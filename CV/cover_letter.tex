%%%%%%%%%%%%%%%%%%%%%%%%%%%%%%%%%%%%%%%%%
% Awesome Cover Letter
% XeLaTeX Template
% Version 1.3 (30/3/2020)
%
% This template has been downloaded from:
% http://www.LaTeXTemplates.com
%
% Original authors:
% Claud D. Park (posquit0.bj@gmail.com)
% Lars Richter (mail@ayeks.de)
% With modifications by:
% Vel (vel@latextemplates.com)
%
% License:
% CC BY-NC-SA 3.0 (http://creativecommons.org/licenses/by-nc-sa/3.0/)
%
% Important note:
% This template must be compiled with XeLaTeX, the below lines will ensure this
%!TEX TS-program = xelatex
%!TEX encoding = UTF-8 Unicode
%
%%%%%%%%%%%%%%%%%%%%%%%%%%%%%%%%%%%%%%%%%

%----------------------------------------------------------------------------------------
%	PACKAGES AND OTHER DOCUMENT CONFIGURATIONS
%----------------------------------------------------------------------------------------

\documentclass[11pt, a4paper]{awesome-cv} % A4 paper size by default, use 'letterpaper' for US letter

\geometry{left=2cm, top=1.5cm, right=2cm, bottom=2cm, footskip=.5cm} % Configure page margins with geometry
 
\fontdir[fonts/] % Specify the location of the included fonts

% Color for highlights
\colorlet{awesome}{awesome-skyblue} % Default colors include: awesome-emerald, awesome-skyblue, awesome-red, awesome-pink, awesome-orange, awesome-nephritis, awesome-concrete, awesome-darknight
%\definecolor{awesome}{HTML}{CA63A8} % Uncomment if you would like to specify your own color

% Colors for text - uncomment and modify
%\definecolor{darktext}{HTML}{414141}
%\definecolor{text}{HTML}{414141}
%\definecolor{graytext}{HTML}{414141}
%\definecolor{lighttext}{HTML}{414141}

\renewcommand{\acvHeaderSocialSep}{\quad\textbar\quad} % If you would like to change the social information separator from a pipe (|) to something else

%----------------------------------------------------------------------------------------
%	PERSONAL INFORMATION
%	Comment any of the lines below if they are not required
%----------------------------------------------------------------------------------------

\name{Andrea}{Bonifacio}
\address{Via Broni 3, Milan, Italy}
\mobile{(+39) 3450867737}

\email{andrea.bonifacio.000@gmail.com}
\homepage{www.itsmebonny.com}
\github{itsmebonny}
%\skype{skypeid}
%\stackoverflow{SOid}{SOname}
%\twitter{@twit}

\position{Mathematical Engineer} % Your expertise/field% A quote or statement

%----------------------------------------------------------------------------------------
%	RECIPIENT/POSITION/LETTER INFORMATION
%	All of the below lines must be filled out
%----------------------------------------------------------------------------------------

\recipient{Optiver}{} % The company being applied to

\letterdate{\today} % The date on the letter, default is the date of compilation

\lettertitle{Job Application for Software Engineer Summer Internship} % The title of the letter

\letteropening{Dear HR,} % How the letter is opened

\letterclosing{Sincerely,} % How the letter is closed

\letterenclosure[Attached]{Curriculum Vitae} % Any enclosures with the letter

  
%----------------------------------------------------------------------------------------

\begin{document}

\makecvheader % Print the header

%----------------------------------------------------------------------------------------
%	LETTER CONTENT
%----------------------------------------------------------------------------------------

\begin{cvletter}

%------------------------------------------------

\lettersection{About Me}

I'm a student from Politecnico di Milano, currently looking to graduate in September 2024. During these years I consolidated a passion for programming and solving IT problems.
%------------------------------------------------

\lettersection{Why Optiver?}

I found Optiver in the list of companies in a list that the Associazione Ingegneri Matematici was sharing, and then i found it again in the list of sponsors in the Advent of Code challenge. So I took a peek at the careers section and found this internship. I find financial markets interesting, even though they're not my principal field of study. I look forward to learn better how a big organization works and how code writing is handled.

%------------------------------------------------

\lettersection{Why Me?}

As I said before, I really like to solve IT problems, to the point that, sometimes, I truly forgot about everything around me until i found a solution. I don't consider myself a great coder as much as I consider myself a great learner. I don't think I'm able to make myself look good on paper, I find myself more comfortable speaking to people.
%------------------------------------------------

\end{cvletter}

%----------------------------------------------------------------------------------------

\makeletterclosing % Print the signature and enclosures

\end{document}