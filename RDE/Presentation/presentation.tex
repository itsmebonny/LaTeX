\documentclass{beamer}

\usetheme{polimi}




%------------------------------------------------------------------------------
%	REQUIRED PACKAGES AND  CONFIGURATIONS
%------------------------------------------------------------------------------
% PACKAGES FOR TITLES
\usepackage{color}

% PACKAGES FOR LANGUAGE AND FONT
\usepackage[utf8]{inputenc}
\usepackage[english]{babel}
\usepackage[T1]{fontenc} % Font encoding
% PACKAGES FOR IMAGES
\usepackage{graphicx}
\graphicspath{{Images/}} % Path for images' folder
\usepackage{eso-pic} % For the background picture on the title page
\usepackage{subfig} % Numbered and caption subfigures using \subfloat
\usepackage{caption} % Coloured captions
\usepackage{transparent}

% STANDARD MATH PACKAGES
\usepackage{amsmath}
\usepackage{amsthm}
\usepackage{bm}
\usepackage[overload]{empheq}  % For braced-style systems of equations

% PACKAGES FOR TABLES 
\usepackage{tabularx}
\usepackage{longtable} % tables that can span several pages
\usepackage{colortbl}

% PACKAGES FOR ALGORITHMS (PSEUDO-CODE)
\usepackage{algorithm}
\usepackage{algorithmic}

% PACKAGES FOR THE APPENDIX
\usepackage{appendix}


% OTHER PACKAGES
\usepackage{amsthm,thmtools,xcolor} % Coloured "Theorem"
\usepackage{comment} % Comment part of code
\usepackage{lipsum} % Insert dummy text
\usepackage{tcolorbox} % Create coloured boxes (e.g. the one for the key-words)
\usepackage{stfloats} % Correct position of the tables
\usepackage{multirow}
\usepackage{multicol}





%-------------------------------------------------------------------------
%	NEW COMMANDS DEFINED
%-------------------------------------------------------------------------
% EXAMPLES OF NEW COMMANDS -> here you see how to define new commands
\newcommand{\bea}{\begin{eqnarray}} % Shortcut for equation arrays
\newcommand{\eea}{\end{eqnarray}}
\newcommand{\e}[1]{\times 10^{#1}}  % Powers of 10 notation
\newcommand{\mathbbm}[1]{\text{\usefont{U}{bbm}{m}{n}#1}} % From mathbbm.sty
\newcommand{\pdev}[2]{\frac{\partial#1}{\partial#2}}
% NB: you can also override some existing commands with the keyword \renewcommand



% \setbeamertemplate{theorem begin}
% {%
%   \par\vskip\medskipamount%
%   \begin{beamercolorbox}[colsep*=.75ex]{block title}
%     \usebeamerfont*{block title}%
%       \inserttheoremname
%       \ifx\inserttheoremaddition\empty\else\ (\inserttheoremaddition)\fi%
%   \end{beamercolorbox}%
%   {\parskip0pt\par}%
%   \ifbeamercolorempty[bg]{block title}
%   {}
%   {\ifbeamercolorempty[bg]{block body}{}{\nointerlineskip\vskip-0.5pt}}%
%   \usebeamerfont{block body}%
%   \vskip-.25ex\vbox{}%
% }
% \setbeamertemplate{theorem end}{}




\newcommand*{\theorembreak}{\(\downarrow\)\usebeamertemplate{theorem end}\framebreak\usebeamertemplate{theorem begin}\rotatebox[origin=c]{180}{$\Lsh$}}





\makeatletter
\newenvironment<>{proofs}[1][\proofname]{%
    \par
    \def\insertproofname{#1\@addpunct{.}}%
    \usebeamertemplate{proof begin}#2}
  {\usebeamertemplate{proof end}}
\newenvironment<>{proofc}{%
  \setbeamertemplate{proof begin}{\begin{block}{}}
    \par
    \usebeamertemplate{proof begin}}
  {\usebeamertemplate{proof end}}
\newenvironment<>{proofe}{%
    \par
    \pushQED{\qed}
    \setbeamertemplate{proof begin}{\begin{block}{}}
    \usebeamertemplate{proof begin}}
  {\popQED\usebeamertemplate{proof end}}






%----------------------------------------------------------------------------
%	ADD YOUR PACKAGES (be careful of package interaction)
%----------------------------------------------------------------------------
\usepackage{amsfonts} 
\usepackage[font=footnotesize,labelfont=bf]{caption}

%----------------------------------------------------------------------------
%	ADD YOUR DEFINITIONS AND COMMANDS (be careful of existing commands)
%----------------------------------------------------------------------------
\usepackage{import}
\usepackage{xifthen}
\usepackage{pdfpages}
\usepackage{transparent}
\usepackage{wrapfig}
\usepackage{dsfont}
\usepackage{tikz}
\usepackage{pgfplots}


\parindent=0pt

\newcommand{\incfig}[1]{%
    \def\svgwidth{\columnwidth}
    \import{./Images/}{#1.pdf_tex}
}
\newcommand{\real}{\mathbb{R}}
\newcommand*{\oldepsilon}{\epsilon}
\renewcommand*{\epsilon}{\varepsilon}
\DeclareMathOperator*{\esssup}{ess\,sup}
\DeclareMathOperator*{\essinf}{ess\,inf}

\newcommand{\oldphi}{\phi}
\renewcommand{\phi}{\varphi}

\newcommand{\oldrho}{\rho}
\renewcommand{\rho}{\varrho}

\numberwithin{equation}{section}
%----------------------------------------------------------------------------
%	CONFIGURATION OF THEOREM ENVIRONMENTS


% \newtheoremstyle{break}
%     {\partopsep}{\topsep}%  
%     {\normalfont}{}
%     {\bfseries}{}%
%     {\newline}{}%
% \theoremstyle{break}
% \newtheorem{theorem}{Theorem}[section]
% \newtheorem{corollary}{Corollary}[section]
% \newtheorem{proposition}{Proposition}[section]
% \newtheorem{remark}{Remark}[section]
% \newtheorem{lemma}{Lemma}[section]
% \newtheorem{notation}{Notation}[section]
% \newtheorem{definition}{Definition}[section]

% \newtheorem{definition}{Definition}[section]

\newtheorem*{remark}{Remark}


\setbeamertemplate{theorems}[numbered]

% %----------------------------------------------------------------------------

% Do not change Configuration_files/config.tex file unless you really know what you are doing.
% This file ends the configuration procedures (e.g. customizing commands, definition of new commands)


% Create color bluePoli (-> manuale grafica coordinata:  https://www.polimi.it/fileadmin/user_upload/il_Politecnico/grafica-coordinata/2015_05_11_46xy_manuale_grafica_coordinata.pdf)
\definecolor{bluePoli}{cmyk}{0.4,0.1,0,0.4}

% Custom theorem environments
\declaretheoremstyle[
  shaded={rulecolor=bluePoli!20, rulewidth=1pt, bgcolor=bluePoli!5},
  headfont=\color{bluePoli}\normalfont\bfseries,
  bodyfont=\color{black}\normalfont,
]{colored}

\captionsetup[figure]{labelfont={color=bluePoli}} % Set colour of the captions
\captionsetup[table]{labelfont={color=bluePoli}} % Set colour of the captions
\captionsetup[algorithm]{labelfont={color=bluePoli}} % Set colour of the captions

% \theoremstyle{colored}
% \newtheorem{theorem}{Theorem}[section]
% \newtheorem{proposition}{Proposition}[section]
% \newtheorem{definition}{Definition}[section]
% \newtheorem*{remark}{Remark}
% \newtheorem{lemma}{Lemma}[section]



% Insert here the info that will be displayed into your Title page
% -> title of your work
% -> author name and surname
% -> MSc course
\newcommand\norm[1]{\lVert#1\rVert}





%------------------------------------------------------------------------------
%	TITLE PAGE

%------------------------------------------------------------------------------
\title{The KPP-Fisher equation over simple graphs}
\author{Andrea Bonifacio}

\date{}


\begin{document}

\begin{frame}[plain]
\titlepage
\end{frame}

%------------------------------------------------------------------------------
%	PRESENTATION SLIDES
%------------------------------------------------------------------------------

%------------------------------------------------
\section{Introduction}

% ======================== First Frame ========================

\begin{frame}
    \frametitle{Introduction}
    \begin{itemize}
        \item This report is based on the paper [1], in which the authors study the dynamics of a population in a river network.
        \item The idea is to model the population with a reaction-diffusion equation, in this case the Fisher-KPP equation. 
        \item The interesting idea proposed in this paper is the study of the population dynamics using a graph-based approach. The authors consider a river network as a graph, where the edges of the graph are the rivers and the nodes are the junctions between rivers. This is not the first time that this kind of approach is used, but, according to the authors, the other works in this area are focused on finite graphs.
    \end{itemize}
\end{frame}

% ======================== Second Frame ========================

\begin{frame}
    \frametitle{Introduction}
    \begin{itemize}
        \item This work will be divided in the following way
        \begin{itemize}
            \item Brief recap on the Fisher-KPP equation.
            \item Presentation and main results of the graph-based approach used by the authors.
            \item In depth analysis of the one of the case presented in the paper.
            \item Finally, some conclusions will be drawn.
        \end{itemize}
    \end{itemize}
\end{frame}



%------------------------------------------------

% \input{Sections/02_KPP-Fisher}

%------------------------------------------------

\section{KPP-Fisher equation over simple graphs}

% ======================== First Frame ========================

\begin{frame}
    \frametitle{The river network}
    Instead of considering a river as a continuous line, here it will be taken into consideration the fact that rivers can divide into smaller rivers and merge into larger rivers. This can be modeled as a graph, where the edges are the rivers and the nodes are the junctions between rivers.
    \tikzset{every picture/.style={line width=0.75pt}} %set default line width to 0.75pt 
\begin{figure}[H]
    \centering
    \scalebox{0.6}{\begin{tikzpicture}[x=0.75pt,y=0.75pt,yscale=-1,xscale=1]
%uncomment if require: \path (0,300); %set diagram left start at 0, and has height of 300

%Straight Lines [id:da06835250339153032] 
\draw    (27.17,139.58) -- (116.96,139.51) ;
\draw [shift={(77.06,139.54)}, rotate = 179.95] [fill={rgb, 255:red, 0; green, 0; blue, 0 }  ][line width=0.08]  [draw opacity=0] (8.93,-4.29) -- (0,0) -- (8.93,4.29) -- cycle    ;
%Shape: Circle [id:dp006568808299004081] 
\draw  [fill={rgb, 255:red, 0; green, 0; blue, 0 }  ,fill opacity=1 ] (114.63,139.51) .. controls (114.63,138.22) and (115.67,137.18) .. (116.96,137.18) .. controls (118.25,137.18) and (119.3,138.22) .. (119.3,139.51) .. controls (119.3,140.8) and (118.25,141.84) .. (116.96,141.84) .. controls (115.67,141.84) and (114.63,140.8) .. (114.63,139.51) -- cycle ;
%Straight Lines [id:da5818375484660003] 
\draw    (456.92,139.33) -- (551.24,139.67) ;
\draw [shift={(509.08,139.52)}, rotate = 180.2] [fill={rgb, 255:red, 0; green, 0; blue, 0 }  ][line width=0.08]  [draw opacity=0] (8.93,-4.29) -- (0,0) -- (8.93,4.29) -- cycle    ;
%Straight Lines [id:da41798667092826824] 
\draw    (120,139.58) -- (210.03,139.58) ;
\draw [shift={(170.01,139.58)}, rotate = 180] [fill={rgb, 255:red, 0; green, 0; blue, 0 }  ][line width=0.08]  [draw opacity=0] (8.93,-4.29) -- (0,0) -- (8.93,4.29) -- cycle    ;
%Shape: Circle [id:dp9495741628591831] 
\draw  [fill={rgb, 255:red, 0; green, 0; blue, 0 }  ,fill opacity=1 ] (548.91,139.67) .. controls (548.91,138.38) and (549.95,137.33) .. (551.24,137.33) .. controls (552.53,137.33) and (553.57,138.38) .. (553.57,139.67) .. controls (553.57,140.96) and (552.53,142) .. (551.24,142) .. controls (549.95,142) and (548.91,140.96) .. (548.91,139.67) -- cycle ;
%Shape: Boxed Line [id:dp3634037576105641] 
\draw    (551.24,139.67) -- (632.76,187.12) ;
\draw [shift={(596.32,165.91)}, rotate = 210.2] [fill={rgb, 255:red, 0; green, 0; blue, 0 }  ][line width=0.08]  [draw opacity=0] (8.93,-4.29) -- (0,0) -- (8.93,4.29) -- cycle    ;
%Shape: Boxed Line [id:dp5855966408278438] 
\draw    (551.24,139.67) -- (633.09,92.79) ;
\draw [shift={(596.5,113.75)}, rotate = 150.2] [fill={rgb, 255:red, 0; green, 0; blue, 0 }  ][line width=0.08]  [draw opacity=0] (8.93,-4.29) -- (0,0) -- (8.93,4.29) -- cycle    ;
%Straight Lines [id:da7091000862619046] 
\draw    (319.92,139.33) -- (414.24,139.67) ;
\draw [shift={(372.08,139.52)}, rotate = 180.2] [fill={rgb, 255:red, 0; green, 0; blue, 0 }  ][line width=0.08]  [draw opacity=0] (8.93,-4.29) -- (0,0) -- (8.93,4.29) -- cycle    ;
%Shape: Boxed Line [id:dp8564221650100632] 
\draw    (240.73,91.88) -- (322.25,139.33) ;
\draw [shift={(285.81,118.12)}, rotate = 210.2] [fill={rgb, 255:red, 0; green, 0; blue, 0 }  ][line width=0.08]  [draw opacity=0] (8.93,-4.29) -- (0,0) -- (8.93,4.29) -- cycle    ;
%Shape: Circle [id:dp5640763569876212] 
\draw  [fill={rgb, 255:red, 0; green, 0; blue, 0 }  ,fill opacity=1 ] (319.92,139.33) .. controls (319.92,138.04) and (320.96,137) .. (322.25,137) .. controls (323.54,137) and (324.58,138.04) .. (324.58,139.33) .. controls (324.58,140.62) and (323.54,141.67) .. (322.25,141.67) .. controls (320.96,141.67) and (319.92,140.62) .. (319.92,139.33) -- cycle ;
%Shape: Boxed Line [id:dp8885765217793992] 
\draw    (240.4,186.21) -- (322.25,139.33) ;
\draw [shift={(285.66,160.28)}, rotate = 150.2] [fill={rgb, 255:red, 0; green, 0; blue, 0 }  ][line width=0.08]  [draw opacity=0] (8.93,-4.29) -- (0,0) -- (8.93,4.29) -- cycle    ;


% Text Node
\draw (35,115.4) node [anchor=north west][inner sep=0.75pt]    {$R_{U}{}{}$};
% Text Node
\draw (465,115.4) node [anchor=north west][inner sep=0.75pt]    {$R_{U}{}{}$};
% Text Node
\draw (169,115.4) node [anchor=north west][inner sep=0.75pt]    {$R_{L}{}{}$};
% Text Node
\draw (378,115.4) node [anchor=north west][inner sep=0.75pt]    {$R_{L}{}{}$};
% Text Node
\draw (592,76.4) node [anchor=north west][inner sep=0.75pt]    {$R_{L_{1}}{}{}$};
% Text Node
\draw (592,173.4) node [anchor=north west][inner sep=0.75pt]    {$R_{L_{2}}{}{}$};
% Text Node
\draw (252,76.4) node [anchor=north west][inner sep=0.75pt]    {$R_{U_{1}}{}{}$};
% Text Node
\draw (254,177.4) node [anchor=north west][inner sep=0.75pt]    {$R_{U_{2}}{}{}$};
% Text Node
\draw (28,200) node [anchor=north west][inner sep=0.75pt]   [align=left] {a)};
% Text Node
\draw (240,200) node [anchor=north west][inner sep=0.75pt]   [align=left] {b)};
% Text Node
\draw (458,200) node [anchor=north west][inner sep=0.75pt]   [align=left] {c)};


\end{tikzpicture}}
    \caption{Graph representation of the river. a) The river is split in an upper branch \(R_U\) and a lower branch \(R_L\). b) The upper branch is split in two branches \(R_{U_1}\) and \(R_{U_2}\). c) The lower branch is split in two branches \(R_{L_1}\) and \(R_{L_2}\).}
    \label{fig:river-graph}
\end{figure}
\end{frame}

% ======================== Second Frame ========================

\begin{frame}
    \frametitle{Steady states}
    With the graph based approach, infinitely many steady states can exist. However, only one of them can be a global attractor. And these global attractors can be summed up in three types:
    \begin{enumerate}
        \item \textbf{washing out:} the population density goes to zero in all the branches;
        \item \textbf{persistence at carrying capacity:} the population survives in all the branches;
        \item \textbf{persistence below carrying capacity:} the population density goes to a value below the carrying capacity in some branches.
    \end{enumerate}
\end{frame}

%------------------------------------------------

\section{Preliminaries}

% ======================== First Frame ========================

\begin{frame}[allowframebreaks]
    \frametitle{Comparison principle}
    \begin{lemma}
        Assume that \(c_i(x,t)\) is bounded on \(\real_i \times [0, T]\) for \(i = U, L\) for some \(0 < T < +\infty\). Let \(w_i \in C(\overline{\real}_i \times [0, T]) \cap C^{1,2}(\real_i \times (0, T])\) satisfy
        \begin{equation*}
            \begin{dcases}
                \partial_t w_i - \partial_{xx} w_i + \beta_i \partial_x w_i + c_i(x,t)w_i \leq 0, & x \in \real_i, 0 < t < T, \\
                w_L(0, t) = w_{U}(0, t), & t > 0 \\
                a_{U} \partial_x w_{U}(0, t) = a_L \partial_x w_L(0, t), & 0 < t < T, \\
                w_i(x, 0) \leq 0, & x \in \real_i,
            \end{dcases}
        \end{equation*}
        \theorembreak
        and 
        \begin{equation}
            \liminf_{R \to +\infty} e^{-cR^2}\left[\max_{\substack{0\leq t\leq T \\ \lvert x \rvert = R}} w_i(x,t)\right] \leq 0
            \label{eq:2.1}
        \end{equation}
        for some \(c > 0\). Then
        \[
            w_i(x,t) \leq 0 \quad \text{for all } x \in \real_i, 0 \leq t \leq T.
        \]
        If \(w_i(x,0)\) is strictly lower than zero then the strict inequality holds. 
        \label{lem:2.1}
    \end{lemma}
\end{frame}

% ======================== Second Frame ========================

\begin{frame}[allowframebreaks]
    \frametitle{Supersolutions and subsolutions}
    \begin{definition}
        If \((\tilde{w}_L, \tilde{w}_{U})\) with \(w_i \in C(\overline{\real}_i \times [0, T]) \cap C^{1,2}(\real_i \times (0, T])\) satisfies 
        \begin{equation}
            \begin{dcases}
                \partial_t \tilde{w}_i - \partial_{xx} \tilde{w}_i + \beta_i \partial_x \tilde{w}_i \geq (\leq) f_i(\tilde{w}_i), & x \in \real_i, 0 < t < T, \\
                \tilde{w}_L(0, t) = \tilde{w}_{U}(0, t), & t > 0 \\
                    a_{U} \partial_x \tilde{w}_{U}(0, t) - a_L \partial_x \tilde{w_L}(0, t) \geq (\leq) 0, & 0 < t < T, \\
            \end{dcases}
            \label{eq:2.2}
        \end{equation}
        then \((\tilde{w}_L, \tilde{w}_{U})\) is a supersolution (subsolution) of \ref{eq:2.2}
        \label{def:2.1}
    \end{definition}
    \framebreak
    Using Lemma \ref{lem:2.1} and Definition \ref{def:2.1}
    \begin{lemma}
        Assume \(f_i(w)\) is locally Lipschitz \((i = U, L)\). Let \((\underline{w}_L, \underline{w}_{U})\) and \((\overline{w}_L, \overline{w}_{U})\) be, respectively, a bounded subsolution and bounded supersolution of \ref{eq:2.2} satisfying \(\underline{w}_i(\cdot, 0) \leq \overline{w}_i(\cdot, 0)\). Then, \(\underline{w}_i \leq \overline{w}_i\). Additionally, if \(\underline{w}_j(x,0) < \overline{w}_j(x,0)\) in a strict sense for some \(j = U, L\), then \(\underline{w}_i < \overline{w}_i\).
        \label{lem:2.2} 
    \end{lemma}
\end{frame}

% ======================== Third Frame ========================

\begin{frame}
    \frametitle{Existence and uniqueness of solutions}
    Existence and uniqueness of solutions for the system \ref{eq:1.6} follows from the following theorem.
    \begin{theorem}
        For any nonnegative initial data \((w_{L,0}, w_{U,0}) \in C_{comp}(\real) \times C_{comp}(\real) \) \, the problem \eqref{eq:1.6} has a unique classical solution \((w_L, w_{U})\) which is defined and uniformly bounded for all \(t > 0\).
        \label{thm:2.4}
    \end{theorem}
\end{frame}

% ======================== Fourth Frame ========================

\begin{frame}
    \frametitle{Sketch of the proof}
    \begin{proofs}[Sketch of the proof]
        It is possible to transform the system \ref{eq:1.6} into an equivalent half-line problem with compactly supported initial data. Then, the standard theory guarantees the existence and uniqueness of a classical solution. The only remaining step is to show the uniform boundedness of \(w_i\) for \(i = U, L\).
        It is possible to prove that the half-line problem has a unique solution \(w_L^\ell, w_{U}^\ell\), where the superscript \(\ell\) stands for the extremum of the domain \((0, \ell)\). 
    \end{proofs}
\end{frame}

% ======================== Fifth Frame ========================
\begin{frame}
    \frametitle{Sketch of the proof}
    \begin{proofe}
        Those solutions are nondecreasing in \(\ell\), and it holds 
        \begin{equation}
            0 \leq w_i^\ell(x,t) \leq 1 + \sup_{\real_L} w_{L,0} + \sup_{\real_{U}} w_{U,0}, \quad i = U, L.
            \label{eq:2.4}
        \end{equation}
        So, the limit \(\ell \to +\infty\) exists, denoted as \(w_i^\infty\).
        It is possible to show that \(w_i^\ell\) converges to \(w_i^\infty\), and that \(w_i^\infty\) is a solution of the original problem \ref{eq:1.6}. 
        At this point, by uniqueness, \(w_i = w_i^\infty\), and thanks to \eqref{eq:2.4}, the solution is uniformly bounded.
    \end{proofe}
\end{frame}
    

%------------------------------------------------

\section{One upper branch and one lower branch}


% ========================= First Frame =========================
% \begin{frame}
%     \frametitle{Proof of Theorem \ref{thm:1.2}}
%     Now it is possible to prove Theorem \ref{thm:1.2} using Theorem \ref{thm:1.1}.
    
%     Each part of the theorem will be proved separately with a sketch of the proof.

%     The proof of Theorem \ref{thm:1.1} is not provided here, but it's based on a phase-plane analysis approach.

% \end{frame}

% ========================= Second Frame =========================

\begin{frame}
    \frametitle{Theorem \ref{thm:1.2}: \texorpdfstring{\(\beta_U < 2\)}{betaU < 2}}
    \begin{theorem}
        Assume that \(\beta_U < 2\) and the initial datum has compact support \(w_{0} \in C_{comp}(\real)\), nontrivial and nonnegative. Let \((w_L, w_{U})\) be the solution of \eqref{eq:1.6}. Then,
        \[
            (w_L(\cdot, t), w_{U}(\cdot, t)) \to (1, 1) \quad \text{locally uniformly as } t \to +\infty.
        \]
        \label{thm:3.1} 
    \end{theorem}
\end{frame}

% ========================= Third Frame =========================

\begin{frame}
    \frametitle{Proof of Theorem \ref{thm:3.1} I}
    \begin{proofs}[Sketch of the proof]
        From Theorem \ref{thm:2.4} is known that 
    \begin{equation}
        0 \leq w_{i} \leq 1 + \sup_{\real} w_{0}, \quad i = U, L.
        \label{eq:3.1}
    \end{equation}
    As \(\norm{w_{0}}_{L^{\infty}(\real)} > 0\), take \(\overline{u}(t) = 1 + \norm{w_{0}}_{L^{\infty}(\real)}e^{-t}\). 
    It is possible to prove that \((\overline{u}, \overline{u})\) is a supersolution of \eqref{eq:1.6}. Using Lemma \ref{lem:2.2} and \eqref{eq:3.1}, it is possible to obtain 
    \begin{equation}
        1 = \lim_{t \to +\infty} \overline{u}(t) \geq \limsup_{t \to +\infty} \norm{w_i(\cdot, t)}_{L^{\infty}(\real)}, \quad i = U, L.
        \label{eq:3.2}
    \end{equation}
    \end{proofs}
\end{frame}

% ========================= Fourth Frame =========================
% \begin{frame}
%     \frametitle{Proof of Theorem \ref{thm:3.1} II}
%     \begin{proofc}
%         From the assumptions, \(w_0\) is nontrivial and nonnegative, so \(w_U(x, 1) > 0\) for \(x \in \real_U\). Since \(0 < \beta_U < 2\), there exists a unique constant \(\ell_0 > 0\) such that the following problem
%         \begin{equation}
%             \begin{dcases}
%                 -\ddot{w} + \beta_U \dot{w} = w(1 - w), & x \in (-\ell, 0), \\
%                 w(0) = w(-\ell) = 0
%             \end{dcases}
%             \label{eq:3.3}
%         \end{equation}
%         has a positive solution iff \(\ell > \ell_0\), and the positive solution \(w_\ell\) is unique. Fixing \(\ell\), it is possible to ensure that the unique solution \(w_\ell\) of \eqref{eq:3.3} satisfies \(w_\ell(x) < w_U(x, 1)\) for \(x \in [-\ell, 0]\).
%     \end{proofc}
% \end{frame}

% ========================= Fifth Frame =========================
\begin{frame}
    \frametitle{Proof of Theorem \ref{thm:3.1} II}
    \begin{proofc}
        Now set 
    \begin{equation*}
        w_\ell^0(x) \coloneqq \begin{cases}
            w_\ell & \text{if } x \in [-\ell, 0], \\
            0 & \text{if } x \in (-\infty, -\ell).
        \end{cases}
    \end{equation*}
    Then, let \((\underline{w}_L. \underline{w}_U)\) be the solution of \eqref{eq:1.6} with initial datum \((0, w_\ell^0)\). Clearly,
    \begin{align*}
        \underline{w}_U(x, t) > 0 & \text{ for } x \in \real_U, t > 0, \\
        \underline{w}_L(x, t) > 0 & \text{ for } x \in \real_L, t > 0.
    \end{align*}
    By the parabolic comparison principle it is possible to conclude that \(\underline{w}_U \geq w_0^l\) for all \((x,t) \in [-\ell, 0] \times [0, +\infty)\).
    \end{proofc}
\end{frame}

% ========================= Sixth Frame =========================

\begin{frame}
    \frametitle{Proof of Theorem \ref{thm:3.1} III}
    \begin{proofc}
        Hence,
        \begin{equation*}
            (\underline{w}_L(\cdot, t), \underline{w}_U(\cdot, t)) \geq (0, w_\ell^0) \quad \text{for all } t > 0.
        \end{equation*}
        Using Lemma \ref{lem:2.2} it follows, with \(\delta > 0\):
        \begin{equation*}
            (\underline{w}_L(\cdot, t + \delta), \underline{w}_U(\cdot, t + \delta)) \geq (\underline{w}_L(\cdot, t), \underline{w}_U(\cdot, t)) \quad \text{for all } t > 0,
        \end{equation*}
        meaning that \((\underline{w}_L, \underline{w}_U)\) is nondecreasing in \(t\). 
    \end{proofc}
\end{frame}

% ========================= Seventh Frame =========================

\begin{frame}
    \frametitle{Proof of Theorem \ref{thm:3.1} IV}
    \begin{proofc}
        By denoting 
        \[
            \left(\underline{w}_{L, \infty}(x), \underline{w}_{U, \infty}(x)\right) \coloneqq \left(\lim_{t \to +\infty} \underline{w}_L(x, t), \lim_{t \to +\infty} \underline{w}_U(x, t)\right), 
        \]
        it is possible to use a similar argument as in proof of Theorem \ref{thm:2.4} to conclude that
        \[
            (\underline{w}_{L}(\cdot, t), \underline{w}_{U}(\cdot, t)) \to (\underline{w}_{L, \infty}(\cdot), \underline{w}_{U, \infty}(\cdot)),
        \]
        locally uniformly as \(t \to +\infty\) and that \((\underline{w}_{L, \infty}(x), \underline{w}_{U, \infty}(x))\) is a positive stationary solution of \eqref{eq:1.6}. 
    \end{proofc}
\end{frame}

% ========================= Eighth Frame =========================

\begin{frame}
    \frametitle{Proof of Theorem \ref{thm:3.1} V}
    \begin{proofc}
        By Theorem \ref{thm:1.1}, the only possible positive stationary solution is \((1, 1)\), therefore
        \begin{equation}
            (\underline{w}_{L}(\cdot, t), \underline{w}_{U}(\cdot, t)) \to (1, 1) \quad \text{locally uniformly as } t \to +\infty.
            \label{eq:3.4}
        \end{equation}
        On the other hand, as \(w_\ell^0(x) < w_U(x, 1)\) for \(x \in (-\infty, 0)\), thanks to Lemma \ref{lem:2.2},
        \[
            (\underline{w}_L(\cdot, t), \underline{w}_U(\cdot, t)) \leq (\underline{w}_L(\cdot, t + 1), \underline{w}_U(\cdot, t + 1)) \quad \text{for all } t > 0.
        \]
    \end{proofc}
\end{frame}

% ========================= Ninth Frame =========================

\begin{frame}
    \frametitle{Proof of Theorem \ref{thm:3.1} VI}
    \begin{proofe}
        This and \eqref{eq:3.4} imply that
        \begin{equation}
            (\liminf_{t \to +\infty} w_L(\cdot, t), \liminf_{t \to +\infty} w_U(\cdot, t)) \to (1, 1),
            \label{eq:3.5}
        \end{equation}
        locally uniformly as \(t \to +\infty\).
        Combining \eqref{eq:3.2} and \eqref{eq:3.5} concludes the proof.
    \end{proofe}
\end{frame}
        

%------------------------------------------------

% % ================== First Frame =========================

\begin{frame}
    \frametitle{Theorem \ref{thm:1.2}: \texorpdfstring{\(\beta_U, \beta_L\geq 2\)}{betaU, betaL >= 2}}
    \begin{theorem}
        Assume that \(\beta_U, \beta_L \geq 2\) and the initial datum has compact support \(w_{0} \in C_{comp}(\real)\), nontrivial and nonnegative. Let \((w_L, w_{U})\) be the solution of \eqref{eq:1.6}. Then,
        \[
            (w_L(\cdot, t), w_{U}(\cdot, t)) \to (0, 0) \quad \text{locally uniformly as } t \to +\infty.
        \]
        Moreover, \(\norm{w_L(\cdot, t)}_{L^{\infty}(\real)} \to 1\) and \(\norm{w_U(\cdot, t)}_{L^{\infty}(\real)} \to 0\) as \(t \to +\infty\).
        \label{thm:3.2}
    \end{theorem}
\end{frame}

% ================== Second Frame =========================

\begin{frame}
    \frametitle{Proof of Theorem \ref{thm:3.2} I}
    \begin{proofs}[Sketch of the proof]
        Fix \(\alpha \in (0, 1)\) and define a supersolution \((\overline{w}_L, \overline{w}_U)\) as
        \begin{align*}
            \overline{w}_L(x, t) & \coloneqq \phi_L(x; \alpha) + M e^{-\lambda t} e^{\frac{\beta_L}{2}x} \quad x \in \real_L, t \geq 0, \\
            \overline{w}_U(x, t) & \coloneqq \phi_U(x; \alpha) + M e^{-\lambda t} e^{\frac{\beta_U}{2}x} \quad x \in \real_U, t \geq 0,
        \end{align*}
        where \(M\) and \(\lambda\) will be determined later. Now, having defined 
        \[
            v(x) \coloneqq M e^{kx} \quad \text{with } k = \frac{1}{2}\beta_U^ + \sqrt{\beta_U^2 - 4}.
        \]
        
    \end{proofs}
\end{frame}

% ================== Third Frame =========================

\begin{frame}
    \frametitle{Proof of Theorem \ref{thm:3.2} II}
    \begin{proofc}
        Clearly,
        \[
            \ddot{v} + \beta_U \dot{v} = v \geq v - v^2.
        \]
        Thanks to Theorem \ref{thm:1.1}(iv), as \(x \to -\infty\), \(\phi_U(x; \alpha)\) satisfies \eqref{eq:1.9}. Therefore, it exists \(l = l_\alpha > 0\) such that
        \[
            \phi_U(x; \alpha) \geq v(x) \quad \text{for } x \leq -l.
        \]
        Fix now \(l = l_\alpha\). Because \(\beta_U, \beta_L \geq 2\), \(\phi_L(0;\alpha) = \alpha > 0\) and \(2\phi_U(-l;\alpha) > 0\) it is always possible to find a sufficiently small \(\lambda(l) > 0\) such that 
        \begin{align*}
            \partial_t \overline{w}_L - \partial_{xx} \overline{w}_L + \beta_L \partial_x \overline{w}_L - \overline{w}_L + \overline{w}_L^2 \geq 0, \quad x \in (0, +\infty), t > 0, \\
            \partial_t \overline{w}_U - \partial_{xx} \overline{w}_U + \beta_U \partial_x \overline{w}_U - \overline{w}_U + \overline{w}_U^2 \geq 0, \quad x \in [-l, 0), t > 0.
        \end{align*}
    \end{proofc}
\end{frame}

% ================== Fourth Frame =========================

\begin{frame}
    \frametitle{Proof of Theorem \ref{thm:3.2} III}
    \begin{proofc}
        Since \(w_0 \in C_{comp}(\real)\), \(M\) can be chosen as \(M > \max{1, \norm{w_0}_{L^{\infty}(\real)}}\) such that \(v(x) > w_0(x)\) in \(\real\) and 
        \[
            \overline{w}_i(x, 0) \geq w_i(x, 0) \quad \text{for } x \in \real, i = L, U.
        \] 
        Since \(v(0) = M > \max{1, \norm{w_0}_{L^{\infty}(\real)}} \geq w_U(0, t)\) for all \(t \geq 0\), by the comparison principle
        \[
            w_U(x, t) \leq v(x) \quad \text{for } x \in \real_U, t \geq 0.
        \]
        Let now \(\xi(t) \coloneqq \frac{2\lambda}{\beta_L} t\). Then 
        \[
            \overline{w}_L(\xi(t), t) > M \geq w_L(\xi(t), t) \quad \text{for all } t > 0.
        \]
    \end{proofc}
\end{frame}

% ================== Fifth Frame =========================

\begin{frame}
    \frametitle{Proof of Theorem \ref{thm:3.2} IV}
    \begin{proofc}
        Then, \((\overline{w}_L, \overline{w}_U)\) is a supersolution of \eqref{eq:1.6} in the region \(x \in [-l, \xi(t)]\) for all \(t \geq 0\). It's possible to conclude that
        \[
            w_U(x, t) \leq \overline{w}_U(x, t) \quad \text{for } x \in \real_U, t \geq 0,
        \]
        Therefore,
        \[
            \limsup_{t \to +\infty} w_i(x, t) \leq \lim \overline{w}_i(x, t) = \phi_i(x; \alpha) = 0 \quad \text{for } x \in \real_i, i = L, U.
        \]
        Since \(\alpha\) is arbitrary, it is possible to conclude that
        \[
            \lim_{t \to +\infty} w_i(x, t) = 0 \quad \text{locally uniformly for } x \in \real_i, i = L, U.
        \]
    \end{proofc}
\end{frame}

% ================== Sixth Frame =========================

\begin{frame}
    \frametitle{Proof of Theorem \ref{thm:3.2} V}
    \begin{proofe}
        Since \(\overline{w}_U\) is increasing in \(x\), it is easily concluded that \(\norm{w_U(\cdot, t)}_{L^{\infty}(\real)} \to 0\) as \(t \to +\infty\).

        By creating a problem similar to \eqref{eq:1.6} and constructing a subsolution for that problem it is possible to prove that \(\norm{w_L(\cdot, t)}_{L^{\infty}(\real)} \to 1\) as \(t \to +\infty\).
    \end{proofe}
\end{frame}

%------------------------------------------------

% % =================== First Frame ===================

\begin{frame}
    \frametitle{Theorem \ref{thm:1.2}: \texorpdfstring{\(\beta_U \geq 2 > \beta_L\)}{betaU >= 2 > betaL}}
    \begin{theorem}
        Assume that \(\beta_U \geq 2 > \beta_L\) and the initial datum has compact support \(w_{0} \in C_{comp}(\real)\), nontrivial and nonnegative. Let \((w_L, w_{U})\) be the solution of \eqref{eq:1.6}. Then,
        \begin{equation}
            (w_L(\cdot, t), w_{U}(\cdot, t)) \to (\phi_L(\cdot; \alpha_0), \phi_U(\cdot; \alpha_0))
            \label{eq:3.6}
        \end{equation}
        locally uniformly as \(t \to +\infty\).
        \label{thm:3.3}
    \end{theorem}
\end{frame}

% =================== Second Frame ===================

\begin{frame}
    \frametitle{Proof of Theorem \ref{thm:3.3} I}
    \begin{proofs}[Sketch of the proof]
        For \(M > 1\), set 
        \begin{align*}
            &k \coloneqq \frac{\beta_U + \sqrt{\beta_U^2 - 4}}{2}, \\
            &\overline{\phi}_{U,0} \coloneqq M e^{kx}, \quad x \in \real_U, \\
            &\overline{\phi}_{L,0} \coloneqq M, \quad x \in \real_L.
        \end{align*}
        Since \(w_0\) has compact support, it is possible to choose \(M > 1\) large enough that
        \begin{align*}
            \overline{\phi}_{U,0}(x) = M e^{kx} > w_0(x) \quad \text{for } x \in \real_U, \\
            \overline{\phi}_{L,0}(x) = M > w_0(x) \quad \text{for } x \in \real_L. 
        \end{align*}
    \end{proofs}
\end{frame}

% =================== Third Frame ===================

\begin{frame}
    \frametitle{Proof of Theorem \ref{thm:3.3} II}
    \begin{proofc}
        Therefore, \((\overline{\phi}_{L,0}, \overline{\phi}_{U,0})\) is a supersolution of the corresponding elliptic problem of \eqref{eq:1.6}. It follows that the unique solution \((\overline{\phi}_U, \overline{\phi}_L)\) of \eqref{eq:1.6} with initial datum \((\overline{\phi}_{L,0}, \overline{\phi}_{U,0})\) is nondecreasing in \(t\).
        As \(t \to +\infty\),
        \[
                (\overline{\phi}_L(\cdot, t), \overline{\phi}_U(\cdot, t)) \to (\hat{\phi}_L(\cdot), \hat{\phi}_U(\cdot)) \quad \text{locally uniformly},
        \]
        and \((\hat{\phi}_L, \hat{\phi}_U)\) is a nonnegative stationary solution of \eqref{eq:1.6}. Clearly,
        \begin{align*}
            \hat{\phi}_U(x) \leq \overline{\phi}_{U,0}(x) = M e^{kx} \quad \text{for } x \in \real_U, \\
            \hat{\phi}_L(x) \leq \overline{\phi}_{L,0}(x) = M \quad \text{for } x \in \real_L.
        \end{align*}
    \end{proofc}
\end{frame}

% =================== Fourth Frame ===================

\begin{frame}
    \frametitle{Proof of Theorem \ref{thm:3.3} III}
    \begin{proofc}
        Defining 
        \begin{align*}
                \underline{\phi}_{U,0} \coloneqq 0, \quad x \in \real_U, \\
                    \underline{\phi}_{L,0} \coloneqq \phi_l(x), \quad x \in \real_L,
        \end{align*}
        where \(\phi_l(x)\) is the unique solution of
        \begin{equation*}
            \begin{cases}
                -\ddot{\phi} + \beta_L \dot{\phi} = \phi(1 - \phi), & x \in (0, l), \\
                \phi(0) = \phi(l) = 0.
            \end{cases}
        \end{equation*}
        It is possible to prove that \((\underline{\phi}_L, \underline{\phi}_U)\) is a subsolution of the elliptic problem of \eqref{eq:1.6}, and again it follows that the unique solution \((\underline{\phi}_L, \underline{\phi}_U)\) of \eqref{eq:1.6} with initial datum \((\underline{\phi}_{L,0}, \underline{\phi}_{U,0})\) is nondecreasing in \(t\).
    \end{proofc}
\end{frame}

% =================== Fifth Frame ===================

\begin{frame}
    \frametitle{Proof of Theorem \ref{thm:3.3} IV}
    \begin{proofc}
        As \(t \to +\infty\),
        \[
            (\underline{\phi}_L(\cdot, t), \underline{\phi}_U(\cdot, t)) \to (\tilde{\phi}_L(\cdot), \tilde{\phi}_U(\cdot)) \quad \text{locally uniformly},
        \]
        and \((\tilde{\phi}_L, \tilde{\phi}_U)\) is a nonnegative stationary solution of \eqref{eq:1.6}. Since
        \[
            \underline{\phi}_{U,0}(x) < \hat{\phi}_U(x), \quad \underline{\phi}_{L,0}(x) < \hat{\phi}_L(x)
        \]
        it's easy to see
        \begin{align*}
            0 < \tilde{\phi}_U(x) \leq \hat{\phi}_U(x)  \leq M e^{kx} \quad \text{for } x \in \real_U, \\
            \phi_l(x) \leq \tilde{\phi}_L(x) \leq \hat{\phi}_L(x) \leq M \quad \text{for } x \in \real_L.
        \end{align*}
    \end{proofc}
\end{frame}

% =================== Sixth Frame ===================

\begin{frame}
    \frametitle{Proof of Theorem \ref{thm:3.3} V}
    \begin{proofe}
        Then \((\tilde{\phi}_L, \tilde{\phi}_U)\) and \((\hat{\phi}_L, \hat{\phi}_U)\) must be solutions of \eqref{eq:1.8} with some \(\hat{\alpha}, \tilde{\alpha} \in (0,1)\), with \(\hat{\alpha} \geq \tilde{\alpha}\). With Theorem \ref{thm:1.1}(iv) it is possible to prove that \(\hat{\alpha} = \tilde{\alpha} = \alpha_0\), and therefore the two solutions coincide and can be defined as \(\phi_L(\cdot; \alpha_0), \phi_U(\cdot; \alpha_0)\).
        With the same arguments as in the previous cases, it is possible to conclude that
        \begin{align*}
            \liminf_{t \to +\infty} w_i(x, t) \geq \phi_i(x; \alpha_0) \quad \text{locally uniformly for } x \in \real_i, \\
            \limsup_{t \to +\infty} w_i(x, t) \leq \phi_i(x; \alpha_0) \quad \text{locally uniformly for } x \in \real_i,
        \end{align*}
        for \(i = U, L\).
        Therefore, \eqref{eq:3.6} holds.
    \end{proofe}
\end{frame}

\end{document}