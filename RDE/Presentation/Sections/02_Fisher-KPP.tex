\section{The KPP-Fisher equation}

% ======================== First Frame ========================

\begin{frame}
    \frametitle{Boundary Value Problem}
    The Fisher-KPP equation is a reaction-diffusion equation that models the spread of a population in a given environment. The boundary value problem is given by
    \begin{equation}
        \begin{dcases}
            u_t - D \Delta u = f(\bm{x}, u) & \text{in} \quad Q_T = \Omega \times (0, T), \\
            \partial_{n}u = 0 & \text{on} \quad S_T = \partial \Omega \times (0, T), \\
            u(\bm{x}, 0) = g & \text{in} \quad \Omega,
        \end{dcases}
        \label{eq:fisher-kpp-bvp}
    \end{equation}
    where \(g = g(\bm{x})\) is the initial population density.
\end{frame}

% ======================== Second Frame ========================

\begin{frame}
    \frametitle{Long-time behavior}
    The long-time behavior of the solution of \eqref{eq:fisher-kpp-bvp} is related to the existence of nontrivial steady states \(U = U(\bm{x})\), satisfying:
    \begin{equation}
        \begin{dcases}
            -D \Delta U = f(\bm{x}, U) & \text{in} \quad \Omega, \\
            \partial_{n}U = 0 & \text{on} \quad \partial \Omega.
        \end{dcases}
        \label{eq:fisher-kpp-steady}
    \end{equation}
    This is necessary to understand questions about persistence and extinction of the population.
\end{frame}

% ======================== Third Frame ========================

\begin{frame}
    \frametitle{Supersolutions and subsolutions I}
    In order to understand the behavior of the solution of \eqref{eq:fisher-kpp-bvp}, it is necessary to define supersolutions and subsolutions. 
    \begin{definition}
        A function \(u \in L^2(0, T; V)\) with \(\dot{u} \in L^2(0, T; L^2(\Omega))\), is a weak supersolution (resp. subsolution) of \eqref{eq:fisher-kpp-bvp} if \(u(\bm{x}, 0) \geq g\) (resp. \(u(\bm{x}, 0) \leq g\)) and
        \begin{equation}
            \begin{split}
                \left(\dot{u}(t), v\right)_0 + D \left(\nabla u(t), \nabla v\right)_0 \geq (f(\bm{x}, u(t)), v)_0 \quad \text{(resp } \leq \text{)} \\ \quad \text{for a.e. } t \in (0, T), \forall v \in V, v \geq 0 \text{ a.e. in } \Omega.
            \end{split}
        \end{equation}
    \end{definition}
\end{frame}

% ======================== Fourth Frame ========================

\begin{frame}
    \frametitle{Supersolutions and subsolutions II}
    Another important result related to supersolutions and subsolutions is the following:
    \begin{theorem}
        Let \(\overline{u}\) and \(\underline{u}\) be a bounded weak supersolution and subsolution of \eqref{eq:fisher-kpp-bvp}, respectively. Then, \(\exists!\) weak solution \(u = u_g\) such that
        \begin{equation}
            \underline{u} \leq u_g \leq \overline{u} \quad \text{a.e. in } \quad Q_T.
        \end{equation}
        \label{thm:10.18}
    \end{theorem}
    Thanks to this theorem, it is possible to find a solution that is global in time, meaning that it exists for all \(t > 0\).
\end{frame}

% ======================== Fifth Frame ========================

\begin{frame}
    \frametitle{Supersolutions and subsolutions III}
    \begin{lemma}
        Let \(\phi, \psi \in L^\infty(\Omega) \cap H^1(\Omega)\) be time independent subsolution and supersolution of \eqref{eq:fisher-kpp-bvp}, respectively, and let \(u_\phi, u_\psi\) be the corresponding solutions of the same problem, with initial condition \(u_\phi(0) = \phi\) and \(u_\psi(0) = \psi\). Then
        \begin{enumerate}
            \item \(u_\phi, u_\psi \text{ and } u_g\) exists for all \(t \geq 0\), and 
            \begin{equation}
                \begin{split}
                    \phi(\bm{x}) \leq u_\phi(\bm{x}, t) \leq u_g(\bm{x}, t) \leq u_\psi(\bm{x}, t) \leq \psi(\bm{x}) \\ \quad \text{a.e in } \Omega \times [0, +\infty) \text{ for all }\phi \leq g \leq \psi,
                \end{split}
            \end{equation}
            \item \(u_\phi(\bm{x}, t_2) \geq u_\phi(\bm{x}, t_1)\) a.e. in \(\Omega\), if \(t_2 > t_1\);
            \item \(u_\psi(\bm{x}, t_2) \leq u_\psi(\bm{x}, t_1)\) a.e. in \(\Omega\), if \(t_2 < t_1\).
        \end{enumerate}
        \label{lem:10.21}
    \end{lemma}
\end{frame}