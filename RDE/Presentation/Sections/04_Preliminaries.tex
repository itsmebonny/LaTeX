\section{Preliminaries}

% ======================== First Frame ========================

\begin{frame}[allowframebreaks]
    \frametitle{Comparison principle}
    \begin{lemma}
        Assume that \(c_i(x,t)\) is bounded on \(\real_i \times [0, T]\) for \(i = U, L\) for some \(0 < T < +\infty\). Let \(w_i \in C(\overline{\real}_i \times [0, T]) \cap C^{1,2}(\real_i \times (0, T])\) satisfy
        \begin{equation*}
            \begin{dcases}
                \partial_t w_i - \partial_{xx} w_i + \beta_i \partial_x w_i + c_i(x,t)w_i \leq 0, & x \in \real_i, 0 < t < T, \\
                w_L(0, t) = w_{U}(0, t), & t > 0 \\
                a_{U} \partial_x w_{U}(0, t) = a_L \partial_x w_L(0, t), & 0 < t < T, \\
                w_i(x, 0) \leq 0, & x \in \real_i,
            \end{dcases}
        \end{equation*}
        \theorembreak
        and 
        \begin{equation}
            \liminf_{R \to +\infty} e^{-cR^2}\left[\max_{\substack{0\leq t\leq T \\ \lvert x \rvert = R}} w_i(x,t)\right] \leq 0
            \label{eq:2.1}
        \end{equation}
        for some \(c > 0\). Then
        \[
            w_i(x,t) \leq 0 \quad \text{for all } x \in \real_i, 0 \leq t \leq T.
        \]
        If \(w_i(x,0)\) is strictly lower than zero then the strict inequality holds. 
        \label{lem:2.1}
    \end{lemma}
\end{frame}

% ======================== Second Frame ========================

\begin{frame}[allowframebreaks]
    \frametitle{Supersolutions and subsolutions}
    \begin{definition}
        If \((\tilde{w}_L, \tilde{w}_{U})\) with \(w_i \in C(\overline{\real}_i \times [0, T]) \cap C^{1,2}(\real_i \times (0, T])\) satisfies 
        \begin{equation}
            \begin{dcases}
                \partial_t \tilde{w}_i - \partial_{xx} \tilde{w}_i + \beta_i \partial_x \tilde{w}_i \geq (\leq) f_i(\tilde{w}_i), & x \in \real_i, 0 < t < T, \\
                \tilde{w}_L(0, t) = \tilde{w}_{U}(0, t), & t > 0 \\
                    a_{U} \partial_x \tilde{w}_{U}(0, t) - a_L \partial_x \tilde{w_L}(0, t) \geq (\leq) 0, & 0 < t < T, \\
            \end{dcases}
            \label{eq:2.2}
        \end{equation}
        then \((\tilde{w}_L, \tilde{w}_{U})\) is a supersolution (subsolution) of \ref{eq:2.2}
        \label{def:2.1}
    \end{definition}
    \framebreak
    Using Lemma \ref{lem:2.1} and Definition \ref{def:2.1}
    \begin{lemma}
        Assume \(f_i(w)\) is locally Lipschitz \((i = U, L)\). Let \((\underline{w}_L, \underline{w}_{U})\) and \((\overline{w}_L, \overline{w}_{U})\) be, respectively, a bounded subsolution and bounded supersolution of \ref{eq:2.2} satisfying \(\underline{w}_i(\cdot, 0) \leq \overline{w}_i(\cdot, 0)\). Then, \(\underline{w}_i \leq \overline{w}_i\). Additionally, if \(\underline{w}_j(x,0) < \overline{w}_j(x,0)\) in a strict sense for some \(j = U, L\), then \(\underline{w}_i < \overline{w}_i\).
        \label{lem:2.2} 
    \end{lemma}
\end{frame}

% ======================== Third Frame ========================

\begin{frame}
    \frametitle{Existence and uniqueness of solutions}
    Existence and uniqueness of solutions for the system \ref{eq:1.6} follows from the following theorem.
    \begin{theorem}
        For any nonnegative initial data \((w_{L,0}, w_{U,0}) \in C_{comp}(\real) \times C_{comp}(\real) \) \, the problem \eqref{eq:1.6} has a unique classical solution \((w_L, w_{U})\) which is defined and uniformly bounded for all \(t > 0\).
        \label{thm:2.4}
    \end{theorem}
\end{frame}

% ======================== Fourth Frame ========================

\begin{frame}
    \frametitle{Sketch of the proof}
    \begin{proof}[Sketch of the proof]
        It will be only shown the uniform boundedness of \(w_i\) for \(i = U, L\).
        It is possible to prove that the half-line problem has a unique solution \(w_L^\ell, w_{U}^\ell\), where the superscript \(\ell\) stands for the extremum of the domain \((0, \ell)\). 
        These solutions are nondecreasing in \(\ell\), and it holds 
        \begin{equation}
            0 \leq w_i^\ell(x,t) \leq 1 + \sup_{\real_L} w_{L,0} + \sup_{\real_{U}} w_{U,0}, \quad i = U, L.
            \label{eq:2.4}
        \end{equation}
        So, the limit \(\ell \to +\infty\) exists, denoted as \(w_i^\infty\).
        It is possible to show that \(w_i^\ell\) converges to \(w_i^\infty\), and that \(w_i^\infty\) is a solution of the original problem \ref{eq:1.6}. 
        At this point, by uniqueness, \(w_i = w_i^\infty\), and thanks to \eqref{eq:2.4}, the solution is uniformly bounded.
    \end{proof}
\end{frame}

% ======================== Fifth Frame ========================
% \begin{frame}
%     \frametitle{Sketch of the proof}
%     \begin{proofe}
        
%     \end{proofe}
% \end{frame}
    