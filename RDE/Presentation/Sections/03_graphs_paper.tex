\section{KPP-Fisher equation over simple graphs}

% ======================== First Frame ========================

\begin{frame}
    \frametitle{The river network}
    Instead of considering a river as a continuous line, here it will be taken into consideration the fact that rivers can divide into smaller rivers and merge into larger rivers. This can be modeled as a graph, where the edges are the rivers and the nodes are the junctions between rivers.
    \tikzset{every picture/.style={line width=0.75pt}} %set default line width to 0.75pt 
\begin{figure}[H]
    \centering
    \scalebox{0.6}{\begin{tikzpicture}[x=0.75pt,y=0.75pt,yscale=-1,xscale=1]
%uncomment if require: \path (0,300); %set diagram left start at 0, and has height of 300

%Straight Lines [id:da06835250339153032] 
\draw    (27.17,139.58) -- (116.96,139.51) ;
\draw [shift={(77.06,139.54)}, rotate = 179.95] [fill={rgb, 255:red, 0; green, 0; blue, 0 }  ][line width=0.08]  [draw opacity=0] (8.93,-4.29) -- (0,0) -- (8.93,4.29) -- cycle    ;
%Shape: Circle [id:dp006568808299004081] 
\draw  [fill={rgb, 255:red, 0; green, 0; blue, 0 }  ,fill opacity=1 ] (114.63,139.51) .. controls (114.63,138.22) and (115.67,137.18) .. (116.96,137.18) .. controls (118.25,137.18) and (119.3,138.22) .. (119.3,139.51) .. controls (119.3,140.8) and (118.25,141.84) .. (116.96,141.84) .. controls (115.67,141.84) and (114.63,140.8) .. (114.63,139.51) -- cycle ;
%Straight Lines [id:da5818375484660003] 
\draw    (456.92,139.33) -- (551.24,139.67) ;
\draw [shift={(509.08,139.52)}, rotate = 180.2] [fill={rgb, 255:red, 0; green, 0; blue, 0 }  ][line width=0.08]  [draw opacity=0] (8.93,-4.29) -- (0,0) -- (8.93,4.29) -- cycle    ;
%Straight Lines [id:da41798667092826824] 
\draw    (120,139.58) -- (210.03,139.58) ;
\draw [shift={(170.01,139.58)}, rotate = 180] [fill={rgb, 255:red, 0; green, 0; blue, 0 }  ][line width=0.08]  [draw opacity=0] (8.93,-4.29) -- (0,0) -- (8.93,4.29) -- cycle    ;
%Shape: Circle [id:dp9495741628591831] 
\draw  [fill={rgb, 255:red, 0; green, 0; blue, 0 }  ,fill opacity=1 ] (548.91,139.67) .. controls (548.91,138.38) and (549.95,137.33) .. (551.24,137.33) .. controls (552.53,137.33) and (553.57,138.38) .. (553.57,139.67) .. controls (553.57,140.96) and (552.53,142) .. (551.24,142) .. controls (549.95,142) and (548.91,140.96) .. (548.91,139.67) -- cycle ;
%Shape: Boxed Line [id:dp3634037576105641] 
\draw    (551.24,139.67) -- (632.76,187.12) ;
\draw [shift={(596.32,165.91)}, rotate = 210.2] [fill={rgb, 255:red, 0; green, 0; blue, 0 }  ][line width=0.08]  [draw opacity=0] (8.93,-4.29) -- (0,0) -- (8.93,4.29) -- cycle    ;
%Shape: Boxed Line [id:dp5855966408278438] 
\draw    (551.24,139.67) -- (633.09,92.79) ;
\draw [shift={(596.5,113.75)}, rotate = 150.2] [fill={rgb, 255:red, 0; green, 0; blue, 0 }  ][line width=0.08]  [draw opacity=0] (8.93,-4.29) -- (0,0) -- (8.93,4.29) -- cycle    ;
%Straight Lines [id:da7091000862619046] 
\draw    (319.92,139.33) -- (414.24,139.67) ;
\draw [shift={(372.08,139.52)}, rotate = 180.2] [fill={rgb, 255:red, 0; green, 0; blue, 0 }  ][line width=0.08]  [draw opacity=0] (8.93,-4.29) -- (0,0) -- (8.93,4.29) -- cycle    ;
%Shape: Boxed Line [id:dp8564221650100632] 
\draw    (240.73,91.88) -- (322.25,139.33) ;
\draw [shift={(285.81,118.12)}, rotate = 210.2] [fill={rgb, 255:red, 0; green, 0; blue, 0 }  ][line width=0.08]  [draw opacity=0] (8.93,-4.29) -- (0,0) -- (8.93,4.29) -- cycle    ;
%Shape: Circle [id:dp5640763569876212] 
\draw  [fill={rgb, 255:red, 0; green, 0; blue, 0 }  ,fill opacity=1 ] (319.92,139.33) .. controls (319.92,138.04) and (320.96,137) .. (322.25,137) .. controls (323.54,137) and (324.58,138.04) .. (324.58,139.33) .. controls (324.58,140.62) and (323.54,141.67) .. (322.25,141.67) .. controls (320.96,141.67) and (319.92,140.62) .. (319.92,139.33) -- cycle ;
%Shape: Boxed Line [id:dp8885765217793992] 
\draw    (240.4,186.21) -- (322.25,139.33) ;
\draw [shift={(285.66,160.28)}, rotate = 150.2] [fill={rgb, 255:red, 0; green, 0; blue, 0 }  ][line width=0.08]  [draw opacity=0] (8.93,-4.29) -- (0,0) -- (8.93,4.29) -- cycle    ;


% Text Node
\draw (35,115.4) node [anchor=north west][inner sep=0.75pt]    {$R_{U}{}{}$};
% Text Node
\draw (465,115.4) node [anchor=north west][inner sep=0.75pt]    {$R_{U}{}{}$};
% Text Node
\draw (169,115.4) node [anchor=north west][inner sep=0.75pt]    {$R_{L}{}{}$};
% Text Node
\draw (378,115.4) node [anchor=north west][inner sep=0.75pt]    {$R_{L}{}{}$};
% Text Node
\draw (592,76.4) node [anchor=north west][inner sep=0.75pt]    {$R_{L_{1}}{}{}$};
% Text Node
\draw (592,173.4) node [anchor=north west][inner sep=0.75pt]    {$R_{L_{2}}{}{}$};
% Text Node
\draw (252,76.4) node [anchor=north west][inner sep=0.75pt]    {$R_{U_{1}}{}{}$};
% Text Node
\draw (254,177.4) node [anchor=north west][inner sep=0.75pt]    {$R_{U_{2}}{}{}$};
% Text Node
\draw (28,200) node [anchor=north west][inner sep=0.75pt]   [align=left] {a)};
% Text Node
\draw (240,200) node [anchor=north west][inner sep=0.75pt]   [align=left] {b)};
% Text Node
\draw (458,200) node [anchor=north west][inner sep=0.75pt]   [align=left] {c)};


\end{tikzpicture}}
    \caption{Graph representation of the river. a) The river is split in an upper branch \(R_U\) and a lower branch \(R_L\). b) The upper branch is split in two branches \(R_{U_1}\) and \(R_{U_2}\). c) The lower branch is split in two branches \(R_{L_1}\) and \(R_{L_2}\).}
    \label{fig:river-graph}
\end{figure}
\end{frame}

% ======================== Second Frame ========================

\begin{frame}
    \frametitle{Steady states}
    With the graph based approach, infinitely many steady states can exist. However, only one of them can be a global attractor. And these global attractors can be summed up in three types:
    \begin{enumerate}
        \item \textbf{washing out:} the population density goes to zero in all the branches;
        \item \textbf{persistence at carrying capacity:} the population survives in all the branches;
        \item \textbf{persistence below carrying capacity:} the population density goes to a value below the carrying capacity in some branches.
    \end{enumerate}
\end{frame}

% ======================== Third Frame ========================

\begin{frame}
    \frametitle{One upper branch and one lower branch I}
    Consider the case where the river is split in an upper branch \(R_U\) and a lower branch \(R_L\). The system of equations is given by
    \begin{equation}
        \begin{dcases}
            \partial_t w_L - \partial_{xx}w_L + \beta_L \partial_x w_L = w_L(1 - w_L) &  x \in \real_L, t > 0, \\
            \partial_t w_U - \partial_{xx}w_U + \beta_U \partial_x w_U = w_U(1 - w_U) & x \in \real_U, t > 0, \\
            w_L(0, t) = w_U(0, t) &  t > 0, \\
            a_L \partial_x w_L(0, t) = a_U \partial_x w_U(0, t) &  t > 0, \\
            w_L(x, 0) = w_{0}(x) &  x \in \real_L, \\
            w_U(x, 0) = w_{0}(x) &  x \in \real_U.
        \end{dcases}
        \label{eq:1.6}
    \end{equation}
\end{frame}

% ======================== Fourth Frame ========================

\begin{frame}
    \frametitle{One upper branch and one lower branch II}
    In Equation \eqref{eq:1.6} the initial condition is the same for both branches and is given by \(w_0(x) \in C_{comp}(\real)\), where \(C_{comp}(\real)\) is the set of continuous functions with compact support in \(\real\).

    The compactness of the support of the initial condition is crucial to determine the long-time behavior of the solution. In fact, the steady state chosen as the global attractor is the one that decays to zero faster at infinity.
\end{frame}

% ======================== Fifth Frame ========================

\begin{frame}
    \frametitle{One upper branch and one lower branch III}
    The long-time behavior of the solution of \eqref{eq:1.6} is related to the solutions of the stationary problem
    \begin{equation}
        \begin{dcases}
            - \ddot{\phi}_L + \beta_L \dot{\phi}_L = \phi_L(1 - \phi_L), & 0 < \phi_L < 1, x \in \real_L, \\
            - \ddot{\phi}_U + \beta_U \dot{\phi}_U = \phi_U(1 - \phi_U), & 0 < \phi_U < 1, x \in \real_U, \\
            \phi_L(0) = \phi_U(0) = \alpha \in [0, 1], \\
            a_L \dot{\phi}_L(0) = a_U \dot{\phi}_U(0).
        \end{dcases}
        \label{eq:1.8}
    \end{equation}
\end{frame}

% ======================== Sixth Frame ========================

\begin{frame}[allowframebreaks]
    \frametitle{Behaviour of solutions}
    The following theorems shed some light on the behavior of the solutions of \eqref{eq:1.6} and their long-time dynamics.
    \begin{theorem}
        Since the assumption is that \(D_L = D_U = 1\), the maximum speed of the population invasion is \(c_* = 2\). 
        
        (i) If \(0 < \beta_U < 2\), then \eqref{eq:1.8} has no solution for \(\alpha \in (0, 1)\). 


        (ii) If \(\beta_L, \beta_U \geq 2\), then for every \(\alpha \in (0, 1)\) \eqref{eq:1.8} has a unique solution.


        (iii) If \(\beta_U \geq 2 > \beta_L > 0\), then there exists \(\alpha_0 \in (0, 1)\) such that \eqref{eq:1.8} has a unique solution for each \(\alpha \in [\alpha_0, 1)\), and no solution for \(\alpha \in (0, \alpha_0)\).

        \theorembreak
        (iv)  Whenever \eqref{eq:1.8} has a solution \((\phi_L, \phi_U)\), it is true that 
            \[
                \dot{\phi}_L > 0, \quad \dot{\phi}_U > 0, \quad \phi_L(\infty) = 1, \quad \phi_U(-\infty) = 0.
            \]
            Moreover, in case (ii) and (iii) with \(\alpha \in [\alpha_0, 1)\), as \(x \to -\infty\), there exists some \(c = c(\alpha) > 0\) such that
            \begin{equation}
                \phi_U(x) = \begin{cases}
                    (c + o(1))e^{\frac{1}{2}(\beta_U - \sqrt{\beta_U^2 - 4})x} & \text{if } \beta_U > 2, \\
                    (c + o(1))\lvert x\rvert e^{x} & \text{if } \beta_U = 2, \\
                \end{cases} 
                \label{eq:1.9}
            \end{equation}
        \theorembreak
            while in case (iii) with \(\alpha = \alpha_0\), as \(x \to -\infty\), there exists some \(c > 0\) such that
            \begin{equation}
                \phi_U(x) = (c + o(1))e^{\frac{1}{2}(\beta_U - \sqrt{\beta_U^2 - 4})x}.
                \label{eq:1.10}
            \end{equation}
        \label{thm:1.1}
    \end{theorem}
\end{frame}

% ======================== Seventh Frame ========================

\begin{frame}[allowframebreaks]
    \frametitle{Long-time dynamics}
    \begin{theorem}
        Assuming that \(w_0 \in C_{comp}(\real)\) is nonnegative and nontrivial, let \((w_L, w_U)\) be the solution of \eqref{eq:1.6}. Then the following holds:

            (i) If \(0 < \beta_U < 2\), then \((w_L(\cdot, t), w_U(\cdot, t)) \to (1, 1)\) locally uniformly as \(t \to +\infty\).

            (ii) If \(\beta_U, \beta_L \geq 2\), then \((w_L(\cdot, t), w_U(\cdot, t)) \to (0, 0)\) locally uniformly as \(t \to +\infty\). Also, \(\norm{w_L(\cdot, t)}_{L^\infty(\real_L)} \to 1\) and \(\norm{w_U(\cdot, t)}_{L^\infty(\real_U)} \to 0\) as \(t \to +\infty\).
            \theorembreak
            (iii) If \(\beta_U \geq 2 > \beta_L > 0\), then \((w_L(\cdot, t), w_U(\cdot, t)) \to (\phi_L(\cdot; \alpha_0), \phi_U(\cdot; \alpha_0))\)\footnote{\((\phi_L(\cdot; \alpha_0), \phi_U(\cdot; \alpha_0))\) is the unique solution of \eqref{eq:1.8} with \(\alpha = \alpha_0\).} locally uniformly as \(t \to +\infty\).
        \label{thm:1.2}
    \end{theorem}
\end{frame}

% ======================== Eighth Frame ========================

\begin{frame}
    \frametitle{Long-time dynamics III}
    In case (ii) of Theorem \ref{thm:1.2}, the population density goes to zero locally in all the branches, because the river's flow is too strong for the population to survive. However, the \(L^\infty\) norm of the population density in the lower branch goes to \(1\), meaning that the population is not out of the network.

    In case (i) and (ii) it is clear that it's far more important the advection coefficient of the upper branch.

    The last case shows that the steady state solution is increasing in both branches, and its limit at \(x \to -\infty\) is 0, while at \(x \to +\infty\) is 1. This is the selected steady state because it decays to zero faster at infinity.
\end{frame}