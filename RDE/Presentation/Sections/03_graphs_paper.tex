\section{KPP-Fisher equation over simple graphs}

% ======================== First Frame ========================

\begin{frame}
    \frametitle{The river network}
    Instead of considering a river as a continuous line, here it will be taken into consideration the fact that rivers can divide into smaller rivers and merge into larger rivers. This can be modeled as a graph, where the edges are the rivers and the nodes are the junctions between rivers.
    \tikzset{every picture/.style={line width=0.75pt}} %set default line width to 0.75pt 
\begin{figure}[H]
    \centering
    \scalebox{0.6}{\begin{tikzpicture}[x=0.75pt,y=0.75pt,yscale=-1,xscale=1]
%uncomment if require: \path (0,300); %set diagram left start at 0, and has height of 300

%Straight Lines [id:da06835250339153032] 
\draw    (27.17,139.58) -- (116.96,139.51) ;
\draw [shift={(77.06,139.54)}, rotate = 179.95] [fill={rgb, 255:red, 0; green, 0; blue, 0 }  ][line width=0.08]  [draw opacity=0] (8.93,-4.29) -- (0,0) -- (8.93,4.29) -- cycle    ;
%Shape: Circle [id:dp006568808299004081] 
\draw  [fill={rgb, 255:red, 0; green, 0; blue, 0 }  ,fill opacity=1 ] (114.63,139.51) .. controls (114.63,138.22) and (115.67,137.18) .. (116.96,137.18) .. controls (118.25,137.18) and (119.3,138.22) .. (119.3,139.51) .. controls (119.3,140.8) and (118.25,141.84) .. (116.96,141.84) .. controls (115.67,141.84) and (114.63,140.8) .. (114.63,139.51) -- cycle ;
%Straight Lines [id:da5818375484660003] 
\draw    (456.92,139.33) -- (551.24,139.67) ;
\draw [shift={(509.08,139.52)}, rotate = 180.2] [fill={rgb, 255:red, 0; green, 0; blue, 0 }  ][line width=0.08]  [draw opacity=0] (8.93,-4.29) -- (0,0) -- (8.93,4.29) -- cycle    ;
%Straight Lines [id:da41798667092826824] 
\draw    (120,139.58) -- (210.03,139.58) ;
\draw [shift={(170.01,139.58)}, rotate = 180] [fill={rgb, 255:red, 0; green, 0; blue, 0 }  ][line width=0.08]  [draw opacity=0] (8.93,-4.29) -- (0,0) -- (8.93,4.29) -- cycle    ;
%Shape: Circle [id:dp9495741628591831] 
\draw  [fill={rgb, 255:red, 0; green, 0; blue, 0 }  ,fill opacity=1 ] (548.91,139.67) .. controls (548.91,138.38) and (549.95,137.33) .. (551.24,137.33) .. controls (552.53,137.33) and (553.57,138.38) .. (553.57,139.67) .. controls (553.57,140.96) and (552.53,142) .. (551.24,142) .. controls (549.95,142) and (548.91,140.96) .. (548.91,139.67) -- cycle ;
%Shape: Boxed Line [id:dp3634037576105641] 
\draw    (551.24,139.67) -- (632.76,187.12) ;
\draw [shift={(596.32,165.91)}, rotate = 210.2] [fill={rgb, 255:red, 0; green, 0; blue, 0 }  ][line width=0.08]  [draw opacity=0] (8.93,-4.29) -- (0,0) -- (8.93,4.29) -- cycle    ;
%Shape: Boxed Line [id:dp5855966408278438] 
\draw    (551.24,139.67) -- (633.09,92.79) ;
\draw [shift={(596.5,113.75)}, rotate = 150.2] [fill={rgb, 255:red, 0; green, 0; blue, 0 }  ][line width=0.08]  [draw opacity=0] (8.93,-4.29) -- (0,0) -- (8.93,4.29) -- cycle    ;
%Straight Lines [id:da7091000862619046] 
\draw    (319.92,139.33) -- (414.24,139.67) ;
\draw [shift={(372.08,139.52)}, rotate = 180.2] [fill={rgb, 255:red, 0; green, 0; blue, 0 }  ][line width=0.08]  [draw opacity=0] (8.93,-4.29) -- (0,0) -- (8.93,4.29) -- cycle    ;
%Shape: Boxed Line [id:dp8564221650100632] 
\draw    (240.73,91.88) -- (322.25,139.33) ;
\draw [shift={(285.81,118.12)}, rotate = 210.2] [fill={rgb, 255:red, 0; green, 0; blue, 0 }  ][line width=0.08]  [draw opacity=0] (8.93,-4.29) -- (0,0) -- (8.93,4.29) -- cycle    ;
%Shape: Circle [id:dp5640763569876212] 
\draw  [fill={rgb, 255:red, 0; green, 0; blue, 0 }  ,fill opacity=1 ] (319.92,139.33) .. controls (319.92,138.04) and (320.96,137) .. (322.25,137) .. controls (323.54,137) and (324.58,138.04) .. (324.58,139.33) .. controls (324.58,140.62) and (323.54,141.67) .. (322.25,141.67) .. controls (320.96,141.67) and (319.92,140.62) .. (319.92,139.33) -- cycle ;
%Shape: Boxed Line [id:dp8885765217793992] 
\draw    (240.4,186.21) -- (322.25,139.33) ;
\draw [shift={(285.66,160.28)}, rotate = 150.2] [fill={rgb, 255:red, 0; green, 0; blue, 0 }  ][line width=0.08]  [draw opacity=0] (8.93,-4.29) -- (0,0) -- (8.93,4.29) -- cycle    ;


% Text Node
\draw (35,115.4) node [anchor=north west][inner sep=0.75pt]    {$R_{U}{}{}$};
% Text Node
\draw (465,115.4) node [anchor=north west][inner sep=0.75pt]    {$R_{U}{}{}$};
% Text Node
\draw (169,115.4) node [anchor=north west][inner sep=0.75pt]    {$R_{L}{}{}$};
% Text Node
\draw (378,115.4) node [anchor=north west][inner sep=0.75pt]    {$R_{L}{}{}$};
% Text Node
\draw (592,76.4) node [anchor=north west][inner sep=0.75pt]    {$R_{L_{1}}{}{}$};
% Text Node
\draw (592,173.4) node [anchor=north west][inner sep=0.75pt]    {$R_{L_{2}}{}{}$};
% Text Node
\draw (252,76.4) node [anchor=north west][inner sep=0.75pt]    {$R_{U_{1}}{}{}$};
% Text Node
\draw (254,177.4) node [anchor=north west][inner sep=0.75pt]    {$R_{U_{2}}{}{}$};
% Text Node
\draw (28,200) node [anchor=north west][inner sep=0.75pt]   [align=left] {a)};
% Text Node
\draw (240,200) node [anchor=north west][inner sep=0.75pt]   [align=left] {b)};
% Text Node
\draw (458,200) node [anchor=north west][inner sep=0.75pt]   [align=left] {c)};


\end{tikzpicture}}
    \caption{Graph representation of the river. a) The river is split in an upper branch \(R_U\) and a lower branch \(R_L\). b) The upper branch is split in two branches \(R_{U_1}\) and \(R_{U_2}\). c) The lower branch is split in two branches \(R_{L_1}\) and \(R_{L_2}\).}
    \label{fig:river-graph}
\end{figure}
\end{frame}

% ======================== Second Frame ========================

\begin{frame}
    \frametitle{Steady states}
    With the graph based approach, infinitely many steady states can exist. However, only one of them can be a global attractor. And these global attractors can be summed up in three types:
    \begin{enumerate}
        \item \textbf{washing out:} the population density goes to zero in all the branches;
        \item \textbf{persistence at carrying capacity:} the population survives in all the branches;
        \item \textbf{persistence below carrying capacity:} the population density goes to a value below the carrying capacity in some branches.
    \end{enumerate}
\end{frame}