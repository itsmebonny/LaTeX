\section{One upper branch and one lower branch}


% ========================= First Frame =========================
% \begin{frame}
%     \frametitle{Proof of Theorem \ref{thm:1.2}}
%     Now it is possible to prove Theorem \ref{thm:1.2} using Theorem \ref{thm:1.1}.
    
%     Each part of the theorem will be proved separately with a sketch of the proof.

%     The proof of Theorem \ref{thm:1.1} is not provided here, but it's based on a phase-plane analysis approach.

% \end{frame}

% ========================= Second Frame =========================

\begin{frame}
    \frametitle{Theorem \ref{thm:1.2}: \texorpdfstring{\(\beta_U < 2\)}{betaU < 2}}
    \begin{theorem}
        Assume that \(\beta_U < 2\) and the initial datum has compact support \(w_{0} \in C_{comp}(\real)\), nontrivial and nonnegative. Let \((w_L, w_{U})\) be the solution of \eqref{eq:1.6}. Then,
        \[
            (w_L(\cdot, t), w_{U}(\cdot, t)) \to (1, 1) \quad \text{locally uniformly as } t \to +\infty.
        \]
        \label{thm:3.1} 
    \end{theorem}
\end{frame}

% ========================= Third Frame =========================

\begin{frame}
    \frametitle{Proof of Theorem \ref{thm:3.1} I}
    \begin{proofs}[Sketch of the proof]
        From Theorem \ref{thm:2.4} is known that 
    \begin{equation}
        0 \leq w_{i} \leq 1 + \sup_{\real} w_{0}, \quad i = U, L.
        \label{eq:3.1}
    \end{equation}
    As \(\norm{w_{0}}_{L^{\infty}(\real)} > 0\), take \(\overline{u}(t) = 1 + \norm{w_{0}}_{L^{\infty}(\real)}e^{-t}\). 
    It is possible to prove that \((\overline{u}, \overline{u})\) is a supersolution of \eqref{eq:1.6}. Using Lemma \ref{lem:2.2} and \eqref{eq:3.1}, it is possible to obtain 
    \begin{equation}
        1 = \lim_{t \to +\infty} \overline{u}(t) \geq \limsup_{t \to +\infty} \norm{w_i(\cdot, t)}_{L^{\infty}(\real)}, \quad i = U, L.
        \label{eq:3.2}
    \end{equation}
    \end{proofs}
\end{frame}

% ========================= Fourth Frame =========================
\begin{frame}
    \frametitle{Proof of Theorem \ref{thm:3.1} II}
    \begin{proofc}
        From the assumptions, \(w_0\) is nontrivial and nonnegative, so \(w_U(x, 1) > 0\) for \(x \in \real_U\). Since \(0 < \beta_U < 2\), there exists a unique constant \(\ell_0 > 0\) such that the following problem
        \begin{equation}
            \begin{dcases}
                -\ddot{w} + \beta_U \dot{w} = w(1 - w), & x \in (-\ell, 0), \\
                w(0) = w(-\ell) = 0
            \end{dcases}
            \label{eq:3.3}
        \end{equation}
        has a positive solution iff \(\ell > \ell_0\), and the positive solution \(w_\ell\) is unique. Fixing \(\ell\), it is possible to ensure that the unique solution \(w_\ell\) of \eqref{eq:3.3} satisfies \(w_\ell(x) < w_U(x, 1)\) for \(x \in [-\ell, 0]\).
    \end{proofc}
\end{frame}

% ========================= Fifth Frame =========================
\begin{frame}
    \frametitle{Proof of Theorem \ref{thm:3.1} III}
    \begin{proofc}
        Now set 
    \begin{equation*}
        w_\ell^0(x) \coloneqq \begin{cases}
            w_\ell & \text{if } x \in [-\ell, 0], \\
            0 & \text{if } x \in (-\infty, -\ell).
        \end{cases}
    \end{equation*}
    Then, let \((\underline{w}_L. \underline{w}_U)\) be the solution of \eqref{eq:1.6} with initial datum \((0, w_\ell^0)\). Clearly,
    \begin{align*}
        \underline{w}_U(x, t) > 0 & \text{ for } x \in \real_U, t > 0, \\
        \underline{w}_L(x, t) > 0 & \text{ for } x \in \real_L, t > 0.
    \end{align*}
    By the parabolic comparison principle it is possible to conclude that \(\underline{w}_U \geq w_0^l\) for all \((x,t) \in [-\ell, 0] \times [0, +\infty)\).
    \end{proofc}
\end{frame}

% ========================= Sixth Frame =========================

\begin{frame}
    \frametitle{Proof of Theorem \ref{thm:3.1} IV}
    \begin{proofc}
        Hence,
        \begin{equation*}
            (\underline{w}_L(\cdot, t), \underline{w}_U(\cdot, t)) \geq (0, w_\ell^0) \quad \text{for all } t > 0.
        \end{equation*}
        Using Lemma \ref{lem:2.2} it follows
        \begin{equation*}
            (\underline{w}_L(\cdot, t + \delta), \underline{w}_U(\cdot, t + \delta)) \geq (\underline{w}_L(\cdot, t), \underline{w}_U(\cdot, t)) \quad \text{for all } t > 0,
        \end{equation*}
        meaning that \((\underline{w}_L, \underline{w}_U)\) is nondecreasing in \(t\). 
    \end{proofc}
\end{frame}

% ========================= Seventh Frame =========================

\begin{frame}
    \frametitle{Proof of Theorem \ref{thm:3.1} V}
    \begin{proofc}
        By denoting 
        \[
            \left(\underline{w}_{L, \infty}(x), \underline{w}_{U, \infty}(x)\right) \coloneqq \left(\lim_{t \to +\infty} \underline{w}_L(x, t), \lim_{t \to +\infty} \underline{w}_U(x, t)\right), 
        \]
        it is possible to use a similar argument as in proof of Theorem \ref{thm:2.4} to conclude that
        \[
            (\underline{w}_{L}(\cdot, t), \underline{w}_{U}(\cdot, t)) \to (\underline{w}_{L, \infty}(\cdot), \underline{w}_{U, \infty}(\cdot)),
        \]
        locally uniformly as \(t \to +\infty\) and that \((\underline{w}_{L, \infty}(x), \underline{w}_{U, \infty}(x))\) is a positive stationary solution of \eqref{eq:1.6}. 
    \end{proofc}
\end{frame}

% ========================= Eighth Frame =========================

\begin{frame}
    \frametitle{Proof of Theorem \ref{thm:3.1} VI}
    \begin{proofc}
        By Theorem \ref{thm:1.1}, the only possible positive stationary solution is \((1, 1)\), therefore
        \begin{equation}
            (\underline{w}_{L}(\cdot, t), \underline{w}_{U}(\cdot, t)) \to (1, 1) \quad \text{locally uniformly as } t \to +\infty.
            \label{eq:3.4}
        \end{equation}
        On the other hand, as \(w_\ell^0(x) < w_U(x, 1)\) for \(x \in (-\infty, 0)\), thanks to Lemma \ref{lem:2.2},
        \[
            (\underline{w}_L(\cdot, t), \underline{w}_U(\cdot, t)) \leq (\underline{w}_L(\cdot, t + 1), \underline{w}_U(\cdot, t + 1)) \quad \text{for all } t > 0.
        \]
    \end{proofc}
\end{frame}

% ========================= Ninth Frame =========================

\begin{frame}
    \frametitle{Proof of Theorem \ref{thm:3.1} VII}
    \begin{proofe}
        This and \eqref{eq:3.4} imply that
        \begin{equation}
            (\liminf_{t \to +\infty} w_L(\cdot, t), \liminf_{t \to +\infty} w_U(\cdot, t)) \to (1, 1),
            \label{eq:3.5}
        \end{equation}
        locally uniformly as \(t \to +\infty\).
        Combining \eqref{eq:3.2} and \eqref{eq:3.5} concludes the proof.
    \end{proofe}
\end{frame}
        