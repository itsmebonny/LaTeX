% =================== First Frame ===================

\begin{frame}
    \frametitle{Theorem \ref{thm:1.3}: \texorpdfstring{\(\beta_U \geq 2 > \beta_L\)}{betaU >= 2 > betaL}}
    \begin{theorem}
        Assume that \(\beta_U \geq 2 > \beta_L\) and the initial datum has compact support \(w_{0} \in C_{comp}(\real)\), nontrivial and nonnegative. Let \((w_L, w_{U})\) be the solution of \eqref{eq:1.6}. Then,
        \begin{equation}
            (w_L(\cdot, t), w_{U}(\cdot, t)) \to (\phi_L(\cdot; \alpha_0), \phi_U(\cdot; \alpha_0)) \quad \text{locally uniformly as } t \to +\infty.
            \label{eq:3.6}
        \end{equation}
        \label{thm:3.3}
    \end{theorem}
\end{frame}

% =================== Second Frame ===================

\begin{frame}
    \frametitle{Proof of Theorem \ref{thm:3.3} I}
    \begin{proofs}[Sketch of the proof]
        For \(M > 1\), set 
        \begin{align*}
            &k \coloneqq \frac{1\beta_U + \sqrt{\beta_U^2 - 4}}{2}, \\
            &\overline{\phi}_{U,0} \coloneqq M e^{kx}, \quad x \in \real_U, \\
            &\overline{\phi}_{L,0} \coloneqq M, \quad x \in \real_L.
        \end{align*}
        Since \(w_0\) has compact support, it is possible to choose \(M > 1\) large enough that
        \begin{align*}
            \overline{\phi}_{U,0}(x) = M e^{kx} > w_0(x) \quad \text{for } x \in \real_U, \\
            \overline{\phi}_{L,0}(x) = M > w_0(x) \quad \text{for } x \in \real_L. 
        \end{align*}
    \end{proofs}
\end{frame}

% =================== Third Frame ===================

\begin{frame}
    \frametitle{Proof of Theorem \ref{thm:3.3} II}
    \begin{proofc}
        Therefore, \((\overline{\phi}_{L,0}, \overline{\phi}_{U,0})\) is a supersolution of the corresponding elliptic problem of \eqref{eq:1.6}. It follows that the unique solution \((\overline{\phi}_U, \overline{\phi}_L)\) of \eqref{eq:1.6} with initial datum \((\overline{\phi}_{L,0}, \overline{\phi}_{U,0})\) is nondecreasing in \(t\).
        As \(t \to +\infty\),
        \[
                (\overline{\phi}_L(\cdot, t), \overline{\phi}_U(\cdot, t)) \to (\hat{\phi}_L(\cdot), \hat{\phi}_U(\cdot)) \quad \text{locally uniformly},
        \]
        and \((\hat{\phi}_L, \hat{\phi}_U)\) is a nonnegative stationary solution of \eqref{eq:1.6}. Clearly,
        \begin{align*}
            \hat{\phi}_U(x) \leq \overline{\phi}_{U,0}(x) = M e^{kx} \quad \text{for } x \in \real_U, \\
            \hat{\phi}_L(x) \leq \overline{\phi}_{L,0}(x) = M \quad \text{for } x \in \real_L.
        \end{align*}
    \end{proofc}
\end{frame}

% =================== Fourth Frame ===================

\begin{frame}
    \frametitle{Proof of Theorem \ref{thm:3.3} III}
    \begin{proofc}
        Defining 
        \begin{align*}
                \underline{\phi}_{U,0} \coloneqq 0, \quad x \in \real_U, \\
                    \underline{\phi}_{L,0} \coloneqq \phi_l(x), \quad x \in \real_L,
        \end{align*}
        where \(\phi_l(x)\) is the unique solution of
        \begin{equation*}
            \begin{cases}
                -\ddot{\phi} + \beta_L \dot{\phi} = \phi(1 - \phi), & x \in (0, l), \\
                \phi(0) = \phi(l) = 0.
            \end{cases}
        \end{equation*}
        It is possible to prove that \((\underline{\phi}_L, \underline{\phi}_U)\) is a subsolution of the elliptic problem of \eqref{eq:1.6}, and again it follows that the unique solution \((\underline{\phi}_L, \underline{\phi}_U)\) of \eqref{eq:1.6} with initial datum \((\underline{\phi}_{L,0}, \underline{\phi}_{U,0})\) is nondecreasing in \(t\).
    \end{proofc}
\end{frame}

% =================== Fifth Frame ===================

\begin{frame}
    \frametitle{Proof of Theorem \ref{thm:3.3} IV}
    \begin{proofc}
        As \(t \to +\infty\),
        \[
            (\underline{\phi}_L(\cdot, t), \underline{\phi}_U(\cdot, t)) \to (\tilde{\phi}_L(\cdot), \tilde{\phi}_U(\cdot)) \quad \text{locally uniformly},
        \]
        and \((\tilde{\phi}_L, \tilde{\phi}_U)\) is a nonnegative stationary solution of \eqref{eq:1.6}. Since
        \[
            \underline{\phi}_{U,0}(x) < \hat{\phi}_U(x), \quad \underline{\phi}_{L,0}(x) < \hat{\phi}_L(x)
        \]
        it's easy to see
        \begin{align*}
            0 < \tilde{\phi}_U(x) \leq \hat{\phi}_U(x)  \leq M e^{kx} \quad \text{for } x \in \real_U, \\
            \phi_l(x) \leq \tilde{\phi}_L(x) \leq \hat{\phi}_L(x) \leq M \quad \text{for } x \in \real_L.
        \end{align*}
    \end{proofc}
\end{frame}

% =================== Sixth Frame ===================

\begin{frame}
    \frametitle{Proof of Theorem \ref{thm:3.3} V}
    \begin{proofe}
        Then \((\tilde{\phi}_L, \tilde{\phi}_U)\) and \((\hat{\phi}_L, \hat{\phi}_U)\) must be solutions of \eqref{eq:1.8} with some \(\hat{\alpha}, \tilde{\alpha} \in (0,1)\), with \(\hat{\alpha} \geq \tilde{\alpha}\). With Theorem \ref{thm:1.1}(iv) it is possible to prove that \(\hat{\alpha} = \tilde{\alpha} = \alpha_0\), and therefore the two solutions coincide and can be defined as \(\phi_L(\cdot; \alpha_0), \phi_U(\cdot; \alpha_0)\).
        With the same arguments as in the previous cases, it is possible to conclude that
        \begin{align*}
            \liminf_{t \to +\infty} w_U(x, t) \geq \phi_U(x; \alpha_0) \quad \text{locally uniformly for } x \in \real_U, \\
            \limsup_{t \to +\infty} w_L(x, t) \leq \phi_L(x; \alpha_0) \quad \text{locally uniformly for } x \in \real_L.
        \end{align*}
        Therefore, \eqref{eq:3.6} holds.
    \end{proofe}
\end{frame}