\section{Conclusions}

This work analyzed the behavior of the Fisher-KPP equation over simple graphs. Working with infinite graphs was a novelty idea that allowed to study the behavior of the solution in a more general setting, but also add some complexity to the problem. As it was stressed in the previous sections, there are three main types of stationary behavior: \textit{washing out} (the solution converges to zero locally), \textit{persistence at carrying capacity} (the solution converges to 1 locally), and \textit{persistence below carrying capacity} (the solution converges to a value less than 1 locally). These three behaviors are related to the flow speed in the river. This model is able to capture the dynamics of a population in a local river, thanks to the infinite length of its branches. It's also true that once the population has spread inside the system, then it needs to be modeled differently. Also, a lot of factors are not taken into account, such as the presence of predators, seasonal variations, and the presence of other species. The model is a good starting point to understand the dynamics of a population in a river, but even in a heterogeneous environment it is not enough to capture the complexity of the real world.