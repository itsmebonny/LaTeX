\section{Two branches problem} 
Here a brief sketch of the proof of Theorem \ref{thm:1.2} is presented. It is possible to prove Theorem \ref{thm:1.2} with the help of Theorem \ref{thm:1.1}

The theorem can be divided into three separate cases:
\begin{enumerate}[label=(\roman*)]
    \item \(\beta_U < 2\);
    \item \(\beta_U, \beta_L \geq 2\);
    \item \(\beta_U \geq 2 > \beta_L\).
\end{enumerate}
In case (i), it is necessary to prove that
\begin{theorem}
    Assume that \(\beta_U < 2\) and the initial datum has compact support \(w_{0} \in C_{comp}(\real)\), nontrivial and nonnegative. Let \((w_L, w_{U})\) be the solution of \eqref{eq:1.6}. Then,
    \[
        (w_L(\cdot, t), w_{U}(\cdot, t)) \to (1, 1) \quad \text{locally uniformly as } t \to +\infty.
    \]
    \label{thm:3.1}
\end{theorem}
Here is presented a sketch of the proof of this theorem:
\begin{proof}[Sketch of the proof]
    From Theorem \ref{thm:2.4} is known that 
    \begin{equation}
        0 \leq w_{i} \leq 1 + \sup_{\real} w_{0}, \quad i = U, L.
        \label{eq:3.2}
    \end{equation}
    As \(\norm{w_{0}}_{L^{\infty}(\real)} > 0\), take \(\overline{u}(t) = 1 + \norm{w_{0}}_{L^{\infty}(\real)}e^{t}\). 
    It is possible to prove that \((\overline{u}, \overline{u})\) is a supersolution of \eqref{eq:1.6}.
\end{proof}