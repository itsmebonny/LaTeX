\section{Two branches problem} 
Here a brief sketch of the proof of Theorem \ref{thm:1.2} is presented. It is possible to prove Theorem \ref{thm:1.2} with the help of Theorem \ref{thm:1.1}

The theorem can be divided into three separate cases:
\begin{enumerate}[label=(\roman*)]
    \item \(\beta_U < 2\);
    \item \(\beta_U, \beta_L \geq 2\);
    \item \(\beta_U \geq 2 > \beta_L\).
\end{enumerate}
In case (i), it is necessary to prove that
\begin{theorem}
    Assume that \(\beta_U < 2\) and the initial datum has compact support \(w_{0} \in C_{comp}(\real)\), nontrivial and nonnegative. Let \((w_L, w_{U})\) be the solution of \eqref{eq:1.6}. Then,
    \[
        (w_L(\cdot, t), w_{U}(\cdot, t)) \to (1, 1) \quad \text{locally uniformly as } t \to +\infty.
    \]
    \label{thm:3.1}
\end{theorem}
Here is presented a sketch of the proof of this theorem:
\begin{proof}[Sketch of the proof]
    From Theorem \ref{thm:2.4} is known that 
    \begin{equation}
        0 \leq w_{i} \leq 1 + \sup_{\real} w_{0}, \quad i = U, L.
        \label{eq:3.1}
    \end{equation}
    As \(\norm{w_{0}}_{L^{\infty}(\real)} > 0\), take \(\overline{u}(t) = 1 + \norm{w_{0}}_{L^{\infty}(\real)}e^{t}\). 
    It is possible to prove that \((\overline{u}, \overline{u})\) is a supersolution of \eqref{eq:1.6}. Using Lemma \ref{lem:2.2} and \eqref{eq:3.1}, it is possible to obtain 
    \begin{equation}
        1 = \lim_{t \to +\infty} \overline{u}(t) \geq \limsup_{t \to +\infty} \norm{w_i(\cdot, t)}_{L^{\infty}(\real)}, \quad i = U, L.
        \label{eq:3.2}
    \end{equation}
    From the assumptions, \(w_0\) is nontrivial and nonnegative, so \(w_U(x, 1) > 0\) for \(x \in \real_U\). Since \(0 < \beta_U < 2\), it is possible to prove, using standard results on the logistic equation, that there exists a unique constant \(l_0 > 0\) such that the following problem
    \begin{equation}
        \begin{dcases}
            -ddot{w} + \beta_U \dot{w} = w(1 - w), & x \in (-l, 0), \\
            w(0) = w(-l) = 0
        \end{dcases}
        \label{eq:3.3}
    \end{equation}
    has a positive solution if and only if \(l > l_0\), and the positive solution \(w_l\) is unique and satisfies \(\norm{w_l}_\infty \to 0\) as \(l \to l_0\). Fixing \(l > l_0\) close to \(l_0\), it is possible to ensure that the unique solution \(w_l\) of \eqref{eq:3.3} satisfies \(w_l(x) < w_U(x, 1)\) for \(x \in [-l, 0]\). Now set 
    \begin{equation*}
        w_l^0(x) \coloneqq \begin{cases}
            w_l & \text{if } x \in [-l, 0], \\
            0 & \text{if } x \in (-\infty, -l).
        \end{cases}
    \end{equation*}
    Then, let \((\underline{w}_L. \underline{w}_U)\) be the solution of \eqref{eq:1.6} with initial datum \((0, w_l^0)\). Clearly,
    \begin{align*}
        \underline{w}_U(x, t) > 0 & \text{ for } x \in \real_U, t > 0, \\
        \underline{w}_L(x, t) > 0 & \text{ for } x \in \real_L, t > 0.
    \end{align*}
    By the parabolic comparison principle it is possible to conclude that \(\underline{w}_U \geq w_0^l\) for all \((x,t) \in [-l, 0] \times [0, +\infty)\). Hence,
    \begin{equation*}
        (\underline{w}_L(\cdot, t), \underline{w}_U(\cdot, t)) \geq (0, w_l^0) \quad \text{for all } t > 0.
    \end{equation*}
    Thus, for any \(\delta > 0\), 
    \begin{equation*}
        (\underline{w}_L(\cdot, \delta), \underline{w}_U(\cdot, \delta)) \geq (0, w_l^0).
    \end{equation*}
    Using Lemma \ref{lem:2.2} it follows
    \begin{equation*}
        (\underline{w}_L(\cdot, t + \delta), \underline{w}_U(\cdot, t + \delta)) \geq (\underline{w}_L(\cdot, t), \underline{w}_U(\cdot, t)) \quad \text{for all } t > 0,
    \end{equation*}
    meaning that \((\underline{w}_L, \underline{w}_U)\) is nondecreasing in \(t\). 
    By denoting 
    \[
        (\underline{w}_{L, \infty}(x), \underline{w}_{U, \infty}(x)) \coloneqq (\lim_{t \to +\infty} \underline{w}_L(x, t), \lim_{t \to +\infty} \underline{w}_U(x, t)), 
    \]
    it is possible to use a similar argument as in proof of Theorem \ref{thm:2.4} to conclude that
    \[
        (\underline{w}_{L}(\cdot, t), \underline{w}_{U}(\cdot, t)) \to (\underline{w}_{L, \infty}(\cdot), \underline{w}_{U, \infty}(\cdot)) \quad \text{locally uniformly as } t \to +\infty.
    \]
    and that \((\underline{w}_{L, \infty}(x), \underline{w}_{U, \infty}(x))\) is a positive stationary solution of \eqref{eq:1.6}. By theorem \ref{thm:1.1}, the only possible positive stationary solution is \((1, 1)\), therefore
    \begin{equation}
        (\underline{w}_{L}(\cdot, t), \underline{w}_{U}(\cdot, t)) \to (1, 1) \quad \text{locally uniformly as } t \to +\infty.
        \label{eq:3.4}
    \end{equation}
    On the other hand, as \(w_l^0(x) < w_U(x, 1)\) for \(x \in (-\infty, 0)\), thanks to Lemma \ref{lem:2.2},
    \[
        (\underline{w}_L(\cdot, t), \underline{w}_U(\cdot, t)) \leq (\underline{w}_L(\cdot, t + 1), \underline{w}_U(\cdot, t + 1)) \quad \text{for all } t > 0.
    \]
    This and \eqref{eq:3.4} imply that
    \begin{equation}
        (\liminf_{t \to +\infty} w_L(\cdot, t), \liminf_{t \to +\infty} w_U(\cdot, t)) \to (1, 1) \quad \text{locally uniformly as } t \to +\infty.
        \label{eq:3.5}
    \end{equation}
    Combining \eqref{eq:3.2} and \eqref{eq:3.5} concludes the proof.
\end{proof}