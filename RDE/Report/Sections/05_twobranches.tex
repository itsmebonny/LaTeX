\section{Two branches problem} 
Here a brief sketch of the proof of Theorem \ref{thm:1.2} is presented. It is possible to prove Theorem \ref{thm:1.2} with the help of Theorem \ref{thm:1.1}

The theorem can be divided into three separate cases:
\begin{enumerate}[label=(\roman*)]
    \item \(\beta_U < 2\);
    \item \(\beta_U, \beta_L \geq 2\);
    \item \(\beta_U \geq 2 > \beta_L\).
\end{enumerate}
In case (i), it is necessary to prove that
\begin{theorem}
    Assume that \(\beta_U < 2\) and the initial datum has compact support \(w_{0} \in C_{comp}(\real)\), nontrivial and nonnegative. Let \((w_L, w_{U})\) be the solution of \eqref{eq:1.6}. Then,
    \[
        (w_L(\cdot, t), w_{U}(\cdot, t)) \to (1, 1) \quad \text{locally uniformly as } t \to +\infty.
    \]
    \label{thm:3.1}
\end{theorem}
Here is presented a sketch of the proof of this theorem:
\begin{proof}[Sketch of the proof]
    From Theorem \ref{thm:2.4} is known that 
    \begin{equation}
        0 \leq w_{i} \leq 1 + \sup_{\real} w_{0}, \quad i = U, L.
        \label{eq:3.1}
    \end{equation}
    As \(\norm{w_{0}}_{L^{\infty}(\real)} > 0\), take \(\overline{u}(t) = 1 + \norm{w_{0}}_{L^{\infty}(\real)}e^{t}\). 
    It is possible to prove that \((\overline{u}, \overline{u})\) is a supersolution of \eqref{eq:1.6}. Using Lemma \ref{lem:2.2} and \eqref{eq:3.1}, it is possible to obtain 
    \begin{equation}
        1 = \lim_{t \to +\infty} \overline{u}(t) \geq \limsup_{t \to +\infty} \norm{w_i(\cdot, t)}_{L^{\infty}(\real)}, \quad i = U, L.
        \label{eq:3.2}
    \end{equation}
    From the assumptions, \(w_0\) is nontrivial and nonnegative, so \(w_U(x, 1) > 0\) for \(x \in \real_U\). Since \(0 < \beta_U < 2\), it is possible to prove, using standard results on the logistic equation, that there exists a unique constant \(l_0 > 0\) such that the following problem
    \begin{equation}
        \begin{dcases}
            -ddot{w} + \beta_U \dot{w} = w(1 - w), & x \in (-l, 0), \\
            w(0) = w(-l) = 0
        \end{dcases}
        \label{eq:3.3}
    \end{equation}
    has a positive solution if and only if \(l > l_0\), and the positive solution \(w_l\) is unique and satisfies \(\norm{w_l}_\infty \to 0\) as \(l \to l_0\). Fixing \(l > l_0\) close to \(l_0\), it is possible to ensure that the unique solution \(w_l\) of \eqref{eq:3.3} satisfies \(w_l(x) < w_U(x, 1)\) for \(x \in [-l, 0]\). Now set 
    \begin{equation*}
        w_l^0(x) \coloneqq \begin{cases}
            w_l & \text{if } x \in [-l, 0], \\
            0 & \text{if } x \in (-\infty, -l).
        \end{cases}
    \end{equation*}
    Then, let \((\underline{w}_L. \underline{w}_U)\) be the solution of \eqref{eq:1.6} with initial datum \((0, w_l^0)\). Clearly,
    \begin{align*}
        \underline{w}_U(x, t) > 0 & \text{ for } x \in \real_U, t > 0, \\
        \underline{w}_L(x, t) > 0 & \text{ for } x \in \real_L, t > 0.
    \end{align*}
    By the parabolic comparison principle it is possible to conclude that \(\underline{w}_U \geq w_0^l\) for all \((x,t) \in [-l, 0] \times [0, +\infty)\). Hence,
    \begin{equation*}
        (\underline{w}_L(\cdot, t), \underline{w}_U(\cdot, t)) \geq (0, w_l^0) \quad \text{for all } t > 0.
    \end{equation*}
    Thus, for any \(\delta > 0\), 
    \begin{equation*}
        (\underline{w}_L(\cdot, \delta), \underline{w}_U(\cdot, \delta)) \geq (0, w_l^0).
    \end{equation*}
    Using Lemma \ref{lem:2.2} it follows
    \begin{equation*}
        (\underline{w}_L(\cdot, t + \delta), \underline{w}_U(\cdot, t + \delta)) \geq (\underline{w}_L(\cdot, t), \underline{w}_U(\cdot, t)) \quad \text{for all } t > 0,
    \end{equation*}
    meaning that \((\underline{w}_L, \underline{w}_U)\) is nondecreasing in \(t\). 
    By denoting 
    \[
        (\underline{w}_{L, \infty}(x), \underline{w}_{U, \infty}(x)) \coloneqq (\lim_{t \to +\infty} \underline{w}_L(x, t), \lim_{t \to +\infty} \underline{w}_U(x, t)), 
    \]
    it is possible to use a similar argument as in proof of Theorem \ref{thm:2.4} to conclude that
    \[
        (\underline{w}_{L}(\cdot, t), \underline{w}_{U}(\cdot, t)) \to (\underline{w}_{L, \infty}(\cdot), \underline{w}_{U, \infty}(\cdot)) \quad \text{locally uniformly as } t \to +\infty.
    \]
    and that \((\underline{w}_{L, \infty}(x), \underline{w}_{U, \infty}(x))\) is a positive stationary solution of \eqref{eq:1.6}. By theorem \ref{thm:1.1}, the only possible positive stationary solution is \((1, 1)\), therefore
    \begin{equation}
        (\underline{w}_{L}(\cdot, t), \underline{w}_{U}(\cdot, t)) \to (1, 1) \quad \text{locally uniformly as } t \to +\infty.
        \label{eq:3.4}
    \end{equation}
    On the other hand, as \(w_l^0(x) < w_U(x, 1)\) for \(x \in (-\infty, 0)\), thanks to Lemma \ref{lem:2.2},
    \[
        (\underline{w}_L(\cdot, t), \underline{w}_U(\cdot, t)) \leq (\underline{w}_L(\cdot, t + 1), \underline{w}_U(\cdot, t + 1)) \quad \text{for all } t > 0.
    \]
    This and \eqref{eq:3.4} imply that
    \begin{equation}
        (\liminf_{t \to +\infty} w_L(\cdot, t), \liminf_{t \to +\infty} w_U(\cdot, t)) \to (1, 1) \quad \text{locally uniformly as } t \to +\infty.
        \label{eq:3.5}
    \end{equation}
    Combining \eqref{eq:3.2} and \eqref{eq:3.5} concludes the proof.
\end{proof}
In case (ii), by Theorem \ref{thm:1.1}(iii), for any \(\alpha \in (0, 1)\), \eqref{eq:1.14} has a unique solution \(\phi_L(\cdot; \alpha), \phi_U(\cdot; \alpha)\) and both \(\phi_L(x; \alpha), \phi_U(x; \alpha)\) are increasing in \(x\). These solutions will be used to construct a suitable supersolution to estabilish the desired asymptotic behavior of \((w_L, w_U)\). 

\begin{theorem}
    Assume that \(\beta_U, \beta_L \geq 2\) and the initial datum has compact support \(w_{0} \in C_{comp}(\real)\), nontrivial and nonnegative. Let \((w_L, w_{U})\) be the solution of \eqref{eq:1.6}. Then,
    \[
        (w_L(\cdot, t), w_{U}(\cdot, t)) \to (0, 0) \quad \text{locally uniformly as } t \to +\infty.
    \]
    Moreover, \(\norm{w_L(\cdot, t)}_{L^{\infty}(\real)} \to 1\) and \(\norm{w_U(\cdot, t)}_{L^{\infty}(\real)} \to 0\) as \(t \to +\infty\).
    \label{thm:3.2}
\end{theorem}
As above, a sketch of the proof is presented:
\begin{proof}[Sketch of the proof]
    Fix \(\alpha \in (0, 1)\) and define a supersolution \((\overline{w}_L, \overline{w}_U)\) as
    \begin{align*}
        \overline{w}_L(x, t) & \coloneqq \phi_L(x; \alpha) + M e^{-\lambda t} e^{\frac{\beta_L}{2}x} \quad x \in \real_L, t \geq 0, \\
        \overline{w}_U(x, t) & \coloneqq \phi_U(x; \alpha) + M e^{-\lambda t} e^{\frac{\beta_U}{2}x} \quad x \in \real_U, t \geq 0,
    \end{align*}
    where \(M\) and \(\lambda\) will be determined later. Now, having defined 
    \[
        v(x) \coloneqq M e^{kx} \quad \text{with } k = \frac{1}{2}\beta_U^ + \sqrt{\beta_U^2 - 4}.
    \]
    Clearly,
    \[
        \ddot{v} + \beta_U \dot{v} = v \geq v - v^2.
    \]
    Thanks to Theorem \ref{thm:1.1}(iv), as \(x \to -\infty\), \(\phi_U(x; \alpha)\) satisfies \eqref{eq:1.9}. Therefore, it exists \(l = l_\alpha > 0\) such that
    \[
        \phi_U(x; \alpha) \geq v(x) \quad \text{for } x \leq -l.
    \]
    Fix now \(l = l_\alpha\). Because \(\beta_U, \beta_L \geq 2\), \(\phi_L(0;\alpha) = \alpha > 0\) and \(2\phi_U(-l;\alpha) > 0\) it is always possible to find a sufficiently small \(\lambda(l) > 0\) such that 
    \begin{align*}
        \partial_t \overline{w}_L - \partial_{xx} \overline{w}_L + \beta_L \partial_x \overline{w}_L - \overline{w}_L + \overline{w}_L^2 \geq 0, \quad x \in (0, +\infty), t > 0, \\
        \partial_t \overline{w}_U - \partial_{xx} \overline{w}_U + \beta_U \partial_x \overline{w}_U - \overline{w}_U + \overline{w}_U^2 \geq 0, \quad x \in [-l, 0), t > 0.
    \end{align*}
    Since \(w_0 \in C_{comp}(\real)\), \(M\) can be chosen as \(M > \max{1, \norm{w_0}_{L^{\infty}(\real)}}\) such that \(v(x) > w_0(x)\) in \(\real\) and 
    \[
        \overline{w}_i(x, 0) \geq w_i(x, 0) \quad \text{for } x \in \real, i = L, U.
    \] 
    Since \(v(0) = M > \max{1, \norm{w_0}_{L^{\infty}(\real)}} \geq w_U(0, t)\) for all \(t \geq 0\), by the comparison principle
    \[
        w_U(x, t) \leq v(x) \quad \text{for } x \in \real_U, t \geq 0.
    \]
    Let now \(\xi(t) \coloneqq \frac{2\lambda}{\beta_L} t\). Then 
    \[
        \overline{w}_L(\xi(t), t) > M \geq w_L(\xi(t), t) \quad \text{for all } t > 0.
    \]
    Then, \((\overline{w}_L, \overline{w}_U)\) is a supersolution of \eqref{eq:1.6} in the region \(x \in [-l, \xi(t)]\) for all \(t \geq 0\). It's possible to conclude that
    \[
        w_U(x, t) \leq \overline{w}_U(x, t) \quad \text{for } x \in \real_U, t \geq 0,
    \]
    Therefore,
    \[
        \limsup_{t \to +\infty} w_i(x, t) \leq \lim \overline{w}_i(x, t) = \phi_i(x; \alpha) = 0 \quad \text{for } x \in \real_i, i = L, U.
    \]
    Since \(\alpha\) is arbitrary, it is possible to conclude that
    \[
        \lim_{t \to +\infty} w_i(x, t) = 0 \quad \text{locally uniformly for } x \in \real_i, i = L, U.
    \]
    Since \(\overline{w}_U\) is increasing in \(x\), it is easily concluded that \(\norm{w_U(\cdot, t)}_{L^{\infty}(\real)} \to 0\) as \(t \to +\infty\).

    By creating a problem similar to \eqref{eq:1.6} and constructing a subsolution for that problem it is possible to prove that \(\norm{w_L(\cdot, t)}_{L^{\infty}(\real)} \to 1\) as \(t \to +\infty\).
\end{proof}

In case (iii), by Theorem \ref{thm:1.1}(iii), there exists a constant \(\alpha_0 \in (0, 1)\) such that \eqref{eq:1.8} has a unique solution \(\phi_L(x; \alpha_0), \phi_U(x; \alpha_0)\) when \(\alpha \in [\alpha_0, 1)\) and no solution when \(\alpha \in (0, \alpha_0)\). 
\begin{theorem}
    Assume that \(\beta_U \geq 2 > \beta_L\) and the initial datum has compact support \(w_{0} \in C_{comp}(\real)\), nontrivial and nonnegative. Let \((w_L, w_{U})\) be the solution of \eqref{eq:1.6}. Then,
    \begin{equation}
        (w_L(\cdot, t), w_{U}(\cdot, t)) \to (\phi_L(\cdot; \alpha_0), \phi_U(\cdot; \alpha_0)) \quad \text{locally uniformly as } t \to +\infty.
        \label{eq:3.6}
    \end{equation}
    \label{thm:3.3}
\end{theorem}

\begin{proof}[Sketch of the proof]
    For \(M > 1\), set 
    \begin{align*}
        &k \coloneqq \frac{1\beta_U + \sqrt{\beta_U^2 - 4}}{2}, \\
        &\overline{\phi}_{U,0} \coloneqq M e^{kx}, \quad x \in \real_U, \\
        &\overline{\phi}_{L,0} \coloneqq M, \quad x \in \real_L.
    \end{align*}
    Since \(w_0\) has compact support, it is possible to choose \(M > 1\) large enough that
    \begin{align*}
        \overline{\phi}_{U,0}(x) = M e^{kx} > w_0(x) \quad \text{for } x \in \real_U, \\
        \overline{\phi}_{L,0}(x) = M > w_0(x) \quad \text{for } x \in \real_L. 
   \end{align*}
   Therefore, \((\overline{\phi}_{L,0}, \overline{\phi}_{U,0})\) is a supersolution of the corresponding elliptic problem of \eqref{eq:1.6}. It follows that the unique solution \((\overline{\phi}_U, \overline{\phi}_L)\) of \eqref{eq:1.6} with initial datum \((\overline{\phi}_{L,0}, \overline{\phi}_{U,0})\) is nondecreasing in \(t\).  As \(t \to +\infty\),
   \[
         (\overline{\phi}_L(\cdot, t), \overline{\phi}_U(\cdot, t)) \to (\hat{\phi}_L(\cdot), \hat{\phi}_U(\cdot)) \quad \text{locally uniformly},
   \]
   and \((\hat{\phi}_L, \hat{\phi}_U)\) is a nonnegative stationary solution of \eqref{eq:1.6}. Clearly,
   \begin{align*}
    \hat{\phi}_U(x) \leq \overline{\phi}_{U,0}(x) = M e^{kx} \quad \text{for } x \in \real_U, \\
    \hat{\phi}_L(x) \leq \overline{\phi}_{L,0}(x) = M \quad \text{for } x \in \real_L.
   \end{align*}
   Defining 
   \begin{align*}
         \underline{\phi}_{U,0} \coloneqq 0, \quad x \in \real_U, \\
            \underline{\phi}_{L,0} \coloneqq \phi_l(x), \quad x \in \real_L,
   \end{align*}
   where \(\phi_l(x)\) is the unique solution of
   \begin{equation*}
    \begin{cases}
        -\ddot{\phi} + \beta_L \dot{\phi} = \phi(1 - \phi), & x \in (0, l), \\
        \phi(0) = \phi(l) = 0.
    \end{cases}
   \end{equation*}
    It is possible to prove that \((\underline{\phi}_L, \underline{\phi}_U)\) is a subsolution of the elliptic problem of \eqref{eq:1.6}, and again it follows that the unique solution \((\underline{\phi}_L, \underline{\phi}_U)\) of \eqref{eq:1.6} with initial datum \((\underline{\phi}_{L,0}, \underline{\phi}_{U,0})\) is nondecreasing in \(t\). As \(t \to +\infty\),
    \[
        (\underline{\phi}_L(\cdot, t), \underline{\phi}_U(\cdot, t)) \to (\tilde{\phi}_L(\cdot), \tilde{\phi}_U(\cdot)) \quad \text{locally uniformly},
    \]
    and \((\tilde{\phi}_L, \tilde{\phi}_U)\) is a nonnegative stationary solution of \eqref{eq:1.6}. Since
    \[
        \underline{\phi}_{U,0}(x) < \hat{\phi}_U(x), \quad \underline{\phi}_{L,0}(x) < \hat{\phi}_L(x)
    \]
    it's easy to see
    \begin{align*}
        0 < \tilde{\phi}_U(x) \leq \hat{\phi}_U(x)  \leq M e^{kx} \quad \text{for } x \in \real_U, \\
        \phi_l(x) \leq \tilde{\phi}_L(x) \leq \hat{\phi}_L(x) \leq M \quad \text{for } x \in \real_L.
    \end{align*}
    Then \((\tilde{\phi}_L, \tilde{\phi}_U)\) and \((\hat{\phi}_L, \hat{\phi}_U)\) must be solutions of \eqref{eq:1.8} with some \(\hat{\alpha}, \tilde{\alpha} \in (0,1)\), with \(\hat{\alpha} \geq \tilde{\alpha}\). With Theorem \ref{thm:1.1}(iv) it is possible to prove that \(\hat{\alpha} = \tilde{\alpha} = \alpha_0\), and therefore the two solutions coincide and can be defined as \(\phi_L(\cdot; \alpha_0), \phi_U(\cdot; \alpha_0)\).
    With the same arguments as in the previous cases, it is possible to conclude that
    \begin{align*}
        \liminf_{t \to +\infty} w_U(x, t) \geq \phi_U(x; \alpha_0) \quad \text{locally uniformly for } x \in \real_U, \\
        \limsup_{t \to +\infty} w_L(x, t) \leq \phi_L(x; \alpha_0) \quad \text{locally uniformly for } x \in \real_L.
    \end{align*}
    Therefore, \eqref{eq:3.6} holds.
\end{proof}