\section{The Fisher-KPP equation}
Fisher-KPP equation is a reaction-diffusion equation that models the spread of an advantageous gene in a population in a given environment \cite{SalsaVerziniPDE}. The equation is given by
\begin{equation}
    u_t = Du_{xx} + ru(1-u) \quad t > 0, x \in \real,
    \label{eq:fisher-kpp}
\end{equation}
where $u = u(x, t)$ is the population density, $r, D$ are positive parameters. It was first introduced by Fisher \cite{Fisher1937} and a similar result was obtained by Kolmogorov, Petrovsky and Piskunov \cite{KPP1937}. It's an example of a semilinear reaction-diffusion equation, which can be used to model phenomena in which some density experience both a diffusive effect (e.g. dispersion) and a reactive one (e.g. growth or decay). Throughout this work, \eqref{eq:fisher-kpp} will be referred to as the Fisher-KPP equation.

\subsection{Boundary value problem}
To better understand the dynamics of a population, a boundary value problem (BVP) is considered. Let \(u = u(\bm{x}, t) > 0\) be the (nonnegative) population density at time \(t > 0\) and position \(\bm{x} \in \Omega \subset \real^n\), where \(\Omega\) is a smooth, bounded domain with boundary \(\partial \Omega\). 
The general form of the KPP-Fisher equation can be expressed as 
\begin{equation}
    u_t - D \Delta u = f(\bm(x), u) \quad \text{in} \quad Q_T = \Omega \times (0, T),
    \label{eq:fisher-kpp-general}
\end{equation}
with \(D > 0\) as diffusion coefficient, the motility of the population, and \(f(\bm{x}, u)\) as a nonlinear reaction term, governing the growth of the population. The term \(f\) should model the balance between the growth of the population and its decay. Natural basic assumptions for \(f\) are:
\begin{itemize}
    \item \(f(\bm{x}, 0) = 0\) for all \(\bm{x} \in \Omega\),
    \item for some \(M > 0, f(\bm{x}, u) < 0\) whenever \(u > M\).
\end{itemize}
The interpretation of the first assumption is that, in absence of the population, there is no growth or decay. The second assumption states that the population cannot grow indefinitely, there is a maximum population density, called maximal carrying capacity. Any time the population density exceeds this value, the population starts to decay. An example of a reaction term that satisfies these assumptions is the logistic growth term, introduced by Verhulst \cite{Verhulst1847}, 
\begin{equation}
    f(\bm{x}, u) = ru\left(1 - \frac{u}{M}\right), 
\end{equation}
where \(r > 0\) is a growth rate. The second term \(-rM^{-1}u^2\) describes the competition for resources.

A closer look at the domain \(\Omega\) is needed, once one has found a fitting reaction term for his problem. Clearly, boundary conditions are needed. In the case of population dynamics, homogeneous Dirichlet boundary conditions represent a lethal boundary, where the population density drops to zero. If one wants to model its population inside an isolated environment, homogeneous Neumann boundary conditions are necessary, creating a reflecting boundary. 

It is now possible to define a BVP for the Fisher-KPP equation, with homogeneous Neumann boundary conditions, as follows:

\begin{equation}
    \begin{dcases}
        u_t - D \Delta u = f(\bm{x}, u) & \text{in} \quad Q_T = \Omega \times (0, T), \\
        \partial_{n}u = 0 & \text{on} \quad S_T = \partial \Omega \times (0, T), \\
        u(\bm{x}, 0) = g & \text{in} \quad \Omega,
    \end{dcases}
    \label{eq:fisher-kpp-bvp}
\end{equation}
where \(g = g(\bm{x})\) is the initial population density.

It is interesting to consider the long-time behavior of the solution of \eqref{eq:fisher-kpp-bvp}, meaning that: does the population survives at all times, or it will eventually die out? This question is related to the existence of nontrivial steady stated \(U = U(\bm{x})\), satisfying:
\begin{equation}
    \begin{dcases}
        -D \Delta U = f(\bm{x}, U) & \text{in} \quad \Omega, \\
        \partial_{n}U = 0 & \text{on} \quad \partial \Omega.
    \end{dcases}
    \label{eq:fisher-kpp-steady} 
\end{equation}
Here, a definition of super/subsolutions is needed. For a more detailed explanation, refer to \cite{SalsaVerziniPDE}. Here \(V = H^1_0(\Omega)\) and the symbols \(\left(\cdot, \cdot\right)_0\) and \(\norm{\cdot}_0\) denote the inner product and norm in \(L^2(\Omega)\) or \(L^2(\Omega, \real^n)\), respectively. 
\begin{definition}
    A function \(u \in L^2(0, T; V)\) with \(\dot{u} \in L^2(0, T; L^2(\Omega))\), is a weak supersolution (resp. subsolution) of \eqref{eq:fisher-kpp-bvp} if \(u(\bm{x}, 0) \geq g\) (resp. \(u(\bm{x}, 0) \leq g\)) and
    \begin{equation}
        \begin{split}
            \left(\dot{u}(t), v\right)_0 + D \left(\nabla u(t), \nabla v\right)_0 \geq (f(\bm{x}, u(t)), v)_0 \quad \text{(resp } \leq \text{)} \\ \quad \text{for a.e. } t \in (0, T), \forall v \in V, v \geq 0 \text{ a.e. in } \Omega.
        \end{split}
    \end{equation}
\end{definition}
Another important result related to supersolutions and subsolutions is the following:
\begin{theorem}
    Let \(\overline{u}\) and \(\underline{u}\) be a bounded weak supersolution and subsolution of \eqref{eq:fisher-kpp-bvp}, respectively. Then, \(\exists!\) weak solution \(u = u_g\) such that
    \begin{equation}
        \underline{u} \leq u_g \leq \overline{u} \quad \text{a.e. in } \quad Q_T.
    \end{equation}
    \label{thm:10.18}
\end{theorem}

In the case of long-time behavior, it is possible to find time independent supersolutions and subsolutions. Thanks to Theorem \ref{thm:10.18}, it is possible to find a solution that is global in time, meaning that it exists for all \(t > 0\). 
The existence of steady states can be inferred by solving the stationary problem \eqref{eq:fisher-kpp-steady}. These states are quite useful to understand the dynamics of the equation.

Two important results that help to understand better the relation between these steady states are the following:
\begin{lemma}
    Let \(\phi, \psi \in L^\infty(\Omega) \cap H^1(\Omega)\) be time independent subsolution and supersolution of \eqref{eq:fisher-kpp-bvp}, respectively, and let \(u_\phi, u_\psi\) be the corresponding solutions of the same problem, with initial condition \(u_\phi(0) = \phi\) and \(u_\psi(0) = \psi\). Then
    \begin{enumerate}
        \item \(u_\phi, u_\psi \text{ and } u_g\) exists for all \(t \geq 0\), and 
        \begin{equation}
            \phi(\bm{x}) \leq u_\phi(\bm{x}, t) \leq u_g(\bm{x}, t) \leq u_\psi(\bm{x}, t) \leq \psi(\bm{x}) \quad \text{a.e in } \Omega \times [0, +\infty),
        \end{equation}
        for all \(\phi \leq g \leq \psi\);
        \item \(u_\phi(\bm{x}, t_2) \geq u_\phi(\bm{x}, t_1)\) a.e. in \(\Omega\), if \(t_2 > t_1\);
        \item \(u_\psi(\bm{x}, t_2) \leq u_\psi(\bm{x}, t_1)\) a.e. in \(\Omega\), if \(t_2 < t_1\).
    \end{enumerate}
    \label{lem:10.21}
\end{lemma}
\begin{theorem}
    Let \(\phi, \psi\) be as in \ref{lem:10.21} and define 
    \[
        \Phi(\bm{x}) = \lim_{t \to +\infty} u_\phi(\bm{x}, t) \quad \text{and} \quad \Psi(\bm{x}) = \lim_{t \to +\infty} u_\psi(\bm{x}, t),
    \]
    a.e. in \(\Omega\). Then \(\Phi, \Psi \) solve \eqref{eq:fisher-kpp-steady} and any other steady state \(U\), with \(\phi \leq U \leq \psi\), satisfies \(\Phi \leq U \leq \Psi\).
    
    In particular, if \(\Phi = \Psi\) a.e. in \(\Omega\), then such a steady state is unique and 
    \[
        u_g(t) \to \Phi \quad \text{as } t \to +\infty.
    \]
    a.e. and in \(L^2(\Omega)\) for every initial datum \(\phi \leq g \leq \psi\).
    \label{thm:10.22}
\end{theorem}

A brief discussion about persistence and extinction of the population will be introduced here. 
The assumptions on the reaction term were:
\begin{itemize}
    \item \(f(\bm{x}, 0) = 0\) for all \(\bm{x} \in \Omega\),
    \item for some \(M > 0, f(\bm{x}, u) < 0\) whenever \(u > M\).
\end{itemize}
Indeed, \(\underline{u}(\bm{x}, t) \equiv 0\) solves the problem, and if \(g \geq 0\), then \(\underline{u}(\bm{x}, t)\) is a subsolution of \eqref{eq:fisher-kpp-bvp}. Also, taking \(\overline{u}(\bm{x}, t) \equiv M'\) where \(M' = \max\{M, \norm{g}_\infty\}\), one obtains
\[
    \left(\partial_t - D\Delta\right)\overline{u} = 0 \geq f(\bm{x}, \overline{u}) 
\]
and \(\overline{u}\) is a supersolution of \eqref{eq:fisher-kpp-bvp}. 

Thanks to Theorem \ref{thm:10.22} it is possible to note that, if \(\Phi(\bm{x}) \equiv 0\) is the only nonnegative steady state, then \(\Phi = \Psi\) and the population will eventually die out as \(t \to +\infty\), for every initial datum \(g\). In the case, \(\Phi > 0\) in \(\Omega\), there is at least one initial datum \(g\) such that the population will survive for all times. 

Clearly, more precise assumptions on the reaction term are needed. For example, one can assume that local relative growth rate is strictly decreasing:
\begin{equation}
    \frac{f(\bm{x}, u_2)}{u_2} < \frac{f(\bm{x}, u_1)}{u_1} \quad \text{whenever } u_2 > u_1 > 0.
    \label{eq:10.69}
\end{equation}
Since \(f\) is at least of class \(C^1\), taking the limits as \(u_1 \to 0\) one obtains
\[
    \frac{f(\bm{x}, u_2)}{u_2} < \lim_{u_1 \to 0} \frac{f(\bm{x}, u_1)}{u_1} = \lim_{u_1 \to 0} \frac{f(\bm{x}, u_1) - f(\bm{x}, 0)}{u_1} = \partial_u f(\bm{x}, 0).
\]
It is possible to define 
\begin{equation}
    m(\bm{x}) = \partial_u f(\bm{x}, 0)
\end{equation}
For a complete discussion on the eigenvalue problem, refer to \cite{SalsaVerziniPDE}. Here will be only presented the most interesting result.

It turns out that the persistence and existence of the population are completely characterized by the sign of the principal eigenvalue \(\lambda_1[m, D]\).