\section{The Fisher-KPP equation}
Fisher-KPP equation is a reaction-diffusion equation that models the spread of a population in a given environment \cite{SalsaVerziniPDE}. The equation is given by
\begin{equation}
    u_t = Du_{xx} + ru(1-u) \quad t > 0, x \in \real
    \label{eq:fisher-kpp}
\end{equation}
where $u = u(x, t)$ is the population density, $r, D$ are positive parameters. It was first introduced by Fisher \cite{Fisher1937} and a similar result was obtained by Kolmogorov, Petrovsky and Piskunov \cite{KPP1937}. It's an example of a semilinear reaction-diffusion equation, which can be used to model phenomena in which some density experience both a diffusive effect (e.g. dispersion) and a reactive one (e.g. growth or decay). Throughout this work, \eqref{eq:fisher-kpp} will be referred to as the Fisher-KPP equation.

\subsection{Boundary value problem}
To better understand the dynamics of a population, a boundary value problem is considered. Let \(u = u(\bm{x}, t) > 0\) be the (nonnegative) population density at time \(t > 0\) and position \(\bm{x} \in \Omega \subset \real^n\), where \(\Omega\) is a smooth, bounded domain with boundary \(\partial \Omega\). 