\section{General overview of the paper} 
After a brief recap on the main results on the KPP-Fisher equation, it is possible to introduce the main topic of this report, the graph-based approach to the domain \(\Omega\). 
\subsection{Introduction}
The authors consider the evolution of a population living in a river, and observe what happens if the river taken into account has two branches, either because of a bifurcation or a confluence. In such a case, the river can be considered as a graph, where the nodes are the junctions and the edges are the branches of the river. One of the novel ideas proposed is to work with infinite graphs, instead of finite ones. If one wants to model the dynamics of an invasive species, usually unbounded spatial regions are better suited for the problem.

One can model the population with the following Cauchy problem:
\begin{equation}
    \begin{dcases}
        u_t - Du_{xx} + \beta u_x = f(u) & \text{for } x \in \real, t > 0, \\
        u(x, 0) = u_0(x) & \text{for } x \in \real,
    \end{dcases}
    \label{eq:1.1}
\end{equation}
where \(u = u(x, t)\) is the population density, \(D > 0\) is the diffusion coefficient, \(\beta \in \real\) is the advection coefficient denoting the flow of the river, \(f(u)\) is the reaction term, and \(u_0(x)\) is the initial population density.

The solution of \eqref{eq:1.1} \(u(x, t)\) is a travelling wave with a certain speed. The idea is to answer some questions about what happens if the river is split in branches, in a way that each branch has infinite length. 

\tikzset{every picture/.style={line width=0.75pt}} %set default line width to 0.75pt 
\begin{figure}
    \centering
    \begin{tikzpicture}[x=0.75pt,y=0.75pt,yscale=-1,xscale=1]
%uncomment if require: \path (0,300); %set diagram left start at 0, and has height of 300

%Straight Lines [id:da06835250339153032] 
\draw    (27.17,139.58) -- (116.96,139.51) ;
\draw [shift={(77.06,139.54)}, rotate = 179.95] [fill={rgb, 255:red, 0; green, 0; blue, 0 }  ][line width=0.08]  [draw opacity=0] (8.93,-4.29) -- (0,0) -- (8.93,4.29) -- cycle    ;
%Shape: Circle [id:dp006568808299004081] 
\draw  [fill={rgb, 255:red, 0; green, 0; blue, 0 }  ,fill opacity=1 ] (114.63,139.51) .. controls (114.63,138.22) and (115.67,137.18) .. (116.96,137.18) .. controls (118.25,137.18) and (119.3,138.22) .. (119.3,139.51) .. controls (119.3,140.8) and (118.25,141.84) .. (116.96,141.84) .. controls (115.67,141.84) and (114.63,140.8) .. (114.63,139.51) -- cycle ;
%Straight Lines [id:da5818375484660003] 
\draw    (456.92,139.33) -- (551.24,139.67) ;
\draw [shift={(509.08,139.52)}, rotate = 180.2] [fill={rgb, 255:red, 0; green, 0; blue, 0 }  ][line width=0.08]  [draw opacity=0] (8.93,-4.29) -- (0,0) -- (8.93,4.29) -- cycle    ;
%Straight Lines [id:da41798667092826824] 
\draw    (120,139.58) -- (210.03,139.58) ;
\draw [shift={(170.01,139.58)}, rotate = 180] [fill={rgb, 255:red, 0; green, 0; blue, 0 }  ][line width=0.08]  [draw opacity=0] (8.93,-4.29) -- (0,0) -- (8.93,4.29) -- cycle    ;
%Shape: Circle [id:dp9495741628591831] 
\draw  [fill={rgb, 255:red, 0; green, 0; blue, 0 }  ,fill opacity=1 ] (548.91,139.67) .. controls (548.91,138.38) and (549.95,137.33) .. (551.24,137.33) .. controls (552.53,137.33) and (553.57,138.38) .. (553.57,139.67) .. controls (553.57,140.96) and (552.53,142) .. (551.24,142) .. controls (549.95,142) and (548.91,140.96) .. (548.91,139.67) -- cycle ;
%Shape: Boxed Line [id:dp3634037576105641] 
\draw    (551.24,139.67) -- (632.76,187.12) ;
\draw [shift={(596.32,165.91)}, rotate = 210.2] [fill={rgb, 255:red, 0; green, 0; blue, 0 }  ][line width=0.08]  [draw opacity=0] (8.93,-4.29) -- (0,0) -- (8.93,4.29) -- cycle    ;
%Shape: Boxed Line [id:dp5855966408278438] 
\draw    (551.24,139.67) -- (633.09,92.79) ;
\draw [shift={(596.5,113.75)}, rotate = 150.2] [fill={rgb, 255:red, 0; green, 0; blue, 0 }  ][line width=0.08]  [draw opacity=0] (8.93,-4.29) -- (0,0) -- (8.93,4.29) -- cycle    ;
%Straight Lines [id:da7091000862619046] 
\draw    (319.92,139.33) -- (414.24,139.67) ;
\draw [shift={(372.08,139.52)}, rotate = 180.2] [fill={rgb, 255:red, 0; green, 0; blue, 0 }  ][line width=0.08]  [draw opacity=0] (8.93,-4.29) -- (0,0) -- (8.93,4.29) -- cycle    ;
%Shape: Boxed Line [id:dp8564221650100632] 
\draw    (240.73,91.88) -- (322.25,139.33) ;
\draw [shift={(285.81,118.12)}, rotate = 210.2] [fill={rgb, 255:red, 0; green, 0; blue, 0 }  ][line width=0.08]  [draw opacity=0] (8.93,-4.29) -- (0,0) -- (8.93,4.29) -- cycle    ;
%Shape: Circle [id:dp5640763569876212] 
\draw  [fill={rgb, 255:red, 0; green, 0; blue, 0 }  ,fill opacity=1 ] (319.92,139.33) .. controls (319.92,138.04) and (320.96,137) .. (322.25,137) .. controls (323.54,137) and (324.58,138.04) .. (324.58,139.33) .. controls (324.58,140.62) and (323.54,141.67) .. (322.25,141.67) .. controls (320.96,141.67) and (319.92,140.62) .. (319.92,139.33) -- cycle ;
%Shape: Boxed Line [id:dp8885765217793992] 
\draw    (240.4,186.21) -- (322.25,139.33) ;
\draw [shift={(285.66,160.28)}, rotate = 150.2] [fill={rgb, 255:red, 0; green, 0; blue, 0 }  ][line width=0.08]  [draw opacity=0] (8.93,-4.29) -- (0,0) -- (8.93,4.29) -- cycle    ;


% Text Node
\draw (35,115.4) node [anchor=north west][inner sep=0.75pt]    {$R_{U}{}{}$};
% Text Node
\draw (465,115.4) node [anchor=north west][inner sep=0.75pt]    {$R_{U}{}{}$};
% Text Node
\draw (169,115.4) node [anchor=north west][inner sep=0.75pt]    {$R_{L}{}{}$};
% Text Node
\draw (378,115.4) node [anchor=north west][inner sep=0.75pt]    {$R_{L}{}{}$};
% Text Node
\draw (592,76.4) node [anchor=north west][inner sep=0.75pt]    {$R_{L_{1}}{}{}$};
% Text Node
\draw (592,173.4) node [anchor=north west][inner sep=0.75pt]    {$R_{L_{2}}{}{}$};
% Text Node
\draw (252,76.4) node [anchor=north west][inner sep=0.75pt]    {$R_{U_{1}}{}{}$};
% Text Node
\draw (254,177.4) node [anchor=north west][inner sep=0.75pt]    {$R_{U_{2}}{}{}$};
% Text Node
\draw (28,200) node [anchor=north west][inner sep=0.75pt]   [align=left] {a)};
% Text Node
\draw (240,200) node [anchor=north west][inner sep=0.75pt]   [align=left] {b)};
% Text Node
\draw (458,200) node [anchor=north west][inner sep=0.75pt]   [align=left] {c)};


\end{tikzpicture}
    \caption{Graph representation of the river. a) The river is split in an upper branch \(R_U\) and a lower branch \(R_L\). b) The upper branch is split in two branches \(R_{U_1}\) and \(R_{U_2}\). c) The lower branch is split in two branches \(R_{L_1}\) and \(R_{L_2}\).}
    \label{fig:river-graph}
\end{figure}

The main results presented in the paper take into account the existence of steady states and the relationship between those states and the domain topology and will be discussed in the next sections.
In Figure \ref{fig:river-graph} it is possible to see the possible configurations of the river that will be studied here. 

If the reaction term present in \eqref{eq:1.1} is assumed to be the logistic law presented in the previous section, \(f(u) = u(1 - u)\) and one defines \(v(x, t) \coloneqq u(x + \beta t, t)\), then the equation becomes
\begin{equation}
    \begin{dcases}
        v_t - Dv_{xx} = v(1 - v) & \text{for } x \in \real, t > 0, \\
        v(x, 0) = u_0(x) & \text{for } x \in \real.
    \end{dcases}
    \label{eq:1.2}
\end{equation}

Clearly, \eqref{eq:1.2} is the KPP-Fisher equation already introduced. One well known result is that 
\begin{equation*}
    \lim_{t \to +\infty} v(x, t) = 1 \quad \text{locally uniform in } x \in \real.
\end{equation*}
This means that the population will eventually occupy all the available space.
It is also possible to define \(c_* \coloneqq 2\sqrt{D} \) as the speed at which the population invades the space. It is possible to see that 
\[
    \lim_{t \to +\infty} u(x, t) = 0 \quad \text{locally uniform in } x \in \real \text{ if } \beta \geq c_*,
\]
and
\[
    \lim_{t \to +\infty} u(x, t) = 1 \quad \text{locally uniform in } x \in \real \text{ if } \beta < c_*.
\]
\eqref{eq:1.1} has two positive steady states, \(\phi_\beta\) and \(1\) if \(\beta \geq c_*\), and only one positive steady state, \(1\), if \(\beta < c_*\). 
This means that the population will eventually die out if the river is too fast, and will occupy all the available space if the river is slow enough.

One of the differences that arises when considering the domain as a graph is that the long-time behavior of the solution is not the same. If \eqref{eq:1.1} has a finite number of steady states when studied over \(\real\), if the domain has a graph structure, whenever a steady state exists, infinitely many of them exist. However, only one of those states can be a global attractor. 

Global attractors are always nonnegative steady states, and they can be classified in three categories:
\begin{enumerate}[label=(\roman*)]
    \item \textbf{washing out:} the population will be washed out by the river in every branch. It happens if the speed of the river \(\beta\) is greater than the speed of the population proliferation \(c_*\) in every branch;
    \item \textbf{persistence at carrying capacity:} the population will survive in all the branches. It happens if the speed of the river \(\beta\) is smaller than the speed of the population proliferation \(c_*\) in every branch;
    \item \textbf{persistence below carrying capacity:} the population will survive in some branches, but not in all of them. It happens if the speed of the river \(\beta\) is smaller than the speed of the population proliferation \(c_*\) in some branches, but not in all of them.
\end{enumerate}

From a biological point of view, assuming that the flow of the river is fixed in time and that the volume of the river is constant, then the speed of the water is completely determined by the topological structure of the river. If every branch in Figure \ref{fig:river-graph} has a small enough cross-sectional area, then the speed of the river will be greater than \(c_*\) in every branch, and the population will be washed out. On the other hand, if the cross-sectional area of the branches is large enough, then the population will survive in all the branches.

A little more counterintuitive is the case in which the cross-sectional area of the upper branches is big enough to let the river flow slower than \(c_*\), the population will survive in all the branches, regardless of the speed of the river in the lower branches. This is due to the fact that the population can invade the lower branches from the upper ones, but not the other way around.

One can ask how those global attractors should be chosen from the infinitely many steady states existing. The mechanism that determines the global attractor is the compact support of the initial datum. The positive steady state that is selected as the global attractor is the one that decays to \(0\) the fastest at the proper end of the graph at infinity. The purpose of this report is to understand the dynamics of the population starting from a bounded area, so only initial data with compact support will be considered.

\subsection{One upper branch and one lower branch}

Here will be observed the behavior of the population when the river is split in two branches, an upper branch and a lower branch. This is the same configuration that can be seen in Figure \ref{fig:river-graph}a. While this configuration may seem a little artificial, since it is not common to see a river split in two branches, unless there is a man-made intervention, like a dam or a gate, which clearly need a different treatment of the junction, it is a good starting point to understand the dynamics of the population in a river network.

Let \(\real_L \coloneqq (0, +\infty)\) and \(\real_U \coloneqq (-\infty, 0)\) be the lower and upper branches. The densities of the population in the lower and upper branches are denoted by \(w_L = w_L(x, t)\) and \(w_U = w_U(x, t)\), respectively. The equation that describes the evolution of the population in the system is
\begin{equation}
    \begin{dcases}
        \partial_t w_L - D \partial_{xx}w_L + \beta_L \partial_x w_L = f_L(w_L) & \text{for } x \in \real_L, t > 0, \\
        \partial_t w_U - D \partial_{xx}w_U + \beta_U \partial_x w_U = f_U(w_U) & \text{for } x \in \real_U, t > 0, \\
        w_L(0, t) = w_U(0, t) & \text{for } t > 0, \\
        D_L a_L \partial_x w_L(0, t) = D_U a_U \partial_x w_U(0, t) & \text{for } t > 0, \\
        w_L(x, 0) = w_{0}(x) & \text{for } x \in \real_L, \\
        w_U(x, 0) = w_{0}(x) & \text{for } x \in \real_U,
    \end{dcases}
    \label{eq:1.4}
\end{equation}
where the parameters \(D_L, D_U > 0\) are the diffusion coefficients, \(\beta_L, \beta_U > 0\) are the advection coefficients, \(a_L, a_U > 0\) are the cross-sectional areas of the branches. The nonlinear reaction terms are \(f_L(w_L), f_U(w_U)\) and are assumed to be locally Lipschitz on \([0, +\infty)\). As already the initial datum \(w_0(x)\) is assumed to have compact support, so it belongs to \(C_{comp}(\real)\), the space of continuous functions on \(\real\) with compact support.

Another assumption is that the volume of the river is constant, so the flow at the junction is the same in both branches, \(a_L \beta_L = a_U \beta_U\). The third line of \eqref{eq:1.4} is the natural continuity connection condition, while the fourth line is the Kirchhoff law, which follows from the natural continuity connection condition and the constant volume assumption.
\[
    a_L \left(D_L \partial_x w_L(0, t) - \beta_L w_L(0, t)\right) = a_U \left(D_U \partial_x w_U(0, t) - \beta_U w_U(0, t)\right).
\]
Without loss of generality, one can assume that \(D_L = D_U = 1\). Also, it is possible to define the reaction terms as \(f_L(w_L) = w_L(1 - w_L)\) and \(f_U(w_U) = w_U(1 - w_U)\).
With these assumptions, the system \eqref{eq:1.4} becomes
\begin{equation}
    \begin{dcases}
        \partial_t w_L - \partial_{xx}w_L + \beta_L \partial_x w_L = w_L(1 - w_L) & \text{for } x \in \real_L, t > 0, \\
        \partial_t w_U - \partial_{xx}w_U + \beta_U \partial_x w_U = w_U(1 - w_U) & \text{for } x \in \real_U, t > 0, \\
        w_L(0, t) = w_U(0, t) & \text{for } t > 0, \\
        a_L \partial_x w_L(0, t) = a_U \partial_x w_U(0, t) & \text{for } t > 0, \\
        w_L(x, 0) = w_{0}(x) & \text{for } x \in \real_L, \\
        w_U(x, 0) = w_{0}(x) & \text{for } x \in \real_U.
    \end{dcases}
    \label{eq:1.6}
\end{equation}
Given any nonnegative initial datum \(w_0(x) \in C_{comp}(\real)\), the system \eqref{eq:1.6} has a unique nonnegative classical solution.

Let now \((\phi_L, \phi_U)\) be the stationary solutions of the system \eqref{eq:1.6}. They are determined by the following system
\begin{equation}
    \begin{dcases}
        - \ddot{\phi}_L + \beta_L \dot{\phi}_L = \phi_L(1 - \phi_L), & 0 < \phi_L < 1, x \in \real_L, \\
        - \ddot{\phi}_U + \beta_U \dot{\phi}_U = \phi_U(1 - \phi_U), & 0 < \phi_U < 1, x \in \real_U, \\
        \phi_L(0) = \phi_U(0) = \alpha \in [0, 1], \\
        a_L \dot{\phi}_L(0) = a_U \dot{\phi}_U(0).
    \end{dcases}
    \label{eq:1.8}
\end{equation}
By the maximum principle, \(\alpha = 0\) implies \((\phi_L, \phi_U) = (0, 0)\), while \(\alpha = 1\) implies \((\phi_L, \phi_U) = (1, 1)\). 
The interesting part is when \(\alpha \in (0, 1)\), which implies 
\[
    0 < \phi_L(x) < 1 \quad \text{for } x \geq 0, \quad \text{and} \quad 0 < \phi_U(x) < 1 \quad \text{for } x \leq 0.
\]
It is possible to rewrite the system \eqref{eq:1.8} in a way that \(\alpha \in (0, 1)\), and from now on it will be considered in this way. 

A complete description of all the solutions of the system \eqref{eq:1.6} is given by the following theorem.

\begin{theorem}
    Since the assumption is that \(D_L = D_U = 1\), the maximum speed of the population invasion is \(c_* = 2\). 
    \begin{enumerate}[label=(\roman*)]
        \item If \(0 < \beta_U < 2\), then \eqref{eq:1.8} has no solution for \(\alpha \in (0, 1)\). 
        \item If \(\beta_L, \beta_U \geq 2\), then for every \(\alpha \in (0, 1)\) \eqref{eq:1.8} has a unique solution.
        \item If \(\beta_U \geq 2 > \beta_L > 0\), then there exists \(\alpha_0 \in (0, 1)\) such that \eqref{eq:1.8} has a unique solution for each \(\alpha \in [\alpha_0, 1)\), and no solution for \(\alpha \in (0, \alpha_0)\).
        \item  Whenever \eqref{eq:1.8} has a solution \((\phi_L, \phi_U)\), it is true that 
        \[
            \dot{\phi}_L > 0, \quad \dot{\phi}_U > 0, \quad \phi_L(\infty) = 1, \quad \phi_U(-\infty) = 0.
        \]
        Moreover, in case (ii) and (iii) with \(\alpha \in (\alpha_0, 1)\), as \(x \to -\infty\), there exists some \(c = c(\alpha) > 0\) such that
        \begin{equation}
            \phi_U(x) = \begin{cases}
                (c + o(1))e^{\frac{1}{2}(\beta_U - \sqrt{\beta_U^2 - 4})x} & \text{if } \beta_U > 2, \\
                (c + o(1))\lvert x\rvert e^{x} & \text{if } \beta_U = 2, \\
            \end{cases} 
            \label{eq:1.9}
        \end{equation}
        while in case (iii) with \(\alpha = \alpha_0\), as \(x \to -\infty\), there exists some \(c > 0\) such that
        \begin{equation}
            \phi_U(x) = (c + o(1))e^{\frac{1}{2}(\beta_U - \sqrt{\beta_U^2 - 4})x}.
            \label{eq:1.10}
        \end{equation}
    \end{enumerate}
    \label{thm:1.1}
\end{theorem}
The long-time behavior of the solutions of the system \eqref{eq:1.6} is described by the following theorem.
\begin{theorem}
    Assuming that \(w_0 \in C_{comp}(\real)\) is nonnegative and nontrivial, let \((w_L, w_U)\) be the solution of \eqref{eq:1.6}. Then the following holds:
    \begin{enumerate}[label=(\roman*)]
        \item If \(0 < \beta_U < 2\), then \((w_L(\cdot, t), w_U(\cdot, t)) \to (1, 1)\) locally uniformly as \(t \to +\infty\).
        \item If \(\beta_U, \beta_L \geq 2\), then \((w_L(\cdot, t), w_U(\cdot, t)) \to (0, 0)\) locally uniformly as \(t \to +\infty\). Also, \(\norm{w_L(\cdot, t)}_{L^\infty(\real_L)} \to 1\) and \(\norm{w_U(\cdot, t)}_{L^\infty(\real_U)} \to 0\) as \(t \to +\infty\).
        \item If \(\beta_U \geq 2 > \beta_L > 0\), then \((w_L(\cdot, t), w_U(\cdot, t)) \to (\phi_L(\cdot; \alpha_0), \phi_U(\cdot; \alpha_0))\)\footnote{\((\phi_L(\cdot; \alpha_0), \phi_U(\cdot; \alpha_0))\) is the unique solution of \eqref{eq:1.8} with \(\alpha = \alpha_0\).} locally uniformly as \(t \to +\infty\).
    \end{enumerate}
    \label{thm:1.2}
\end{theorem}

A brief explanation of Theorem \ref{thm:1.2} is presented here. Case (ii) is the more natural and easy to understand: since in both branches water flows faster than the critical speed \(c_*\), the population will be eventually washed put, so locally it converges to \(0\). But looking at \(\norm{w_L(\cdot, t)}_{L^\infty(\real_L)}\) is it possible to see that the population in the lower branch will be washed downstream, but not wiped out from the network. The population in the upper branch will be moved to the lower branch, so it converges to \(0\) even in the \(L^\infty\) norm.
For case (i) and (ii) it is clear that the flow in the upper branch has more impact on the problem than the flow on the lower one. In case (iii) the steady state solution is increasing in both branches, and its limit at \(-\infty\) is \(0\), while at \(+\infty\) is \(1\). This is the selected state because it decays to \(0\) with the fastest rate compared to the other steady states. 

\subsection{Two upper branches and one lower branch}
In this case the topology of the river is the one presented in Figure \ref{fig:river-graph}b. The domains used for each branch are \(\real_{U_1} \coloneqq (-\infty, 0)\), \(\real_{U_2} \coloneqq (-\infty, 0)\), and \(\real_L \coloneqq (0, +\infty)\). The system that describes the evolution of the population in this case is
\begin{equation}
    \begin{dcases}
        \partial_t w_{U_1} - \partial_{xx}w_{U_1} + \beta_{U_1} \partial_x w_{U_1} = w_{U_1}(1 - w_{U_1}) & \text{for } x \in \real_{U_1}, t > 0, \\
        \partial_t w_{U_2} - \partial_{xx}w_{U_2} + \beta_{U_2} \partial_x w_{U_2} = w_{U_2}(1 - w_{U_2}) & \text{for } x \in \real_{U_2}, t > 0, \\
        \partial_t w_L - \partial_{xx}w_L + \beta_L \partial_x w_L = w_L(1 - w_L) & \text{for } x \in \real_L, t > 0, \\
        w_{U_1}(0, t) = w_{U_2}(0, t) = w_L(0, t) & \text{for } t > 0, \\
        a_{U_1} \partial_x w_{U_1}(0, t) + a_{U_2} \partial_x w_{U_2}(0, t) = a_L \partial_x w_L(0, t) & \text{for } t > 0, \\
    \end{dcases}
    \label{eq:1.11}
\end{equation}
where the parameters \(\beta_{U_1}, \beta_{U_2}, \beta_L, a_L, a_{U_1}, a_{U_2}\) are positive constants with the same meaning as in the previous section. The conservation law at the junction point now is given by
\begin{equation}
    a_L \beta_L = a_{U_1} \beta_{U_1} + a_{U_2} \beta_{U_2}.
    \label{eq:1.12}
\end{equation}
A few assumptions on the initial conditions
\begin{equation}
    \begin{dcases}
        w_i(x, 0) = w_0(x) \text{ is nonnegative and continuous in } \overline{\real}_i, & \text{ for } i = U_1, U_2, L, \\
        w_{L,0} = w_{U_1,0} = w_{U_2,0}, \\
        w_{i,0}(x) \equiv 0 \text{ for all large negative } x, & \text{ for } i = U_1, U_2,
        w_{L,0}(x) \equiv 0 \text{ for all large positive } x.
    \end{dcases}
    \label{eq:1.13}
\end{equation}

Again, it is possible to find the stationary solutions of the system \eqref{eq:1.11} by looking for triplets \((\phi_{U_1}, \phi_{U_2}, \phi_L)\) that satisfies \(0 < \phi_{U_1} < 1\), \(0 < \phi_{U_2} < 1\), and \(0 < \phi_L < 1\). Moreover, either \((\phi_{U_1}, \phi_{U_2}, \phi_L) \equiv (0, 0, 0)\), or \((\phi_{U_1}, \phi_{U_2}, \phi_L) \equiv (1, 1, 1)\), or it solves the following system
\begin{equation}
    \begin{dcases}
        - \ddot{\phi}_{U_1} + \beta_{U_1} \dot{\phi}_{U_1} = \phi_{U_1}(1 - \phi_{U_1}), & 0 < \phi_{U_1} < 1, x \in \real_{U_1}, \\
        - \ddot{\phi}_{U_2} + \beta_{U_2} \dot{\phi}_{U_2} = \phi_{U_2}(1 - \phi_{U_2}), & 0 < \phi_{U_2} < 1, x \in \real_{U_2}, \\
        - \ddot{\phi}_L + \beta_L \dot{\phi}_L = \phi_L(1 - \phi_L), & 0 < \phi_L < 1, x \in \real_L, \\
        \phi_{U_1}(0) = \phi_{U_2}(0) = \phi_L(0) = \alpha \in (0, 1), \\
        a_{U_1} \dot{\phi}_{U_1}(0) + a_{U_2} \dot{\phi}_{U_2}(0) = a_L \dot{\phi}_L(0).
    \end{dcases}
    \label{eq:1.14}
\end{equation}
Same as before, a complete description of all the solutions of the system \eqref{eq:1.11} is given by the following theorem.
\begin{theorem}
    \begin{enumerate}[label=(\roman*)]
        \item If \(\beta_{U_1}, \beta_{U_2} < 2\), then \eqref{eq:1.14} has no solution for \(\alpha \in (0, 1)\).
        \item If \(\beta_{U_1}, \beta_{U_2}, \beta_L \geq 2\), then the following holds:
        \begin{enumerate}[label=(\alph*)]
            \item For every \(\alpha \in (0, 1)\), \eqref{eq:1.14} has a continuum of solutions satisfying
            \begin{equation}
                \dot{\phi}_{U_1} > 0, \quad \dot{\phi}_{U_2} > 0, \quad \dot{\phi}_L > 0, \quad \phi_{U_1}(-\infty) = \phi_{U_2}(-\infty) = 0, \quad \phi_L(+\infty) = 1.
                \label{eq:1.15}
            \end{equation}
            \item For \(i = 1,2\) and \(j = 3 - 1\), there exists \(\hat{\alpha}_i \in (0, 1)\) such that for each \(\alpha \in [\hat{\alpha}_i, 1)\), \eqref{eq:1.14} has a unique solution satisfying
            \begin{equation}
                \dot{\phi}_{U_i} < 0, \quad \dot{\phi}_{U_j} > 0, \quad \dot{\phi}_L > 0, \quad \phi_{U_j}(-\infty) = 0, \quad \phi_L(+\infty) = 1,
                \label{eq:1.16}
            \end{equation}
            and no solution for \(\alpha \in (0, \hat{\alpha}_i)\).
            \item Any solution of \eqref{eq:1.14} with \(\alpha \in (0,1)\) satisfies \eqref{eq:1.15} or \eqref{eq:1.16}. Moreover, for any \(\alpha \in (0,1)\), there exists \(c_i = c_i(\alpha) > 0\) for \(i = 1,2\) and a solution of \eqref{eq:1.14} such that
            \begin{equation}
                \phi_{U_i}(x) = \begin{cases}
                    (c_i + o(1))e^{\frac{1}{2}(\beta_{U_i} - \sqrt{\beta_{U_i}^2 - 4})x} & \text{if } \beta_{U_i} > 2, \\
                    (c_i + o(1))\lvert x\rvert e^{x} & \text{if } \beta_{U_i} = 2.
                \end{cases}
                \label{eq:1.17}
            \end{equation}
        \end{enumerate}
        \item If \(\beta_{U_1}, \beta_{U_2} \geq 2 > \beta_L > 0\), then the following holds:
        \begin{enumerate}[label=(\alph*)]
            \item There exists \(\alpha^* \in (0,1)\) such that \eqref{eq:1.14} has a continuum of solutions satisfying \eqref{eq:1.15} for each \(\alpha \in (\alpha^*, 1)\), has a unique solution for \(\alpha = \alpha^*\) and has no solution for \(\alpha \in (0, \alpha^*)\).
            \item For \(i = 1,2\) and \(j = 3 - i\), there exists \(\hat{\alpha}_i^* \in (0, 1)\) with \(\hat{\alpha}_i^* > \alpha^*\), such that for each \(\alpha \in [\hat{\alpha}_i^*, 1)\), \eqref{eq:1.14} has a unique solution satisfying \eqref{eq:1.16} and no solution for \(\alpha \in (0, \hat{\alpha}_i^*)\).
            \item Any solution of \eqref{eq:1.14} with \(\alpha \in (0,1)\) satisfies either \eqref{eq:1.15} or \eqref{eq:1.16}. 
            \item If \(\alpha \in (\alpha^*, 1)\), then for any solution of \eqref{eq:1.14} satisfying \eqref{eq:1.15}, there exists \(c_i^* = c_i^*(\alpha) > 0\) for \(i = 1,2\) such that, as \(x \to -\infty\), \eqref{eq:1.17} holds.
            \item If \(\alpha = \alpha^*\), then the unique solution of \eqref{eq:1.14} has the following asymptotic behavior as \(x \to -\infty\):
            \begin{equation}
                \phi_{U_i}(x) = (c_i^* + o(1))e^{\frac{1}{2}(\beta_{U_i} - \sqrt{\beta_{U_i}^2 - 4})x}, \quad \text{for some } c_i = c_i(\alpha) > 0, i = 1,2.
                \label{eq:1.18}
            \end{equation}
        \end{enumerate} 
        \item If \(\max\{\beta_{U_1}, \beta_{U_2}\} \geq 2 >  \min\{\beta_{U_1}, \beta_{U_2}\}\), then there exists \(\alpha^{**} \in (0,1)\) such that \eqref{eq:1.14} has a unique solution for each \(\alpha \in (\alpha^{**}, 1)\) and no solution for \(\alpha \in (0, \alpha^{**})\). Moreover, when \(\alpha \in (\alpha^{**}, 1)\), and \(\beta_{U_j} =\max\{\beta_{U_1}, \beta_{U_2}\}\), the solution \((\phi_{U_1}, \phi_{U_2}, \phi_L)\) satisfies \eqref{eq:1.16} for \(i = 1,2\) and \(j = 3 - i\). Also, as \(x \to -\infty\), \eqref{eq:1.17} holds for \(\phi_{U_j}\) when \(\alpha \in (\alpha^{**}, 1)\), and \eqref{eq:1.18} holds for \(\phi_{U_j}\) when \(\alpha = \alpha^{**}\).
    \end{enumerate}
    \label{thm:1.3}
\end{theorem}
There is a lot to unpack here, the stationary solutions of the system \eqref{eq:1.11} are rather complex as Theorem \ref{thm:1.3} shows. However, the long-time behavior of the solutions is easier to understand. The following theorem describes the long-time behavior of the solutions of the system \eqref{eq:1.11}.
\begin{remark}
    The behaviour of the solution at \(x \to -\infty\) plays a crucial role in determining the long-time dynamics.
\end{remark}
\begin{theorem}
    Assuming that \((w_{U_1, 0}, w_{U_2, 0}, w_{L, 0})\) satisfies \eqref{eq:1.13} and \(w_{i,0}\) is nontrivial for some \(i = U_1, U_2, L\). Let \((w_{U_1}, w_{U_2}, w_L)\) be the solution of \eqref{eq:1.11}. Then the following holds:
    \begin{enumerate}[label=(\roman*)]
        \item If \(\beta_{U_1}, \beta_{U_2} < 2\), then \((w_{U_1}(\cdot, t), w_{U_2}(\cdot, t), w_L(\cdot, t)) \to (1, 1, 1)\) locally uniformly as \(t \to +\infty\).
        \item If \(\beta_{U_1}, \beta_{U_2}, \beta_L \geq 2\), then \((w_{U_1}(\cdot, t), w_{U_2}(\cdot, t), w_L(\cdot, t)) \to (0, 0, 0)\) locally uniformly as \(t \to +\infty\). Also, \(\norm{w_{U_1}(\cdot, t)}_{L^\infty(\real_{U_1})} \to 0\), \(\norm{w_{U_2}(\cdot, t)}_{L^\infty(\real_{U_2})} \to 0\), and \(\norm{w_L(\cdot, t)}_{L^\infty(\real_L)} \to 1\) as \(t \to +\infty\).
        \item 
        \item If \(\beta_{U_1}, \beta_{U_2} \geq 2 > \beta_L\), then 
        \[
            (w_{U_1}(\cdot, t), w_{U_2}(\cdot, t), w_L(\cdot, t)) \to (\phi_{U_1}(\cdot; \alpha^{*}), \phi_{U_2}(\cdot; \alpha^{*}), \phi_L(\cdot; \alpha^{*}))
        \]
        locally uniformly as \(t \to +\infty\).
        \item If \(\beta_{U_1} \geq 2 > \beta_{U_2}\) (same reasoning if \(\beta_{U_2} \geq 2 > \beta_{U_1}\)), then
        \[
            (w_{U_1}(\cdot, t), w_{U_2}(\cdot, t), w_L(\cdot, t)) \to (\phi_{U_1}(\cdot; \alpha^{**}), \phi_{U_2}(\cdot; \alpha^{**}), \phi_L(\cdot; \alpha^{**}))
        \]
        locally uniformly as \(t \to +\infty\).
    \end{enumerate}
    \label{thm:1.4}
\end{theorem}
The long-time behavior of the solutions of the system \eqref{eq:1.11} have different behaviours in cases (iii) and (iv). In case (iii) both the upper branches have a flow speed higher than the critical speed \(c_*\), but the lower branch doesn't. This means that the population persists but stabilizes at a function which is increasing in the water direction. In case (iv) one of the upper branches has a flow speed higher than the critical speed \(c_*\), while the other one doesn't. In first branch the population will follow the same dynamics as in case (iii), while in the latter branch it will stabilize at a function that is increasing in the direction opposite to the water flow.

\begin{remark}
    It is possible to sum up the long-time behaviour of the solutions in all the different topologies in three cases:
    \begin{enumerate}[label=(\roman*)]
        \item \textbf{washing out:} if the flow speed in all the branches is higher than the critical speed \(c_*\), then the population will eventually be washed out.
        \item \textbf{persistence at carrying capacity:} if the flow speed in every upper branch is lower than the critical speed \(c_*\), then the population will reach the normalized carrying capacity \(1\) in every branch of the network.
        \item \textbf{persistence below carrying capacity:} if the flow speed in some upper branches is lower than the critical speed \(c_*\), but not in all of them, then the population in every branch of the network will reach a positive steady state strictly below the carrying capacity.
    \end{enumerate}
\end{remark}
