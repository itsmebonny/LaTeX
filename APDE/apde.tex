\documentclass[a4paper,12pt]{article}
\usepackage{amssymb}
\usepackage{amsmath}
\usepackage{hhline}
\usepackage{hyperref}
\usepackage{mathtools}
\usepackage{bm}
\usepackage[margin=2cm]{geometry}
\usepackage{pgfplots}

\usepgfplotslibrary{fillbetween}
\usetikzlibrary{patterns}

\usepackage{amsthm}
\usepackage{chngcntr}
\usepackage[makeroom]{cancel}

\usepackage{tabularx}
\usepackage{graphicx}
\usepackage{physics}
\usepackage{textcomp}
\setlength\parindent{0pt}

\newlength\mylength
\setlength\mylength{0.1cm}
\newcolumntype{Y}{>{\Centering\arraybackslash}X}

\AtBeginEnvironment{array}{\everymath{\displaystyle}}
\newtheoremstyle{break}
  {\partopsep}{\topsep}%  
  {\normalfont}{}
  {\bfseries}{}%
  {\newline}{}%
  \theoremstyle{break}
\newtheorem{theorem}{Theorem}[section]
\newtheorem{corollary}{Corollary}[section]
\newtheorem{proposition}{Proposition}[section]
\newtheorem{remark}{Remark}
\newtheorem{lemma}{Lemma}[section]
\newtheorem{exercise}{Exercise}
\counterwithin*{exercise}{subsection}
\counterwithin*{remark}{subsection}


\renewcommand*{\proofname}{\textbf{Proof}}
\renewcommand\qedsymbol{$\bigstar$}
\renewcommand*{\grad}{\nabla}
\renewcommand*{\div}{\text{div}}
\newtheorem{definition}{Definition}[section]
\renewcommand\labelenumi{(\theenumi)}
\newcommand{\vect}[1]{\textbf{#1}}

\let\oldemptyset\emptyset
\let\emptyset\varnothing
\let\oldepsilon\epsilon
\let\epsilon\varepsilon
\let\oldphi\phi
\let\phi\varphi


\newcommand*{\txt}[1]{\text{#1}}

\newcommand{\ind}{\perp\!\!\!\!\perp} 
\newcommand{\measurespace}{(X, \mathcal{M}, \mu)}
\newcommand{\sigalg}{\sigma\mbox{-algebra}}
\newcommand{\boreal}{\mathcal{B}(\mathbb{R})}
\renewcommand{\real}{\mathbb{R}}
\renewcommand{\natural}{\mathbb{N}}
\newcommand{\barreal}{\overline{\mathbb{R}}}
\newcommand{\code}[1]{\texttt{#1}}
\newcommand{\xdownarrow}[1]{%
  {\left\downarrow\vbox to #1{}\right.\kern-\nulldelimiterspace}
}
\newcommand{\xuparrow}[1]{%
  {\left\uparrow\vbox to #1{}\right.\kern-\nulldelimiterspace}
}
\newcommand{\arrvline}{\hfil\kern\arraycolsep\vline\kern-\arraycolsep\hfilneg}
\newcommand{\esssup}{\text{ess}\, \text{sup}}
\newcommand{\normdot}{\norm{\cdot}}
\newcommand{\scalardot}{\langle \cdot,\cdot \rangle}
\newcommand{\scalarproduct}[2]{\langle #1,#2 \rangle}

\newcommand{\interior}[1]{%
 {\kern0pt#1}^{\mathrm{o}}%
}
\def\stackbelow#1#2{\underset{\displaystyle\overset{\displaystyle\shortparallel}{#2}}{#1}}
\def\stackbelowlittle#1#2{\underset{\textstyle\overset{\textstyle\shortparallel}{#2}}{#1}}



\long\def\symbolfootnotemark[#1]#2{\begingroup%
\def\thefootnote{\fnsymbol{footnote}}\footnotetext[#1]{#2}\footnotemark[#1]\endgroup}

\long\def\symbolfootnotetext[#1]#2{\begingroup%
\def\thefootnote{\fnsymbol{footnote}}\footnotetext[#1]{#2}\endgroup}


\numberwithin{equation}{section}


\usepackage[many]{tcolorbox}


\tcolorboxenvironment{exercise}{
  breakable,
  colback=green!15!white,
  boxrule=0pt,
  boxsep=1pt,
  left=2pt,right=2pt,top=5pt,bottom=5pt,
  oversize=2pt,
  before skip=\topsep,
  after skip=\topsep,
}
\tcolorboxenvironment{remark}{
  breakable,
  colback=blue!20!white,
  boxrule=0pt,
  boxsep=1pt,
  left=2pt,right=2pt,top=2pt,bottom=2pt,
  oversize=2pt,
  before skip=\topsep,
  after skip=\topsep,
}



\begin{document}
\title{Advanced Partial Differential Equations \\ Exams}
\author{}
\date{}
\maketitle
This is a small collection of some of the past exams for the Advanced Partial Differential Equations course at Politecnico di Milano. I compiled this document to help other students prepare for the exam. I do not guarantee the correctness of the solutions, some are mine, some are my friends' and some are from the files available on the AIM website. If you find any mistakes, please let me know or correct them yourself. The source code for this document can be found on \href{https://github.com/itsmebonny/LaTeX/tree/main/APDE}{GitHub} (sorry in advance for my terrible LaTeX code).
\section{Exams 2021/22}
\subsection{June 2021}
\begin{exercise}
    For \(a, \gamma \in \real\), consider the Cauchy problem for the wave equation
    \begin{equation*}
        \begin{cases}
            u_{tt} - 4u_{xx} = 0 & (x,t) \in \real \times (0, \infty) \\
            u(x, 0) = e^{-x^2} + \gamma e^{-(x-a)^2} & x \in \real \\
            u_t(x, 0) = 0 & x \in \real
        \end{cases}
    \end{equation*}
    Show that the mass M of the solution is constant, then find the couples \((a, \gamma)\) such that:
    \begin{itemize}
        \item \(M = 0.\)
        \item The solution \(u(x,1)\) at \(t=1\) consists of only two ``bumps''.
    \end{itemize}
\end{exercise}

To show that the mass \(M\) of the solution is constant we need to define such a mass, and then check its behavior over time.


The mass \(M\) of the solution is defined as \(M(t) \coloneqq \int_\real u(x,t) \, dx\).

\[M''(t) = \frac{\partial^2}{\partial^2 t} \int_\real u(x,t) \, dx = \int_\real u_{tt} \, dx = \int_\real 4 u_{xx}\, dx = 4 \underbrace{\int_\real (u_x)_x\, dx}_{\text{div. form} = 0} = 0\]

So \(M(t) = A + Bt\). But \(M'(t) = B\), and also \(M'(t) = \int_\real u_t(x,t)\, dx\). Since \(M'(t)\) is constant we take \(M'(0) = \int_\real u_t(x,0)\, dx = 0 \Rightarrow B = 0\).

Then we conclude that \(M(t) = A\) is constant too, and \(M(0) = \int_\real e^{-x^2} + \gamma e^{-(x-a)^2} \, dx\)

\[
    \int_\real e^{-x^2} + \underbrace{\gamma e^{-(x-a)^2}}_{\substack{x-a=y \\ dx = dy}} \, dx = (1+ \gamma) \int_\real e^{-y^2} \, dy = \sqrt{\pi} (1 + \gamma)
\]

After that we need to show that \(M = 0\), so \(\sqrt{\pi}(1 + \gamma) = 0 \iff \gamma = -1\)

Then we look for the values of \(a\) such that the solution consists of two ``bumps''. 

\begin{remark}
    For a hyperbolic equation, we know that the solution \(u(x,t) = \frac{1}{2}(g(x+ct) + g(x-ct))\)
\end{remark}
In this case we have \(u(x, t) = \frac{1}{2}(\overbrace{g(x+2t)}^{\substack{\text{c is }2 \\\text{bc }4u_{xx}}} + g(x -2t))\), which becomes
\[
    u_{a, \gamma}(x, t) = \frac{1}{2}\left(e^{-(x+2t)^2} + \gamma e^{-(x-a+2t)^2} + e^{-(x-2t)^2} + \gamma e^{-(x-a-2t)^2} \right)
\]
that, for \(\gamma = - 1\) and \(t = 1\) 
\[
    u_{a, \gamma}(x, 1) = \frac{1}{2}\left(e^{-(x+2)^2} - e^{-(x-a+2)^2} + e^{-(x-2)^2} - e^{-(x-a-2)^2}\right)
\]
We can see that this solution has four ``bumps'' in \(x = -2, x = 2, x = a - 2, x = a + 2\). To obtain two bumps we manipulate \(a\) and see that 
\begin{itemize}
    \item \(-2 = a - 2 \Rightarrow a = 0 \Rightarrow u_{0, \gamma}(x, 1) = \frac{1}{2}(e^{-(x+2)^2} - e^{-(x+2)^2} + e^{-(x-2)^2} - e^{-(x-2)^2}) = 0\) 
    \item \(-2 = a + 2 \Rightarrow a = -4 \Rightarrow u_{-4, \gamma}(x, 1) = \frac{1}{2}(e^{-(x+2)^2} - e^{-(x+6)^2} + e^{-(x-2)^2} - e^{-(x+2)^2}) =\frac{1}{2}(- e^{-(x+6)^2} + e^{-(x-2)^2}) \) 
    \item \(2 = a -2 \Rightarrow a = 4 \Rightarrow u_{-4, \gamma}(x, 1) = \frac{1}{2}(e^{-(x+2)^2} - e^{-(x-2)^2} + e^{-(x-2)^2} - e^{-(x-6)^2}) =\frac{1}{2}(- e^{-(x-6)^2} + e^{-(x+2)^2})\) 
    \item \(-2 = a - 2 \Rightarrow a = 0 \Rightarrow u_{0, \gamma}(x, 1) = \frac{1}{2}(e^{-(x+2)^2} - e^{-(x+2)^2} + e^{-(x-2)^2} - e^{-(x-2)^2}) = 0\) 
\end{itemize}

We can see that the solution has only two ``bumps'' in the cases \(a = \pm 4\), so we conclude that the desired couples of \((a, \gamma)\) are 
\[
    \begin{cases}
        \gamma = -1\\
        a = -4
    \end{cases}
    \vee
    \begin{cases}
        \gamma = -1 \\ 
        a = 4
    \end{cases}
\]
\newpage
\begin{exercise}
Let \(\Omega \subset \real^n (n \geq 2)\) be a bounded smooth domain, let \(a\) be a measurable function in \(\Omega\).
Consider the problem
\[
    \begin{cases}
        - \Delta u = a(x) u^3 & \Omega \\
        u = 0 & \partial\Omega
    \end{cases}
    \tag*{(P)}
\]
Under which assumptions on the space dimension n can we write a variational formulation of problem (P) in
\(H^1_0(\Omega)\)? 

For each of these dimensions find the most general assumptions on \(a\) that allow to write the variational formulation. 

Finally, write the variational formulation.
\end{exercise}

First, a quick reminder on Sobolev embedding, which will be very useful a.e. in this document
\begin{remark}\label{sobolev_embedding}
    Let \(\Omega \subseteq \real^n\) open with \(\partial\Omega \in \text{Lip}\), \(s \geq 0\),
    \[
        H^s(\Omega) \subset 
        \begin{cases}
            L^p(\Omega) \qquad \forall \, 2 \leq p \leq 2^* & \text{if } n > 2s \\
            L^p(\Omega) \qquad \forall \, 2 \leq p < \infty & \text{if } n = 2s \\
            C^0(\bar{\Omega})  & \text{if } n < 2s
        \end{cases}
    \]
    Increasing \(s\) increases the regularity, while increasing \(n\) decreases it.

    The exponent \(2^*\) is called critical exponent and is defined as \(2^* \coloneqq \frac{2n}{n - 2s}\).

    If \(Omega\) is bounded, all these embeddings are compact except \(H^s(\Omega) \subset L^{2^*}\) when \(n > s\).
\end{remark}

Since we want to know the variational formulation in \(H^1_0\) we have \(s = 1\) and need to check \(n = 2, n \geq 3\). Remember a variational formulation makes sense if \(\int_\Omega fv < \infty\).
\begin{itemize}
    \item[\(n = 2\).] In this case we have \(u^3, v \in H^1_0(\Omega)\), so by Sobolev embedding we know \(u^3, v \in L^p(\Omega)\) for \(2 \leq p < \infty\). 
    \[
        \abs{\int_\Omega a(x) u^3 v}  \, dx \leq \int_\Omega \abs{a(x)} \abs{u^3} \abs{v} \, dx \underset{Holder}{\leq} \left(\int_\Omega \abs{a(x)}^r\right)^{\frac{1}{r}} \left(\int_\Omega \abs{u^3}^p \right)^{\frac{1}{p}} \left(\int_\Omega \abs{v}^q \right)^{\frac{1}{q}} < \infty.
    \]
    To use Holder inequality we need to find \(r, p, q\) such that \(\frac{1}{r} + \frac{1}{p} + \frac{1}{q} = 1\). We see that, 
    \[
        \frac{1}{r} + \frac{1}{p} + \frac{1}{q} = 1 \iff a(x) \in L^r(\Omega) \qquad \text{with } r > 1
    \]
    \item[\(n \geq 3\).] In this case we have \(u^3, v \in H^1_0(\Omega)\), so by Sobolev embedding we know \(u^3, v \in L^p(\Omega)\) for \(2 \leq p \leq 2^*\).
    We proceed as before, using Holder inequality, but decide to use \(p = \frac{2^*}{3}\) and \(q = \frac{1}{2^*}.\)
    \[
        \abs{\int_\Omega a(x) u^3 v}  \, dx \leq \int_\Omega \abs{a(x)} \abs{u^3} \abs{v} \, dx \underset{\mathclap{Holder}}{\leq} \left(\int_\Omega \abs{a(x)}^r\right)^{\frac{1}{r}} \left(\int_\Omega \abs{u}^{2^*} \right)^{\frac{3}{2^*}} \left(\int_\Omega \abs{v}^{2^*} \right)^{\frac{1}{2^*}} < \infty.
    \]
    In this case Holder inequality gives us 
    \[
        \frac{1}{r} + \frac{3}{2^*} + \frac{1}{2^*} = 1 \iff \frac{1}{r} = 1 - \frac{4}{2^*} \iff r = \frac{2^*}{2^* - 4}
    \]
    Substituting \(2^* = \frac{2n}{n - 2}\) we get \(r = \frac{n}{-n + 4}\). Since \(r > 0\) we need \(n < 4\).
    In this case we have \(a(x) \in L^3(\Omega)\) for \(n = 3\), but also \(a(x) \in L^\infty(\Omega)\) for \(n = 4\).
\end{itemize}

At this point we can write the weak formulation of the problem. We multiply the equation by a test function \(v \in H^1_0(\Omega)\) and obtain 
\[
    \int_\Omega - \Delta u v \, dx = \int_\Omega a(x) u^3 v \, dx \qquad \forall v \in H^1_0(\Omega)
\]
We integrate by parts the left-hand side and obtain
\[
    \int_\Omega \nabla u \nabla v \, dx = \int_\Omega a(x) u^3 v \, dx \qquad \forall v \in H^1_0(\Omega)
\]
This is the weak formulation of the problem. This is well posed if 
\begin{table}[h]
    \centering
    \begin{tabular}{|c|c|}
        \hline
        Dimension & Assumptions on $a(x)$ \\
        \hline
        $n = 2$ & $a \in L^r(\Omega)$, $r > 1$ \\
        $n = 3$ & $a \in L^3(\Omega)$ \\
        $n = 4$ & $a \in L^\infty(\Omega)$ \\
        $n \geq 5$ & No variational formulation \\
        \hline
    \end{tabular}
\end{table}
\newpage

\begin{exercise}
    Let \(\Omega \subset \real^n\) be a bounded open set of class \(C^1\), and let \(u\) be a sufficiently regular solution of the problem
    \[
        \begin{cases}
            u_t - \Delta u = 0 & \Omega \times (0, \infty) \\
            u = 0 & \partial\Omega \times (0, \infty) \\
            u(x, 0) = \alpha(x) & x \in \Omega
        \end{cases}
    \]
    Study monotonicity/boundedness properties of the energy \(E_u(t) = \int_\Omega \abs{\nabla u}^2 \, dx\).
\end{exercise}
The energy functional is defined as \(E_u(t) = \int_\Omega \abs{\nabla u}^2 \, dx = \norm{\nabla u}^2_{L^2(\Omega)}\). We want to study its behavior over time, so we need to compute its derivative with respect to time.
\[
    \begin{split}
        \frac{d}{dt} E_u(t) = \frac{d}{dt} \int_\Omega \abs{\nabla u}^2 \, dx = \int_\Omega \frac{d}{dt} \abs{\nabla u}^2 \, dx = \int_\Omega 2 \nabla u \cdot \nabla u_t \, dx = \\
        = \int_{\partial\Omega} 2 \underbrace{u \cdot \nu}_{=0} u_t - \int_\Omega 2 \Delta u u_t \, dx = - \int_\Omega 2 (\Delta u)^2 \, dx \leq 0.
    \end{split}
    \Rightarrow E \text{ is non-increasing}
\]
We see that the energy is non-increasing, since we obtain a positive quantity with a negative sign. Now we want to study the boundedness of the energy. We start by multiplying the equation by \(u\)  

\[
    \int_\Omega u_t u \, dx - \int_\Omega \Delta u u \, dx = 0
\]
Integrating by parts the second term we obtain
\[
    \int_\Omega u_t u \, dx - \int_{\partial\Omega} u \nabla u \cdot \nu \, dS + \int_\Omega (\nabla u)^2 \, dx = 0
\]
Since \(u = 0\) on the boundary we have
\[
    \int_\Omega u_t u \, dx = - \int_\Omega (\nabla u)^2 \, dx 
\]
We can rewrite the energy as
\[
    E_u(t) = \int_\Omega \abs{\nabla u}^2 \, dx = - \int_\Omega u_t u \, dx = - \frac{1}{2} \int_\Omega (u^2)_t \, dx
\]
Since the energy is non-increasing, we have that \(E_u(t) \leq E_u(0) \forall t \geq 0\), so we have 
\[
    E_u(t) = - \frac{1}{2} \int_\Omega (u^2)_t \, dx \leq - \frac{1}{2} \int_\Omega (u(x, 0)^2)_t \, dx = - \frac{1}{2} \int_\Omega \alpha(x)^2 \, dx
\]
Since \(\alpha(x)\) is bounded (is a function in \(H^1_0\)) we have that the energy is bounded too.

\newpage
\begin{exercise}
    Let \((X, \normdot)\) be a Banach space, and let \(u \in C^1([0, T]; X)\). Using the following abstract version of the \textit{Fundamental Theorem of Calculus}:
    \[
        \int_0^T u'(t) \, dt = u(T) - u(0)
    \]
    prove that \(\Lambda_{u'} = (\Lambda_u)' \in \mathcal{D}(0, T; X)\) where 
    \[
        \Lambda_f(\phi) = \int_0^T \phi(t) f(t) \, dt \qquad \forall f \in L^1(0, T; X)
    \]
\end{exercise}

By the definition of distributional derivative we have
\[
    (\Lambda_u(\phi))' = - \Lambda_u(\phi') \forall \phi \in \mathcal{D}(0, T)
\]
where 
\[
    \Lambda_u(\phi)' = - \int_0^T \phi'(t) u(t) \, dt
\]
We can integrate by parts the above expression
\[
    \begin{split}
        (\Lambda_u(\phi))' = - \int_0^T \phi'(t) u(t) \, dt =  \underbrace{\left. - \phi(t) u(t) \right|_0^T}_{=0} + \int_0^T \phi(t) u'(t) \, dt = \int_0^T \phi(t) u'(t) \, dt = \Lambda_{u'}(\phi)
    \end{split}
\]
We have shown that \(\Lambda_{u'} = (\Lambda_u)'\) in \(\mathcal{D}(0, T; X)\).

\newpage
\subsection{July 2021}
\begin{exercise}
    Let \(\Omega \subseteq \real^2\) be a bounded open set of class \(C^\infty\), let \(f \in L^2(\Omega)\). Consider the Dirichlet problem
    \[
        \begin{cases}
            2\partial^2_{x} u + 3\partial^2_{y} u + 2\partial_{xy} u = f & \Omega \\
            u = 0 & \partial\Omega
        \end{cases}
        \tag*{(P)}
    \]
    \begin{enumerate}
        \item Prove that (P) admits a unique solution \(u \in H^1_0(\Omega)\).
        \item What is the minimum \(m \in \natural\) for which \(f \in H^m(\Omega)\) implies \(u \in H^5(\Omega)\)?
    \end{enumerate}
\end{exercise}
We can rewrite the equation in the form \(\div( A \grad u)\) with \(A = \begin{pmatrix} 2 & 1 \\ 1 & 3 \end{pmatrix}\). Let's check if \(A\) is the correct matrix.
\[
    \begin{split}
        \div(A \grad u) = \div\left(\begin{pmatrix} 2 & 1 \\ 1 & 3 \end{pmatrix} \begin{pmatrix} u_x \\ u_y \end{pmatrix}\right) = \div\left(\begin{pmatrix} 2u_x + u_y \\ u_x + 3u_y \end{pmatrix}\right) = 2u_{xx} + 3u_{yy} + 2u_{xy}
    \end{split}
\]
Our Hilbert triplet is \(H^1_0(\Omega) \subset L^2(\Omega) \subset H^{-1}(\Omega)\). We define \(V = H^1_0(\Omega)\), \(H = L^2(\Omega)\), \(V' = H^{-1}(\Omega)\). We can now write the weak formulation of the problem. We multiply the equation by a test function \(v \in V\) and obtain
\[
    \int_\Omega \div(A \grad u) v \, dx = \int_\Omega f v \, dx \qquad \forall v \in V
\]
We integrate by parts the left-hand side and obtain
\[
    \int_\Omega A \grad u \cdot \grad v \, dx = \int_\Omega f v \, dx \qquad \forall v \in V
\]
This is the weak formulation of the problem. Now we use Lax-Milgram theorem to prove the existence and uniqueness of the solution. We need to check the coercivity and boundedness of the bilinear form. We have that the bilinear form is 
\[
    a(u, v) = \int_\Omega A \grad u \cdot \grad v \, dx
\]
A bilinear form is continuous if there exists a constant \(C > 0\) such that
\[
    \abs{a(u, v)} \leq C \norm{u}_V \norm{v}_V
\]
\begin{remark}
    In \(H^1_0(\Omega)\) we have the norm \(\norm{u}_V = \norm{\grad u}_{L^2}\)
\end{remark}
We write the bilinear form explicitly and bound it
\[
    \begin{split}
        \abs{a(u, v)} = \abs{\int_\Omega A \grad u \cdot \grad v \, dx} \leq \int_\Omega \abs{A} \abs{\grad u} \abs{\grad v} \, dx \leq \abs{A} \norm{\grad u}_{L^2} \norm{\grad v}_{L^2} = \abs{A} \norm{u}_V \norm{v}_V
    \end{split}
\]
We have that the bilinear form is continuous. We need to check the coercivity of the bilinear form. A bilinear form is coercive if there exists a constant \(c > 0\) such that
\[
    a(u, u) \geq c \norm{u}_V^2
\]
We write the bilinear form explicitly 
\[
    \begin{split}
        a(u, u) = \int_\Omega A \grad u \cdot \grad u \, dx = \int_\Omega \abs{A} \abs{\grad u}^2 \, dx = \abs{A} \norm{\grad u}_{L^2}^2 \geq \norm{\grad u}_{L^2}^2 = \norm{u}_V^2
    \end{split}
\]
We have that the bilinear form is coercive. By Lax-Milgram theorem we have that the problem admits a unique solution \(u \in V\).

The second request is about the minimum \(m\) such that \(f \in H^m(\Omega)\) implies \(u \in H^5(\Omega)\). 
\begin{remark}
    We know that if \(f \in H^m(\Omega)\) then \(u \in H^{m+2}(\Omega)\).
\end{remark}
We have that \(f \in H^m(\Omega)\) implies \(u \in H^{m+2}(\Omega)\), so we need \(m+2 \geq 5 \Rightarrow m \geq 3\). The minimum \(m\) is 3.

\newpage
\begin{exercise}
    Find solitary waves for the problem
    \[
        \begin{cases}
            u_t - u_{xx} - u_x^2 = 0 & \real \times (0, \infty) \\
            u(x, 0) = g(x) & x \in \real
        \end{cases}
    \]
    Moreover, discuss mass and momentum conservation for general solutions \(u \in S(\real)\) of (P).
\end{exercise}
Quick reminder about the solitary waves for parabolic equations. 
\begin{remark}
    In the case of a parabolic equation, we have that the solution \(u(x,t) =  g(x + ct)\) where \(c\) is the speed of the wave.
\end{remark}
We are working with solution of the form \(u(x,t) = g(x + ct)\), so we substitute this solution in the equation and obtain
\[
    \begin{split}
        cg'(x+ct) - g''(x+ct) - (g'(x+ct))^2 = 0 \Rightarrow cg'(x+ct) - g''(x+ct) = (g'(x+ct))^2
    \end{split}
\]
We perform a change of variable \(s = x + ct\) and obtain
\[
    \begin{split}
        cg'(s) - g''(s) = (g'(s))^2
    \end{split}
\]
At this point we are working with an ODE, so we can solve it. We start by defining \(y(s) = g'(s)\) and obtain
\[
    \begin{split}
        cy(s) - y'(s) = y(s)^2 \Rightarrow y'(s) = y(s)^2 - cy(s) 
    \end{split}
\]
To solve this we introduce 
\[
    z(s) = \frac{1}{y(s)} \Rightarrow z'(s) = - \frac{y'(s)}{y(s)^2} 
\]
We substitute \(y'(s)\) and obtain
\[
    \begin{split}
        z'(s) = - \frac{cy(s) - y(s)^2}{y(s)^2} = - c \frac{1}{y(s)} + 1 \Rightarrow z'(s) + c z(s) = 1
    \end{split}
\]
Solving this ODE we obtain
\[
    \begin{split}
        z(s) = e^{-cs} \left(k + \int_0^s e^{ct} \, dt\right) = e^{-cs} \left(k + \left. \frac{e^{ct}}{c} \right|_0^s\right) = e^{-cs} \left(k + \frac{e^{cs} - 1}{c}\right) = k e^{-cs} + \frac{1}{c} - \frac{e^{-cs}}{c} = \\
        = e^{-cs} \left(k - \frac{1}{c}\right) + \frac{1}{c} = k_0 e^{-cs} + \frac{1}{c}
    \end{split}
\]
At this point we use the definition of \(z(s)\) and obtain
\[
    \begin{split}
        y(s) = \frac{1}{z(s)} = \frac{1}{k_0 e^{-cs} + \frac{1}{c}} = \frac{c e^{cs}}{c k_0 + e^{cs}} = \frac{c e^{cs}}{k_1 + e^{cs}}
    \end{split}
\]
We have found the solution for \(g'(s)\), so we can integrate it to find \(g(s)\)
\[
    \begin{split}
        g(s) = \int_0^s \frac{c e^{cs}}{k_1 + e^{cs}} \, ds = \log(k_1 + e^{cs}) + k_2
    \end{split}
\]
We have found the solution for \(g(s) = \log(k_1 + e^{cs}) + k_2\).

Now we can discuss mass and momentum conservation for general solutions \(u \in S(\real)\) of (P). We start by defining the mass and momentum of the solution
\[
    M(t) = \int_\real u(x,t) \, dx
\]
\[
    \mathcal{M}(t) = \int_\real u(x,t)^2 \, dx
\]
We compute the derivative of the mass
\[
    \begin{split}
        M'(t) = \frac{d}{dt} \int_\real u(x,t) \, dx = \int_\real u_t(x,t) \, dx = \int_\real u_{xx}(x,t) + u_x(x,t)^2 \, dx =\\
        = \int_\real \underbrace{(u_x)_x}_{\text{div. form} = 0} + u_x^2 \, dx = \int_\real u_x^2 \, dx \geq 0
    \end{split}
\]
We do not have mass conservation, since mass is not constant over time. 

We compute the derivative of the momentum
\[
    \begin{split}
        \mathcal{M}'(t) = \frac{d}{dt} \int_\real u(x,t)^2 \, dx = \int_\real 2 u(x,t) u_t(x,t) \, dx = \int_\real 2 u u_{xx} + 2 u(x,t) u_x^2 \, dx =  \\
        = \int_\real 2 u u_{xx} + \int_\real 2 u u_x^2 \, dx = 2\left(\cancel{\left. u u_x \right|_\real} - \int_\real u_x^2 \, dx\right) + \int_\real 2 u u_x^2 \, dx = \\
        = 2 \int_\real (u - 1) u_x^2 \, dx
    \end{split}
\]
As we can see, the momentum is not conserved either.

\newpage
\begin{exercise}
    By using the Helmoltz-Weyl theorem and the variational formulation of the Stokes problem, explain how to derive the role of pressure.
\end{exercise}
\begin{remark}
    We introduce three spaces:
    \begin{itemize}
        \item \(\bm{G}_1 \coloneqq \left\{ f \in \bm{L}^2(\Omega) \mid \grad \cdot f = 0, \gamma_\nu f = 0 \right\}\)
        \item \(\bm{G}_2 \coloneqq \left\{ f \in \bm{L}^2(\Omega) \mid \grad \cdot f = 0, \exists g \in H^1(\Omega) \text{ s.t. } f = \grad g \right\}\)
        \item \(\bm{G}_3 \coloneqq \left\{ f \in \bm{L}^2(\Omega) \mid \exists g \in H^1_0(\Omega) \text{ s.t. } f = \grad g \right\}\)
    \end{itemize}
    We also introduce the space \(\bm{V} \coloneqq \left\{ f \in \bm{H}^1_0(\Omega) \mid \grad \cdot f = 0 \right\}\) which is the space of divergence-free functions.
    We know that \(\bm{V}\) is dense in \(\bm{G}_1\).

    A famous result by Helmoltz and Weyl states that the spaces \(\bm{G}_1, \bm{G}_2, \bm{G}_3\) are mutually orthogonal in \(\bm{L}^2(\Omega)\) and that \(\bm{L}^2(\Omega) = \bm{G}_1 \oplus \bm{G}_2 \oplus \bm{G}_3\).
\end{remark}

We start by writing the strong formulation of the Stokes problem with \(f \in \bm{L}^2(\Omega)\)
\[
    \begin{cases}
        - \eta \Delta u + \grad p = f & \Omega \\
        \grad \cdot u = 0 & \Omega \\
        u = 0 & \partial\Omega
    \end{cases}
    \tag*{(S)}
\]
Then we multiply the equation by a test function \(v \in \bm{V}\) to obtain the weak formulation
\[
    \int_\Omega - \eta \grad u : \grad v + \int_\Omega p \grad \cdot v = \int_\Omega f v \qquad \forall v \in \bm{V}
\]
By the Helmoltz-Weyl theorem we know that \(\grad p \in \bm{G}_2 \oplus \bm{G}_3\), so when we multiply the equation by a test function \(v \in V\) we have
\[
    \int_\Omega \grad p \cdot v = 0
\]
since \(\bm{V}\) is dense in \(\bm{G}_1\) and \(\bm{G_1}\) is orthogonal to \(\bm{G}_2 \oplus \bm{G}_3\). We can now write the variational formulation of the Stokes problem
\[
    \int_\real - \eta \grad u : \grad v = \int_\Omega f v \qquad \forall v \in \bm{V}
\]
Now we observe that for every \(f \in \bm{L}^2\) the function \(v \mapsto \int_\Omega f v\) is a bounded linear functional on \(\bm{V}\). Then, by Lax-Milgram corollary we obtain 
\[
    \forall f \in \bm{L}^2 \quad \exists! u \in \bm{V} \text{ s.t. } \int_\Omega - \eta \grad u : \grad v = \int_\Omega f v \qquad \forall v \in \bm{V}
\]
Also, thanks to elliptic regularity we have that \(u \in \bm{H}^2(\Omega)\), so we have 
\[
    \forall f \in \bm{L}^2 \quad \exists! u \in \bm{H}^2 \cap \bm{V} \text{ s.t. } \int_\Omega - \eta \grad u : \grad v = \int_\Omega f v \qquad \forall v \in \bm{V}
\]
Since \(\bm{V}\) is dense in \(\bm{G}_1\) we rewrite it as 
\[
    \forall f \in \bm{L}^2 \quad \exists! u \in \bm{H}^2 \cap \bm{V} \text{ s.t. } \int_\Omega (\eta \Delta u + f) v = 0 \qquad \forall v \in \bm{G}_1
\]
As for \(\grad p\), this means that \((\eta \Delta u + f) \in \bm{G}_2 \oplus \bm{G}_3\).
Thanks to this finding we can write 
\[
    \exists! p \in \bm{H}^1/\real \text{ s.t. } -\grad p = \eta \Delta u + f
\]
where the space \(\bm{H}^1/\real\) is the space of functions in \(\bm{H}^1\) up to a constant. 

So we have \(\underbrace{-\eta \Delta u}_{\in \bm{G}_1 \oplus \bm{G}_2} + \underbrace{\grad p}_{\in \bm{G}_2 \oplus \bm{G}_3} = \underbrace{f}_{{\qquad   \mathclap{\in \bm{G}_1 \oplus \bm{G}_2 \oplus \bm{G}_3}}} \in \bm{L}^2\). This means that the role of the pressure is to satisfy the equation projected on \(\bm{G}_2\).


\newpage
\subsection{September 2021}
\begin{exercise}
    Find solitary waves for the problem
    \[
        \begin{cases}
            u_t - u_{xxx} = 0 & \real \times (0, \infty) \\
            u(x, 0) = g(x) & x \in \real
        \end{cases}
        \tag*{(P)}
    \]
    Moreover, discuss mass and momentum conservation for general solutions \(u \in S(\real)\) of (P).
\end{exercise}
We start by finding the solitary waves for the problem. We know that the solution is of the form \(u(x,t) = g(x + ct)\), so we substitute this solution in the equation and obtain
\[
    \begin{split}
        cg'(x+ct) - g'''(x+ct) = 0 \Rightarrow cg'(x+ct) = g'''(x+ct)
    \end{split}
\]
We perform a change of variable \(s = x + ct\) and obtain
\[
    \begin{split}
        cg'(s) = g'''(s)
    \end{split}
\]
At this point we are working with an ODE, so we can solve it. We start by defining \(p(l)\) as the characteristic polynomial of the ODE
\[
    \begin{split}
        p(l) = l^3 - c l = l(l^2 - c) = 0
    \end{split}
\]
and now study the behavior of the roots of the polynomial when \(c > 0\), \(c = 0\), \(c < 0\).
\begin{itemize}
    \item[\(c > 0\).] We have three real roots \(l = 0, \sqrt{c}, -\sqrt{c}\). The general solution is
    \[
        g(s) = k_1 + k_2 e^{\sqrt{c}s} + k_3 e^{-\sqrt{c}s}
    \]
    \item[\(c = 0\).] We have a triple root \(l = 0\). The general solution is
    \[
        g(s) = k_1 + k_2 s + k_3 s^2
    \]
    \item[\(c < 0\).] We have a complex conjugate pair of roots \(l = 0, \pm i \sqrt{-c}\). The general solution is
    \[
        g(s) = k_1 + k_2 \cos(\sqrt{-c}s) + k_3 \sin(\sqrt{-c}s)
    \]
\end{itemize}
We have found the solution for \(g(s)\). Now we can discuss mass and momentum conservation.

Defining the mass and momentum of the solution as
\[
    M(t) = \int_\real u(x,t) \, dx
\]
\[
    \mathcal{M}(t) = \int_\real u(x,t)^2 \, dx
\]
Starting from the mass, we take its derivative
\[
    \begin{split}
        M'(t) = \frac{d}{dt} \int_\real u(x,t) \, dx = \int_\real u_t(x,t) \, dx = \int_\real u_{xxx}(x,t) \, dx = \int_\real \underbrace{(u_xx)_x}_{\text{div. form} = 0} \, dx = 0
    \end{split}
\]
We have mass conservation, since mass is constant over time.

We compute the derivative of the momentum
\[
    \begin{split}
        \mathcal{M}'(t) = \frac{d}{dt} \int_\real u(x,t)^2 \, dx = \int_\real 2 u(x,t) u_t(x,t) \, dx = \int_\real 2 u u_{xxx} \, dx = 2\left(\cancel{\left. u u_{xx} \right|_\real} - \int_\real u_x u_{xx} \, dx\right) = \\
        = - \int_\real \underbrace{(u_x^2)_x}_{\text{div. form} = 0} \, dx = 0
    \end{split}
\]
We have also momentum conservation.

\newpage
\begin{exercise}
    Let \(\Omega \coloneqq B(0, 1) \subset \real^n\) with \(n \geq 2\), and let \(f \in H^3(\Omega)\). Justify or confute the following statements:
    \begin{enumerate}
        \item one can surely conclude that \(f \in C(\Omega)\);
        \item one can surely conclude that \(\gamma_2(f) \in H^{1/2}(\partial\Omega)\);
        \item one can surely exclude that \(\gamma_0(f) \in H^{1}(\partial\Omega)\);
        \item if \(n = 8\), then \(f \in L^{15/2}(\Omega)\).
    \end{enumerate}
\end{exercise}
We start by recalling the Sobolev embeddings with \(2s = 6\)
\begin{align*}
    H^3(\Omega) &\subset C(\Omega) && \text{ if } n < 6 \\
    H^3(\Omega) &\subset L^p(\Omega) \qquad \forall 2 \leq p < \infty && \text{ if } n = 6 \\
    H^3(\Omega) &\subset L^p(\Omega) \qquad \forall 2 \leq p \leq \frac{2n}{n - 6} && \text{ if } n > 6
\end{align*}
Then we check the statements
\begin{enumerate}
    \item In this case we have that \(f \in C(\Omega)\) if \(n < 6\). Since \(n \geq 2\) we can surely conclude that \(f \in C(\Omega)\).
    \item In this case we recall that \(\gamma_j(f) \in H^{s - j - 1/2}(\partial\Omega)\). In the case of \(f \in H^3(\Omega)\) we have that \(\gamma_2(f) \in H^{3 - 2 - 1/2}(\partial\Omega) = H^{1/2}(\partial\Omega)\). We can surely conclude that \(\gamma_2(f) \in H^{1/2}(\partial\Omega)\).
    \item In this case we proceed as before, but with \(j = 0\). We have that \(\gamma_0(f) \in H^{3 - 0 - 1/2}(\partial\Omega) = H^{5/2}(\partial\Omega)\). Since \(H^{5/2} \subset H^1\) we cannot surely exclude that \(\gamma_0(f) \in H^{1}(\partial\Omega)\).
    \item In this case we have \(n = 8\) so we need to check if \(f \in L^p(\Omega)\) with \(2\leq p \leq 2^*\). The critical exponent is \(p = \frac{2\cdot 8}{8 - 6} = 8\). Since \(15/2 < 8\) we can surely conclude that \(f \in L^{15/2}(\Omega)\). 
\end{enumerate}

\newpage
\begin{exercise}
    Let \(\ell > 0\) and consider the eigenvalue problem
    \[
        \begin{cases}
            \Delta^2 u + \lambda u_{xx} = 0 & (0, \pi) \times (-\ell, \ell) \\
            u =  \Delta u = 0 & \partial\left[(0, \pi) \times (-\ell, \ell)\right]
        \end{cases}
        \tag*{(P)}
    \]
    Prove that \(\lambda = 1\) is not an eigenvalue of (P). For which values of \(\ell\) is the least eigenvalue double?
\end{exercise}

For this problem we know that the eigenvalues of the problem are of the form
\[
    \lambda_{m,n} = m^2 +  \frac{n^2 \pi^2}{\ell^2} \qquad m, n \in \natural \iff \begin{cases}
        -\Delta u = \lambda u & \text{ in } \Omega \\
        u = 0 & \text{ on } \partial\Omega
    \end{cases} \text{ has a non-trivial solution}
\]
The space of eigenfunctions is given by
\[
    EF = \left\{ u_{m,n}(x, y) = \sin(mx) \sin\left(\frac{n \pi y}{\ell}\right) \mid m, n \in \natural \right\}
\]
Now we can rewrite the Laplacian operator as
\[
    \Delta u_{m,n} = - \lambda_{m,n} u_{m,n} 
\]
While the two derivatives are
\[
    \begin{split}
        u_x = -m \cos(mx) \sin\left(\frac{n \pi y}{\ell}\right) \qquad u_{xx} = -m^2 \sin(mx) \sin\left(\frac{n \pi y}{\ell}\right) 
    \end{split}
\]
so \(u_{xx} = -m^2 u_{m,n}\). As for the bi-Laplacian operator we have
\[
    \Delta^2 u_{m,n} = - \Delta \lambda_{m,n} u_{m,n} = - \lambda_{m,n}^2 u_{m,n}
\]
We can now substitute the derivatives in the equation and obtain
\[
    \begin{split}
        - \lambda_{m,n}^2 u_{m,n} + \lambda m^2 u_{m,n} = 0 \Rightarrow \lambda = \frac{\lambda_{m,n}^2}{m^2}
    \end{split}
\]
We have obtained an explicit expression for \(\lambda\)
\[
    \lambda = \frac{1}{m^2} \left(m^2 + \frac{n^2 \pi^2}{\ell^2}\right)^2
\] 
Putting this expression equal to \(1\) we obtain
\[
    \begin{split}
        m^2 = \left(m^2 + \frac{n^2 \pi^2}{\ell^2}\right)^2 \Rightarrow  m = m^2 + \frac{n^2 \pi^2}{\ell^2} \Rightarrow m - m^2 = \frac{n^2 \pi^2}{\ell^2}
    \end{split}
\]
Then, since \(m, n \in \natural\) we have
\[
    \begin{split}
        n^2 = \frac{\pi^2}{\ell^2} \left(m - m^2\right) > 0 \Rightarrow (m - m^2) > 0 \Rightarrow m(m - 1) < 0 \Rightarrow m \in (0,1)
    \end{split}
\]
Since \(m \in \natural\) we have that this is impossible, so \(\lambda = 1\) is not an eigenvalue of (P).

Now we want to find the values of \(\ell\) for which the least eigenvalue is double. To do so we fix \(n = 1\), since we know that the smallest eigenvalue is \(\lambda_{1,1} = \left(1 + \frac{\pi^2}{\ell^2}\right)^2\), and obtain
\[
    \lambda_{m, n}^2 \geq \lambda_{m, 1}^2 = \left( m^2 + \frac{\pi^2}{\ell^2} \right)^2 \qquad \forall m \in \natural
\]
Since we are looking for the least eigenvalue, we are looking at a minimization problem. 
\[
    \begin{split}
        \min_{m \in \natural} \lambda_{m, 1}^2 = \min_{m \in \natural} \left( m + \frac{\pi^2}{\ell^2 m} \right)^2 = \mu^*
    \end{split}
\]
This is the same as minimizing the function \(f(x) = (x + \frac{a^2}{x})^2\) with \(a = \frac{\pi}{\ell}\).
What we know about \(f\)?
\begin{itemize}
    \item \(f : [1, \infty) \to \real\)
    \item \(f \in C^\infty\)
    \item \(f'(x) = \geq 0 \iff x \geq a\)
    \item \(\lim_{x \to \infty} f(x) = \infty\)
\end{itemize}
If \(a > 1, a \in \natural\) means that \(\mu^*\) can have multiplicity greater than \(1\). We choose \(1 < a < 2\) so that \((\ell < \pi < 2\ell)\) and \(\mu^* = \min\left\{\lambda_{1,1}^2, \lambda_{2,1}^2\right\}\). We define \(\mu_{m,n} = \lambda_{m,n}^2\) and we have 
\[
    \begin{split}
        \mu_{1,1} = \left(1 + \frac{\pi^2}{\ell^2}\right)^2 \qquad \mu_{2,1} = \left(2 + \frac{\pi^2}{2 \ell^2}\right)^2.
    \end{split}
\]
To have them ordered \(\mu_{1,1} < \mu_{2,1} \iff \pi \leq \ell \sqrt{2}\). Then we have three cases:
\begin{itemize}
    \item \(\ell \leq \pi < \ell \sqrt{2} \Rightarrow \mu^* = \mu_{1,1}\) which is simple;
    \item \(\ell \sqrt{2} < \pi < 2\ell \Rightarrow \mu^* = \mu_{2,1}\) which is simple;
    \item \(\pi = \ell \sqrt{2} \Rightarrow \mu^* = \mu_{1,1} = \mu_{2,1}\) which is double.
\end{itemize}
So, the value of \(\ell\) for which the least eigenvalue is double is \(\ell = \frac{\pi}{\sqrt{2}}\).

\newpage
\subsection{January 2022}
\begin{exercise}
    For the Korteweg-de Vries equation
    \[
        \begin{cases}
            u_t + u_{xxx} + 6 u u_x = 0 & \real \times (0, \infty) \\
            u(x, 0) = g(x) & x \in \real
        \end{cases}
        \tag*{(P)}
    \]
    prove that the ``energy'' \(E(t) = \int_\real (u_x^2 - 2 u^3) \, dx\) is conserved for general solutions \(u \in S(\real)\) of (P).
\end{exercise}
We start by computing the derivative of the energy
\[
    \begin{split}
        E'(t) = \frac{d}{dt} \int_\real (u_x^2 - 2 u^3) \, dx = \int_\real 2 u_x u_{xt} - 6 u^2 u_t \, dx = \int_\real 2 u_x u_{xt} \, dx - 6 \int_\real u^2 u_t \, dx
    \end{split}
\]
Dividing the problem in two parts we have
\[
    \begin{split}
        \int_\real 2 u_x u_{xt} \, dx = \cancel{\left.2 u_x u_t \right|_\real} - \int_\real 2u_{xx} u_t \, dx = - \int_\real 2u_{xx} ( - u_{xxx} - 6 u u_x) \, dx = \\
        = \int_\real 2u_{xx} u_{xxx} + 12 u \underbrace{u_x u_{xx}}_{=\left(\frac{u_x^2}{2}\right)_x} \, dx = \int_\real \underbrace{(u_{xx}^2)_x}_{=0} + \cancel{\left. 12 u \frac{u_x^2}{2} \right|_\real} - 12 \int_\real u_x \frac{u_x^2}{2} \, dx = - 6 \int_\real u_x^3 \, dx
    \end{split}
\]
For the second part we have
\[
    \begin{split}
        6 \int_\real u^2 u_t \, dx = 6 \int_\real u^2 (-u_{xxx} - 6 u u_x) \, dx = -\int_\real 6 u^2 u_{xxx} - \int_\real 36 \underbrace{u^3 u_x}_{=\left(\frac{u^4}{4}\right)_x} \, dx = \\ 
        = \cancel{\left. - 6 u^2 u_{xx} \right|_\real} + 6 \int_\real 2 u u_{x} u_{xx} - 9 \int_\real \underbrace{(u^4)_x}_{=0} \, dx = 12 \int_\real u u_x u_{xx} \, dx = \\
        =12 \int_\real u\left(\frac{u^2}{2}\right)_x \, dx = \cancel{\left. 6 u u_x^2 \right|_\real} - 6 \int_\real u_x u^2_x \, dx = - 6 \int_\real u_x^3 \, dx
    \end{split}
\]
To recap, we have
\begin{align*}
    E'(t) &= 2\int_\real u_x u_{xt} \, dx - 6 \int_\real u^2 u_t \, dx  \\
    2\int_\real  u_x u_{xt} \, dx &= - 6 \int_\real u_x^3 \, dx \\
    6\int_\real  u^2 u_t \, dx &= - 6 \int_\real u_x^3 \, dx
\end{align*} 
We can now substitute these results in the derivative of the energy
\[
    \begin{split}
        E'(t) = - 6 \int_\real u_x^3 \, dx + 6 \int_\real u_x^3 \, dx = 0
    \end{split}
\]
We have that the energy is conserved for general solutions \(u \in S(\real)\) of (P).

\newpage
\begin{exercise}
    Let \(\Omega \subset \real^n\) be a bounded domain of class \(C^1\), and let \(f \in L^2(0, T; L^2(\Omega))\). Moreover, let \(u_0 \in H^1(\Omega)\) and \(u_1 \in L^2(\Omega)\). Prove that, for every \(\gamma > 0\), the Galerkin method can be applied to the problem
    \[
        \begin{cases}
            u_{tt} - \Delta u + \gamma u = f & \Omega \times (0, T) \\
            \partial_\nu u = 0 & \partial\Omega \times (0, T) \\
            u = u_0 & \Omega \times \{0\} \\
            u_t = u_1 & \Omega \times \{0\}
        \end{cases}
        \tag*{(P)}
    \]
    which, therefore, admits a unique solution.
\end{exercise}
We start by writing the weak formulation of the problem, choosing adequate function spaces. We define the spaces
\[
    \begin{split}
        V = H^1(\Omega) \subseteq H = L^2(\Omega) \qquad V' = \left(H^1(\Omega)\right)'
    \end{split}
\]
We also need to introduce the space of weakly continuous functions over \([0, T]\).
\begin{remark}
    Let \(H\) be a Hilbert space. The space of weakly continuous functions over \([0, T]\) is defined as
    \[
        \begin{split}
            C_w^0([0, T]; H) = \left\{ u \in L^\infty(0, T; H) \mid \lim_{t \to t_0} (u(t) - u(t_0), v)_H = 0, \quad \forall t_0 \in [0, T], \forall v \in H \right\}
        \end{split}
    \] 
\end{remark}
We can now write the weak formulation of the problem (P), by multiplying the equation by a test function \(v \in V\) and integrating over \(\Omega\)
\begin{align*}
    \int_\Omega f(t) v \, dx &= \int_\Omega u_{tt} v \, dx + \int_\Omega (-\Delta u  + \gamma u)v \, dx =\\
    &= \int_\Omega u_{tt} v \, dx - \cancel{\int_{\partial\Omega} \partial_\nu u v \, d\sigma} + \int_\Omega \left[\grad u \cdot \grad v + \gamma u v\right] \, dx = \\
    &= \int_\Omega u_{tt} v \, dx + \int_\Omega \underbrace{\grad u \cdot \grad v + \gamma u v}_{B(u, v)} \, dx
\end{align*}
To write it more compactly we have 
\[
    \begin{split}
        \frac{d^2}{dt^2} (u, v)_{L^2} + B(u, v) = (f, v)_{L^2} \qquad \forall v \in V
    \end{split}
\]
Another important abstract result is the following
\begin{remark}
    If \(B(u, v)\) is continuous and coercive, \(u_0 \in V\) and \(u_1 \in H\) and \(f \in L^2(0, T; V')\), then we know that 
    \[
        \exists! u \in C_w^0([0, T]; V) \cap C^0([0, T]; H) \text{ with } u_t \in C_w^0([0, T]; H), \quad u_{tt} \in L^2(0, T; V') \text{ for (P).}
    \]
\end{remark}
So, we need to check if \(B(u, v)\) is continuous and coercive. We start by checking the continuity of \(B(u, v)\)
\[
    \begin{split}
        \abs{B(u, v)} \leq \int_\Omega \abs{\grad u \cdot \grad v} + \abs{\gamma} \abs{u} \abs{v} \overset{\text{CS + H}}{\leq} \underbrace{\norm{\grad u}_{L^2}}_{\leq \norm{u}_V} \norm{\grad v}_{L^2} + \abs{\gamma} \norm{u}_{L^2} \norm{v}_{L^2} \leq \\ 
        \leq (1 + \abs{\gamma}) \norm{u}_V \norm{v}_V \leq C \norm{u}_V \norm{v}_V
    \end{split}
\]
We have that \(B(u, v)\) is continuous. We now check if it is coercive
\[
    \begin{split}
        B(u, u) = \int_\Omega \abs{\grad u}^2 + \abs{\gamma} \abs{u}^2 = (1 + \abs{\gamma}) \int_\real \abs{\grad u}^2  + \abs{u}^2 \geq (1 + \abs{\gamma}) \norm{u}_V^2
    \end{split}
\]
We need that \(\alpha \geq 0\) to obtain \(B(u,u) \geq \min\{\alpha, 1\} \norm{u}_V^2\). Since \(\gamma > 0\) by hypothesis, we have that \(B(u, v)\) is coercive. We can now apply the abstract result and conclude that the solution of (P) exists and is unique.

\newpage
\begin{exercise}
    Let \(\Omega \coloneqq B(0, 1) \subset \real^n\), and let 
    \[
        f(x) \coloneqq \frac{e^{\abs{x}}-1}{\abs{x}^\alpha}, \quad \text{with } \alpha > 0
    \]
    Find the values of \(\alpha\) for which \(f \in H^1(\Omega)\).
\end{exercise}
To check that a function is in \(H^1(\Omega)\) we need that
\begin{remark}
    \[
        H^1(\Omega) = \left\{ f \in L^2(\Omega) \mid \grad f \in L^2(\Omega) \right\}
    \]  
\end{remark}
We start by checking if \(f \in L^2(\Omega)\), so we check \(f(x) \in L^2(\Omega) \iff \int_\Omega \abs{f(x)}^2 \, dx < \infty\).
\[
    \begin{split}
        \int_\Omega \abs{f(x)}^2 \, dx = \int_\Omega \abs{\frac{e^{\abs{x}}-1}{\abs{x}^\alpha}}^2 \, dx
    \end{split}
\]
Since our domain is a ball, we can use spherical coordinates to compute the integral
\[
    \begin{split}
        \int_\Omega \abs{\frac{e^{\abs{x}}-1}{\abs{x}^\alpha}}^2 \, dx = \int_0^1 \int_{\left\{\norm{x} = \rho\right\}} \abs{\frac{e^{\abs{\rho}}-1}{\rho^\alpha}}^2 \, d\sigma \, d\rho = \\
        = \sigma_n \int_0^1 \abs{\frac{e^{\abs{\rho}}-1}{\rho^\alpha}}^2 \rho^{n-1} \, d\rho = \sigma_n \int_0^1 \abs{\frac{(e^{\abs{\rho}}-1)^2}{\rho^{2\alpha - n +1}}} \, d\rho
    \end{split}
\]
At this point we know 
\[
    \begin{split}
        e^{\abs{\rho}} -1 \overset{\rho \to 0}{\longrightarrow} \rho \Rightarrow \int_0^1 \abs{\frac{\rho^2}{\rho^{2\alpha - n + 1}}} \, d\rho < \infty \iff 2 \alpha - n + 1 -2 < 1 \iff \alpha < \frac{n + 2}{2}
    \end{split}
\]
So we have that \(f \in L^2(\Omega)\) if \(\alpha < \frac{n + 2}{2}\). We now need to check if \(\grad f \in L^2(\Omega)\). We start by computing the gradient of \(f\)
\[
    \begin{split}
        \partial x_i f &= \frac{e^{\abs{x}} \abs{x}^{\alpha} - \left(e^{\abs{x}} - 1\right) \alpha \abs{x}^{\alpha - 1} x_i}{\abs{x}^{2\alpha}} \frac{x_i}{\abs{x}} = \frac{e^{\abs{x}} \abs{x} - \left(e^{\abs{x}} - 1\right) \alpha}{\abs{x}^{2\alpha + 1}} \frac{x_i}{\abs{x}} \\
        \norm{\grad f}^2 &= \frac{\norm{\abs{x} e^{\abs{x}} - \alpha e^{\abs{x}} + \alpha}}{\abs{x}^{\alpha + 1}}
    \end{split}
\]
Now we need to check if this function is in \(L^2(\Omega)\)
\[
    \begin{split}
        \int_\Omega \norm{\grad f}^2 \, dx = \int_\Omega \left(\frac{\norm{\abs{x} e^{\abs{x}} - \alpha e^{\abs{x}} + \alpha}}{\abs{x}^{2\alpha + 1}}\right)^2 \, dx = \\
        = \int_0^1 \int_{\left\{\norm{x} = \rho\right\}} \left(\frac{\rho e^{\rho} - \alpha e^{\rho} + \alpha}{\rho^{\alpha + 1}}\right)^2 \, d\sigma \, d\rho = \sigma_n \int_0^1 \frac{\left(\rho e^{\rho} - \alpha e^{\rho} + \alpha\right)^2}{\rho^{2\alpha + 2}} \rho^{n-1} \, d\rho = \\
    \end{split}
\]
We can now check if this integral is finite
\[
    \begin{split}
        \rho e^{\rho} - \alpha e^{\rho} + \alpha \overset{\rho \to 0}{\longrightarrow} \rho \cancel{-\alpha} + \cancel{\alpha} = \rho \\
        \Rightarrow \int_0^1 \frac{\rho^2}{\rho^{2\alpha + 2 - n + 1}} \, d\rho < \infty \iff 2\alpha + 2 - n + 1 - 2 < 1 \iff \alpha < \frac{n}{2}
    \end{split}
\]
Now that we have checked that \(\grad f \in L^2(\Omega)\) and \(f \in L^2(\Omega)\) we only need to check that the weak derivative of \(f\) exists, since if it exists it is equal to the classical one. We have that the weak derivative of \(f\) exists if
\[
    \begin{split}
        \int_\Omega f \partial x_i \phi \, dx = - \int_\Omega \partial x_i f \phi \, dx \qquad \forall \phi \in \mathcal{D}(\Omega)
    \end{split}
    \tag*{E1}
\]
To check that this condition is satisfied we need to cut off the singularity of \(f\) in the origin. We can do this by defining
\[
    \Omega_\epsilon = B(0, 1) \setminus B(0, \epsilon) = B_1 \setminus B_\epsilon = \left\{ x \in \real^n \mid \epsilon < \norm{x} < 1 \right\}
\]
Since \(f \in C^1(\Omega_\epsilon)\) we can apply the divergence theorem to the weak derivative definition and obtain
\[
    \begin{split}
        \int_{\Omega_\epsilon} f \partial x_i \phi \, dx = \int_{\Omega_\epsilon} \partial x_i f \phi \, dx - \int_{\partial\Omega_\epsilon} f \phi \nu_i \, d\sigma
    \end{split}
    \tag*{E2}
\]
We want check that taking the limit \(\epsilon \to 0\) in (E2) we obtain (E1). To do so we need to check that the boundary term goes to zero, and the other two terms are equal at the ones in (E1). We start with the first term
\[
    \begin{split}
        \int_{\Omega_\epsilon} f \partial x_i \phi \, dx = \int_{B_1} f \partial x_i \phi \chi_{\Omega_\epsilon} \, dx
    \end{split}
\]
We want to claim that 
\[
    \begin{split}
        \lim_{\epsilon \to 0^+} \int_{\Omega_\epsilon} f \partial x_i \phi \, dx = \int_{B_1} f \partial x_i \phi \, dx
    \end{split}
\]
To do so we need to swap the limit and the integral. We see that 
\begin{itemize}
    \item \(f \partial x_i \phi \chi_{\Omega_\epsilon} \underset{\epsilon \to 0}{\longrightarrow} f \partial x_i \phi\) a.e. in \(B_1\)
    \item \(\abs{f \partial x_i \phi \chi_{\Omega_\epsilon}} \leq \underbrace{\abs{f}}_{\in L^p(B_1)} \overbrace{\abs{\partial x_i \phi}}^{\in L^q(B_1)} \in L^1(B_1)\)
\end{itemize}
We can now apply the Dominated Convergence Theorem and obtain the desired result. 
The same process can be applied to
\[
    \begin{split}
        \int_{\Omega_\epsilon} \partial x_i f \phi \, dx = \int_{B_1} \partial x_i f \phi \chi_{\Omega_\epsilon} \, dx
    \end{split}
\]
Then we have shown that 
\[
    \begin{split}
        \int_{B_1} f \partial x_i \phi \, dx = \int_{\partial\Omega_\epsilon} f \phi \nu_i \, d\sigma + \int_{B_1} \partial x_i f \phi \, dx
    \end{split}
\]
It is clear that we need to check that the boundary term goes to zero. Since we know that \(\text{supp } \phi \subseteq B_\epsilon\).
\[
    \begin{split}
        \int_{\partial\Omega_\epsilon} f \phi \nu_i \, d\sigma = \cancel{\int_{\partial B_1} f \phi \nu_i \, d\sigma} + \int_{\partial B_\epsilon} f \phi \nu_i \, d\sigma
    \end{split}
\]
Moreover,
\[
    \begin{split}
        \abs{\int_{\partial B_\epsilon} f \phi \nu_i \, d\sigma} \leq \int_{\partial B_\epsilon} \abs{f} \abs{\phi} \underbrace{\abs{\nu_i}}_{=1} \, d\sigma \leq \max_{\partial B_\epsilon} \abs{\phi} \int_{\{ \norm{x} = \epsilon \}} \abs{f} \, d\sigma = \\
        \max_{\partial B_\epsilon} \abs{\phi} \frac{e^{\epsilon} - 1}{\epsilon^\alpha} \mu\{\norm{x} = \epsilon\} \leq \max_{\partial B_\epsilon} \abs{\phi} \frac{e^{\epsilon} - 1}{\epsilon^\alpha} \epsilon^{n-1} 
    \end{split}
\]
We have that 
\[
    \begin{split}
        \lim_{\epsilon \to 0^+} \max_{\partial B_\epsilon} \abs{\phi} \frac{e^{\epsilon} - 1}{\epsilon^\alpha} \epsilon^{n-1} = 0 \iff \alpha - 1 - n + 1 > 1 \iff \alpha < n + 1
    \end{split}
\]

Quick recap of the values of \(\alpha\) for which \(f \in H^1(\Omega)\) %in a table 
\begin{table}[h]
    \centering
        \begin{tabular}{c|c}
            \(\alpha \in (0, \frac{n + 2}{2})\) & \(f \in L^2(\Omega)\) \\
            \(\alpha \in (0, \frac{n}{2})\) & \(\grad f \in L^2(\Omega)\) \\
            \(\alpha \in (0, n + 1)\) & \(\text{weak derivative of } f \text{ exists}\)
        \end{tabular}
\end{table}

Since all three conditions are necessary, we have that \(\alpha \in (0, \min\{\frac{n + 2}{2}, \frac{n}{2}, n + 1\}) = (0, \frac{n}{2})\).

\newpage
\begin{exercise}
Let \(\Omega \coloneqq (0,2) \subseteq \real\),and let 
\[
    f(x) \coloneqq \begin{cases}
        x & x \in (0, 1) \\
        1 & x \in [1, 2)
    \end{cases}
\]
Find the values of \(m\) for which \(f \in H^m(\Omega)\).
\end{exercise}
For a function to be in \(H^m(\Omega)\) we need that its weak derivative exists up to order \(m\) and belongs to \(L^2(\Omega)\). 
\begin{remark}
    \[
        H^m(\Omega) = \left\{ f \in L^2(\Omega) \mid D^\alpha f \in L^2(\Omega) \quad \forall \alpha \leq m \right\}
    \]
\end{remark}
We start by computing the classical derivatives of \(f\)
\[
    \begin{split}
        f'(x) = \begin{cases}
            1 & x \in (0, 1) \\
            0 & x \in [1, 2)
        \end{cases} \qquad f''(x) = f'''(x)  = \ldots = f^{(m)}(x) = 0 \quad \forall m \geq 2
    \end{split}
\]
Let's now check if \(f \in L^2(\Omega)\).
\[
    \begin{split}
        \int_\Omega \abs{f(x)}^2 \, dx = \int_0^1 x^2 \, dx + \int_1^2 1 \, dx = \frac{1}{3} + 2 - 1 = \frac{4}{3} < \infty \Rightarrow f \in L^2(\Omega)
    \end{split}
\]
Now check if, for \(m = 1\), \(f^{(m)} \in L^2(\Omega)\), since it is the only non-trivial derivative of \(f\).
\[
    \begin{split}
        \int_\Omega \abs{f'(x)}^2 \, dx = \int_0^1 1 \, dx = 1 < \infty \Rightarrow f' \in L^2(\Omega)
    \end{split}
\] 
At this point let's see if the weak derivative of \(f\) exists. We need to check if
\[
    \begin{split}
        \int_\Omega f \phi' \, dx = - \int_\Omega f' \phi \, dx \qquad \forall \phi \in \mathcal{D}(\Omega)
    \end{split}
\]
Substituting both \(f\) and \(f'\) we have
\[
    \begin{split}
        \int_0^1 x \phi' \, dx + \int_1^2 \phi' \, dx = - \int_0^1 \phi \, dx \qquad \forall \phi \in \mathcal{D}(\Omega)
    \end{split}
\]
Check by integrating by parts
\begin{align*}
    \int_0^1 x \phi' \, dx &= \left. x \phi \right|_0^1 - \int_0^1 \phi \, dx = \phi(1) - \int_0^1 \phi \, dx \\
    \int_1^2 \phi' \, dx &= \left. \phi \right|_1^2 = \phi(2) - \phi(1) \underset{\phi \in (0,2)}{=} - \phi(1)
\end{align*}
Let's put everything together
\[
    \begin{split}
        \int_\Omega f \phi' \, dx = \cancel{\phi(1)} - \int_0^1 \phi \, dx - \cancel{\phi(1)} = - \int_0^1 \phi \, dx \qquad \forall \phi \in \mathcal{D}(\Omega)
    \end{split}
\]
We have shown that the weak derivative of \(f\) exists and is equal to the classical one. We have shown that \(f \in H^1(\Omega)\).

Now let's see if the weak derivative of \(f'\) exists. We need to check if
\[
    \begin{split}
        \int_\Omega f' \phi' \, dx = - \int_\Omega f'' \phi \, dx \qquad \forall \phi \in \mathcal{D}(\Omega)
    \end{split}
\]
Substituting both \(f'\) and \(f''\) we have
\[
    \begin{split}
        \int_0^1 \phi' \, dx = - \int_0^1 0 \, dx \qquad \forall \phi \in \mathcal{D}(\Omega)
    \end{split}
\]
Computing both integrals we have
\begin{align*}
    \int_0^1 \phi' \, dx &= \left. \phi \right|_0^1 = \phi(1) - \phi(0) = \phi(1) \\
    \int_0^1 0 \, dx &= 0
\end{align*}
As we can see the two integrals are not equal, so the weak derivative of \(f'\) does not exist. So the only weak derivative of \(f\) that exists is the first one, and we have shown that \(f \in H^1(\Omega)\).

\newpage
\subsection{February 2022}
\begin{exercise}
    Let \(\Omega \subset \real^n (n \geq 2)\) be a bounded smooth domain, let \(a\) be a measurable function in \(\Omega\).
    Consider the problem
    \[
        \begin{cases}
            - \Delta u = a(x) u^5 & \Omega \\
            u = 0 & \partial\Omega
        \end{cases}
        \tag*{(P)}
    \]
    Under which assumptions on the space dimension n can we write a variational formulation of problem (P) in
    \(H^1_0(\Omega)\)? 
    
    For each of these dimensions find the most general assumptions on \(a\) that allow to write the variational formulation. 
    
    Finally, write the variational formulation.
    \end{exercise}

    Since we want to know the variational formulation in \(H^1_0\) we have \(s = 1\) and need to check \(n = 2, n \geq 3\). 
    
    Remember a variational formulation makes sense if \(\int_\Omega fv < \infty\).
    \begin{itemize}
        \item[\(n = 2\).] In this case we have \(u^5, v \in H^1_0(\Omega)\), so by Sobolev embedding we know \(u^5, v \in L^p(\Omega)\) for \(2 \leq p < \infty\). 
        \[
            \abs{\int_\Omega a(x) u^5 v}  \, dx \leq \int_\Omega \abs{a(x)} \abs{u^5} \abs{v} \, dx \underset{Holder}{\leq} \left(\int_\Omega \abs{a(x)}^r\right)^{\frac{1}{r}} \left(\int_\Omega \abs{u^5}^p \right)^{\frac{1}{p}} \left(\int_\Omega \abs{v}^q \right)^{\frac{1}{q}} < \infty.
        \]
        To use Holder inequality we need to find \(r, p, q\) such that \(\frac{1}{r} + \frac{1}{p} + \frac{1}{q} = 1\). We see that, 
        \[
            \frac{1}{r} + \frac{1}{p} + \frac{1}{q} = 1 \iff a(x) \in L^r(\Omega) \qquad \text{with } r > 1
        \]
        \item[\(n \geq 3\).] In this case we have \(u^5, v \in H^1_0(\Omega)\), so by Sobolev embedding we know \(u^5, v \in L^p(\Omega)\) for \(2 \leq p \leq 2^*\).
        We proceed as before, using Holder inequality, but decide to use \(p = \frac{2^*}{5}\) and \(q = \frac{1}{2^*}.\)
        \[
            \begin{split}
                \abs{\int_\Omega a(x) u^5 v}  \, dx \leq \int_\Omega \abs{a(x)} \abs{u^5} \abs{v} \, dx \underset{{Holder}}{\leq} \\
                \leq \left(\int_\Omega \abs{a(x)}^r\right)^{\frac{1}{r}} \left(\int_\Omega \abs{u}^{2^*} \right)^{\frac{5}{2^*}} \left(\int_\Omega \abs{v}^{2^*} \right)^{\frac{1}{2^*}} < \infty.
            \end{split}
        \]
        In this case Holder inequality gives us 
        \[
            \frac{1}{r} + \frac{5}{2^*} + \frac{1}{2^*} = 1 \iff \frac{1}{r} = 1 - \frac{6}{2^*} \iff r = \frac{2^*}{2^*-6}
        \]
        Substituting \(2^* = \frac{2n}{n - 2}\) we get \(r = \frac{n}{-2n + 6}\). Since \(r > 0\) we need \(n < 3\).
        In this case we only have \(a(x) \in L^\infty(\Omega)\) for \(n = 3\).
    \end{itemize}
    \newpage
    
    At this point we can write the weak formulation of the problem. We multiply the equation by a test function \(v \in H^1_0(\Omega)\) and obtain 
    \[
        \int_\Omega - \Delta u v \, dx = \int_\Omega a(x) u^5 v \, dx \qquad \forall v \in H^1_0(\Omega)
    \]
    We integrate by parts the left-hand side and obtain
    \[
        \int_\Omega \nabla u \nabla v \, dx = \int_\Omega a(x) u^5 v \, dx \qquad \forall v \in H^1_0(\Omega)
    \]
    This is the weak formulation of the problem. This is well posed if 
    \begin{table}[h]
        \centering
        \begin{tabular}{|c|c|}
            \hline
            Dimension & Assumptions on $a(x)$ \\
            \hline
            $n = 2$ & $a \in L^r(\Omega)$, $r > 1$ \\
            $n = 3$ & $a \in L^\infty(\Omega)$ \\
            $n \geq 4$ & No variational formulation \\
            \hline
        \end{tabular}
    \end{table}
    
    \newpage
\begin{exercise}
    Let \(B = \left\{ x \in \real^n \mid \abs{x} < 1 \right\}\). For which values of \(p \in [1, \infty) \) is the function
    \[
        f(x) = \frac{e^{-\abs{x}^2}}{\abs{x}^2}
    \]
    in \(L^p(B)\) and in \(W^{1, p}(B)\)?
\end{exercise}
The strategy for this exercise remain the same as before.
\begin{remark}
    A function belong to \(W^{1, p}(B)\) if its weak derivative exists and belongs to \(L^p(B)\).
\end{remark}
We start by checking if \(f \in L^p(B)\)
\[
    \begin{split}
        \int_B \abs{f(x)}^p \, dx = \int_B \abs{\frac{e^{-\abs{x}^2}}{\abs{x}^2}}^p \, dx = \int_B \frac{e^{-\abs{x}^2p}}{\abs{x}^{2p}} \, dx = \\
        = \int_0^1 \int_{\left\{\abs{x} = \rho\right\}} \frac{e^{-\rho^2 p}}{\rho^{2p}} \, d\sigma \, d\rho = \sigma_n \int_0^1 \frac{e^{-\rho^2 p}}{\rho^{2p}} \rho^{n-1} \, d\rho
    \end{split}
\]
Since \(e^{-\rho^2 p} \overset{\rho \to 0}{\longrightarrow} 1\) we have that the integral is finite if
\[
    \begin{split}
        \int_0^1 \abs{\frac{1}{\rho^{2p}} \rho^{n-1}} \, d\rho < \infty \iff 2p - n + 1 < 1 \iff 1 \leq p < \frac{n}{2} \Rightarrow n \geq 3
    \end{split}
\]
We have that \(f \in L^p(B)\) if \(p \in [1, \frac{n}{2})\). We now need to check if \(f \in W^{1, p}(B)\). We start by computing the gradient of \(f\)
\[
    \begin{split}
        \partial x_i f = \frac{-2 \abs{x} e^{-\abs{x}^2} \abs{x}^2 - e^{-\abs{x}^2} 2\abs{x}}{\abs{x}^4} \frac{x_i}{\abs{x}} = \frac{-2 \abs{x}^2 e^{-\abs{x}^2} - 2 e^{-\abs{x}^2}}{\abs{x}^3} \frac{x_i}{\abs{x}} 
    \end{split}
\]
with 
\[
    \begin{split}
        \norm{\grad f} = \frac{2 e^{-\abs{x}^2} \left(\abs{x}^2 + 1\right)}{\abs{x}^3}
    \end{split}
\]
Now we need to check if \(\grad f \in L^p(B)\)
\[
    \begin{split}
        \int_B \abs{\grad f}^p \, dx = \int_B \abs{\frac{2 e^{-\abs{x}^2} \left(\abs{x}^2 + 1\right)}{\abs{x}^3}}^p \, dx = \int_B  \frac{2e^{-\abs{x}^2p} \left(\abs{x}^2 + 1\right)^p} {\abs{x}^{-3p}} \, dx = \\
        \int_0^1 \int_{\left\{\abs{x} = \rho\right\}}  \frac{2e^{-\rho^2 p} \left(\rho^2 + 1\right)^p} {\rho^{3p}} \, d\sigma \, d\rho = \sigma_n \int_0^1  \frac{2e^{-\rho^2 p} \left(\rho^2 + 1\right)^p} {\rho^{3p}} \rho^{n-1} \, d\rho
    \end{split}
\]
Also in this case we have \(e^{-\rho^2 p} \overset{\rho \to 0}{\longrightarrow} 1\) and \(\rho^2 + 1 \overset{\rho \to 0}{\longrightarrow} 1\), so we need that
\[
    \begin{split}
        \int_0^1 \abs{\frac{1} {\rho^{3p}} \rho^{n-1}} \, d\rho < \infty \iff 3p - n + 1 < 1 \iff 1 \leq p < \frac{n}{3} \Rightarrow n \geq 4
    \end{split}
\]
Now that we have checked that \(\grad f \in L^2(\Omega)\) and \(f \in L^2(\Omega)\) we only need to check that the weak derivative of \(f\) exists, since if it exists it is equal to the classical one. The weak derivative of \(f\) exists if
\[
    \begin{split}
        \int_\Omega f \partial x_i \phi \, dx = - \int_\Omega \partial x_i f \phi \, dx \qquad \forall \phi \in \mathcal{D}(\Omega)
    \end{split}
    \tag*{E1}
\]
To check that this condition is satisfied we need to cut off the singularity of \(f\) in the origin. We can do this by defining
\[
    \Omega_\epsilon = B(0, 1) \setminus B(0, \epsilon) = B_1 \setminus B_\epsilon = \left\{ x \in \real^n \mid \epsilon < \norm{x} < 1 \right\}
\]
Since \(f \in C^1(\Omega_\epsilon)\) we can apply the divergence theorem to the weak derivative definition and obtain
\[
    \begin{split}
        \int_{\Omega_\epsilon} f \partial x_i \phi \, dx = \int_{\Omega_\epsilon} \partial x_i f \phi \, dx - \int_{\partial\Omega_\epsilon} f \phi \nu_i \, d\sigma
    \end{split}
    \tag*{E2}
\]
We want check that taking the limit \(\epsilon \to 0\) in (E2) we obtain (E1). To do so we need to check that the boundary term goes to zero, and the other two terms are equal at the ones in (E1). We start with the first term
\[
    \begin{split}
        \int_{\Omega_\epsilon} f \partial x_i \phi \, dx = \int_{B_1} f \partial x_i \phi \chi_{\Omega_\epsilon} \, dx
    \end{split}
\]
We want to claim that 
\[
    \begin{split}
        \lim_{\epsilon \to 0^+} \int_{\Omega_\epsilon} f \partial x_i \phi \, dx = \int_{B_1} f \partial x_i \phi \, dx
    \end{split}
\]
To do so we need to swap the limit and the integral. We see that 
\begin{itemize}
    \item \(f \partial x_i \phi \chi_{\Omega_\epsilon} \underset{\epsilon \to 0}{\longrightarrow} f \partial x_i \phi\) a.e. in \(B_1\)
    \item \(\abs{f \partial x_i \phi \chi_{\Omega_\epsilon}} \leq \underbrace{\abs{f}}_{\in L^p(B_1)} \overbrace{\abs{\partial x_i \phi}}^{\in L^q(B_1)} \in L^1(B_1)\)
\end{itemize}
We can now apply the Dominated Convergence Theorem and obtain the desired result. 
The same process can be applied to
\[
    \begin{split}
        \int_{\Omega_\epsilon} \partial x_i f \phi \, dx = \int_{B_1} \partial x_i f \phi \chi_{\Omega_\epsilon} \, dx
    \end{split}
\]
Then we have shown that 
\[
    \begin{split}
        \int_{B_1} f \partial x_i \phi \, dx = \int_{\partial\Omega_\epsilon} f \phi \nu_i \, d\sigma + \int_{B_1} \partial x_i f \phi \, dx
    \end{split}
\]
It is clear that we need to check that the boundary term goes to zero. Since we know that \(\text{supp } \phi \subseteq B_\epsilon\).
\[
    \begin{split}
        \int_{\partial\Omega_\epsilon} f \phi \nu_i \, d\sigma = \cancel{\int_{\partial B_1} f \phi \nu_i \, d\sigma} + \int_{\partial B_\epsilon} f \phi \nu_i \, d\sigma
    \end{split}
\]
Moreover,
\[
    \begin{split}
        \abs{\int_{\partial B_\epsilon} f \phi \nu_i \, d\sigma} \leq \int_{\partial B_\epsilon} \abs{f} \abs{\phi} \underbrace{\abs{\nu_i}}_{=1} \, d\sigma \leq \max_{\partial B_\epsilon} \abs{\phi} \int_{\{ \norm{x} = \epsilon \}} \abs{f} \, d\sigma = \\
        \max_{\partial B_\epsilon} \abs{\phi} \frac{e^{\epsilon^2}}{\epsilon^2} \mu\{\norm{x} = \epsilon\} \leq \max_{\partial B_\epsilon} \abs{\phi} \frac{e^{\epsilon^2}}{\epsilon^2} \epsilon^{n-1}
    \end{split}
\]
We have that 
\[
    \begin{split}
        \lim_{\epsilon \to 0^+} \max_{\partial B_\epsilon} \abs{\phi} \frac{e^{\epsilon^2}}{\epsilon^2} \epsilon^{n-1} = \lim_{\epsilon \to 0^+} \max_{\partial B_\epsilon} \abs{\phi} e^{\epsilon^2} \epsilon^{n-3} = 0 \text{ since } n \geq 4
    \end{split}
\]

So we have shown that \(f \in W^{1, p}(B) \iff n > 3 \land 1 \leq p < \frac{n}{3}\).

\newpage
\begin{exercise}
    Let \((X, \norm{\cdot})\) be a Banach space and let \(v \in X\) such that \(\norm{v} = 1\). Moreover, let \(T > 0\) and let \(t_0 \in (0, T)\) be fixed. Prove that the map 
    \begin{align*}
        \Lambda_{t_0, v} \colon \mathcal{D}([0, T]) &\to X \\
        \Lambda_{t_0, v}(\phi) &\coloneqq \phi(t_0) v \in X
    \end{align*}
    belongs to \(\mathcal{D}'([0, T], X)\).
\end{exercise}
To show that \(\Lambda_{t_0, v} \in \mathcal{D}'([0, T], X)\) we need to show that it is a continuous linear functional on \(\mathcal{D}([0, T])\). 
Before we start, a quick reminder on the properties of series of functions in \(\mathcal{D}([0, T])\).
\begin{remark}
    Let \(\left\{ \phi_n \right\}_{n} \subseteq \mathcal{D}([0, T])\) be a sequence of functions that converges to \(\phi \in \mathcal{D}([0, T])\). Then we have that
    \begin{itemize}
        \item \(\exists \text{ compact }[a, b] \subseteq [0, T] \colon \text{supp } \phi_n \subset [a, b] \quad \forall n \in \natural\)
        \item \(\phi_n^{(k)} \to \phi^{(k)}\) uniformly on \([a, b]\) for all \(k \in \natural\)
    \end{itemize}
\end{remark}
Checking if \(\Lambda_{t_0, v}\) is linear we need to show that \(\Lambda_{t_0, v}(\alpha \phi + \beta \psi) = \alpha \Lambda_{t_0, v}(\phi) + \beta \Lambda_{t_0, v}(\psi)\). We have that
\[
    \begin{split}
        \Lambda_{t_0, v}(\alpha \phi + \beta \psi) = \alpha \phi(t_0) v + \beta \psi(t_0) v = \alpha \Lambda_{t_0, v}(\phi) + \beta \Lambda_{t_0, v}(\psi)
    \end{split}
\]
So our map is linear. Now we need to show that it is continuous. 
To do that we take advantage of the properties of the sequence of functions in \(\mathcal{D}([0, T])\). Let \(\left\{ \phi_n \right\}_{n} \subseteq \mathcal{D}([0, T])\) be a sequence of functions that converges to \(\phi \in \mathcal{D}([0, T])\). 

We have that \(\norm{\phi_n - \phi} \to 0\) as \(n \to \infty\). We can now compute the norm of the difference of the images of the functions
\[
    \begin{split}
        \norm{\Lambda_{t_0, v}(\phi_n) - \Lambda_{t_0, v}(\phi)}_X = \norm{\phi_n(t_0) v - \phi(t_0) v}_X =\\
        = \norm{(\phi_n(t_0) - \phi(t_0)) v}_X \leq \abs{\phi_n(t_0) - \phi(t_0)} \norm{v}_X \underset{n \to \infty}{\longrightarrow} 0
    \end{split}
\]
So our map is also continuous. We have shown that \(\Lambda_{t_0, v} \in \mathcal{D}'([0, T], X)\).

\newpage
\begin{exercise}
    Let \(\Omega \subset \real^n\) be a bounded open set. Prove that exists no solutions \(u \in C^2(\Omega) \cap C(\overline{\Omega})\) of the problem
    \[
        \begin{cases}
            \Delta u = 1 - u^2 & \Omega \\
            u = 0 & \partial\Omega
        \end{cases}
    \]
    which satisfy the a-priori estimate \(0 \leq u \leq 1\) in \(\Omega\).
\end{exercise}
To solve this we will need the maximum principle. 
\begin{remark}
    Let \(u \in C^2(\Omega) \cap C(\overline{\Omega})\), then 
    \[
        \begin{cases}
            \Delta u \geq 0 & \Omega \\
            u \leq 0 & \partial\Omega
        \end{cases} \Rightarrow u \leq 0 \text{ in } \Omega
    \]
    Which means the sign of \(u\) on the boundary is preserved in the interior. Also \(\max_{\overline{\Omega}} u = \max_{\partial\Omega} u\).
\end{remark}
In this case we have an a-priori estimate on \(u\), which means that \(0 \leq u \leq 1 \Rightarrow 1 - u^2 \geq 0\). We can now apply the maximum principle to the equation
\[
    \begin{cases}
        \Delta u = 1 - u^2 \geq 0 & \Omega \\
        u = 0 & \partial\Omega
    \end{cases}
\]
Since the maximum of \(u\) is on the boundary, we have that \(u \leq 0\) in \(\Omega\). By our estimate this means that \(0 \leq u \leq 1 \cap u \leq 0 \Rightarrow u = 0\). But this is not a solution to the equation, because
\[
    \begin{split}
        \Delta 0 = 1 - 0 = 1 \neq 0
    \end{split}
\]
So we have shown that there are no solutions \(u \in C^2(\Omega) \cap C(\overline{\Omega})\) of the problem which satisfy the a-priori estimate \(0 \leq u \leq 1\) in \(\Omega\).
\section{Exams 2022/23}
\subsection{June 2022}
\begin{exercise}
  Let \(\Omega \subseteq \real^2\) be a bounded open set of class \(C^1\), and let \(f \in L^2(\Omega)\). Consider the Dirichlet problem
    \begin{equation*}
        \begin{cases}
        -\left(2\partial_{x}^2 u + \partial_{y}^2 u + \partial_{xy} u\right) = f & \text{in } \Omega, \\
        u = 0 & \text{on } \partial \Omega.
        \end{cases}
        \tag*{(P)}
    \end{equation*}
\begin{itemize}
    \item For a suitable symmetric matrix \(A\), write the PDE appearing in (P) in the form \(\div(A \cdot \grad u) = f\).
    \item Write the variational formulation of (P) and show that there exists a unique solution \(u \in H_0^1(\Omega)\) using Lax-Milgram's theorem.
\end{itemize}
\end{exercise}
First, we choose a suitable symmetric matrix \(A\). In \(n = 2\) we have that the PDE can be written as \(A_{11} \partial_{x}^2 u + A_{22} \partial_{y}^2 u + 2A_{12} \partial_{xy} u\), so a suitable choice is
\[
A = \begin{pmatrix}
2 & 1/2 \\
1/2 & 1
\end{pmatrix}.
\]
Then, the PDE can be written as
\[
 - \div(A \cdot \grad u) = f.
\]
Now we write the variational formulation of (P). We start by choosing an appropriate Hilbert triple. Since we are dealing with a Dirichlet problem, we choose \(V = H_0^1(\Omega)\) and \(V' = H^{-1}(\Omega)\), with \(H = L^2(\Omega)\) to have \(V \subseteq H \subseteq V'\), and multiply the PDE by a test function \(v \in V\), integrate by parts, and obtain
\[
\int_{\Omega} - \div(A \cdot \grad u) v \, dx = \int_{\Omega} \, dx f \forall v \in V.
\]
Now we integrate by parts and obtain
\[
\int_{\Omega} A \cdot \grad u \cdot \grad v \, dx + \cancel{\int_{\partial \Omega} A \cdot \grad u \cdot \vect{n} v \, d\sigma}= \int_{\Omega} f v \, dx \forall v \in V.
\]
We define the bilinear form \(a(u,v) = \int_{\Omega} A \cdot \grad u \cdot \grad v \, dx\) and apply Lax-Milgram's theorem. Then we need to check that \(a\) is continuous and coercive. First we check continuity:
\[
\abs{a(u,v)} = \abs{\int_{\Omega} A \cdot \grad u \cdot \grad v \, dx} \leq \abs{A} \norm{\grad u}_{L^2} \norm{\grad v}_{L^2} \leq \norm{A} \norm{u}_{V} \norm{v}_{V}.
\]
Now we check coercivity, taking advantage of these two facts (\(\norm{u}_{L^2} \leq C_p \norm{\grad u}_{L^2}\) and \(\norm{u}_{V} = \norm{\grad u}_{L^2}\)):
\[
a(u,u) = \int_{\Omega} A \cdot \grad u \cdot \grad u \, dx \geq \abs{A} \norm{\grad u}_{L^2}^2 = \abs{A} \frac{1}{1 + C_p^2} \norm{u}_{V}^2 = \alpha \norm{u}_{V}^2.
\]
Therefore, we have that the bilinear form is continuous and coercive, and by Lax-Milgram's theorem, there exists a unique solution \(u \in V\) to the variational formulation of (P).

\newpage
\begin{exercise}
    Find solitary waves for the problem
    \[
        \begin{cases}
            u_t - 2u_{xx} - u_x^3 = 0 & \real \times (0, \infty) \\
            u(x, 0) = g(x) & x \in \real
        \end{cases}
    \]
    Moreover, discuss mass and momentum conservation for general solutions \(u \in S(\real)\) of (P).
\end{exercise}
Quick reminder about the solitary waves for parabolic equations. 
\begin{remark}
    In the case of a parabolic equation, we have that the solution \(u(x,t) =  g(x + ct)\) where \(c\) is the speed of the wave.
\end{remark}
We are working with solution of the form \(u(x,t) = g(x + ct)\), so we substitute this solution in the equation and obtain
\[
    \begin{split}
        cg'(x+ct) - 2g''(x+ct) - (g'(x+ct))^3 = 0 \Rightarrow cg'(x+ct) - g''(x+ct) = (g'(x+ct))^3
    \end{split}
\]
We perform a change of variable \(s = x + ct\) and obtain
\[
    \begin{split}
        cg'(s) - 2g''(s) = (g'(s))^3
    \end{split}
\]
At this point we are working with an ODE, so we can solve it. We start by defining \(y(s) = g'(s)\) and obtain
\[
    \begin{split}
        cy(s) - 2y'(s) = y(s)^3 \Rightarrow 2y'(s) = cy(s) - y(s)^3 
    \end{split}
\]
To solve this we introduce 
\[
    z(s) = \frac{1}{y(s)^2} \Rightarrow z'(s) = -2 \frac{y'(s)}{y(s)^3} 
\]
We substitute \(y'(s)\) and obtain
\[
    \begin{split}
        z'(s) = -2 \frac{y(s)^2 - cy(s)}{y(s)^3} = -c \frac{1}{y(s)^2} + 1 = -c z(s) + 1
    \end{split}
\]
Solving this ODE we obtain
\[
    \begin{split}
        z(s) = e^{-cs} \left(k + \int_0^s e^{ct} \, dt\right) = e^{-cs} \left(k + \left. \frac{e^{ct}}{c} \right|_0^s\right) = e^{-cs} \left(k + \frac{e^{cs} - 1}{c}\right) = k e^{-cs} + \frac{1}{c} - \frac{e^{-cs}}{c} = \\
        = e^{-cs} \left(k - \frac{1}{c}\right) + \frac{1}{c} = k_0 e^{-cs} + \frac{1}{c}
    \end{split}
\]
At this point we use the definition of \(z(s)\) and obtain
\[
    \begin{split}
        y(s)^2 = \frac{1}{z(s)} = \frac{1}{k_0 e^{-cs} + \frac{1}{c}} = \frac{c e^{cs}}{c k_0 + e^{cs}} = \frac{c e^{cs}}{k_1 + e^{cs}}
    \end{split}
\]
Then we compute the square root of this expression and obtain
\[
    \begin{split}
        y(s) = \pm \sqrt{\frac{c e^{cs}}{k_1 + e^{cs}}} = g'(s)
    \end{split}
\]
Since 
\[
    \begin{split}
        g'(s) = \pm \sqrt{\frac{c e^{cs}}{k_1 + e^{cs}}} 
    \end{split}
\]
At this point we can conclude \(\nexists \text{ solitary waves}\) for the problem and then \(\nexists \text{ global solutions}\) for the problem.

Now we can discuss mass and momentum conservation for general solutions \(u \in S(\real)\) of (P). We start by defining the mass and momentum of the solution
\[
    M(t) = \int_\real u(x,t) \, dx
\]
\[
    \mathcal{M}(t) = \int_\real u(x,t)^3 \, dx
\]
We compute the derivative of the mass
\[
    \begin{split}
        M'(t) = \frac{d}{dt} \int_\real u(x,t) \, dx = \int_\real u_t(x,t) \, dx = \int_\real 2u_{xx}(x,t) + u_x(x,t)^3 \, dx =\\ 
        = \int_\real \underbrace{2(u_x)_x}_{\text{div. form} = 0} + u_x^3 \, dx = \int_\real u_x^3 \, dx
    \end{split}
\]
We do not have mass conservation, since mass is not constant over time. 

We compute the derivative of the momentum
\[
    \begin{split}
        \mathcal{M}'(t) = \frac{d}{dt} \int_\real u(x,t)^2 \, dx = \int_\real 2 u(x,t) u_t(x,t) \, dx = \int_\real 2 u(x,t) 2u_{xx} + 2u u_x^3 \, dx= \\
         = \int_\real 4 u u_{xx} + \underbrace{2u u_x^3}_{= 2 u u_x u_x^2} \, dx =  \cancel{\left. 4 u u_x \right|_\real} - \int_\real 4 u_x^2 \, dx + \cancel{\left. u^2 u_x^2 \right|_\real} - \int_\real u^2 2 u_x u_{xx} \, dx =\\
         = - \int_\real u_x (4 u_x - 2 u^2 u_{xx}) \, dx
    \end{split}
\]
As we can see, the momentum is not conserved either.

\newpage
\begin{exercise}
    Let \(f : \real \to \real\) be the function defined by
    \[
        f(x) = \begin{cases}
            1 - \abs{x} & \text{if } \abs{x} < 1, \\
            0 &  \text{if } \abs{x} \geq 1.
        \end{cases}
    \]
    Prove that \(f \in H^s(\real)\) for all \(s < 3/2\). Hint: use the Fourier transform.
\end{exercise}
We start by recalling the defintion of \(H^s(\real)\).
\begin{remark}
    Let \(s \in \real\). We define the Sobolev space \(H^s(\real)\) as 
    \[
        H^s(\real) = \left\{ f \in L^2(\real) : (1 + \abs{\xi}^2)^{\frac{s}{2}} \hat{f}(\xi) \in L^2(\real) \right\}
    \]
\end{remark}
We start by computing the Fourier transform of \(f\). We have that
\[
    \begin{split}
        \hat{f}(\xi) = \int_\real f(x) e^{-i x \xi} \, dx = \int_{-1}^0 (1 + x) e^{-i x \xi} \, dx + \int_0^1 (1 - x) e^{-i x \xi} \, dx = \\
        = \int_{-1}^0 e^{-i x \xi} \, dx + \int_0^1 e^{-i x \xi} \, dx + \int_{-1}^0 x e^{-i x \xi} \, dx - \int_0^1 x e^{-i x \xi} \, dx = \\
        = \left.  \frac{e^{-i x \xi}}{-i \xi} \right|_{-1}^0 + \left. \frac{e^{-i x \xi}}{-i \xi} \right|_0^1 + \int_{-1}^0 x e^{-i x \xi} \, dx - \int_0^1 x e^{-i x \xi} \, dx = \\
        = \cancel{\frac{1}{- i \xi}} - \frac{e^{i \xi}}{- i \xi} + \frac{e^{-i \xi}}{- i \xi} - \cancel{\frac{1}{- i \xi}} + \left. x \frac{e^{-i x \xi}}{-i \xi} \right|_{-1}^0 - \int_{-1}^0 \frac{e^{-i x \xi}}{-i \xi} \, dx - \left. x \frac{e^{-i x \xi}}{-i \xi} \right|_0^1 + \int_0^1 \frac{e^{-i x \xi}}{-i \xi} \, dx = \\
        = \cancel{- \frac{e^{i \xi}}{- i \xi}} + \cancel{\frac{e^{-i \xi}}{- i \xi}} + \cancel{\frac{e^{i \xi}}{- i \xi}} - \left. \frac{e^{-i x \xi}}{(-i \xi)^2} \right|_{-1}^0 - \cancel{\frac{e^{-i \xi}}{- i \xi}} + \left. \frac{e^{-i x \xi}}{(-i \xi)^2} \right|_0^1 = \\ 
        = -\frac{1}{\xi^2} + \frac{e^{i \xi}}{\xi^2} + \frac{e^{-i \xi}}{\xi^2} - \frac{1}{\xi^2} = \frac{e^{i \xi} + e^{-i \xi} - 2}{\xi^2}  
    \end{split}
\]
Remembering that \(\cos(\xi) = \frac{e^{i \xi} + e^{-i \xi}}{2}\) and \(\sin(\xi) = \frac{e^{i \xi} - e^{-i \xi}}{2i}\), we can rewrite the Fourier transform as
\[
    \begin{split}
        \hat{f}(\xi) = \frac{(e^{i \xi} + e^{-i \xi} - 2)}{\xi^2} = \frac{2 \cos(\xi) - 2}{\xi^2} 
    \end{split}
\]
Now to check that \(f \in H^s(\real)\) for all \(s < 3/2\), we need to check that \((1 + \abs{\xi}^2)^{\frac{s}{2}} \hat{f}(\xi) \in L^2(\real)\). We have that
\[
    \begin{split}
        \left(1 + \abs{\xi}^2\right)^{\frac{s}{2}} \frac{2 \cos(\xi) - 2}{\xi^2}  \in L^2(\real) \iff \int_\real \left(1 + \abs{\xi}^2\right)^{s} \abs{\frac{2 \cos(\xi) - 2}{\xi^2}}^2 \, d\xi < \infty
    \end{split}
\]
We know that \(\left(1 + \abs{\xi}^2\right) \overset{\xi \to \infty}{\longrightarrow} \abs{\xi}^2\), and we can bound \(2 \cos(\xi) - 2\), so we have that
\[
    \begin{split}
        \int_\real \left(1 + \abs{\xi}^2\right)^{s} \abs{\frac{2 \cos(\xi) - 2}{\xi^2}}^2 \, d\xi  \leq
        \int_\real \abs{\frac{\xi^{s}}{\xi^2}}^2 \, d\xi = \int_\real \frac{\xi^{2s}}{\xi^4} \, d\xi  
        = \int_\real \frac{1}{\xi^{4 - 2s}} \, d\xi
    \end{split}
\]
We know that the integral \(\int_\real \frac{1}{\xi^{4 - 2s}} \, d\xi\) converges if \(4 - 2s > 1 \Rightarrow s < 3/2\), so we have that \(f \in H^s(\real)\) for all \(s < 3/2\). 

\newpage
\subsection{July 2022}
\begin{exercise}
    Let \(\Omega \subset \real^n (n \geq 2)\) be a bounded smooth domain, let \(a\) be a measurable function in \(\Omega\).
    Consider the problem
    \[
        \begin{cases}
            - \Delta u = a(x) u^4 & \Omega \\
            u = 0 & \partial\Omega
        \end{cases}
        \tag*{(P)}
    \]
    Under which assumptions on the space dimension n can we write a variational formulation of problem (P) in
    \(H^1_0(\Omega)\)? 
    
    For each of these dimensions find the most general assumptions on \(a\) that allow to write the variational formulation. 
    
    Finally, write the variational formulation.
    \end{exercise}

    Since we want to know the variational formulation in \(H^1_0\) we have \(s = 1\) and need to check \(n = 2, n \geq 3\). 
    
    Remember a variational formulation makes sense if \(\int_\Omega fv < \infty\).
    \begin{itemize}
        \item[\(n = 2\).] In this case we have \(u^4 v \in H^1_0(\Omega)\), so by Sobolev embedding we know \(u^4, v \in L^p(\Omega)\) for \(2 \leq p < \infty\). 
        \[
            \abs{\int_\Omega a(x) u^4 v}  \, dx \leq \int_\Omega \abs{a(x)} \abs{u^4} \abs{v} \, dx \underset{Holder}{\leq} \left(\int_\Omega \abs{a(x)}^r\right)^{\frac{1}{r}} \left(\int_\Omega \abs{u^4}^p \right)^{\frac{1}{p}} \left(\int_\Omega \abs{v}^q \right)^{\frac{1}{q}} < \infty.
        \]
        To use Holder inequality we need to find \(r, p, q\) such that \(\frac{1}{r} + \frac{1}{p} + \frac{1}{q} = 1\). We see that, 
        \[
            \frac{1}{r} + \frac{1}{p} + \frac{1}{q} = 1 \iff a(x) \in L^r(\Omega) \qquad \text{with } r > 1
        \]
        \item[\(n \geq 3\).] In this case we have \(u^4, v \in H^1_0(\Omega)\), so by Sobolev embedding we know \(u^4, v \in L^p(\Omega)\) for \(2 \leq p \leq 2^*\).
        We proceed as before, using Holder inequality, but decide to use \(p = \frac{2^*}{4}\) and \(q = \frac{1}{2^*}.\)
        \[
            \begin{split}
                \abs{\int_\Omega a(x) u^4 v}  \, dx \leq \int_\Omega \abs{a(x)} \abs{u^4} \abs{v} \, dx \underset{{Holder}}{\leq} \\
                \leq \left(\int_\Omega \abs{a(x)}^r\right)^{\frac{1}{r}} \left(\int_\Omega \abs{u}^{2^*} \right)^{\frac{4}{2^*}} \left(\int_\Omega \abs{v}^{2^*} \right)^{\frac{1}{2^*}} < \infty.
            \end{split}
        \]
        In this case Holder inequality gives us 
        \[
            \frac{1}{r} + \frac{4}{2^*} + \frac{1}{2^*} = 1 \iff \frac{1}{r} = 1 - \frac{5}{2^*} \iff r = \frac{2^*}{2^* - 5}
        \]
        Substituting \(2^* = \frac{2n}{n - 2}\) we get \(r = \frac{2n}{-3n + 10}\). Since \(r > 0\) we need \(n < 10/3\), so we have that the variational formulation is well posed if \(n < 3\).
        In this case we have that \(2^* = \frac{2n}{n - 2} = \frac{2 \cdot 3}{3 - 2} = 6\), so we have that \(r = \frac{6}{6 - 5} = 6\), so we need \(a(x) \in L^6(\Omega)\).
    \end{itemize}
    \newpage
    
    At this point we can write the weak formulation of the problem. We multiply the equation by a test function \(v \in H^1_0(\Omega)\) and obtain 
    \[
        \int_\Omega - \Delta u v \, dx = \int_\Omega a(x) u^4 v \, dx \qquad \forall v \in H^1_0(\Omega)
    \]
    We integrate by parts the left-hand side and obtain
    \[
        \int_\Omega \nabla u \nabla v \, dx = \int_\Omega a(x) u^4 v \, dx \qquad \forall v \in H^1_0(\Omega)
    \]
    This is the weak formulation of the problem. This is well posed if 
    \begin{table}[h]
        \centering
        \begin{tabular}{|c|c|}
            \hline
            Dimension & Assumptions on $a(x)$ \\
            \hline
            $n = 2$ & $a \in L^r(\Omega)$, $r > 1$ \\
            $n = 3$ & $a \in L^6(\Omega)$ \\
            $n \geq 4$ & No variational formulation \\
            \hline
        \end{tabular}
    \end{table}

\newpage
\begin{exercise}
    Explain how to proceed in order to find solitary waves for the Korteweg-de Vries equation
    \[
        \begin{cases}
            u_t + u_{xxx} + 6 u u_x = 0 & \real \times (0, \infty) \\
            u(x, 0) = g(x) & x \in \real
        \end{cases}
    \]
    Derive the related couple of first order ODEs, without solving them.
\end{exercise}
We start by recalling the definition of solitary waves for the KdV equation.
\begin{remark}
    In the case of the KdV equation, we have that the solution \(u(x,t) =  g(x - ct)\) where \(c\) is the speed of the wave.
\end{remark}
As always, we substitute this solution in the equation and obtain
\[
    \begin{split}
        -c g'(x - ct) + g'''(x - ct) + 6 g(x - ct) g'(x - ct) = 0
    \end{split}
\]
We perform a change of variable \(s = x - ct\) and obtain
\[
    \begin{split}
        -c g'(s) + g'''(s) + 6 g(s) g'(s) = 0
    \end{split}
\]
We can see that this equation can be rewritten as
\[
    \begin{split}
        \frac{d}{ds}\left[-c g(s) + g''(s) + 3 g(s)^2\right] = 0
    \end{split}
\]
By integrating this equation we obtain
\[
    \begin{split}
        -c g(s) + g''(s) + 3 g(s)^2 = \frac{a}{2} \qquad \text{with } a \in \real
    \end{split}
\]
where we choose the constant of integration equal to \(a/2\) to simplify the calculations. Now we multiply this equation by \(g'(s)\) and obtain
\[
    \begin{split}
        -c g(s) g'(s) + g''(s) g'(s) + 3 g(s)^2 g'(s) = \frac{a}{2} g'(s)
    \end{split}
\]
Again this can be rewritten as
\[
    \begin{split}
        \frac{d}{ds}\left[-\frac{c}{2} g(s)^2 + \frac{1}{2} g'(s)^2 + g(s)^3 - \frac{a}{2} g(s)\right] = 0
    \end{split}
\]
A second integration gives us
\[
    \begin{split}
        -\frac{c}{2} g(s)^2 + \frac{1}{2} g'(s)^2 + g(s)^3 - \frac{a}{2} g(s) = \frac{b}{2} \qquad \text{with } b \in \real
    \end{split}
\]
In the end the equation looks like
\[
    \begin{split}
        g'(s)^2 = -2 g(s)^3 + c g(s)^2 + a g(s) + b
    \end{split}
\]
We can see that 
\[
    g'(s)^2 = P_3(g(s)) 
\]
where \(P_3\) is a polynomial of degree 3, with the coefficients depending on \(c, a, b\), where \(c\) is the speed of the wave, and \(a, b\) are constants of integration.

Then by taking the square root of this equation we obtain the couple of first order ODEs
\[
    \begin{cases}
        g'(s) = \sqrt{-2 g(s)^3 + c g(s)^2 + a g(s) + b} \\
        g'(s) = -\sqrt{-2 g(s)^3 + c g(s)^2 + a g(s) + b}
    \end{cases}
\]

\newpage
\begin{exercise}
        Let \(B = \left\{ x \in \real^n \mid \abs{x} < 1 \right\}\). For which values of \(p \in [1, \infty) \) is the function
        \[
            f(x) = \frac{\sin\left(\abs{x}\right)}{\abs{x}^3}
        \]
        in \(W^{1, p}(B)\)?
\end{exercise}
The strategy for this exercise remain the same as before.
\begin{remark}
    A function belong to \(W^{1, p}(B)\) if its weak derivative exists and belongs to \(L^p(B)\).
\end{remark}
We start by checking if \(f \in L^p(B)\)
\[
    \begin{split}
        \int_B \abs{f}^p \, dx = \int_B \abs{\frac{\sin\left(\abs{x}\right)}{\abs{x}^3}}^p \, dx = \int_B \frac{\abs{\sin\left(\abs{x}\right)}^p}{\abs{x}^{3p}} \, dx = \int_0^1 \int_{\left\{\abs{x} = \rho\right\}} \frac{\abs{\sin\left(\rho\right)}^p}{\rho^{3p}} \, d\sigma \, d\rho = \\
        = \sigma_n \int_0^1 \frac{\abs{\sin\left(\rho\right)}^p}{\rho^{3p}} \rho^{n-1} \, d\rho
    \end{split}
\]
Since \(\sin\left(\rho\right) \overset{\rho \to 0}{\longrightarrow} 1\) we have that the integral is finite if
\[
    \begin{split}
        \int_0^1 \abs{\frac{1}{\rho^{3p}} \rho^{n-1}} \, d\rho < \infty \iff 3p - n + 1 < 1 \iff 1 \leq p < \frac{n}{3} \Rightarrow n \geq 4
    \end{split}
\]
We have that \(f \in L^p(B)\) if \(p \in [1, \frac{n}{2})\). We now need to check if \(f \in W^{1, p}(B)\). We start by computing the gradient of \(f\)
\[
    \begin{split}
        \partial_{x_i} f = \frac{\cos\left(\abs{x}\right) \abs{x}^3 - 3 \sin\left(\abs{x}\right) \abs{x}^2}{\abs{x}^6} \frac{x_i}{\abs{x}} = \frac{\cos\left(\abs{x}\right) \abs{x} - 3 \sin\left(\abs{x}\right)}{\abs{x}^4} \frac{x_i}{\abs{x}}
    \end{split}
\]
with 
\[
    \begin{split}
        \norm{\grad f} = \frac{\cos\left(\abs{x}\right) \abs{x} - 3 \sin\left(\abs{x}\right)}{\abs{x}^4} 
    \end{split}
\]
Now we need to check if \(\grad f \in L^p(B)\)
\[
    \begin{split}
        \int_B \abs{\grad f}^p \, dx = \int_B \abs{\frac{\cos\left(\abs{x}\right) \abs{x} - 3 \sin\left(\abs{x}\right)}{\abs{x}^4}}^p \, dx = \int_B  \frac{\left[\cos\left(\abs{x}\right) \abs{x} - 3 \sin\left(\abs{x}\right)\right]^p} {\abs{x}^{4p}} \, dx = \\
        \int_0^1 \int_{\left\{\abs{x} = \rho\right\}}  \frac{\left[\cos\left(\abs{\rho}\right) \abs{\rho} - 3 \sin\left(\abs{\rho}\right)\right]^p} {\rho^{4p}} \, d\sigma \, d\rho = \sigma_n \int_0^1  \frac{\left[\cos\left(\abs{\rho}\right) \abs{\rho} - 3 \sin\left(\abs{\rho}\right)\right]^p} {\rho^{4p}} \rho^{n-1} \, d\rho
    \end{split}
\]
Also in this case we have \(\cos\left(\abs{\rho}\right) \abs{\rho} - 3 \sin\left(\abs{\rho}\right) \overset{\rho \to 0}{\longrightarrow} -1\), so we need that
\[
    \begin{split}
        \int_0^1 \abs{\frac{1} {\rho^{4p}} \rho^{n-1}} \, d\rho < \infty \iff 4p - n + 1 < 1 \iff 1 \leq p < \frac{n}{4} \Rightarrow n \geq 5
    \end{split}
\]
Now that we have checked that \(\grad f \in L^2(\Omega)\) and \(f \in L^2(\Omega)\) we only need to check that the weak derivative of \(f\) exists, since if it exists it is equal to the classical one. The weak derivative of \(f\) exists if
\[
    \begin{split}
        \int_\Omega f \partial x_i \phi \, dx = - \int_\Omega \partial x_i f \phi \, dx \qquad \forall \phi \in \mathcal{D}(\Omega)
    \end{split}
    \tag*{E1}
\]
To check that this condition is satisfied we need to cut off the singularity of \(f\) in the origin. We can do this by defining
\[
    \Omega_\epsilon = B(0, 1) \setminus B(0, \epsilon) = B_1 \setminus B_\epsilon = \left\{ x \in \real^n \mid \epsilon < \norm{x} < 1 \right\}
\]
Since \(f \in C^1(\Omega_\epsilon)\) we can apply the divergence theorem to the weak derivative definition and obtain
\[
    \begin{split}
        \int_{\Omega_\epsilon} f \partial x_i \phi \, dx = \int_{\Omega_\epsilon} \partial x_i f \phi \, dx - \int_{\partial\Omega_\epsilon} f \phi \nu_i \, d\sigma
    \end{split}
    \tag*{E2}
\]
We want check that taking the limit \(\epsilon \to 0\) in (E2) we obtain (E1). To do so we need to check that the boundary term goes to zero, and the other two terms are equal at the ones in (E1). We start with the first term
\[
    \begin{split}
        \int_{\Omega_\epsilon} f \partial x_i \phi \, dx = \int_{B_1} f \partial x_i \phi \chi_{\Omega_\epsilon} \, dx
    \end{split}
\]
We want to claim that 
\[
    \begin{split}
        \lim_{\epsilon \to 0^+} \int_{\Omega_\epsilon} f \partial x_i \phi \, dx = \int_{B_1} f \partial x_i \phi \, dx
    \end{split}
\]
To do so we need to swap the limit and the integral. We see that 
\begin{itemize}
    \item \(f \partial x_i \phi \chi_{\Omega_\epsilon} \underset{\epsilon \to 0}{\longrightarrow} f \partial x_i \phi\) a.e. in \(B_1\)
    \item \(\abs{f \partial x_i \phi \chi_{\Omega_\epsilon}} \leq \underbrace{\abs{f}}_{\in L^p(B_1)} \overbrace{\abs{\partial x_i \phi}}^{\in L^q(B_1)} \in L^1(B_1)\)
\end{itemize}
We can now apply the Dominated Convergence Theorem and obtain the desired result. 
The same process can be applied to
\[
    \begin{split}
        \int_{\Omega_\epsilon} \partial x_i f \phi \, dx = \int_{B_1} \partial x_i f \phi \chi_{\Omega_\epsilon} \, dx
    \end{split}
\]
Then we have shown that 
\[
    \begin{split}
        \int_{B_1} f \partial x_i \phi \, dx = \int_{\partial\Omega_\epsilon} f \phi \nu_i \, d\sigma + \int_{B_1} \partial x_i f \phi \, dx
    \end{split}
\]
It is clear that we need to check that the boundary term goes to zero. Since we know that \(\text{supp } \phi \subseteq B_\epsilon\).
\[
    \begin{split}
        \int_{\partial\Omega_\epsilon} f \phi \nu_i \, d\sigma = \cancel{\int_{\partial B_1} f \phi \nu_i \, d\sigma} + \int_{\partial B_\epsilon} f \phi \nu_i \, d\sigma
    \end{split}
\]
Moreover,
\[
    \begin{split}
        \abs{\int_{\partial B_\epsilon} f \phi \nu_i \, d\sigma} \leq \int_{\partial B_\epsilon} \abs{f} \abs{\phi} \underbrace{\abs{\nu_i}}_{=1} \, d\sigma \leq \max_{\partial B_\epsilon} \abs{\phi} \int_{\{ \norm{x} = \epsilon \}} \abs{f} \, d\sigma = \\
        = \max_{\partial B_\epsilon} \abs{\phi} \int_{\{ \norm{x} = \epsilon \}} \abs{\frac{\sin\left(\abs{x}\right)}{\abs{x}^3}} \, d\sigma = \max_{\partial B_\epsilon} \abs{\phi}  \abs{\frac{\sin\left(\epsilon\right)}{\epsilon^3}} \mu\left\{\norm{x} = \epsilon\right\} \leq \max_{\partial B_\epsilon} \abs{\phi} \frac{1}{\epsilon^3} \epsilon^{n-1} = \\ 
    \end{split}
\]
We have that 
\[
    \begin{split}
        \lim_{\epsilon \to 0^+} \max_{\partial B_\epsilon} \abs{\phi} \frac{1}{\epsilon^3} \epsilon^{n-1} = \lim_{\epsilon \to 0^+} \max_{\partial B_\epsilon} \abs{\phi} \epsilon^{n-4} = 0 \text{ since } n \geq 5
    \end{split}
\]

So we have shown that \(f \in W^{1, p}(B) \iff n > 4 \land 1 \leq p < \frac{n}{4}\).

\newpage
\subsection{September 2022}
\begin{exercise}
    Let \(\Omega \subset \real^2\) be a bounded open set of class \(C^1\), \(u_0 \in L^2(\Omega)\). Moreover, let \(T > 0\) be a fixed time and let \(f \in L^2(0, T; L^2(\Omega))\). Prove that there exists a unique weak solution \(u\) for the problem
    \[
        \begin{cases}
            u_t - \left( 3\partial_{x}^2 u + 2\partial_{y}^2 u - 4 \partial_{xy} u \right) = f & \Omega \times (0, T) \\
            u = 0 & \partial\Omega \times (0, T) \\
            u(x, 0) = u_0(x) & \Omega
        \end{cases}
    \]
\end{exercise}
We start by finding an adequate matrix \(A\) such that the equation can be written as
\[
    u_t - \div(A \grad u) = f
\]
We choose the matrix
\[
    A = \begin{pmatrix}
        3 & -2 \\
        -2 & 2
    \end{pmatrix}
\]
Now we are dealing with the problem
\[
    \begin{cases}
        u_t - \div(A \grad u) = f & \Omega \times (0, T) \\
        u = 0 & \partial\Omega \times (0, T) \\
        u(x, 0) = u_0(x) & \Omega
    \end{cases}
    \tag*{(P)}
\]
To obtain its weak formulation we multiply the equation by a test function \(\phi \in \mathcal{D}(\Omega)\) 
\begin{align*}
    \int_\Omega u_t \phi - \div(A \grad u) \phi \, dx &= \int_\Omega f \phi \, dx \qquad \forall \phi \in \mathcal{D}(\Omega) \\
    \Updownarrow &\text{ using the divergence theorem} \\
    \frac{d}{dt} \underbrace{\int_\Omega u \phi \, dx}_{(u, \phi)_{L^2}} + \underbrace{\int_\Omega A \grad u \grad \phi \, dx}_{a(u,\phi)} &= \underbrace{\int_\Omega f \phi \, dx}_{(f, \phi)_{L^2}} \qquad \forall \phi \in \mathcal{D}(\Omega) \\
\end{align*}
Taking into account that \(u = 0\) on the boundary, we choose an adequate triplet of Hilbert spaces
\[
    V = H^1_0(\Omega) \subseteq H = L^2(\Omega) \subseteq V' = H^{-1}(\Omega)
\]
We can now write the weak formulation of the problem
\[
    \begin{split}
        \text{Find } u \in L^2(0, T; V) \cap C^0([0, T]; H) \text{ such that } u(0) = u_0 \text{ and }\\
        \frac{d}{dt} (u, v)_{L^2} + a(u, v) = (f, v)_{H} \qquad \forall v \in V
    \end{split}
\]
For the existence and uniqueness of the solution we need the following:
\begin{itemize}
    \item \(a(u, v)\) is continuous and coercive
    \item \(f \in L^2(0, T; H)\)
    \item \(u_0 \in H\)
\end{itemize}
We see that the third condition is satisfied, since \(u_0 \in L^2(\Omega)\), and also the second condition is satisfied because \(f \in L^2(0, T; L^2(\Omega)) \subseteq L^2(0, T; H)\). We need to check the first condition. We start by checking for continuity
\[
    \begin{split}
        \abs{a(u, v)} = \abs{\int_\Omega A \grad u \grad v \, dx} \leq \abs{A} \norm{\grad u}_{L^2} \norm{\grad v}_{L^2} \leq \abs{A} \norm{u}_{V} \norm{v}_{V} 
    \end{split}
\]
We can see that \(a(u, v)\) is continuous. We now need to check for coercivity. We have that
\[
    \begin{split}
        a(u, u) = \int_\Omega A \grad u \grad u \, = \abs{A} \norm{\grad u}_{L^2}^2 \geq \frac{\abs{A}}{1 + C_p^2} \norm{u}_{V}^2
    \end{split}
\]
where \(C_p\) is the Poincaré constant.

Since the bilinear form \(a(u, v)\) is continuous and coercive, all the requirements are met and, by abstract results we can conclude that
\[
    \begin{split}
        \exists! u \in L^2(0, T; V) \cap C^0([0, T]; H) \text{ such that is a weak solution of (P)}
    \end{split}
\]

\newpage
\begin{exercise}
    Let \(\Omega \in \real^n (n \geq 2)\) be a bounded open domain with \(\partial\Omega \in C^\infty\). Consider the problem
    \[
        \begin{cases}
            -\Delta u = f & \Omega \\
            u = g & \partial\Omega
        \end{cases}
        \tag*{(P)}
    \]
    \begin{enumerate}
        \item Assuming that \(f \in H^{-1}(\Omega)\) and \(g \in H^{\frac{1}{2}}(\partial\Omega)\), write the weak formulation of problem (P).
        \item Prove that this problem admits a unique solution and characterize its regularity.
        \item What are the minimal regularity assumptions on \(f\) and \(g\) that guarantee \(u \in H^2(\Omega)\)?
    \end{enumerate}
\end{exercise}
\begin{enumerate}
    \item We start by writing the weak formulation of the problem. First we define a suitable function space, since \(f \in H^{-1}(\Omega, g \in H^{1/2}(\Omega))\) and \(\alpha = 0 > -\lambda_1\) \(\exists u_0 \in H^1(\Omega)\) such that \(\gamma_0(u_0) = g\), where \(\gamma_0\) is the trace operator. We define the function space
    \[
        K = \left\{ u \in H^1(\Omega) \mid u - u_0 \in H^1_0(\Omega) \right\}
    \]
    Then we multiply the equation by a test function \(v \in H^1_0(\Omega)\) and obtain
    \[
        \begin{split}
            -\int_\Omega \Delta u v \, dx = \int_\Omega f v \, dx \qquad \forall v \in H^1_0(\Omega)
        \end{split}
    \]
    We integrate by parts the left-hand side and obtain
    \[
        \begin{split}
            \int_\Omega \grad u \grad v \, dx = \int_\Omega f v \, dx \qquad \forall v \in H^1_0(\Omega)
        \end{split}
    \]
    Then we obtain the weak formulation of the problem
    \[
        \begin{split}
            \text{Find } u \in K \text{ such that } \int_\Omega \grad u \grad v \, dx =  \langle f, v \rangle \qquad \forall v \in H^1_0(\Omega)
        \end{split}
    \]
    \item We now need to prove that the problem admits a unique solution and characterize its regularity. The Dirichlet principle states that 
    \begin{remark}
        Let \(\alpha > -\lambda_1\) and \(f \in H^{-1}(\Omega)\), \(g \in H^{\frac{1}{2}}(\partial\Omega)\). Then the problem (P) admits a unique solution \(u \in K\), Moreover \(u\) is a weak solution if and only if minimizes the functional
        \[
            J(u) = \frac{1}{2} \int_\Omega \abs{\grad u}^2 \, dx - \langle f, u \rangle
        \]
    \end{remark}
    We provide a sketch of the proof of the uniqueness of the solution. We start by letting \(u_0 \in H^1(\Omega)\) be such that \(\gamma_0(u_0) = g \in H^{1/2}(\partial\Omega)\). Now let \(u = z + u_0 \iff z = u - u_0\). We have that \(z \in H^1_0(\Omega)\). 
    \[
        \begin{split}
            u \in K \text{ is weak solution of (P)} \iff \int_\Omega \grad u \grad v \, dx = \langle f, v \rangle \qquad \forall v \in H^1_0(\Omega) \\
            \iff \int_\Omega \grad z \grad v \, dx = \langle f, v \rangle - \int_\Omega \grad u_0 \grad v \, dx \qquad \forall v \in H^1_0(\Omega) 
        \end{split}
    \]
    Now we define the functional
    \begin{align*}
        \Lambda: H^1_0(\Omega) &\longrightarrow \real \\
        v &\longmapsto \langle f, v \rangle - \int_\Omega \grad u_0 \grad v \, dx
    \end{align*}
    We have that \(\Lambda\) is a linear and continuous functional. 

    Then we take a look at \(a(z, v) = \int_\Omega \grad z \grad v \, dx\). We have that \(a(z, v)\) is a continuous and coercive bilinear form since \(\alpha = 0 > -\lambda_1\). Then, by the Lax-Milgram theorem, we have that there exists a unique solution \(z \in H^1_0(\Omega)\) such that
    \[
        \begin{split}
            a(z, v) = \Lambda(v) \qquad \forall v \in H^1_0(\Omega)
        \end{split}
    \]
    Since \(z\) is the unique solution of the problem we have that \(u = z + u_0\) is the unique solution of the problem (P).
\end{enumerate}

\newpage
\begin{exercise}
    Write the weak formulation of the stationary Navier-Stokes equations under Dirichlet boundary conditions in a smooth bounded domain \(\Omega \subset \real^n\) and explain why the assumption \(n \leq 4\) is necessary.
\end{exercise}
We start by writing the stationary Navier-Stokes equations
\[
    \begin{cases}
        - \eta \Delta u + \left( u \cdot \grad \right) u + \grad p = f & \Omega \\
        \div u = 0 & \Omega \\
        u = 0 & \partial\Omega
    \end{cases}
\]
where \(\Omega \subset \real^n\), \(\partial\Omega \in C^1\), and \(n \leq 4\).
\begin{remark}
    About the term \(\left( u \cdot \grad \right) u\), we have that (\(n = 3\))
    \[
        u = \begin{pmatrix}
            u_1 \\
            u_2 \\
            u_3
        \end{pmatrix} \quad \text{and} \quad \grad = \begin{pmatrix}
            \partial_1 \\
            \partial_2 \\
            \partial_3
        \end{pmatrix}
    \]
    So combining the two we have
    \[
        \begin{split}
            \left( u \cdot \grad \right) u = \begin{pmatrix}
                u_1 \partial_1 + u_2 \partial_2 + u_3 \partial_3
            \end{pmatrix} \begin{pmatrix}
                u_1 \\
                u_2 \\
                u_3
            \end{pmatrix} = \begin{pmatrix}
                u_1 \partial_1 u_1 + u_2 \partial_2 u_1 + u_3 \partial_3 u_1 \\
                u_1 \partial_1 u_2 + u_2 \partial_2
                u_2 + u_3 \partial_3 u_2 \\
                u_1 \partial_1 u_3 + u_2 \partial_2 u_3 + u_3 \partial_3 u_3
            \end{pmatrix}
        \end{split}
    \]
    which is the so-called convective term.
\end{remark}
As functional spaces we choose the spaces introduced in the previous exercise about the Stokes problem \(\bm{V}, \bm{G_1}, \bm{G_2}, \bm{G_3}\). 
\begin{remark}
    \(\bm{V} \coloneqq \left\{ f \in \bm{H}^1_0(\Omega) \mid \grad \cdot f = 0 \right\}\) is the space of divergence-free functions.
    We also introduce three spaces:
    \begin{itemize}
        \item \(\bm{G}_1 \coloneqq \left\{ f \in \bm{L}^2(\Omega) \mid \grad \cdot f = 0, \gamma_\nu f = 0 \right\}\)
        \item \(\bm{G}_2 \coloneqq \left\{ f \in \bm{L}^2(\Omega) \mid \grad \cdot f = 0, \exists g \in H^1(\Omega) \text{ s.t. } f = \grad g \right\}\)
        \item \(\bm{G}_3 \coloneqq \left\{ f \in \bm{L}^2(\Omega) \mid \exists g \in H^1_0(\Omega) \text{ s.t. } f = \grad g \right\}\)
    \end{itemize}
\end{remark}
We now multiply the equation by a test function \(v \in \bm{V}\) and obtain
\[
    \begin{split}
        - \eta \int_\Omega \Delta u v \, dx + \int_\Omega \left( u \cdot \grad \right) u v \, dx + \int_\Omega \grad p \cdot v \, dx = \int_\Omega f v \, dx
    \end{split}
\]
We integrate by parts the first term and obtain
\[
    \begin{split}
        \eta \int_\Omega \grad u : \grad v \, dx + \int_\Omega \left( u \cdot \grad \right) u \cdot v \, dx + \underbrace{\int_\Omega \grad p \cdot v \, dx}_{=0} = \int_\Omega f v \, dx
    \end{split}
\]
The term \(\int_\Omega \grad p v \, dx\) is zero because \(v \in \bm{V} \subseteq \bm{G_1}\) and \(\grad p \in \bm{G_2} \oplus \bm{G_3} = \bm{G_1}^\perp\). We can now write the weak formulation of the problem
\[
    \begin{split}
        \text{Find } u \in \bm{V} \text{ such that } \eta \int_\Omega \grad u : \grad v \, dx + \int_\Omega \left( u \cdot \grad \right) u \cdot v \, dx = \langle f, v \rangle \qquad \forall v \in \bm{V}
    \end{split}
\]
Now we need to explain why the assumption \(n \leq 4\) is necessary. We need to understand in which space the convective term lies and that \(v \in \bm{V} \Rightarrow \grad v \in \bm{L}^2(\Omega)\).  
\begin{itemize}
    \item[\(n = 2\)] In the bidimensional case, we have that \(u \in \bm{V} \Rightarrow u \in \bm{L}^p(\Omega) \, \forall 1 \leq p < \infty\). Since \(u \cdot \grad u\) is a product of a function in \(\bm{L}^p(\Omega)\) and a function in \(\bm{L}^2(\Omega)\), we have that \(u \cdot \grad u \in \bm{L}^2(\Omega) \, \forall q < 2\).
    \[
        \begin{pmatrix*}
            \bm{H}^1_0(\Omega) \subset \bm{L}^p{(\Omega)} & \forall 1 \leq p < \infty \\
            \bm{L}^{p'}(\Omega) \subset \bm{H}^{-1}(\Omega) & \forall 1 < p \leq \infty
        \end{pmatrix*}
        \Rightarrow u \cdot \grad u \in \bm{H}^{-1}(\Omega)
    \]
    \item[\(n = 3\)] In the tridimensional case, Sobolev embedding gives us \(u \in \bm{V} \Rightarrow u \in \bm{L}^6(\Omega)\). In this case we have that \(u \cdot \grad u \in \bm{L}^{3/2}(\Omega)\), because we have that \(u \in \bm{L}^6(\Omega)\) and \(\grad u \in \bm{L}^2(\Omega)\), and by Holder's inequality \(\frac{1}{6} + \frac{1}{2} + \frac{1}{r} = 1 \Rightarrow r = 3\) and the dual of \(\bm{L}^{3}(\Omega)\) is \(\bm{L}^{3/2}(\Omega)\).
    \[
        \begin{pmatrix*}
            \bm{H}^1_0(\Omega) \subset \bm{L}^6{(\Omega)}  \\
            \bm{L}^{6/5}(\Omega) \subset \bm{H}^{-1}(\Omega)
        \end{pmatrix*}
        \Rightarrow u \cdot \grad u \in \bm{L}^{3/2}(\Omega) \subset \bm{L}^{6/5}(\Omega) \subset \bm{H}^{-1}(\Omega)
    \]
    \item[\(n = 4\)] In the four-dimensional case, we have that \(u \in \bm{V} \Rightarrow u \in \bm{L}^4(\Omega)\). In this case we have that \(u \cdot \grad u \in \bm{L}^{4/3}(\Omega)\), because we have that \(u \in \bm{L}^4(\Omega)\) and \(\grad u \in \bm{L}^2(\Omega)\), and by Holder's inequality \(\frac{1}{4} + \frac{1}{2} + \frac{1}{r} = 1 \Rightarrow r = 4\) and the dual of \(\bm{L}^{4}(\Omega)\) is \(\bm{L}^{4/3}(\Omega)\).
    \[
        \begin{pmatrix*}
            \bm{H}^1_0(\Omega) \subset \bm{L}^4{(\Omega)}  \\
            \bm{L}^{4/3}(\Omega) \subset \bm{H}^{-1}(\Omega)
        \end{pmatrix*}
        \Rightarrow u \cdot \grad u \in \bm{H}^{-1}(\Omega)
    \]
    \item[\(n = 5\)] In the five-dimensional case, we have that \(u \in \bm{V} \Rightarrow u \in \bm{L}^{10/3}(\Omega)\). In this case we have that \(u \cdot \grad u \in \bm{L}^{5/4}(\Omega)\), because we have that \(u \in \bm{L}^{10/3}(\Omega)\) and \(\grad u \in \bm{L}^2(\Omega)\), and by Holder's inequality \(\frac{3}{10} + \frac{1}{2} + \frac{1}{r} = 1 \Rightarrow r = 5\) and the dual of \(\bm{L}^{5}(\Omega)\) is \(\bm{L}^{5/4}(\Omega)\).
    \[
        \begin{pmatrix*}
            \bm{H}^1_0(\Omega) \subset \bm{L}^{10/3}{(\Omega)}  \\
            \bm{L}^{10/7}(\Omega) \subset \bm{H}^{-1}(\Omega)
        \end{pmatrix*}
        \nRightarrow u \cdot \grad u \in \bm{L}^{10/7}(\Omega) \subset \bm{H}^{-1}(\Omega) \text{ because } \frac{5}{4} < \frac{10}{7}
    \]
    Because \(L^q \subset L^p\) if \(p < q\), in \(n = 5\) we do not have that \(u \cdot \grad u \in \bm{H}^{-1}(\Omega)\).
\end{itemize}

\newpage
\subsection{January 2023}

\section{Exams 2023/24}
\subsection{June 2023}
\begin{exercise}
    Write the strong formulation of the initial-value problem under Dirichlet boundary conditions for the evolution Navier-Stokes equations in a smooth bounded domain $\Omega \subset \mathbb{R}^d$ with $d=2,3$. Then explain why the argument used for the proof of uniqueness of solution for \(n = 2\) cannot be extended to \(n = 3\).
\end{exercise}
First we write the strong formulation of evolutional Navier-Stokes equations
\begin{equation*}
    \begin{cases}
        \partial_t u - \nu \Delta u + (u \cdot \nabla)u + \nabla p = f, & \text{in } \Omega \times (0,T), \\
        \nabla \cdot u = 0, & \text{in } \Omega \times (0,T), \\
        u = 0, & \text{on } \partial \Omega \times (0,T), \\
        u(\cdot,0) = u_0, & \text{in } \Omega,
    \end{cases}
\end{equation*}

\begin{remark}[Weak formulation]
    We choose \(\bm{V}, \bm{H} = \bm{G_1}, \bm{V'}\). By choosing a test function \(v \in \bm{V}\), doing the usual integration by parts, and using the boundary condition, we obtain the weak formulation
    \begin{equation*}
        \text{Find } u \in \bm{L}^2(0,T;\bm{V}) \text{ s.t. }
        \begin{cases}
            \begin{aligned}
                \frac{d}{dt}(u(t), v)_H + \nu \int_\Omega \nabla u(t) : \nabla v \, dx +\\
                + \int_\Omega \left[ (u(t) \cdot \grad) u(t)\right] \cdot v \, dx = \langle f, v \rangle,
            \end{aligned} & \forall v \in \bm{V} \qquad \text{in } \mathcal{D}(0,T) \\
            u(0) = u_0.
        \end{cases}
    \end{equation*}
    This is the weakest possible formulation of the Navier-Stokes equations, we have no information on \(u\). We start with the least possible assumptions and we gain more informations step by step.
\end{remark}
For \(n=2\) we have global uniqueness for the solution, thanks to Ladyzhenskaya's inequality. 
\begin{itemize}
    \item[\(n=2\)] We have the following inequality
    \[
        \norm{v}^2_{\bm{L}^4} \sqrt{2} \norm{v}_{\bm{L}^2} \norm{\grad v}_{\bm{L}^2} \qquad \forall v \in \bm{H}^1_0(\Omega).
    \]
    This is an interpolation inequality: \(\bm{L}^4\) has a better regularity than \(\bm{L}^2\), but it is worse than \(\bm{H}^1\), so we can bound the \(\bm{L}^4\) norm with the \(\bm{H}^1\) and \(\bm{L}^2\) norms. 
    \item[\(n=3\)] In this case Ladyzhenskaya's inequality states
    \[
        \norm{v}^2_{\bm{L}^4} \leq 2 \norm{v}_{\bm{L}^2}^{1/2} \norm{\grad v}_{\bm{L}^2}^{3/2} \qquad \forall v \in \bm{H}^1_0(\Omega).
    \]
    Due to Sobolev embeddings, increasing dimensions reduces the regularity of the solution. This inequality is not enough to prove uniqueness of the solution.
\end{itemize}
Since we know that if a solution satisfies \(u \in L^s(0,T;\bm{L}^r) \text{ with } \frac{2}{s}+\frac{n}{r} \leq 1\), then it is the unique solution. We can immediately see that for \(n=2\) we have \(s=4, r=4\), but increasing \(n\) breaks the inequality. This results means the we always have the existence of a solution, but only if we guarantee \(u \in L^s(0,T;\bm{L}^r)\) with the correct regularity we can have uniqueness.

\newpage
\begin{exercise}
    Consider the conservation law with two different initial conditions
    \begin{equation*}
        a)\,\begin{cases}
            u_t - \log(u) u_x = 0, & x \in \real, t > 0, \\
            u(x,0) = e^{-x}, & x \in \real,
        \end{cases}
        \qquad 
        b)\,\begin{cases}
            u_t - \log(u) u_x = 0, & x \in \real, t > 0, \\
            u(x,0) = \begin{cases}
                1, & x < 0, \\
                e, & x \geq 0.
            \end{cases}
        \end{cases}
    \end{equation*}
    Discuss existence and uniqueness of classical and weak solutions for Cauchy problems. Then estabilish whether the found solutions satisfy the entropy condition.
\end{exercise}
\begin{itemize}
    \item[\textbf{a)}] We look for solutions \(u(, t) \in C^1(\real \times (0, \infty))\) that satisfy the conservation law and the initial condition. 
    We start with existence and uniqueness of solutions, by rewriting the conservation law as
    \[
        u_t + [F(u)] u_x = 0, \quad F(u) = -\log(u).
    \]
    Its characteristics are given by
    \[
        \frac{du}{dt} = \frac{dt}{ds} \frac{du}{dt} + \frac{dx}{ds} \frac{du}{dx} = 0
    \]
    means that \(u\) is constant along the characteristics and then 
    \[
        u(x,t) = u(x_0,0) = e^{-x_0}.
    \]
    with \(t=s\) since the initial condition is given at \(t=0\). 
    \[
        \frac{dx}{ds} = -\log(u_0) \quad \Rightarrow \quad x(s) = x_0 - s \log(u_0).
    \]
    Since \(\log(u_0) = \log(e^{-x_0}) = -x_0\), we have
    \[
        x(s) = x_0 + s x_0 = x_0(1+s) \quad \Rightarrow \quad x_0 = \frac{x}{1+s}.
    \]
    The solution is then
    \[
        u(x,t) = e^{-x/(1+t)}.
    \]
    This is a classical solution and, thanks to the method of characteristics, we have uniqueness. A classical solution is also a weak solution. Moreover, the solution satisfies the entropy condition, since it is a global solution.
    \item[\textbf{b)}] We have two different initial conditions, so we have to solve the conservation law for both cases. 
    \begin{enumerate}
        \item For \(x < 0\) we have
        \[
            u(x,0) = 1
        \]
        By the method of characteristics we have again 
        \[
            \frac{dx}{ds} = -\log(u_0) \quad \Rightarrow \quad x(s) = -\log(1)s = x_0.
        \]
        which is a constant. Therefore the characteristics are vertical lines and the solution is constant 
        \[
            u(x,t) = 1 \quad \forall x < 0.
        \]
        \item For \(x > 0\) we have
        \[
            u(x,0) = e
        \]
        By the method of characteristics we have
        \[
            \frac{dx}{ds} = -\log(u_0) \quad \Rightarrow \quad x(s) = -\log(e)s = -s + x_0.
        \]
        therefore the characteristics are given by
        \[
            u(x,t) = e \quad \forall x > 0.
        \]
        The solution is then
        \[
            u(x,t) = \begin{cases}
                1, & x < 0, \\
                e, & x > 0.
            \end{cases}
        \]
        This cannot be a classical solution, since it is not differentiable at \(x=0\). However, it is a weak solution. 
        Now we have to check the entropy condition, since there is a discontinuity. We first verify the Rankine-Hugoniot condition
        \[
            s = \frac{f(u_r) - f(u_l)}{u_r - u_l} = \frac{\log(e) - \log(1)}{e - 1} = -1.frac{e}{e-1}.
        \]
        To satisfy the entropy condition we need to have that the characteristics on both sides of the shock point towards the shock. To ensure this the condition \(s\) should be less than the speed of the characteristic on the right side and greater than the speed of the characteristic on the left side. 
        \[
            \begin{aligned}
                \text{Left side:} \quad & u_l = 1, \quad f(u_l) = -\log(1) = 0 \quad \Rightarrow \quad -\frac{1}{e-1} > 0 \qquad \text{True} \\
                \text{Right side:} \quad & u_r = e, \quad f(u_r) = -\log(e) = -1 \quad \Rightarrow \quad -\frac{1}{e-1} < -1 \qquad \text{False}
            \end{aligned}
        \]
        The entropy condition is not satisfied.
    \end{enumerate}
\end{itemize}

\newpage
\begin{exercise}
    Let \(\Omega \subset \real^3\) be a bounded Lipschitz domain. Prove that the problem
    \begin{equation*}
        \begin{cases}
            \grad \cdot u = 0, & \text{in } \Omega, \\
            u  = 0, & \text{on } \partial \Omega,
        \end{cases}
    \end{equation*}
    has infinitely many solutions in \(C^\infty(\Omega) \cap C^0(\overline{\Omega})\).
\end{exercise}
We can prove this with the help of Helmoltz decomposition. We can write any vector field \(u\) as the sum of a solenoidal and an irrotational part
\[
    u = \grad \phi + \grad \times \psi.
\]
where \(\phi\) is the scalar potential and \(\psi\) is the vector potential. Our constraint is that the divergence of \(u\) is zero, so we have
\[
    \grad \cdot u = \grad \cdot \grad \phi + \grad \cdot \grad \times \psi = \Delta \phi = 0 \quad \Rightarrow \quad \phi = \text{const.}
\]
This means that the scalar potential is constant and we can rewrite the decomposition
\[
    u = \grad \phi + \grad \times \psi = \grad \times \psi.
\]
Now, choosing \(\psi\) such that satisfies the boundary condition, we can have infinitely many solutions, since 
\[
    \grad \cdot u = \grad \cdot \grad \times \psi = 0 \quad \forall \psi \in C^\infty(\Omega) \cap C^0(\overline{\Omega}).
\]

\newpage


\end{document}