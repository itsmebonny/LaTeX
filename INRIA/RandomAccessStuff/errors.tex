\documentclass[10pt]{beamer}
\usetheme{Montpellier}
\beamertemplatenavigationsymbolsempty


\usepackage{amsmath}
\usepackage{amssymb}
\usepackage{graphicx}
\graphicspath{{/home/andrea/Documents/Internship/Codes/Beam/Sofa_Playground/images}}

\begin{document}
\begin{frame}
    \frametitle{Methodology}
    Here are the results of testing the model trained on a mesh with \(n\) nodes and tested on a set of meshes with varying number of nodes. The set of meshes has the following number of nodes \(n\):
    \[
        \begin{split}
            Mesh = \{36, 56, 60, 75, 80, 96, 109, 128, 143, 168, 177, 183,\\ 272, 318, 345, 431, 501, 576, 660, 902\}.
        \end{split}
    \]

    The model was trained on the mesh with 56, 75, 128, 183 nodes in different runs. The testing was made by running a simulation with all the meshes simultaneously and comparing the results of the model with the ground truth.
\end{frame}

\begin{frame}
    \frametitle{Model trained on 56 nodes}
    \begin{figure}
        \centering
        \includegraphics[width=0.8\textwidth]{/home/andrea/Documents/Internship/Codes/Beam/Sofa_Playground/images/beam_error_56_nodes.png}
    \end{figure}
\end{frame}

\begin{frame}
    \frametitle{Model trained on 75 nodes}
    \begin{figure}
        \centering
        \includegraphics[width=0.8\textwidth]{/home/andrea/Documents/Internship/Codes/Beam/Sofa_Playground/images/beam_error_75_nodes.png}
    \end{figure}

\end{frame}

\begin{frame}
    \frametitle{Model trained on 128 nodes}
    \begin{figure}
        \centering
        \includegraphics[width=0.8\textwidth]{/home/andrea/Documents/Internship/Codes/Beam/Sofa_Playground/images/beam_error_128_nodes.png}
    \end{figure}
\end{frame}

\begin{frame}
    \frametitle{Model trained on 183 nodes}
    \begin{figure}
        \centering
        \includegraphics[width=0.8\textwidth]{/home/andrea/Documents/Internship/Codes/Beam/Sofa_Playground/images/beam_error_183_nodes.png}
    \end{figure}
\end{frame}

\end{document}
