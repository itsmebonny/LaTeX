\documentclass{beamer}
\usetheme{Pittsburgh}
\beamertemplatenavigationsymbolsempty


\usepackage{amsmath}
\usepackage{amssymb}
\usepackage{bm} % For bold math symbols
\usepackage{graphicx}


\usepackage{subfigure}
\usepackage{multirow}
\usepackage{multicol}
\usepackage{color}
\usepackage{url}
\usepackage{hyperref}
\usepackage{listings}
\usepackage{algorithm}
\usepackage{physics}
% add image path
\graphicspath{{../Images/}}

\title{Weekly Updates\\
\tiny{Wednesday, 19 February 2025}}
\author{Andrea Bonifacio}
\date{}

\begin{document}

\begin{frame}
\titlepage
\end{frame}


\begin{frame}{Issues - Sampling Nodes}
    \begin{columns}
        \column{0.5\textwidth}
        \centering Pros
        \begin{itemize}
            \item Exact solution
            \item Easy comparisons between meshes
        \end{itemize}
        \column{0.5\textwidth}
        \centering Cons
        \begin{itemize}
            \item Stops working with complex geometries
            \item Still need to interpolate back
        \end{itemize}
    \end{columns}

    \vspace{1cm}

    Also, they are not really needed when working with a GNN, only for fully connected networks.
    \begin{itemize}
        \item A possible solution could be to map the low resolution mesh to the high resolution mesh in the solver to create a comparable ground truth.
        \item The downside is that we lose a bit of topology indepe
    \end{itemize}  

\end{frame}



\begin{frame}
\frametitle{Issues - Physics}
\begin{itemize}
    \item Impossible to implement the physics in the loss, because the prediction is not physically correct.
    \item Should somehow take into account the change in the physics of the system.
    \item In the dynamic case, the solver will never be able to predict the step \(t+1\), because the corrected solution at time \(t\) is ``breaking'' the physics.
\end{itemize}
\end{frame}

\begin{frame}
    \frametitle{Ideas - Neural Operators}
    \begin{itemize}
        \item Geometric Neural Operators \href{https://arxiv.org/pdf/2207.05209}{GNO}, where the mesh is deformed to a grid before being fed to the network.
        \item Super-Resolution Neural Operators \href{https://arxiv.org/pdf/2303.02584}{SRNO}, which is somewhat similar to our problem, but its output is a high-resolution tensor.
    \end{itemize}

    The main issue with our problem is that our target mesh needs to be of the same size as the input mesh, but our ground truth is a high-resolution mesh.
    

\end{frame}



\end{document}

