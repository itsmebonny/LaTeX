\section{Introduction}

Numerical simulations play a critical role in a wide array of scientific and engineering applications, providing insights into the behavior of physical systems under various conditions. Among the most prominent techniques for performing such simulations is Finite Element Modeling (FEM). FEM discretizes a continuous domain into a mesh of finite elements, allowing for the approximation of solutions to complex partial differential equations (PDEs). However, one significant drawback of FEM is its computational intensity, especially when high resolution is required for accurate results. This research aims to explore the potential of Deep Learning (DL) techniques to accelerate FEM simulations, focusing specifically on the deformation of objects subjected to external forces.

The deformation of an object under an applied force is directly tied to the object's discretization. In FEM, the object is represented by a mesh, where the resolution of the mesh—i.e., the size and number of elements—clearly impacts the accuracy and computational cost of the simulation. High-resolution meshes can capture fine details of deformation, leading to more accurate simulations, but they are computationally expensive and time-consuming. 

The main goal of this work is to study the efficacy of a method that combines both Finite Element Modeling (FEM) and DL to obtain a realistic simulation of an object in a fraction of the time that would be required by a traditional FEM simulation. The idea is to, somehow, train a DL model to have inside the information given by the refined discretization and pass them on a coarser discretization.

The idea of using DL techniques to solve scientific problem is not new. Thanks to the rise of new frameworks and libraries, such as TensorFlow and PyTorch, it is now possible to train very complex models on large datasets in a reasonable amount of time. For the problem at hand, a lot of different approaches can be found in the existing literature: a lot of them are based on the idea that the deep learning model should predict the whole dynamic of the system, for example MeshGraphNet \citep{pfaffLearningMeshBasedSimulation2021a} or its multiscale version \citep{fortunatoMultiScaleMeshGraphNets2022}, but these are just two examples of the many possible approaches \citep{jiangMeshfreeFlowNetPhysicsConstrainedDeep2020}, \citep{djeumouNeuralNetworksPhysicsInformed2022}, \citep{hanPredictingPhysicsMeshreduced2022a}. Other methods rely on solving a time independent problem, using various architectures, such as PINNs \citep{djeumouNeuralNetworksPhysicsInformed2022} or GNNs \citep{gaoPhysicsinformedGraphNeural2022}. The proposed method falls into the second category, as it will be explained in the following sections.

PIPPONE SU SUPER RESOLUTION PER ALLUNGARE IL BRODO


