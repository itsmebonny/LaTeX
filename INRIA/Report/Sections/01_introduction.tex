\section{Introduction}
This work will be dedicated to the study of a possible method to accelerate numerical simulations using Deep Learning (DL) techniques. The problem that will be study is the deformation of an object under the action of a force. The main idea is that the same object can be discretized with different resolutions and this will cause different results in the solution. In particular, the ``stiffness'' of the object is directly tied to its discretization. 

The main goal of this work is to study the efficacy of a method that combines both Finite Element Modeling (FEM) and DL to obtain a realistic simulation of an object in a fraction of the time that would be required by a traditional FEM simulation. The idea is to, somehow, train a DL model to have inside the information given by the refined discretization and pass them on a coarser discretization.

There are some works that have already tried to use DL to accelerate FEM simulations, mainly in the context of computational fluid dynamics (CFD). The main difference between the problem here and the CFD ones is that the CFD problems are usually solved on a fixed grid, while the problem here is solved on a moving grid. This excludes a lot of possible solutions that rely on the fact that the domain will stay the same during the simulation.\citep{Allaire_1992}
