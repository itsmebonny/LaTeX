\section{Problem setting}
\label{sec:problem_setting}

In this section, we outline the mathematical description of the problem using a Neo-Hookean material model, which is well-suited for large deformations in soft tissues and other hyperelastic materials.

We consider the boundary value problem of computing the deformation of a hyperelastic material under both Dirichlet and Neumann boundary conditions. The solid occupies a domain $\Omega \subset \mathbb{R}^d$ with $d=2,3$, whose boundary is $\partial\Omega = \Gamma_D \cup \Gamma_N$ with $\Gamma_D \cap \Gamma_N = \emptyset$. We assume Dirichlet conditions on $\Gamma_D$ known a priori, while Neumann boundary conditions on $\Gamma_N$ can change at any time step.

We use a Lagrangian description of the deformation with material coordinates given by the vector $\bm{X}$. The deformed state of each point of the solid is given by
\begin{equation}
    \bm{x} = \bm{X} + \bm{u}
\label{eq:deformation}
\end{equation}
where $\bm{u}$ is the displacement field.

For the Neo-Hookean material model, we first compute the deformation gradient $\bm{F} \in \mathbb{R}^{d\times d}$:
\begin{equation}
    \bm{F} = \frac{\partial \bm{x}}{\partial \bm{X}} = \bm{I} + \nabla_X \bm{u}
\label{eq:deformation_gradient}
\end{equation}
where $\bm{I} \in \mathbb{R}^{d\times d}$ is the identity matrix and $\nabla_X$ denotes the gradient with respect to material coordinates. The right Cauchy-Green deformation tensor is then defined as $\bm{C} = \bm{F}^T\bm{F}$, and the first invariant of $\bm{C}$ is $I_C = \text{tr}(\bm{C})$.

The Neo-Hookean strain-energy density function $\Psi$ is given by:
\begin{equation}
    \Psi(\bm{F}) = \frac{\mu}{2} (I_C - 3 - 2\ln(J)) + \frac{\lambda}{4} (J^2 - 1 - 2\ln(J))
\label{eq:neo_hookean_energy}
\end{equation}
where $J = \det(\bm{F})$ is the determinant of the deformation gradient (representing the volume change ratio), $\mu$ is the shear modulus, and $\lambda$ is the first Lamé parameter. These material parameters are related to the Young's modulus $E$ and the Poisson's ratio $\nu$ by:
\begin{equation}
    \mu = \frac{E}{2(1+\nu)} \quad \text{and} \quad \lambda = \frac{E\nu}{(1+\nu)(1-2\nu)}
\end{equation}

From the strain energy density function, we derive the first Piola-Kirchhoff stress tensor $\bm{P}$:
\begin{equation}
    \bm{P} = \frac{\partial \Psi}{\partial \bm{F}} = \mu \bm{F} - \mu \bm{F}^{-T} + \frac{\lambda}{2}(J^2-1)\bm{F}^{-T}
\label{eq:piola_stress}
\end{equation}

For dynamic problems, the equation of motion including inertial forces is:
\begin{equation}
    \begin{cases}
        \rho \frac{\partial^2 \bm{u}}{\partial t^2} - \nabla_X \cdot \bm{P} = \bm{b} \quad \text{in} \quad \Omega \\
        \bm{u} = \bm{u}_D \quad \text{on} \quad \Gamma_D \\
        \bm{P} \cdot \bm{N} = \bm{t} \quad \text{on} \quad \Gamma_N
    \end{cases}
\label{eq:dynamic_problem}
\end{equation}
where $\rho$ is the material density in the reference configuration, $\bm{b}$ is the external body force, $\bm{N}$ is the unit normal to $\Gamma_N$ in the reference configuration, and $\bm{t}$ is the traction applied to the boundary.

The weak form of equation \eqref{eq:dynamic_problem}, obtained from the principle of virtual work, is:
\begin{equation}
    \int_{\Omega} \rho \frac{\partial^2 \bm{u}}{\partial t^2} \cdot \bm{\eta} \, d\Omega + \int_{\Omega} \bm{P} : \nabla_X \bm{\eta} \, d\Omega = \int_{\Omega} \bm{b} \cdot \bm{\eta} \, d\Omega + \int_{\Gamma_N} \bm{t} \cdot \bm{\eta} \, d\Gamma
\label{eq:weak_form}
\end{equation}
where $\bm{\eta} \in \{\bm{\eta} \in H^1(\Omega) | \bm{\eta} = \bm{0} \text{ on } \Gamma_D\}$ is any vector-valued test function ($H^1(\Omega)$ being a Hilbert space). The first term on the left side represents the inertial forces, the second term denotes the internal virtual work, and the right side represents the virtual work from the applied external loads.

To numerically solve this system, we discretize the domain into finite elements and use the Newton-Raphson method to handle the nonlinearities inherent in the Neo-Hookean model. Due to these nonlinearities, at each time step, we solve iteratively for displacement corrections until convergence is achieved. This approach provides an accurate representation of large deformations in hyperelastic materials, making it suitable for simulating soft tissues and other materials that undergo significant shape changes.