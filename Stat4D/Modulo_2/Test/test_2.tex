\documentclass[12pt, a4paper]{article}
\usepackage[utf8]{inputenc}
\usepackage[T1]{fontenc}
\usepackage[italian]{babel}
\usepackage{amsmath}
\usepackage{amssymb}
\usepackage{geometry}
\geometry{a4paper, left=2.5cm, right=2.5cm, top=2.5cm, bottom=2.5cm}
\usepackage{parskip}
\usepackage{enumitem}
\usepackage{ FiraSans }
\usepackage{graphicx} % Necessario per \newpage

% ----- Impostazioni Liste -----
\setlist[enumerate,1]{label=\arabic*)} % Numeri per domande principali
\setlist[enumerate,2]{label=(\alph*), leftmargin=*} % Lettere per opzioni

\begin{document}

\begin{center}
    \Large\textbf{Test di Autovalutazione Aggiuntivo - Modulo 2} \\
    \vspace{0.2cm}
    \large\textit{Tecniche di Ricerca e Analisi dei Dati} \\
    \vspace{0.5cm}
    \normalsize{Istruzioni: Scegli l'unica risposta corretta per ogni domanda. Le soluzioni sono disponibili nell'ultima pagina.}
\end{center}
\vspace{1cm}

\begin{enumerate} % Inizio elenco domande principali
    \item Considerando i seguenti punteggi: 10, 15, 12, 15, 18, 11. Qual è la Mediana?
    \begin{enumerate} % Inizio opzioni domanda 1
        \item 15
        \item 13.5
        \item 12
        \item 14
    \end{enumerate}

    \item In una distribuzione fortemente asimmetrica negativa (con coda a sinistra), quale relazione è generalmente vera tra media, mediana e moda?
    \begin{enumerate}
        \item Media < Mediana < Moda
        \item Moda < Mediana < Media
        \item Media = Mediana = Moda
        \item Mediana < Moda < Media
    \end{enumerate}

    \item La somma dei quadrati degli scarti dalla media ($\sum (X_i - \bar{X})^2$) è nota come:
    \begin{enumerate}
        \item Varianza
        \item Deviazione Standard
        \item Devianza (o Somma dei Quadrati)
        \item Errore Standard
    \end{enumerate}

    \item Perché nella formula della varianza campionaria ($s^2$) si usa (n-1) al denominatore invece di n?
    \begin{enumerate}
        \item Per semplificare i calcoli.
        \item Perché la varianza del campione è sempre maggiore di quella della popolazione.
        \item Per ottenere una stima più accurata (non distorta) della varianza della popolazione.
        \item Perché n-1 rappresenta i gradi di libertà della moda.
    \end{enumerate}

    \item Un punteggio Z di -1.00 indica che il punteggio grezzo originale si trova:
    \begin{enumerate}
        \item Esattamente sulla media della distribuzione.
        \item Una deviazione standard sopra la media.
        \item Una deviazione standard sotto la media.
        \item A una distanza di 1 punto dalla media.
    \end{enumerate}

    \item Qual è il principale vantaggio della trasformazione dei punteggi grezzi in punteggi Z?
    \begin{enumerate}
        \item Rende tutti i punteggi positivi.
        \item Permette di confrontare punteggi provenienti da scale o distribuzioni diverse.
        \item Elimina gli outlier dalla distribuzione.
        \item Aumenta la media della distribuzione.
    \end{enumerate}

    \item Secondo la teoria classica della probabilità, qual è la probabilità di estrarre una pallina rossa da un'urna contenente 3 palline rosse e 7 palline blu?
    \begin{enumerate}
        \item 0.7
        \item 0.3
        \item 3/7 (circa 0.43)
        \item 7/10
    \end{enumerate}

    \item In una distribuzione normale standard (Z), qual è approssimativamente l'area (probabilità) compresa tra Z = 0 e Z = +2?
    \begin{enumerate}
        \item 68%
        \item 95%
        \item 47.5% (metà del 95% circa)
        \item 34% (metà del 68% circa)
    \end{enumerate}

    \item Cosa afferma il Teorema del Limite Centrale riguardo alla forma della distribuzione campionaria della media?
    \begin{enumerate}
        \item È sempre identica alla forma della popolazione da cui sono estratti i campioni.
        \item È sempre asimmetrica positiva.
        \item Tende alla normalità all'aumentare della dimensione del campione (n>=30), indipendentemente dalla forma della popolazione.
        \item Ha sempre media 0 e deviazione standard 1.
    \end{enumerate}

    \item L'Errore Standard ($\sigma_{\bar{X}}$) misura:
    \begin{enumerate}
        \item La variabilità dei singoli punteggi nella popolazione.
        \item L'errore commesso nel calcolare la media del campione.
        \item La variabilità media delle medie campionarie attorno alla media della popolazione.
        \item La differenza tra la media del campione e la media della popolazione.
    \end{enumerate}

    \item Cosa succede all'Errore Standard ($\sigma_{\bar{X}} = \sigma / \sqrt{n}$) se la dimensione del campione (n) aumenta?
    \begin{enumerate}
        \item Aumenta.
        \item Diminuisce.
        \item Rimane invariato.
        \item Diventa uguale alla deviazione standard della popolazione ($\sigma$).
    \end{enumerate}

    \item Nel processo di verifica delle ipotesi, l'ipotesi alternativa ($H_1$) solitamente afferma:
    \begin{enumerate}
        \item Che non c'è alcun effetto o differenza nella popolazione.
        \item Che l'effetto osservato nel campione è dovuto al caso.
        \item Che esiste un effetto o una differenza significativa nella popolazione.
        \item Che la media del campione è uguale alla media della popolazione.
    \end{enumerate}

    \item Commettere un Errore di I Tipo significa:
    \begin{enumerate}
        \item Accettare $H_0$ quando è falsa.
        \item Rifiutare $H_0$ quando è vera.
        \item Accettare $H_1$ quando è falsa.
        \item Rifiutare $H_1$ quando è vera.
    \end{enumerate}

    \item Il livello di significatività $\alpha$ rappresenta:
    \begin{enumerate}
        \item La probabilità di accettare $H_0$ quando è vera.
        \item La probabilità di commettere un Errore di I Tipo (rifiutare $H_0$ vera).
        \item La probabilità che $H_1$ sia vera.
        \item Il valore della statistica test calcolata.
    \end{enumerate}

    \item Se, in un test di ipotesi, il valore della statistica test calcolata (es. Z = 2.50) è maggiore del valore critico (es. Z critico = 1.96) per $\alpha=0.05$:
    \begin{enumerate}
        \item Si accetta l'ipotesi nulla $H_0$.
        \item Si rifiuta l'ipotesi nulla $H_0$.
        \item Non si può prendere una decisione.
        \item Si deve aumentare il livello di significatività $\alpha$.
    \end{enumerate}

    \item La distribuzione t di Student viene utilizzata al posto della distribuzione Z nella verifica di ipotesi sulla media quando:
    \begin{enumerate}
        \item La varianza della popolazione ($\sigma^2$) è nota e il campione è grande (n>=30).
        \item La varianza della popolazione ($\sigma^2$) non è nota e/o il campione è piccolo (n<30).
        \item La distribuzione della popolazione è asimmetrica.
        \item Si vuole calcolare la moda invece della media.
    \end{enumerate}

    \item Rispetto alla distribuzione normale standard, la distribuzione t di Student (specialmente per bassi gradi di libertà):
    \begin{enumerate}
        \item È più appuntita al centro e ha code più leggere.
        \item È identica.
        \item È più appiattita al centro e ha code più pesanti (maggior variabilità).
        \item È asimmetrica.
    \end{enumerate}

    \item Quanti gradi di libertà (gl) ha la distribuzione t utilizzata per un test t su un campione singolo di 25 soggetti?
    \begin{enumerate}
        \item 25
        \item 24 (n-1)
        \item 30
        \item Non ha gradi di libertà.
    \end{enumerate}

    \item Se un ricercatore ottiene un risultato con p-value = 0.03 e aveva fissato $\alpha = 0.05$, cosa dovrebbe concludere?
    \begin{enumerate}
        \item Accettare $H_0$ perché p > $\alpha$.
        \item Rifiutare $H_0$ perché p < $\alpha$.
        \item Accettare $H_0$ perché p < $\alpha$.
        \item Rifiutare $H_0$ perché p > $\alpha$.
    \end{enumerate}

    \item Quale indice di tendenza centrale è più appropriato usare se la distribuzione dei dati è fortemente asimmetrica e si vuole un indice non influenzato dagli outlier?
    \begin{enumerate}
        \item Media
        \item Moda
        \item Mediana
        \item Deviazione Standard
    \end{enumerate}

\end{enumerate} % Fine elenco domande principali

\newpage % Inizia una nuova pagina per le soluzioni

\begin{center}
    \Large\textbf{Griglia delle Soluzioni} \\
    \vspace{0.5cm}
    \normalsize{(Test di Autovalutazione Aggiuntivo - Modulo 2)}
\end{center}
\vspace{1cm}

\begin{enumerate}[leftmargin=*, label=\arabic*.]
    \item (b) 13.5 (Dati ordinati: 10, 11, \textbf{12}, \textbf{15}, 15, 18. N=6 è pari. Mediana = (12+15)/2 = 27/2 = 13.5).
    \item (a) Media < Mediana < Moda (La media è "trascinata" verso la coda lunga a sinistra).
    \item (c) Devianza (o Somma dei Quadrati).
    \item (c) Per ottenere una stima più accurata (non distorta) della varianza della popolazione (la varianza campionaria tende a sottostimare quella della popolazione).
    \item (c) Una deviazione standard sotto la media. (Segno negativo = sotto la media; Valore 1.00 = 1 DS).
    \item (b) Permette di confrontare punteggi provenienti da scale o distribuzioni diverse.
    \item (b) 0.3 (Casi favorevoli = 3 rosse; Casi possibili = 3 rosse + 7 blu = 10 totali. P = 3/10 = 0.3).
    \item (c) 47.5% (L'area tra -2 e +2 è circa 95%. Essendo simmetrica, l'area tra 0 e +2 è metà di 95%, cioè circa 47.5%).
    \item (c) Tende alla normalità all'aumentare della dimensione del campione (n>=30), indipendentemente dalla forma della popolazione.
    \item (c) La variabilità media delle medie campionarie attorno alla media della popolazione.
    \item (b) Diminuisce (Essendo n al denominatore sotto radice, se n aumenta, il denominatore aumenta e la frazione diminuisce).
    \item (c) Che esiste un effetto o una differenza significativa nella popolazione.
    \item (b) Rifiutare $H_0$ quando è vera. (È l'errore del "falso allarme").
    \item (b) La probabilità di commettere un Errore di I Tipo.
    \item (b) Si rifiuta l'ipotesi nulla $H_0$. (Il valore calcolato cade nella regione di rifiuto, essendo più estremo del valore critico).
    \item (b) La varianza della popolazione ($\sigma^2$) non è nota e/o il campione è piccolo (n<30). (Se $\sigma^2$ non è nota ma n>=30 si può ancora usare Z approssimato).
    \item (c) È più appiattita al centro e ha code più pesanti (maggior variabilità).
    \item (b) 24 (gl = n - 1 = 25 - 1 = 24).
    \item (b) Rifiutare $H_0$ perché p < $\alpha$ (0.03 è minore di 0.05).
    \item (c) Mediana (È l'indice di tendenza centrale meno sensibile ai valori estremi).
\end{enumerate}

\end{document}
% ----- FINE DOCUMENTO -----