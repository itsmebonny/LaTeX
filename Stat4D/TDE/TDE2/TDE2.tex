\documentclass[12pt, a4paper]{article}
\usepackage[utf8]{inputenc}
\usepackage[T1]{fontenc}
\usepackage[italian]{babel}
\usepackage{amsmath}
\usepackage{amssymb}
\usepackage{geometry}
\geometry{a4paper, left=2.5cm, right=2.5cm, top=2.5cm, bottom=2.5cm}
\usepackage{parskip} % Spazio tra paragrafi invece di indentazione
\usepackage{enumitem} % Per personalizzare elenchi
\usepackage{ FiraSans } % Font più moderno
\usepackage{graphicx} % Necessario per \newpage
\usepackage{multicol} % Per le soluzioni su più colonne
\usepackage{amsfonts} % Per simboli matematici aggiuntivi se necessari

% ----- Impostazioni Liste -----
\setlist[enumerate,1]{label=\arabic*.} % Numeri per domande principali
\setlist[enumerate,2]{label=(\alph*), leftmargin=*} % Lettere per opzioni

\begin{document}

\begin{center}
    \Large\textbf{Simulazione Esame - Tecniche di Ricerca e Analisi dei Dati (Sim. 2)} \\
    \vspace{0.2cm}
    \large\textit{Psicologia} \\
    \vspace{0.5cm}
    \normalsize{Istruzioni: Scegli l'unica risposta corretta per ogni domanda. Le soluzioni sono disponibili nell'ultima pagina.}
\end{center}
\vspace{1cm}

% ----- MODULO 1 -----
\section*{Modulo 1: Introduzione alla Ricerca in Psicologia}
\begin{enumerate}[resume]
    \item Qual è la distinzione fondamentale tra "metodo" e "metodologia" nella ricerca?
    \begin{enumerate}
        \item Il metodo si riferisce alle tecniche statistiche, la metodologia alla raccolta dati.
        \item Il metodo è la procedura specifica usata, la metodologia è lo studio e l'applicazione dei metodi.
        \item Il metodo è usato nella ricerca qualitativa, la metodologia in quella quantitativa.
        \item Non c'è una distinzione significativa tra i due termini.
    \end{enumerate}
    \vspace{0.3cm}

    \item Un ricercatore vuole studiare "l'intelligenza emotiva". Poiché è un costrutto astratto, deve prima:
    \begin{enumerate}
        \item Scegliere un campione rappresentativo.
        \item Definirlo operativamente attraverso indicatori misurabili.
        \item Formulare un'ipotesi nulla.
        \item Selezionare il test statistico più appropriato.
    \end{enumerate}
    \vspace{0.3cm}

    \item La scala di misura che classifica le persone in base al loro partito politico di preferenza (es. Partito A, Partito B, Partito C) è di tipo:
    \begin{enumerate}
        \item Ordinale.
        \item A intervalli.
        \item A rapporti.
        \item Nominale.
    \end{enumerate}
    \vspace{0.3cm}

    \item In uno studio sugli effetti del sonno sulla memoria, un ricercatore assegna casualmente i partecipanti a tre gruppi: 4 ore di sonno, 6 ore di sonno, 8 ore di sonno. Le "ore di sonno" rappresentano:
    \begin{enumerate}
        \item La variabile dipendente.
        \item Una variabile di disturbo.
        \item La variabile indipendente.
        \item Un errore sistematico.
    \end{enumerate}
    \vspace{0.3cm}

    \item Un errore casuale in una misurazione è caratterizzato dal fatto che:
    \begin{enumerate}
        \item È sempre dovuto a un difetto dello strumento.
        \item Varia in modo imprevedibile da una misurazione all'altra.
        \item Influenza tutte le misurazioni per eccesso.
        \item Può essere completamente eliminato con un buon disegno di ricerca.
    \end{enumerate}
    \vspace{0.3cm}
\end{enumerate}

% ----- MODULO 2 -----
\section*{Modulo 2: Elementi di Statistica Descrittiva e Inferenziale}
\begin{enumerate}[resume]
    \item Quale indice di tendenza centrale è definito come il valore che divide la distribuzione ordinata dei punteggi esattamente a metà?
    \begin{enumerate}
        \item Media.
        \item Moda.
        \item Mediana.
        \item Devianza.
    \end{enumerate}
    \vspace{0.3cm}

    \item La varianza è calcolata come:
    \begin{enumerate}
        \item La radice quadrata della deviazione standard.
        \item La media degli scarti dalla media.
        \item La media dei quadrati degli scarti dalla media.
        \item La somma degli scarti quadratici divisa per (n-2).
    \end{enumerate}
    \vspace{0.3cm}

    \item Se la media di una distribuzione di punteggi Z è 0 e la deviazione standard è 1, un punteggio grezzo che corrisponde a Z = 0 si trova:
    \begin{enumerate}
        \item Una deviazione standard sotto la media dei punteggi grezzi.
        \item Esattamente sulla media dei punteggi grezzi.
        \item Una deviazione standard sopra la media dei punteggi grezzi.
        \item All'estremo inferiore della distribuzione.
    \end{enumerate}
    \vspace{0.3cm}

    \item La distribuzione campionaria della media tende ad avere una forma normale se:
    \begin{enumerate}
        \item La popolazione di origine è normale, indipendentemente dalla numerosità del campione.
        \item La numerosità del campione è sufficientemente elevata (es. n >= 30), anche se la popolazione non è normale.
        \item La varianza della popolazione è nota.
        \item Sia (a) che (b) sono vere.
    \end{enumerate}
    \vspace{0.3cm}

    \item Rifiutare l'ipotesi nulla ($H_0$) quando essa è vera, nel contesto della verifica delle ipotesi, è definito:
    \begin{enumerate}
        \item Errore di II Tipo.
        \item Decisione corretta.
        \item Errore standard.
        \item Errore di I Tipo.
    \end{enumerate}
    \vspace{0.3cm}
\end{enumerate}

% ----- MODULO 3 -----
\section*{Modulo 3: La Significatività Statistica e il Confronto fra Due Campioni}
\begin{enumerate}[resume]
    \item La probabilità di commettere un Errore di II Tipo ($\beta$) è la probabilità di:
    \begin{enumerate}
        \item Rifiutare $H_0$ quando $H_0$ è vera.
        \item Accettare $H_0$ quando $H_0$ è falsa.
        \item Rifiutare $H_0$ quando $H_0$ è falsa.
        \item Accettare $H_1$ quando $H_1$ è vera.
    \end{enumerate}
    \vspace{0.3cm}

    \item Un ricercatore calcola il d di Cohen e ottiene un valore di 0.75. Secondo le convenzioni, questo indica una dimensione dell'effetto:
    \begin{enumerate}
        \item Piccola.
        \item Media.
        \item Grande.
        \item Trascurabile.
    \end{enumerate}
    \vspace{0.3cm}

    \item Se si riduce il livello di significatività $\alpha$ (es. da 0.05 a 0.01), a parità di altre condizioni, la probabilità di commettere un Errore di I Tipo:
    \begin{enumerate}
        \item Aumenta.
        \item Diminuisce.
        \item Rimane invariata.
        \item Diventa uguale a $\beta$.
    \end{enumerate}
    \vspace{0.3cm}

    \item Il t-test per campioni indipendenti è appropriato per confrontare:
    \begin{enumerate}
        \item Le mediane di due gruppi.
        \item Le medie di due gruppi distinti di soggetti.
        \item Le medie dello stesso gruppo di soggetti misurate in due momenti diversi.
        \item Le varianze di due gruppi.
    \end{enumerate}
    \vspace{0.3cm}

    \item L'analisi della potenza "a priori" viene condotta principalmente per:
    \begin{enumerate}
        \item Determinare la significatività statistica di un risultato già ottenuto.
        \item Calcolare la dimensione dell'effetto dopo aver condotto lo studio.
        \item Stimare il numero di soggetti necessari per condurre uno studio con una potenza adeguata.
        \item Verificare l'assunto di normalità dei dati.
    \end{enumerate}
    \vspace{0.3cm}
\end{enumerate}

% ----- MODULO 4 -----
\section*{Modulo 4: L'Analisi Della Varianza (ANOVA)}
\begin{enumerate}[resume]
    \item Qual è il principale vantaggio dell'ANOVA rispetto all'esecuzione di multipli t-test per confrontare le medie di più di due gruppi?
    \begin{enumerate}
        \item L'ANOVA è più semplice da calcolare.
        \item L'ANOVA controlla meglio l'inflazione dell'Errore di I Tipo complessivo.
        \item L'ANOVA non richiede l'assunto di omogeneità delle varianze.
        \item L'ANOVA può essere usata solo con due gruppi.
    \end{enumerate}
    \vspace{0.3cm}

    \item Se la varianza "tra i gruppi" è molto più grande della varianza "entro i gruppi", il valore del test F tenderà ad essere:
    \begin{enumerate}
        \item Vicino a 0.
        \item Vicino a 1.
        \item Significativamente maggiore di 1.
        \item Negativo.
    \end{enumerate}
    \vspace{0.3cm}

    \item In un disegno fattoriale 2x3 (due fattori, il primo con 2 livelli e il secondo con 3 livelli), quante condizioni sperimentali (o gruppi) ci sono in totale?
    \begin{enumerate}
        \item 2
        \item 3
        \item 5
        \item 6
    \end{enumerate}
    \vspace{0.3cm}

    \item I test post-hoc (es. test di Tukey, test di Bonferroni) nell'ANOVA vengono utilizzati quando:
    \begin{enumerate}
        \item Il test F non è risultato significativo.
        \item Si vuole aumentare la potenza del test F.
        \item Il test F è risultato significativo e si vuole identificare quali specifiche medie di gruppo differiscono tra loro.
        \item L'assunto di normalità è violato.
    \end{enumerate}
    \vspace{0.3cm}

    \item Il test t per misure ripetute è concettualmente un caso particolare di quale tipo di ANOVA?
    \begin{enumerate}
        \item ANOVA a una via per campioni indipendenti.
        \item ANOVA fattoriale.
        \item ANOVA per misure ripetute (con due soli livelli della variabile indipendente).
        \item MANOVA.
    \end{enumerate}
    \vspace{0.3cm}
\end{enumerate}

% ----- MODULO 5 -----
\section*{Modulo 5: Correlazione e Regressione Lineare}
\begin{enumerate}[resume]
    \item Se osserviamo che al diminuire dei punteggi di ansia aumentano i punteggi di rendimento scolastico, che tipo di correlazione lineare ci aspettiamo?
    \begin{enumerate}
        \item Positiva.
        \item Negativa.
        \item Nulla.
        \item Curvilinea.
    \end{enumerate}
    \vspace{0.3cm}

    \item L'interpretazione di un coefficiente di correlazione di Pearson (r) deve sempre tenere conto che:
    \begin{enumerate}
        \item Un valore elevato implica sempre una relazione di causa-effetto.
        \item È sensibile solo a relazioni non lineari.
        \item Misura solo la forza della relazione, non la direzione.
        \item Una correlazione non implica necessariamente causalità.
    \end{enumerate}
    \vspace{0.3cm}

    \item Se il coefficiente di determinazione $r^2$ tra ore di studio e voto all'esame è 0.49, cosa significa?
    \begin{enumerate}
        \item Il 49\% degli studenti ha superato l'esame.
        \item Le ore di studio spiegano il 49\% della variabilità dei voti all'esame.
        \item La correlazione tra ore di studio e voto è 0.70.
        \item Il 51\% della varianza è dovuto all'errore di misurazione.
    \end{enumerate}
    \vspace{0.3cm}

    \item Nella retta di regressione, il metodo dei minimi quadrati viene utilizzato per:
    \begin{enumerate}
        \item Massimizzare la somma dei quadrati degli errori di previsione.
        \item Minimizzare la somma dei quadrati degli errori di previsione (differenza tra Y osservato e Y predetto).
        \item Assicurare che il coefficiente di correlazione sia uguale a 1.
        \item Garantire che l'intercetta sia uguale a 0.
    \end{enumerate}
    \vspace{0.3cm}

    \item Per poter generalizzare i risultati di un'analisi di regressione campionaria alla popolazione, è necessario verificare:
    \begin{enumerate}
        \item Solo la significatività del coefficiente di correlazione r.
        \item La significatività dell'intercetta ($\alpha$) e/o del coefficiente di regressione ($\beta$) a livello poblazionale.
        \item Che la varianza residua sia uguale a zero.
        \item Che il campione sia composto da almeno 100 soggetti.
    \end{enumerate}
    \vspace{0.3cm}
\end{enumerate}

% ----- MODULO 6 -----
\section*{Modulo 6: Le Statistiche Non Parametriche}
\begin{enumerate}[resume]
    \item Quale delle seguenti situazioni NON giustificherebbe l'uso di un test non parametrico?
    \begin{enumerate}
        \item Dati misurati su scala nominale.
        \item Campioni molto piccoli con distribuzioni dei dati palesemente non normali.
        \item Dati quantitativi che soddisfano tutti gli assunti dei test parametrici (normalità, omoschedasticità).
        \item Presenza di outlier che distorcono marcatamente la media.
    \end{enumerate}
    \vspace{0.3cm}

    \item Il test Chi-quadrato per l'indipendenza valuta:
    \begin{enumerate}
        \item Se le medie di due o più categorie sono diverse.
        \item Se esiste un'associazione o dipendenza tra due variabili categoriali.
        \item Se la distribuzione di una singola variabile categoriale si adatta a un modello teorico.
        \item Se le varianze di due variabili sono omogenee.
    \end{enumerate}
    \vspace{0.3cm}

    \item Il test H di Kruskal-Wallis è l'alternativa non parametrica:
    \begin{enumerate}
        \item Al t-test per campioni indipendenti.
        \item All'ANOVA a una via per campioni indipendenti.
        \item Al test Chi-quadrato.
        \item All'ANOVA per misure ripetute.
    \end{enumerate}
    \vspace{0.3cm}

    \item Il test dei ranghi con segno di Wilcoxon è utilizzato per:
    \begin{enumerate}
        \item Confrontare due campioni indipendenti quando la variabile è ordinale.
        \item Confrontare due misurazioni ripetute (appaiate) sullo stesso campione quando i dati non sono normali.
        \item Verificare l'indipendenza tra due variabili nominali.
        \item Confrontare tre o più misurazioni ripetute.
    \end{enumerate}
    \vspace{0.3cm}

    \item Un limite dei test non parametrici, quando si potrebbero usare i test parametrici (cioè quando i loro assunti sono rispettati), è che i test non parametrici:
    \begin{enumerate}
        \item Sono sempre più difficili da interpretare.
        \item Richiedono campioni più grandi per raggiungere la stessa potenza.
        \item Hanno generalmente una potenza statistica inferiore.
        \item Non possono essere utilizzati con variabili quantitative.
    \end{enumerate}
    \vspace{0.3cm}
\end{enumerate}

\newpage
\begin{center}
    \Large\textbf{Griglia delle Soluzioni con Spiegazioni - Simulazione Esame (Sim. 2)}
\end{center}
\vspace{1cm}

\begin{footnotesize}
\begin{multicols}{2}
\textbf{Modulo 1}
\begin{enumerate}
    \item (b) \textit{Metodologia è lo studio dei metodi, metodo è la procedura specifica.}
    \item (b) \textit{Un costrutto va definito operativamente per essere misurato.}
    \item (d) \textit{La scala nominale classifica in categorie senza ordine.}
    \item (c) \textit{La VI è quella manipolata (ore di sonno).}
    \item (b) \textit{Gli errori casuali variano imprevedibilmente.}
\end{enumerate}
\vspace{0.5cm}
\textbf{Modulo 2}
\begin{enumerate}
    \setcounter{enumi}{5}
    \item (c) \textit{La mediana divide la distribuzione a metà.}
    \item (c) \textit{La varianza è la media dei quadrati degli scarti.}
    \item (b) \textit{Z=0 corrisponde alla media nella scala grezza.}
    \item (d) \textit{Tutte e due: popolazione normale o n elevato.}
    \item (d) \textit{Errore di I Tipo: rifiuto H0 vera.}
\end{enumerate}
\vspace{0.5cm}
\textbf{Modulo 3}
\begin{enumerate}
    \setcounter{enumi}{10}
    \item (b) \textit{$\beta$ è accettare H0 falsa.}
    \item (c) \textit{d=0.75 è una dimensione dell'effetto grande.}
    \item (b) \textit{Riducendo $\alpha$, diminuisce la probabilità di Errore I.}
    \item (b) \textit{t-test ind. confronta medie di gruppi distinti.}
    \item (c) \textit{Potenza "a priori" stima n per potenza adeguata.}
\end{enumerate}
\columnbreak
\textbf{Modulo 4}
\begin{enumerate}
    \setcounter{enumi}{15}
    \item (b) \textit{ANOVA controlla l'inflazione dell'Errore I.}
    \item (c) \textit{F grande indica varianza "tra" > "entro".}
    \item (d) \textit{2x3 = 6 condizioni sperimentali.}
    \item (c) \textit{Post-hoc identificano quali medie differiscono.}
    \item (c) \textit{t-test RM è un caso di ANOVA RM a due livelli.}
\end{enumerate}
\vspace{0.5cm}
\textbf{Modulo 5}
\begin{enumerate}
    \setcounter{enumi}{20}
    \item (b) \textit{Correlazione negativa: al crescere di X, Y diminuisce.}
    \item (d) \textit{Correlazione non implica causalità.}
    \item (b) \textit{$r^2$ spiega la proporzione di varianza.}
    \item (b) \textit{Minimi quadrati minimizzano gli errori di previsione.}
    \item (b) \textit{Verificare la significatività di $\alpha$ e/o $\beta$ nella popolazione.}
\end{enumerate}
\vspace{0.5cm}
\textbf{Modulo 6}
\begin{enumerate}
    \setcounter{enumi}{25}
    \item (c) \textit{Se gli assunti sono rispettati, meglio il parametrico.}
    \item (b) \textit{Chi-quadrato per indipendenza valuta l'associazione.}
    \item (b) \textit{Kruskal-Wallis è l'alternativa non parametrica all'ANOVA a una via.}
    \item (b) \textit{Wilcoxon confronta misure ripetute non normali.}
    \item (c) \textit{I non parametrici hanno potenza inferiore se si possono usare i parametrici.}
\end{enumerate}
\end{multicols}
\end{footnotesize}

\end{document}