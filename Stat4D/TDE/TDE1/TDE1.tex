\documentclass[12pt, a4paper]{article}
\usepackage[utf8]{inputenc}
\usepackage[T1]{fontenc}
\usepackage[italian]{babel}
\usepackage{amsmath}
\usepackage{amssymb}
\usepackage{geometry}
\geometry{a4paper, left=2.5cm, right=2.5cm, top=2.5cm, bottom=2.5cm}
\usepackage{parskip} % Spazio tra paragrafi invece di indentazione
\usepackage{enumitem} % Per personalizzare elenchi
\usepackage{ FiraSans } % Font più moderno
\usepackage{graphicx} % Necessario per \newpage
\usepackage{multicol} % Per le soluzioni su più colonne
\usepackage{amsfonts} % Per simboli matematici aggiuntivi se necessari

% ----- Impostazioni Liste -----
\setlist[enumerate,1]{label=\arabic*.} % Numeri per domande principali
\setlist[enumerate,2]{label=(\alph*), leftmargin=*} % Lettere per opzioni

\begin{document}

\begin{center}
    \Large\textbf{Simulazione Esame - Tecniche di Ricerca e Analisi dei Dati} \\
    \vspace{0.2cm}
    \large\textit{Psicologia} \\
    \vspace{0.5cm}
    \normalsize{Istruzioni: Scegli l'unica risposta corretta per ogni domanda. Le soluzioni sono disponibili nell'ultima pagina.}
\end{center}
\vspace{1cm}

% ----- MODULO 1 -----
\section*{Modulo 1: Introduzione alla Ricerca in Psicologia}
\begin{enumerate}[resume]
    \item La ricerca di base in psicologia ha come obiettivo principale:
    \begin{enumerate}
        \item Fornire strategie pratiche per la risoluzione di problemi concreti.
        \item Far progredire la conoscenza teorica su un dato argomento.
        \item Valutare l'efficacia di interventi terapeutici specifici.
        \item Descrivere le caratteristiche di un singolo individuo o caso.
    \end{enumerate}
    \vspace{0.3cm}

    \item Quale delle seguenti è la corretta sequenza delle fasi principali del processo di ricerca?
    \begin{enumerate}
        \item Analisi dei dati, interpretazione, raccolta dati, pianificazione.
        \item Identificazione del problema, pianificazione, raccolta dati, analisi dei dati, interpretazione.
        \item Raccolta dati, pianificazione, interpretazione, comunicazione dei risultati.
        \item Interpretazione, analisi dei dati, comunicazione, identificazione del problema.
    \end{enumerate}
    \vspace{0.3cm}

    \item L'operazionalizzazione di un costrutto come "l'ansia sociale" consiste nel:
    \begin{enumerate}
        \item Definirlo in termini teorici astratti.
        \item Scegliere la scala di misura più appropriata.
        \item Tradurlo in indicatori comportamentali osservabili e misurabili (es. frequenza di evitamento di situazioni sociali).
        \item Confrontarlo con altri costrutti psicologici simili.
    \end{enumerate}
    \vspace{0.3cm}

    \item Un ricercatore misura il livello di soddisfazione lavorativa (da 1=molto insoddisfatto a 7=molto soddisfatto) in un gruppo di impiegati. Questa variabile è misurata su scala:
    \begin{enumerate}
        \item Nominale.
        \item Ordinale.
        \item A intervalli.
        \item A rapporti.
    \end{enumerate}
    \vspace{0.3cm}

    \item L'errore sistematico in una misurazione si differenzia dall'errore casuale perché:
    \begin{enumerate}
        \item Si verifica solo in campioni molto piccoli.
        \item Tende ad annullarsi con misurazioni ripetute.
        \item Influenza tutte le misurazioni nella stessa direzione (eccesso o difetto).
        \item È impossibile da identificare o correggere.
    \end{enumerate}
    \vspace{0.3cm}
\end{enumerate}

% ----- MODULO 2 -----
\section*{Modulo 2: Elementi di Statistica Descrittiva e Inferenziale}
\begin{enumerate}[resume]
    \item Se in una distribuzione di punteggi la media è 50, la mediana è 55 e la moda è 60, la distribuzione è probabilmente:
    \begin{enumerate}
        \item Simmetrica unimodale.
        \item Asimmetrica positiva (coda a destra).
        \item Asimmetrica negativa (coda a sinistra).
        \item Bimodale.
    \end{enumerate}
    \vspace{0.3cm}

    \item La deviazione standard è una misura di:
    \begin{enumerate}
        \item Tendenza centrale.
        \item Posizione relativa di un punteggio.
        \item Variabilità o dispersione dei punteggi attorno alla media.
        \item Frequenza relativa di un punteggio.
    \end{enumerate}
    \vspace{0.3cm}

    \item Un punteggio Z = +2.5 indica che il punteggio grezzo originale:
    \begin{enumerate}
        \item È 2.5 volte la media.
        \item Si trova a 2.5 deviazioni standard sopra la media.
        \item È inferiore alla media di 2.5 punti.
        \item Corrisponde al 2.5° percentile.
    \end{enumerate}
    \vspace{0.3cm}

    \item Qual è la principale funzione della statistica inferenziale?
    \begin{enumerate}
        \item Descrivere e riassumere grandi quantità di dati.
        \item Trasformare i punteggi grezzi in punteggi standardizzati.
        \item Trarre conclusioni sulla popolazione basandosi sui dati di un campione.
        \item Calcolare la media, la mediana e la moda di una distribuzione.
    \end{enumerate}
    \vspace{0.3cm}

    \item Il livello di significatività $\alpha$ convenzionalmente fissato a 0.05 indica:
    \begin{enumerate}
        \item Una probabilità del 95\% che l'ipotesi nulla sia vera.
        \item Una probabilità del 5\% di rifiutare l'ipotesi nulla quando essa è vera (Errore di I Tipo).
        \item Una probabilità del 5\% che l'ipotesi alternativa sia falsa.
        \item Che l'effetto osservato è sicuramente importante.
    \end{enumerate}
    \vspace{0.3cm}
\end{enumerate}

% ----- MODULO 3 -----
\section*{Modulo 3: La Significatività Statistica e il Confronto fra Due Campioni}
\begin{enumerate}[resume]
    \item Se si rifiuta l'ipotesi nulla ($H_0$) quando in realtà essa è falsa, si commette:
    \begin{enumerate}
        \item Un Errore di I Tipo ($\alpha$).
        \item Un Errore di II Tipo ($\beta$).
        \item Una decisione corretta.
        \item Un errore nella scelta del test statistico.
    \end{enumerate}
    \vspace{0.3cm}

    \item La dimensione dell'effetto (effect size) fornisce informazioni su:
    \begin{enumerate}
        \item La probabilità che il risultato sia dovuto al caso.
        \item Quanto è "grande" o "importante" l'effetto osservato o la differenza tra gruppi.
        \item Il numero di soggetti necessari per lo studio.
        \item La forma della distribuzione campionaria.
    \end{enumerate}
    \vspace{0.3cm}

    \item La potenza statistica (1 - $\beta$) di un test è la probabilità di:
    \begin{enumerate}
        \item Accettare $H_0$ quando è vera.
        \item Rifiutare $H_0$ quando è vera.
        \item Accettare $H_0$ quando è falsa.
        \item Rifiutare $H_0$ quando è falsa (individuare un effetto reale).
    \end{enumerate}
    \vspace{0.3cm}

    \item Nel t-test per campioni indipendenti, l'assunto di omoschedasticità si riferisce al fatto che:
    \begin{enumerate}
        \item Le medie dei due gruppi devono essere uguali nella popolazione.
        \item Le varianze dei due gruppi devono essere uguali nelle popolazioni da cui sono estratti i campioni.
        \item I dati devono essere distribuiti normalmente in entrambi i campioni.
        \item I due campioni devono avere la stessa numerosità.
    \end{enumerate}
    \vspace{0.3cm}

    \item A parità di altre condizioni, cosa succede alla potenza di un test statistico se si aumenta la dimensione del campione?
    \begin{enumerate}
        \item Diminuisce.
        \item Aumenta.
        \item Rimane invariata.
        \item Diventa uguale al livello $\alpha$.
    \end{enumerate}
    \vspace{0.3cm}
\end{enumerate}

% ----- MODULO 4 -----
\section*{Modulo 4: L'Analisi Della Varianza (ANOVA)}
\begin{enumerate}[resume]
    \item L'ANOVA a una via viene utilizzata per confrontare:
    \begin{enumerate}
        \item Le medie di due gruppi su una variabile dipendente.
        \item Le medie di tre o più gruppi (definiti da una variabile indipendente) su una variabile dipendente.
        \item La relazione tra due variabili quantitative.
        \item Le frequenze di categorie di una variabile nominale.
    \end{enumerate}
    \vspace{0.3cm}

    \item Nell'ANOVA, il rapporto F si calcola come:
    \begin{enumerate}
        \item Varianza entro i gruppi / Varianza tra i gruppi.
        \item Varianza totale / Varianza tra i gruppi.
        \item Varianza tra i gruppi / Varianza entro i gruppi.
        \item Devianza tra i gruppi / Devianza entro i gruppi.
    \end{enumerate}
    \vspace{0.3cm}

    \item Cosa indica un "effetto di interazione" in un'ANOVA fattoriale (es. a due vie)?
    \begin{enumerate}
        \item Che entrambe le variabili indipendenti hanno un effetto significativo sulla variabile dipendente.
        \item Che l'effetto di una variabile indipendente sulla variabile dipendente dipende dal livello dell'altra variabile indipendente.
        \item Che le medie dei gruppi sono tutte significativamente diverse tra loro.
        \item Che la varianza entro i gruppi è molto elevata.
    \end{enumerate}
    \vspace{0.3cm}

    \item Se un'ANOVA a una via con 3 gruppi produce un risultato F significativo, cosa si può concludere immediatamente?
    \begin{enumerate}
        \item Tutte e tre le medie dei gruppi sono significativamente diverse tra loro.
        \item Almeno una media di un gruppo è significativamente diversa da almeno un'altra media.
        \item L'ipotesi nulla è vera.
        \item Non ci sono differenze significative tra le medie dei gruppi.
    \end{enumerate}
    \vspace{0.3cm}

    \item L'assunto di sfericità nell'ANOVA per misure ripetute è analogo a quale assunto dell'ANOVA per campioni indipendenti?
    \begin{enumerate}
        \item Normalità della distribuzione.
        \item Indipendenza delle osservazioni.
        \item Omogeneità delle varianze (omoschedasticità).
        \item Additività degli effetti.
    \end{enumerate}
    \vspace{0.3cm}
\end{enumerate}

% ----- MODULO 5 -----
\section*{Modulo 5: Correlazione e Regressione Lineare}
\begin{enumerate}[resume]
    \item Un coefficiente di correlazione di Pearson r = -0.85 indica una relazione:
    \begin{enumerate}
        \item Positiva forte.
        \item Negativa forte.
        \item Positiva debole.
        \item Negativa debole.
    \end{enumerate}
    \vspace{0.3cm}

    \item Se due variabili X e Y hanno un coefficiente di correlazione r = 0, significa che:
    \begin{enumerate}
        \item Non esiste alcuna relazione di alcun tipo tra X e Y.
        \item Esiste una forte relazione non lineare tra X e Y.
        \item Non esiste una relazione lineare tra X e Y.
        \item X causa Y.
    \end{enumerate}
    \vspace{0.3cm}

    \item Il coefficiente di determinazione ($r^2$) esprime:
    \begin{enumerate}
        \item La forza e la direzione della relazione lineare.
        \item La probabilità che la correlazione osservata sia dovuta al caso.
        \item La proporzione di varianza della variabile dipendente spiegata dalla variabile indipendente.
        \item L'errore standard della stima nella regressione.
    \end{enumerate}
    \vspace{0.3cm}

    \item Nella retta di regressione $Y = a + bX$, il coefficiente 'a' rappresenta:
    \begin{enumerate}
        \item La pendenza della retta.
        \item Il valore predetto di Y quando X=0 (intercetta).
        \item Il coefficiente di correlazione.
        \item L'errore di previsione.
    \end{enumerate}
    \vspace{0.3cm}

    \item Qual è uno dei principali limiti dell'interpretazione di un coefficiente di correlazione elevato?
    \begin{enumerate}
        \item Non indica la direzione della relazione.
        \item Non può essere utilizzato per fare previsioni.
        \item Non implica necessariamente una relazione di causa-effetto.
        \item È valido solo per campioni molto grandi.
    \end{enumerate}
    \vspace{0.3cm}
\end{enumerate}

% ----- MODULO 6 -----
\section*{Modulo 6: Le Statistiche Non Parametriche}
\begin{enumerate}[resume]
    \item Il test Chi-quadrato per la bontà di adattamento è utilizzato per verificare se:
    \begin{enumerate}
        \item Le medie di due campioni sono diverse.
        \item Esiste un'associazione tra due variabili nominali.
        \item Le frequenze osservate in un campione si conformano a una distribuzione teorica attesa.
        \item La varianza di un campione è uguale alla varianza di una popolazione.
    \end{enumerate}
    \vspace{0.3cm}

    \item Il test U di Mann-Whitney è l'alternativa non parametrica al:
    \begin{enumerate}
        \item t-test per campioni appaiati.
        \item t-test per campioni indipendenti.
        \item ANOVA a una via.
        \item Test Chi-quadrato.
    \end{enumerate}
    \vspace{0.3cm}

    \item Per confrontare i punteggi di ansia (scala ordinale) misurati in tre momenti diversi (pre-trattamento, post-trattamento, follow-up) sullo stesso gruppo di pazienti, quale test non parametrico è più appropriato se gli assunti parametrici non sono soddisfatti?
    \begin{enumerate}
        \item Test di Kruskal-Wallis.
        \item Test di Wilcoxon per ranghi segnati.
        \item Test di Friedman.
        \item Test U di Mann-Whitney.
    \end{enumerate}
    \vspace{0.3cm}

    \item Quale dei seguenti è un vantaggio del metodo bootstrap?
    \begin{enumerate}
        \item Aumenta sempre la potenza statistica del test.
        \item Non richiede alcuna assunzione sulla distribuzione dei dati per stimare la distribuzione campionaria.
        \item È più semplice da calcolare manualmente rispetto ai test tradizionali.
        \item Elimina la necessità di raccogliere dati campionari.
    \end{enumerate}
    \vspace{0.3cm}

    \item I test non parametrici, rispetto ai test parametrici (quando gli assunti di questi ultimi sono validi):
    \begin{enumerate}
        \item Hanno generalmente maggiore potenza.
        \item Hanno generalmente minore potenza.
        \item Hanno sempre la stessa potenza.
        \item Sono sempre preferibili per la loro semplicità.
    \end{enumerate}
    \vspace{0.3cm}
\end{enumerate}

\newpage
\begin{center}
    \Large\textbf{Griglia delle Soluzioni con Spiegazioni - Simulazione Esame}
\end{center}
\vspace{1cm}

\begin{footnotesize}
\begin{multicols}{2}
\textbf{Modulo 1}
\begin{enumerate}
    \item (b) \textit{La ricerca di base mira all'avanzamento teorico.}
    \item (b) \textit{La sequenza corretta è: problema, piano, dati, analisi, interpretazione.}
    \item (c) \textit{Operazionalizzare significa tradurre in indicatori misurabili.}
    \item (b) \textit{La scala di soddisfazione è ordinale (ordine, ma non intervalli uguali).}
    \item (c) \textit{L'errore sistematico influenza tutte le misure nella stessa direzione.}
\end{enumerate}
\vspace{0.5cm}
\textbf{Modulo 2}
\begin{enumerate}
    \setcounter{enumi}{5} % Riprende la numerazione da 6
    \item (c) \textit{Media < Mediana < Moda indica asimmetria negativa.}
    \item (c) \textit{La deviazione standard misura la dispersione attorno alla media.}
    \item (b) \textit{Z=+2.5 è 2.5 deviazioni standard sopra la media.}
    \item (c) \textit{La statistica inferenziale generalizza dal campione alla popolazione.}
    \item (b) \textit{$\alpha$=0.05 è la probabilità di Errore di I Tipo.}
\end{enumerate}
\vspace{0.5cm}
\textbf{Modulo 3}
\begin{enumerate}
    \setcounter{enumi}{10} % Riprende la numerazione da 11
    \item (c) \textit{Rifiutare H0 falsa è una decisione corretta.}
    \item (b) \textit{La dimensione dell'effetto indica l'importanza pratica dell'effetto.}
    \item (d) \textit{La potenza è la probabilità di rifiutare H0 falsa.}
    \item (b) \textit{Omoschedasticità: varianze uguali nei gruppi.}
    \item (b) \textit{Aumentare n aumenta la potenza.}
\end{enumerate}
\columnbreak % Passa alla colonna successiva
\textbf{Modulo 4}
\begin{enumerate}
    \setcounter{enumi}{15} % Riprende la numerazione da 16
    \item (b) \textit{ANOVA confronta 3+ medie.}
    \item (c) \textit{F = Varianza tra / Varianza entro.}
    \item (b) \textit{Interazione: l'effetto di una VI dipende dall'altra.}
    \item (b) \textit{F significativo: almeno una media differisce.}
    \item (c) \textit{Sfericità (RM) analoga a omoschedasticità (indipendenti).}
\end{enumerate}
\vspace{0.5cm}
\textbf{Modulo 5}
\begin{enumerate}
    \setcounter{enumi}{20} % Riprende la numerazione da 21
    \item (b) \textit{r=-0.85 è una correlazione negativa forte.}
    \item (c) \textit{r=0: nessuna relazione LINEARE.}
    \item (c) \textit{$r^2$: varianza spiegata.}
    \item (b) \textit{'a' è l'intercetta (Y quando X=0).}
    \item (c) \textit{Correlazione non implica causalità.}
\end{enumerate}
\vspace{0.5cm}
\textbf{Modulo 6}
\begin{enumerate}
    \setcounter{enumi}{25} % Riprende la numerazione da 26
    \item (c) \textit{Chi-quadrato bontà di adattamento: frequenze osservate vs attese.}
    \item (b) \textit{U di Mann-Whitney: alternativa a t-test ind.}
    \item (c) \textit{Friedman: misure ripetute non parametriche.}
    \item (b) \textit{Bootstrap: non richiede assunti sulla distribuzione.}
    \item (b) \textit{Non parametrici: minore potenza se si possono usare i parametrici.}
\end{enumerate}
\end{multicols}
\end{footnotesize}


\end{document}