\documentclass[12pt, a4paper]{article}
\usepackage[utf8]{inputenc}
\usepackage[T1]{fontenc}
\usepackage[italian]{babel}
\usepackage{amsmath}
\usepackage{amssymb}
\usepackage{geometry}
\geometry{a4paper, left=2.5cm, right=2.5cm, top=2.5cm, bottom=2.5cm}
\usepackage{parskip} % Spazio tra paragrafi invece di indentazione
\usepackage{enumitem} % Per personalizzare elenchi
\usepackage{ FiraSans } % Font più moderno
\usepackage{graphicx} % Necessario per \newpage
\usepackage{multicol} % Per le soluzioni su più colonne
\usepackage{amsfonts} % Per simboli matematici aggiuntivi se necessari

% ----- Impostazioni Liste -----
\setlist[enumerate,1]{label=\arabic*.} % Numeri per domande principali
\setlist[enumerate,2]{label=(\alph*), leftmargin=*} % Lettere per opzioni

\begin{document}

\begin{center}
    \Large\textbf{Simulazione Esame - Tecniche di Ricerca e Analisi dei Dati (Sim. 3)} \\
    \vspace{0.2cm}
    \large\textit{Psicologia} \\
    \vspace{0.5cm}
    \normalsize{Istruzioni: Scegli l'unica risposta corretta per ogni domanda. Le soluzioni sono disponibili nell'ultima pagina.}
\end{center}
\vspace{1cm}

% ----- MODULO 1 -----
\section*{Modulo 1: Introduzione alla Ricerca in Psicologia}
\begin{enumerate}[resume]
    \item La psicologia è considerata una scienza perché:
    \begin{enumerate}
        \item Si basa su opinioni soggettive e intuizioni.
        \item Utilizza metodi empirici e sperimentali per raccogliere e analizzare dati.
        \item Studia unicamente il comportamento osservabile, ignorando i processi mentali.
        \item I suoi risultati sono sempre incontrovertibili e definitivi.
    \end{enumerate}
    \vspace{0.3cm}

    \item In un disegno di ricerca, una "variabile di disturbo" (o confondente procedurale) è una variabile che:
    \begin{enumerate}
        \item È stata intenzionalmente manipolata dal ricercatore.
        \item Covaria con la variabile indipendente a causa di errori metodologici e può influenzare la variabile dipendente.
        \item È misurata come risultato dell'esperimento.
        \item È intrinsecamente associata alla variabile indipendente nella sua operazionalizzazione.
    \end{enumerate}
    \vspace{0.3cm}

    \item Se un ricercatore misura il "livello di istruzione" classificando i partecipanti come "Scuola primaria", "Scuola secondaria", "Laurea", "Post-laurea", su quale scala di misura sta operando?
    \begin{enumerate}
        \item Nominale.
        \item Ordinale.
        \item A intervalli.
        \item A rapporti.
    \end{enumerate}
    \vspace{0.3cm}

    \item Qual è lo scopo principale di una condizione di controllo in un esperimento?
    \begin{enumerate}
        \item Aumentare il numero di partecipanti allo studio.
        \item Fornire una baseline di confronto per valutare l'effetto della condizione sperimentale.
        \item Misurare la variabile indipendente.
        \item Assicurare che tutti i partecipanti ricevano il trattamento.
    \end{enumerate}
    \vspace{0.3cm}

    \item La sezione "Discussione" di un articolo scientifico ha tipicamente lo scopo di:
    \begin{enumerate}
        \item Descrivere dettagliatamente le procedure utilizzate per raccogliere i dati.
        \item Presentare i risultati statistici in forma numerica e grafica.
        \item Interpretare i risultati alla luce delle ipotesi e della letteratura esistente, evidenziando limiti e implicazioni.
        \item Fornire un riassunto conciso dell'intero studio.
    \end{enumerate}
    \vspace{0.3cm}
\end{enumerate}

% ----- MODULO 2 -----
\section*{Modulo 2: Elementi di Statistica Descrittiva e Inferenziale}
\begin{enumerate}[resume]
    \item Quale indice di tendenza centrale è meno influenzato dalla presenza di valori estremi (outlier) in una distribuzione?
    \begin{enumerate}
        \item Media.
        \item Mediana.
        \item Moda.
        \item Deviazione standard.
    \end{enumerate}
    \vspace{0.3cm}

    \item La devianza (Somma dei Quadrati, SS) rappresenta:
    \begin{enumerate}
        \item La media dei quadrati degli scarti dalla media.
        \item La radice quadrata della varianza.
        \item La somma dei quadrati delle differenze tra ogni punteggio e la media.
        \item L'errore standard della media.
    \end{enumerate}
    \vspace{0.3cm}

    \item Se trasformiamo un'intera distribuzione di punteggi grezzi in punteggi Z, la nuova distribuzione Z avrà sempre:
    \begin{enumerate}
        \item Media 100 e deviazione standard 15.
        \item Media 0 e deviazione standard 1.
        \item Media uguale alla media dei punteggi grezzi e deviazione standard 0.
        \item Media 1 e deviazione standard 0.
    \end{enumerate}
    \vspace{0.3cm}

    \item La probabilità teorica di ottenere "testa" nel lancio di una moneta non truccata è un esempio di probabilità:
    \begin{enumerate}
        \item Frequentista (o a posteriori).
        \item Soggettiva.
        \item Classica (o a priori).
        \item Condizionata.
    \end{enumerate}
    \vspace{0.3cm}

    \item Nel processo di verifica delle ipotesi, l'ipotesi nulla ($H_0$) viene formulata in modo da:
    \begin{enumerate}
        \item Affermare l'esistenza di un effetto o di una differenza.
        \item Essere quella che il ricercatore spera di confermare.
        \item Essere sottoposta direttamente a verifica statistica, ipotizzando l'assenza di un effetto o differenza significativa.
        \item Coincidere sempre con l'ipotesi alternativa ($H_1$).
    \end{enumerate}
    \vspace{0.3cm}
\end{enumerate}

% ----- MODULO 3 -----
\section*{Modulo 3: La Significatività Statistica e il Confronto fra Due Campioni}
\begin{enumerate}[resume]
    \item Se un test statistico produce un p-value di 0.02, e il livello $\alpha$ scelto è 0.05, la decisione corretta è:
    \begin{enumerate}
        \item Accettare $H_0$.
        \item Rifiutare $H_0$.
        \item Aumentare il campione e ripetere il test.
        \item Concludere che non c'è effetto.
    \end{enumerate}
    \vspace{0.3cm}

    \item La "dimensione dell'effetto dei punteggi grezzi" in un confronto tra due medie si riferisce semplicemente a:
    \begin{enumerate}
        \item La differenza tra le due medie campionarie.
        \item Il rapporto tra la differenza delle medie e la deviazione standard.
        \item La probabilità che la differenza sia dovuta al caso.
        \item Il numero di deviazioni standard di cui le medie differiscono.
    \end{enumerate}
    \vspace{0.3cm}

    \item A parità di altre condizioni, quale dei seguenti fattori NON influenza direttamente la potenza statistica di un test?
    \begin{enumerate}
        \item Il livello di significatività ($\alpha$) scelto.
        \item L'ampiezza del campione (n).
        \item La dimensione dell'effetto nella popolazione (d).
        \item Il software statistico utilizzato per l'analisi.
    \end{enumerate}
    \vspace{0.3cm}

    \item Il t-test per campioni indipendenti assume che le osservazioni all'interno di ciascun gruppo siano:
    \begin{enumerate}
        \item Dipendenti tra loro.
        \item Correlate positivamente.
        \item Indipendenti le une dalle altre.
        \item Misurate su scala nominale.
    \end{enumerate}
    \vspace{0.3cm}

    \item Quando si confrontano due campioni indipendenti e si sospetta una violazione dell'assunto di omogeneità delle varianze con numerosità campionarie diverse, è consigliabile:
    \begin{enumerate}
        \item Utilizzare sempre il test Z.
        \item Ignorare la violazione se l'effetto è grande.
        \item Considerare versioni modificate del t-test (es. test di Welch) o test non parametrici.
        \item Aumentare il livello $\alpha$ per compensare.
    \end{enumerate}
    \vspace{0.3cm}
\end{enumerate}

% ----- MODULO 4 -----
\section*{Modulo 4: L'Analisi Della Varianza (ANOVA)}
\begin{enumerate}[resume]
    \item Nell'ANOVA, la "varianza entro i gruppi" (within-group variance) riflette principalmente:
    \begin{enumerate}
        \item L'effetto della variabile indipendente.
        \item Le differenze sistematiche tra i gruppi.
        \item La variabilità casuale e le differenze individuali non spiegate dalla VI.
        \item La media generale di tutti i punteggi.
    \end{enumerate}
    \vspace{0.3cm}

    \item Se il valore F calcolato in un'ANOVA è molto vicino a 1, è probabile che:
    \begin{enumerate}
        \item L'ipotesi nulla sia falsa e ci siano differenze significative tra i gruppi.
        \item L'ipotesi nulla sia vera e non ci siano differenze significative tra i gruppi.
        \item Ci sia un forte effetto di interazione.
        \item L'assunto di normalità sia stato violato.
    \end{enumerate}
    \vspace{0.3cm}

    \item Un disegno fattoriale è definito tale perché:
    \begin{enumerate}
        \item Coinvolge una sola variabile indipendente con molti livelli.
        \item Esamina l'effetto simultaneo di due o più variabili indipendenti (fattori) sulla variabile dipendente.
        \item Utilizza sempre misure ripetute.
        \item Si applica solo a variabili dipendenti misurate su scala nominale.
    \end{enumerate}
    \vspace{0.3cm}

    \item Nell'ANOVA per misure ripetute, come viene gestita la variabilità dovuta alle differenze individuali tra i soggetti?
    \begin{enumerate}
        \item Viene ignorata, assumendo che sia trascurabile.
        \item Viene inclusa nella varianza tra i trattamenti.
        \item Viene stimata e rimossa dalla varianza d'errore, aumentando la sensibilità del test.
        \item Si controlla aumentando la numerosità del campione.
    \end{enumerate}
    \vspace{0.3cm}

    \item L'Eta quadrato ($\eta^2$) nell'ANOVA indica:
    \begin{enumerate}
        \item La probabilità di commettere un Errore di I Tipo.
        \item La proporzione di varianza della variabile dipendente spiegata dalla variabile indipendente (fattore).
        \item Il numero di gruppi che differiscono significativamente.
        \item La media dei quadrati della varianza d'errore.
    \end{enumerate}
    \vspace{0.3cm}
\end{enumerate}

% ----- MODULO 5 -----
\section*{Modulo 5: Correlazione e Regressione Lineare}
\begin{enumerate}[resume]
    \item Se il grafico di dispersione tra due variabili mostra i punti che si dispongono approssimativamente lungo una linea retta che sale da sinistra a destra, la correlazione è:
    \begin{enumerate}
        \item Negativa e lineare.
        \item Positiva e lineare.
        \item Nulla.
        \item Curvilinea.
    \end{enumerate}
    \vspace{0.3cm}

    \item Quale dei seguenti valori del coefficiente di correlazione di Pearson (r) indica la relazione lineare più forte?
    \begin{enumerate}
        \item +0.70
        \item -0.85
        \item +0.50
        \item -0.10
    \end{enumerate}
    \vspace{0.3cm}

    \item Se il coefficiente di correlazione tra X e Y è +0.60, il coefficiente di determinazione ($r^2$) sarà:
    \begin{enumerate}
        \item 0.60
        \item 0.36
        \item -0.60
        \item 1.20
    \end{enumerate}
    \vspace{0.3cm}

    \item Nella retta di regressione $Y = a + bX$, il coefficiente 'b' (pendenza) indica:
    \begin{enumerate}
        \item Il valore di Y quando X è zero.
        \item La variazione media attesa in Y per ogni variazione unitaria in X.
        \item La forza della relazione tra X e Y.
        \item L'errore standard della stima.
    \end{enumerate}
    \vspace{0.3cm}

    \item L'analisi della regressione è più appropriata della semplice correlazione quando:
    \begin{enumerate}
        \item Si vuole solo descrivere la forza e la direzione di un'associazione.
        \item Si vuole predire i valori di una variabile dipendente basandosi su una variabile indipendente, ipotizzando una relazione causale.
        \item Entrambe le variabili sono misurate su scala nominale.
        \item Non si hanno ipotesi specifiche sulla relazione tra le variabili.
    \end{enumerate}
    \vspace{0.3cm}
\end{enumerate}

% ----- MODULO 6 -----
\section*{Modulo 6: Le Statistiche Non Parametriche}
\begin{enumerate}[resume]
    \item Il test Chi-quadrato per la bontà di adattamento confronta:
    \begin{enumerate}
        \item Le medie osservate con le medie attese.
        \item Le frequenze osservate in diverse categorie con le frequenze teoricamente attese per quelle categorie.
        \item Le varianze di due campioni.
        \item I ranghi di due campioni indipendenti.
    \end{enumerate}
    \vspace{0.3cm}

    \item Quale test non parametrico è appropriato per valutare se esiste un'associazione tra il genere (Maschio/Femmina) e la preferenza per un tipo di film (Azione/Commedia/Drammatico)?
    \begin{enumerate}
        \item Test U di Mann-Whitney.
        \item Test H di Kruskal-Wallis.
        \item Test Chi-quadrato per l'indipendenza.
        \item Test di Friedman.
    \end{enumerate}
    \vspace{0.3cm}

    \item Il test di Kruskal-Wallis è utilizzato quando si vuole confrontare:
    \begin{enumerate}
        \item Due misurazioni ripetute su dati ordinali.
        \item Le mediane/distribuzioni di tre o più campioni indipendenti su dati ordinali o non normali.
        \item L'associazione tra due variabili ordinali.
        \item Le frequenze di più di due categorie per una singola variabile.
    \end{enumerate}
    \vspace{0.3cm}

    \item Il test dei ranghi con segno di Wilcoxon è l'alternativa non parametrica al:
    \begin{enumerate}
        \item t-test per campioni indipendenti.
        \item ANOVA a una via.
        \item t-test per campioni appaiati (misure ripetute).
        \item Coefficiente di correlazione di Pearson.
    \end{enumerate}
    \vspace{0.3cm}

    \item Uno dei principali motivi per cui si ricorre ai test non parametrici è la violazione dell'assunto di:
    \begin{enumerate}
        \item Indipendenza delle osservazioni.
        \item Linearità della relazione.
        \item Normalità della distribuzione dei dati (specialmente con campioni piccoli).
        \item Presenza di un effetto di interazione.
    \end{enumerate}
    \vspace{0.3cm}
\end{enumerate}

\newpage
\begin{center}
    \Large\textbf{Griglia delle Soluzioni con Spiegazioni - Simulazione Esame (Sim. 3)}
\end{center}
\vspace{1cm}

\begin{footnotesize}
\begin{multicols}{2} % Ridotto a 2 colonne per più spazio alle spiegazioni
\textbf{Modulo 1}
\begin{enumerate}
    \item (b) \textit{La scienza si basa su metodi empirici e sperimentali, non su opinioni.}
    \item (b) \textit{Una variabile di disturbo covaria con la VI per errori metodologici.}
    \item (b) \textit{Le categorie ordinate (primaria, secondaria, laurea) definiscono una scala ordinale.}
    \item (b) \textit{Il controllo fornisce una baseline per valutare l'effetto del trattamento.}
    \item (c) \textit{La discussione interpreta i risultati, evidenziando limiti e implicazioni.}
\end{enumerate}
\vspace{0.5cm}
\textbf{Modulo 2}
\begin{enumerate}
    \setcounter{enumi}{5}
    \item (b) \textit{La mediana è meno sensibile agli outlier rispetto alla media.}
    \item (c) \textit{La devianza è la somma dei quadrati degli scarti dalla media.}
    \item (b) \textit{La standardizzazione Z produce media 0 e deviazione standard 1.}
    \item (c) \textit{La probabilità teorica (es. lancio moneta) è un esempio di probabilità classica.}
    \item (c) \textit{$H_0$ è formulata per essere testata, ipotizzando l'assenza di effetto.}
\end{enumerate}
\vspace{0.5cm}
\textbf{Modulo 3}
\begin{enumerate}
    \setcounter{enumi}{10}
    \item (b) \textit{Se p < $\alpha$, si rifiuta $H_0$.}
    \item (a) \textit{La dimensione dell'effetto dei punteggi grezzi è semplicemente la differenza tra le medie.}
    \item (d) \textit{Il software non influenza direttamente la potenza statistica.}
    \item (c) \textit{Il t-test assume indipendenza delle osservazioni all'interno di ciascun gruppo.}
    \item (c) \textit{In caso di violazione dell'omogeneità, si usano test modificati o non parametrici.}
\end{enumerate}
\columnbreak
\textbf{Modulo 4}
\begin{enumerate}
    \setcounter{enumi}{15}
    \item (c) \textit{La varianza entro i gruppi riflette la variabilità casuale non spiegata dalla VI.}
    \item (b) \textit{F vicino a 1 suggerisce che l'ipotesi nulla è vera.}
    \item (b) \textit{Un disegno fattoriale esamina l'effetto di due o più VI (fattori).}
    \item (c) \textit{L'ANOVA RM rimuove la variabilità dovuta alle differenze individuali.}
    \item (b) \textit{$\eta^2$ indica la proporzione di varianza spiegata dalla VI.}
\end{enumerate}
\vspace{0.5cm}
\textbf{Modulo 5}
\begin{enumerate}
    \setcounter{enumi}{20}
    \item (b) \textit{Una linea che sale da sinistra a destra indica una correlazione positiva.}
    \item (b) \textit{Il valore assoluto più alto (in questo caso, |-0.85|) indica la relazione più forte.}
    \item (b) \textit{$r^2 = r * r = 0.60 * 0.60 = 0.36$.}
    \item (b) \textit{Il coefficiente 'b' indica la variazione attesa in Y per ogni unità di X.}
    \item (b) \textit{La regressione è usata per predire Y da X, implicando una relazione causale.}
\end{enumerate}
\vspace{0.5cm}
\textbf{Modulo 6}
\begin{enumerate}
    \setcounter{enumi}{25}
    \item (b) \textit{Il Chi-quadrato per la bontà di adattamento confronta frequenze osservate e attese.}
    \item (c) \textit{Il Chi-quadrato per l'indipendenza valuta l'associazione tra due variabili nominali.}
    \item (b) \textit{Kruskal-Wallis confronta le distribuzioni di tre o più gruppi indipendenti.}
    \item (c) \textit{Wilcoxon è l'alternativa non parametrica al t-test per campioni appaiati.}
    \item (c) \textit{I test non parametrici sono usati quando non è soddisfatto l'assunto di normalità.}
\end{enumerate}
\end{multicols}
\end{footnotesize}
\end{document}