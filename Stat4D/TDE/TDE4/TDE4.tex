\documentclass[12pt, a4paper]{article}
\usepackage[utf8]{inputenc}
\usepackage[T1]{fontenc}
\usepackage[italian]{babel}
\usepackage{amsmath}
\usepackage{amssymb}
\usepackage{geometry}
\geometry{a4paper, left=2.5cm, right=2.5cm, top=2.5cm, bottom=2.5cm}
\usepackage{parskip} % Spazio tra paragrafi invece di indentazione
\usepackage{enumitem} % Per personalizzare elenchi
\usepackage{ FiraSans } % Font più moderno
\usepackage{graphicx} % Necessario per \newpage
\usepackage{multicol} % Per le soluzioni su più colonne
\usepackage{amsfonts} % Per simboli matematici aggiuntivi se necessari

% ----- Impostazioni Liste -----
\setlist[enumerate,1]{label=\arabic*.} % Numeri per domande principali
\setlist[enumerate,2]{label=(\alph*), leftmargin=*} % Lettere per opzioni

\begin{document}

\begin{center}
    \Large\textbf{Simulazione Esame - Tecniche di Ricerca e Analisi dei Dati (Sim. 4)} \\
    \vspace{0.2cm}
    \large\textit{Psicologia} \\
    \vspace{0.5cm}
    \normalsize{Istruzioni: Scegli l'unica risposta corretta per ogni domanda. Le soluzioni sono disponibili nell'ultima pagina.}
\end{center}
\vspace{1cm}

% ----- MODULO 1 -----
\section*{Modulo 1: Introduzione alla Ricerca in Psicologia}
\begin{enumerate}[resume]
    \item Quale dei seguenti è un esempio di variabile misurata su scala nominale?
    \begin{enumerate}
        \item Il tempo di reazione a uno stimolo.
        \item Il genere dei partecipanti (maschio, femmina, altro).
        \item Il punteggio a un test di intelligenza.
        \item L'ordine di arrivo in una gara.
    \end{enumerate}
    \vspace{0.3cm}

    \item La fase del processo di ricerca che implica la definizione precisa di come un costrutto verrà misurato è chiamata:
    \begin{enumerate}
        \item Formulazione dell'ipotesi.
        \item Operazionalizzazione.
        \item Analisi statistica.
        \item Interpretazione dei risultati.
    \end{enumerate}
    \vspace{0.3cm}

    \item Un ricercatore studia l'effetto di un nuovo metodo di insegnamento sul rendimento scolastico. Il "rendimento scolastico" in questo caso è:
    \begin{enumerate}
        \item La variabile indipendente.
        \item La variabile dipendente.
        \item Una variabile di controllo.
        \item Un parametro della popolazione.
    \end{enumerate}
    \vspace{0.3cm}

    \item La differenza principale tra un disegno sperimentale e un disegno quasi-sperimentale risiede:
    \begin{enumerate}
        \item Nel tipo di analisi statistica utilizzata.
        \item Nella presenza o assenza di una variabile dipendente.
        \item Nella capacità del ricercatore di manipolare la variabile indipendente e/o assegnare casualmente i soggetti ai gruppi.
        \item Nel numero di partecipanti coinvolti.
    \end{enumerate}
    \vspace{0.3cm}

    \item L'errore di campionamento è inevitabile quando:
    \begin{enumerate}
        \item Si utilizza uno strumento di misura non valido.
        \item Si studia un'intera popolazione invece di un campione.
        \item Si traggono conclusioni su una popolazione basandosi sui dati di un campione.
        \item Il ricercatore commette errori nel registrare i dati.
    \end{enumerate}
    \vspace{0.3cm}
\end{enumerate}

% ----- MODULO 2 -----
\section*{Modulo 2: Elementi di Statistica Descrittiva e Inferenziale}
\begin{enumerate}[resume]
    \item In una distribuzione con i seguenti punteggi: 5, 7, 7, 8, 10, 12, 7. Qual è la moda?
    \begin{enumerate}
        \item 8
        \item 7.5
        \item 7
        \item 10
    \end{enumerate}
    \vspace{0.3cm}

    \item La radice quadrata della varianza è conosciuta come:
    \begin{enumerate}
        \item Devianza.
        \item Errore standard.
        \item Deviazione standard.
        \item Coefficiente di variazione.
    \end{enumerate}
    \vspace{0.3cm}

    \item Un punteggio Z negativo indica che il punteggio grezzo originale si trova:
    \begin{enumerate}
        \item Esattamente sulla media.
        \item Al di sopra della media.
        \item Al di sotto della media.
        \item All'estremo superiore della distribuzione.
    \end{enumerate}
    \vspace{0.3cm}

    \item La distribuzione normale standard ha:
    \begin{enumerate}
        \item Media 1 e deviazione standard 0.
        \item Media 0 e deviazione standard 1.
        \item Media e deviazione standard variabili a seconda del campione.
        \item Media 100 e deviazione standard 15.
    \end{enumerate}
    \vspace{0.3cm}

    \item Se si conduce un test di ipotesi con $\alpha = 0.01$, significa che si è disposti ad accettare una probabilità dell'1\% di:
    \begin{enumerate}
        \item Commettere un Errore di II Tipo.
        \item Commettere un Errore di I Tipo.
        \item Ottenere un risultato non significativo.
        \item Che l'ipotesi alternativa sia vera.
    \end{enumerate}
    \vspace{0.3cm}
\end{enumerate}

% ----- MODULO 3 -----
\section*{Modulo 3: La Significatività Statistica e il Confronto fra Due Campioni}
\begin{enumerate}[resume]
    \item Accettare l'ipotesi nulla ($H_0$) quando in realtà essa è falsa, è un:
    \begin{enumerate}
        \item Errore di I Tipo.
        \item Errore di II Tipo.
        \item Decisione corretta.
        \item Errore di campionamento.
    \end{enumerate}
    \vspace{0.3cm}

    \item La dimensione dell'effetto standardizzata (es. d di Cohen) è utile perché:
    \begin{enumerate}
        \item Indica se il risultato è statisticamente significativo.
        \item Permette di confrontare la grandezza degli effetti tra studi diversi, anche con scale di misura differenti.
        \item Aumenta la potenza del test statistico.
        \item È più facile da calcolare rispetto alla differenza tra le medie.
    \end{enumerate}
    \vspace{0.3cm}

    \item Aumentando l'ampiezza dell'effetto nella popolazione, a parità di altre condizioni, la potenza statistica di un test:
    \begin{enumerate}
        \item Diminuisce.
        \item Aumenta.
        \item Rimane invariata.
        \item Diventa uguale a 1.
    \end{enumerate}
    \vspace{0.3cm}

    \item Il t-test per campioni indipendenti richiede che la variabile dipendente sia misurata almeno su scala:
    \begin{enumerate}
        \item Nominale.
        \item Ordinale.
        \item A intervalli.
        \item Categoriale.
    \end{enumerate}
    \vspace{0.3cm}

    \item La distribuzione campionaria delle differenze tra le medie, utilizzata nel t-test per campioni indipendenti, ha una media uguale a:
    \begin{enumerate}
        \item 0, se l'ipotesi nulla è vera (nessuna differenza tra le medie delle popolazioni).
        \item 1, indipendentemente dall'ipotesi nulla.
        \item La media del primo campione.
        \item La differenza tra le medie dei due campioni.
    \end{enumerate}
    \vspace{0.3cm}
\end{enumerate}

% ----- MODULO 4 -----
\section*{Modulo 4: L'Analisi Della Varianza (ANOVA)}
\begin{enumerate}[resume]
    \item L'ipotesi nulla nell'ANOVA a una via afferma che:
    \begin{enumerate}
        \item Almeno una delle medie dei gruppi è diversa dalle altre.
        \item Tutte le medie dei gruppi sono diverse tra loro.
        \item Tutte le medie delle popolazioni da cui i campioni sono estratti sono uguali.
        \item Le varianze dei gruppi sono diverse.
    \end{enumerate}
    \vspace{0.3cm}

    \item Un valore elevato del test F nell'ANOVA suggerisce che:
    \begin{enumerate}
        \item La varianza entro i gruppi è maggiore della varianza tra i gruppi.
        \item La varianza tra i gruppi è maggiore della varianza entro i gruppi, indicando possibili differenze tra le medie.
        \item Non ci sono differenze significative tra le medie dei gruppi.
        \item L'ipotesi alternativa è falsa.
    \end{enumerate}
    \vspace{0.3cm}

    \item In un'ANOVA fattoriale, un effetto principale si riferisce a:
    \begin{enumerate}
        \item L'effetto combinato di due o più variabili indipendenti.
        \item L'effetto di una singola variabile indipendente sulla variabile dipendente, ignorando le altre variabili indipendenti.
        \item La differenza tra la media più alta e quella più bassa tra tutti i gruppi.
        \item L'errore residuo del modello.
    \end{enumerate}
    \vspace{0.3cm}

    \item Se il test di Mauchly per la sfericità in un'ANOVA per misure ripetute risulta significativo (es. p < 0.05), cosa implica?
    \begin{enumerate}
        \item L'assunto di sfericità è rispettato e si può procedere con l'ANOVA standard.
        \item L'assunto di sfericità è violato e potrebbero essere necessarie correzioni ai gradi di libertà.
        \item Non ci sono differenze significative tra le misurazioni ripetute.
        \item I dati non sono distribuiti normalmente.
    \end{enumerate}
    \vspace{0.3cm}

    \item L'ANOVA per misure ripetute è vantaggiosa rispetto all'ANOVA per campioni indipendenti perché:
    \begin{enumerate}
        \item Richiede un minor numero di calcoli.
        \item È meno sensibile alla violazione dell'assunto di normalità.
        \item Riduce la varianza d'errore rimuovendo la variabilità dovuta alle differenze individuali tra i soggetti.
        \item Può essere utilizzata solo con variabili dipendenti nominali.
    \end{enumerate}
    \vspace{0.3cm}
\end{enumerate}

% ----- MODULO 5 -----
\section*{Modulo 5: Correlazione e Regressione Lineare}
\begin{enumerate}[resume]
    \item Quale delle seguenti affermazioni descrive una correlazione negativa?
    \begin{enumerate}
        \item All'aumentare di una variabile, aumenta anche l'altra.
        \item All'aumentare di una variabile, l'altra variabile non cambia.
        \item All'aumentare di una variabile, l'altra variabile diminuisce.
        \item Le due variabili non sono correlate.
    \end{enumerate}
    \vspace{0.3cm}

    \item Il coefficiente di correlazione di Pearson (r) può assumere valori compresi tra:
    \begin{enumerate}
        \item 0 e +1.
        \item -1 e +1.
        \item 0 e 100.
        \item Meno infinito e più infinito.
    \end{enumerate}
    \vspace{0.3cm}

    \item Se il coefficiente di determinazione $r^2$ è 0.64, quale percentuale di varianza della variabile dipendente NON è spiegata dalla variabile indipendente?
    \begin{enumerate}
        \item 64\%
        \item 36\%
        \item 80\%
        \item 16\%
    \end{enumerate}
    \vspace{0.3cm}

    \item L'errore standard della stima in un modello di regressione lineare misura:
    \begin{enumerate}
        \item La pendenza della retta di regressione.
        \item La dispersione media dei punti osservati attorno alla retta di regressione.
        \item Il coefficiente di correlazione.
        \item La significatività dell'intercetta.
    \end{enumerate}
    \vspace{0.3cm}

    \item La verifica delle ipotesi nel contesto della regressione lineare serve principalmente a determinare se:
    \begin{enumerate}
        \item Il coefficiente di correlazione r è uguale a 1.
        \item Il coefficiente di regressione $\beta$ (pendenza) nella popolazione è significativamente diverso da zero.
        \item Tutti i punti osservati cadono esattamente sulla retta di regressione.
        \item La variabile indipendente causa la variabile dipendente.
    \end{enumerate}
    \vspace{0.3cm}
\end{enumerate}

% ----- MODULO 6 -----
\section*{Modulo 6: Le Statistiche Non Parametriche}
\begin{enumerate}[resume]
    \item In un test Chi-quadrato per la bontà di adattamento, se le frequenze osservate sono molto diverse da quelle attese, il valore del Chi-quadrato calcolato tenderà ad essere:
    \begin{enumerate}
        \item Vicino a 0.
        \item Elevato.
        \item Negativo.
        \item Uguale al numero di categorie.
    \end{enumerate}
    \vspace{0.3cm}

    \item Il test U di Mann-Whitney è più appropriato del t-test per campioni indipendenti quando:
    \begin{enumerate}
        \item I campioni sono molto grandi e i dati normali.
        \item La variabile dipendente è misurata su scala nominale.
        \item Gli assunti di normalità e/o omoschedasticità del t-test sono palesemente violati, specialmente con campioni piccoli.
        \item Si vogliono confrontare più di due gruppi.
    \end{enumerate}
    \vspace{0.3cm}

    \item Il test di Friedman è l'alternativa non parametrica:
    \begin{enumerate}
        \item Al test Chi-quadrato per l'indipendenza.
        \item All'ANOVA a una via per campioni indipendenti.
        \item Al t-test per campioni appaiati.
        \item All'ANOVA per misure ripetute (con tre o più condizioni).
    \end{enumerate}
    \vspace{0.3cm}

    \item Quale delle seguenti è una caratteristica dei test non parametrici basati sui ranghi?
    \begin{enumerate}
        \item Sono molto sensibili ai valori esatti dei punteggi e agli outlier.
        \item Trasformano i dati grezzi in posizioni ordinali (ranghi), riducendo l'impatto degli outlier.
        \item Richiedono sempre che la distribuzione dei dati sia perfettamente simmetrica.
        \item Possono essere utilizzati solo per variabili nominali.
    \end{enumerate}
    \vspace{0.3cm}

    \item Se si confrontano due gruppi su una variabile dipendente ordinale e si vuole evitare l'assunzione di normalità, quale test è indicato?
    \begin{enumerate}
        \item t-test per campioni indipendenti.
        \item Test U di Mann-Whitney.
        \item ANOVA a una via.
        \item Test di correlazione di Pearson.
    \end{enumerate}
    \vspace{0.3cm}
\end{enumerate}

\newpage
\begin{center}
    \Large\textbf{Griglia delle Soluzioni con Spiegazioni - Simulazione Esame (Sim. 4)}
\end{center}
\vspace{1cm}

\begin{footnotesize}
\begin{multicols}{3}
\textbf{Modulo 1}
\begin{enumerate}
    \item (b) \textit{Il genere è una variabile categoriale senza ordine intrinseco.}
    \item (b) \textit{Operazionalizzare definisce come misurare un costrutto.}
    \item (b) \textit{Il rendimento scolastico è influenzato dal metodo di insegnamento.}
    \item (c) \textit{Nei quasi-esperimenti manca la manipolazione e/o l'assegnazione casuale.}
    \item (c) \textit{L'errore di campionamento si verifica quando si generalizza dal campione alla popolazione.}
\end{enumerate}
\vspace{0.5cm}
\textbf{Modulo 2}
\begin{enumerate}
    \setcounter{enumi}{5}
    \item (c) \textit{La moda è il valore più frequente.}
    \item (c) \textit{La deviazione standard è la radice quadrata della varianza.}
    \item (c) \textit{Un punteggio Z negativo è sotto la media.}
    \item (b) \textit{La distribuzione normale standard ha media 0 e DS 1.}
    \item (b) \textit{$\alpha$ è la probabilità di Errore di I Tipo (falso positivo).}
\end{enumerate}
\vspace{0.5cm}
\textbf{Modulo 3}
\begin{enumerate}
    \setcounter{enumi}{10}
    \item (b) \textit{Accettare H0 falsa è un Errore di II Tipo (falso negativo).}
    \item (b) \textit{La dimensione dell'effetto standardizzata permette confronti tra studi.}
    \item (b) \textit{Aumentare l'effetto aumenta la potenza.}
    \item (c) \textit{Il t-test richiede scala a intervalli o rapporti.}
    \item (a) \textit{La media della distribuzione campionaria è 0 se H0 è vera.}
\end{enumerate}
\columnbreak
\textbf{Modulo 4}
\begin{enumerate}
    \setcounter{enumi}{15}
    \item (c) \textit{H0 nell'ANOVA: tutte le medie delle popolazioni sono uguali.}
    \item (b) \textit{F elevato indica che la varianza "tra" è maggiore di quella "entro".}
    \item (b) \textit{L'effetto principale è l'effetto di una VI ignorando le altre.}
    \item (b) \textit{Se Mauchly è significativo, l'assunto di sfericità è violato.}
    \item (c) \textit{L'ANOVA RM riduce la varianza d'errore.}
\end{enumerate}
\vspace{0.5cm}
\textbf{Modulo 5}
\begin{enumerate}
    \setcounter{enumi}{20}
    \item (c) \textit{Correlazione negativa: X aumenta, Y diminuisce.}
    \item (b) \textit{r varia tra -1 e +1.}
    \item (b) \textit{Se $r^2$=0.64, allora 1-0.64=0.36 (36\%) non è spiegata.}
    \item (b) \textit{L'errore standard della stima misura la dispersione attorno alla retta.}
    \item (b) \textit{Si verifica se $\beta$ è significativamente diverso da zero.}
\end{enumerate}
\vspace{0.5cm}
\textbf{Modulo 6}
\begin{enumerate}
    \setcounter{enumi}{25}
    \item (b) \textit{Chi-quadrato elevato: frequenze osservate diverse da quelle attese.}
    \item (c) \textit{Mann-Whitney si usa se non ci sono normalità e omoschedasticità.}
    \item (d) \textit{Friedman è l'alternativa non parametrica all'ANOVA RM.}
    \item (b) \textit{I test sui ranghi riducono l'impatto degli outlier.}
    \item (b) \textit{Mann-Whitney si usa con VD ordinale e senza normalità.}
\end{enumerate}
\end{multicols}
\end{footnotesize}

\end{document}