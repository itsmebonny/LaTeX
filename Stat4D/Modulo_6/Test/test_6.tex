\documentclass[12pt, a4paper]{article}
\usepackage[utf8]{inputenc}
\usepackage[T1]{fontenc}
\usepackage[italian]{babel}
\usepackage{amsmath}
\usepackage{amssymb}
\usepackage{geometry}
\geometry{a4paper, left=2.5cm, right=2.5cm, top=2.5cm, bottom=2.5cm}
\usepackage{parskip} % Spazio tra paragrafi invece di indentazione
\usepackage{enumitem} % Per personalizzare elenchi
\usepackage{ FiraSans } % Font più moderno
\usepackage{graphicx} % Necessario per \newpage
\usepackage{bm} % Per simboli grassetto eventuali
\usepackage{array} % Per tabelle più avanzate (es. m{} per allineamento verticale)
\usepackage{longtable} % Per tabelle che possono spezzarsi su più pagine
\usepackage{booktabs} % Per tabelle con linee professionali

% ----- Impostazioni Liste -----
\setlist[enumerate,1]{label=\arabic*)} % Numeri per domande principali
\setlist[enumerate,2]{label=(\alph*), leftmargin=*} % Lettere per opzioni

% ----- Macro per simboli comuni (Modulo 6) -----
\newcommand{\chisq}{\chi^2} % Chi-quadrato
\newcommand{\Hnull}{H_0} % Ipotesi nulla
\newcommand{\Halt}{H_1} % Ipotesi alternativa
\newcommand{\df}{df} % gradi di libertà (o gl)
\newcommand{\fo}{f_o} % frequenze osservate
\newcommand{\fa}{f_a} % frequenze attese (o f_e)
\newcommand{\phiCoeff}{\phi} % Coefficiente Phi
\newcommand{\phiCramer}{\phi_C} % Phi di Cramér (o V di Cramér)
\newcommand{\Umann}{U} % U di Mann-Whitney
\newcommand{\Hkruskal}{H} % H di Kruskal-Wallis
\newcommand{\FrFried}{F_r} % Fr di Friedman (o \chi^2_r)
\newcommand{\Ntot}{N_{\text{tot}}} % N totale

\begin{document}

\begin{center}
    \Large\textbf{Test di Autovalutazione Aggiuntivo - Modulo 6} \\
    \vspace{0.2cm}
    \large\textit{Tecniche di Ricerca e Analisi dei Dati} \\
    \vspace{0.5cm}
    \normalsize{Istruzioni: Scegli l'unica risposta corretta per ogni domanda. Le soluzioni sono disponibili nell'ultima pagina.}
\end{center}
\vspace{1cm}

\begin{enumerate} % Inizio elenco domande principali
    \item I test non parametrici sono particolarmente utili quando:
    \begin{enumerate} % Inizio opzioni domanda 1
        \item I campioni sono molto grandi e i dati perfettamente normali.
        \item Si lavora con variabili misurate su scala nominale o ordinale.
        \item Si vuole stimare con precisione la media e la varianza della popolazione.
    \end{enumerate}
    \vspace{0.5cm}

    \item Quale delle seguenti affermazioni descrive meglio un vantaggio dei test non parametrici?
    \begin{enumerate}
        \item Hanno sempre una potenza statistica superiore ai test parametrici.
        \item Richiedono meno assunzioni sulla distribuzione dei dati nella popolazione.
        \item Sono più adatti per analizzare interazioni complesse in disegni fattoriali.
    \end{enumerate}
    \vspace{0.5cm}

    \item Il termine "distribution-free" riferito ai test non parametrici significa che:
    \begin{enumerate}
        \item Non utilizzano alcuna distribuzione campionaria per la verifica delle ipotesi.
        \item Non fanno ipotesi specifiche sulla forma della distribuzione della popolazione (es. normalità).
        \item I risultati del test sono sempre distribuiti uniformemente.
    \end{enumerate}
    \vspace{0.5cm}

    \item Nel test Chi-quadrato per la bontà di adattamento, l'ipotesi nulla afferma tipicamente che:
    \begin{enumerate}
        \item Le frequenze osservate sono significativamente diverse da quelle attese.
        \item Esiste un'associazione significativa tra due variabili categoriali.
        \item Le frequenze delle categorie nel campione si conformano a una distribuzione teorica attesa.
    \end{enumerate}
    \vspace{0.5cm}

    \item Per calcolare le frequenze attese ($\fa$) nel test Chi-quadrato per la bontà di adattamento, se si ipotizza che tre categorie siano equiprobabili in un campione di 90 persone, quale sarà $\fa$ per ciascuna categoria?
    \begin{enumerate}
        \item 90
        \item 30
        \item Non si può determinare senza conoscere le frequenze osservate.
    \end{enumerate}
    \vspace{0.5cm}

    \item Il test Chi-quadrato per l'indipendenza è usato per:
    \begin{enumerate}
        \item Verificare se un campione proviene da una popolazione con una distribuzione di frequenze nota.
        \item Valutare se esiste una relazione (associazione) tra due variabili qualitative.
        \item Confrontare le medie di due o più gruppi.
    \end{enumerate}
    \vspace{0.5cm}

    \item In una tabella di contingenza 3x4 (3 righe, 4 colonne) usata per un test Chi-quadrato per l'indipendenza, quanti gradi di libertà avrà il test?
    \begin{enumerate}
        \item 2
        \item 6
        \item 12
    \end{enumerate}
    \vspace{0.5cm}

    \item Se il risultato di un test Chi-quadrato per l'indipendenza è statisticamente significativo, cosa possiamo concludere?
    \begin{enumerate}
        \item Le due variabili sono sicuramente causa l'una dell'altra.
        \item Le due variabili sono associate (non indipendenti) nella popolazione.
        \item Le medie delle categorie sono significativamente diverse.
    \end{enumerate}
    \vspace{0.5cm}

    \item Il coefficiente Phi ($\phiCoeff$) è una misura di dimensione dell'effetto per il test Chi-quadrato per l'indipendenza. Quando è appropriato usarlo?
    \begin{enumerate}
        \item Per qualsiasi tabella di contingenza, indipendentemente dalle dimensioni.
        \item Specificamente per tabelle di contingenza 2x2.
        \item Solo quando il test Chi-quadrato non è significativo.
    \end{enumerate}
    \vspace{0.5cm}

    \item Il test U di Mann-Whitney è considerato l'alternativa non parametrica a quale test parametrico?
    \begin{enumerate}
        \item ANOVA a una via.
        \item t-test per campioni appaiati.
        \item t-test per campioni indipendenti.
    \end{enumerate}
    \vspace{0.5cm}

    \pagebreak % Inizia le domande successive su una nuova pagina se necessario

    \item Quale indice di tendenza centrale è tipicamente confrontato dal test U di Mann-Whitney quando si formulano le ipotesi?
    \begin{enumerate}
        \item La media.
        \item La moda.
        \item La mediana.
    \end{enumerate}
    \vspace{0.5cm}

    \item Il test H di Kruskal-Wallis è utilizzato per confrontare:
    \begin{enumerate}
        \item Le mediane di due campioni indipendenti.
        \item Le mediane di tre o più campioni indipendenti.
        \item Le frequenze di tre o più categorie.
    \end{enumerate}
    \vspace{0.5cm}

    \item Se un ricercatore vuole confrontare i livelli di soddisfazione (misurati su scala ordinale) tra quattro gruppi diversi di dipendenti, quale test non parametrico sarebbe più indicato?
    \begin{enumerate}
        \item Test U di Mann-Whitney.
        \item Test di Wilcoxon.
        \item Test H di Kruskal-Wallis.
    \end{enumerate}
    \vspace{0.5cm}

    \item Il metodo del bootstrap è una tecnica che permette di:
    \begin{enumerate}
        \item Aumentare artificialmente la numerosità del campione originale.
        \item Stimare la distribuzione campionaria di una statistica ricampionando dai dati originali.
        \item Trasformare sempre dati non normali in dati normali.
    \end{enumerate}
    \vspace{0.5cm}

    \item Nel test dei ranghi con segno di Wilcoxon, su cosa si basa principalmente il calcolo della statistica test?
    \begin{enumerate}
        \item Sulla differenza tra le medie delle due misurazioni.
        \item Sui ranghi delle differenze assolute tra coppie di osservazioni, considerando il segno.
        \item Sulle frequenze osservate in ciascuna delle due condizioni.
    \end{enumerate}
    \vspace{0.5cm}

    \item Un ricercatore misura l'ansia in un gruppo di pazienti prima (T1), subito dopo (T2) e a un mese dalla fine (T3) di una terapia. Se i dati non rispettano gli assunti per un'ANOVA RM, quale test non parametrico è più adatto?
    \begin{enumerate}
        \item Test H di Kruskal-Wallis.
        \item Test Chi-quadrato per l'indipendenza.
        \item Test di Friedman.
    \end{enumerate}
    \vspace{0.5cm}

    \item Qual è uno dei principali limiti dei test non parametrici rispetto ai test parametrici, quando questi ultimi sono applicabili?
    \begin{enumerate}
        \item Sono più difficili da calcolare manually.
        \item Hanno generalmente una minore potenza nel rilevare un effetto.
        \item Non permettono di calcolare la dimensione dell'effetto.
    \end{enumerate}
    \vspace{0.5cm}

    \item La trasformazione dei dati in ranghi, comune in molti test non parametrici, ha lo scopo di:
    \begin{enumerate}
        \item Rendere i dati più vicini a una distribuzione normale.
        \item Ridurre l'impatto di valori anomali (outlier) e della forma specifica della distribuzione.
        \item Aumentare la varianza dei dati per rendere gli effetti più evidenti.
    \end{enumerate}
    \vspace{0.5cm}

    \item Se un test non parametrico produce un risultato statisticamente significativo (es. p < 0.05), cosa significa?
    \begin{enumerate}
        \item L'effetto osservato è sicuramente di grande importanza pratica.
        \item È improbabile che l'effetto osservato nel campione sia dovuto solo al caso, assumendo l'ipotesi nulla vera.
        \item Il test parametrico corrispondente avrebbe dato lo stesso risultato.
    \end{enumerate}
    \vspace{0.5cm}

    \item Quando si sceglie tra un test parametrico e uno non parametrico per confrontare due gruppi, quale fattore è MENO determinante se il campione è molto piccolo (es. n < 10 per gruppo)?
    \begin{enumerate}
        \item La scala di misura della variabile dipendente.
        \item La robustezza del test parametrico alla violazione dell'assunto di normalità.
        \item La dimensione dell'effetto attesa.
    \end{enumerate}
    \vspace{0.5cm}

\end{enumerate} % Fine elenco domande principali

\newpage % Inizia una nuova pagina per le soluzioni

\begin{center}
    \Large\textbf{Griglia delle Soluzioni} \\
    \vspace{0.5cm}
    \normalsize{(Test di Autovalutazione Aggiuntivo - Modulo 6)}
\end{center}
\vspace{1cm}

\begin{enumerate}[leftmargin=*, label=\arabic*.]
    \item (b) Si lavora con variabili misurate su scala nominale o ordinale. (I non parametrici sono ideali per dati categoriali/ordinali o quantitativi non normali).
    \item (b) Richiedono meno assunzioni sulla distribuzione dei dati nella popolazione. (Questo li rende più flessibili, soprattutto con campioni piccoli o dati "difficili").
    \item (b) Non fanno ipotesi specifiche sulla forma della distribuzione della popolazione (es. normalità). (Sono "liberi dalla distribuzione").
    \item (c) Le frequenze delle categorie nel campione si conformano a una distribuzione teorica attesa. (Verifica se il campione "si adatta" al modello teorico).
    \item (b) 30 (Con equiprobabilità, la frequenza attesa per ogni categoria è il totale diviso il numero di categorie: $\fa = N/k = 90/3 = 30$).
    \item (b) Valutare se esiste una relazione (associazione) tra due variabili qualitative. (Verifica se le categorie di una variabile variano al variare delle categorie dell'altra).
    \item (b) 6 (I gradi di libertà si calcolano come $(righe - 1) \times (colonne - 1) = (3-1) \times (4-1) = 2 \times 3 = 6$).
    \item (b) Le due variabili sono associate (non indipendenti) nella popolazione. (Si rifiuta l'ipotesi di indipendenza; la significatività non implica causalità).
    \item (b) Specificamente per tabelle di contingenza 2x2. (Per tabelle più grandi si usa la V di Cramér o $\phiCramer$).
    \item (c) t-test per campioni indipendenti. (Entrambi confrontano due gruppi indipendenti, ma il Mann-Whitney non assume normalità e usa ranghi/mediane).
    \item (c) La mediana. (I test non parametrici basati sui ranghi sono sensibili alla posizione centrale, rappresentata dalla mediana).
    \item (b) Le mediane di tre o più campioni indipendenti. (È l'estensione non parametrica dell'ANOVA a una via).
    \item (c) Test H di Kruskal-Wallis. (Adatto per confrontare più di due gruppi indipendenti quando la VD è ordinale).
    \item (b) Stimare la distribuzione campionaria di una statistica ricampionando dai dati originali. (È una tecnica di simulazione basata sui dati osservati).
    \item (b) Sui ranghi delle differenze assolute tra coppie di osservazioni, considerando il segno. (Combina informazione sull'ordine delle differenze e sulla loro direzione).
    \item (c) Test di Friedman. (È l'alternativa non parametrica all'ANOVA per misure ripetute con più di due misurazioni).
    \item (b) Hanno generalmente una minore potenza nel rilevare un effetto. (Quando gli assunti parametrici sono soddisfatti, i test parametrici sono più efficienti).
    \item (b) Ridurre l'impatto di valori anomali (outlier) e della forma specifica della distribuzione. (I ranghi preservano l'ordine ma non la magnitudo esatta delle distanze).
    \item (b) È improbabile che l'effetto osservato nel campione sia dovuto solo al caso, assumendo l'ipotesi nulla vera. (Questa è l'interpretazione standard del p-value).
    \item (b) La robustezza del test parametrico alla violazione dell'assunto di normalità. (Con campioni molto piccoli, la robustezza è bassa; quindi, la violazione della normalità diventa un problema più serio e spinge verso i test non parametrici, rendendo la robustezza un fattore meno determinante nella scelta *tra* i due).
\end{enumerate}

\end{document}
% ----- FINE DOCUMENTO -----