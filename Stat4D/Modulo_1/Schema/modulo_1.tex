\documentclass[12pt, a4paper]{article}
\usepackage[utf8]{inputenc}
\usepackage[T1]{fontenc}
\usepackage[italian]{babel}
\usepackage{amsmath}
\usepackage{amssymb} % Per spunte e simboli
\usepackage{graphicx} % Per includere immagini (se servisse)
\usepackage{geometry}
\geometry{a4paper, left=2.5cm, right=2.5cm, top=2.5cm, bottom=2.5cm} % Margini leggermente più ampi
\usepackage{parskip}
\usepackage{enumitem}
\usepackage{tikz} % Per eventuali diagrammi futuri semplici
\usepackage{framed} % Per i box delle domande
\usepackage{ FiraSans } % Un font più moderno e leggibile
\usepackage{xcolor} % Per colori

% ----- Definizioni Colori e Stili -----
\definecolor{boxbgcolor}{rgb}{0.95, 0.95, 1.0} % Azzurrino chiaro per i box
\definecolor{boxtitlecolor}{rgb}{0.1, 0.1, 0.6} % Blu scuro per titoli box
\newenvironment{reflectionbox}{%
    \begin{framed}\par\medskip\noindent
    \textbf{\color{boxtitlecolor}Domande} \par
    \begin{itemize}[leftmargin=*, label=$\blacktriangleright$]
}{%
    \end{itemize}\par\medskip
    \end{framed}
}

\setlist{nosep}
\renewcommand{\labelitemi}{$\bullet$}

% ----- INIZIO DOCUMENTO -----
\begin{document}

\begin{center}
    \Large\textbf{Modulo 1: Introduzione alla Ricerca e all'Analisi Dati} \\
    \vspace{0.5cm}
    \large\textit{(Concetti chiave e spunti dal test di autovalutazione)}
\end{center}

\section*{1. Introduzione alla Ricerca in Psicologia}
\begin{itemize}
    \item \textbf{Oggetto:} Studio del \textbf{comportamento} umano (come varia, cause, previsione).
    \item \textbf{Tipi Principali:}
        \begin{itemize}
            \item \textbf{Ricerca di Base (Pura):} Aumentare la \textbf{conoscenza teorica}. (Es: Come funziona la memoria?)
            \item \textbf{Ricerca Applicata:} Trovare \textbf{soluzioni pratiche} a problemi. (Es: Quale terapia è più efficace?)
        \end{itemize}
    \item \textbf{Metodologia:} Insieme delle regole e procedure usate.
    \item \textbf{Carattere Scientifico:} Basato su \textbf{dati empirici} (osservabili, misurabili) e \textbf{metodo sperimentale} (controllo ipotesi).
\end{itemize}

\begin{reflectionbox}
    \item Qual è l'obiettivo principale della \textbf{ricerca di base}? (Vedere Domanda 16)
\end{reflectionbox}

\section*{2. Il Processo di Ricerca}
Percorso \textbf{circolare} standard (vedi Fig 1.1, pag 5):
\begin{enumerate}
    \item \textbf{Identificazione del problema:} Trovare la domanda di ricerca, basata su teorie o lacune esistenti. (\textbf{Prima fase!})
    \item \textbf{Pianificazione del disegno:} Come condurre lo studio (chi, cosa, come misurare).
    \item \textbf{Osservazioni/Raccolta Dati:} Mettere in atto il piano e raccogliere i dati.
    \item \textbf{Analisi dei Dati:} Usare la statistica per elaborare i dati.
    \item \textbf{Interpretazione dei Dati:} Dare un significato ai risultati (supportano l'ipotesi?).
    \item \textbf{Comunicazione dei Risultati:} Condividere le scoperte (articoli, report).
\end{enumerate}

\begin{reflectionbox}
    \item Qual è la \textbf{primissima fase} del processo di ricerca? (Vedere Domanda 1)
    \item Come può essere descritta la relazione tra le varie fasi (specialmente tra campione e popolazione)? (Vedere Domanda 19)
\end{reflectionbox}

\section*{3. Popolazione, Campione, Parametri, Indici}
\begin{itemize}
    \item \textbf{Popolazione (N):} \textbf{TUTTO} l'insieme di individui/elementi che interessano al ricercatore. (Es: Tutti gli studenti UniCusano). Spesso troppo grande da studiare interamente (\textit{Indagine Totale}).
    \item \textbf{Campione (n):} Un \textbf{sottoinsieme selezionato} dalla popolazione, che viene effettivamente studiato (\textit{Indagine Campionaria}). (Es: 100 studenti UniCusano).
    \item \textbf{Rappresentatività:} Il campione deve "assomigliare" alla popolazione per poter generalizzare i risultati. Si ottiene tramite estrazione \textbf{casuale}.
    \item \textbf{Parametro:} Valore numerico che descrive una caratteristica della \textbf{Popolazione} (Es: Età media di \textit{tutti} gli studenti UniCusano). \textit{Spesso è l'obiettivo della ricerca, in quanto il suo valore non è noto.}
    \item \textbf{Indice Statistico (o Statistica):} Valore numerico calcolato sul \textbf{Campione} (Es: Età media dei \textit{100 studenti} studiati). \textit{Si calcola per stimare il parametro.}
    \item \textbf{Errore di Campionamento:} Differenza naturale e inevitabile tra indice statistico (campione) e parametro (popolazione). \textbf{Non è uno sbaglio!} Si tratta di una variazione casuale dovuta al fatto di studiare solo una parte. Si verifica nelle \textit{indagini campionarie}.
    \item \textbf{Modalità di Estrazione Campione:}
        \begin{itemize}
            \item \textit{Casuale Semplice:} Ogni individuo ha la stessa probabilità di essere scelto.
            \item \textit{Casuale Stratificata:} Popolazione divisa in sottogruppi (strati, es. per età), poi estrazione casuale da ogni strato.
        \end{itemize}
\end{itemize}

\begin{reflectionbox}
    \item Un indice statistico (es. la media del campione) a cosa si riferisce? (Vedere Domanda 3)
    \item Qual è la relazione corretta tra parametri e indici statistici? (Vedere Domanda 4)
    \item Per essere rappresentativo, come deve essere scelto un campione? (Vedere Domanda 5)
    \item L'errore di campionamento è tipico di quale tipo di indagine? (Vedere Domanda 6)
    \item Le indagini che studiano l'intera popolazione come si chiamano? (Vedere Domanda 11)
\end{reflectionbox}

\section*{4. Statistica Descrittiva e Inferenziale}
\begin{itemize}
    \item \textbf{Statistica Descrittiva:}
        \begin{itemize}
            \item \textbf{Scopo:} \textbf{Riassumere, organizzare, descrivere} i dati raccolti (del campione o popolazione). Rendere i dati comprensibili.
            \item \textbf{Strumenti:} Tabelle, \textbf{grafici}, indici numerici (media, mediana, moda, deviazione standard, ecc.).
        \end{itemize}
    \item \textbf{Statistica Inferenziale:}
        \begin{itemize}
            \item \textbf{Scopo:} Usare i dati del campione per fare \textbf{inferenze} (deduzioni, generalizzazioni) sulla \textbf{popolazione}.
            \item \textbf{Strumenti:} Stima dei parametri, verifica delle ipotesi.
            \item \textbf{Logica:} Aiuta a capire se le differenze/relazioni osservate nel campione sono "reali" (generalizzabili alla popolazione) o dovute solo all'errore di campionamento (caso).
        \end{itemize}
\end{itemize}

\begin{reflectionbox}
    \item Quale strumento \textbf{non} è tipicamente usato per fare inferenze sulla popolazione? (Vedere Domanda 20)
    \item Qual è l'utilità principale della statistica \textbf{inferenziale}? (Vedere Domanda 29)
\end{reflectionbox}

\section*{5. Variabili e Costrutti}
\begin{itemize}
    \item \textbf{Variabile:} Caratteristica o condizione che \textbf{varia} (assume valori diversi) tra individui o situazioni. (Es: Età, altezza, voto, sesso, tipo di trattamento).
    \item \textbf{Costrutto:} Concetto \textbf{teorico/astratto}, non direttamente osservabile (Es: Intelligenza, ansia, motivazione).
    \item \textbf{Operazionalizzazione:} Processo fondamentale per definire un costrutto in termini di \textbf{comportamenti/indicatori osservabili e misurabili} (variabili). Trasforma l'astratto in concreto. (Es: Operazionalizzo l'"ansia" misurando il battito cardiaco).
\end{itemize}

\begin{reflectionbox}
    \item L'"età anagrafica" è una variabile o un costrutto? (Vedere Domanda 9)
    \item Per rendere misurabile un costrutto (es. "felicità"), a cosa devo riferirmi? (Vedere Domanda 24)
\end{reflectionbox}

\section*{6. Come Misuriamo: Le Scale di Misura}
\begin{itemize}
    \item \textbf{Misurare significa:} Attribuire etichette (parole o numeri) a caratteristiche di persone o eventi, seguendo regole precise. Il tipo di etichetta usata (la "scala") determina cosa possiamo dire e quali calcoli possiamo fare sensatamente.
    \item \textbf{Tipi di Scale (dal più semplice al più completo):}
        \begin{itemize}
            \item \textbf{Scala Nominale (Qualitativa - Etichette):}
                \subitem \textit{Cosa fa:} Classifica elementi in categorie distinte, senza alcun ordine intrinseco. Serve solo a \textbf{distinguere} i gruppi.
                \subitem \textit{Operazioni logiche permesse:} Possiamo solo verificare se due elementi sono \textbf{uguali (=)} o \textbf{diversi ($\neq$)} (cioè, se appartengono alla stessa categoria o a categorie diverse). Non ha senso chiedere se "blu" è maggiore di "rosso".
                \subitem \textit{Esempi comuni:} Genere (Maschio, Femmina, Altro), Stato Civile (Celibe/Nubile, Coniugato/a, Divorziato/a), Colore degli occhi (Azzurro, Marrone, Verde).

            \item \textbf{Scala Ordinale (Qualitativa - Classifica):}
                \subitem \textit{Cosa fa:} Mette le categorie in un \textbf{ordine logico} (es. dal più piccolo al più grande, dal meno al più preferito), ma non ci dice \textit{quanto} sia la distanza tra una posizione e l'altra. Sappiamo che "oro" è meglio di "argento", ma non di quanto.
                \subitem \textit{Operazioni logiche permesse:} Oltre a uguale/diverso, possiamo stabilire un \textbf{ordine (maggiore di >, minore di <)}.
                \subitem \textit{Esempi comuni:} Titolo di studio (Licenza media < Diploma < Laurea), Livello di soddisfazione (Poco < Abbastanza < Molto), Classifica in una competizione (1°, 2°, 3°).

            \item \textbf{Scala a Intervalli (Quantitativa - Numeri con distanze uguali):}
                \subitem \textit{Cosa fa:} Usa numeri reali dove la \textbf{distanza} (l'intervallo) tra un numero e il successivo è costante e significativa. Possiamo quantificare le differenze tra i valori.
                \subitem \textit{Operazioni logiche/matematiche permesse:} Oltre a uguale/diverso e ordine, possiamo fare \textbf{somme e sottrazioni sulle differenze} tra i valori (es. la differenza tra 10 e 15 è uguale alla differenza tra 20 e 25).
                \subitem \textit{Punto chiave:} Lo \textbf{zero è convenzionale} (arbitrario), non indica l'assenza reale della caratteristica misurata. Esempio classico: 0° Celsius non significa "assenza di temperatura", è solo un punto di riferimento (il punto di congelamento dell'acqua). La differenza tra -5°C e +5°C è la stessa (10 gradi) della differenza tra +10°C e +20°C, mostrando che gli intervalli sono significativi. Tuttavia, non possiamo dire che +10°C sia "il doppio più caldo" di +5°C, perché lo zero non è assoluto (cosa significherebbe "il doppio più caldo" di -5°C?). Per questo, non possiamo fare rapporti sensati.
                \subitem \textit{Esempi comuni:} Temperatura in gradi Celsius o Fahrenheit, Punteggio in un test psicologico standardizzato (es. QI), Anno del calendario.

            \item \textbf{Scala a Rapporti (Quantitativa - Numeri con zero vero):}
                \subitem \textit{Cosa fa:} È la scala più completa. Ha tutte le proprietà della scala a intervalli (numeri, ordine, intervalli uguali), ma in più possiede uno \textbf{zero assoluto}, che indica veramente l'"assenza" della quantità misurata.
                \subitem \textit{Operazioni logiche/matematiche permesse:} Consente \textbf{tutte le operazioni}: uguale/diverso, ordine, somma/sottrazione, e anche \textbf{moltiplicazione e divisione (rapporti)}. Qui ha senso dire che un valore è il doppio, la metà, un terzo di un altro.
                \subitem \textit{Esempi comuni:} Altezza (0 cm = nessuna altezza), Peso (0 kg = nessun peso), Età (0 anni = nascita), Numero di figli (0 = nessun figlio), Tempo di reazione (0 secondi = istantaneo/nessun tempo), Reddito (\$0 = nessun reddito).
        \end{itemize}
\end{itemize}

\begin{reflectionbox}
    \item Quale scala ha solo le proprietà di "differenza" e "ordine"? (Vedere Domanda 10)
    \item La variabile "colore degli occhi" su che tipo di scala si misura? (Vedere Domanda 23)
    \item In quale scala lo zero significa veramente "quantità nulla"? (Vedere Domanda 25)
    \item La scala a intervalli permette di dire che un valore è "il doppio" di un altro? (Vedere Domanda 30)
\end{reflectionbox}

\section*{7. Classificazione delle Variabili}
Oltre alla scala di misura, le variabili si classificano per:
\begin{itemize}
    \item \textbf{Precisione/Natura:}
        \begin{itemize}
            \item \textbf{Discrete:} Valori separati, "a salti" (spesso numeri interi, ma anche categorie). Non ci sono valori intermedi possibili. (Es: Numero figli, Numero esami superati, Sesso, Professione). Possono essere sia Qualitative (nominale, ordinale) che Quantitative (intervalli, rapporti).
            \item \textbf{Continue:} Possono assumere qualsiasi valore in un intervallo (anche con decimali), dipende solo dalla precisione dello strumento. (Es: Altezza, Peso, Tempo). Sono Quantitative (intervalli, rapporti).
        \end{itemize}
    \item \textbf{Ruolo nella Ricerca:}
        \begin{itemize}
            \item \textbf{Variabile Indipendente (VI):} Quella che il ricercatore \textbf{manipola} (negli esperimenti) o usa per definire i gruppi da confrontare. È la presunta \textbf{causa}. Può essere sia qualitativa (es. tipo di terapia) che quantitativa (es. dose di farmaco).
            \item \textbf{Variabile Dipendente (VD):} Quella che viene \textbf{osservata e misurata} per vedere l'effetto della VI. È il presunto \textbf{effetto}. Solitamente quantitativa.
            \item \textbf{Variabili Confuse/Confondenti Operazionali:} Non previste, ma \textbf{intrinsecamente legate} alla VI o alla sua operazionalizzazione (covariano con essa). Creano ambiguità nell'interpretazione.
            \item \textbf{Variabili di Disturbo/Confondenti Procedurali:} Estranee alla VI, ma \textbf{covariano} con essa per errori metodologici (artefatti), influenzando la VD. Devono essere controllate.
        \end{itemize}
\end{itemize}

\begin{reflectionbox}
    \item La Variabile Indipendente può essere solo quantitativa? (Vedere Domanda 2)
    \item Le variabili limitate a numeri interi (come "numero di gatti posseduti") come si chiamano? (Vedere Domanda 13)
    \item Le variabili discrete possono essere sia qualitative che quantitative? (Vedere Domanda 26)
    \item Le Variabili Indipendenti sono quelle misurate o quelle manipolate? (Vedere Domanda 14)
    \item In uno studio sull'effetto di una nuova dieta (VI) sul peso (VD), l'attività fisica non controllata che differenza tra i gruppi potrebbe essere una variabile...? (Vedere Domanda 15)
    \item Se studio l'effetto di un farmaco sulla pressione sanguigna, cos'è la pressione sanguigna? E il farmaco? (Vedere Domanda 21)
\end{reflectionbox}

\section*{8. I Disegni di Ricerca in Psicologia}
Classificati in base al \textbf{grado di controllo} del ricercatore:
\begin{itemize}
    \item \textbf{Metodi Descrittivi:} Descrivono le variabili così come sono. Controllo minimo/nullo. (Es: Sondaggi, osservazione naturalistica).
    \item \textbf{Metodo Correlazionale:} Misura due (o più) variabili per vedere se esiste una \textbf{relazione} (covariazione) tra loro. \textbf{Non stabilisce causa-effetto}. (Es: C'è relazione tra ore di studio e voti?).
    \item \textbf{Metodo Sperimentale:} Obiettivo: stabilire relazioni \textbf{causa-effetto}. Massimo controllo. Caratteristiche chiave:
        \begin{itemize}
            \item \textbf{Manipolazione} della VI (il ricercatore crea i diversi livelli/condizioni).
            \item \textbf{Controllo} delle variabili estranee (disturbo, concorrenti, ambientali) per evitare confusioni. Tecniche: \textbf{assegnazione casuale} dei partecipanti alle condizioni, mantenere costanti le condizioni ambientali.
        \end{itemize}
    \item \textbf{Metodo Quasi-Sperimentale:} Simile all'esperimento (confronto tra gruppi), ma il ricercatore \textbf{non può manipolare} pienamente la VI (es. è una caratteristica pre-esistente come sesso, età) o non può fare l'assegnazione casuale. Controllo minore, più difficile stabilire causa-effetto certa.
    \item \textbf{Disegni Specifici:}
        \begin{itemize}
            \item \textbf{Between-subjects (o Tra Gruppi):} Gruppi diversi di partecipanti sono assegnati a condizioni diverse.
            \item \textbf{Within-subjects (o Entro Gruppi / Misure Ripetute):} Gli stessi partecipanti sono sottoposti a tutte le condizioni (o misurati più volte, es. prima e dopo un trattamento).
        \end{itemize}
    \item \textbf{Condizione di Controllo:} Gruppo/condizione che non riceve il trattamento sperimentale (o riceve un placebo), serve come baseline per il confronto.
\end{itemize}

\begin{reflectionbox}
    \item Quale disegno di ricerca offre il maggior livello di controllo? (Vedere Domanda 8)
    \item Quali sono le due caratteristiche fondamentali di un vero esperimento? (Vedere Domanda 17)
    \item Quale tecnica aiuta a controllare le differenze individuali tra partecipanti (variabili concorrenti) in un esperimento? (Vedere Domanda 18)
    \item Se misuro l'ansia e la depressione in un gruppo per vedere se sono associate, che studio sto facendo? (Vedere Domanda 27)
    \item Se testo la memoria degli stessi anziani prima e dopo un training cognitivo, che disegno sto usando (between o within)? (Vedere Domanda 22)
\end{reflectionbox}

\section*{9. L'Errore di Misurazione}
Ogni misura contiene una parte "vera" e una parte di errore: $X_i = V_i + e_i$. L'errore \textbf{non è uno sbaglio}, ma imprecisione intrinseca.
\begin{itemize}
    \item \textbf{Errore Sistematico:} Costante, influenza tutte le misure nello \textbf{stesso modo} (es. bilancia starata che aggiunge sempre 0.5 kg). Legato a difetti dello strumento/metodo. Riducibile identificando la causa.
    \item \textbf{Errore Casuale (o Accidentale):} Imprevedibile, varia da misura a misura (a volte in eccesso, a volte in difetto). Dovuto a fattori incontrollabili (ambientali, fluttuazioni attenzione). Non eliminabile totalmente, ma tende a compensarsi su molte misure. La statistica inferenziale ne tiene conto.
\end{itemize}

\begin{reflectionbox}
    \item È possibile eliminare totalmente l'errore di misurazione? (Vedere Domanda 28)
\end{reflectionbox}

\section*{10. La Presentazione della Ricerca}
Comunicare i risultati è fondamentale per il progresso scientifico (replicabilità, confronto). Struttura tipica di un articolo scientifico:
\begin{itemize}
    \item \textbf{Introduzione:} Background teorico, letteratura, ipotesi, obiettivi.
    \item \textbf{Materiali e Metodi:} Descrizione \textbf{dettagliatissima} di partecipanti, strumenti, procedure, disegno, analisi statistiche. \textbf{Cruciale per la replicabilità!}
    \item \textbf{Risultati:} Presentazione oggettiva dei dati e delle analisi (indici descrittivi, test inferenziali, tabelle, grafici).
    \item \textbf{Discussione:} Interpretazione dei risultati alla luce delle ipotesi e della letteratura, conclusioni, limiti, prospettive future.
\end{itemize}

\begin{reflectionbox}
    \item Quale sezione dell'articolo deve essere descritta nel minimo dettaglio per permettere ad altri di rifare lo studio? (Vedere Domanda 12)
\end{reflectionbox}

\section*{11. Cenni di Notazione Statistica}
Simboli comuni:
\begin{itemize}
    \item \textbf{X, Y:} Indicano le variabili misurate.
    \item \textbf{N:} Numero totale di soggetti/osservazioni nella \textbf{Popolazione}.
    \item \textbf{n:} Numero totale di soggetti/osservazioni nel \textbf{Campione}.
    \item \textbf{X\textsubscript{i} (o Y\textsubscript{i}):} Rappresenta il punteggio/valore dell'\textit{i}-esimo soggetto (individuo) sulla variabile X (o Y). L'indice `i' varia da 1 a N (o n).
\end{itemize}

\begin{reflectionbox}
    \item Con quale lettera si indica solitamente la numerosità (quanti soggetti) di un campione? (Vedere Domanda 7)
\end{reflectionbox}



\end{document}
% ----- FINE DOCUMENTO -----