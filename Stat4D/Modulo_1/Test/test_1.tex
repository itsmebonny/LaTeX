\documentclass[12pt, a4paper]{article}
\usepackage[utf8]{inputenc}
\usepackage[T1]{fontenc}
\usepackage[italian]{babel}
\usepackage{amsmath}
\usepackage{amssymb} % Per simboli matematici se servissero
\usepackage{geometry}
\geometry{a4paper, left=2.5cm, right=2.5cm, top=2.5cm, bottom=2.5cm}
\usepackage{parskip} % Spazio tra paragrafi
\usepackage{enumitem} % Per personalizzare liste
\usepackage{ FiraSans } % Font leggibile
\usepackage{xcolor} % Per colori (se volessimo usarli)
\usepackage{graphicx} % Necessario per \newpage

% ----- Impostazioni Liste -----
% Rende le liste più compatte
\setlist[enumerate,1]{label=\arabic*)} % Numeri per domande principali
\setlist[enumerate,2]{label=(\alph*), leftmargin=*} % Lettere per opzioni

\begin{document}

\begin{center}
    \Large\textbf{Test di Autovalutazione Aggiuntivo - Modulo 1} \\
    \vspace{0.2cm}
    \large\textit{Tecniche di Ricerca e Analisi dei Dati} \\
    \vspace{0.5cm}
    \normalsize{Istruzioni: Scegli l'unica risposta corretta per ogni domanda. Le soluzioni sono disponibili nell'ultima pagina.}
\end{center}
\vspace{1cm}

\begin{enumerate} % Inizio elenco domande principali
    \item Quale fase del processo di ricerca segue immediatamente l'analisi statistica dei dati?
    \begin{enumerate} % Inizio opzioni domanda 1
        \item La raccolta dei dati.
        \item La pianificazione del disegno sperimentale.
        \item L'interpretazione dei risultati.
        \item L'identificazione del problema di ricerca.
    \end{enumerate}

    \item La statistica descrittiva ha lo scopo principale di:
    \begin{enumerate}
        \item Verificare ipotesi sulla popolazione partendo dal campione.
        \item Riassumere e organizzare i dati raccolti per renderli comprensibili.
        \item Stimare il valore dei parametri della popolazione.
        \item Determinare se le differenze osservate sono dovute al caso.
    \end{enumerate}

    \item Un ricercatore calcola il reddito medio annuo di 500 cittadini italiani selezionati casualmente. Questo valore (il reddito medio dei 500) rappresenta:
    \begin{enumerate}
        \item Un parametro della popolazione.
        \item Un indice statistico del campione.
        \item Un errore di campionamento.
        \item Una variabile nominale.
    \end{enumerate}

    \item Perché è fondamentale l'operazionalizzazione di un costrutto (es. "aggressività") nella ricerca psicologica?
    \begin{enumerate}
        \item Per renderlo più interessante per i partecipanti.
        \item Per poterlo misurare attraverso indicatori osservabili e concreti.
        \item Per dimostrare che il costrutto non esiste realmente.
        \item Per classificarlo su una scala nominale.
    \end{enumerate}

    \item Quale delle seguenti variabili è misurata su una scala a rapporti?
    \begin{enumerate}
        \item Il tipo di diploma conseguito (es. liceo classico, tecnico, professionale).
        \item La temperatura corporea misurata in gradi Celsius.
        \item Il tempo (in secondi) impiegato per completare un puzzle.
        \item Il livello di accordo con un'affermazione (scala da 1=Per niente a 5=Completamente).
    \end{enumerate}

    \item In uno studio sperimentale che valuta l'effetto di diverse dosi di caffeina (0mg, 100mg, 200mg) sulla velocità di reazione, la variabile "velocità di reazione" è:
    \begin{enumerate}
        \item La variabile indipendente qualitativa.
        \item La variabile indipendente quantitativa.
        \item La variabile dipendente.
        \item Una variabile di disturbo.
    \end{enumerate}

    \item Quale tra i seguenti è un esempio di variabile discreta?
    \begin{enumerate}
        \item L'altezza di una persona in centimetri.
        \item Il numero di libri letti in un anno.
        \item Il peso di un neonato in grammi.
        \item La distanza percorsa in chilometri.
    \end{enumerate}

    \item Un ricercatore vuole studiare la relazione tra il livello di autostima e il rendimento accademico in un gruppo di studenti, misurando entrambe le variabili tramite questionari. Che tipo di disegno di ricerca sta utilizzando principalmente?
    \begin{enumerate}
        \item Sperimentale.
        \item Quasi-sperimentale.
        \item Correlazionale.
        \item Descrittivo (studio di caso singolo).
    \end{enumerate}

    \item L'errore sistematico in una misurazione:
    \begin{enumerate}
        \item È dovuto a fluttuazioni casuali e imprevedibili.
        \item Tende ad annullarsi se si ripetono molte misurazioni.
        \item Influenza tutte le misurazioni in modo simile (es. sempre per eccesso o difetto).
        \item È sinonimo di errore di campionamento.
    \end{enumerate}

    \item Nella notazione statistica standard, cosa rappresenta generalmente la lettera 'N' maiuscola?
    \begin{enumerate}
        \item Il numero di soggetti nel campione.
        \item Il punteggio del primo soggetto.
        \item Il numero totale di soggetti nella popolazione.
        \item Il numero di variabili misurate.
    \end{enumerate}

    \item Quale scala di misura permette di classificare gli elementi in categorie distinte, senza alcun ordine implicito?
    \begin{enumerate}
        \item Scala ordinale.
        \item Scala a intervalli.
        \item Scala a rapporti.
        \item Scala nominale.
    \end{enumerate}

    \item L'obiettivo principale della ricerca applicata è:
    \begin{enumerate}
        \item Sviluppare nuove teorie psicologiche.
        \item Trovare soluzioni pratiche a problemi specifici del mondo reale.
        \item Descrivere dettagliatamente un singolo caso clinico.
        \item Aumentare la conoscenza pura, indipendentemente dalle applicazioni.
    \end{enumerate}

    \item Un disegno di ricerca "within-subjects" (o a misure ripetute) si caratterizza perché:
    \begin{enumerate}
        \item Confronta gruppi diversi di partecipanti assegnati a condizioni diverse.
        \item Gli stessi partecipanti vengono sottoposti a tutte le condizioni sperimentali (o misurati più volte).
        \item Non prevede alcuna misurazione della variabile dipendente.
        \item Viene utilizzato solo per studi correlazionali.
    \end{enumerate}

    \item Se, in un esperimento, il gruppo che riceve un nuovo farmaco viene confrontato con un gruppo che riceve una pillola senza principio attivo (placebo), quest'ultimo gruppo rappresenta:
    \begin{enumerate}
        \item Il campione rappresentativo.
        \item La popolazione di riferimento.
        \item La condizione sperimentale.
        \item La condizione di controllo.
    \end{enumerate}

    \item L'errore di campionamento rappresenta:
    \begin{enumerate}
        \item Uno sbaglio commesso dal ricercatore nella selezione del campione.
        \item La differenza naturale tra le caratteristiche del campione e quelle della popolazione da cui è estratto.
        \item Un difetto dello strumento di misura utilizzato.
        \item L'effetto della variabile indipendente sulla variabile dipendente.
    \end{enumerate}

     \item Quale delle seguenti NON è una caratteristica fondamentale di un vero esperimento?
    \begin{enumerate}
        \item La manipolazione di una variabile indipendente.
        \item La misurazione di una variabile dipendente.
        \item L'assenza totale di qualsiasi errore di misurazione.
        \item Il controllo delle variabili estranee.
    \end{enumerate}

    \item La sezione "Materiali e Metodi" di un articolo scientifico è cruciale per:
    \begin{enumerate}
        \item Interpretare il significato teorico dei risultati.
        \item Fornire un riassunto breve dello studio per i lettori frettolosi.
        \item Permettere ad altri ricercatori di replicare lo studio.
        \item Presentare grafici e tabelle riassuntive dei dati.
    \end{enumerate}

    \item Il Quoziente Intellettivo (QI) è tipicamente considerato misurato su quale scala?
    \begin{enumerate}
        \item Nominale.
        \item Ordinale.
        \item A intervalli.
        \item A rapporti.
    \end{enumerate}

    \item Una variabile che può assumere qualsiasi valore numerico all'interno di un dato intervallo (es. 1.75 metri, 65.3 kg) è detta:
    \begin{enumerate}
        \item Discreta.
        \item Nominale.
        \item Continua.
        \item Dipendente.
    \end{enumerate}

    \item L'assegnazione casuale dei partecipanti alle diverse condizioni sperimentali serve principalmente a:
    \begin{enumerate}
        \item Rendere l'esperimento più interessante.
        \item Controllare le variabili di disturbo legate alle differenze individuali tra i partecipanti.
        \item Garantire che la variabile dipendente sia misurata correttamente.
        \item Scegliere la popolazione di riferimento.
    \end{enumerate}

\end{enumerate} % Fine elenco domande principali

\newpage % Inizia una nuova pagina per le soluzioni

\begin{center}
    \Large\textbf{Griglia delle Soluzioni} \\
    \vspace{0.5cm}
    \normalsize{(Test di Autovalutazione Aggiuntivo - Modulo 1)}
\end{center}
\vspace{1cm}

\begin{enumerate}[leftmargin=*, label=\arabic*.]
    \item (c) L'interpretazione dei risultati.
    \item (b) Riassumere e organizzare i dati raccolti per renderli comprensibili.
    \item (b) Un indice statistico del campione.
    \item (b) Per poterlo misurare attraverso indicatori osservabili e concreti.
    \item (c) Il tempo (in secondi) impiegato per completare un puzzle. (Lo zero è assoluto: 0 secondi = assenza di tempo).
    \item (c) La variabile dipendente. (È l'effetto misurato).
    \item (b) Il numero di libri letti in un anno. (Non puoi leggere 2.5 libri).
    \item (c) Correlazionale. (Si cerca una relazione, non si manipola nulla).
    \item (c) Influenza tutte le misurazioni in modo simile.
    \item (c) Il numero totale di soggetti nella popolazione. (n minuscola indica il campione).
    \item (d) Scala nominale.
    \item (b) Trovare soluzioni pratiche a problemi specifici del mondo reale.
    \item (b) Gli stessi partecipanti vengono sottoposti a tutte le condizioni sperimentali.
    \item (d) La condizione di controllo.
    \item (b) La differenza naturale tra le caratteristiche del campione e quelle della popolazione.
    \item (c) L'assenza totale di qualsiasi errore di misurazione. (L'errore è sempre presente).
    \item (c) Permettere ad altri ricercatori di replicare lo studio.
    \item (c) A intervalli. (Le differenze tra punteggi sono significative, ma lo zero non è assoluto e non si possono fare rapporti diretti tipo "QI 120 è il doppio di QI 60").
    \item (c) Continua.
    \item (b) Controllare le variabili di disturbo legate alle differenze individuali tra i partecipanti.
\end{enumerate}

\end{document}
% ----- FINE DOCUMENTO -----