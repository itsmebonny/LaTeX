\documentclass[12pt, a4paper]{article}
\usepackage[utf8]{inputenc}
\usepackage[T1]{fontenc}
\usepackage[italian]{babel}
\usepackage{amsmath}
\usepackage{amssymb} % Per simboli matematici se servissero
\usepackage{geometry}
\geometry{a4paper, left=2.5cm, right=2.5cm, top=2.5cm, bottom=2.5cm}
\usepackage{parskip} % Spazio tra paragrafi
\usepackage{enumitem} % Per personalizzare liste
\usepackage{ FiraSans } % Font leggibile
\usepackage{xcolor} % Per colori (se volessimo usarli)
\usepackage{graphicx} % Necessario per \newpage

% ----- Impostazioni Liste -----
% Rende le liste più compatte
\setlist[enumerate,1]{label=\arabic*)} % Numeri per domande principali
\setlist[enumerate,2]{label=(\alph*), leftmargin=*} % Lettere per opzioni

% ----- Macro utili (se servissero per le soluzioni, qui non usate) -----
\newcommand{\Hnull}{H_0} % Ipotesi nulla
\newcommand{\Halt}{H_1} % Ipotesi alternativa
\newcommand{\Fscore}{F} % Statistica F
\newcommand{\etasq}{\eta^2} % Eta quadro
\newcommand{\MSb}{MS_{\text{tra}}} % Varianza Between
\newcommand{\MSw}{MS_{\text{entro}}} % Varianza Within

\begin{document}

\begin{center}
    \Large\textbf{Test di Autovalutazione Originale - Modulo 4} \\
    \vspace{0.2cm}
    \large\textit{Tecniche di Ricerca e Analisi dei Dati} \\
    \vspace{0.5cm}
    \normalsize{Istruzioni: Scegli l'unica risposta corretta per ogni domanda. Le soluzioni sono disponibili nell'ultima pagina.}
\end{center}
\vspace{1cm}

\begin{enumerate} % Inizio elenco domande principali
    \item Qual è il motivo principale per cui si utilizza l'ANOVA invece di eseguire molteplici t-test quando si confrontano più di due gruppi?
    \begin{enumerate} % Inizio opzioni domanda 1
        \item L'ANOVA è più semplice da calcolare a mano.
        \item Per controllare l'inflazione della probabilità dell'errore di I tipo.
        \item L'ANOVA richiede meno partecipanti.
        \item Il t-test non può confrontare medie.
    \end{enumerate}

    \item In un'ANOVA a una via, cosa rappresenta la Varianza ENTRO i gruppi ($\MSw$)?
    \begin{enumerate}
        \item L'effetto della variabile indipendente.
        \item La variabilità totale dei dati.
        \item La variabilità dovuta alle differenze tra le medie dei gruppi.
        \item La variabilità non spiegata (errore casuale e differenze individuali).
    \end{enumerate}

    \item Se il rapporto $\Fscore$ calcolato in un'ANOVA è molto vicino a 1, cosa suggerisce riguardo all'ipotesi nulla ($\Hnull$)?
    \begin{enumerate}
        \item $\Hnull$ è quasi certamente falsa.
        \item I dati supportano $\Hnull$ (probabile assenza di effetto).
        \item C'è un errore nei calcoli.
        \item La dimensione dell'effetto è molto grande.
    \end{enumerate}

    \item L'ipotesi alternativa ($\Halt$) nell'ANOVA a una via afferma che:
    \begin{enumerate}
        \item Tutte le medie delle popolazioni sono diverse tra loro.
        \item Almeno due medie delle popolazioni sono diverse tra loro.
        \item Le varianze delle popolazioni sono diverse.
        \item La media del campione è diversa da quella della popolazione.
    \end{enumerate}

    \item Un ricercatore studia l'effetto del tipo di musica (Classica, Rock, Silenzio) e del livello di rumore ambientale (Basso, Alto) sulla concentrazione (punteggio 0-100). Che tipo di ANOVA dovrebbe usare?
    \begin{enumerate}
        \item ANOVA a una via per campioni indipendenti.
        \item ANOVA Fattoriale.
        \item ANOVA per misure ripetute.
        \item Test t per campioni indipendenti.
    \end{enumerate}

    \item Nell'ANOVA fattoriale, un "effetto di interazione" significa che:
    \begin{enumerate}
        \item Entrambi i fattori hanno un effetto significativo sulla variabile dipendente.
        \item L'effetto di un fattore sulla variabile dipendente dipende dal livello dell'altro fattore.
        \item Almeno un fattore non ha alcun effetto.
        \item Gli errori residui sono molto alti.
    \end{enumerate}

    \item L'indice Eta Quadro ($\etasq$) nell'ANOVA misura:
    \begin{enumerate}
        \item La probabilità di commettere un errore di tipo II.
        \item I gradi di libertà dell'errore.
        \item La proporzione di varianza della VD spiegata dalla VI (fattore).
        \item Il valore critico della statistica F.
    \end{enumerate}

    \item Un valore di $\etasq = 0.15$ indica convenzionalmente una dimensione dell'effetto:
    \begin{enumerate}
        \item Trascurabile.
        \item Piccola.
        \item Media.
        \item Grande.
    \end{enumerate}

    \item Il test di Levene è utilizzato per verificare quale assunto dell'ANOVA per campioni indipendenti?
    \begin{enumerate}
        \item Normalità delle distribuzioni.
        \item Indipendenza delle osservazioni.
        \item Omogeneità delle varianze (Omoschedasticità).
        \item Linearità.
    \end{enumerate}

    \item Qual è il vantaggio principale di un disegno a misure ripetute rispetto a un disegno tra gruppi?
    \begin{enumerate}
        \item È sempre applicabile.
        \item Elimina gli effetti dell'ordine.
        \item Richiede meno calcoli statistici.
        \item Controlla la variabilità dovuta alle differenze individuali stabili tra i partecipanti.
    \end{enumerate}

     \item Il test t per misure ripetute è adatto quando si confrontano:
    \begin{enumerate}
        \item Due gruppi indipendenti su una variabile quantitativa.
        \item Tre o più misure ripetute sugli stessi soggetti.
        \item Due misure ripetute sugli stessi soggetti (o coppie appaiate).
        \item Due variabili qualitative sullo stesso campione.
    \end{enumerate}

     \item Su cosa si basa principalmente il calcolo del test t per misure ripetute?
    \begin{enumerate}
        \item Sul confronto diretto delle medie delle due misurazioni.
        \item Sulla varianza combinata dei due momenti di misurazione.
        \item Sull'analisi dei punteggi differenza calcolati per ogni soggetto tra le due misure.
        \item Sulla correlazione tra le due misure.
    \end{enumerate}

     \item L'ipotesi nulla ($\Hnull$) per un test t per misure ripetute è tipicamente:
    \begin{enumerate}
        \item $\mu_1 = \mu_2$
        \item $\mu_d = 0$ (la media delle differenze è zero)
        \item $d = 0$ (la differenza individuale è zero)
        \item $\bar{X}_1 = \bar{X}_2$
    \end{enumerate}

     \item L'ANOVA per misure ripetute (ANOVA RM) viene utilizzata quando:
    \begin{enumerate}
        \item Si confrontano le medie di 3 o più gruppi indipendenti.
        \item Si confrontano le medie di 3 o più misure sugli stessi soggetti.
        \item Si vuole verificare la correlazione tra 3 o più variabili.
        \item La variabile dipendente è qualitativa.
    \end{enumerate}

     \item Nell'ANOVA RM, la varianza "Entro Soggetti" viene scomposta in:
    \begin{enumerate}
        \item Varianza Tra Misure e Varianza Residua (Errore).
        \item Varianza Tra Soggetti e Varianza Tra Misure.
        \item Varianza Spiegata e Varianza Totale.
        \item Effetto Principale e Interazione.
    \end{enumerate}

     \item Quale termine di errore viene usato al denominatore del rapporto F nell'ANOVA RM?
    \begin{enumerate}
        \item La Varianza Entro Gruppi ($\MSw$).
        \item La Varianza Residua ($MS_{Residua}$).
        \item La Varianza Tra Soggetti.
        \item La Varianza Totale.
    \end{enumerate}

     \item L'assunto di sfericità è particolarmente rilevante per:
    \begin{enumerate}
        \item Il t-test per campioni indipendenti.
        \item L'ANOVA a una via per campioni indipendenti.
        \item Il t-test per misure ripetute (con solo 2 misure).
        \item L'ANOVA per misure ripetute (con 3 o più misure).
    \end{enumerate}

     \item Se il test di Mauchly risulta significativo ($p \le 0.05$) in un'ANOVA RM, cosa indica?
    \begin{enumerate}
        \item L'assunto di sfericità è rispettato.
        \item L'assunto di normalità è violato.
        \item L'assunto di sfericità è violato.
        \item L'effetto del trattamento è statisticamente significativo.
    \end{enumerate}

     \item Quale correzione si può applicare ai gradi di libertà del test F quando l'assunto di sfericità è violato?
    \begin{enumerate}
        \item Correzione di Bonferroni.
        \item Correzione di Welch.
        \item Correzione di Greenhouse-Geisser o Huynh-Feldt.
        \item Test di Levene.
    \end{enumerate}

     \item Un ricercatore misura la soddisfazione lavorativa (scala 1-10) in un gruppo di dipendenti prima, subito dopo e 6 mesi dopo un intervento formativo. Quale test è più appropriato per analizzare i dati?
    \begin{enumerate}
        \item Test t per campioni indipendenti.
        \item ANOVA a una via per campioni indipendenti.
        \item Test t per misure ripetute.
        \item ANOVA per misure ripetute.
    \end{enumerate}


\end{enumerate} % Fine elenco domande principali

\newpage % Inizia una nuova pagina per le soluzioni

\begin{center}
    \Large\textbf{Griglia delle Soluzioni} \\
    \vspace{0.5cm}
    \normalsize{(Test di Autovalutazione Originale - Modulo 4)}
\end{center}
\vspace{1cm}

\begin{enumerate}[leftmargin=*, label=\arabic*.]
    \item (b) Per controllare l'inflazione della probabilità dell'errore di I tipo.
    \item (d) La variabilità non spiegata (errore casuale e differenze individuali).
    \item (b) I dati supportano $\Hnull$ (probabile assenza di effetto) (Perché sia la varianza spiegata che quella d'errore sono simili).
    \item (b) Almeno due medie delle popolazioni sono diverse tra loro. (È un test omnibus).
    \item (b) ANOVA Fattoriale (Ci sono due VI: tipo di musica e livello di rumore).
    \item (b) L'effetto di un fattore sulla variabile dipendente dipende dal livello dell'altro fattore.
    \item (c) La proporzione di varianza della VD spiegata dalla VI (fattore).
    \item (d) Grande ($\etasq = 0.14$ è la soglia per 'grande').
    \item (c) Omogeneità delle varianze (Omoschedasticità).
    \item (d) Controlla la variabilità dovuta alle differenze individuali stabili tra i partecipanti (rendendo il test più potente).
    \item (c) Due misure ripetute sugli stessi soggetti (o coppie appaiate).
    \item (c) Sull'analisi dei punteggi differenza calcolati per ogni soggetto tra le due misure.
    \item (b) $\mu_d = 0$ (la media delle differenze è zero).
    \item (b) Si confrontano le medie di 3 o più misure sugli stessi soggetti.
    \item (a) Varianza Tra Misure e Varianza Residua (Errore). (La varianza Tra Soggetti viene separata).
    \item (b) La Varianza Residua ($MS_{Residua}$) (che rappresenta l'errore specifico per i disegni RM).
    \item (d) L'ANOVA per misure ripetute (con 3 o più misure). (Con 2 misure, la sfericità è sempre soddisfatta).
    \item (c) L'assunto di sfericità è violato.
    \item (c) Correzione di Greenhouse-Geisser o Huynh-Feldt (che aggiustano i gl).
    \item (d) ANOVA per misure ripetute (Ci sono 3 misure ripetute sugli stessi soggetti).
\end{enumerate}

\end{document}
% ----- FINE DOCUMENTO -----