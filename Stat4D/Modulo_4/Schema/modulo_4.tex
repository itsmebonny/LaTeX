\documentclass[12pt, a4paper]{article}
\usepackage[utf8]{inputenc}
\usepackage[T1]{fontenc}
\usepackage[italian]{babel}
\usepackage{amsmath}
\usepackage{amssymb}
\usepackage{graphicx}
\usepackage{geometry}
\geometry{a4paper, left=2.5cm, right=2.5cm, top=2.5cm, bottom=2.5cm}
\usepackage{parskip} % Spazio tra paragrafi invece di indentazione
\usepackage{enumitem} % Per personalizzare elenchi
\usepackage{ FiraSans } % Font più moderno
\usepackage{xcolor} % Per colori
\usepackage{framed} % Per i box
\usepackage{bm} % Per grassetto matematico (se serve)
\usepackage{tikz} % Per diagrammi
\usetikzlibrary{shapes, arrows.meta, positioning, calc, shadows, backgrounds} % Librerie TikZ utili (aggiunto backgrounds, arrows.meta)
\usepackage{booktabs} % Per tabelle professionali
\usepackage{colortbl} % Per colorare celle tabelle
\usepackage{multirow} % Per celle multi-riga
\usepackage{appendix} % Per gestire le appendici

% ----- Definizioni Colori e Stili -----
\definecolor{boxbgcolor}{rgb}{0.95, 0.95, 1.0} % Azzurrino per reflection
\definecolor{boxtitlecolor}{rgb}{0.1, 0.1, 0.6} % Blu scuro per reflection title
\definecolor{examplebgcolor}{rgb}{1.0, 0.98, 0.9} % Giallo pallido per esempi
\definecolor{examplebordercolor}{rgb}{0.9, 0.8, 0.5} % Bordo ocra per esempi
\definecolor{diagrambgcolor}{rgb}{0.92, 0.92, 0.92} % Grigio chiaro per sfondi diagrammi
\definecolor{decisioncolor}{rgb}{0.9, 1.0, 0.9} % Verde pallido per decisioni flowchart
\definecolor{testcolor}{rgb}{1.0, 0.9, 0.9} % Rosso pallido per test flowchart
\definecolor{startcolor}{rgb}{0.9, 0.9, 1.0} % Blu pallido per inizio flowchart

% Environment per Reflection Box
\newenvironment{reflectionbox}{%
    \medskip
    \begin{framed}\par\noindent
    \textbf{\color{boxtitlecolor}Domande per Riflettere (Basate sul Test)} \par
    \begin{itemize}[leftmargin=*, label=$\blacktriangleright$]
}{%
    \end{itemize}\par
    \end{framed}
    \medskip
}

% Environment per Esempi
\newenvironment{example}[1][Esempio Pratico]{%
    \medskip
    \begin{center}
    \begin{tikzpicture}
        \node[rectangle, draw=examplebordercolor, fill=examplebgcolor, rounded corners, inner sep=10pt, text width=0.9\textwidth] (box) \bgroup\medskip
        \par\noindent{\textbf{#1:}}\par\smallskip\noindent\ignorespaces
}{%
        \egroup;
    \end{tikzpicture}
    \end{center}
    \medskip
}

\setlist{nosep}
\renewcommand{\labelitemi}{$\bullet$}

% ----- Macro per simboli comuni -----
\newcommand{\popmean}{\mu}
\newcommand{\samplemean}{\bar{X}}
\newcommand{\popvar}{\sigma^2}
\newcommand{\samplevar}{s^2}
\newcommand{\popsd}{\sigma}
\newcommand{\samplesd}{s}
\newcommand{\stderr}{\sigma_{\samplemean}}
\newcommand{\zscore}{Z}
\newcommand{\tscore}{t}
\newcommand{\Fscore}{F} % Statistica F
\newcommand{\alphaerr}{\alpha} % Livello alpha
\newcommand{\betaerr}{\beta}  % Errore beta
\newcommand{\power}{1-\betaerr} % Potenza
\newcommand{\Hnull}{H_0} % Ipotesi nulla
\newcommand{\Halt}{H_1} % Ipotesi alternativa
\newcommand{\cohend}{d} % d di Cohen
\newcommand{\etasq}{\eta^2} % Eta quadro
\newcommand{\etasqpart}{\eta_p^2} % Eta quadro parziale
\newcommand{\df}{gl} % gradi di libertà
\newcommand{\SSb}{SS_{\text{tra}}} % Devianza Between (Tra Gruppi)
\newcommand{\SSw}{SS_{\text{entro}}} % Devianza Within (Entro Gruppi)
\newcommand{\SStot}{SS_{\text{tot}}} % Devianza Totale
\newcommand{\MSb}{MS_{\text{tra}}} % Varianza Between (Mean Square)
\newcommand{\MSw}{MS_{\text{entro}}} % Varianza Within (Mean Square)
\newcommand{\kgr}{k} % Numero gruppi/livelli
\newcommand{\Ntot}{N_{\text{tot}}} % N totale
\newcommand{\niind}{n_i} % n per gruppo i
\newcommand{\dfb}{\df_{\text{tra}}} % gl Between
\newcommand{\dfw}{\df_{\text{entro}}} % gl Within
\newcommand{\dftot}{\df_{\text{tot}}} % gl Totale

% ----- Stili per Flowchart TikZ -----
\tikzstyle{startstop} = [rectangle, rounded corners, minimum width=3cm, minimum height=1cm, text centered, draw=black, fill=startcolor, text width=4cm]
\tikzstyle{decision} = [diamond, minimum width=3cm, minimum height=1cm, text centered, draw=black, aspect=2, fill=decisioncolor, text width=3.5cm]
\tikzstyle{process} = [rectangle, minimum width=3cm, minimum height=1cm, text centered, text width=4cm, draw=black, fill=testcolor]
\tikzstyle{arrow} = [thick,->,>=stealth]

% ----- INIZIO DOCUMENTO -----
\begin{document}

\begin{center}
    \Large\textbf{Modulo 4: Analisi della Varianza (ANOVA) e Confronto tra Campioni Dipendenti} \\
    \vspace{0.5cm}
    \large\textit{(ANOVA Indipendente, ANOVA Fattoriale, t-test e ANOVA per Misure Ripetute)}
\end{center}

% #####################################################################
\section*{Parte 1: L'Analisi della Varianza (ANOVA) per Campioni Indipendenti}
% #####################################################################

% ... (contenuto della Parte 1 come nel codice precedente) ...
\subsection*{1. Introduzione all'Analisi della Varianza}
\begin{itemize}
    \item \textbf{Scopo:} Estende la logica del t-test per campioni indipendenti, permettendo di confrontare le medie di una variabile dipendente quantitativa tra \textbf{due o più} gruppi (definiti da una o più variabili indipendenti qualitative).
    \item \textbf{Origine del Nome:} Si chiama Analisi della Varianza perché, per decidere se le medie dei gruppi sono diverse, analizza (scompone) la \textbf{variabilità totale} dei dati in diverse fonti.
    \item \textbf{Problema dei Confronti Multipli:} Perché non fare tanti t-test a coppie se ho più di due gruppi (es. Gruppo A vs B, A vs C, B vs C)?
        \begin{itemize}
            \item Ogni test ha una probabilità $\alphaerr$ (es. 0.05) di commettere un Errore di I Tipo (rifiutare $\Hnull$ vera).
            \item Facendo molti test, la probabilità \textit{complessiva} di fare almeno un errore di I tipo aumenta notevolmente (inflazione dell'errore alfa).
            \item L'ANOVA esegue un \textbf{test omnibus} (unico) che confronta tutte le medie simultaneamente, mantenendo la probabilità di errore di I tipo sotto controllo al livello $\alphaerr$ prefissato.
        \end{itemize}
    \item \textbf{Tipi di ANOVA Univariata (una VD):}
        \begin{itemize}
            \item \textbf{ANOVA a una via (One-Way ANOVA):} Una sola variabile indipendente (fattore) con 3 o più livelli (gruppi). Es: Effetto di 3 dosaggi di caffeina (placebo, basso, alto) sul voto.
            \item \textbf{ANOVA Fattoriale:} Due o più variabili indipendenti (fattori). Es: Effetto del dosaggio di caffeina (3 livelli) E del sesso (2 livelli) sul voto. Permette di studiare gli \textbf{effetti principali} di ciascun fattore e la loro \textbf{interazione}.
        \end{itemize}
    \item \textbf{ANOVA Multivariata (MANOVA):} Studia l'effetto di una o più VI su \textbf{più variabili dipendenti} contemporaneamente. (Accennata, non trattata in dettaglio qui).
    \item \textbf{Disegni di Ricerca:} L'ANOVA può essere usata sia in disegni \textbf{tra gruppi (between-subjects)} che \textbf{entro i gruppi (within-subjects)} (vedi Parte 2).
\end{itemize}

\begin{reflectionbox}
    \item Il termine analisi della varianza identifica tecniche utilizzabili in quali tipi di disegni? (Vedere Domanda 1 - Risposta: In entrambe le tipologie)
    \item L'analisi della varianza univariata (ANOVA) consente di studiare quante variabili dipendenti alla volta? (Vedere Domanda 2)
    \item L'analisi della varianza confronta le medie analizzando cosa? (Vedere Domanda 3)
    \item Nell'analisi della varianza, la variabile dipendente che caratteristiche deve avere? (Vedere Domanda 8 - Risposta: Sempre quantitativa)
    \item Nell'analisi della varianza, la variabile indipendente può avere quanti livelli? (Vedere Domanda 22)
    \item L'analisi della varianza multivariata (MANOVA) studia gli effetti di...? (Vedere Domanda 27)
    \item In disegni di ricerca between, quale parametro definisce il numero di campioni da reclutare? (Vedere Domanda 29 - Risposta: I livelli dalla variabile indipendente)
\end{reflectionbox}

\subsection*{2. Logica della Verifica delle Ipotesi nell'ANOVA a Una Via}
\begin{itemize}
    \item \textbf{Ipotesi:}
        \begin{itemize}
            \item $\Hnull$: Le medie delle popolazioni da cui provengono i campioni sono \textbf{tutte uguali} tra loro ($\popmean_1 = \popmean_2 = \dots = \popmean_k$). Qualsiasi differenza tra le medie campionarie è dovuta solo al caso (errore campionario).
            \item $\Halt$: \textbf{Almeno una} media delle popolazioni è diversa dalle altre. Non dice quali medie differiscono, ma solo che non sono tutte uguali. È sempre \textbf{bilaterale}.
        \end{itemize}
    \item \textbf{Scomposizione della Varianza:} L'idea centrale è scomporre la variabilità totale dei dati in due fonti principali:
        \begin{itemize}
            \item \textbf{Varianza TRA Gruppi ($\MSb$ - Mean Square Between):} Misura quanto le medie di ciascun gruppo ($\samplemean_k$) differiscono dalla media generale ($\bar{X}_{tot}$). Rappresenta la variabilità \textbf{spiegata} dalla variabile indipendente (effetto del trattamento) + l'errore casuale (differenze individuali, errore di misura).
            \item \textbf{Varianza ENTRO Gruppi ($\MSw$ - Mean Square Within):} Misura quanto i punteggi individuali all'interno di ciascun gruppo variano attorno alla media del proprio gruppo. Rappresenta la variabilità \textbf{non spiegata} dalla variabile indipendente, dovuta solo all'errore casuale (differenze individuali, errore di misura). È considerata una stima della varianza d'errore della popolazione.
        \end{itemize}
    \item \textbf{Il Rapporto F (Statistica Test):} L'ANOVA calcola il rapporto tra queste due varianze:
        $$ \Fscore = \frac{\text{Varianza TRA Gruppi}}{\text{Varianza ENTRO Gruppi}} = \frac{\MSb}{\MSw} $$
        \begin{itemize}
            \item Se $\Hnull$ è \textbf{vera} ($\popmean_1 = \dots = \popmean_k$): La VI non ha effetto. La varianza TRA gruppi riflette solo l'errore casuale, così come la varianza ENTRO gruppi. Quindi $\MSb \approx \MSw$, e il rapporto $\Fscore$ sarà \textbf{vicino a 1}.
            \item Se $\Hnull$ è \textbf{falsa} (c'è un effetto della VI): La varianza TRA gruppi rifletterà l'effetto della VI \textit{più} l'errore casuale, mentre la varianza ENTRO riflette solo l'errore. Quindi $\MSb > \MSw$, e il rapporto $\Fscore$ sarà \textbf{maggiore di 1}.
            \item Più $\Fscore$ è grande (maggiore di 1), più è probabile che la differenza tra le medie sia reale (statisticamente significativa) e non dovuta al caso.
        \end{itemize}
    \item \textbf{Distribuzione F di Fisher:} Per prendere una decisione, il valore $\Fscore$ calcolato ($\Fscore_{test}$) viene confrontato con un valore critico ($\Fscore_{critico}$) sulla \textit{distribuzione F di Fisher}.
        \begin{itemize}
            \item È una distribuzione teorica di probabilità \textbf{asimmetrica positiva} (definita solo per valori $\ge 0$).
            \item La sua forma dipende da \textbf{due parametri}: i gradi di libertà della varianza TRA gruppi ($\dfb$) e i gradi di libertà della varianza ENTRO gruppi ($\dfw$).
        \end{itemize}
    \item \textbf{Decisione:}
        \begin{itemize}
            \item Se $\Fscore_{test} > \Fscore_{critico}$ (o se $p \le \alphaerr$): Si \textbf{rifiuta $\Hnull$}. Si conclude che c'è una differenza significativa tra almeno due delle medie dei gruppi.
            \item Se $\Fscore_{test} \le \Fscore_{critico}$ (o se $p > \alphaerr$): Si \textbf{non rifiuta $\Hnull$}. Non c'è evidenza sufficiente per dire che le medie delle popolazioni siano diverse.
        \end{itemize}
    \item \textbf{Confronti Post-Hoc o Pianificati:} Se l'ANOVA risulta significativa (si rifiuta $\Hnull$), indica che c'è una differenza \textit{da qualche parte}, ma non dice \textit{dove}. Per scoprire quali coppie di medie specifiche differiscono significativamente, si usano test aggiuntivi:
        \begin{itemize}
            \item \textbf{Test Post-Hoc} (es. Tukey, Bonferroni, Scheffé): Si usano dopo aver trovato un F significativo, per confrontare tutte le possibili coppie di medie, controllando l'errore alfa complessivo.
            \item \textbf{Confronti Pianificati (o a priori):} Ipotesi specifiche su quali medie differiranno, formulate prima di raccogliere i dati. Hanno maggiore potenza dei post-hoc.
            \item Questi test si usano \textbf{solo se} l'ANOVA generale è risultata significativa.
        \end{itemize}
\end{itemize}

\begin{reflectionbox}
    \item Nell'analisi della varianza, cosa predice l'ipotesi nulla ($\Hnull$)? (Vedere Domanda 5)
    \item L'ipotesi alternativa ($\Halt$) nell'analisi della varianza può essere...? (Vedere Domanda 6 - Risposta: Esclusivamente bilaterale)
    \item In un disegno between-groups, la varianza ENTRO i gruppi da quali fattori è influenzata? (Vedere Domanda 7 - Risposta: differenze individuali e errori residui)
    \item Nella verifica delle ipotesi con ANOVA, a quale distribuzione teorica si fa riferimento per il valore critico? (Vedere Domanda 11)
    \item Quando $\Hnull$ è vera, la stima della varianza TRA gruppi è...? (Vedere Domanda 14 - Risposta: Uguale alla stima della varianza entro i gruppi)
    \item La distribuzione F di Fisher è...? (Vedere Domanda 16 - Risposta: Asimmetrica)
    \item Quando $\Hnull$ è vera, il rapporto F tra varianza TRA e varianza ENTRO è...? (Vedere Domanda 17 - Risposta: Uguale a 1 circa)
    \item L'ANOVA prevede che la varianza totale sia scomposta in...? (Vedere Domanda 19)
    \item Se $\Fscore_{test} > \Fscore_{critico}$, quale decisione si prende? (Vedere Domanda 24)
    \item Da cosa è definita la distribuzione F di Fisher? (Vedere Domanda 25 - Risposta: I gradi di libertà)
    \item La statistica test F è uguale al rapporto tra...? (Vedere Domanda 28)
    \item I test post-hoc si usano...? (Vedere Domanda 4)
\end{reflectionbox}

\subsection*{3. Calcolo dell'ANOVA a Una Via (Cenni)}
Anche se il calcolo viene fatto dal software, è utile capire i passaggi:
\begin{enumerate}
    \item \textbf{Calcolare le Devianze (Somme dei Quadrati, SS):}
        \begin{itemize}
            \item \textbf{Devianza Totale ($\SStot$):} Somma degli scarti al quadrato di ogni singolo punteggio ($X_i$) dalla media generale ($\bar{X}_{tot}$). Misura la variabilità totale.
              $$ \SStot = \sum (X_i - \bar{X}_{tot})^2 $$
            \item \textbf{Devianza TRA Gruppi ($\SSb$):} Somma degli scarti al quadrato della media di ciascun gruppo ($\bar{X}_k$) dalla media generale ($\bar{X}_{tot}$), ponderata per la numerosità ($n_k$) di ciascun gruppo. Misura la variabilità dovuta alle differenze tra le medie dei gruppi.
              $$ \SSb = \sum n_k (\bar{X}_k - \bar{X}_{tot})^2 $$
            \item \textbf{Devianza ENTRO Gruppi ($\SSw$):} Somma degli scarti al quadrato di ogni punteggio ($X_{ik}$) dalla media del proprio gruppo ($\bar{X}_k$). Si può ottenere sommando le devianze interne a ciascun gruppo, oppure per differenza: $\SSw = \SStot - \SSb$. Misura la variabilità residua/errore.
        \end{itemize}
    \item \textbf{Calcolare i Gradi di Libertà ($\df$):}
        \begin{itemize}
            \item $\dftot = \Ntot - 1$ (dove $\Ntot$ è il numero totale di soggetti)
            \item $\dfb = \kgr - 1$ (dove $\kgr$ è il numero di gruppi/livelli della VI)
            \item $\dfw = \Ntot - \kgr$ (oppure $\dfw = \dftot - \dfb$)
        \end{itemize}
    \item \textbf{Calcolare le Varianze (Mean Squares, MS):} Si ottengono dividendo ogni devianza per i rispettivi gradi di libertà.
        \begin{itemize}
            \item $\MSb = \frac{\SSb}{\dfb}$ (Varianza TRA)
            \item $\MSw = \frac{\SSw}{\dfw}$ (Varianza ENTRO)
        \end{itemize}
    \item \textbf{Calcolare il Rapporto F:}
        $$ \Fscore = \frac{\MSb}{\MSw} $$
    \item \textbf{Tabella Riassuntiva ANOVA:} I risultati vengono solitamente presentati in una tabella standard:
        \begin{center}
        \begin{tabular}{lcccc} % Rimossa colonna p-value per semplificare
            \toprule
            Fonte Variazione & SS (Devianza) & gl (df) & MS (Varianza) & F  \\
            \midrule
            TRA Gruppi (Effetto VI) & $\SSb$ & $\dfb$ & $\MSb$ & \multirow{2}{*}{$\frac{\MSb}{\MSw}$}  \\
            ENTRO Gruppi (Errore) & $\SSw$ & $\dfw$ & $\MSw$ &    \\
            \midrule
            Totale & $\SStot$ & $\dftot$ & &  \\
            \bottomrule
        \end{tabular}
        \end{center}
\end{enumerate}

\begin{reflectionbox}
    \item Nell'ANOVA a una via, quante variabili indipendenti è possibile studiare? (Vedere Domanda 15)
    \item La varianza totale può essere calcolata come rapporto tra...? (Vedere Domanda 26 - Risposta: Devianza totale e gradi di libertà totali)
\end{reflectionbox}

\subsection*{4. La Verifica delle Ipotesi nell'ANOVA Fattoriale}
\begin{itemize}
    \item \textbf{Scopo:} Studiare l'effetto di \textbf{due o più VI (fattori)} su una VD quantitativa. Permette di analizzare:
        \begin{itemize}
            \item \textbf{Effetti Principali:} L'effetto medio di ciascun fattore sulla VD, considerato isolatamente (ignorando l'altro fattore). C'è un effetto principale per ogni VI.
            \item \textbf{Effetto di Interazione:} Se l'effetto di un fattore sulla VD \textbf{dipende dal livello} dell'altro fattore. L'interazione indica che i fattori agiscono congiuntamente in modo non semplicemente additivo.
        \end{itemize}
    \textbf{Esempio (2x3):} Studio sull'effetto del Sesso (Fattore A: M, F) e del Dosaggio Caffeina (Fattore B: Placebo, Basso, Alto) sul voto. % Aggiunto itemize per chiarezza esempio
    \begin{itemize}
        \item Effetto Principale A (Sesso): In media, i maschi hanno voti diversi dalle femmine, indipendentemente dal dosaggio?
        \item Effetto Principale B (Dosaggio): In media, i diversi dosaggi portano a voti diversi, indipendentemente dal sesso?
        \item Interazione AxB: L'effetto del dosaggio sul voto è diverso per maschi e femmine? (Es: la caffeina aiuta solo i maschi ma non le femmine, o viceversa).
    \end{itemize}
    \item \textbf{Logica della Scomposizione (estesa):} La Varianza TRA gruppi viene ulteriormente scomposta:
        $$ \SSb = SS_{\text{Fattore A}} + SS_{\text{Fattore B}} + SS_{\text{Interazione AxB}} $$
        La Varianza ENTRO gruppi ($\SSw$) rimane la misura dell'errore.
    \item \textbf{Verifica delle Ipotesi:} Si conducono test F separati per ogni effetto:
        \begin{itemize}
            \item Test F per Effetto Principale A: $F_A = MS_A / MS_W$
            \item Test F per Effetto Principale B: $F_B = MS_B / MS_W$
            \item Test F per Interazione AxB: $F_{AxB} = MS_{AxB} / MS_W$
        \end{itemize}
        Ognuno viene valutato con i propri gradi di libertà e confrontato con $\Fscore_{critico}$. Si possono avere effetti principali significativi senza interazione, interazione significativa senza effetti principali, o entrambi.
    \item \textbf{Importanza dell'Interazione:} Quando l'interazione è significativa, interpretare gli effetti principali richiede cautela, perché l'effetto di un fattore non è costante, ma varia a seconda dei livelli dell'altro.
\end{itemize}

\subsection*{5. Dimensione dell'Effetto nell'Analisi della Varianza}
Un F significativo dice che c'è un effetto, ma non quanto è grande. Si usano indici di effect size:
\begin{itemize}
    \item \textbf{Eta Quadro ($\etasq$):} Usato principalmente nell'ANOVA a una via.
        $$ \etasq = \frac{\SSb}{\SStot} $$
        \begin{itemize}
            \item \textbf{Interpretazione:} Indica la \textbf{proporzione di varianza totale} della VD che è \textbf{spiegata} dalla VI (dal fattore). Varia da 0 a 1 (o 0\% a 100\%).
            \item \textbf{Convenzioni (Cohen, per psicologia):}
                \begin{itemize}
                    \item $\etasq \approx 0.01$ (1\%): Effetto piccolo
                    \item $\etasq \approx 0.06$ (6\%): Effetto medio
                    \item $\etasq \approx 0.14$ (14\%): Effetto grande
                \end{itemize}
            \item \textbf{Limite:} Tende a sovrastimare la dimensione dell'effetto nella popolazione, specialmente con campioni piccoli.
        \end{itemize}
    \item \textbf{Eta Quadro Parziale ($\etasqpart$):} Usato nell'ANOVA Fattoriale (e spesso riportato anche per l'ANOVA a una via dai software).
        $$ \etasqpart (\text{per un effetto}) = \frac{SS_{\text{effetto}}}{SS_{\text{effetto}} + SS_{\text{errore}}} $$
        (Dove $SS_{\text{errore}} = \SSw$)
        \begin{itemize}
            \item \textbf{Interpretazione:} Indica la proporzione di varianza della VD \textbf{spiegata da un singolo effetto} (un fattore o un'interazione), dopo aver rimosso la variabilità spiegata dagli altri effetti inclusi nel modello. Varia da 0 a 1.
            \item Le convenzioni per piccolo/medio/grande sono le stesse di $\etasq$.
            \item Permette di valutare l'importanza relativa di ciascun effetto principale e interazione.
        \end{itemize}
\end{itemize}

\begin{reflectionbox}
    \item Quale indice si usa per calcolare la dimensione dell'effetto in ANOVA? (Vedere Domanda 21)
\end{reflectionbox}

\subsection*{6. Assunti dell'Analisi della Varianza per Campioni Indipendenti}
Perché i risultati dell'ANOVA (il valore F e il p-value) siano validi, devono essere soddisfatti alcuni assunti:
\begin{enumerate}
    \item \textbf{Indipendenza delle Osservazioni:} I punteggi dei soggetti all'interno di un gruppo devono essere indipendenti da quelli degli altri soggetti nello stesso gruppo e nei gruppi diversi. È un assunto legato al disegno di ricerca (assegnazione casuale, ecc.) e non può essere testato statisticamente facilmente, ma è cruciale.
    \item \textbf{Normalità:} La variabile dipendente deve essere distribuita normally all'interno di \textit{ciascuna} popolazione da cui provengono i campioni.
        \begin{itemize}
            \item Si può verificare con test come Shapiro-Wilk o Kolmogorov-Smirnov (se $p > 0.05$, l'assunto è rispettato) o con ispezione visiva (istogrammi, Q-Q plot).
            \item L'ANOVA è considerata \textbf{robusta} a violazioni moderate di questo assunto, specialmente se i gruppi hanno numerosità simili e sufficientemente grandi (es. $n \ge 20-30$ per gruppo).
        \end{itemize}
    \item \textbf{Omoschedasticità (Omogeneità delle Varianze):} Le varianze della variabile dipendente devono essere uguali in tutte le popolazioni da cui sono estratti i campioni ($\popvar_1 = \popvar_2 = \dots = \popvar_k$).
        \begin{itemize}
            \item Si verifica con il \textbf{Test di Levene} (preferito) o il test di Bartlett. Se il test \textit{non} è significativo ($p > 0.05$), l'assunto è rispettato (le varianze sono omogenee).
            \item Se l'assunto è violato ($p \le 0.05$), specialmente con numerosità dei gruppi diverse, l'F test standard può essere inaffidabile. Si possono usare correzioni (es. F di Welch o Brown-Forsythe nell'ANOVA a una via) o test non parametrici (es. Kruskal-Wallis).
        \end{itemize}
    \item \textbf{Additività degli Effetti (per ANOVA Fattoriale):} Gli effetti dei fattori si sommano. Violazioni gravi possono emergere come interazioni significative.
\end{enumerate}

\clearpage % Inizia nuova pagina per la Parte 2

% ##############################################################
\section*{Parte 2: Il Confronto tra Campioni Dipendenti (Misure Ripetute)}
% ##############################################################

\subsection*{7. La Logica dei Disegni di Ricerca a Misure Ripetute}
\begin{itemize}
    \item \textbf{Contrapposizione con Disegni Indipendenti (Between):} Nei disegni between-subjects, gruppi diversi di partecipanti sono assegnati a condizioni diverse (es. Gruppo A fa trattamento, Gruppo B fa placebo). Le differenze individuali tra i partecipanti contribuiscono alla varianza d'errore ($\MSw$).
    \item \textbf{Disegni a Misure Ripetute (Within-Subjects):} Gli \textbf{stessi partecipanti} vengono sottoposti a \textbf{tutte} le condizioni sperimentali o vengono misurati più volte nel tempo (es. pre-test, post-test, follow-up).
    \item \textbf{Vantaggi:}
        \begin{itemize}
            \item \textbf{Controllo delle Differenze Individuali:} Poiché ogni partecipante fa da controllo a se stesso, la variabilità dovuta alle differenze stabili tra persone viene eliminata o ridotta dalla componente d'errore. Questo rende il test \textbf{più potente} (più facile rilevare effetti reali).
            \item \textbf{Efficienza:} Richiedono meno partecipanti rispetto ai disegni between per ottenere la stessa potenza statistica.
        \end{itemize}
    \item \textbf{Svantaggi Potenziali:}
        \begin{itemize}
            \item \textbf{Effetti dell'Ordine (o di Trascinamento):} La partecipazione a una condizione può influenzare la performance nelle condizioni successive (es. apprendimento, fatica, noia). Si controllano con il \textit{controbilanciamento} delle condizioni.
            \item Non sempre applicabili (es. se il trattamento ha effetti permanenti).
        \end{itemize}
    \item \textbf{Struttura Dati:} I dati sono "appaiati": per ogni soggetto abbiamo più misurazioni (una per ogni condizione/tempo).
    \item \textbf{Numero di Misure:} Definito dai \textbf{livelli della variabile indipendente} (es. 2 livelli = 2 misure; 3 tempi = 3 misure).
\end{itemize}

\begin{reflectionbox}
    \item In un disegno within (entro i gruppi), il numero di misurazioni è definito da...? (Vedere Domanda 23)
    \item Relativamente all'ANOVA per misure ripetute, quale fattore NON è una fonte di variabilità dei dati nei disegni within (nel senso che viene controllato/rimosso)? (Vedere Domanda 30 - Risposta: Le differenze individuali, anche se presenti, sono separate dall'errore)
\end{reflectionbox}


\subsection*{8. Il test t per Misure Ripetute (o Campioni Appaiati)}
\begin{itemize}
    \item \textbf{Scopo:} Confrontare le medie di una VD quantitativa misurata in \textbf{due sole} occasioni (due condizioni, due tempi) sugli \textbf{stessi soggetti} (o su coppie appaiate).
    \item \textbf{Logica:} Invece di confrontare direttamente le medie delle due misurazioni ($\bar{X}_1$ vs $\bar{X}_2$), il test lavora sulle \textbf{differenze} tra le due misurazioni calcolate per \textbf{ciascun soggetto}.
        \begin{enumerate}
            \item Si calcola il punteggio differenza ($d = X_{post} - X_{pre}$ o $d = X_{cond1} - X_{cond2}$) per ogni partecipante.
            \item Si ottiene così una \textbf{singola distribuzione di punteggi differenza}.
            \item Il test verifica se la \textbf{media di queste differenze} ($\bar{d}$) nella popolazione ($\mu_d$) è significativamente diversa da zero.
        \end{enumerate}
    \item \textbf{Ipotesi:}
        \begin{itemize}
            \item $\Hnull: \mu_d = 0$ (Non c'è differenza media tra le due misurazioni nella popolazione; il trattamento/condizione non ha avuto effetto).
            \item $\Halt: \mu_d \neq 0$ (C'è una differenza media significativa tra le due misurazioni nella popolazione).
        \end{itemize}
    \item \textbf{Statistica Test (t):} Si calcola un test t su singolo campione applicato ai punteggi differenza:
        $$ t = \frac{\bar{d} - \mu_{d0}}{S_{\bar{d}}} = \frac{\bar{d} - 0}{s_d / \sqrt{n}} $$
        Dove:
        \begin{itemize}
            \item $\bar{d}$ è la media dei punteggi differenza nel campione.
            \item $s_d$ è la deviazione standard dei punteggi differenza nel campione.
            \item $n$ è il numero di partecipanti (coppie di misure).
            \item $S_{\bar{d}} = s_d / \sqrt{n}$ è l'errore standard della media delle differenze.
        \end{itemize}
    \item \textbf{Gradi di Libertà:} $\df = n - 1$.
    \item \textbf{Decisione:} Si confronta il $t_{test}$ con il $t_{critico}$ dalla distribuzione t di Student con $n-1$ gradi di libertà e $\alphaerr$ scelto. Se $|t_{test}| > |t_{critico}|$ (o $p \le \alphaerr$), si rifiuta $\Hnull$.
\end{itemize}

\begin{reflectionbox}
    \item Un ricercatore misura l'ansia prima e dopo un trattamento su 50 persone. Quale test usa per vedere se c'è differenza significativa? (Vedere Domanda 10)
\end{reflectionbox}

\subsection*{9. L'Analisi della Varianza per Misure Ripetute (ANOVA RM)}
\begin{itemize}
    \item \textbf{Scopo:} Estensione del test t per misure ripetute. Confronta le medie di una VD quantitativa misurata in \textbf{tre o più} occasioni (condizioni/tempi) sugli \textbf{stessi soggetti}.
    \item \textbf{Logica Simile all'ANOVA Indipendente:} Si basa sempre sulla scomposizione della varianza e sul calcolo di un rapporto F.
        $$ \Fscore = \frac{\text{Varianza TRA le Misure (Spiegata)}}{\text{Varianza Residua (Errore)}} $$
    \item \textbf{Differenza Chiave nella Scomposizione:} La variabilità totale viene scomposta diversamente per tener conto della dipendenza delle misure:
        \begin{itemize}
            \item \textbf{Varianza TRA Soggetti:} Variabilità dovuta alle differenze medie tra i partecipanti (viene calcolata ma poi "ignorata" o rimossa dal test F principale).
            \item \textbf{Varianza ENTRO Soggetti:} Variabilità all'interno dei punteggi di ciascun soggetto. Questa viene ulteriormente scomposta in:
                   \begin{itemize}
                       \item \textbf{Varianza TRA Misure (o Effetto Tempo/Condizione):} Misura quanto le medie delle diverse misurazioni/condizioni ($\bar{X}_k$) differiscono tra loro. Rappresenta l'effetto della VI (tempo, trattamento, ecc.) + errore residuo. Corrisponde a $\MSb$ dell'ANOVA indipendente.
                       \item \textbf{Varianza Residua (o Errore):} Misura la variabilità rimanente dopo aver tolto l'effetto delle condizioni e le differenze individuali. Rappresenta l'interazione Soggetto x Condizione (cioè, quanto l'effetto della condizione varia da soggetto a soggetto) e funge da termine di errore. Corrisponde a $\MSw$ dell'ANOVA indipendente.
                   \end{itemize}
        \end{itemize}
    \item \textbf{Calcolo del Rapporto F:}
        $$ F = \frac{MS_{\text{TRA Misure}}}{MS_{\text{Residua}}} $$
            \item \textbf{Gradi di Libertà:}
                \begin{itemize}
                    \item $\df_{\text{TRA Misure}} = k - 1$ (dove $k$ è il numero di misure/livelli VI)
                    \item $\df_{\text{Residua}} = (n - 1)(k - 1)$ (dove $n$ è il numero di soggetti)
                \end{itemize}
    \item \textbf{Decisione e Post-Hoc:} Come per l'ANOVA indipendente. Se F è significativo, si rifiuta $\Hnull$ ($\mu_1=\mu_2=\dots=\mu_k$) e si procede con test post-hoc (es. Bonferroni) per confronti a coppie tra le medie delle misure, usando $MS_{Residua}$ come stima dell'errore.
\end{itemize}

\begin{reflectionbox}
    \item Per confrontare 3 misurazioni consecutive sugli stessi soggetti, quale test si usa? (Vedere Domanda 18)
    \item Nell'ANOVA per misure ripetute, la variabile dipendente può essere...? (Vedere Domanda 12 - Risposta: Esclusivamente quantitativa)
\end{reflectionbox}

\subsection*{10. Dimensione dell'Effetto per Misure Ripetute}
\begin{itemize}
    \item \textbf{Test t per Misure Ripetute:} Si può calcolare il \textbf{d di Cohen} basato sulle differenze:
      $$ d = \frac{|\bar{d}|}{s_d} $$
      Interpretazione standard (0.2 piccolo, 0.5 medio, 0.8 grande).
    \item \textbf{ANOVA per Misure Ripetute:} Si usa l'\textbf{Eta Quadro ($\etasq$) o Eta Quadro Parziale ($\etasqpart$)}. La logica è simile all'ANOVA indipendente, ma le formule usano le SS specifiche del disegno RM.
        \begin{itemize}
            \item $\etasq = \frac{SS_{\text{TRA Misure}}}{SS_{\text{Totale}}}$ (proporzione della variabilità totale dovuta all'effetto).
            \item $\etasqpart = \frac{SS_{\text{TRA Misure}}}{SS_{\text{TRA Misure}} + SS_{\text{Residua}}}$ (proporzione della variabilità "entro soggetti" dovuta all'effetto).
            \item Le convenzioni (0.01, 0.06, 0.14) rimangono le stesse.
            \item L'Eta quadro parziale è spesso preferito perché non è influenzato dalla variabilità tra soggetti.
        \end{itemize}
\end{itemize}
\textbf{Nota Importante:} La Domanda 20 del test afferma che non è possibile calcolare l'effect size nei disegni within. Questo è \textbf{errato} secondo le dispense (p. 29-30) e la pratica comune, dove si calcolano sia $d$ per il t-test RM che $\etasq / \etasqpart$ per l'ANOVA RM.

\begin{reflectionbox}
    \item Nei disegni sperimentali within (entro i gruppi), è possibile calcolare la dimensione dell'effetto? (Vedere Domanda 20 - Nota: La risposta corretta basata sulle dispense è SÌ (Vero), anche se il test indica Falso. Si calcolano d di Cohen o Eta Quadro).
\end{reflectionbox}

\subsection*{11. Assunti dell'Analisi della Varianza per Misure Ripetute}
Oltre agli assunti comuni a molte procedure parametriche:
\begin{enumerate}
    \item \textbf{Scala di Misura:} La VD deve essere quantitativa (intervalli o rapporti).
    \item \textbf{Normalità:} La VD dovrebbe essere distribuita normalmente ad \textit{ogni} livello della VI (ad ogni tempo/condizione). Oppure, nel caso del t-test RM, le \textit{differenze} tra le misure dovrebbero essere distribuite normalmente. Il test è relativamente robusto, specialmente con $n$ grande.
    \item \textbf{Indipendenza delle Osservazioni (tra soggetti):} Le osservazioni \textit{tra} soggetti diversi devono essere indipendenti. (Le osservazioni \textit{dello stesso} soggetto sono ovviamente dipendenti).
\end{enumerate}
Assume particolare importanza un assunto specifico dei disegni RM con 3 o più livelli (ANOVA RM):
\begin{enumerate}
    \setcounter{enumi}{3}
    \item \textbf{Sfericità (o Circolarità):} Le \textbf{varianze delle differenze} tra tutte le possibili coppie di livelli della VI (misure ripetute) devono essere uguali nella popolazione. Es: $\text{Var}(X_1-X_2) = \text{Var}(X_1-X_3) = \text{Var}(X_2-X_3)$. È l'analogo dell'omoschedasticità per i disegni RM.
        \begin{itemize}
            \item \textbf{Verifica:} Si usa il \textbf{Test di Mauchly}.
            \begin{itemize}
                \item Se il test di Mauchly \textbf{NON} è significativo ($p > 0.05$), l'assunto di sfericità è rispettato. Si può usare l'output standard dell'ANOVA RM.
                \item Se il test di Mauchly \textbf{è} significativo ($p \le 0.05$), l'assunto è violato. L'F test standard tende ad essere troppo liberale (troppi errori di I tipo).
            \end{itemize}
            \item \textbf{Correzioni se violato:} Si usano valori F corretti, aggiustando i gradi di libertà ($\df_{\text{TRA Misure}}$ e $\df_{\text{Residua}}$) usando un fattore di correzione chiamato epsilon ($\epsilon$). I p-value associati saranno più conservativi. Correzioni comuni (ordinate dalla più conservativa):
            \begin{itemize}
                \item \textit{Lower-bound} ($\epsilon$ stimato al minimo possibile)
                \item \textit{Greenhouse-Geisser} ($\epsilon_{GG}$ stimato dai dati, conservativa)
                \item \textit{Huynh-Feldt} ($\epsilon_{HF}$ stimato dai dati, meno conservativa)
                \item La scelta tra GG e HF dipende da quanto $\epsilon$ stimato è lontano da 1 (se $\epsilon > 0.75$, HF è preferita, altrimenti GG).
            \end{itemize}
        \end{itemize}
\end{enumerate}

\begin{reflectionbox}
    \item Nell'ANOVA per misure ripetute, il test di Mauchly è utile per valutare quale assunto? (Vedere Domanda 9)
    \item Quale dei seguenti NON è un assunto dell'ANOVA per misure ripetute? (Vedere Domanda 13 - Risposta: Indipendenza dei campioni, perché il campione è uno solo)
\end{reflectionbox}


% ##############################################################
% #############           APPENDICE             ################
% ##############################################################
\newpage % Inizia l'appendice in una nuova pagina
\appendix
\section*{Appendice: Diagramma di Flusso per la Scelta del Test di Ipotesi (Confronto Medie)}
\label{sec:appendice_flowchart}

Questo diagramma aiuta a scegliere il test statistico appropriato per confrontare le medie, basandosi sulle caratteristiche principali del disegno di ricerca e delle variabili coinvolte. Si assume che la \textbf{Variabile Dipendente (VD)} sia sempre \textbf{quantitativa} (scala a intervalli o rapporti) per questi test.

\begin{center}
    \begin{tikzpicture}[
        node distance=1.8cm and 1.5cm, % Vertical and Horizontal distance
        >=latex,
        startstop/.style={rectangle, rounded corners, minimum width=2.5cm, minimum height=1cm, text centered, draw=black, fill=startcolor},
        decision/.style={diamond, minimum width=2cm, minimum height=1cm, text centered, draw=black, aspect=2, fill=decisioncolor, inner sep=1pt},
        process/.style={rectangle, minimum width=3cm, minimum height=1cm, text centered, text width=4cm, draw=black, fill=testcolor, rounded corners},
        arrow/.style={thick,->,>=stealth},
        connector/.style={coordinate} % Style for invisible connector nodes
    ]
    
        % Nodi - Starting top-left
        \node (start) [startstop] {Inizio: Confronto Medie?};
    
        \node (q_ngroups) [decision, below=of start] {Quanti Gruppi/ Misure?};
    
        % --- Path 1: One Group ---
        \node (test_1group) [process, left=2cm of q_ngroups] {Test Z o t per campione singolo (Modulo 2)};
    
        % --- Path 2: Two Groups/Measures ---
        \node (q_indep2) [decision, below left=1.5cm and 0.5cm of q_ngroups] {Indipendenti o Dipendenti?};
        \node (test_t_ind) [process, below left=1.5cm and 0.2cm of q_indep2] {t-test campioni indipendenti (Modulo 3)};
        \node (test_t_dep) [process, below=1.0cm of test_t_ind] {t-test misure ripetute (Modulo 4)};
    
        % --- Path 3: Three+ Groups/Measures ---
        % Position this branch further down and right
        \node (connector3) [connector, right=3cm of q_indep2] {}; % Helper node for positioning
        \node (q_indep3) [decision, below=of connector3] {Indipendenti o Dipendenti?};
    
        % Sub-branch: 3+ Independent
        \node (q_factors_ind) [decision, below=1.0cm of test_t_dep] {Quanti Fattori (VI)?};
        \node (anova_1way) [process, below=of q_factors_ind, yshift=-0.5cm] {ANOVA a una via (Modulo 4)};
        \node (anova_fact) [process, right=1cm of anova_1way] {ANOVA Fattoriale (Modulo 4)};
    
        % Sub-branch: 3+ Dependent
        \node (q_factors_dep) [decision, below right=2.5cm and 0.2cm of q_indep3] {Quanti Fattori (VI)?};
        \node (anova_rm) [process, below=of q_factors_dep, yshift=-0.5cm] {ANOVA misure ripetute (Modulo 4)};
        \node (anova_rm_fact) [process, right=1cm of anova_rm] {ANOVA RM Fattoriale / Mista (Avanzato)};
    
    
        % Frecce
        \draw [arrow] (start) -- (q_ngroups);
    
        % Ramo 1 gruppo
            \draw [arrow] (q_ngroups.west) -- node[above, pos=0.6] {Uno} (test_1group);
        % Ramo 2 gruppi/misure
        \draw [arrow] (q_ngroups.south) -- node[above, sloped, pos=0.4] {Due} (q_indep2);
        \draw [arrow] (q_indep2.west) -- node[above, sloped, pos=0.4] {Indipendenti} (test_t_ind);
        \draw [arrow] (q_indep2.south) |- node[pos=0.35, left] {Dipendenti} (test_t_dep.east);
    
        % Ramo 3+ gruppi/misure - Use orthogonal paths for clarity (Down then Right)
        \draw [arrow] (q_ngroups.east) -- node[right, pos=0.25] {Tre o più} (q_indep3.north);

        \draw [arrow] (q_indep3.west) |- node[above, sloped, pos=0.4] {Indipendenti} (q_factors_ind);
        \draw [arrow] (q_indep3.south) -- node[above, sloped, pos=0.4] {Dipendenti} (q_factors_dep);
    
        % Sotto-rami ANOVA Indipendente
        \draw [arrow] (q_factors_ind.west) -- node[above, sloped, pos=0.4] {Una} (anova_1way);
        \draw [arrow] (q_factors_ind.east) -- node[above, sloped, pos=0.4] {Due+} (anova_fact);
    
        % Sotto-rami ANOVA Dipendente
        \draw [arrow] (q_factors_dep.west) -- node[above, sloped, pos=0.4] {Una} (anova_rm);
        \draw [arrow] (q_factors_dep.east) -- node[above, sloped, pos=0.4] {Due+} (anova_rm_fact);
    
    \end{tikzpicture}
    \end{center}


\textbf{Legenda e Spiegazioni:}
\begin{itemize}
    \item \textbf{Obiettivo:} Il diagramma si focalizza sui test per confrontare le medie ($\mu$) di una variabile dipendente (VD) quantitativa.
    \item \textbf{Numero Gruppi/Misure:}
        \begin{itemize}
            \item \textbf{Uno:} Confronti la media del tuo campione con un valore noto o ipotizzato ($\mu_0$). La scelta tra Z e t dipende dalla conoscenza della varianza della popolazione ($\sigma^2$) e dalla numerosità ($n$) (vedi Modulo 2).
            \item \textbf{Due:} Confronti le medie di due gruppi o due misure.
            \item \textbf{Tre o più:} Confronti le medie di tre o più gruppi o tre o più misure.
        \end{itemize}
    \item \textbf{Indipendenti (Between) vs Dipendenti (Within):}
        \begin{itemize}
            \item \textbf{Indipendenti:} I partecipanti in un gruppo sono diversi da quelli negli altri gruppi (es. maschi vs femmine; gruppo trattamento vs gruppo controllo). Si usano t-test per campioni indipendenti o ANOVA per campioni indipendenti (a una via o fattoriale).
            \item \textbf{Dipendenti/Appaiati (Within):} Gli stessi partecipanti sono misurati più volte (es. pre-test vs post-test; condizione A vs condizione B fatte dagli stessi soggetti) o i partecipanti sono appaiati in base a caratteristiche rilevanti. Si usano t-test per misure ripetute o ANOVA per misure ripetute.
        \end{itemize}
    \item \textbf{Numero di Variabili Indipendenti (Fattori):}
        \begin{itemize}
            \item \textbf{Una:} Hai una sola variabile che definisce i gruppi (es. tipo di trattamento con 3 livelli) o le misure (es. tempo con 3 livelli). Usi ANOVA a una via (between) o ANOVA RM (within).
            \item \textbf{Due o più:} Hai due o più variabili che definiscono i gruppi/misure (es. tipo di trattamento E sesso). Usi ANOVA Fattoriale (between) o ANOVA RM Fattoriale/Mista (within/mista).
        \end{itemize}
    \item \textbf{Nota:} Questo diagramma non copre test per variabili dipendenti non quantitative (es. Chi-quadro) o analisi di relazione/associazione (es. Correlazione, Regressione) che verranno trattate nei prossimi moduli.
\end{itemize}


\end{document}
% ----- FINE DOCUMENTO -----