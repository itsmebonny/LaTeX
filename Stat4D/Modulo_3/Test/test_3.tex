\documentclass[12pt, a4paper]{article}
\usepackage[utf8]{inputenc}
\usepackage[T1]{fontenc}
\usepackage[italian]{babel}
\usepackage{amsmath}
\usepackage{amssymb}
\usepackage{geometry}
\geometry{a4paper, left=2.5cm, right=2.5cm, top=2.5cm, bottom=2.5cm}
\usepackage{parskip}
\usepackage{enumitem}
\usepackage{ FiraSans }
\usepackage{graphicx} % Necessario per \newpage
\usepackage{bm} % Per simboli grassetto eventuali

% ----- Impostazioni Liste -----
\setlist[enumerate,1]{label=\arabic*)} % Numeri per domande principali
\setlist[enumerate,2]{label=(\alph*), leftmargin=*} % Lettere per opzioni

% ----- Macro per simboli comuni -----
% Modulo 1 & 2
\newcommand{\popmean}{\mu} % Media popolazione
\newcommand{\samplemean}{\bar{X}} % Media campione (o M)
\newcommand{\popvar}{\sigma^2} % Varianza popolazione
\newcommand{\samplevar}{s^2} % Varianza campione
\newcommand{\popsd}{\sigma} % Deviazione standard popolazione
\newcommand{\samplesd}{s} % Deviazione standard campione
\newcommand{\stderr}{\sigma_{\samplemean}} % Errore standard media singola
\newcommand{\zscore}{Z} % Punteggio Z
\newcommand{\tscore}{t} % Punteggio t


% Modulo 3
\newcommand{\alphaerr}{\alpha} % Livello alpha
\newcommand{\betaerr}{\beta}  % Errore beta
\newcommand{\power}{1-\betaerr} % Potenza
\newcommand{\Hnull}{H_0} % Ipotesi nulla
\newcommand{\Halt}{H_1} % Ipotesi alternativa
\newcommand{\cohend}{d} % d di Cohen
\newcommand{\df}{df} % gradi di libertà
\newcommand{\Spool}{S^2_{\text{combinata}}} % Varianza combinata (FIXED)
\newcommand{\SEdiff}{S_{\samplemean_1 - \samplemean_2}} % Errore standard differenza (FIXED)



\begin{document}

\begin{center}
    \Large\textbf{Test di Autovalutazione Aggiuntivo - Modulo 3} \\
    \vspace{0.2cm}
    \large\textit{Tecniche di Ricerca e Analisi dei Dati} \\
    \vspace{0.5cm}
    \normalsize{Istruzioni: Scegli l'unica risposta corretta per ogni domanda. Le soluzioni sono disponibili nell'ultima pagina.}
\end{center}
\vspace{1cm}

\begin{enumerate} % Inizio elenco domande principali
    \item Se un ricercatore conclude che un trattamento è efficace (rifiuta $\Hnull$) quando in realtà non lo è ( $\Hnull$ è vera), ha commesso:
    \begin{enumerate} % Inizio opzioni domanda 1
        \item Un errore di I tipo ($\alphaerr$).
        \item Un errore di II tipo ($\betaerr$).
        \item Una decisione corretta.
        \item Un errore nella stima della dimensione dell'effetto.
    \end{enumerate}

    \item La probabilità di commettere un errore di II tipo è indicata con:
    \begin{enumerate}
        \item $\alphaerr$
        \item $1-\alphaerr$
        \item $\betaerr$
        \item $1-\betaerr$
    \end{enumerate}

    \item La Potenza Statistica ($1-\betaerr$) rappresenta la probabilità di:
    \begin{enumerate}
        \item Accettare $\Hnull$ quando è vera.
        \item Rifiutare $\Hnull$ quando è vera.
        \item Accettare $\Hnull$ quando è falsa.
        \item Rifiutare $\Hnull$ quando è falsa (cioè, rilevare un effetto reale).
    \end{enumerate}

    \item Se un ricercatore decide di usare un livello di significatività $\alphaerr$ più basso (es. 0.01 invece di 0.05), cosa accade generalmente alla potenza statistica del test (a parità di altre condizioni)?
    \begin{enumerate}
        \item Aumenta.
        \item Diminuisce.
        \item Rimane invariata.
        \item Diventa uguale ad $\alphaerr$.
    \end{enumerate}

    \item Quale dei seguenti fattori, se aumentato, tende ad \textbf{aumentare} la potenza statistica di un test?
    \begin{enumerate}
        \item La probabilità dell'errore di II tipo ($\betaerr$).
        \item L'ampiezza del campione ($n$).
        \item La varianza della popolazione ($\sigma^2$).
        \item Il livello di protezione ($1-\alphaerr$).
    \end{enumerate}

    \item La Dimensione dell'Effetto ($\cohend$ di Cohen) è un indice che esprime:
    \begin{enumerate}
        \item La probabilità che il risultato sia dovuto al caso.
        \item La numerosità del campione utilizzata nello studio.
        \item La grandezza standardizzata della differenza tra medie o della forza di un'associazione.
        \item Il livello di significatività $\alphaerr$ scelto dal ricercatore.
    \end{enumerate}

    \item Un valore di $\cohend$ di Cohen pari a 0.50 è convenzionalmente considerato indicativo di un effetto:
    \begin{enumerate}
        \item Piccolo.
        \item Medio.
        \item Grande.
        \item Nullo.
    \end{enumerate}

    \item L'analisi della potenza   `a priori' viene condotta solitamente per:
    \begin{enumerate}
        \item Calcolare la dimensione dell'effetto dopo aver raccolto i dati.
        \item Determinare la numerosità campionaria necessaria prima di iniziare lo studio.
        \item Scegliere il livello di significatività $\alphaerr$.
        \item Verificare l'assunto di normalità.
    \end{enumerate}

    \item Il test t per campioni indipendenti è appropriato per confrontare:
    \begin{enumerate}
        \item La media di un singolo campione con un valore noto della popolazione.
        \item Le medie di una variabile quantitativa tra due gruppi di soggetti diversi e non correlati.
        \item Le medie della stessa variabile misurata due volte sugli stessi soggetti.
        \item Le frequenze di una variabile categorica tra due gruppi.
    \end{enumerate}

    \item Qual è l'ipotesi nulla ($\Hnull$) tipica per un test t per campioni indipendenti?
    \begin{enumerate}
        \item $\mu_1 \neq \mu_2$
        \item $\mu_1 = \mu_2$ (o $\mu_1 - \mu_2 = 0$)
        \item $\bar{X}_1 = \bar{X}_2$
        \item $\cohend > 0$
    \end{enumerate}

    \item La distribuzione campionaria di riferimento per il test t per campioni indipendenti è:
    \begin{enumerate}
        \item La distribuzione normale standard (Z).
        \item La distribuzione delle medie campionarie ($\bar{X}$).
        \item La distribuzione F di Fisher.
        \item La distribuzione delle differenze tra le medie campionarie.
    \end{enumerate}

    \item La stima della varianza combinata ($\Spool$) nel test t per campioni indipendenti:
    \begin{enumerate}
        \item È necessaria solo se le numerosità dei campioni sono uguali.
        \item Serve a stimare l'unica varianza comune assunta per le due popolazioni (omoschedasticità).
        \item È uguale alla media delle due varianze campionarie, indipendentemente dalle numerosità.
        \item Viene utilizzata solo se l'assunto di normalità è violato.
    \end{enumerate}

    \item L'errore standard della differenza tra le medie ($S_{\samplemean_1 - \samplemean_2}$):
    \begin{enumerate}
        \item Misura la differenza media osservata tra i due campioni.
        \item È il denominatore nella formula del test t per campioni indipendenti.
        \item Aumenta all'aumentare della numerosità dei campioni.
        \item È uguale alla radice quadrata della differenza tra le varianze campionarie.
    \end{enumerate}

    \item L'assunto di omoschedasticità nel test t per campioni indipendenti richiede che:
    \begin{enumerate}
        \item Le medie delle due popolazioni siano uguali.
        \item Le varianze delle due popolazioni siano uguali.
        \item Le distribuzioni campionarie siano normali.
        \item I campioni abbiano la stessa numerosità.
    \end{enumerate}

    \item Il test di Levene o il test F di Fisher (rapporto tra varianze) sono spesso usati per verificare quale assunto del test t per campioni indipendenti?
    \begin{enumerate}
        \item Indipendenza delle osservazioni.
        \item Normalità della distribuzione della variabile dipendente.
        \item Omogeneità delle varianze (omoschedasticità).
        \item Linearità della relazione.
    \end{enumerate}

    \item Il test t per campioni indipendenti è considerato   `robusto'a moderate violazioni dell'assunto di:
    \begin{enumerate}
        \item Indipendenza.
        \item Normalità (specialmente con campioni grandi).
        \item Omogeneità delle varianze (solo se le numerosità sono molto diverse).
        \item Presenza di outlier.
    \end{enumerate}

    \item Per calcolare la stima della dimensione dell'effetto ($d$ di Cohen) dopo un test t per campioni indipendenti, si utilizza solitamente la differenza tra le medie campionarie divisa per:
    \begin{enumerate}
        \item L'errore standard della differenza.
        \item La radice quadrata della varianza combinata (stima della deviazione standard comune).
        \item La media delle deviazioni standard dei due campioni.
        \item Il valore t calcolato.
    \end{enumerate}

    \item In un test t per campioni indipendenti, i gradi di libertà ($df$) sono calcolati come:
    \begin{enumerate}
        \item $N_1 - 1$
        \item $N_2 - 1$
        \item $N_1 + N_2 - 1$
        \item $N_1 + N_2 - 2$
    \end{enumerate}

    \item Un risultato   `statisticamente significativo'(es. p < 0.05) in un test t per campioni indipendenti indica che:
    \begin{enumerate}
        \item La differenza osservata tra le medie campionarie è sicuramente grande e importante.
        \item È improbabile che una differenza così grande (o più grande) sia dovuta solo al caso, se le medie delle popolazioni fossero uguali.
        \item L'ipotesi nulla è stata dimostrata vera con certezza.
        \item L'assunto di omogeneità delle varianze è stato rispettato.
    \end{enumerate}

     \item L'analisi della potenza   `a posteriori'è utile, ad esempio, quando:
    \begin{enumerate}
        \item Si deve decidere quanti soggetti reclutare per uno studio.
        \item Si ottiene un risultato statisticamente significativo e si vuole confermarlo.
        \item Si ottiene un risultato non statisticamente significativo e si vuole capire se il test aveva sufficiente   `forza'per rilevare un eventuale effetto reale.
        \item Si deve calcolare il livello di significatività $\alphaerr$.
    \end{enumerate}


\end{enumerate} % Fine elenco domande principali

\newpage % Inizia una nuova pagina per le soluzioni

\begin{center}
    \Large\textbf{Griglia delle Soluzioni} \\
    \vspace{0.5cm}
    \normalsize{(Test di Autovalutazione Aggiuntivo - Modulo 3)}
\end{center}
\vspace{1cm}

\begin{enumerate}[leftmargin=*, label=\arabic*.]
    \item (a) Un errore di I tipo ($\alphaerr$) (Rifiutare $\Hnull$ quando è vera).
    \item (c) $\betaerr$.
    \item (d) Rifiutare $\Hnull$ quando è falsa (cioè, rilevare un effetto reale).
    \item (b) Diminuisce (Se $\alphaerr$ è più piccolo, la regione di rifiuto si restringe, è più difficile rifiutare $\Hnull$, quindi aumenta la probabilità di errore $\betaerr$ e diminuisce la potenza $1-\betaerr$).
    \item (b) L'ampiezza del campione ($n$) (Campioni più grandi danno stime più precise e riducono l'errore standard, aumentando la potenza).
    \item (c) La grandezza standardizzata della differenza tra medie o della forza di un'associazione.
    \item (b) Medio.
    \item (b) Determinare la numerosità campionaria necessaria prima di iniziare lo studio.
    \item (b) Le medie di una variabile quantitativa tra due gruppi di soggetti diversi e non correlati.
    \item (b) $\mu_1 = \mu_2$ (o $\mu_1 - \mu_2 = 0$) (L'ipotesi nulla postula l'assenza di differenza tra le medie delle popolazioni).
    \item (d) La distribuzione delle differenze tra le medie campionarie (la cui forma è descritta dalla distribuzione t di Student in questo caso).
    \item (b) Serve a stimare l'unica varianza comune assunta per le due popolazioni (omoschedasticità).
    \item (b) È il denominatore nella formula del test t per campioni indipendenti (misura la variabilità attesa della differenza tra medie).
    \item (b) Le varianze delle due popolazioni siano uguali.
    \item (c) Omogeneità delle varianze (omoschedasticità).
    \item (b) Normalità (specialmente con campioni grandi). (È molto sensibile alla violazione dell'indipendenza).
    \item (b) La radice quadrata della varianza combinata (stima della deviazione standard comune).
    \item (d) $N_1 + N_2 - 2$.
    \item (b) È improbabile che una differenza così grande (o più grande) sia dovuta solo al caso, se le medie delle popolazioni fossero uguali. (Non dice nulla sulla grandezza   `pratica'dell'effetto).
    \item (c) Si ottiene un risultato non statisticamente significativo e si vuole capire se il test aveva sufficiente   `forza'per rilevare un eventuale effetto reale.
\end{enumerate}

\end{document}
% ----- FINE DOCUMENTO -----