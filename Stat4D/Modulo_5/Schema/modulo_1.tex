\documentclass[12pt, a4paper]{article}
\usepackage[utf8]{inputenc}
\usepackage[T1]{fontenc}
\usepackage[italian]{babel}
\usepackage{amsmath}
\usepackage{amssymb} % Per spunte e simboli
\usepackage{graphicx} % Per includere immagini (se servisse)
\usepackage{geometry}
\geometry{a4paper, left=2.5cm, right=2.5cm, top=2.5cm, bottom=2.5cm} % Margini leggermente più ampi
\usepackage{parskip}
\usepackage{enumitem}
\usepackage{tikz} % Per eventuali diagrammi futuri semplici
\usepackage{framed} % Per i box delle domande
\usepackage{ FiraSans } % Un font più moderno e leggibile
\usepackage{xcolor} % Per colori

% ----- Definizioni Colori e Stili -----
\definecolor{boxbgcolor}{rgb}{0.95, 0.95, 1.0} % Azzurrino chiaro per i box
\definecolor{boxtitlecolor}{rgb}{0.1, 0.1, 0.6} % Blu scuro per titoli box
\newenvironment{reflectionbox}{%
    \begin{framed}\par\medskip\noindent
    \textbf{\color{boxtitlecolor}Domande per Riflettere (Basate sul Test)} \par
    \begin{itemize}[leftmargin=*, label=$\blacktriangleright$]
}{%
    \end{itemize}\par\medskip
    \end{framed}
}

\setlist{nosep}
\renewcommand{\labelitemi}{$\bullet$}

% ----- INIZIO DOCUMENTO -----
\begin{document}

\begin{center}
    \Large\textbf{Schemino Modulo 1: Introduzione alla Ricerca e all'Analisi Dati} \\
    \vspace{0.5cm}
    \large\textit{(Concetti chiave e spunti dal test di autovalutazione)}
\end{center}

\section*{1. Introduzione alla Ricerca in Psicologia}
\begin{itemize}
    \item \textbf{Oggetto:} Studio del \textbf{comportamento} umano (come varia, cause, previsione).
    \item \textbf{Tipi Principali:}
        \begin{itemize}
            \item \textbf{Ricerca di Base (Pura):} Aumentare la \textbf{conoscenza teorica}. (Es: Come funziona la memoria?)
            \item \textbf{Ricerca Applicata:} Trovare \textbf{soluzioni pratiche} a problemi. (Es: Quale terapia è più efficace?)
        \end{itemize}
    \item \textbf{Metodologia:} Insieme delle regole e procedure usate.
    \item \textbf{Carattere Scientifico:} Basato su \textbf{dati empirici} (osservabili, misurabili) e \textbf{metodo sperimentale} (controllo ipotesi).
\end{itemize}

\begin{reflectionbox}
    \item Qual è l'obiettivo principale della \textbf{ricerca di base}? (Vedi Q16)
\end{reflectionbox}

\section*{2. Il Processo di Ricerca}
Percorso \textbf{circolare} standard (vedi Fig 1.1, pag 5):
\begin{enumerate}
    \item \textbf{Identificazione del problema:} Trovare la domanda di ricerca, basata su teorie o lacune esistenti. (\textbf{Prima fase!})
    \item \textbf{Pianificazione del disegno:} Come condurre lo studio (chi, cosa, come misurare).
    \item \textbf{Osservazioni/Raccolta Dati:} Mettere in atto il piano e raccogliere i dati.
    \item \textbf{Analisi dei Dati:} Usare la statistica per elaborare i dati.
    \item \textbf{Interpretazione dei Dati:} Dare un significato ai risultati (supportano l'ipotesi?).
    \item \textbf{Comunicazione dei Risultati:} Condividere le scoperte (articoli, report).
\end{enumerate}

\begin{reflectionbox}
    \item Qual è la \textbf{primissima fase} del processo di ricerca? (Vedi Q1)
    \item Come può essere descritta la relazione tra le varie fasi (specialmente tra campione e popolazione)? (Vedi Q19)
\end{reflectionbox}

\section*{3. Popolazione, Campione, Parametri, Indici}
\begin{itemize}
    \item \textbf{Popolazione (N):} \textbf{TUTTO} l'insieme di individui/elementi che interessano al ricercatore. (Es: Tutti gli studenti UniCusano). Spesso troppo grande da studiare interamente (\textit{Indagine Totale}).
    \item \textbf{Campione (n):} Un \textbf{sottoinsieme selezionato} dalla popolazione, che viene effettivamente studiato (\textit{Indagine Campionaria}). (Es: 100 studenti UniCusano).
    \item \textbf{Rappresentatività:} Il campione deve "assomigliare" alla popolazione per poter generalizzare i risultati. Si ottiene meglio con estrazione \textbf{casuale}.
    \item \textbf{Parametro:} Valore numerico che descrive una caratteristica della \textbf{Popolazione} (Es: Età media di \textit{tutti} gli studenti UniCusano). \textit{Obiettivo della ricerca, spesso sconosciuto.}
    \item \textbf{Indice Statistico (o Statistica):} Valore numerico calcolato sul \textbf{Campione} (Es: Età media dei \textit{100 studenti} studiati). \textit{Si calcola per stimare il parametro.}
    \item \textbf{Errore di Campionamento:} Differenza naturale e inevitabile tra indice statistico (campione) e parametro (popolazione). \textbf{Non è uno sbaglio!}, ma variazione casuale dovuta al fatto di studiare solo una parte. Si verifica nelle \textit{indagini campionarie}.
    \item \textbf{Modalità di Estrazione Campione:}
        \begin{itemize}
            \item \textit{Casuale Semplice:} Ogni individuo ha la stessa probabilità di essere scelto.
            \item \textit{Casuale Stratificata:} Popolazione divisa in sottogruppi (strati, es. per età), poi estrazione casuale da ogni strato.
        \end{itemize}
\end{itemize}

\begin{reflectionbox}
    \item Un indice statistico (es. la media del campione) a cosa si riferisce? (Vedi Q3)
    \item Qual è la relazione corretta tra parametri e indici statistici? (Vedi Q4)
    \item Per essere rappresentativo, come dovrebbe essere idealmente scelto un campione? (Vedi Q5)
    \item L'errore di campionamento è tipico di quale tipo di indagine? (Vedi Q6)
    \item Le indagini che studiano l'intera popolazione come si chiamano? (Vedi Q11)
\end{reflectionbox}

\section*{4. Statistica Descrittiva e Inferenziale}
\begin{itemize}
    \item \textbf{Statistica Descrittiva:}
        \begin{itemize}
            \item \textbf{Scopo:} \textbf{Riassumere, organizzare, descrivere} i dati raccolti (del campione o popolazione). Rendere i dati comprensibili.
            \item \textbf{Strumenti:} Tabelle, \textbf{grafici}, indici numerici (media, mediana, moda, deviazione standard, ecc.).
        \end{itemize}
    \item \textbf{Statistica Inferenziale:}
        \begin{itemize}
            \item \textbf{Scopo:} Usare i dati del campione per fare \textbf{inferenze} (deduzioni, generalizzazioni) sulla \textbf{popolazione}.
            \item \textbf{Strumenti:} Stima dei parametri, verifica delle ipotesi.
            \item \textbf{Logica:} Aiuta a capire se le differenze/relazioni osservate nel campione sono "reali" (generalizzabili alla popolazione) o dovute solo all'errore di campionamento (caso).
        \end{itemize}
\end{itemize}

\begin{reflectionbox}
    \item Quale strumento \textbf{non} è tipicamente usato per fare inferenze sulla popolazione? (Vedi Q20)
    \item Qual è l'utilità principale della statistica \textbf{inferenziale}? (Vedi Q29)
\end{reflectionbox}

\section*{5. Variabili e Costrutti}
\begin{itemize}
    \item \textbf{Variabile:} Caratteristica o condizione che \textbf{varia} (assume valori diversi) tra individui o situazioni. (Es: Età, altezza, voto, sesso, tipo di trattamento).
    \item \textbf{Costrutto:} Concetto \textbf{teorico/astratto}, non direttamente osservabile (Es: Intelligenza, ansia, motivazione).
    \item \textbf{Operazionalizzazione:} Processo fondamentale per definire un costrutto in termini di \textbf{comportamenti/indicatori osservabili e misurabili} (variabili). Trasforma l'astratto in concreto. (Es: Operazionalizzo l'"ansia" misurando il battito cardiaco).
\end{itemize}

\begin{reflectionbox}
    \item L'"età anagrafica" è una variabile o un costrutto? (Vedi Q9)
    \item Per rendere misurabile un costrutto (es. "felicità"), a cosa devo riferirmi? (Vedi Q24)
\end{reflectionbox}

\section*{6. Misurazione e Scale di Misura}
\begin{itemize}
    \item \textbf{Misurazione:} Assegnare valori (numeri o categorie) a eventi/individui secondo regole precise.
    \item \textbf{Scale di Misura} (determinano le analisi statistiche possibili):
        \begin{itemize}
            \item \textbf{Nominale (Qualitativa):} Categorie con nomi diversi, solo classificazione (\textit{uguaglianza/differenza}). No ordine. (Es: Sesso, Colore occhi, Nazionalità).
            \item \textbf{Ordinale (Qualitativa):} Categorie ordinate (\textit{uguaglianza/differenza + ordine}). Distanza tra categorie non definita/costante. (Es: Titolo di studio, Classifica, Livello di soddisfazione "poco/medio/alto").
            \item \textbf{A Intervalli (Quantitativa):} Valori numerici ordinati con \textbf{intervalli uguali}. Si possono fare somme/sottrazioni di differenze. Lo \textbf{zero è arbitrario} (non indica assenza). (Es: Temperatura °C, QI).
            \item \textbf{A Rapporti (Quantitativa):} Come intervalli, ma con uno \textbf{zero assoluto} (indica assenza totale). Permette tutte le operazioni, inclusi i \textbf{rapporti} (doppio, metà). (Es: Altezza, Peso, Età, Tempo di reazione, Numero di errori).
        \end{itemize}
\end{itemize}

\begin{reflectionbox}
    \item Quale scala ha solo le proprietà di "differenza" e "ordine"? (Vedi Q10)
    \item La variabile "colore degli occhi" su che tipo di scala si misura? (Vedi Q23)
    \item In quale scala lo zero significa veramente "quantità nulla"? (Vedi Q25)
    \item La scala a intervalli permette di dire che un valore è "il doppio" di un altro? (Vedi Q30)
\end{reflectionbox}

\section*{7. Classificazione delle Variabili}
Oltre alla scala di misura, le variabili si classificano per:
\begin{itemize}
    \item \textbf{Precisione/Natura:}
        \begin{itemize}
            \item \textbf{Discrete:} Valori separati, "a salti" (spesso numeri interi, ma anche categorie). Non ci sono valori intermedi possibili. (Es: Numero figli, Numero esami superati, Sesso, Professione). Possono essere sia Qualitative (nominale, ordinale) che Quantitative (intervalli, rapporti).
            \item \textbf{Continue:} Possono assumere qualsiasi valore in un intervallo (anche con decimali), dipende solo dalla precisione dello strumento. (Es: Altezza, Peso, Tempo). Sono Quantitative (intervalli, rapporti).
        \end{itemize}
    \item \textbf{Ruolo nella Ricerca:}
        \begin{itemize}
            \item \textbf{Variabile Indipendente (VI):} Quella che il ricercatore \textbf{manipola} (negli esperimenti) o usa per definire i gruppi da confrontare. È la presunta \textbf{causa}. Può essere sia qualitativa (es. tipo di terapia) che quantitativa (es. dose di farmaco).
            \item \textbf{Variabile Dipendente (VD):} Quella che viene \textbf{osservata e misurata} per vedere l'effetto della VI. È il presunto \textbf{effetto}. Solitamente quantitativa.
            \item \textbf{Variabili Confuse/Confondenti Operazionali:} Non previste, ma \textbf{intrinsecamente legate} alla VI o alla sua operazionalizzazione (covariano con essa). Creano ambiguità nell'interpretazione.
            \item \textbf{Variabili di Disturbo/Confondenti Procedurali:} Estranee alla VI, ma \textbf{covariano} con essa per errori metodologici (artefatti), influenzando la VD. Devono essere controllate.
        \end{itemize}
\end{itemize}

\begin{reflectionbox}
    \item La Variabile Indipendente può essere solo quantitativa? (Vedi Q2)
    \item Le variabili limitate a numeri interi (come "numero di gatti posseduti") come si chiamano? (Vedi Q13)
    \item Le variabili discrete possono essere sia qualitative che quantitative? (Vedi Q26)
    \item Le Variabili Indipendenti sono quelle misurate o quelle manipolate? (Vedi Q14)
    \item In uno studio sull'effetto di una nuova dieta (VI) sul peso (VD), l'attività fisica non controllata che differenza tra i gruppi potrebbe essere una variabile...? (Vedi Q15 - Attenzione alla definizione esatta del testo!)
    \item Se studio l'effetto di un farmaco (VI) sulla pressione sanguigna (VD), cos'è la pressione sanguigna? E il farmaco? (Ispirato a Q21)
\end{reflectionbox}

\section*{8. I Disegni di Ricerca in Psicologia}
Classificati in base al \textbf{grado di controllo} del ricercatore:
\begin{itemize}
    \item \textbf{Metodi Descrittivi:} Descrivono le variabili così come sono. Controllo minimo/nullo. (Es: Sondaggi, osservazione naturalistica).
    \item \textbf{Metodo Correlazionale:} Misura due (o più) variabili per vedere se esiste una \textbf{relazione} (covariazione) tra loro. \textbf{Non stabilisce causa-effetto}. (Es: C'è relazione tra ore di studio e voti?).
    \item \textbf{Metodo Sperimentale:} Obiettivo: stabilire relazioni \textbf{causa-effetto}. Massimo controllo. Caratteristiche chiave:
        \begin{itemize}
            \item \textbf{Manipolazione} della VI (il ricercatore crea i diversi livelli/condizioni).
            \item \textbf{Controllo} delle variabili estranee (disturbo, concorrenti, ambientali) per evitare confusioni. Tecniche: \textbf{assegnazione casuale} dei partecipanti alle condizioni, mantenere costanti le condizioni ambientali.
        \end{itemize}
    \item \textbf{Metodo Quasi-Sperimentale:} Simile all'esperimento (confronto tra gruppi), ma il ricercatore \textbf{non può manipolare} pienamente la VI (es. è una caratteristica pre-esistente come sesso, età) o non può fare l'assegnazione casuale. Controllo minore, più difficile stabilire causa-effetto certa.
    \item \textbf{Disegni Specifici:}
        \begin{itemize}
            \item \textbf{Between-subjects (o Tra Gruppi):} Gruppi diversi di partecipanti sono assegnati a condizioni diverse.
            \item \textbf{Within-subjects (o Entro Gruppi / Misure Ripetute):} Gli stessi partecipanti sono sottoposti a tutte le condizioni (o misurati più volte, es. prima e dopo un trattamento).
        \end{itemize}
    \item \textbf{Condizione di Controllo:} Gruppo/condizione che non riceve il trattamento sperimentale (o riceve un placebo), serve come baseline per il confronto.
\end{itemize}

\begin{reflectionbox}
    \item Quale disegno di ricerca offre il maggior livello di controllo? (Vedi Q8)
    \item Quali sono le due caratteristiche fondamentali di un vero esperimento? (Vedi Q17)
    \item Quale tecnica aiuta a controllare le differenze individuali tra partecipanti (variabili concorrenti) in un esperimento? (Vedi Q18)
    \item Se misuro l'ansia e la depressione in un gruppo per vedere se sono associate, che studio sto facendo? (Vedi Q27)
    \item Se testo la memoria degli stessi anziani prima e dopo un training cognitivo, che disegno sto usando (between o within)? (Vedi Q22)
\end{reflectionbox}

\section*{9. L'Errore di Misurazione}
Ogni misura contiene una parte "vera" e una parte di errore: $X_i = V_i + e_i$. L'errore \textbf{non è uno sbaglio}, ma imprecisione intrinseca.
\begin{itemize}
    \item \textbf{Errore Sistematico:} Costante, influenza tutte le misure nello \textbf{stesso modo} (es. bilancia starata che aggiunge sempre 0.5 kg). Legato a difetti dello strumento/metodo. Riducibile identificando la causa.
    \item \textbf{Errore Casuale (o Accidentale):} Imprevedibile, varia da misura a misura (a volte in eccesso, a volte in difetto). Dovuto a fattori incontrollabili (ambientali, fluttuazioni attenzione). Non eliminabile totalmente, ma tende a compensarsi su molte misure. La statistica inferenziale ne tiene conto.
\end{itemize}

\begin{reflectionbox}
    \item Quale affermazione sugli errori di misura è corretta riguardo alla loro eliminazione? (Vedi Q28)
\end{reflectionbox}

\section*{10. La Presentazione della Ricerca}
Comunicare i risultati è fondamentale per il progresso scientifico (replicabilità, confronto). Struttura tipica di un articolo scientifico:
\begin{itemize}
    \item \textbf{Introduzione:} Background teorico, letteratura, ipotesi, obiettivi.
    \item \textbf{Materiali e Metodi:} Descrizione \textbf{dettagliatissima} di partecipanti, strumenti, procedure, disegno, analisi statistiche. \textbf{Cruciale per la replicabilità!}
    \item \textbf{Risultati:} Presentazione oggettiva dei dati e delle analisi (indici descrittivi, test inferenziali, tabelle, grafici).
    \item \textbf{Discussione:} Interpretazione dei risultati alla luce delle ipotesi e della letteratura, conclusioni, limiti, prospettive future.
\end{itemize}

\begin{reflectionbox}
    \item Quale sezione dell'articolo deve essere descritta nel minimo dettaglio per permettere ad altri di rifare lo studio? (Vedi Q12)
\end{reflectionbox}

\section*{11. Cenni di Notazione Statistica}
Simboli comuni:
\begin{itemize}
    \item \textbf{X, Y:} Indicano le variabili misurate.
    \item \textbf{N:} Numero totale di soggetti/osservazioni nella \textbf{Popolazione}.
    \item \textbf{n:} Numero totale di soggetti/osservazioni nel \textbf{Campione}.
    \item \textbf{X\textsubscript{i} (o Y\textsubscript{i}):} Rappresenta il punteggio/valore dell'\textit{i}-esimo soggetto (individuo) sulla variabile X (o Y). L'indice 'i' varia da 1 a N (o n).
\end{itemize}

\begin{reflectionbox}
    \item Con quale lettera si indica solitamente la numerosità (quanti soggetti) di un campione? (Vedi Q7)
\end{reflectionbox}



\end{document}
% ----- FINE DOCUMENTO -----