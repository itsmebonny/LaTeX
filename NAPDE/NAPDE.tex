\documentclass[a4paper,11pt]{article}
\usepackage{amssymb}
\usepackage{amsmath}
\usepackage{hhline}
\usepackage{hyperref}
\usepackage{mathtools}
\usepackage{bm}
\usepackage[margin=2cm]{geometry}

\usepackage{amsthm}

\usepackage{tabularx}
\usepackage{graphicx}
\usepackage{physics}
\usepackage{textcomp}

\usepackage{mdframed}
\usepackage{xcolor}

\usepackage[makeroom]{cancel}
\usepackage{mathrsfs}


\setlength\parindent{0pt}

\newlength\mylength
\setlength\mylength{0.1cm}
\newcolumntype{Y}{>{\Centering\arraybackslash}X}
\newtheoremstyle{break}
  {\partopsep}{\topsep}%  
  {\normalfont}{}
  {\bfseries}{}%
  {\newline}{}%
  \theoremstyle{break}
\newtheorem{theorem}{Theorem}[section]
\newtheorem{corollary}{Corollary}[section]
\newtheorem{proposition}{Proposition}[section]
\newtheorem{remark}{Remark}[section]
\newtheorem{lemma}{Lemma}[section]
\renewcommand*{\proofname}{\textbf{Proof}}
\renewcommand\qedsymbol{$\bigstar$}
\renewcommand*{\grad}{\nabla}
\renewcommand*{\div}{\text{div}}
\newtheorem{definition}{Definition}[section]
\renewcommand\labelenumi{(\theenumi)}
\newcommand{\vect}[1]{\textbf{#1}}

\let\oldemptyset\emptyset
\let\emptyset\varnothing
\let\oldepsilon\epsilon
\let\epsilon\varepsilon
\let\oldphi\phi
\let\phi\varphi

\newcommand*{\txt}[1]{\text{#1}}

\newcommand{\ind}{\perp\!\!\!\!\perp} 
\newcommand{\measurespace}{(X, \mathcal{M}, \mu)}
\newcommand{\sigalg}{\sigma\mbox{-algebra}}
\newcommand{\boreal}{\mathcal{B}(\mathbb{R})}
\renewcommand{\real}{\mathbb{R}}
\renewcommand{\natural}{\mathbb{N}}
\newcommand{\barreal}{\overline{\mathbb{R}}}
\newcommand{\code}[1]{\texttt{#1}}
\newcommand{\xdownarrow}[1]{%
  {\left\downarrow\vbox to #1{}\right.\kern-\nulldelimiterspace}
}
\newcommand{\xuparrow}[1]{%
  {\left\uparrow\vbox to #1{}\right.\kern-\nulldelimiterspace}
}
\newcommand{\arrvline}{\hfil\kern\arraycolsep\vline\kern-\arraycolsep\hfilneg}
\newcommand{\esssup}{\text{ess}\, \text{sup}}
\newcommand{\normdot}{\norm{\cdot}}
\newcommand{\scalardot}{\langle \cdot,\cdot \rangle}
\newcommand{\scalarproduct}[2]{\langle #1,#2 \rangle}
\newcommand*{\limited}[1]{\vert_{#1}}
\newcommand{\interior}[1]{%
 {\kern0pt#1}^{\mathrm{o}}%
}
\newcommand*{\seminorm}[2]{\left\lvert#1\right\rvert_{#2}}
\newcommand*{\threenorm}[3][\null]{\left\lvert\mspace{-2mu}\left\lvert\mspace{-2mu}\left\lvert#2\right\rvert\mspace{-2mu}\right\rvert\mspace{-2mu}\right\rvert_{#3}^{#1}}

\newcommand{\ltwonorm}[1]{\norm{#1}_{L^2(\Omega)}}
\newcommand{\honenorm}[1]{\norm{#1}_{H^1(\Omega)}}

\newcommand{\jump}[1]{\left[\!\left[#1\right]\!\right]}
\newcommand{\average}[1]{\left\{\mspace{-6.5mu}\left\{#1\right\}\mspace{-6.5mu}\right\}}

\newcommand{\element}{\mathcal{K}}
\newcommand{\triangulation}{\mathcal{T}_h}

\newcommand{\boundary}{\partial\Omega}
\newcommand{\eit}{{\element\in\triangulation}}
\newcommand{\scalingfactor}{\tau_{\element}}
\newcommand{\peclet}{\mathbb{P}e}

\newcommand{\vecbabs}{\abs{\vect{b}}}
\newcommand{\loperator}{\mathcal{L}}
\newcommand{\xvec}{\vect{x}}
\newcommand{\tstep}{\Delta t}
\newcommand{\onehalf}{\frac{1}{2}}
\newcommand{\bilinear}{a(\cdot, \cdot)}
\newmdenv[backgroundcolor=blue!20]{background}


\def\stackbelow#1#2{\underset{\displaystyle\overset{\displaystyle\shortparallel}{#2}}{#1}}
\def\stackbelowlittle#1#2{\underset{\textstyle\overset{\textstyle\shortparallel}{#2}}{#1}}



\long\def\symbolfootnotemark[#1]#2{\begingroup%
\def\thefootnote{\fnsymbol{footnote}}\footnotetext[#1]{#2}\footnotemark[#1]\endgroup}

\long\def\symbolfootnotetext[#1]#2{\begingroup%
\def\thefootnote{\fnsymbol{footnote}}\footnotetext[#1]{#2}\endgroup}


\numberwithin{equation}{section}





\begin{document}
\title{Numerical Analysis for Partial Differential Equations}
\author{Andrea Bonifacio}
\date{\today}
\maketitle
\section{Boundary Value Problems}
\subsection{Weak Formulation}
Let's consider a problem 
\begin{equation}
    \begin{cases}
        \mathcal{L}u = f & \text{in }\Omega \\
        + \text{B.C.} & \text{on }\partial\Omega
    \end{cases}
\end{equation}
\begin{itemize}
    \item \(\Omega\): open bounded domain in \(\real^d\), with \(d = 2,3\)
    \item \(\partial\Omega\): boundary of \(\Omega\)
    \item \(f\): given 
    \item B.C. accordingly to \(\mathcal{L}\)
    \item \(\mathcal{L}\): \(2^{\text{nd}}\) order operator, like:
    \begin{enumerate}
        \item \(\mathcal{L}u = -\div(\mu\grad u) +\vect{b}\cdot\grad u + \sigma u\) {\hspace*{\fill} (non-conservative form)}
        \item \(\mathcal{L}u = -\div(\mu \grad u) + \div(\vect{b}u) + \sigma u\) {\hspace*{\fill} (conservative form)}
    \end{enumerate}
    \begin{itemize}
        \item \(\mu \in L^\infty(\Omega), \quad \mu(\vect{x})\geq \mu_0 > 0\) {\hspace*{\fill} uniformly bounded from below}
        \item \(\vect{b} \in (L^\infty(\Omega))^d\) {\hspace*{\fill} transport term}
        \item \(\sigma \in L^2(\Omega)\) {\hspace*{\fill} reaction term}
        \item \(f \in L^2(\Omega)\) {\hspace*{\fill} can be less regular}
    \end{itemize}
\end{itemize}


\subsection*{General elliptic problems}
Consider 
\begin{equation}
    \begin{cases}
        -\div(\mu \grad u)+ \vect{b}\cdot \grad u + \sigma u = f& \text{in } \Omega \quad g \in L^2(\Gamma_N) \\
        u = 0 & \txt{on }\Gamma_D \quad \partial\Omega = \Gamma_D \cup \Gamma_N \\

        \mu \grad u \cdot \vect{n} = g & \txt{on } \Gamma_N \quad \interior{\Gamma_D} \cap \interior{\Gamma_N} = \emptyset
    \end{cases}
\end{equation}
Suppose that \(f \in L^2(\Omega)\) and \(\mu, \sigma \in L^\infty(\Omega)\). Also suppose that \(\exists \; \mu_0 > 0 \txt{ s.t. } \mu(\vect{x}) \geq \mu_0\), and \(\sigma(\vect{x}) \geq 0\) a.e. on \(\Omega\).  
Then, given a test function \(v\), we multiply the equation by \(v\), and integrate on the domain \(\Omega\)
\begin{equation*}
    \int_\Omega \left[-\div(\mu \grad u)+ \vect{b}\cdot \grad u + \sigma u\right] v \,  = \int_\Omega f v 
\end{equation*}
By applying Green's formula 
\begin{equation*}
    \underbrace{\int_\Omega \mu\grad u \cdot \grad v  + \int_\Omega \vect{b} \cdot \grad u v + \int_\Omega \sigma u v}_{=: a(u,v)} = \int_\Omega f v +\underbrace{\int_{\Gamma_D} \mu \grad u \cdot \vect{n} v}_{= 0 \txt{ if } v\vert_{\Gamma_D} = 0} + \int_{\Gamma_N} \underbrace{\mu \grad u \cdot \vect{n}}_{= g} v 
\end{equation*}
So the weak formulation of the problem is 
\begin{equation}
    \begin{cases}
        \txt{Find } u \in V & V = \left\{v \in H^1(\Omega), v\vert_{\Gamma_D} = 0\right\} =: H^1_{\Gamma_D}(\Omega)\\
        a(u,v) = \scalarproduct{F}{v} & \forall \; v \in V \label{Weak Formulation of Boundary Value Problems}
    \end{cases}
\end{equation}
where \(a: V \times V \to \real\) is a bilinear form and \(F:V \to \real\) is a linear form s.t. \(\scalarproduct{F}{v} \equiv F(v) = \int_\Omega fv + \int_{\Gamma_N} gv\).
\begin{background}\begin{theorem}[Lax-Milgram]
    Assume that 
    \begin{itemize}
        \item \(V\) Hilbert space with \(\normdot\) and inner product \((\cdot, \cdot)\)
        \item \(F \in V^*: \abs*{F(v)} \leq \norm{F}_{V^*}\norm{v} \; \forall \; v \in V\)
        \item \(a\) continuous: \(\exists \; M > 0: \abs{a(u,v)} \leq M \norm{u}\norm{v} \; \forall \; u,v \in V\)
        \item \(a\) coercive: \(\exists \; \alpha > 0: a(v,v) \geq \alpha\norm{v}^2 \; \forall \; v \in V\)
    \end{itemize}
    Then, there exists a unique solution \(u\) of \ref*{Weak Formulation of Boundary Value Problems}
\end{theorem}\end{background}
Moreover 
\[
    \alpha \norm{u}^2 \leq a(u,u) = F(u) \leq \norm{F}_{V^*} \norm{u}
\]
where \(\alpha\) is the coercivity costant. Hence
\[
    \norm{u} \leq \frac{\norm{F}_{V^*}}{\alpha} \to \txt{stability/continuous dependence on data}
\]
But what if some of the assumptions of Lax-Milgram (in particular coercivity) are not satisfied?

We need a slightly more general problem to formulate Nečas theorem:
\begin{equation}
    \begin{cases}
        \txt{find } u \in V \\
        a(u,w) = \scalarproduct{F}{w} & \forall \; w \in W \label{Generalized weak formulation}
    \end{cases}
\end{equation}
They belong to different spaces: W for the test function, V the solutions
\begin{background}\begin{theorem}[Nečas]
    Assume that \(F \in W^*\). Consider the following conditions:
    \begin{itemize}
        \item \(a\) continuous: \(\exists \; M > 0: \abs*{a(u,w)} \leq M \norm{u}_V \norm{w}_W \; \forall \; u \in V, w \in W\)
        \item \(\inf-\sup\) condition: \(\exists \; \alpha > 0: \forall \; v \in V \quad \sup_{w \in W \setminus \left\{0\right\}} \frac{a(v,m)}{\norm{w}_W} \geq \alpha \norm{v}_V\)
        \item \(\forall \; w \in W, w \neq 0, \exists \; v \in V : a(v,w) \neq 0\)
    \end{itemize}
    These conditions are necessary and sufficient for the existence and uniqueness of a solution of \ref*{Generalized weak formulation}, for any \(F \in W^*\). Moreover 
    \[
        \norm{u}_V \leq \frac{1}{\alpha}\norm{F}_{W^*}
    \]
\label{Nečas}
\end{theorem}
\end{background}
When \(W=V\) Lax-Milgram provides necessary and sufficient conditions for existence and uniqueness of solutions.

Going back to 
\begin{equation*}
    \begin{cases}
        \mathcal{L}u = f & \text{in }\Omega \\
        + \text{B.C.} & \text{on }\partial\Omega
    \end{cases}
\end{equation*}
What could be our choice of \(V\)? Given that
\[
    u\in V : a(u,v) = F(v) \quad \forall \; v \in V
\]
and 
\[
    a(u,v) = \int_\Omega \mu \underbrace{\grad u \grad v}_{\grad u, \grad v \in L^2} + \int_\Omega b \underbrace{\grad u v}_{\in L^1} + \int_\Omega \sigma \underbrace{u v}_{\in L^1}
\]
We want to choose \(v\) in order to have all of these integrable \[\Rightarrow V = \left\{v \in L^2(\Omega), \grad u \in \left[L^2(\Omega)\right]^d, v\vert_{\Gamma_D} = 0\right\} = V_{\Gamma_D}\].

Knowing that a Sobolev space 
\[
    H^1 = \left\{v \in L^2(\Omega), \grad u \in \left[L^2(\Omega)\right]^d\right\}
\]
we can say \(V_{\Gamma_D} = \left\{v \in H^1(\Omega): v\limited{\Gamma_D}= 0\right\}\), and if \(\Gamma_D = \partial\Omega\), then \(V_{\Gamma_D} = H^1_0\)
\subsection{Approximation}
Recall for a moment the weak formulation of a generic elliptic problem 
\begin{equation}
    \begin{cases}
        \txt{Find } u \in V \\
        a(u,v) = \scalarproduct{F}{v} & \forall \; v \in V \label{Weak Formulation of Boundary Value Problems - 2}
    \end{cases}
\end{equation}
with \(V\) being an appropriate Hilbert space, subset of \(H^1(), a(\cdot,\cdot)\) being a continuous and coercive bilinear forrm from \(V \times V \to \real\), \(F(\cdot)\) being a continuous linear functional from \(V \to \real\).

Let \(V_h \subset V\) be a family of spaces that depends on a parameter \(h > 0\), such that \(\dim V_h = N_h < \infty\).
We can rewrite the weak formulation 
\begin{equation}
    \begin{cases}
        \txt{Find } u_h \in V_h\\
        a(u_h,v_h) = \scalarproduct{F}{v_h} & \forall \; v_h \in V_h \label{Galerkin Problem Formula}
    \end{cases}
\end{equation}
and is called a \textbf{Galerkin problem}. Denoting with \(\left\{\phi_j, j = 1,2,\ldots,N_h\right\}\) a basis of \(V_h\), it is sufficient that the \eqref{Galerkin Problem Formula} is verified for each function of the basis. 
Also we need that 
\[
    a(u_h, \phi_i) = F(\phi_i) \quad i = 1, 2, \ldots , N_h
\]
Since \(u_h \in V_h\)
\[
    u_h(\vect{x})=\sum_{j=1}^{N_h} u_j \phi_j(\vect{x})
\]
where \(u_j\) are unknown coefficients. Then
\[
    \sum_{j=1}^{N_h}u_j a(\phi_j, \phi_i) = F(\phi_i)
\]
We denote by \(A\) the matrix made by \(a_{ij} = a(\phi_j, \phi_i)\) and \(\vect{f}\) the vector of \(F(\phi_i) = f_i\) components. If we denote the vector \(\vect{u}\) made by the unknown coefficients \(u_h\).
\begin{equation}
    A\vect{u} = \vect{f}
    \label{Linear system Galerkin}
\end{equation}
\begin{background}
    \begin{theorem}
    The stiffness matrix \(A\) associated to the Galerkin discretization of an elliptic problem, whose bilinear form is coercive is positive definite.
\end{theorem}
\begin{proof}
    Recall that a matrix \(B \in \real^{n\times n}\) is said to be positive definite if 
    \[
        \vect{v}^TB\vect{v} \geq 0 \quad \forall \; \vect{v} \in \real^n
    \]
    and
    \[
        \vect{v}^T B \vect{v} = 0 \Leftrightarrow \vect{v} = \vect{0}
    \]
    The correspondence 
    \[
        \vect{v} = (v_i) \in \real^{N_h} \longrightarrow v_h(x) = \sum_{j=1}^{N_h} v_j\phi_j \in V_h
    \]
    defines a bijection between \(V_h\) and \(\real^{N_h}\). Given a generic vector \(\vect{v} = (v_i)\) of \(\real^{N_h}\), thanks to the bilinearity and coercivity of \(a\) we obtain 
    
        \begin{align*}
        \vect{v}^T A \vect{v} & = \sum_{j=1}^{N_h}\sum_{i=1}^{N_h} v_i ai_j v_j \\
        & = \sum_{j=1}^{N_h}\sum_{i=1}^{N_h} v_ia(\phi_j, \phi_i)v_j\\
        & = \sum_{j=1}^{N_h}\sum_{i=1}^{N_h} a(v_j\phi_j, v_i \phi_i)\\
        & = a\left(\sum_{j=1}^{N_h}v_j\phi_j\sum_{i=1}^{N_h}v_i\phi_i\right)\\
        & = a(v_h,v_h) \geq \alpha \norm{v_h}_V^2 \geq 0
        \end{align*}

    Moreover, if \(\vect{v}^T A \vect{v} = 0\), then \(\norm{v_h}_V^2 = 0\).
\end{proof}
\end{background}

\subsubsection*{Existence and uniqueness}
\begin{background}
    \begin{corollary}
    The solution of the Galerkin problem \eqref{Galerkin Problem Formula} exists and is unique.
\end{corollary}
\end{background}
To prove this we can prove that the solution to \eqref{Linear system Galerkin} exists and is unique. The matrix \(A\) is invertible as the unique solution of \(A\vect{u} = \vect{0}\) is the null solution, meaning that \(A\) is definite positive.
\subsection*{Stability}
\begin{corollary}
    The Galerkin method is stable, uniformly with respect to \(h\), by virtue of the following upper bound for the solution
    \[
        \norm{u_h}_V \leq \frac{1}{\alpha}\norm{F}_{V^*}
    \]
\end{corollary}
The stability of the method guarantees that the norm \(\norm{u_h}_V\) of the discrete solution remains bounded for \(h \to 0\). Equivalently it guarantees that \(\norm{u_h-w_h}_V \leq \frac{1}{\alpha}\norm{F-G}_{V^*}\) with \(u_h\) and \(w_h\) being numerical solution corresponding to different data \(F\) and \(G\).
\subsection*{Convergence}
\begin{background}
    \begin{lemma}[Galerkin orthogonality]
    The solution \(u_h\) of the Galerkin method satisfies 
    \begin{equation}
        a(u-u_h, v_h) = 0 \quad \forall \; v_h \in V_h \label{Galerkin orthogonality}
    \end{equation}
\end{lemma}

\begin{proof}
    Since \(V_h \subset V\), the exact solution \(u\) satisfies the weak problem \eqref{Weak Formulation of Boundary Value Problems - 2} for each element \(v = v_h \in V_h\), hence we have 
    \begin{equation}
        a(u, v_h) = F(v_h) \forall \; v_h \in V_h \label{Application of exact solution}
    \end{equation}
    By subtracting side by side \eqref{Galerkin Problem Formula} from \eqref{Application of exact solution}, we obtain 
    \[
        a(u,v_h)-a(u_h, v_h) = 0 \forall \; v_h \in V_h
    \]
    from which the claim follows.
\end{proof}
\end{background}

Also this can be generalized in the cases in which \(a(\cdot, \cdot)\) is not symmetric. Consider the value taken by the bilinear form when both its arguments are \(u-u_h\). If \(v_h\) is an arbitrary element of \(V_h\) we obtain 
\[
    a(u-u_h, u-u_h) = a(u-u_h, u-v_h) + a(u-u_h, v_h-u_h)
\]
The last term is null by \eqref{Galerkin orthogonality}. Moreover 
\[
    \abs*{a(u-u_h, u-v_h)} \leq M \norm{u-u_h}_V\norm{u-v_h}_V
\]
having exploited the continuity of the bilinear form. Also by the coercivity 
\[
    a(u-u_h, u-u_h) \geq \alpha \norm{u-u_h}^2_V
\]
hence 
\[
    \norm{u-u_h}_V \leq \frac{M}{\alpha} \norm{u-v_h}_V \quad \forall \; v_h \in V_h
\]
Such inequality holds for all functions \(v_h \in V_h\) and therefore we find 
\begin{equation}
   \underbrace{ \norm{u-u_h}_V}_{\text{Galerkin error}} \leq \frac{M}{\alpha} \underbrace{\inf_{w_h \in V_h} \norm{u - w_h}_V}_{\text{Best Approximation Error}} \label{Ceà Lemma}
\end{equation}
In order for the method to converge, it is sufficient that, for \(h \to 0\) the space \(V_h\) tends to saturate the entire space \(V\). 
\begin{equation}
    \lim_{h \to 0} \inf_{v_h \in V_h} \norm{v-v_h}_V= 0 \quad \forall \; v \in V \label{Saturation Property}
\end{equation}
In that case the Galerkin method is convergent and it can be written that
\[
    \lim_{h\to 0} \norm{u-u_h}_V = 0 \Leftrightarrow \text{convergence}
\]
Thie space \(V_h\) must be chosen carefully to satisfy the saturation property \eqref{Saturation Property}.
\subsection{Finite Element Method}
\subsubsection*{Partitions}
\begin{itemize}
    \item[\textbf{1D}]
    Let us suppose that \(\Omega\) is an interval \((a,b)\). How to create an approximation of the space \(H^1(a,b)\) that depend on a parameter \(h\). Consider a partition \(\mathcal{T}_h\) in \(N+1\) subintervals \(K_j = x_{j-1}, x_j\), having width \(h_j = x_j - x_{j-1}\) with 
    \begin{equation}
        a = x_0 < x_1 < \ldots < x_{N-1} < x_N = b \label{Partition of an interval}
    \end{equation}
    and set \(h = \max_j h_j\). 
    \item[\textbf{2D}] Now we can extend the FEM for multi-dimensional problems. For simplicity we will consider \(\Omega \subset \real^2\) with polygonal shapes \(\mathcal{T}_h\). In this case the partition is called a triangulation. We can define the discretized domain 
    \[
        \Omega_h = \text{int}\left(\bigcup_{K \in \mathcal{T}_h} K\right)
    \]
    in a way that the internal part of the union of the triangles \(\mathcal{T}_h\). Having set \(\text{diam}(K) = \max_{x,y \in K} \abs{x-y} = h_k\). Also, given \(\rho_K\) the measure of the diameter of the circle inscribed in the triangle \(K\), must be satisfied the condition that, for a suitable \(\delta > 0\) 
    \begin{equation}
        \frac{h_k}{\rho_k} \leq \delta \quad \forall \; K \in \mathcal{T}_h \label{Regularity of K}
    \end{equation}
    The condition \eqref{Regularity of K} excludes very deformed triangles.

\end{itemize}
\begin{background}
\begin{definition}[Seminorms]
    A seminorm is defined as 
    \[
        \abs{f}_k = \abs{f}_H^k (\Omega) = \sqrt{\sum_{\abs{\alpha} = k} \int_\Omega \left(D^\alpha f\right)^2 \, d\Omega}
    \]
    In particular 

        \begin{align*}
            \text{1D:}\quad  \abs{u}_{H^1(a,b)} & = \left(\norm{u_x}^2_{L^2(a,b)}\right)^{\frac{1}{2}} = \norm{u_x}^2_{L^2(a,b)} \\
              \abs{u}_{H^2(a,b)} & = \norm{u_{xx}}^2_{L^2(a,b)} \\
            \text{2D:}\quad  \abs{u}_{H^1(a,b)} & = \left(\norm{u_x}^2_{L^2(a,b)} + \norm{u_y}^2_{L^2(a,b)}\right)^{\frac{1}{2}} \\
             \abs{u}_{H^1(a,b)} & = \left(\norm{u_{xx}}^2_{L^2(a,b)} + \norm{u_{xy}}^2_{L^2(a,b)} + \norm{u_{yx}}^2_{L^2(a,b)} +\norm{u_{yy}}^2_{L^2(a,b)}\right)^{\frac{1}{2}} \\
        \end{align*}
    Always true that \(\abs{u}_{H^q} \leq \norm{u}_{H^q}\)
\end{definition}
\end{background}
The problem is always: 
\begin{align}
    \begin{split} 
        \text{find } u_h \in \;&V_h : a(u_h, v_h) = F(v_h) \quad \forall \; v_h \in V_h \\
        &\downarrow\\
        &V_h = \left\lbrace v_h \in X^r_h : v_h\vert_{\Gamma_D} = 0\right\rbrace \; \; r \geq 1 \label{Galerkin weak formulation}
    \end{split}   
\end{align}

Since the functions of \(H^1(a,b)\) are continuous on \([a,b]\), it is possible to create the family of spaces 
\begin{equation}
    X_h^r = \left\{v_h \in \mathcal{C}^0\left(\overline{\Omega}\right) : v_h \vert_{K_j} \in \mathbb{P}_r \;\; \forall \; K_j \in \mathcal{T}_h\right\}, \quad r= 1,2,\ldots \label{Family of spaces FEM}
\end{equation}
having denoted by \(\mathbb{P}_r\) the space of polynomials with degree lower or equal to \(r\) in the variable \(x\). All these spaces are subspaces of \(H^1(a,b)\) as they are constituted by differentiable functions except for at most a finite number of points (the vertices of the partition). It is convenient to sekect a basis for the \(X^r_h\) space that is \textit{Lagrangian}.

    \begin{align*}
        \mathbb{P}^r:\quad &\text{1D} \quad &p(x) =& \sum_{k=0}^r a_k x^k &\text{intervals} \\
        \\
                      &\text{2D}  \quad &p(x_1, x_2) =& \sum_{\mathclap{\substack{k,m=0 \\ k + m \leq r}}}^r a_{km} x_1^k x_2^m  &\text{triangles} \\
         \\
                      &\text{3D} \quad &p(x_1, x_2, x_3) =& \sum_{\mathclap{\substack{k,m,n=0 \\ k + m + n  \leq r}}}^r a_{kmn} x_1^k x_2^m x_3^n &\text{tetrahedra}
    \end{align*}

\subsubsection*{The space \(X^1_h\)}
The space is constituted by the functions of the partition \eqref{Partition of an interval}. Since only a straight line can pass through different points, the degrees of freedom (DOF, the number of values we need to assign to the basis to define the functions)  of the functions will be equal to the number \(N+2\) of vertices of the partition. It follows naturally that \(\left\lbrace\phi_i\right\rbrace, i = 0, 1, \ldots, N, N+1\).
In this case the basis functions are characterized by the following properties 
\[
    \phi_i \in X^1_h \text{ s.t } \phi_i(x_j) = \delta_{ij}, \quad i, j = 0, 1, \ldots, N, N+1
\]
where \(\delta_{ij}\) is the Kronecker delta. So we have our basis function that have value \(1\) in the node \(x_j\) and \(0\) elsewhere.

The formula for the basis function is then given by 
\begin{equation}
    \phi_i(x) = \begin{cases}
        \frac{x-x_{i-1}}{x_i - x_{i-1}} & \text{for }x_{i-1} \leq x \leq x_i \\
        \frac{x-x_{i+1}}{x_{i+1} - x_i} & \text{for }x_i \leq x \leq x_{i+1} \\
        0 & \text{otherwise}
    \end{cases}
\end{equation}
\subsubsection*{The space \(X^2_h\)}
In this case polynomials are of degree 2, so the points necessary to evaluate them are \(3\). The chosen points for every element of the partition \(\mathcal{T}_h\). The nodes from the interval goes from \(a = x_0\) to \(b = x_{2N + 2}\), so that midpoints are the nodes with odd indices. As the previous case the basis is Lagrangian
\[
    \phi_i \in X^2_h \text{ s.t } \phi_i(x_j) = \delta_{ij}, \quad i, j = 0, 1, \ldots, 2N+2
\]
\subsubsection*{The space \(V_h\)}
This space is generated by 
\[
    V_h = \left\{v_h \in X^r_h : v_h(a) = v_h(b) = 0 \right\}
\]
Having defined a basis \(\left\lbrace\phi_j(\vect{x})\right\rbrace_{j=1}^{N_h}\) for the space \(V_h\), each \(v_h\) can be expanded as a linear combination of elements of the basis, suitably weighted by coefficients \(\left\{v_j\right\}_{j=1}^{N_h}\)
\[
    v_h(\vect{x}) = \sum_{j=1}^{N_h} v_j \phi_j(\vect{x})
\]
A basis is called Lagrangian if it satisfies the following properties
\[
    \phi_i(\vect{x}_j) = \delta_{ij} \quad \forall \; 1 \leq i,j \leq N_h
\]
and then the following property holds:
\[
    v_h(\vect{x}_j) = v_j \quad \forall \; 1 \leq i,j \leq N_h
\]
The solution of the Finite Element Method, \(u_h\) can be written as 
\begin{equation}
    u_h(\vect{x}) = \sum_{j=1}^{N_h} u_j\phi_j (\vect{x}) \label{Solution of Galerkin FEM}
\end{equation}
In \eqref{Galerkin weak formulation} take \(v_h = \phi_j\) \(\forall \; j = 1,\ldots, N_h \) such that \(a(u_h,\phi_i) = F(\phi_i)\) \(\forall \; i = 1,\ldots,N_h\). Then use \eqref{Solution of Galerkin FEM} to obtain 
\begin{align*}
    &a\left( \sum_{j=1}^{N_h} u_j\phi_j (\vect{x}), \phi_i \right) = \underbrace{F(\phi_i)}_{F_i} \\ 
    &\Rightarrow \sum_{j=1}^{N_h} \underbrace{a(\phi_j, \phi_i)}_{\mathclap{\substack{a_{ij} \text{ elements}\\ \text{of } A}}} u_j (\vect{x}) = F_i \quad i = 1, \ldots, N_h \\
    &\Rightarrow A\,\vect{u} = \vect{F}
\end{align*}
Which is a linear system of dimension \(N_h \times N_h\) with \(\vect{F}\) the right hand side (RHS), \(A\) the stiffness matrix and \(\vect{u}\) a vector of unknown nodal values of the solution \(u_h\).
\subsection{Advection Diffusion Reaction Problem}
\begin{equation*}
   \begin{cases}
       Lu = \underbrace{-\div (\mu\grad u)}_{\txt{diffusion}} + \underbrace{\vect{b} \cdot \grad u}_{\txt{advection}} + \underbrace{\sigma u}_{\txt{reaction}} = f & \txt{in }\Omega \\
       \txt{BC }& \txt{on }\partial\Omega\\
   \end{cases}
\end{equation*}
Lax-Milgram tells us that if \(\sigma - \frac{1}{2}\div\vect{b} \geq \gamma > 0\) then \(\exists!\) a solution to the problem. But what if these conditions are not satisfied? We can use Nečas theorem (\eqref{Nečas}) with equivalent assumptions:
\begin{itemize}
    \item Weak coercivity (Gårding inequality): 
    \[
        \exists \; \alpha, \lambda : a(v,v) \geq \alpha \norm{v}^2-\norm{v}_{L^2(\Omega)}^2 \quad \forall \; v \in V
    \]
    \item Uniqueness condition (typically proven by maximum principle):
    \[
        (a(u,v) = 0 \; \forall \; v \in V) \Rightarrow u = 0
    \]
\end{itemize}
If \(A\) is spd (symmetric positive defined) then \(K_2(A) = \frac{\lambda_{max}(A)}{\lambda_{min}(A)}\)
\begin{background}
    \begin{proposition}
        If \(a(\cdot,\cdot)\) is symmetric and coercive, then \(A\) is spd.
    \end{proposition}

    \begin{proof}
        Symmetry: \(A_{ij} = a(\phi_j,\phi_i) = a(\phi_i,\phi_j) = A_{ji}\)

        \(\forall \; \vect{v} \in \real^{N_h}\):
        \begin{align*}
            \vect{v}^TA\vect{v} & = \sum_{i,j}A_{ij}v_iv_j = \sum_{i,j}a(\phi_j, \phi_i)v_i v_j \\
            & = a(\sum_j v_j \phi_j, \sum_i v_i \phi_i) = a(v_h, v_h) \geq \alpha\norm{v_h}^2 > 0
        \end{align*}
    if (\(v_h \neq 0 \Leftrightarrow \vect{v} \neq \vect{0}\)). Hence \(A\) is positive defined.
    \end{proof}
\end{background}
\begin{definition}
    If \(A\) is spd, we define the \textit{A-norm} of \(\vect{v}\) as 
    \begin{align*}
        \norm{v}_A :=& (A\vect{v}, \vect{v})^{\frac{1}{2}}\\
        =& \left(\sum_{i,j}a_{ij} v_i v_j\right)^{\frac{1}{2}}
    \end{align*}
\end{definition}
Since \(A\) is positive defined \(\Rightarrow \text{Re}(\lambda_k(A)) \Rightarrow \lambda_k(A) \neq 0\). Then, by symmetry of \(A\) \(\Rightarrow \lambda_k(A) \in \real\). Combining the two we have that \(A\) sdp \(\Rightarrow \lambda_k (A) > 0 \Rightarrow \exists! \txt{ solution of } A\vect{u} = \vect{f}\)
\begin{definition}
    If \(A\) is sdp, then \(K_2(A) = \frac{\lambda_{\txt{max}}}{\lambda_{\txt{min}}}\) is called \textbf{spectral condition number}
\end{definition}
If \(K_2(A) \gg 1 \Rightarrow A\) is ill-conditioned \(\Rightarrow\) solving \(A\vect{u} = \vect{f}\) is hard.

We can also prove that \(\exists \; C_1, C_2 > 0 : \forall \; \lambda_h\) eigenvalue of \(A\): 
\[
    \alpha C_1 h^d \leq \lambda_h \leq M C_2 h^{d-2} \qquad d = 1, 2, 3
\]
whence 
\[
    \frac{\lambda_{\txt{max}}(A)}{\lambda_{\txt{min}}(A)} \leq \frac{MC_2}{\alpha C_1} h^{-2}
\]
Then
\[
    K_2(A) = \mathcal{O}(h^{-2})
\]
If we use the conjugate gradient method to solve \(A\vect{u} = \vect{f}\), then:
\[
    \norm{\vect{u}^{(k)} - \vect{u}}_A \leq 2\left(\frac{\sqrt{K_2(A)}+1}{\sqrt{K_2(A)}-1}\right)^k \norm{\vect{u}^{(k)} - \vect{u}}_A
\]
Same with gradient method, with \(K_2(A)\) instead of \(\sqrt{K_2(A)} \Rightarrow\) need for preconditioners.
\subsection{Interpolant estimates}
\begin{equation}
    \norm{u-u_h}_V \leq \frac{M}{\alpha} \inf_{v_h \in V_h} \norm{u-v_h} \underset{\txt{saturation}}{\longrightarrow} 0 \Leftrightarrow \txt{convergence} 
\end{equation}
But how fast it saturates? 

Note: \(\inf_{v_h \in V_h} \norm{u-v_h}_V \leq \norm{u-\bar{u}_h}_V\) \(\forall \; \bar{u}_h\) suitable chosen in \(V_h\) and \(\bar{u}_h\) is a smart guy chosen in a smart way (close enough to \(u\)). 

In \(1\)D the finite element interpolant can be defined as \(\prod_{h}^{r} u(x_k) = u(x_k)\) \(\forall \; x_k\) node. Then \(\bar{u}_h = \prod_{h}^r u \in V_h\). 

How good is \(\bar{u}_h\)? 
\[
    \prod_{h}^{r} u(x) = \sum_{j=1}^{N_h} u(x_j)\phi_j(x)
\]
which is a good approximation.
\subsection*{Interpolant error estimates}
Then, for \(m=0,1\) \(\exists \; C = C(r,m,\hat{k})\) s.t.
\begin{equation}
    \seminorm{v-\prod_{h}^{r}v}{H^m(\Omega)} \leq C\left(\sum_{K \in \mathcal{T}_h}h_K^{2(r+1-m)}\seminorm{v}{H^{r+1}(K)}^2\right)^{\frac{1}{2}} \label{some_estimate_bvp}
\end{equation}
    

where \(h_K = \txt{diam}(K)\) and \(h_K \leq h\) \(\forall \; K\) this yields:
\begin{equation}
    \seminorm{v-\prod_{h}^{r}v}{H^m(\Omega)} \leq C h^{r+1-m}\seminorm{v}{H^{r+1}(K)} \quad \forall \; v \in H^{r+1}(\Omega), m = 0,1 \label{some_other_estimate_bvp}
\end{equation}
Recall also that 
\begin{align*}
    \norm{u-u_h} &= \norm{u-u_h}_{H^1(\Omega)} \\
    &\leq \frac{M}{\alpha} \inf_{v_h \in V_h} \norm{u-v_h} \\
    &\leq \frac{M}{\alpha} \norm{u-\prod_h^r u}_{H^1(\Omega)}
\end{align*}
Using \eqref{some_estimate_bvp} we obtain
\begin{equation}
    \norm{u-u_h} \leq C\frac{M}{\alpha}\left(\sum_{K \in \mathcal{T}_h}h_K^{2r}\seminorm{v}{H^{r+1}(K)}^2\right)^{\frac{1}{2}} \label{first_estimate_H1_bvp}
\end{equation}
Then, by using \eqref{some_other_estimate_bvp}:
\begin{equation}
    \norm{u-u_h} \leq C \frac{M}{\alpha} h^r \seminorm{u}{H^{r+1}(\Omega)} \label{second_estimate_H1_bvp}
\end{equation}

\begin{definition}
    Consider a bilinear form \(a: V \times V \to \real\). The \textit{adjoint} form \(a^*\) is defined as \(a^*:V \times V \to \real\)
    \[
        a^*(v,w) = a(w,v) \quad \forall \; v,w \in V
    \]
\end{definition}
Now let's consider the adjoint problem 
\begin{equation}
    \begin{cases}
        \txt{Find } \phi = \phi(g) \in V & \forall \; g \in L^2(\Omega)\\ 
        a^*(\phi,v) = (g,v) = \int_{\Omega} g v & \forall \; v \in V
    \end{cases}
    \label{Adjoint_problem_bvp}
\end{equation}
Assuming that \(\phi \in H^2(\Omega) \cap V\) (elliptic regularity). Consider now, for example, \(\mathcal{L} = -\Delta\). Then the solution of 
\begin{equation*}
    \begin{cases}
        -\Delta \phi = g & \txt{in } \Omega \\
        u = 0 & \txt{on } \partial \Omega
    \end{cases}
\end{equation*}
satisfies \(\phi \in H^2(\Omega)\). Moreover
\begin{equation}
    \exists \; C_1 > 0 : \norm{\phi(g)}_{H^2(\Omega)} \leq C_1 \norm{g}_{L^2(\Omega)}  \label{L2_norm_bvp}
\end{equation}
    
Take now \(g = e_h = u-u_h\) in \eqref{Adjoint_problem_bvp}. Then
\begin{align*}
    \norm{e_h}^2_{L^2(\Omega)} &= a^*(\phi, e_h) = a(e_h, \phi) \\ 
    &= a(e_h, \phi-\phi_h) \tag{\txt{Galerkin orthogonality}} \\
    &\leq M\norm{e_h}_{H^1(\Omega)}\norm{\phi-\phi_h}_{H^1(\Omega)}
\end{align*}
Take then \(\phi_h = \prod_{h}^{1}\phi\):
\begin{align*}
    \norm{e_h}^2_{L^2(\Omega)} &\leq M\norm{e_h}_{H^1(\Omega)}\norm{\phi - {\textstyle \prod_{h}^{1}}\phi}_{H^1(\Omega)} \\
    &\leq M\honenorm{e_h}C_2 h\seminorm{\phi}{H^2(\Omega)} \tag{for \eqref{some_other_estimate_bvp} with m=r=1} \\
    &\leq M \honenorm{e_h} C_2 h C_1 \ltwonorm{e_h} \tag{for \eqref{L2_norm_bvp}}
\end{align*}
Whence: 
\begin{align*}
    \ltwonorm{e_h} &\leq C_1 C_2 h h \honenorm{e_h} \\
    &\leq M C_1 C_2 h C_3 h^r \seminorm{u}{H^{r+1}(\Omega)} \tag{for \eqref{second_estimate_H1_bvp}}
\end{align*}
So 
\begin{equation}
    \ltwonorm{e_h} \leq \overline{C} h^{r+1} \seminorm{u}{H^{r+1}(\Omega)}
\end{equation}
\newpage
\section{Spectral Element Method}
\subsection{Introduction}
The problem with the Finite Element Method is that the rate of convergence is limited by the degree of the polynomials used. An alternative can be the Spectral Element Method, for which the convergence rate is limited by the regularity of the solution. 
\subsection{Legendre polynomials}
The Legendre polynomials \(\{L_k(x) \in \mathbb{P}_k, k = 0, 1, \ldots\}\) are the eigenfunctions of the singular Sturm-Liouville problem:
\[
    ((1-x^2)L'_k(x))' + k(k+1)L_k(x) = 0 \quad -1 < x < 1
\]
So they satisfy the recurrence relation
\begin{equation}
    \begin{split}
        &L_0(x) = 1, \ L_1(x) = x, \txt{ and for } k \geq 1 \\
        &L_{k+1}(x) = \frac{2k+1}{k+1}xL_k(x) - \frac{k}{k+1}L_{k-1}(x)
    \end{split}
    \label{legendre_polynomials}
\end{equation}
Given a weight function \(w(x) \equiv 1\), they are mutually orthogonal with respect to it on the interval \((-1, 1)\)
\[
    \int_{-1}^1 L_k(x)L_m(x) \; dx = \begin{cases}
        \frac{2}{2k+1} & \txt{if } k = m \\
        0 & \txt{if } k \neq m
    \end{cases}
\]
The expansion of \(u \in L^2(-1,1)\) in terms of \(L_k\) is 
\[
    u(x) = \sum_{k=0}^{\infty} \hat{u}_k L_(x)
\]
Given that \((f,g) = \int_{-1}^1 fg \, dx\) we know that:
\[
    (u,L_m) = \sum_{k=0} \hat{u}_k (L_k, L_m) \underset{\txt{orth.}}{=} \hat{u}_m\frac{2}{2m+1} \Rightarrow \hat{u}_k = \frac{2k+1}{2} \int_{-1}^1 u L_k \, dx
\]
The truncated Legendre series of \(u\) is the \(L^2-\txt{projection of } u \txt{ over } \mathbb{P}_N\) is 
\begin{equation}
    P_Nu = \sum_{k=0}^{N} \hat{u}_k L_k
\end{equation}
Given any \(u \in H^s(-1,1)\) with \(s \in N\), the projection error \((u-P_Nu)\) satisfies the estimates 
\begin{align*}
    \norm{u-P_Nu}_{L^2(-1,1)} &\leq CN^{-s} \norm{u}_{H^s(-1,1)} & \forall \; s \geq 0 \\
    \norm{u-P_Nu}_{L^2(-1,1)} &\leq CN^{-s} \seminorm{u}{H^s(-1,1)} & \forall \; s \leq N+1 \\
\end{align*}
There is also a ``modified'' Legendre basis for function that vanish at \(\pm 1\). This is because the Legendre basis is not suited to impose Dirichlet B.C.
\begin{align*}
    \psi_0(x) = \frac{1}{2} (L_0(x) - L_1(x)) = \frac{1-x}{2} \\ 
    \psi_N(x) = \frac{1}{2} (L_0(x) + L_1(x)) = \frac{1+x}{2} \\ 
    \psi_{k-1}(x) = \frac{1}{\sqrt{2(2k-1)}} (L_{k-2}(x) - L_k(x))\\ 
    \txt{for } k = 2, \ldots, N \ -1 < x < 1 \\ 
\end{align*}

\subsection{Spectral Galerkin formulation}
Given \(\Omega = (-1, 1), \mu, b, \sigma > 0\) const., \(f: \Omega \to \real\). Look for \(u:\Omega \to \real\) s.t. 
\[
    \begin{cases}
        -(\mu u')'+(bu)' + \sigma u = f & \txt{in } \Omega \\
        u(-1) = 0 \\
        u(1) = 0
    \end{cases}
\]
Set \(V  = H^1_0(\Omega)\), then the weak form of the differential problem reads: 
\[
    \txt{find } u \in V \txt{ s.t } a(u,v) = (f,v)_{L^2(\Omega)} \quad \forall \; v \in V, \ f \in L^2(\Omega)
\]
where 
\begin{align*}
    & a(u,v) = \int_{\Omega} (\mu u' - bu)v'\, dx + \int_{\Omega} \sigma uv \, dx \\
    & (f,v)_{L^2(\Omega)} = \int_{\Omega} f v \, dx
\end{align*}
Now set \(V_N = \mathbb{P}^0_N\) 
\begin{equation}
    \txt{find } u_N \in V_N: a(u_N, v_N) = (f, v_N)_{L^2(\Omega)} \label{weak_spec_galerkin_formulation}
\end{equation}
Now expand \(u_N(x) = \sum_{k=1}^{N-1} \tilde{u}_k \psi_k(x)\) and chose \(v_N = \psi_i(x)\) for any \(i = 1, \ldots, N-1\).
The discretization of the problem reads:
\[
    \txt{find } u = \left[\tilde{u}\right]_{k=1}^{N-1} : \sum_{k=1}^{N-1} a(\phi_k, \psi_i)\tilde{u}_k = (f, \psi_i)_{L^2(\Omega)} \quad \txt{for any } i = 1, \ldots, N-1
\]
Given \(u_N \in V_N\) the solution of the problem, then if \(u \in H^{s+1}(\Omega)\) with \(s \geq 0\), thanks to Ceà Lemma, holds that:
\[
    \honenorm{u-u_N} \leq C(s)\left(\frac{1}{N}\right)^s\norm{u}_{H^{s+1}(\Omega)}
\]
So \(u_N\) converges with spectral accuracy with respect to \(N\).
But doing so we would have two full matrices, the stiffness one and the mass one  \(M_{ij} = (\psi_j, \psi_i)_{L^2(-1,1)}\) are quite expensive to compute or invert.

To solve this we can use a Lagrange nodal basis instead of a modal one, by using the Legendre-Gauss-Lobatto quadrature formulas.
In this case we need a Legendre polynomial \(L_N(x)\).

Given a \(L_N(x)\) polynomial, we can put one node at each end of the domain, so \(x_0 = -1, x_N =  1\) and \(x_j = \txt{ zeros of }L'
_N\) with \(j =  1, \ldots, N-1\). We also need a set of weights \(w_j = \frac{2}{N(N+1)} \frac{1}{[L_N(x_j)]^2}\) with \(j = 0, \ldots, N\).

With this set of nodes and weights it's possible to obtain the following interpolatory quadrature formula
\[
    \int_{-1}^1 f(x) \, dx \approx \sum_{j=0}^{N} f(x_j)*w_j
\]
The degree of exactness of this method is \(2N-1\), meaning that 
\[
    \int_{-1}^1 f(x)\,dx = \sum_{j=0}^{N} f(x_j)w_j \quad \forall \; f \in \mathbb{P}_{2N-1}
\]
Some useful operation with LGL nodes
\begin{itemize}
    \item Discrete inner product in \(L^2(-1,1)\):
    \[
        (u,v)_N = \sum_{j=0}^{N} u(x_j)v(x_j)w_j 
    \]
    with degree of exactness \(2N-1\) 
    \[
        (u,v)_{L^2(\Omega)} = (u,v)_N \quad \txt{only if } u, v \in \mathbb{P}_{2N-1}
    \]
    \item Discrete norm in \(L^2(-1,1)\)
    \[
        \norm{u}_N = (u,u)_N^{\frac{1}{2}}
    \]
    with the following norm equivalence: \(\exists \; c_1, c_2 > 0\) s.t.
    \[
        c_1 \norm{v_N}_{L^2(-1,1)} \leq \norm{v_N}_N \leq c_2 \norm{v_N}_{L^2(-1,1)} \quad \forall \; v_N \in \mathbb{P}_N
    \]
\end{itemize}
Given \(\left\{\phi_0, \ldots, \phi_N\right\}\) characteristics Lagrange polynomials in \(\mathbb{P}_N\) w.r.t the LGL nodes. then
\[
    \phi_j = \frac{1}{n(n+1)} \frac{(1-x^2)}{(x_j-x)}\frac{L'_N(x)}{L_N(x_j)} \quad \txt{for } j = 0, \ldots, N
\]
Also true that \(\phi_j(x_k) = \delta_{kj}\) and \(\left\{\phi_j\right\}\) are orthogonal w.r.t. the discrete inner product \((\cdot,\cdot)_N\), meaning that the mass matrix \(M\) is diagonal. Given \(\left\{w_i\right\}\) the set of weights, then 
\[
    M_{ij} = (\phi_j, \phi_i)_N = \delta_{ij}w_i  \quad i,j  = 0, \ldots, N
\]
\subsection{Galerkin with Numerical Integration}
We can now define the spectral Galerkin method with numerical integration (GNI), by setting our bilinear discrete form as \(a_N(u_N, v_N) = (\mu u'_N-b u_n, v'_N)_N+(\sigma u_n, v_n)_N\), and the problem as
\[
    \txt{find } u_N^{\txt{GNI}} \in V_N : a_N(u_N^{\txt{GNI}},v_N)=(f, v_N)_N \quad \forall \; v_N \in V_N
\]
Then, by the same expansion w.r.t. the Lagrange basis: \(u_N^{\txt{GNI}}(x) =  \sum_{i=0}^{N}u_N^{\txt{GNI}}(x_i)\phi_i(x)\) and choose \(v_N(x) = \phi_i(x)\) for any \(i = 1, \ldots, N-1\).  

The GNI discretization of the weak problem reads:
\[
    \txt{look for } u^{\txt{GNI}} = \left[u_N^{\txt{GNI}}(x_j)\right]_{j=0}^N : \begin{cases}
        u_N^{\txt{GNI}}(x_0) = u_N^{\txt{GNI}}(x_N) \\
        \sum_{j=0}^{N} a_N(\phi_j, \phi_i)u_N^{\txt{GNI}} (x_j) = (f, \phi_i)_N & \forall \; i = 1, \ldots, N-1
    \end{cases} 
\]

Now let's have a closer look to the \(\left\{\phi_j\right\}\):
\[
    \phi_j \in \mathbb{P}_N : \phi_j(x_i) = \delta_{ij} = \begin{cases}
        1 & \txt{if } i = j \\
        0 & \txt{otherwise}
    \end{cases}   
\]
Given the discrete inner product \((u,v)_N = \sum_{j=0}^N u(x_j)v(x_j)w_j\) we can write:
\begin{align*}
    (\phi_k, \phi_m)_N &= \sum_{j=0}^{N} \underbrace{\phi_k(x_j)}_{\delta_{kj}} \underbrace{\phi_m(x_j)}_{\delta_{mj}} w_j \quad 0 \leq k,m \leq N \\
    &= \sum_{k=0}^{N} = \begin{cases}
        w_m & \txt{if } k = m \\
        0 & \txt{otherwise}
    \end{cases}
\end{align*}
so \(\left\{\phi_k\right\}\) is orthogonal under the discrete inner product.

The GNI solution is 
\[
    u_N(x) = \sum_{i=0}^{N} \alpha_i\phi_i(x) \quad \left\{\alpha_i\right\} \txt{ unknown coefficients}
\]
Set now \(x = x_j\) with LGL nodes:
\[
    u_N(x_j) = \sum_{i=0}^{N} \alpha_i \underbrace{\phi_i(x_j)}_{\delta{ij}} = \alpha_j
\]
So, given \(u_n^{\txt{GNI}}(x_j)\) the nodal values, we obtain the nodal expansion:
\[
    u_N^{\txt{GNI}}(x) = \sum_{j=0}^{N} u_N^{\txt{GNI}} (x_j)\phi_j(x)
\]
\subsection*{Algebraic form of Spectral GNI}
Now it's about solving the following linear system 
\[
    A^{\txt{GNI}}\vect{u}^{\txt{GNI}} = \vect{f}^{\txt{GNI}}
\]
with \(A_{ij}^{\txt{GNI}} = a_N(\phi_j, \phi_i)\) for \(i = 1,\ldots, N-1, j = 0, \ldots, N\) and \(\vect{f}^{\txt{GNI}} = (f,\phi_i)_N\) for \(i = 1, \ldots, N-1\):
\[
    A^{\txt{GNI}} = \left[\begin{matrix*}
        1 & 0 & \cdots & 0 & 0 \\
        \vdots & \ddots & & &\vdots\\
        \vdots & & a_N(\phi_j, \phi_i) & & \vdots \\
        \vdots & & & \ddots & \vdots \\
        0 & 0 & \cdots & 0 & 1
    \end{matrix*}\right], \quad \vect{f}^{\txt{GNI}} = \left[\begin{matrix*}
        0 \\ \vdots \\ f_i^{\txt{GNI}} \\ \vdots \\ 0
    \end{matrix*}\right]
\]
Given that \(a(u,v) = \int_{-1}^1 \mu u' v' - \int_{-1}^1 bu v' + \int_{-1}^1 \sigma u v\) and \((f,v) = \int_{-1}^1 fv\). We estabilished that \(a_n(u,v) = (\mu u', v')_N - (bu,v')_N + (\sigma u, v)_N\) and that \((f,v)_N = (f,v)_N\), so we obtain 
\[
    A_{ij}^{\txt{GNI}} = a_N(\phi_j, \phi_i) = \underbrace{\left(\mu \phi_j', \phi_i'\right)_N}_{\txt{A}} - \underbrace{\left(b\phi_j, \phi_i'\right)_N}_{\txt{B}}+\underbrace{\left(\sigma \phi_j, \phi_i\right)_N}_{\txt{C}}
\]
Assuming \(\mu, b, \sigma \in \real\) we have that 
\begin{align*}
    C&: \sigma(\phi_j, \phi_i)_N = \sigma \delta_{ij} w_i = \begin{cases}
        \sigma w_i & i = j \\
        0 & i\neq j
    \end{cases} &\rightarrow &\quad M = \sigma \underbrace{\left[\begin{matrix}
        w_0 & 0 & 0 \\ 
        0 & \ddots & 0 \\
        0 & 0 & w_N
    \end{matrix}\right]}_{\txt{diagonal weight matrix}} \\
    B&: - b(\phi_j, \phi_i')_N = -b\sum_{k=0}^{N} \underbrace{\phi_j(x_k)}_{\delta_{jk}}\underbrace{\phi_i'(x_k)}_{D_{ki}\neq 0} w_k &\rightarrow& \quad \txt{full matrix} \\ 
    A&: \mu(\phi_j', \phi_i')_N = \mu\sum_{k=0}^{N} \underbrace{\phi_j'(x_k)}_{D_{kj}}\underbrace{\phi_i'(x_k)}_{D_{ki}} w_k &\rightarrow& \quad \txt{full matrix}
\end{align*}
where \(D = (D_{ki}) = \phi_k'(x_i)\) is the differentiation matrix that can be computed only once. The computation of \((f,\phi_i)_N\) can be made this way 
\[
    (f, \phi_i)_N = \sum_m w_m f(x_m)\underbrace{\phi_i(x_m)}_{\delta_{im}} = w_i f(x_i) 
\]
In conclusion the GNI method is still as full as the spectral one, but much easier to compute thanks to the nodal expansion.
\subsection*{Accuracy}
We can define the Global Lagrange polynomial of degree \(N\) that interpolates \(u\) at LGL nodes as:
\[
    I_nu(x) = \sum_{j=0}^{N}u(x_j)\phi_j(x)
\]
And the interpolation error, for any \(u \in H^{s+1}(-1, 1)\) with \(s \geq 0 \), the interpolation error \(u - I_N u\) satisfies the estimate:
\[
    \norm{u-I_Nu}_{H^k(-1,1)} \leq C(s)\left(\frac{1}{N}\right)^{s+1-k}\norm{u}_{H^{s+1}(-1,1)} \quad \txt{for }k = 0,1
\]
One important feature of LGL nodes is that they are not uniformly spaced (otherwise there could be problems), so that 
\[
    I_nu(x_k) = u(x_k) \quad 0 \leq k \leq N
\]
It's also possible to estimate the \(L^2\) norm of the error as:
\[
    \norm{u - I_N u}_{L^2(-1,1)} \leq C(s) \left(\frac{1}{N}\right)^{s+1} \norm{u}_{H^{s+1}(-1,1)} \quad s \geq 1
\]
\begin{theorem}[Quadrature error]
    \(\exists \; c > 0 : \forall \; f \in H^q(-1,1)\), with \(q \geq 1\), \(\forall \; v_N \in \mathbb{P}_N\) it holds 
    \[
        \abs*{\int_{-1}^1 f v_N \, dx - (f, v_N)_N} \geq c \left(\frac{1}{N}\right)^q \norm{f}_{H^q(-1,1)} \norm{v_N}_{L^2(-1,1)}
    \]
\end{theorem}
Let now \(u_N^{\txt{GNI}} \in V_N\) be the solution of
\[
     a_N(u_N^{\txt{GNI}}, v_N) = (f, v_N)_N \quad \forall \; v_N \in V_N
\]
If \(u \in H^{s+1}(\Omega)\) and \(f \in H^{s}(\Omega)\) with \(s \geq 0\), then:
\[
    \norm{u-u_N^{\txt{GNI}}}_{H^1(\Omega)} \leq C(s) \left(\frac{1}{N}\right)^s \left(\norm{u}_{H^{s+1}(\Omega)}+\norm{f}_{H^s(\Omega)}\right)
\]
So \(u_N^{\txt{GNI}}\) converges with spectral accuracy w.r.t. to \(N\) to the exact solution when the latter is smooth.

\subsubsection*{General ideas}
The idea proposed until now are the following:
\[
    \begin{array}{lccc}
        \txt{(WP)} & V \txt{ Hilbert} & a \txt{ bilinear form} & F \txt{ functional} \\ 
        \txt{(SG)} & V_h \txt{ instead of } V & \txt{same }a &  \txt{same } F \\ 
        \txt{(GNI)} & V_N & a_N  & F_N\\ 
    \end{array}
\]
\begin{itemize}
    \item For the Galerkin method one can use Ceà Lemma 
    \begin{align*}
        \honenorm{u-u_N} &\leq \underbrace{\inf_{v_N \in V_N}\honenorm{u-v_N}}_{\txt{distance of } V \txt{ from }V_N} \\
        &\leq \honenorm{u-I_nu}
    \end{align*}
    \item For the Galerkin with Numerical Integration we need something more:
    \begin{align*}
        \honenorm{u-u_N} \leq\quad & \txt{``distance'' of } V \txt{ from }V_N \\
        &+ \txt{``distance'' of } a(\cdot,\cdot) \txt{ from }a_N(\cdot,\cdot) \\
        &+ \txt{``distance'' of } F(\cdot) \txt{ from }F_N(\cdot)
    \end{align*}
\end{itemize}
\subsection{Strang Lemma}
\begin{lemma}[Strang lemma]
Consider the problem
\begin{equation}
    \txt{find } u \in V: a(u,v) = F(v) \quad \forall \; v \in V
    \label{another_weak_formulation}
\end{equation}
and its approximation 
\begin{equation}
    \txt{find } u_h \in V_h: a_h(u_h,v_h) = F_h(v_h) \quad \forall \; v_h \in V_h
    \label{another_weak_approximation}
\end{equation}
with \(\left\{V_h\right\}\) being a family of subspaces of \(V\). Suppose that \(a_h(\cdot,\cdot)\) is continuous on \(V_h \times V_h\) and uniformly coercive on \(V_h\) meaning that:
\[
    \exists \; \alpha^* > 0 \txt{ independent of }h : a_h(v_h, v_h) \geq \alpha^* \norm{v_h}^2_V \quad \forall \; v_h \in V_h
\]
Also suppose that \(F_h\) is linear and bounded on \(V_h\). Then:
\begin{itemize}
    \item exist a unique solution \(u_h\) to the problem.
    \item such solution depends continuously on the data, i.e. we have 
    \[
        \norm{u_h}_V \leq \frac{1}{\alpha^*} \sup_{v_h \in V_h \backslash \left\{0\right\}} \frac{F_h(v_h)}{\norm{v_h}_V}
    \]
    \item  finally, the following a priori error estimate holds 
    \begin{align*}
        \norm{u-u_h}_V \quad \leq & \inf_{w_h \in V_h} \bigg\{\left(1+\frac{M}{\alpha^*}\right) \norm{u-w_h}_V \\
        &+ \frac{1}{\alpha^*} \sup_{v_h \in V_h \backslash \left\{0\right\}} \frac{\abs*{a(w_h, v_h) - a_h(w_h, v_h)}}{\norm{v_h}_V} \bigg\} \\
        &+ \frac{1}{\alpha^*} \sup_{v_h \in V_h \backslash \left\{0\right\}} \frac{\abs*{F(v_h)-F_h(v_h)}}{\norm{v_h}_V} 
    \end{align*}
\end{itemize}
with \(M\) being the continuity constant of \(a(\cdot, \cdot)\)
\end{lemma}
\begin{proof}
    The assumption of Lax-Milgram are satisfied for \eqref{another_weak_approximation}, so the solution exists and is unique. Moreover
    \[
        \norm{u_h}_V \leq \frac{1}{\alpha^*} \norm{F_h}_{V'_h} 
    \]
    with \(\norm{F_h}_{V'_h} = \sup_{v_h \in V_h \backslash \left\{0\right\}} \frac{F_h(v_h)}{\norm{v_h}_V}\) being the norm of the dual space \(V'_h\).

    Now the only thing missing is the error inequality. Let \(w_h\) be any function of the subspace \(V_h\). Setting \(\sigma_h = u_h -w_h \in V_h\), we have:
    \begin{align*}
        \alpha^* \norm{\sigma_h}^2_V &\leq a_h(\sigma_h, \sigma_h) \tag*{(by coercivity of \(a_h\))} \\
        &= a_h(u_h, \sigma_h) - a_h(w_h, \sigma_h) \\
        &= F_h(\sigma_h) - a_h(w_h, \sigma_h) \tag*{(by \eqref{another_weak_approximation})} \\
        &= F_h(\sigma_h) - F(\sigma_h) + F(\sigma_h) - a_h(w_h, \sigma_h) \\
        &= \left[F_h(\sigma_h)-F(\sigma_h)\right] + a(u, \sigma_h) - a_h(w_h, \sigma_h) \tag*{(by \eqref{another_weak_formulation})} \\
        &= \left[F_h(\sigma_h) - F(\sigma_h)\right] + a(u-w_h, \sigma_h) + \left[a(w_h, \sigma_h) - a_h(w_h, \sigma_h)\right]
    \end{align*}
    If \(\sigma_h \neq 0\), we can divide everything by \(\alpha^*\norm{\sigma_h}_V\)
    \begin{align*}
        \norm{\sigma_h}_V &\leq \frac{1}{\alpha^*} \left\{\frac{\abs*{F_h(\sigma) - F(\sigma_h)}}{\norm{\sigma_h}_V} + \frac{\abs*{a(u-w_h,\sigma_h)}}{\norm{\sigma_h}_v} + \frac{\abs*{a(w_h, \sigma_h) - a_h(w_h, \sigma_h)}}{\norm{\sigma_h}_V}\right\} \\
        &\leq \frac{1}{\alpha^*} \left\{M\norm{u-w_h}_V + \sup_{\mathclap{v_h \in V_h \backslash \left\{0\right\}}} \frac{\abs*{a(w_h, v_h) - a_h(w_h, v_h)}}{\norm{v_h}_V} + \sup_{\mathclap{v_h \in V_h \backslash \left\{0\right\}}}\frac{\abs*{F_h(\sigma_h)-F(\sigma_h)}}{\norm{v_h}_V}\right\}
    \end{align*}
    Clearly, if \(\sigma_h = 0\), the inequality still holds.

    We can now estimate the error between \(u\) and \(u_h\). Since \(u-u_h = (u-w_h) - \sigma_h\) we obtain
    \begin{align*}
        \norm{u-u_h} &\leq  \norm{u-w_h}_V + \norm{\sigma_h}_V \\
        &\leq \norm{u-w_h}_V + \frac{1}{\alpha^*} \bigg\{M\norm{u-w_h}_V + \sup_{\mathclap{v_h \in V_h \backslash \left\{0\right\}}} \frac{\abs*{a(w_h, v_h) - a_h(w_h, v_h)}}{\norm{v_h}_V}  \\
        & \quad + \sup_{\mathclap{v_h \in V_h \backslash \left\{0\right\}}} \frac{\abs*{F_h(\sigma_h)-F(\sigma_h)}}{\norm{v_h}_V}\bigg\} \\
        &= \left(1+\frac{M}{\alpha^*}\right)\norm{u-w_h}_V + \frac{1}{\alpha^*} \sup_{v_h \in V_h \backslash \left\{0\right\}} \frac{\abs*{a(w_h, v_h) - a_h(w_h, v_h)}}{\norm{v_h}_V} \\
        &\quad + \sup_{v_h \in V_h \backslash \left\{0\right\}}\frac{\abs*{F_h(\sigma_h)-F(\sigma_h)}}{\norm{v_h}_V}
    \end{align*}
    If this inequality holds \(\forall \; w_h \in V_h\), then it holds when taking the infimum.
\end{proof}
Now we should try to apply Strang's lemma to GNI method in one dimension, to verify its convergence. Obviously, we will have \(V_N\) instead of \(V_h\) and everything that follows from there.

First of all, the error of the LGL numerical integration formula 
\[
    E(g, v_N) = (g, v_N) - (g, v_N)_N
\]
with \(g\) and \(v_N\) being a generic continuous function and a generic polynomial of \(\mathbb{Q}_N\) respectively. Introducing the interpolation polynomial \(I_Ng\), we obtain:
\begin{align*}
    E(g, v_n) &= (g, v_N) - (I_Ng, v_N) \\
    &= (g, v_N) - (I_{N-1}g, v_N) + \underbrace{(\overbrace{I_{N-1}g}^{\in \mathbb{Q}_{N-1}}, \overbrace{v_N}^{\in \mathbb{Q}_N})}_{\in \mathbb{Q}_{2N-1}} \\
    &= (g, v_N) - (I_{N-1}g, v_N) +(I_{N-1}g, v_N)_N - (I_{N}g, v_N)_N \\
    &= (g - I_{N-1}g, v_N) + (I_{N-1}g - I_N g, v_N)_N
\end{align*}
The first summand of the right-hand side can be bounded from above using Cauchy-Schwartz:
\[
    \abs*{(g-I_{N-1}g, v_N)} \leq \norm{g-I_{N-1}g}_{L^2(-1,1)}\norm{v_N}_{L^2(-1,1)}
\]
For the second term, it's a bit more difficult, we need to introduce two new lemmas
\begin{lemma}
    The discrete scalar product \((\cdot, \cdot)_N\) is a scalar product on \(\mathbb{Q}_N\) and, as such, it satisfies the Cauchy-Schwartz inequality
    \[
        \abs*{(\phi, \psi)_N} \leq \norm{\phi}_N \norm{\psi}_N
    \]
    where the discrete norm is defined as 
    \[
        \norm{\phi}_N = \sqrt{(\phi, \phi)_N} \quad \forall \; \phi \in \mathbb{Q}_N
    \]
\end{lemma}
\begin{lemma}
    The ``continuous'' norm of \(L^"(-1,1)\) and the ``discrete'' norm \(\norm{\cdot}_N\) verify the inequalities
    \[
        \norm{v_N}_{L^2(-1,1)} \leq \norm{v_N}_N \leq \sqrt{3}\norm{v_N}_{L^2(-1,1)}
    \]
    hence they are uniformly equivalent on \(\mathbb{Q}_N\)
\end{lemma}
By using these two lemmas we are able to obtain 
\begin{equation*}
    \begin{split}
        &\abs*{(I_{N-1}g - I_Ng, v_N)_N} \leq \norm{I_{N-1}g - I_Ng}_N \norm{v_N}_N  \\
        &\leq 3 \left[\norm{I_{N-1}g-g}_{L^2(-1,1)} + \norm{I_{N}g-g}_{L^2(-1,1)}\right] \norm{v_N}_{L^2(-1,1)}
    \end{split}
\end{equation*}
Putting all together we obtain the upper bound 
\[
    \abs*{E(g,v_N)} \leq \left[4\norm{I_{N-1}g-g}_{L^2(-1,1)} + 3 \norm{I_{N-1}g-g}_{L^2(-1,1)} \right] \norm{v_N}_{L^2(-1,1)}
\]
Using then the interpolation estimate 
\[
    \norm{f- I_Nf}_{H^k(-1,1)} \leq C(s)\left(\frac{1}{N}\right)^{s-k} \norm{f}_{H^s(-1,1)} \quad s \geq 1, k = 0,1
\]
we can bound \(\abs*{E(g,v_N)}\) even more
\[
    \abs*{E(g,v_N)} \leq C(s)\left[\left(\frac{1}{N-1}\right)^s + \left(\frac{1}{N}\right)^s\right] \norm{g}_{H^s(-1,1)} \norm{v_N}_{L^2(-1,1)}
\]
assuming that \(g \in H^s(-1,1)\).

Then, since for each \(N\geq 2 \txt{ we have that }\frac{1}{N-1} \leq \frac{2}{N}\), the error for the LGL integration can be written as 
\[
    \abs*{E(g,v_N)} \leq C(s)\left(\frac{1}{N}\right)^s \norm{g}_{H^s(-1,1)} \norm{v_N}_{L^2(-1,1)}
\]

\subsection{GNI as Collocation method}
Let us introduce a problem 
\[
    \begin{cases}
        Lu = -(\mu u')' + (bu)' + \sigma u =  f & -1 < x < 1 \\
        u(-1) = u(1) = 0
    \end{cases}
\]
that has the usual weak formulation 
\[
    \txt{find } u \in V = H^1_0(-1,1): a(u,v) = F(v), \forall \; v \in V
\]
The GNI formulation follows 
\[
    \begin{cases}
        \txt{find } u_N \in V_N = \mathbb{P}^0_N = \left\{v_N  \in \mathbb{P}_N : v_N(\pm 1) = 0\right\} \\
        a_N(u_N, v_N) = F_N(v_N) \quad \forall \; v_N \in V_N
    \end{cases}
\]
Note that, thanks to the exactness of LGL quadrature formula 
\begin{align*}
    a_N(u_N, v_N) &\overset{(\txt{def. of }I_N)}{=} (I_N\underbrace{(\mu u'_N - bu_N)}_{\in \mathbb{P}_N}, \underbrace{v'_N}_{\in \mathbb{P}_{N-1}})_N + (\sigma u_N, v_N)_N \\
    &\overset{(\txt{exactness})}{=} (I_N(\mu u' - b u), v_N) + (\sigma u_N, v_N)_N \\
    &\overset{(\txt{int. by parts})}{=} -(\underbrace{I_N(\mu u' - b u)'}_{\in \mathbb{P}_{N-1}}, \underbrace{v_N}_{\in \mathbb{P}_N}) + (\sigma u_N, v_N)_N \\
    &\overset{(\txt{exactness})}{=} (\underbrace{-(I_N(\mu u' - b u_N))' + \sigma u_N}_{= L_Nu_N}, v_N)_N
\end{align*}
So it's obvious that \((\txt{GNI}) \iff (L_Nu_N, v_N)_N = F_N(v_N) \ \forall \; v_N \in V_N\), so it's a collocation method.
\subsection{1D Spectral Elements}
Let \(p \geq 1\) integer and \(\mathbb{P}_p\) the space of polynomials of degree \(\leq p\). We can divide the domain \(\Omega = \bigcup_{n=1}^{N_e} I_k\) with \(I_k\) disjoint elements s.t. \(I_k = F_k ((-1,1))\) and 
\[
    F_k : \xi \mapsto x = \frac{b_k -a_k}{2}\xi + \frac{b_k+a_k}{2}
\]
with \(N_p = p\cdot N_e +1\) the total number of nodes in \(\Omega\). 
Then we use the Lagrange basis functions \(\left\{ \phi_i\right\}_{i=1}^{N_p}\) w.r.t. the LGL nodes.

Now set \(X_\delta = \left\{ v \in \mathcal{C}^0 : v\vert_{I_k} \in \mathbb{P}_p, \forall \; I_k \right\}\) with \(h_k = meas(I_k)\), mesh size \(h =  max_k h_k\) and polynomial degree \(p\) we can define \(\delta = (h,p)\) and 
\[
    v_\delta(x) = \sum_{i=1}^{N_p} v_\delta (x_i)\phi_i(x) \quad \forall \; v_\delta \in X_\delta
\]
Let now \((\hat{\xi}_j, \hat{w}_j) \txt{ for } j = 0, \ldots, p\) be the LGL nodes and respective weights in \(\hat{\Omega} = (-1,1)\).
We can define the local LGL quadrature as
\[
    \int_{I_k} u(x)v(x) \, dx \approx (u,v)_{\delta, I_k} = \sum_{j=0}^{p} u(\xi_j)v(\xi_j)w_j
\]
with \(\xi_j = \frac{b_k - a_k}{2}\hat{\xi}_j + \frac{b_k + a_k}{2}\) and \(w_j = \frac{b_k -a_k}{2}\hat{w}_j\).
Meanwhile we can pass this quadrature to the whole domain, obtaining the composite LGL quadrature:
\[
    \int_\Omega u(x)v(x)\, dx \approx (u,v)_{\delta, \Omega} = \sum_{k=1}^{N_e} (u,v)_{\delta, I_k}
\]
with its relative error \(\exists \; c > 0 : \forall \; f \in H^r(\Omega), r\geq 1, p \geq 1 : \forall \; v_\delta \in X_\delta \):
\[
    \abs*{\int_{\Omega} fv_\delta \, dx - (f, v_\delta)_{\delta, \Omega}} \leq c h^{\min(p,r)}\left( \frac{1}{p} \right)^r \norm{f}_{H^r(\Omega)}\norm{v_\delta}_{L^2(\Omega)}
\]
and its interpolation error as: \(\exists \; c > 0 : \forall \;\in H^{s+1}(\Omega), s \geq 1\)
\[
    \norm{v - {\textstyle \prod_{\delta}^{LGL}}v}_{H^k(\Omega)} \leq Ch^{\min(p+1, s+1)-k}\left( \frac{1}{p} \right)^{s+1-k}\norm{v}_{H^{s+1}(\Omega)}
\] 
\subsection{Spectral Element Method with Numerical Integration}
Let's go back to the problem 
\[
    \begin{cases}
        -(\mu u')' - (bu)' + \sigma u = f & \txt{in }  \Omega \\
        u(a) = u(b) = 0
    \end{cases}
\]
Given \(V = H^1_0(\Omega)\), the weak formulation reads
\[
    \txt{find }u \in V : a(u,v) = (f,v)_{L^2(\Omega)} \quad \forall \; v \in V, f \in L^2(\Omega)
\]
with 
\begin{flalign*}
    &a(u,v) = \int_\Omega (\mu u' - bu)v' \, dx + \int_\Omega \sigma uv \, dx \\
    &(f,v)_{L^2(\Omega)} = \int_\Omega fv \, dx 
\end{flalign*}
Now set \(a_\delta(\phi_j, \phi_i) = (\mu \phi'_j- b\phi_j, \phi_i)_{\delta, \Omega} + (\sigma \phi_j, \phi_i)_{\delta, \Omega}\) to get the SEM-GNI formulation:
\begin{equation}
    \txt{find }u_\delta^{\txt{GNI}} \in V_\delta : a_\delta(u_\delta^{\txt{GNI}}, v_\delta) = (f,v_\delta)_{\delta, \Omega} \quad \forall \; v_\delta \in V_\delta
\end{equation}
Now expand \(u_\delta^{\txt{GNI}}\) w.r.t the Lagrange basis \(u_\delta^{\txt{GNI}}(x) = \sum_{i=1}^{N_p}u\delta^{\txt{GNI}}(x_i)\phi_i(x)\) adn choose \(v_\delta(x) = \phi_i(x)\) for any \(i = 1, \ldots, N_p\). We can now write the SEM-GNI discretization of the weak formulation:
\[
    \begin{cases}
        \txt{find } u^{\txt{GNI}} = \left[ u_\delta^{\txt{GNI}}(x_j) \right]_{j=1}^{N_p} \\
        u_\delta^{\txt{GNI}}(x_1) = u_\delta^{\txt{GNI}}(x_{N_p}) = 0 \\
        \sum_{j=1}^{N_p} a_\delta(\phi_j, \phi_i)u_\delta^{\txt{GNI}}(x_j) = (f, \phi_i)_{\delta, \Omega} & \forall \; i = 1, \ldots, N_p
    \end{cases}
\]
or, in algebraic form \(A^{\txt{GNI}}u^{\txt{GNI}} = f^{\txt{GNI}}\) with \(A_{ij}^{\txt{GNI}} = a_\delta(\phi_j,\phi_i)\) and \(f_i^{\txt{GNI}} = (f,\phi_i)_{\delta, \Omega}\).

Now, for the error analysis, we will apply the Strang lemma, so:
\begin{align*}
    \norm{u- u _\delta^{\txt{GNI}}}_V \leq \ & \norm{u-u_\delta}_V \\
    & + \frac{1}{\mu^*} \sup_{v_\delta \in V_\delta \backslash\left\{ 0 \right\}} \frac{\abs{a(u_\delta, v_\delta) - a_\delta(u_\delta, v_\delta)}}{\norm{v_\delta}_V} \\
    & + \frac{1}{\mu^*} \sup_{v_\delta \in V_\delta \backslash\left\{ 0 \right\}} \frac{\abs*{f,v_\delta}_{L^2(\Omega) - (f,v_\delta)_{\delta,\Omega}}}{\norm{v_\delta}_V}
\end{align*}
where \(\mu^*\) is the coercivity constant of \(a_\delta\): \(a_\delta(v_\delta, v_\delta) \geq \mu^*\norm{v_\delta}^"_V\) and \(u_\delta\) the SEM-GNI solution.
Thus for any \(u \in H^{s+1}(\Omega)\) and \(f \in H^r(\Omega)\) 
\[
    \norm{u-u_\delta^{\txt{GNI}}}_{H^1(\Omega)} \leq C\left[ h^{\min(p,s)}\left( \frac{1}{p} \right)^s \norm{u}_{H^{s+1}(\Omega)} + h^{\min(p,r)}\left( \frac{1}{p} \right)^r \norm{f}_{H^{r}(\Omega)}  \right]
\]
So \(u_\delta^{\txt{GNI}}\) converges with spectral accuracy w.r.t. \(p\) and algebraic accuracy w.r.t. \(h\) to the exact solution.
\subsection{Convergence rate of SEM-GNI}
When \(s,r\) are large \((s, r > p)\):
\[
    \norm{u-u_\delta}_{H^1{\Omega}} \leq C \left[ h^p \left(\frac{1}{p}\right)^s \norm{u}_{H^{s+1}(\Omega)} + h^p \left(\frac{1}{p}\right)^r \norm{f}_{H^{r}(\Omega)}  \right]
\]
when \(s\) is small \((s \leq p)\):
\[
    \norm{u-u_\delta}_{H^1(\Omega)} \leq C\left( \frac{h}{p} \right)^s \norm{u}_{H^{s+1}(\Omega)}
\]

\section{Discontinuous Galerkin methods}
The idea behind DG methods is to seek the soltion in a discrete space made of polynomials that are completely discontinuous across the elements of the mesh.
\[
    V_h \subsetneq V
\]
\subsection{1D case}
Let us consider a Poisson problem 
\[
    \begin{cases}
        -u'' = f & a < x < b \\
        u(a) = u(b) = 0
    \end{cases}
\]
The aim is to use discontinuous piecewise polynomials, so that between every interval \(I_k\) from one node to another we obtain
\[
    \int_{a}^b -u''v = \int_a^b fv \Rightarrow - \sum_{k=0}^{N-1} \int_{I_k} u''v = \sum_{k=0}^{N-1} \int_{I_k} fv
\]
We must know integrate by parts, but our test functions are discontinuous at the nodes, so we must acknowledge it. Let's call \(x_k^-\) and \(x_k^+\) the left and right side of the \(x_k\) node. Then we can:
\begin{equation}
    -\sum_{k=0}^{N-1} \int_{I_k} u''v = \sum_{k=0}^{N-1} \left[ \int_{I_k}u'v' - \left( u'v\vert_{x_{k+1}^-} - u'v\vert_{x_k^+} \right) \right] \label{integration_by_parts_dg}
\end{equation}
    \begin{align}
    \sum_{k=0}^{N-1} (u'v\vert_{x_{k+1}^-} - u'v\vert_{x_k^+}) &= u'(x_1^-)v(x_1^-) - u'(x_0^+)v(x_0^+) \nonumber\\
    & + u'(x_2^-)v(x_2^-) - u'(x_1^+)v(x_1^+) \nonumber\\
    & + \ldots \\
    & + u'(x_N^-)v(x_N^-) - u'(x_{N-1}^+)v(x_{N-1}^+) \nonumber\\
    & = \sum_{k=0}^{N} \jump{u'(x_k)v(x_k)} \nonumber \label{step_for_jump_dg}
\end{align}
where we have defined the jump function
\begin{align}
    \jump{\phi(x_0)} &:= -\phi(x^+_0) & \nonumber\\
    \jump{\phi(x_k)} &:= \phi(x_k^-) -\phi(x^+_k) & x_k :\txt{ interior node} \\
    \jump{\phi(x_{\!N})} &:= \phi(x^-_N) &\nonumber \label{jump_function}
\end{align}
By using \eqref{integration_by_parts_dg} and \eqref{step_for_jump_dg} we obtain 
\begin{equation}
    \sum_{k=0}^{N-1} \int_{I_k} u'v' - \sum_{k=0}^{N} \jump{u'(x_k)v(x_k)} = \sum_{k=0}^{N-1} \int_{I_k} fv \label{almost_bilinear_dg}
\end{equation}
Now define the average operator 
\begin{align}
    \average{\phi(x_0)} &:= \phi(x^+_0) & \nonumber\\
    \average{\phi(x_k)} &:= \frac{1}{2}\phi(x_k^-) +\phi(x^+_k) & x_k : \txt{ interior node} \\
    \average{\phi(x_N)} &:= \phi(x^-_N) &\nonumber \label{average_function}
\end{align}
This way we obtain this formula 
\begin{equation}
    \sum_{k=0}^{N} \jump{u'(x_k)v(x_k)} = \sum_{k=0}^{N} \average{u'(x_k)} \jump{v(x_k)} + \sum_{k=1}^{N-1} \jump{u'(x_k)}\average{v(x_k)}\label{magic_formula_dg}
\end{equation}
If \(u\) is the exact solution and \(u \in \mathcal{C}^1([a,b])\), then \(\jump{u'(x_k)} = 0\) for every interior node, and the second sum in \eqref{magic_formula_dg} drops.

We end up with the formulation (by collecting \eqref{almost_bilinear_dg} and \eqref{magic_formula_dg})
\begin{equation}
    \begin{split}
        \underbrace{\sum_{k=0}^{N-1}\int_{I_k} u'v' - \sum_{k=0}^{N} \average{u'(x_k)}\jump{v(x_k)} - \sum_{k=1}^{N-1} \jump{u'(x_k)}\average{v(x_k)}}_{\mathcal{A}(u,v)} \\
        = \sum_{k=0}^{N-1}\int_{I_k} fv \quad \forall \; v \in V \label{dg_formulation}
    \end{split}
\end{equation}
where 
\[
    V = H^1_{\txt{broken}}(\Omega) := \left\{ v \in L^2(\Omega) : v\vert_{I_k} \in H^1{I_k} \ \forall \; k = 0,\ldots, N-1 \right\}
\]
with the broken norm 
\[
    \norm{v}_{H^1_{\txt{broken}}(\Omega)} = \left( \sum_{k=0}^{N} \norm{v\vert_{I_k}}_{H^1(\Omega)}^2 \right)^{\frac{1}{2}}
\]
Let now \(V_h \subset V\)
\begin{equation}
    \txt{find } u_h \in V_h : \mathcal{A}(u_h, v_h) = \sum_{k=0}^{N-1} \int_{I_k} fv_h \quad \forall \; v_h \in V_h \label{weak_not_well_posed_formulation_dg}
\end{equation}
\begin{remark}
    \(V_h\) is not a subspace of \(H^1(\Omega)\)
\end{remark}
But \eqref{weak_not_well_posed_formulation_dg} is not well posed, so the \eqref{dg_formulation} must be modified such that:
\begin{itemize}
    \item drop \(3^{rd}\) term because \(\jump{u'(x_k)} = 0\)
    \item add symmetrization term (\(=0\) if \(u\) is the exact solution) 
    \[
        -\sum_{k=0}^{N} \theta \average{v'(x_k)}\jump{u(x_k)}
    \]
    with
    \begin{itemize}
        \item[\qedhere] \(\theta = 1\) SIP (Symmetric Interior Penalty)
        \item[\qedhere] \(\theta = -1\) NIP (Non-symmetric Interior Penalty)
        \item[\qedhere] \(\theta = 0\) IIP (Incomplete Interior Penalty)
    \end{itemize}
    \item add the stabilization term (\(=0\) if \(u\) is the exact solution)
    \[
        +\sum_{k=0}^{N} \gamma \jump{u(x_k)}\jump{v(x_k)}
    \]
\end{itemize}
We can now obtain a new bilinear form
\begin{equation}
    \begin{split}
        \mathcal{A}^*(u_h, v_h) = \underbrace{\sum_{k=0}^{N-1} \int_{I_k} u'_h v'_h}_{(1)} \underbrace{-\sum_{k=0}^{N} \average{u'_h(x_k)} \jump{v_h(x_k)}}_{(2)} \\
        \underbrace{-\sum_{k=0}^{N} \theta \average{v'_h(x_k)} \jump{u_h(x_k)}}_{(3)} \underbrace{+\sum_{k=0}^{N} \gamma \jump{u_h(x_k)}\jump{v_h(x_k)}} \label{new_bilinear_form_dg} 
    \end{split}
\end{equation}
\subsubsection*{Neumann BC}
Impose Neumann BC through \(\average{u'(x_k)}\) in \((2)\). In this case we have \(\sum_{k=1}^{N-1}\) in \((2)\) and, consequently, we write \(\sum_{k=1}^{N-1}\) in \((3)\) for symmetry.
\subsubsection*{Non-homogeneus Dirichlet BC}
Impose Dirichlet BC as follows. In \((3)\) and \((4)\) replace \(\jump{u_h(x_0)}\) and \(\jump{u_h(x_N)}\) with the following definition:
\begin{align*}
    &\jump{u_h(x_0)} := \alpha - u_h(x_0^+) & \txt{if } u(a) = \alpha \\
    &\jump{u_h(x_N)} :=  u_h(x_N^-) - \beta & \txt{if } u(b) = \beta
\end{align*}
In case \(\alpha =  \beta = 0\) we have homogeneus Dirichlet. 

Now in \eqref{new_bilinear_form_dg} split sums as follows: 
\begin{equation}
    \begin{split}        
    & \mathcal{A}^*(u_h, v_h)\\
    & = \sum_{k=0}^{N-1} \int_{I_k} u'_h v'_h \\
        & - \sum_{k=1}^{N-1} \average{u'_h(x_k)}\jump{v_h(x_k)} + u'_h(x_0^+)v_h(x_0^+) - u'_h(x_N^-)v_h(x_N^-) \\
        & - \sum_{k=1}^{N-1} \theta \average{v'_h(x_k)} \jump{u_h(x_k)} - \left[\theta v'_h(x^+_0)(\alpha - u_h(x^+_0)) + \theta v'_h(x^-_N)(u_h(x^-_N) - \beta)\right] \\
        & - \sum_{k=1}^{N-1} \gamma \jump{u_h(x_k)} \jump{v_h(x_k)} + \gamma(\alpha - u_h(x^+_0))(-v_h(x^+_0)) + \gamma (u_h(x_N^-) - \beta)v_h(x_N^-) \label{split_sum_bilinear_dg}
    \end{split}
\end{equation}
Now move terms, including \(\alpha\) and \(\beta\) to the right hand side of the formulation.

On the left hand side it remains 
\begin{equation}
    \begin{split}
        & \tilde{\mathcal{A}} (u_h, v_h) \\
        & = \sum_{k=0}^{N-1} \int_{I_k} u'_hv'_h \\
        & - \sum_{k=1}^{N-1} \average{u'_h(x_k)}\jump{v_h(x_k)} + u'_h(x^+_0)v_h(x^+_0) - u'_h(x_N^-)v_h(x_N^-) \\
        & - \sum_{k=1}^{N-1} \theta \average{v'_h(x_k)} \jump{u_h(x_k)} - \left[\theta v'_h(x^+_0)u_h(x^+_0) + \theta v'_h(x^-_N)u_h(x^-_N)\right] \\
        & - \sum_{k=1}^{N-1} \gamma \jump{u_h(x_k)} \jump{v_h(x_k)} + \gamma u_h(x^+_0)v_h(x^+_0) + \gamma u_h(x_N^-)v_h(x_N^-) \label{split_sum_noalpha_bilinear_dg}
    \end{split}
\end{equation}
On the right hand side instead
\begin{equation}
    \mathcal{v_h} = \sum_{k=0}^{N-1} \int_{I_k} f v_h + \theta(\alpha v'_h(x^+_0) - \beta v'_h(x_N^-)) + \gamma(\alpha v_h(x^+_0) + \beta v_h(x^-_N))
\end{equation}
\begin{remark}
    Note that for \(\theta = 1\), \(\tilde{\mathcal{A}}(u_h, v_h) = \tilde{\mathcal{A}}(v_h, u_h)\), so it's symmetric.
\end{remark}
\subsubsection*{Non-homogeneus Dirichlet conditions}
\begin{equation}
    \txt{find } u_h \in V_h : \tilde{\mathcal{A}}(u_h, v_h) = \mathcal{F}(v_h) \quad \forall \; v_h \in V_h
    \label{functional_dg}
\end{equation}
with \(\mathcal{F}\) depending on \(f\), \(\alpha\) and \(\beta\).

Note that in \eqref{split_sum_noalpha_bilinear_dg}, if we define \(\jump{u_h(x_0)}\) and \(\jump{u_h(x_N)}\) as \(\jump{v_h(x_0)}\) and \(\jump{v_h(x_N)}\)
\begin{equation*}
    \begin{split}
        -\sum_{k=1}^{N-1} \average{u'_h(x_k)} \jump{v_h(x_k)} + u'_h(x_0^+)v_h(x^+_0) - u'_h(x_N^-)v_h(x_N^-) \\
        = -\sum_{k=0}^{N} \average{u'_h(x_k)} \jump{v_h(x_h)} \\
         -\sum_{k=1}^{N-1} \theta \average{v'_h(x_k)} \jump{u_h(x_k)} + (\theta u_h(x_0^+)v'_h(x_0^+) - \theta u_h(x_N^-)v'_h(x_N^-)) \\
        = - \sum_{k=0}^{N} \theta \average{v'_h(x_k)}\jump{u_h(x_k)} \\
        + \sum_{k=1}^{N-1} \gamma \jump{u_h(x_k)} \jump{v_h(x_k)} + \gamma u_h(x_0^+)v_h(x_0^+) + \gamma u_h(x_N^-)v_h(x_N^-) \\
        + \sum_{k=0}^{N} \gamma \jump{u_h(x_k)}\jump{v_h(x_k)}
    \end{split}
\end{equation*}
\subsection{Multidimensional case}
We can take our Poisson problem in multidimension 
\begin{equation}
    \begin{cases}
        -\Delta u = f & \txt{in } \Omega \\
        u = 0 & \txt{on }\partial\Omega
    \end{cases}\label{Poisson_multidim_dg}
\end{equation}
with the triangulation \(\mathcal{T}_h\), but this time we cannot assume that the conformity constraint is present. 

So we need to take a test function \(v\) (element-wise smooth), and integrate over an element \(\mathcal{K} \in \mathcal{T}_h\)
\[
    \int_{\mathcal{K}} -\Delta u v = \int_{\mathcal{K}} fv
\]
As usual, integrate by parts, and sum over all the elements \(\mathcal{K} \in \triangulation\) 
\[
    \sum_{\element \in \triangulation} \int_{\element}\grad u \cdot \grad v - \sum_{\element \in \triangulation} \int_{\partial\element} \grad u \cdot \vect{n}_{\element} v = \int_{\Omega} fv
\]
since for any \(F \in \mathcal{F}'_h\) which is the set of interior faces shared by two elements \(\element^+\) and \(\element^-\)
\begin{align*}
    &\average{v} = \frac{(v^+ + v^-)}{2}  & \jump{v} = v^+ \vect{n}^+ + v^- \vect{n}^-\\
    &\average{\bm{\tau}} = \frac{(\bm{\tau}^+ + \bm{\tau}^-)}{2} & \jump{\bm{\tau}} = \bm{\tau}^+ \vect{n}^+ + \bm{\tau}^- \vect{n}^-
\end{align*}
while, for the set of boundary faces \(F \in \mathcal{F}^B_h\)
\begin{align*}
    &\average{v} = v & \jump{v} = v \vect{n} \\
    &\average{\bm{\tau}} = \bm{\tau} & \jump{\bm{\tau}} = \bm{\tau} \cdot \vect{n}
\end{align*}
in this way we can obtain the following formula \(\forall \;\bm{\tau} \txt{ vector function: }
\)
\begin{equation}
    \sum_{\element \in \triangulation} \int_{\partial\element} \bm{\tau} \cdot \vect{n}_{\element} v = \sum_{F \in \mathcal{F}_h}\int_F\average{\bm{\tau}} \cdot \jump{v} + \sum_{F \in \mathcal{F}'_h} \int_{F}\jump{\bm{\tau}}\average{v} \label{magic_formula_multidim_dg}
\end{equation}
and thanks to that we obtain 
\[
    -\sum_{\element \in \triangulation} \int_{\partial\element} \bm{\tau} \cdot \vect{n}_{\element} v =- \sum_{F \in \mathcal{F}_h}\int_F\average{\grad u} \cdot \jump{v} - \sum_{F \in \mathcal{F}'_h} \int_{F}\jump{\grad u}\average{v} 
\]
then 
\[
    \sum_{\element \in \triangulation} \int_{\element}\grad u \cdot \grad v - \sum_{\element \in \triangulation} \int_{\partial\element} \grad u \cdot \vect{n}_{\element} v = \int_{\Omega} fv
\]
so it becomes 
\[
    \sum_{\element \in \triangulation} \int_{\element} \grad u \grad v - \sum_{F \in \mathcal{F}_h} \int_{F} \average{\grad u} \cdot \jump{v} - \sum_{F \in \mathcal{F}'_h} \int_F \jump{\grad u} \average{v} = \int_{\Omega} fv
\]
but, if we assume \(u \in H^2(\Omega)\), then \(\jump{\grad u} = 0 \ \forall \; F \in \mathcal{F}'_h\). This regularity assumption is fullfilled if \(f \in L^2\) and the domain is a convex polygon, thanks to the property of elliptic regularity.
\[
    sum_{\element \in \triangulation} \int_{\element} \grad u \grad v - \sum_{F \in \mathcal{F}_h} \int_{F} \average{\grad u} \cdot \jump{v} \cancel{- \sum_{F \in \mathcal{F}'_h} \int_F \jump{\grad u} \average{v}} = \int_{\Omega} fv
\]
Now we can assume that \(\jump{u} = 0 \ \forall \; F \in \mathcal{F}_h\) (since \(u \in H^2(\Omega) \cap H^1_0(\Omega)\)) to add a symmetry term 
\[
    \sum_{\element \in \triangulation} \int_\element \grad u \cdot \grad v - \sum_{F \in \mathcal{F}_h} \int_F \average{\grad u} \cdot \jump{v} - \sum_{F \in \mathcal{F}'_h} \int_F \average{\grad_h v} \jump{u} = \int_\Omega fv
\] 
where \(\grad_h\) is the elementwise gradient (\(v\) is only piecewise smooth).
We also add a stabilization term that controls the jumps
\[
    \sum_{\element \in \triangulation} \int_\element \grad u \cdot \grad v  - \sum_{F \in \mathcal{F}_h} \int_F \average{\grad u} \cdot \jump{v} - \sum_{F \in \mathcal{F}_h} \int_F \jump{u} \cdot \average{\grad_h v} + \sum_{F \in \mathcal{F}_h} \int_F \gamma \jump{u} \cdot \jump{v} = \int_\Omega fv
\]
where \(\gamma\) is a stabilization function.

Now we can define the DG discrete space 
\[
    V_h^p = \left\{ v_h \in L^2(\Omega) : v_h\vert_\element \in \mathcal{P}^{p_\element}(\element) \ \forall \; \element \in \triangulation\right\} \not\subseteq H^1_0 (\Omega)
\]
Discretize \(u \leadsto u_h, v \leadsto v_h\) and obtain the following weak formulation
\[
    \txt{find } u_h \in V_h^p \txt{ s.t. } \mathcal{A}(u_h, v_h) = \int_\Omega fv \quad \forall \;v_h \in V_h^p
\]
where 
\begin{equation*}
    \begin{split}
        \mathcal{A}(u, v) =  \sum_{\element \in \triangulation} \int_\element \grad u \cdot \grad v  - \sum_{F \in \mathcal{F}_h} \int_F \average{\grad u} \cdot \jump{v}  - \sum_{F \in \mathcal{F}_h} \int_F \jump{u} \cdot \average{\grad_h v} \\+ \sum_{F \in \mathcal{F}_h} \int_F \gamma \jump{u} \cdot \jump{v}
    \end{split}
\end{equation*}
\subsubsection*{Interior Penalty DG methods}
\[
    \txt{find } u_h \in V_h^p \txt{ s.t. } \mathcal{A}(u_h, v_h) = \int_\Omega fv \quad \forall \;v_h \in V_h^p
\]
Note that \(\mathcal{A}\) depends on the triangulation and it differs from the original weak formulation in the infinite dimension problem.
\begin{equation*}
    \begin{split}
        \mathcal{A}(u, v) =  \sum_{\element \in \triangulation} \int_\element \grad u \cdot \grad v  - \sum_{F \in \mathcal{F}_h} \int_F \average{\grad u} \cdot \jump{v}  - \theta \sum_{F \in \mathcal{F}_h} \int_F \jump{u} \cdot \average{\grad_h v} \\+ \sum_{F \in \mathcal{F}_h} \int_F \gamma \jump{u} \cdot \jump{v}
    \end{split}
\end{equation*}
where 
\begin{itemize}
    \item \(\theta =  1\) Symmetric Interior Penalty (SIP)
    \item \(\theta =  -1\) Non-symmetric Interior Penalty (NIP)
    \item \(\theta =  0\) Incomplete Interior Penalty (IIP)
\end{itemize}
\subsubsection*{Dirichlet BC}
The above formulation holds when applying homogeneus Dirichlet BC, but in the case of non-homogeneus BC, such as 
\[
    u = g_D \quad \txt{ on } \partial\Omega
\]
the right hand side must be modified as
\[
    \int_\Omega fv -\theta \sum_{F \in \mathcal{F}^B_h}\int_F g_D \grad_h v \cdot \vect{n} + \sum_{F \in \mathcal{F}^B_h} \int_F \gamma g_D v 
\]
\subsubsection*{Neumann BC}
In the case of Neumann BC, like 
\[
    \grad u \cdot \vect{n} = g_N \quad \txt{ on } \partial\Omega
\]
the bilinear form has to be modified such as 
\begin{equation*}
    \begin{split}
        \mathcal{A}(u, v) =  \sum_{\element \in \triangulation} \int_\element \grad u \cdot \grad v  - \sum_{F \in \mathcal{F}'_h} \int_F \average{\grad u} \cdot \jump{v}  - \theta \sum_{F \in \mathcal{F}'_h} \int_F \jump{u} \cdot \average{\grad_h v} \\+ \sum_{F \in \mathcal{F}'_h} \int_F \gamma \jump{u} \cdot \jump{v}
    \end{split}
\end{equation*}
and the right hand side 
\[
    \int_\Omega fv - \sum_{F \in \mathcal{F}^B_h}\int_F g_N v
\]
\subsubsection*{The stabilization function \texorpdfstring{\(\gamma\)}{gamma}}
\[
    \sum_{F \in \mathcal{F}_h} \int_F \gamma \jump{u} \cdot \jump{v} \quad \gamma = \alpha \frac{p^2}{h}
\]
where 
\[
    p =\begin{cases}
        \max\left\{ p_{\element^+}, p_{\element^-} \right\} & \txt{ if } F \in \mathcal{F}'_h \\
        p_\element & \txt{ if } F \in \mathcal{F}^B_h
    \end{cases}
\]
and 
\[
    h =\begin{cases}
        \min\left\{ h_{\element^+}, h_{\element^-} \right\} & \txt{ if } F \in \mathcal{F}'_h \\
        h_\element & \txt{ if } F \in \mathcal{F}^B_h
    \end{cases}
\]
Since we can make some assumptions 
\[
    h_F \approx h_{\element^+} \approx h_{\element^-}, \ p_{\element^+} \approx p_{\element^-} \Rightarrow \gamma = \mathcal{O}\left( \frac{p^2}{h} \right)
\]

\subsection{Theoretical reminders}

For an integer \(s \geq 1\) define the broken Sobolev space
\begin{align*}
    H^s(\triangulation) &= \left\{ v \in L^2(\Omega) : v\limited{\element} \in H^s(\element) \ \forall \;\element \in \triangulation\right\} \\
    \norm{v}^2_{H^s(\element)} &= \sum_{\element \in \triangulation} \norm{v}^2_{H^s(\element)}
\end{align*}
Define also 
\[
    \norm{v}^2_{L^2(\mathcal{F}_h)} = \sum_{F \in \mathcal{F}_h} \norm{v}^2_{L^2(F)}
\]
We define then the following norms 
\begin{align*}
    \norm{v}^2_{DG} &= \norm{\grad_h v}^2_{L^2(\Omega)} + \norm{\gamma^{\frac{1}{2}}\jump{v}}^2_{L^2(\mathcal{F}_h)} \quad \forall \;v \in H^2(\triangulation) \\
    \threenorm{v}{DG} &= \norm{v}_{DG}^2 + \norm{\gamma ^{\frac{1}{2}} \average{\grad_h v}}^2_{L^2(\mathcal{F}_h)} \quad \forall \;v \in H^2(\triangulation)
\end{align*}
where \(\grad_h v\) is the elementwise gradient:
\[
    (\grad_h v)\limited{\element} = \grad(v\limited{\element}) \quad\forall \; \element \in \triangulation
\]
Notice that \(V_h^p \subset H^2(\triangulation)\). It can be shown that 
\begin{align*}
    \norm{v}_{DG} &\underset{(trivial)}{\leq} \threenorm{v}{DG} \not\lesssim \norm{v}_{DG} \quad \forall \; v \in H^2(\triangulation) \\
    \norm{v_h}_{DG} &\underset{(trivial)}{\leq} \threenorm{v_h}{DG} \underset{(on \ slides)}{\not\lesssim}\norm{v_h}_{DG} \quad \forall \; v_h \in V_h^p \\
\end{align*}
Some key properties:
\begin{itemize}
    \item Continuity on \(H^2(\triangulation) \times V^p_h\): 
    \[
        \abs{\mathcal{A}(v, w_h)} \lesssim \threenorm{v}{DG} \norm{w_h}_{DG} \quad  \forall \; v \in H^2(\triangulation), \ \forall \; w_h \in V_h^p
    \]
    Also remind that \(\abs{\mathcal{A}(v, w_h)} \not\lesssim \norm{v}_{DG} \norm{w_h}_{DG}\)
    \item Coercivity on \(V_h^p \times V_h^p\): 
    \[
        \mathcal{A}(v_h, v_h) \gtrsim \norm{v_h}_{DG} \quad \forall \; v_h \in V_h^p
    \]
    For SIP and IIP, the penaly parameter \(\alpha\) should be large enough.
    \item  Strong-consistency (Galerkin orthogonality):
    \[
        \mathcal{A}(u,v_h) = \int_\Omega f v_h \ \ \forall \; v_h \in V_h^p \Rightarrow \mathcal{A}(u-u_h, v_h) = 0 \quad \forall \; v_h \in V_h^p
    \]
    \item Approximation. Let \(\prod_{h}^{p}u \in V_h^p\) be a suitable approximation of \(u\), then 
    \[
        \threenorm{u-{\textstyle \prod_{h}^{p}u}}{DG} \lesssim \frac{h^{\min(p,s)}}{p^{s-\frac{1}{2}}}\norm{u}_{H^{s+1}(\triangulation)}
    \]
    If \(p \geq s\)
    \[
        \threenorm{u-{\textstyle \prod_{h}^{p}u}}{DG} \lesssim \left( \frac{h}{p} \right)^s p^{\frac{1}{2}}\norm{u}_{H^{s+1}(\triangulation)}
    \]
\end{itemize}
\subsection{Error estimates}
Recall the abstract error estimate \(\norm{u - u_h}_{DG} \lesssim \threenorm{u-\prod_{h}^{p}}{DG}\).

If \(u\) is sufficiently regular then
\[
    \norm{u-u_h}_{DG} \lesssim \frac{h^{\min(p,s)}}{p^{s-\frac{1}{2}}}\norm{u}_{H^{s+1}(\triangulation)}
\]
Then, by using a duality argument, one can obtain an estimate for the \(L^2\) norm.

Assuming that \(\Omega\) is such that the following elliptic regularity result holds: for any \(g \in L^2(\Omega)\), the solution \(z\) of the problem
\[
   \begin{cases}
     -\Delta z = g & \txt{in }\Omega \\
     z = 0 & \txt{on } \partial\Omega
   \end{cases}
\]
satisfies \(z \in H^2(\Omega)\) and 
\[
    \norm{z}_{H^2(\Omega)} \lesssim \norm{g}_{L^2(\Omega)}
\]
If the exact solution \(u \in H^s(\Omega), s \geq 2\) and, if \(u_h\) is obtained with the SIP method, it holds 
\[
    \norm{u-u_h}_{L^2(\Omega)} \lesssim \frac{h^{\min(p,s)+1}}{p^{s+\frac{1}{2}}}\norm{u}_{H^{s+1}(\Omega)}
\]
while for NIP and IIP holds
\[
    \norm{u-u_h}_{L^2(\Omega)} \lesssim \frac{h^{\min(p,s)}}{p^{s-\frac{1}{2}}}\norm{u}_{H^{s+1}(\Omega)}
\]
\newpage
\section{Advection-Diffusion-Reaction equations}
\subsection{Formulation of the problem}
Consideting the problem \(\mathcal{L}u = f\) in \(\Omega\), \(u=0\) on \(\partial\Omega\) where 
\begin{align*}
    &\mathcal{L} u = -\div \left(\mu \grad u + \vect{b} u\right) + \sigma u & \txt{(conservative form)} \\
    &\mathcal{L} u = -\div \left(\mu \grad u\right) + \vect{b} \cdot \grad u + \sigma u & \txt{(non-conservative form)}
\end{align*}
with the same assumptions as \eqref{bvp_first_page}. 

The weak formulation is written as 
\begin{equation}
    \txt{find } u \in V = H^1_0(\Omega) : a(u,v) = F(v) \ \forall \; v \in V
    \label{weak_formulation_adr}
\end{equation}
with 
\[
    F(v) = \int_\Omega fv
\]
and 
\[
    a(u,v) = \begin{cases}
        \displaystyle\int_\Omega \left( \mu \grad u + \vect{b} u \right) \cdot \grad v + \int_\Omega \sigma u v & \txt{conservative form} \\
        \\
        \displaystyle\int_\Omega \mu \grad u \cdot \grad v + \int_\Omega \vect{b} \cdot \grad u v + \int_\Omega \sigma u v & \txt{non-conservative form}
    \end{cases}
\]
Let's verify the uniqueness of the solution:
\subsubsection*{Coercivity}
Sufficient conditions for coercivity:
\begin{align*}
    &\sigma - \frac{1}{2} \div \vect{b} \geq 0 \txt{ in } \Omega & \txt{non-conservative case} \\
    &\sigma + \frac{1}{2} \div \vect{b} \geq 0 \txt{ in } \Omega & \txt{conservative case}  
\end{align*}
In both cases: \(a(u,v) \geq \mu_0\norm{\grad u}^2 \rightarrow \txt{ coercivity constant } \alpha \simeq \mu_0\)
\subsubsection*{Continuity}
In both cases, continuity constant: \(M \simeq \norm{\mu}_{L^\infty} + \norm{\vect{b}}_{L^\infty} + \norm{\sigma}_{L^2}\)

Given that the hypotheses of Lax-Milgram holds, the solution exists and is unique. We can now bring in the Galerkin formulation
\[
    \txt{find }u_h \in V_h : a(u_h, v_h) = (f,v_h) \quad \forall \; v_h \in V_h
\]
and move to the error estimate 
\[
    \norm{u-u_h} \underset{(\txt{Ceà})}{\leq} \frac{M}{\alpha} \inf_{v_h \in V_h} \norm{u-v_h} \underset{\substack{(\txt{interpolation} \\\txt{error estimate})}}{\leq} C \frac{M}{\alpha} h^r \abs{u}_{H^{r+1}(\Omega)}
\]
If it is a convection dominated flow (or reaction dominated), then \(\frac{M}{\alpha} \gg 1\), then we need to find a tradeoff between \(\frac{M}{\alpha}\) and \(h^r\). Also it is numerically prohibitive. 

The Péclet number tells us if the flow is dominated by advection or diffusion if its greater or smaller than \(1\). We can define it as 
\[
    \mathbb{P}e = h\frac{M}{\alpha}
\]
Should be less than \(1\) for stability issues.
\subsection{Stabilization methods}
The idea now is to stabilize the Galerkin method. 
\begin{itemize}
    \item 1D case: Upwind method \(\iff\) Artificial diffusion
    \item 2D case: Streamline diffusion: 
    \[
        +c(h) \int_\Omega \frac{1}{\norm{\vect{b}}} \left( \vect{b} \cdot \grad u_h \right) \left( \vect{b} \cdot \grad v_h \right)
    \]
    Artificial diffusion: 
    \[
        +c(h)\int_\Omega \grad u_h \cdot \grad v_h 
    \]
\end{itemize}
Now the solution is stabilized, but is not fully consistent. So the solution is to find a way to obtain a fully consistent solution 
\[
    \txt{find } u_h \in V_h : a(u_h, v_h) + \mathscr{L}_h (u_h, f; v_h) = F(v_h) \quad \forall \; v_h \in V_h
\]
with \(\mathscr{L}_h\) suitably chosen such that
\[
    \mathscr{L}(u_h, f; v_h) = 0 \quad \forall \; v_h \in V_h
\]
so we obtain a strongly consistent approximation of the original problem.

One possibility could be to use an operator proportional to the residual:
\[
    \mathscr{L}_h(u_h, f;v_h) = \sum_{\element \in \triangulation} \int_\element(\mathcal{L}u - f)\tau_\element \phi(v_h) \quad \forall \; v_h \in V_h  
\]
with \(\tau_\element\) as a scaling factor. Typically is chosen, given \(h_\element = diam(\element)\):
\[
    \tau_\element (\vect{x}) = \delta \frac{h_\element}{\abs{\vect{b}(\vect{x})}} \quad \forall \; \vect{x} \in \element, \element \in \triangulation
\]
while, for \(\phi(v_h)\) there are many possibilities. Two of them are 
\begin{itemize}
    \item \(\phi(v_h) = \mathcal{L}v_h \rightarrow\) GLS - Galerkin Least Squares method
    \item \(\phi(v_h) = \mathcal{L}_{ss}v_h \rightarrow\) SUPG - Streamline Upwind Petrov-Galerkin method
\end{itemize}
Brief notation remark: \(\mathcal{L} = \mathcal{L}_s +\mathcal{L}_{ss}\) (symmetric + skew-symmetric part)
Which we define as 
\begin{align*}
    {}_{V'}\langle\mathcal{L}_su, v \rangle_{V} &= {}_{V}\langle v, \mathcal{L}_su \rangle_{V'}  \quad \forall \; u, v \in V \\
    {}_{V'}\langle\mathcal{L}_{ss}u, v \rangle_V &= -{}_{V}\langle v, \mathcal{L}_{ss}u \rangle_{V'} \quad \forall \; u, v \in V 
\end{align*}
For matrices it is \(A = A_S + A_{SS}\) 
\[
  A_S = \frac{1}{2} (A + A^T) \quad A_{SS} = \frac{1}{2} (A -A^T)
\]
Let us see an example in the non conservative form 
\begin{align*}
        \mathcal{L}^1 &= -\mu \Delta u + \vect{b} \cdot \grad u + \sigma u \\
        &= \underbrace{\left[ -\mu \Delta u + \left( \sigma - \frac{1}{2} \div \vect{b} \right)u\right]}_{\mathcal{L}^1_s u} + \underbrace{\left[ \frac{1}{2} \left( \div(\vect{b}u) + \vect{b} \cdot \grad u \right) \right]}_{\mathcal{L}^1_{ss}u}
\end{align*}
Indeed we can see
\begin{align*}
    {}_{V'}\langle \mathcal{L}^1_s, v \rangle_V &= \int_\Omega \mu \grad u \cdot \grad v + \left( \sigma - \frac{1}{2} \div \vect{b} \right) u v \\
    &= \int_\Omega \left[ -\mu \Delta v + \left( \sigma - \frac{1}{2} \div \vect{b} \right) v \right] u = {}_{V}\langle v, \mathcal{L}^1_s \rangle_{V'}
\end{align*}
\begin{align*}
    {}_{V'}\langle \mathcal{L}^1_{ss}, v \rangle_V &= \frac{1}{2} \int_\Omega (\div(\vect{b}u)v+(\vect{b} \cdot \grad u)v) \\
    &= \frac{1}{2} \int_\Omega (-(\vect{b}u) \grad v + (\vect{b}v) \cdot \grad u) \\
    &= \frac{1}{2} \int_\Omega (-(\vect{b}\cdot \grad v)u - \div(\vect{b}v)u) = -{}_V\langle u, \mathcal{L}_{ss}^1 \rangle_{V'}
\end{align*}
\begin{remark}
    If \(\div \vect{b} = 0\), which happens if \(\vect{b}\) is constant, then the conservative and non conservative forms coincide.
\end{remark}
\subsection{GLS method (conservative form)}
\[
    \txt{find }u_h \in V_h : a(u_h, v_h) + \sum_{\element \in \triangulation} \int_\Omega \mathcal{L}u_h \tau_\element \mathcal{L}v_h = \int_\Omega f v_h + \sum_{\element \in \triangulation} \int_\Omega f \tau_\element \mathcal{L}v_h \qquad \forall \; v_h \in V_h
\]
\begin{theorem}
    Consider the conservative case. Suppose that 
    \[
        \exists \gamma_0 , \gamma_1 > 0 : 0 < \gamma_0 \leq \gamma(\vect{x}) \leq \gamma_1
    \]
    then, for a suitable constant \(C\), independent of \(h\), we have:
    \[
        \norm{u_h}_{GLS}^2 \leq C \norm{f}_{L^2(\Omega)}^2
    \]
    where \(\normdot_{GLS}\) will be defined later
\end{theorem}
\begin{proof}
    Take \(u_h = v_h\). We have 
    \begin{align*}
        a_h(u_h, u_h) &= \int_\Omega \mu\abs{\grad u_h}^2 + \underbrace{\int_\Omega \div(\vect{b}\,u_h) u_h}_{\mathclap{\substack{=-\int_\Omega \vect{b}\cdot (u_h \grad u_h) \\ = -\frac{1}{2}\int_\Omega \vect{b}\cdot \grad(u^2_h) \\ =\frac{1}{2} \int_\Omega \div \vect{b} \, u^2_h}}} +\int_\Omega \sigma u^2_h + \sum_{\element \in \triangulation} \int_\element \tau_\element (\mathcal{L}u_h)^2 \\
        &= \int_\Omega \mu \abs{\grad u_h}^2 + \int_\Omega \underbrace{\left( \sigma + \frac{1}{2} \div \vect{b} \right)u_h^2}_{=: \gamma(\vect{x})} + \sum_{\element \in \triangulation} \int_\element \tau_\element(\mathcal{L}u_h)^2 \\
        &=: \norm{u_h}_{GLS}^2
    \end{align*}
    On the other hand
    \[
        \abs{F_h(u_h)} \leq \abs{\int_\Omega f u_h} + \abs{{\textstyle \sum_{\element \in \triangulation}}\int_{\element} f \tau_\element \mathcal{L}u_h}
    \]
    where 
    \begin{equation*}
        \begin{split}
            \abs{\int_\Omega f u_h} = \abs{\int_\Omega \frac{1}{\sqrt{\gamma}} f \sqrt{\gamma}u_h} \underset{\txt{\tiny{Cauchy-Schwartz}}}{\leq} \norm{\frac{1}{\sqrt{\gamma}}f}_{L^2(\Omega)} \ltwonorm{\sqrt{\gamma}u_h} \\
            \underset{\txt{\tiny Young}}{\leq} \ltwonorm{\frac{1}{\sqrt{\gamma}}f}^2 + \frac{1}{4} \ltwonorm{\sqrt{\gamma}u_h}^2
        \end{split}
    \end{equation*}
    and where 
    \begin{equation*}
        \begin{split}
            \abs{\sum_{\element \in \triangulation} \int_\element f \tau_\element \mathcal{L}u_h} = \abs{\sum_{\element \in \triangulation} \int_\element \sqrt{\tau_\element} f \sqrt{\tau_\element}\mathcal{L}u_h} \\
            \underset{\txt{\tiny Cauchy-Schwartz}}{\leq} \sum_{\element\in\triangulation}\norm{\sqrt{\tau_\element}f}_{L^2(\element)}\norm{\sqrt{\tau_\element} \mathcal{L}u_h}^2_{L^2(\element)} \\
            \underset{\txt{\tiny Young}}{\leq} \sum_{\element\in\triangulation}\norm{\sqrt{\tau_\element}f}_{L^2(\element)}^2 +\frac{1}{4} \norm{\sqrt{\tau_\element} \mathcal{L}u_h}^2_{L^2(\element)} \\
        \end{split}
    \end{equation*}
    So, \(a_h(u_h, u_h) = F_h(u_h)\) implies:
    \begin{align*}
        \norm{u_h}_{GLS}^2 &= \int_\Omega \mu \abs{\grad u_h}^2 + \int_\Omega \gamma u_h^2 + \sum_{\element \in \triangulation} \int_\element \tau_\element (\mathcal{L}u_h)^2 \\
        &\leq \left[ \norm{\frac{1}{\sqrt{\gamma}}f}_{L^2(\Omega)} + \sum_{\element \in \triangulation} \norm{\sqrt{\tau_\element}f}_{L^2(\Omega)}^2 \right] \\
        & \ + \frac{1}{4} \left[ \int_\Omega \gamma u_h^2 + \sum_{\element \in \triangulation} \tau_\element (\mathcal{L}u_h)^2 \right] \\
        &\leq \underbrace{\left( \frac{1}{\gamma_0} +\max_{\element \in \triangulation} \tau_\element \right)}_{\mathclap{=C(\txt{if }\tau_\element \txt{ uniformly bounded w.r.t. }h)}} \ltwonorm{f}^2 +\frac{1}{4} \norm{u_h}_{GLS}^2
    \end{align*}
    In the end 
    \[
        \norm{u_h}_{GLS}^2 \leq \frac{4}{3} C \norm{f}_{L^2(\Omega)}^2
    \]
\end{proof}
As we already said, a smart choice for \(\scalingfactor\) is \(\delta\frac{h_\element}{\abs{\vect{b}(\vect{x})}}\). But another possibility may be 
\[
    \scalingfactor(\vect{x}) = \frac{h_\element}{2\abs{\vect{b}(\vect{x})}} \xi(\peclet_\element)
\]
with \(\xi(\theta) = \coth(\theta)-\frac{1}{\theta}\). and \(\peclet_\element (\vect{x}) = \frac{\abs{\vect{b}(\vect{x})}}{2\mu(\vect{x})}h_\element\) is the local Péclet number. 
Moreover, if \(\theta \to 0\), then \(\xi(\theta) = \frac{\theta}{3} + o(\theta)\), therefore when \(\peclet_\element(\vect{x}) \ll 1\), we have \(\tau_\element(\vect{x}) \to 0\) and no stabilization is needed.
\subsection{Convergence of GLS}
To state the convergence of GLS we need the inverse inequality, defined as
\begin{equation}
    \sum_{\eit} h^2_\element \int_\element \abs{\Delta v_h}^2 \, d\element \leq C_0 \ltwonorm{\grad v_h}^2 \quad \forall \; v_h \in X_h^r
    \label{inverse_inequality_adr}
\end{equation}
\begin{theorem}[Convergence of GLS]
    Assume that the space \(V_h\) satisfies the following local approximation property: for each \(v \in V \cap H^{r+1}(\Omega)\), there exists a function \(\hat{v}_h \in V_h\) s.t. 
    \begin{equation}
        \norm{v-v_h}_{L^2(\element)} + h_\element \norm{v-\hat{v}_h}_{H^1(\element)} + h^2_\element \abs{v-\hat{v}_h}_{H^2(\element)} \leq Ch^{r+1}_\element \abs{v}_{H^{r+1}}
        \label{convergence_gls_adr}
    \end{equation}
    for each \(\eit\). Moreover, we suppose that for each \(\eit\) the local Péclet number of \(K\) satisfies 
    \begin{equation}
        \peclet_\element (\vect{x}) = \frac{\abs{\vect{b}(\vect{x})}h_\element}{2\mu} > 1 \quad \forall \; \vect{x} \in \element
        \label{local_peclet_adr}
    \end{equation}
    that is, we are in the pre-asymptotic regime. Finally, we suppose that the inverse inequality holds and that the stabilization parameters satisfies the relation \(0 < \delta \leq 2C_0^{-1}\).

    Then, as long as \(u \in H^{r+1}(\Omega)\), the following super-optimal estimate holds:
    \begin{equation}
        \norm{u-u_h}_{GLS} \leq Ch^{r+\frac{1}{2}} \abs{u}_{H^{r+1}(\Omega)}        
        \label{super-estimate_adr}
    \end{equation}
\end{theorem}
\begin{proof}
    First of all, rewrite the error as 
    \begin{equation}
        e_h = u_h - u = \sigma_h - \eta
        \label{error_rewrite_adr}
    \end{equation}
    
    with \(\sigma_h = u_h - \hat{u}_h, \eta = u - \hat{u}_h\), where \(\hat{u}_h\) is a function that depends on \(u\) and that satisfies property \eqref{convergence_gls_adr}. If, for instance, \(V_h = X_h^r \cap H^1_0(\Omega)\), we can choose \(\hat{u}_h = \prod_{h}^{r} u\) that is the finite element interpolant of \(u\). 

    We start by estimating the norm \(\norm{\sigma_h}_{GLS}\). By exploiting the strong consistency of the GLS scheme we obtain 
    \[
        \norm{\sigma_h}^2_{GLS} = a_h(\sigma_h, \sigma_h) = a_h(u_h - u +\eta, \sigma_h) = a_h(\eta, \sigma_h)
    \]
    Now, thanks to the homogeneous Dirichlet boundary conditions it follows that, by adding and subtracting \(\sum_{\eit}\left(\eta, \mathcal{L}\sigma_h\right)_\element\), suitable computations lead to:
    \begin{align*}
        a_h(\eta, \sigma_h) &= \mu \sigma_\Omega \grad \eta \cdot \grad \sigma_h \, d\Omega - \int_\Omega \eta \vect{b} \cdot \grad \sigma_h \, d\Omega + \int_\Omega \sigma \eta \sigma_h \, d\Omega \\
        & \quad + \sum_\eit \delta \left( \mathcal{L}\eta, \frac{h_\element}{\abs{\vect{b}}}\mathcal{L}\sigma_h \right)_{L^2(\element)} \\
        &= \underbrace{\mu\left( \grad \eta, \grad \sigma_h \right)_{L^2(\Omega)}}_{(I)} - \underbrace{\sum_\eit \left(\eta, \mathcal{L}\sigma_h\right)_{L^2(\Omega)}}_{(II)} + \underbrace{2\left( \gamma \eta, \sigma_h \right)_{L^2(\element)}}_{(III)} \\
        &\quad + \underbrace{\sum_\eit \left( \eta, -\mu\Delta \sigma_h \right)_{L^2(\element)}}_{(IV)} + \underbrace{\sum_\eit \delta \left( \mathcal{L}\eta, \frac{h_\element}{\abs{\vect{b}}\mathcal{L}\sigma_h} \right)_{L^2(\element)}}_{(V)} 
    \end{align*}
    Now, we bound each of these terms. By using Cauchy-Schwartz and Young's inequalities we obtain 
    {
        \allowdisplaybreaks
        \begin{align*}
        \abs{(I)} &= \abs{\mu\left( \grad \eta, \grad \sigma_h \right)_{L^2(\element)}} \leq \frac{\mu}{4} \ltwonorm{\grad \sigma_h}^2 + \mu \ltwonorm{\grad \eta}^2 \\
        \abs{(II)} &= \abs{\sum_eit \left( \eta, \mathcal{L}\sigma_h \right)_{L^2(\element)}} \\
        & = \abs{\sum_\eit \left( \sqrt{\frac{\abs{\vect{b}}}{\delta h_\element}} \eta, \sqrt{\frac{\delta h_\element}{\abs{\vect{b}}}} \mathcal{L}\sigma_h \right)_{L^2(\Omega)}} \\
        &\leq \frac{1}{4} \sum_\eit \delta \left( \frac{h_k}{\abs{\vect{b}}} \mathcal{L}\sigma_h, \mathcal{L}\sigma_h \right) \\
        \abs{(III)} &= 2\abs{\left( \gamma \eta, \sigma_h \right)_{L^2(\Omega)}} = 2\abs{\left( \sqrt{\gamma \eta, \sqrt{\gamma}\sigma_h} \right)_{L^2(\Omega)}} \\
        &\leq \frac{1}{2} \ltwonorm{\sqrt{\gamma}\sigma_h}^2 + 2 \ltwonorm{\sqrt{\gamma} \eta}^2
    \end{align*}
    }
    Then, thanks to CS and Young, but also hypotheses \eqref{local_peclet_adr} and \eqref{inverse_inequality_adr}, we obtain 
    \begin{align*}
        \abs{(IV)} &= \abs{\sum_\eit \left( \eta, -\mu \Delta \sigma_h \right)_{L^2(\element)}} \\
        &\leq \frac{1}{4} \sum_\eit \delta \mu^2 \left( \frac{h_\element}{\abs{\vect{b}}}\Delta \sigma_h, \Delta \sigma_h \right)_{L^2(\element)} + \sum_\eit \left( \frac{\abs{\vect{b}}}{\delta h_\element} \eta, \eta \right)_{L^2(\element)} \\
        &\leq \frac{1}{8} \delta \mu \sum_\eit h^2_\element \left( \grad \sigma_h, \grad \sigma_h \right)_{L^2(\element)} + \sum_\eit \left( \frac{\abs{\vect{b}}}{\delta h_\element} \right)_{L^2(\element)} \\
        &\leq \frac{\sigma C_0 \mu}{8} \ltwonorm{\grad \sigma_h}^2 + \sum_\eit \left( \frac{\abs{\vect{b}}}{\delta h_\element}\eta, \eta \right)_{L^2(\element)}
    \end{align*}
    The last one can be bounded once again thanks to CS and Young inequalities as follows 
    \begin{align*}
        \abs{(V)} &= \abs{\sum_\eit \delta \left( \mathcal{L}\eta, \frac{h_\element}{\vecbabs}\mathcal{L}\sigma_h \right)_{L^2(\element)}} \\
        &\leq \frac{1}{4} \sum_\eit \delta \left( \frac{h_\element}{\vecbabs} \mathcal{L}\sigma_h, \mathcal{L}\sigma_h \right)_{L^2(\element)} + \sum_\eit \delta\left( \frac{h_\element}{\vecbabs} \mathcal{L}\eta, \mathcal{L}\eta \right)_{L^2(\element)} 
    \end{align*}
    So we can rewrite everything bounded as 
    \begin{align*}
        \norm{\sigma_h}^2_{GLS} &= a_h(\eta, \sigma_h) \leq \frac{1}{4} \norm{\sigma_h}^2_{GLS} \\
        &\quad + \frac{1}{4} \left( \ltwonorm{\sqrt{\gamma}\sigma_h}^2 + \sum_\eit \delta \left( \frac{h_\element}{\vecbabs} \mathcal{L}\sigma_h,\mathcal{L} \sigma_h \right)_{L^2(\element)} \right) + \frac{\delta C_0 \mu}{8} \ltwonorm{\grad \sigma_h}^2 \\
        &\quad + \underbrace{\mu\ltwonorm{\grad \eta}^2 + 2 \sum_\eit \left( \frac{\vecbabs}{\delta h_\element} \eta, \eta \right)_{L^2(\element)} + 2 \ltwonorm{\sqrt{\gamma}\eta}^2 + \sum_\eit \delta \left( \frac{h_\element}{\vecbabs} \mathcal{L}\eta, \mathcal{L}\eta \right)_{L^2(\element)}}_{\mathcal{E}(\eta)} \\
        &\leq \frac{1}{2} \norm{\sigma_h}_{GLS}^2 + \mathcal{E}(\eta)
    \end{align*}
    Having exploited the assumption that \(\delta \leq 2C^{-1}_0\). We can state then
    \[
        \norm{\sigma_h}_{GLS}^2 \leq 2\mathcal{E}(\eta)
    \]
    It's time to estimate \(\mathcal{E}(\eta)\), by bounding each of it's summands separately. To do this, we will use the local approximation property \eqref{convergence_gls_adr} and the local Péclet \eqref{local_peclet_adr}. Moreover, we observe that the constant \(C\), introduced in the remainder, depends neither on \(h\) nor on \(\peclet_\element\), but can depend on other quantities such as the constant \(\gamma_1\), the reaction constant \(\sigma\) or the norm \(\norm{\vect{b}}_{L^\infty(\Omega)}\), the stabilization parameter \(\delta\).

    Then we have 
    \begin{align*}
        \mu \ltwonorm{\grad \eta}^2 &\leq C\mu h^{2r} \abs{u}_{H^{r+1}(\Omega)}^2 \\
        &\leq C\frac{\norm{\vect{b}}_{L^\infty(\Omega) h}}{2} h^{2r} \abs{u}^2_{H^{r+1}(\Omega)} \leq C h^{2r+1}\abs{u}_{H^{r+1}(\Omega)}^2\\
        2\sum_\eit \left( \frac{\vecbabs}{\delta h_\element} \eta, \eta \right)_{L^2(\element)} &\leq  C\frac{\norm{\vect{b}}_{L^\infty(\Omega) h}}{2} \sum_\eit \frac{1}{h_\element} h^{2r+1}_\element \abs{u}^2_{H^{r+1}(\Omega)} \\
        &\leq c h^{2r+1} \abs{u}^2_{H^{r+1}(\Omega)} \\
        2 \ltwonorm{\sqrt{\gamma}\eta}^2 &\leq 2\gamma_1 \ltwonorm{\eta}^2 \leq Ch^{2r+1} \abs{u}^2_{H^{r+1}(\Omega)} 
    \end{align*} 
    For the fouth term we have 
    \begin{align*}
        \sum_\eit \delta \left( \frac{h_\element}{\vecbabs}\mathcal{L}\eta, \mathcal{L}\eta \right)_{L^2(\element)} &= \sum_\eit \norm{\sqrt{\frac{h_\element}{\vecbabs}} \mathcal{L}\eta}_{L^2(\element)}^2 \\
        &= \sum_\eit \delta \norm{-\mu \sqrt{\frac{h_\element}{\vecbabs}} \Delta \eta + \sqrt{\frac{h_\element}{\vecbabs}} \div(\vect{b}\eta) + \sigma \sqrt{\frac{h_\element}{\vecbabs}} \eta}_{L^2(\element)}^2 \stepcounter{equation}\tag{\theequation}\label{fourth_term_adr}\\
        &\leq C \sum_\eit \delta \Bigg(\norm{\mu \sqrt{\frac{h_\element}{\vecbabs}} \Delta \eta}_{L^2(\element)}^2 + \norm{\sqrt{\frac{h_\element}{\vecbabs}} \div(\vect{b}\eta)}^2_{L^2(\element)} \\
        &\quad + \norm{\sigma \sqrt{\frac{h_\element}{\vecbabs}}\eta}_{L^2(\element)}^2\Bigg) 
    \end{align*}
    Now it is easy to prove that the second and third term of the summands can be bounded using a term or the form \(Ch^{2r+1}\abs{u}_{H^{r+1}(\Omega)}^2\), for a suitable choice of the constant \(C\). For the first term we have 
    \begin{align*}
        \sum_\eit \delta \norm{\mu \sqrt{\frac{h_\element}{\vecbabs}} \Delta \eta}^2_{L^2(\element)} &\leq \sum_\eit \delta \frac{h^2_\element \mu}{2} \norm{\Delta \eta}^2_{L^2(\element)} \\
        &\leq C \delta \norm{\vect{b}}_{L^\infty(\Omega)}\sum_\eit h^3_\element \norm{\Delta}_{L^2(\element)}^2 \leq \abs{u}_{H^{r+1}(\Omega)}^2
    \end{align*}
    having used again \eqref{convergence_gls_adr} and \eqref{local_peclet_adr}. Now we can conclude that 
    \[
        \mathcal{E}(\eta) \leq C h^{2r+1}\abs{u}_{H^{r+1}(\Omega)}^2
    \]
    that is 
    \begin{equation}
        \norm{\sigma_h}_{GLS} \leq C h^{r+\frac{1}{2}}\abs{u}_{H^{r+1}(\Omega)}
        \label{final_proof_adr}
    \end{equation}
    Reverting to \eqref{error_rewrite_adr}, to obtain the desired estimate for the norm \(\norm{u_h-u}_{GLS}\) we need to estimate \(\norm{\eta}_{GLS}\). But thanks to \eqref{fourth_term_adr} we obtain 
    \[
        \norm{\eta}_{GLS} \leq C h^{r+\frac{1}{2}}\abs{u}_{H^{r+1}(\Omega)}
    \]
    Combining this with \eqref{final_proof_adr} we obtain \eqref{super-estimate_adr}.
\end{proof}

\newpage
\section{Parabolic equations}
\subsection{Introduction}
Now we consider parabolic equations of the form
\begin{equation}
    \partialderivative{u}{t} + \loperator u = f \quad \vect{x} \in \Omega, t>0
    \label{example_problem_pbl}
\end{equation}
where: 
\begin{itemize}
    \item \(\Omega\) is a domain of \(\real^d\) with \(d = 1,2,3\)
    \item \(f = f(\vect{x}, t)\) is a given function 
    \item \(\loperator = \loperator (\vect{x})\) is a generic elliptic operator acting on \(u = u(\vect{x}, t)\)
\end{itemize}
When solved for a bounded time interval, for example \(0 < t < T\), the region \(Q_T = \Omega \times (0,T)\) is called cylinder in the space \(\real^d \times \real^+\). 
In \eqref{example_problem_pbl} must be assigned an initial condition 
\begin{equation}
    u(\vect{x}, 0) = u_0(\vect{x}),\quad \xvec \in \Omega
    \label{initial_cond_pbl}
\end{equation}
also we'll need some BC, like 
\begin{equation}
    \begin{aligned}
        u(\xvec, t) &= \phi(\xvec, t) & \xvec \in \Gamma_D \txt{ and } t > 0 \\
        \partialderivative{u(\xvec, t)}{n} &=\psi(\xvec, t) & \xvec \in \Gamma_N \txt{ and } t > 0 
    \end{aligned}
    \label{BC_example_pbl}
\end{equation}
where \(u_0, \phi\) and \(\psi\) are given funcion and \(\left\{ \Gamma_D, \Gamma_N \right\}\) provides a boundary partition that is \(\Gamma_D \cup \Gamma_N = \boundary, \interior{\Gamma_D} \cap \interior{\Gamma_N} = \emptyset\). For obvious reasons \(\Gamma_D\) is the Dirichlet boundary, while \(\Gamma_N\) is the Neumann one.

In the one dimensional case the problem becomes 
\begin{equation}
    \begin{aligned}
        &\partialderivative{u}{t} - \nu \partialderivative{^2u}{x^2} = f & 0 < x <d, t> 0 \\
        &u(x,0) = u_0(x) & 0 < x < d \\
        & u(0,t) = u(d,t) = 0 & t>0
    \end{aligned}
    \label{one_dim_pbl}
\end{equation}
which describes the evolution of the temperature \(u(x,t)\) at point \(x\) and time \(t\) of a metal bar of length \(d\) occupying the interval \([0,d]\), whose thermal conductivity is \(\nu\) and whose endpoints are kept at a constant temperature of zero degrees. The function \(u_0\) describes the temperature in the initial state, while \(f\) represents the heat generated per unit of length by the bar. This is called the heat equation.
\subsection{Weak formulation and approximation}
We proceed as usual by multiplying for each \(t>0\) the differential equation by a test function \(v = v(\xvec)\) and integrating. We set \(V = H^1_{\Gamma_D}(\Omega)\) and, for each \(t>0\), we seek \(u(t)\in V\) s.t. 
\begin{equation}
    \int_{\Omega} \partialderivative{u(t)}{t} v \, d\Omega + a(u(t), v) = \int_\Omega f(t)v \, d\Omega \quad \forall \; v \in V 
    \label{weak_formulation_heat_pbl}
\end{equation}
where 
\begin{itemize}
    \item \(u(0) = u_0\)
    \item \(a(\cdot,\cdot)\) is the bilinear form associated to the operator \(\loperator\)
    \item we have supposed for simplicity \(\phi=0\) and \(\psi=0\)
\end{itemize}
\begin{definition}
    A bilinear form \(a(\cdot, \cdot)\) is said weakly coercive if 
    \[
        \exists \; \lambda \geq 0, \exists \; \alpha > 0 : a(v,v) + \lambda \norm{v}^2_{L^2(\Omega)} \geq \alpha \norm{v}^2_V \quad \forall \; v \in V
    \]
\end{definition}
Rationale for weak coercivity 
\[
    \partialderivative{u}{t} +\loperator u = f 
\]
now we perform a change of variable \((u = e^{\lambda t})\)
\[
    \partialderivative{w}{t} + \loperator w + \lambda w = e^{-lambda t} f
\]
so that the bilinear form 
\[
    \tilde{a}(w,v) := a(w,v)+\lambda(w,v)_{L^2(\Omega)} \Rightarrow \tilde{a}(w,w) := a(w,w) + \lambda \ltwonorm{w}^2
\]
\begin{theorem}
    Suppose that the bilinear form \(a(\cdot,\cdot)\) is continuous and weakly coercive. Moreover, we require \(u_0 \in L^2(\Omega)\) and \(f \in L^2(Q_T)\)- Then, \eqref{weak_formulation_heat_pbl} admits a unique solution \(u \in \mathcal{C}^0 (\real^+;L^2(\Omega))\). Also, \(u \in L^2(\real^+;V)\) and \(\partialderivative{u}{t} \in L^2(\real^+; V')\).
\end{theorem}
\subsection{Algebraic formulation}
Now we can use Galerkin to approximate, for each \(t > 0\), we need to find \(u_h(t) \in V_h\) s.t. 
\begin{equation}
    \int_\Omega \partialderivative{u_h(t)}{t} v_h \, d\Omega + a(u_h(t), v_h) = \int_\Omega f(t)v_h \, d\Omega \quad \forall \; v_h \in V_h
\label{semi-discretization_pbl}
\end{equation}
with \(u_h(0) = u_{0h}\), where \(V_h \subset V\) is a suitable space of finite dimension and \(u_{0h}\) is a convenient approximation of \(u_0\) in the space \(V_h\).

Now we need to discretize the temporal variable, because, as of now, we obtained a semi-discretization of the problem.

We introduce a basis \(\left\{ \phi_j \right\}\) for \(V_h\) and we observe that it suffices that \eqref{semi-discretization_pbl} is verified for the basis function in order to be satisfied by all the functions in the subspace. 

Moreover, since for each \(t > 0\), the solution to the Galerkin problem belongs to the subspace as well, we will have
\[
    u_h(\xvec, t) = \sum_{j=1}^{N_h} u_j(t)\phi_j(\xvec)
\]
where the coefficients \(\left\{ u_j(t) \right\}\) represent the unknown of the problem. 

Denoting by \(\dot{u}_j(t)\) the temporal derivatives of \(u_j(t)\), \eqref{semi-discretization_pbl} becomes
\[
    \int_\Omega \sum_{j=1}^{N_h} \dot{u}_j(t) \phi_j\phi_i \, d\Omega + a\left( \sum_{j=1}^{N_h} u_j(t)\phi_j, \phi_i \right) = \int_\Omega f(t)\phi_i \, d\Omega, \quad i = 1,2,\ldots, N_h
\]
that is 
\begin{equation}
    \sum_{j=1}^{N_h} \dot{u}_j(t) \underbrace{\int_\Omega \phi_j \phi_i \, d\Omega}_{m_{ij}} + \sum_{j=1}^{N_h} u_j(t) \underbrace{a(\phi_j, \phi_i)}_{a_{ij}} = \underbrace{\int_\Omega f(t)\phi_i \, d\Omega} \quad i = 1,2,\ldots, N_h
    \label{algebraic_formulation_pbl}
\end{equation}
If we define the vector of unknowns \(\vect{u} = (u_1(t), u_2(t), \ldots, u_{N_h}(t))^T\), the mass matrix \(M = [m_{ij}]\), the stiffness matrix \(A = [a_{ij}]\) and the right hand side vector \(\vect{f} = (f_1(t), f_2(t), \ldots, f_{N_h}(t))^T\), the system \eqref{algebraic_formulation_pbl} can be rewritten as 
\[
    M\dot{\vect{u}}(t) + A\vect{u}(t) = \vect{f}(t)
\]
\subsection{Time discretization}
For the numerical solution of this ODE system we will use the \(\theta-\)method, which discretizes the temporal difference quotient and replaces the other terms with a linear combination of the value at time \(t^k\) and of the value at time \(t^{k+1}\), depending on the real parameter \(0 \leq \theta \leq 1\).
\begin{equation}
    M\frac{\vect{u}^{k+1} - \vect{u}^k}{\Delta t} + A[\theta \vect{u}^{k+1} + (1-\theta)\vect{u}^k] = \theta \vect{f}^{k+1} + (1-\theta)\vect{f}^k
    \label{theta_method_pbl}
\end{equation}
The real positive parameter \(\Delta t = t^{k+1}-t^k\), \(k=0,1,\ldots\) denotes the discretization step.

Some particular cases of \eqref{algebraic_formulation_pbl} 
\begin{itemize}
    \item \(\theta = 0\) wThe forward Euler method 
    \[
        M\frac{\vect{u}^{k+1}-\vect{u}^k}{\Delta t} + A\vect{u}^{k} = \vect{f}^k
    \]
    \item \(\theta = 1\) The backward Euler method 
    \[
        M\frac{\vect{u}^{k+1}-\vect{u}^k}{\Delta t} + A\vect{u}^{k+1} = \vect{f}^{k+1}        
    \]
    \item \(\theta = \frac{1}{2}\) The Crank-Nicolson (or trapezoidal) method
    \[
        M\frac{\vect{u}^{k+1}-\vect{u}^k}{\Delta t} + \frac{1}{2}A\left( \vect{u}^{k+1} + \vect{u}^k \right) = \frac{1}{2}\left( \vect{f}^{k+1} \vect{f}^k \right)
    \]
    which is of the second order in \(\Delta t\). 
\end{itemize}
Let us consider the two extremal cases \(\theta = 0\) and \(\theta = 1\). In the first case the system to solve is only the mass matrix \(\frac{M}{\Delta t}\), while in the other case \(\frac{M}{\tstep} + A\). \(M\) is invertible, being positively defined.

In the case \(\theta = 0\) the scheme is not unconditionally stable, and in the case where \(V_h\) is a subspace of finite elements, there is the following stability condition 
\[
    \exists \; c > 0 : \tstep \leq ch^2 \quad \forall \; h > 0
\]
so \(\tstep\) cannot be chosen irrespective of \(h\).

In this case, if we make \(M\) diagonal, we actually decouple the system. This operation is called lumping of the mass matrix.

When \(\theta > 0\), the system will have the from \(K\vect{u}^{k+1} = \vect{g}\), where \(\vect{g}\) is the source term and \(K = \frac{M}{\tstep}\). The latter is invariant in time (the operator \(\loperator\) being independent of time), so, if the spatial mesh doesn't change, it can be factorized only once at the beginning of the process. 

Then, since \(M\) is symmetric, if \(A\) is symmetric, also \(K\) will be symmetric too. Hence, we can use, for example, the Cholesky factorization \(K = HH^T\), with \(H\) being lower triangular. At each timestep, then, will be solved two triangular systems 
\begin{align*}
    H\vect{y} &= \vect{g} \\
    H^T \vect{u}^{k+1} = \vect{y}
\end{align*}
\subsection{A priori estimate}
Let us know consider \eqref{weak_formulation_heat_pbl}. Since the corresponding equation must hold for each \(v \in V\), it will be legitimate to set \(v = u(t)\) with \(t\) being given. yielding 
\begin{equation}
    \int_\Omega \partialderivative{u(t)}{t} u(t) = \frac{1}{2}\partialderivative{\null}{t} \int_\Omega \abs{u(t)}^2 \, d\Omega = \frac{1}{2} \partialderivative{\null}{t} \ltwonorm{u(t)}^2
    \label{a_priori_est_pbl}
\end{equation}
Considering the individual terms, the first one is 
\begin{equation}
    \int_\Omega \partialderivative{u(t)}{t} u(t) \, d\Omega = \onehalf \partialderivative{\null}{t} \int_\Omega \abs{u(t)}^2 \, d\Omega = \onehalf \partialderivative{\null}{t} \ltwonorm{u(t)}^2
    \label{first_term_est_pbl}
\end{equation}
Assuming that the bilinear form is coercive, with \(\alpha\) coercivity constant, we obtain 
\[
    a(u(t), u(t)) \geq \alpha \norm{u(t)}^2_V
\]
while, thanks to the CS inequality we find:
\begin{equation}
    (f(t), u(t)) \leq \ltwonorm{f(t)}\ltwonorm{u(t)}
    \label{CS_inequality_pbl}
\end{equation}
So \eqref{a_priori_est_pbl} becomes 
\[
    \frac{1}{2} \partialderivative{\null}{t} \ltwonorm{u(t)}^2 + \alpha \norm{u(t)}_V^2 \leq \ltwonorm{f(t)}\ltwonorm{u(t)}
\] 
Now let's define two important inequalities 
\begin{definition}[Young's inequality]
    \(\forall \; a, b \in \real\)
    \begin{equation}
        ab \leq \epsilon\alpha^2 + \frac{1}{4\epsilon} b^2 \quad \forall \; \epsilon >0 \label{young_inequality_pbl}
    \end{equation}
\end{definition}
\begin{definition}[Poincaré inequality]
    If \(\Gamma_D\) is a set of positive measure, then:
    \begin{equation}
        \exists \; C_{\Omega} > 0 : \ltwonorm{v} \leq C_\Omega \ltwonorm{\grad v} \quad \forall \; v \in H^1_{\Gamma_D(\Omega)} \label{poincare_inequality_pbl}
    \end{equation}
\end{definition}
By using these two we obtain 
\begin{equation}
    \begin{aligned}
        \frac{1}{2} \partialderivative{\null}{t} \ltwonorm{u(t)}^2 + \alpha \ltwonorm{\grad u}^2 &\leq \ltwonorm{f(t)}\ltwonorm{u(t)} \\
        & \leq \frac{C_\Omega^2}{2\alpha} \ltwonorm{f(t)}^2 + \frac{\alpha}{2} \ltwonorm{\grad u(t)}^2
        \label{a_priori_not_integrated_pbl}
    \end{aligned}
\end{equation}
Then, by integrating in time, we obtain, for all \(t > 0\)
\begin{equation}
    \ltwonorm{u(t)}^2 +\alpha \int_0^t \ltwonorm{\grad u(s)}^2 \, ds \leq \ltwonorm{u_0}^2 + \frac{C_\Omega^2}{\alpha} \int_0^t \ltwonorm{f(s)}^2 \, ds
    \label{energy_estimate_pbl}
\end{equation}
This is an a priori energy estimate. Note that 
\[
    \frac{1}{2} \partialderivative{\null}{t} \ltwonorm{u(t)}^2 = \ltwonorm{u(t)} \partialderivative{\null}{t} \ltwonorm{u(t)} 
\]
then, from \eqref{a_priori_est_pbl}, using \eqref{first_term_est_pbl}, \eqref{CS_inequality_pbl} and \eqref{poincare_inequality_pbl}
\begin{equation}
    \begin{split}
        \ltwonorm{u(t)} \partialderivative{\null}{t} \ltwonorm{u(t)} + \frac{\alpha}{C_\Omega} \ltwonorm{u(t)} \ltwonorm{\grad u(t)} \\
        \leq  \ltwonorm{f(t)}\ltwonorm{u(t)}, \quad t>0
    \end{split}
\end{equation}
If \(\ltwonorm{u(t)} \neq 0\), we can divide by \(\ltwonorm{u(t)}\) and integrate in time to obtain another estimate.
\begin{equation}
    \ltwonorm{u(t)} \leq \ltwonorm{u_0} + \int_0^t \ltwonorm{f(s)} \, ds \quad t >0
    \label{another_a_priori_pbl}
\end{equation}
Let us now use the first inequality in \eqref{a_priori_not_integrated_pbl} and integrate in time to yield 
\begin{equation}
    \begin{aligned}
        \ltwonorm{u(t)}^2 + 2\alpha \int_0^t \ltwonorm{\grad u(s)}^2 \, ds 
        &\underset{\eqref{a_priori_not_integrated_pbl}}{\leq} \ltwonorm{u(t)}^2 + 2\int_0^t \ltwonorm{f(s)}\ltwonorm{u(s)} \, ds \\
        &\underset{\eqref{another_a_priori_pbl}}{\leq} \ltwonorm{u_0}^2 + 2\int_0^t \ltwonorm{f(s)} \cdot \Bigg( \ltwonorm{u_0} \\
        &\qquad + \int_0^s \ltwonorm{f(\tau)} \, d\tau \Bigg) \, ds \\
        &= \ltwonorm{u_0}^2 \int_0^t \ltwonorm{f(s)}\ltwonorm{u_0}\, ds \\
        &\qquad+ 2 \int_0^t \ltwonorm{f(s)}\int_0^s \ltwonorm{f(\tau)} \, d\tau ds
    \end{aligned}
    \label{16_slide_pbl}
\end{equation}
Now, noticing that 
\[
    \ltwonorm{f(s)} \int_0^s \ltwonorm{f(\tau)} \, d\tau = \pderivative{s} \left( \int_0^s \ltwonorm{f} \, d\tau \right)^s
\]
we can rewrite the right hand side of \eqref{16_slide_pbl} as:
\begin{equation*}
    \begin{split}
        \ltwonorm{u_0}^2 \int_0^t \ltwonorm{f(s)}\ltwonorm{u_0}\, ds+ 2 \int_0^t \ltwonorm{f(s)}\int_0^s \ltwonorm{f(\tau)} \, d\tau ds \\
        = \left( \ltwonorm{u_0} +\int_0^t \ltwonorm{f(s)} \, ds \right)^2
    \end{split}
\end{equation*}
Therefore we can obtain the following estimate 
\begin{equation}
    \begin{split}
        \left( \ltwonorm{u(t)}^2 + 2\alpha \int_0^t \ltwonorm{\grad u(s)}^2 \, ds \right)^{\onehalf} \\
        \leq \ltwonorm{u_0}^2 + \int_0^t \ltwonorm{f(s)} \, ds \quad t >0
    \end{split}
    \label{hopefully_last_estimate_pbl}
\end{equation}
Now that we have found an estimate for \eqref{weak_formulation_heat_pbl}, we can now estimate its discretization \eqref{semi-discretization_pbl}, like \eqref{energy_estimate_pbl}
\begin{equation}
    \ltwonorm{u_h(t)}^2 +\alpha \int_0^t \ltwonorm{\grad u_h(s)}^2 \, ds \leq \ltwonorm{u_{0h}}^2 + \frac{C_\Omega^2}{\alpha} \int_0^t \ltwonorm{f(s)}^2 \, ds
    \label{energy_est_galerkin_pbl}
\end{equation}
We can proceed as before, simply taking for every \(t>0\), \(v_h = u_h(t)\). Since \(u_h(0) = u_{0h}\) we obtain the discrete counterparts of \eqref{another_a_priori_pbl} and \eqref{hopefully_last_estimate_pbl} 
\begin{equation}
    \ltwonorm{u_h(t)} \leq \ltwonorm{u_{0h}(t)} + \int_0^t
    \ltwonorm{f(s)} \,ds \quad t > 0
    \label{another_a_priori_disc_pbl}
\end{equation}
and 
\begin{equation}
    \begin{split}
            \left( \ltwonorm{u_h(t)}^2 + 2\alpha \int_0^t \ltwonorm{\grad u_h(s)}^2 \, ds\right)^{\onehalf} \\
            \leq \ltwonorm{u_{0h}(t)} + \int_0^t \ltwonorm{f(s)} \, ds \quad t > 0
    \end{split}
\end{equation}
\subsection{Convergence analysis}
\begin{theorem}
    There exists a constant \(C > 0\) independent of both \(t\) and \(h\) s.t. 
    \begin{equation*}
        \begin{split}
            \left\{ \ltwonorm{u(t) - u_h(t)}^2 + \alpha \int_0^t \ltwonorm{\grad u(s) -\grad u_h(s)}^2 \, ds \right\}^{\onehalf} \\
            \leq Ch^r \left\{ \seminorm{u_0}{H^r(\Omega)}^2 + \int_0^t \seminorm{u(s)}{H^{r+1}( Omega)}^2 \, ds + \int_0^t \seminorm{\partialderivative{u(s)}{s}}{H^{r+1}(\Omega)}^2 \, ds \right\}^{\onehalf}
        \end{split}
    \end{equation*}
\end{theorem}
\subsection{Stability analysis of the \texorpdfstring{\(\theta\)}{theta}-method}
We can now analyze the stability of discretized problem. 
By applying the \(\theta\)-method to the Galerkin problem \eqref{semi-discretization_pbl} we obtain 
\begin{equation}
    \begin{split}
        \left( \frac{u_h^{k+1}-u_h^k}{\tstep}, v_h \right) + a(\theta u_h^{k+1} + (1-\theta) u^k_h, v_h) \\
        = \theta F^{k+1}(v_h) + (1-\theta) F^k(v_h) \quad \forall \; v_h \in V_h
    \end{split}
    \label{theta_method_stab_pbl}
\end{equation}
for each \(k \geq 0\), with \(F^k\) indicating the functional evaluated at the time \(k\).
For the analysis we will consider \(F=0\), and the case of implicit Euler \(\theta = 1\).
\[
    \left( \frac{u_h^{k+1}-u^k_h}{\tstep}, v_h \right) + a(u^{k+1}_h, v_h) = 0 \quad \forall \; v_h \in V_h
\]
By choosing \(v_h = u_h^{k+1}\), we obtain 
\[
    ( u_h^{k+1}, u_h^{k+1} ) +\tstep a(u^{k+1}_h, u_h^{k+1}) = (u_h^{k}, u_h^{k+1})
\]
and then, thanks to these inequalities 
\begin{align*}
    a(u_h^{k+1}, u_h^{k+1}) &\geq \alpha \norm{u_h^{k+1}}_V^2 \\
    (u_h^{k},u_h^{k+1}) &\leq \onehalf \ltwonorm{u_h^{k}}^2 + \onehalf \ltwonorm{u_h^{k+1}}^2
\end{align*}
which are derived from the coercivity of \(a(\cdot, \cdot)\) and the CS inequality we obtain 
\begin{equation}
    \ltwonorm{u_h^{k+1}}^2 +2\alpha \tstep \norm{u_h^{k+1}}_V^2 \leq \ltwonorm{u_h^{k}}^2
    \label{22_slide_pbl}
\end{equation}
Observing that \(\norm{u_h^{k+1}}_V \geq \ltwonorm{u_h^{k+1}}\), we deduce from \eqref{22_slide_pbl} that 
\[
    (1+2\alpha \tstep)\ltwonorm{u_h^{k+1}}^2 \leq \ltwonorm{u_h^{k}}^2
\]
hence 
\[
    \ltwonorm{u_h^{k+1}} \leq \frac{1}{\sqrt{1+2\alpha \tstep}} \ltwonorm{u_h^{k}}
\]
which entails 
\[
    \ltwonorm{u_h^{k}} \leq \left( \frac{1}{\sqrt{1+2\alpha \tstep}} \right)^k\ltwonorm{u_{0h}}
\]
and therefore 
\[
    \lim_{k\to \infty} \ltwonorm{u_h^{k}} = 0
\]
so the backward Euler method is absolutely stable without any restriction on \(\tstep\).

Now we assume \(f \neq 0\) 
\[
    \underbrace{\left( \frac{u_h^{k+1} - u_h^{k}}{\tstep}, u_h^{k+1} \right)}_{(1)} + \underbrace{a(u_h^{k+1},u_h^{k+1})}_{(2)} = \underbrace{\int_\Omega f^{k+1}u_h^{k+1}}_{(3)}
\]
where 
\begin{align*}
    (1) &\geq \frac{1}{2\tstep t} \left( \ltwonorm{u_h^{k+1}}^2 -\ltwonorm{u_h^{k}}^2 \right) \\
    (2) &\geq \alpha \ltwonorm{u_h^{k+1}}^2 \\
    (3) &\underset{\txt{\tiny{(CS)}}}{\leq} \ltwonorm{f^{k+1}} \norm{u_h^{k+1}}_V \underset{\txt{\tiny{(Young)}}}{\leq} \frac{1}{2\alpha} \ltwonorm{f^{k+1}}^2 + \frac{\alpha}{2} \norm{u_h^{k+1}}_V^2
\end{align*}
Then, after summation on \(k\), for \(k = 0,\ldots, n-1\):
\begin{equation*}
    \ltwonorm{u_h^2} + \underbrace{\alpha \sum_{k=1}^{n}\tstep \norm{u_h^{k}}_V^2}_{\simeq \alpha \int_0^{t_n} \norm{u_h(t)}_V^2 \, dt} \leq \ltwonorm{u_{0h}}^2 + \underbrace{\frac{1}{\alpha} \sum_{k=1}^{n} \tstep \ltwonorm{f^k}^2}_{\simeq\frac{1}{\alpha} \int_0^{t_n} \ltwonorm{f^k}^2 \, dt}
\end{equation*}
meaning that we have unconditional stability. 

Now, before analyzing the general case, where \(0 \leq \theta \leq 1\), we need the following definition 
\begin{definition}
    We say that the scalar \(\lambda\) is an eigenvalue of the bilinear form \(a(\cdot, \cdot): V \times V \to \real\) and that \(w \in V\) is its corresponding eigenfunction if 
    \[
        a(w,v) = \lambda(w,v) \quad \forall v \in V    
    \]
\end{definition}
If the bilinear form \(\bilinear\) is symmetric and coercive, it has positive and real eigenvalues forming an infinite sequence, moreover, its eigenfunctions form a basis of the space \(V\). The eigenvalues and eigenfunctions of \(\bilinear\) can be approximated by finding the pairs \(\lambda_h \in \real\) and \(w_h \in V_h\), which satisy 
\begin{equation}
    a(w_h, w_h) \lambda_h (w_h, v_h) \quad \forall \; v_h \in V_h
    \label{eigenvalues_of_bilinear_pbl}
\end{equation}
From an algebraic viewpoint, problem \eqref{eigenvalues_of_bilinear_pbl} can be formulated as 
\[
    A\vect{w} = \lambda_h M \vect{w},
\]
where \(A\) is the stiffness matrix and \(M\) the mass matrix. This is a generalized eigenvalue problem. Such eigenvalue are all positive and \(N_h\) (the usual dimension of \(V_h\)). After ordering them in ascending order we have
\[
    \lambda_h^{N_h} \quad \txt{for } N_h \to \infty.
\]
Moreover, the corresponding eigenfunctions form a basis for the subspace \(V_h\) and can be chosen to be orthonormal w.r.t. the scalar product of \(L^2(\Omega)\). This means that, by denoting with \(w^i_h\) the eigenfunction corresponding to the eigenvalue \(\lambda^i_h\), we have \((w^i_h, w_h^j) = \delta_{ij}\), \(\forall \; i,j = 1,\ldots,N_h\). Thus, each function \(v_h \in V_h\) can be represented as follows 
\[
    v_h(\xvec) = \sum_{j=1}^{N_h} v_j w^j_h(\xvec)
\]
and, thanks to the eigenfunctions orthonormality
\begin{equation}
    \ltwonorm{v_h}^2 = \sum_{j=1}^{N_h} v_j^2
    \label{orthonormality_pbl}
\end{equation}

Let us know consider an arbitrary \(\theta \in [0,1]\) and assume that \(\bilinear\) is symmetric. Since \(u_h^k \in V_h\) we can write 
\[
    u_h^k(\xvec) = \sum_{j=1}^{N_h} u^k_j w^j_h(\xvec)
\]
In this case, however, \(u_j^k\) no longer represents the nodal values of \(u_h^k\).

If we now set \(F=0\) in \eqref{theta_method_stab_pbl} and take \(v_h = w_h^i\), we find 
\[
    \frac{1}{\tstep} \sum_{k=1}^{N_h} [u_j^{k+1} - u_j^k](w_h^i, w_h^j) + \sum_{j=1}^{N_h} + \sum_{j=1}^{N_h}[\theta u_j^{k+1} + (1-\theta)u_j^k]a(w_h^j, w_h^i) = 0,
\]
for each \(i = 1,\ldots, N_h\). For each pair \(i,j = 1, \ldots, N_h\) we have 
\[
    a(w_h^j, w_h^i) = \lambda^j_h (w_h^j, w_h^i) = \lambda^i_h,
\]
and thus, for each \(i = 1,\ldots, N_h\), 
\[
    \frac{u_i^{k+1} - u^k_i}{\tstep} + [\theta u^{k+1}_i + (1-\theta)u^k_i] \lambda_h^i = 0.
\]
Solving for \(u^{k+1}_i\), we find:
\[
    u_i^{k+1} = u_i^k \frac{1-(1-\theta) \lambda_h^i \tstep}{1+\theta \lambda_h^i\tstep}.
\]
Now, recalling \eqref{orthonormality_pbl}, we can conclude that absolute stability comes from the followin inequality
\[
    \abs{\frac{1-(1-\theta) \lambda_h^i \tstep}{1+\theta \lambda_h^i\tstep}} < 1
\]
that is 
\[
    -1-\theta \lambda_h^i \tstep < 1 - (1-\theta) \lambda_h^i \theta < 1+\theta \lambda_h^i \theta.
\]
Hence, 
\[
    -\frac{2}{\lambda_h^i \tstep} -\theta < \theta -1 < \theta.
\]
While the second inequality is always verified, we can rewrite the first one as: 
\[
    2\theta - 1 > -\frac{2}{\lambda_h^i \tstep}. 
\]
If \(\theta \geq \onehalf\), the left hand side is nonnegative, while the right hand side is negative, so the relation always holds. In the case \(\theta < \onehalf\), the inequality is satisfied if 
\begin{equation}
    \tstep < \frac{2}{(1-2\theta) \lambda_h^i}.
    \label{stability_forward_euler_pbl}
\end{equation}
So, we have that 
\begin{itemize}
    \item if \(\theta \geq \onehalf\), the \(\theta\)-method is unconditionally absolutely stable, for every value of \(\tstep\).
    \item if \(\theta < \onehalf\) the \(\theta\)-method is absolutely stable only if \eqref{stability_forward_euler_pbl} is satisfied.
\end{itemize}
Thanks to the definition of eigenvalue \eqref{eigenvalues_of_bilinear_pbl} and the continuity property of \(\bilinear\), we deduce that 
\[
    \lambda_h^{N_h} = \frac{a(w_{N_h}, w_{N_h})}{\ltwonorm{w_{N_h}}^2} \leq \frac{M\norm{w_{N_h}}_V^2}{\ltwonorm{w_{N_h}}^2} \leq M(1+C^2h^{-2}).
\]
The constant \(C > 0\), which appears in the latter step derives from the following inverse inequality 
\[
    \exists \; C > 0 : \ltwonorm{\grad v_h} \leq Ch^{-1} \ltwonorm{v_h} \quad \forall \; v_h \in V_h
\]
Hence, for \(h\) small enough, \(\lambda_h^{N_h} \leq Ch^{-2}\). In fact, we can prove that \(\lambda^{N_h}_h\) is indeed of the order \(h^-2\), that is \
\[
    \lambda_h^{N_h} = \max_i \lambda_h^i \simeq ch^{-2}. 
\]
Knowing that, we obtain the stability of the \(\theta\)-method for \(\theta < \onehalf\), which is 
\begin{equation}
    \tstep \leq C(\theta) h^2
    \label{forward_euler_stab_pbl}
\end{equation}
where \(C(\theta)\) denotes a positive constant depending on \(\theta\). We cannot choose \(\tstep\) without keeping in mind \(h\).

\subsection{Convergence analysis}
\begin{theorem}
    Under the hypothesis that \(u_0, f\) and the exact solution are sufficiently regular, the following a priori estimate holds \(\forall \; n \geq 1\): 
    \[
        \ltwonorm{u(t^n) - u_h^n}^2 + 2\alpha \tstep \sum_{k=1}^{n}\norm{u(t^k) - u^k_n}_V^2 \leq C(u_0, f, u) (\tstep^{p(\theta)} + h^{2r})
    \]
    where \(p(\theta) = 2\) if \(\theta \neq \frac{1}{2}\), \(p(\onehalf) = 4\) and \(C\) depends on its arguments, but not on \(\tstep\) or \(h\).
\end{theorem}
\subsection{Parabolic ADR equation}
Consider the parabolic PDE, where \(\Omega \subset \real^2\) is an open bounded domain
\begin{equation}
    \begin{cases}
       \displaystyle \partialderivative{u}{t} - \mu \Delta u + \beta \cdot \grad u + \sigma u = f & \txt{in }\Omega \times (0,T) \\
       u = 0 & \txt{on }\boundary \times (0,T) \\
       u(0) = u_0 & \txt{in } \Omega
   \end{cases}
   \label{parabolic_adr_pbl}
\end{equation}
where \(\mu , \beta, \sigma\) and \(f\) are regular functions, satisfying: 
\begin{align*}
    & 0 < \mu_0 \leq \mu \leq \mu_1 & \txt{a.e. in } \Omega \\
    \abs{\beta} \leq b_1 & \txt{a.e. in } \Omega \\
    0 < \sigma_0 \leq \sigma \leq \sigma_1 & \txt{a.e. in } \Omega
\end{align*}
Then, introducing a finite element space \(V_h \subset H^1_0(\Omega)\), the semi-discrete Galerkin formulation reads 
\begin{equation}
    \begin{split}
        \txt{for all } t \in (0, T] \txt{ find }u_h(t) \in V_h : \int_\Omega \partialderivative{u_h(t)}{t} v_h \, dx + \int_\Omega \mu \grad u_h (t) \cdot \grad v_h + \int_\Omega \beta \cdot \grad u_h(t) v_h \\
        + \int_\Omega \sigma u_h(t) v_h = \int_\Omega f v_h \quad \forall \; v_h \in V_h 
    \end{split}
    \label{semi_weak_form_parabolic_adr_pbl}
\end{equation}
And such that \(u_h(0) = u_{0h}\), where \(u_{0h}\) is the projection of the initial condition into \(V_h\).
\subsubsection{A semimplicit scheme}
Consider now a time-advancin scheme, where the diffusion and reaction term are treated implicitly, while the advection term is treated explicitedly. Let us denote \(t_k = k\Delta t\), for \(k = 0, \ldots, N\), where \(\tstep = \frac{T}{N}\). Let now \(u_h^{k}\) be the approximation of \(u(t_k)\). A fully discretized version of \eqref{semi_weak_form_parabolic_adr_pbl} reads:
\begin{equation}
    \begin{cases}
        \begin{aligned}
            \left( \dfrac{u_h^{k+1}- u^k_H}{\tstep}, v_h \right) &+ \left( \mu \grad u_h^{k+1}, v_h \right) + \left( \beta \cdot \grad u_h^{k} , v_h \right) \\
            &+ \left( \sigma u_h^{k+1}, v_h \right) = (f,v_h) \quad \forall \; v_h \in V_h, \ k = 0, \ldots, N-1
        \end{aligned} \\
        u_h(0) = u_{0h}
    \end{cases}
    \label{weak_form_parabolic_adr_pbl}
\end{equation}
\begin{theorem}
    If the coefficients of the problem staisfy 
    \begin{equation}
        b_1^2 < 4\mu_0 \sigma_0
        \label{30_slide_pbl}
    \end{equation}
    then the semimplicit scheme \eqref{weak_form_parabolic_adr_pbl} is absolutely stable for any chance of \(\tstep\). Consider now the case \(\sigma = 0\). If the coefficients of the problem satisfy 
    \begin{equation}
        b_1 < \frac{\mu_0}{C_p}
    \end{equation}
    with \(C_p\) being the Poincaré constant, then the scheme is absolutely stable, for any choice of \(\tstep\).
\end{theorem}
\begin{proof}
    Let us choose \(v_h = u_h^{k+1}\). We have 
    \begin{align*}
        & \left( \mu \grad u_h^{k+1}, \grad u_h^{k+1} \right) \geq \mu_0 \ltwonorm{\grad u_h^{k+1}}^2 \\
        & \left( \sigma u_h^{k+1}, u_h^{k+1} \right) \geq \sigma_0 \ltwonorm{u_h^{k+1}}^2
    \end{align*}
    which entails, for every \(k\)
    \begin{equation*}
        \begin{split}
            \ltwonorm{u_h^{k+1}}^2 + \tstep \mu_0 \ltwonorm{\grad u_h^{k+1}}^2 + \tstep \sigma_0 \ltwonorm{u_h^{k+1}}^2 \leq \abs{\left( u_h^{k}, u_h^{k+1} \right)} + \tstep \abs{\left( \beta \cdot \grad u_h^{k}, u_h^{k+1} \right)}
        \end{split}
    \end{equation*}
    The two right-hand side terms can be bounded by combining \eqref{young_inequality_pbl} and \eqref{CS_inequality_pbl}: 
    \begin{align*}
        & \abs{\left( u_h^{k}, u_h^{k+1} \right)} \leq \frac{1}{2 \eta_1} \ltwonorm{u_h^{k}}^2 + \frac{\eta_1}{2} \ltwonorm{u_h^{k+1}}^2 \\
        & \abs{\left( \beta \cdot \grad u_h^{k}, u_h^{k+1} \right)} \leq \frac{b_1}{2\eta_2} \ltwonorm{\grad u_h^{k}}^2 + \frac{\eta_2 b_1}{2} \ltwonorm{u_h^{k+1}}^2
    \end{align*}
    where the positive constant \(\eta_1\) and \(\eta_2\) will be later fixed accordingly. Now we end up with the following inequality
    \begin{equation*}
        \begin{split}
            \underbrace{\left[ 1 + \tstep \sigma_0 - \frac{\eta_1}{2} - \frac{\tstep \eta_2 b_1}{2} \right]}_{A} \ltwonorm{u_h^{k+1}}^2 + \underbrace{\tstep\mu_0}_{B} \ltwonorm{\grad u_h^{k+1}}^2 \\
            \leq \underbrace{\frac{1}{2\eta_1}}_{A'} \ltwonorm{u_h^{k}}^2 + \underbrace{\frac{\tstep b_1}{2 \eta_2}}_{B'} \ltwonorm{\grad u_h^{k}}^2 
        \end{split}
    \end{equation*}
    In order to prove stability we need \(A > A'\) and \(B > B'\). In fact, if this were true, we would end up with 
    \[
        A\ltwonorm{u_h^{k+1}}^2 + B\ltwonorm{\grad u_h^{k+1}}^2 \leq \max\left( \frac{A'}{A}, \frac{B'}{B} \right) \left[ A\ltwonorm{u_h^{k+1}}^2 + B \ltwonorm{u_h^{k+1}}^2 \right],
    \]
    which is the seeked stability in the norm \(\normdot_{A,B}:= \left( A\ltwonorm{\cdot}^2 + B\ltwonorm{\grad \cdot}^2 \right)^\onehalf\), equivalent to the standard \(V_h\) norm.

    Therefore, we look for a suitable choice of \(\eta_1\) and \(\eta_2\) that ensures \(A > A'\) and \(B > B'\). The second inequality is satisfied if and only if 
    \[
        \eta_2 = \frac{b_1 +\epsilon}{2\mu_0} 
    \]
    for some \(\epsilon > 0\).
    

    Hence, the first inequality reads 
    \[
        1 + \tstep \sigma_0 -\frac{\tstep b_1(b_1 + \epsilon)}{4\mu_0} > \frac{1}{2\eta_1} + \frac{\eta_1}{2}
    \]
    The rigth hand side is minimized for \(\eta_1 = 1\), thus leading to the condition 
    \begin{equation}
        4\frac{\mu_0\sigma_0}{b_1(b_1+\epsilon)} > 1
        \label{32_slide_pbl}
    \end{equation}
    Clearly, it is possible to find \(\epsilon > 0\) such that \eqref{32_slide_pbl} holds if and only if
    \begin{equation}
        b_1^2 \leq 4\mu_0\sigma_0
        \label{33_slide_pbl}
    \end{equation}
    In conclusion, whenever the coefficients of the problem satisfy the condition \eqref{33_slide_pbl}, the scheme \eqref{weak_form_parabolic_adr_pbl} is absolutely stable for any choice of \(\tstep\).

    Let us now consider \(\sigma= 0\). Proceeding as before, we have: 
    \begin{equation*}
        \begin{split}
            \left[ 1-\frac{\eta_1}{2} - \frac{\tstep \eta_2 b_1}{2} \right] \ltwonorm{u_h^{k+1}}^2 + \tstep \mu_0 \ltwonorm{\grad u_h^{k+1}}^2 \\ 
            \leq \frac{1}{2\eta_1} \ltwonorm{u_h^{k}}^2 + \frac{\tstep b_1}{2 \eta_2} \ltwonorm{\grad u_h^{k}}^2
        \end{split}
    \end{equation*}
    Introducing now a constant \(\omega \in (0,1)\), we have (thanks to \eqref{poincare_inequality_pbl}):
    \begin{equation*}
        \begin{aligned}
            \ltwonorm{\grad u_h^{k+1}}^2 &= (1-\omega) \ltwonorm{\grad u_h^{k+1}}^2 + \omega \ltwonorm{ \grad u_h^{k+1}}^2 \\
            &\geq \frac{1-\omega}{C_p} \ltwonorm{u_h^{k+1}}^2 + \omega \ltwonorm{\grad u_h^{k+1}}^2
        \end{aligned}
    \end{equation*}
    Combining the latter inequalities, we obtain 
    \begin{equation*}
        \begin{split}
            \underbrace{\left[ 1 - \frac{\eta_1}{2} - \frac{\tstep \eta_2 b_1}{2} - \frac{(1-\omega)\tstep \mu_0}{C_p^2}\right]}_{A} \ltwonorm{u_h^{k+1}}^2 + \underbrace{\omega \tstep \mu_0}_{B} \ltwonorm{\grad u_h^{k+1}}^2 \\
            \leq \underbrace{\frac{1}{2\eta_1}}_{A'} \ltwonorm{u_h^{k}}^2 + \underbrace{\frac{\tstep b_1}{2 \eta_2}}_{B'} \ltwonorm{\grad u_h^{k}}^2 
        \end{split}
    \end{equation*}
    As before, we look for conditions such that \(A >A'\) and \(B>B'\). The second inequality is satisfied if and only if
    \[
        \eta_2 =\frac{b_1 + \epsilon}{2\omega \mu_0}
    \]
    for some \(\epsilon > 0\).
    Then, the first inequality reads
    \[
        1 - \frac{\tstep b_1 (b_1 + \epsilon)}{4\omega \mu_0} + \frac{(1-\omega) \tstep \mu_0}{C^2_p} > \frac{1}{2\eta_1} + \frac{\eta_1}{2}
    \]
    The right hand side is minimized for \(\eta_1 = 1\). Rearranging the term, we get 
    \begin{equation}
        -\omega^2 + \omega -\frac{b_1(b_1 +\epsilon)C_p^2}{4\mu_0^2} > 0
        \label{34_slide_pbl}
    \end{equation}
    Real solutions \(\omega \in (0,1)\) exists whenever the discriminant is positive, that is 
    \[
        b_1(b_1+\epsilon)C_p^2 \leq \mu_0^2
    \]
    The latter condition can be satisfied (by suitable choosing of \(\epsilon\)) if and only if 
    \begin{equation}
        b_1 < \frac{\mu_0}{C_p}
        \label{35_slide_pbl}
    \end{equation}
    In conclusion, if \eqref{35_slide_pbl} is satisfied, the scheme is absolutely stable for any choice of \(\tstep\).
\end{proof}
\end{document}
