\documentclass[a4paper,12pt]{article}
\usepackage{amssymb}
\usepackage{amsmath}
\usepackage{hhline}
\usepackage{hyperref}
\usepackage{mathtools}
\usepackage{bm}
\usepackage[margin=2cm]{geometry}

\usepackage{amsthm}

\usepackage{tabularx}
\usepackage{graphicx}
\usepackage{physics}
\usepackage{textcomp}
\setlength\parindent{0pt}

\newlength\mylength
\setlength\mylength{0.1cm}
\newcolumntype{Y}{>{\Centering\arraybackslash}X}

\AtBeginEnvironment{array}{\everymath{\displaystyle}}
\newtheoremstyle{break}
  {\partopsep}{\topsep}%  
  {\normalfont}{}
  {\bfseries}{}%
  {\newline}{}%
  \theoremstyle{break}
\newtheorem{theorem}{Theorem}[section]
\newtheorem{corollary}{Corollary}[section]
\newtheorem{proposition}{Proposition}[section]
\newtheorem{remark}{Remark}[section]
\newtheorem{lemma}{Lemma}[section]
\renewcommand*{\proofname}{\textbf{Proof}}
\renewcommand\qedsymbol{$\bigstar$}
\renewcommand*{\grad}{\nabla}
\renewcommand*{\div}{\text{div}}
\newtheorem{definition}{Definition}[section]
\renewcommand\labelenumi{(\theenumi)}
\newcommand{\vect}[1]{\textbf{#1}}

\let\oldemptyset\emptyset
\let\emptyset\varnothing
\let\oldepsilon\epsilon
\let\epsilon\varepsilon
\let\oldphi\phi
\let\phi\varphi

\newcommand*{\txt}[1]{\text{#1}}

\newcommand{\ind}{\perp\!\!\!\!\perp} 
\newcommand{\measurespace}{(X, \mathcal{M}, \mu)}
\newcommand{\sigalg}{\sigma\mbox{-algebra}}
\newcommand{\boreal}{\mathcal{B}(\mathbb{R})}
\renewcommand{\real}{\mathbb{R}}
\renewcommand{\natural}{\mathbb{N}}
\newcommand{\barreal}{\overline{\mathbb{R}}}
\newcommand{\code}[1]{\texttt{#1}}
\newcommand{\xdownarrow}[1]{%
  {\left\downarrow\vbox to #1{}\right.\kern-\nulldelimiterspace}
}
\newcommand{\xuparrow}[1]{%
  {\left\uparrow\vbox to #1{}\right.\kern-\nulldelimiterspace}
}
\newcommand{\arrvline}{\hfil\kern\arraycolsep\vline\kern-\arraycolsep\hfilneg}
\newcommand{\esssup}{\text{ess}\, \text{sup}}
\newcommand{\normdot}{\norm{\cdot}}
\newcommand{\scalardot}{\langle \cdot,\cdot \rangle}
\newcommand{\scalarproduct}[2]{\langle #1,#2 \rangle}
\newcommand*{\limited}[1]{\vert_{#1}}
\newcommand{\interior}[1]{%
 {\kern0pt#1}^{\mathrm{o}}%
}
\def\stackbelow#1#2{\underset{\displaystyle\overset{\displaystyle\shortparallel}{#2}}{#1}}
\def\stackbelowlittle#1#2{\underset{\textstyle\overset{\textstyle\shortparallel}{#2}}{#1}}



\long\def\symbolfootnotemark[#1]#2{\begingroup%
\def\thefootnote{\fnsymbol{footnote}}\footnotetext[#1]{#2}\footnotemark[#1]\endgroup}

\long\def\symbolfootnotetext[#1]#2{\begingroup%
\def\thefootnote{\fnsymbol{footnote}}\footnotetext[#1]{#2}\endgroup}


\numberwithin{equation}{section}





\begin{document}
\title{Numerical Analysis for Partial Differential Equations}
\author{Andrea Bonifacio}
\date{\today}
\maketitle
\newpage
\section{Boundary Value Problems}
Let's consider a problem 
\begin{equation}
    \begin{cases}
        \mathcal{L}u = f & \text{in }\Omega \\
        + \text{B.C.} & \text{on }\partial\Omega
    \end{cases}
\end{equation}
\begin{itemize}
    \item \(\Omega\): open bounded domain in \(\real^d\), with \(d = 2,3\)
    \item \(\partial\Omega\): boundary of \(\Omega\)
    \item \(f\): given 
    \item B.C. accordingly to \(\mathcal{L}\)
    \item \(\mathcal{L}\): \(2^{\text{nd}}\) order operator, like:
    \begin{enumerate}
        \item \(\mathcal{L}u = -\div(\mu\grad u) +\vect{b}\cdot\grad u + \sigma u\) {\hspace*{\fill} (non-conservative form)}
        \item \(\mathcal{L}u = -\div(\mu \grad u) + \div(\vect{b}u) + \sigma u\) {\hspace*{\fill} (conservative form)}
    \end{enumerate}
    \begin{itemize}
        \item \(\mu \in L^\infty(\Omega), \quad \mu(\vect{x})\geq \mu_0 > 0\) {\hspace*{\fill} uniformly bounded from below}
        \item \(\vect{b} \in (L^\infty(\Omega))^d\) {\hspace*{\fill} transport term}
        \item \(\sigma \in L^2(\Omega)\) {\hspace*{\fill} reaction term}
        \item \(f \in L^2(\Omega)\) {\hspace*{\fill} can be less regular}
    \end{itemize}
\end{itemize}

%\subsection*{Example}
%Advection Diffusion Reaction Problem: 
%\begin{equation*}
%    \begin{cases}
%        Lu = \overbrace{-\div (\mu\grad u)}^{\txt{diffusion}} + \overbrace{\vect{b} \cdot \grad u}^{\txt{advection}} + \overbrace{\sigma u}^{\txt{reaction}} = f & \txt{in }\Omega \\
%        u = 0 & \txt{on }\Gamma_D\\
%        \mu \grad u \cdot \vect{n} = g & \txt{on }\Gamma_N 
%    \end{cases}
%    \qquad\qquad
%    \begin{array}{l}
%        g \in L^2(\Gamma_N) \\
%        \partial\Omega = \Gamma_D \cup \Gamma_N \\
%        \interior{\Gamma_D} \cap \interior{\Gamma_N} = \emptyset
%    \end{array}
%\end{equation*}
\subsection*{General elliptic problems}
Consider 
\begin{equation}
    \begin{cases}
        -\div(\mu \grad u)+ \vect{b}\cdot \grad u + \sigma u = f& \text{in } \Omega \\
        u = 0 & \txt{on }\Gamma_D \\
        \mu \grad u \cdot \vect{n} = g & \txt{on } \Gamma_N
    \end{cases}
    \qquad\qquad
    \begin{array}{l}
        g \in L^2(\Gamma_N) \\
        \partial\Omega = \Gamma_D \cup \Gamma_N \\
        \interior{\Gamma_D} \cap \interior{\Gamma_N} = \emptyset
    \end{array}
\end{equation}
Suppose that \(f \in L^2(\Omega)\) and \(\mu, \sigma \in L^\infty(\Omega)\). Also suppose that \(\exists \; \mu_0 > 0 \txt{ s.t. } \mu(\vect{x}) \geq \mu_0\), and \(\sigma(\vect{x}) \geq 0\) a.e. on \(\Omega\).  
Then, given a test function \(v\), we multiply the equation by \(v\), and integrate on the domain \(\Omega\)
\begin{equation*}
    \int_\Omega \left[-\div(\mu \grad u)+ \vect{b}\cdot \grad u + \sigma u\right] v \,  = \int_\Omega f v 
\end{equation*}
By applying Green's formula 
\begin{equation*}
    \underbrace{\int_\Omega \mu\grad u \cdot \grad v  + \int_\Omega \vect{b} \cdot \grad u v + \int_\Omega \sigma u v}_{=: a(u,v)} = \int_\Omega f v +\underbrace{\int_{\Gamma_D} \mu \grad u \cdot \vect{n} v}_{= 0 \txt{ if } v\vert_{\Gamma_D} = 0} + \int_{\Gamma_N} \underbrace{\mu \grad u \cdot \vect{n}}_{= g} v 
\end{equation*}
So the weak formulation of the problem is 
\begin{equation}
    \begin{cases}
        \txt{Find } u \in V & V = \left\{v \in H^1(\Omega), v\vert_{\Gamma_D} = 0\right\} =: H^1_{\Gamma_D}(\Omega)\\
        a(u,v) = \scalarproduct{F}{v} & \forall \; v \in V \label{Weak Formulation of Boundary Value Problems}
    \end{cases}
\end{equation}
where \(a: V \times V \to \real\) is a bilinear form and \(F:V \to \real\) is a linear form s.t. \(\scalarproduct{F}{v} \equiv F(v) = \int_\Omega fv + \int_{\Gamma_N} gv\).
\begin{theorem}[Lax-Milgram]
    Assume that 
    \begin{itemize}
        \item \(V\) Hilbert space with \(\normdot\) and inner product \((\cdot, \cdot)\)
        \item \(F \in V^*: \abs*{F(v)} \leq \norm*{F}_{V^*}\norm*{v} \; \forall \; v \in V\)
        \item \(a\) continuous: \(\exists \; M > 0: \abs{a(u,v)} \leq M \norm*{u}\norm*{v} \; \forall \; u,v \in V\)
        \item \(a\) coercive: \(\exists \; \alpha > 0: a(v,v) \geq \alpha\norm*{v}^2 \; \forall \; v \in V\)
    \end{itemize}
    Then, there exists a unique solution \(u\) of \ref*{Weak Formulation of Boundary Value Problems}
\end{theorem}
Moreover 
\[
    \alpha \norm*{u}^2 \leq a(u,u) = F(u) \leq \norm*{F}_{V^*} \norm*{u}
\]
where \(\alpha\) is the coercivity costant. Hence
\[
    \norm*{u} \leq \frac{\norm*{F}_{V^*}}{\alpha} \to \txt{stability/continuous dependence on data}
\]
But what if some of the assumptions of Lax-Milgram (in particular coercivity) are not satisfied?

We need a slightly more general problem to formulate Nečas theorem:
\begin{equation}
    \begin{cases}
        \txt{find } u \in V \\
        a(u,w) = \scalarproduct{F}{w} & \forall w \in W \label{Generalized weak formulation}
    \end{cases}
\end{equation}
They belong to different spaces: W for the test function, V the solutions
\begin{theorem}[Nečas]
    Assume that \(F \in W^*\). Consider the following conditions:
    \begin{itemize}
        \item \(a\) continuous: \(\exists \; M > 0: \abs*{a(u,w)} \leq M \norm*{u}_V \norm*{w}_W \; \forall \; u \in V, w \in W\)
        \item \(\inf-\sup\) condition: \(\exists \; \alpha > 0: \forall \; v \in V \quad \sup_{w \in W \setminus \left\{0\right\}} \frac{a(v,m)}{\norm*{w}_W} \geq \alpha \norm*{v}_V\)
        \item \(\forall \; w \in W, w \neq 0, \exists \; v \in V : a(v,w) \neq 0\)
    \end{itemize}
    These conditions are necessary and sufficient for the existence and uniqueness of a solution of \ref*{Generalized weak formulation}, for any \(F \in W^*\). Moreover 
    \[
        \norm*{u}_V \leq \frac{1}{\alpha}\norm*{F}_{W^*}
    \]
\end{theorem}
When \(W=V\) Lax-Milgram provides necessary and sufficient conditions for existence and uniqueness of solutions.

Going back to 
\begin{equation*}
    \begin{cases}
        \mathcal{L}u = f & \text{in }\Omega \\
        + \text{B.C.} & \text{on }\partial\Omega
    \end{cases}
\end{equation*}
What could be our choice of \(V\)? Given that
\[
    u\in V : a(u,v) = F(v) \quad \forall \; v \in V
\]
and 
\[
    a(u,v) = \int_\Omega \mu \underbrace{\grad u \grad v}_{\grad u, \grad v \in L^2} + \int_\Omega b \underbrace{\grad u v}_{\in L^1} + \int_\Omega \sigma \underbrace{u v}_{\in L^1}
\]
We want to choose \(v\) in order to have all of these integrable \[\Rightarrow V = \left\{v \in L^2(\Omega), \grad u \in \left[L^2(\Omega)\right]^d, v\vert_{\Gamma_D} = 0\right\} = V_{\Gamma_D}\].

Knowing that a Sobolev space 
\[
    H^1 = \left\{v \in L^2(\Omega), \grad u \in \left[L^2(\Omega)\right]^d\right\}
\]
we can say \(V_{\Gamma_D} = \left\{v \in H^1(\Omega): v\limited{\Gamma_D}= 0\right\}\), and if \(\Gamma_D = \partial\Omega\), then \(V_{\Gamma_D} = H^1_0\)
\subsection{Approximation}
Recall for a moment the weak formulation of a generic elliptic problem 
\begin{equation}
    \begin{cases}
        \txt{Find } u \in V \\
        a(u,v) = \scalarproduct{F}{v} & \forall \; v \in V \label{Weak Formulation of Boundary Value Problems - 2}
    \end{cases}
\end{equation}
with \(V\) being an appropriate Hilbert space, subset of \(H^1(), a(\cdot,\cdot)\) being a continuous and coercive bilinear forrm from \(V \times V \to \real\), \(F(\cdot)\) being a continuous linear functional from \(V \to \real\).

Let \(V_h \subset V\) be a family of spaces that depends on a parameter \(h > 0\), such that \(\dim V_h = N_h < \infty\).
We can rewrite the weak formulation 
\begin{equation}
    \begin{cases}
        \txt{Find } u_h \in V_h\\
        a(u_h,v_h) = \scalarproduct{F}{v_h} & \forall \; v_h \in V_h \label{Weak Formulation of Boundary Value Problems - Approx}
    \end{cases}
\end{equation}
and is called a \textbf{Galerkin problem}. Denoting with \(\left\{\phi_j, j = 1,2,\ldots,N_h\right\}\) a basis of \(V_h\), it is sufficient that the \eqref{Weak Formulation of Boundary Value Problems - Approx} is verified for each function of the basis. 
\end{document}

