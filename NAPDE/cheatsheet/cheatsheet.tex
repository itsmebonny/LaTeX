\documentclass[a4paper,11pt]{article}
\usepackage{amssymb}
\usepackage{amsmath}
\usepackage{hhline}
\usepackage{hyperref}
\usepackage{mathtools}
\usepackage{bm}
\usepackage[margin=2cm]{geometry}

\usepackage{amsthm}

\usepackage{tabularx}
\usepackage{graphicx}
\usepackage{physics}
\usepackage{textcomp}

\usepackage{mdframed}
\usepackage{xcolor}

\usepackage[makeroom]{cancel}
\usepackage{mathrsfs}

\usepackage[many]{tcolorbox}


\setlength\parindent{0pt}

\newlength\mylength
\setlength\mylength{0.1cm}
\newcolumntype{Y}{>{\Centering\arraybackslash}X}
\newtheoremstyle{break}
  {\partopsep}{\topsep}%  
  {\normalfont}{}
  {\bfseries}{}%
  {\newline}{}%
  \theoremstyle{break}
\newtheorem{theorem}{Theorem}[section]
\newtheorem{corollary}{Corollary}[section]
\newtheorem{proposition}{Proposition}[section]
\newtheorem{remark}{Remark}[section]
\newtheorem{lemma}{Lemma}[section]
\newtheorem{notation}{Notation}[section]
\newtheorem{definition}{Definition}[section]


\tcolorboxenvironment{theorem}{
  breakable,
  colback=purple!30,
  boxrule=0pt,
  boxsep=1pt,
  left=2pt,right=2pt,top=5pt,bottom=5pt,
  oversize=2pt,
  before skip=\topsep,
  after skip=\topsep,
}
\tcolorboxenvironment{corollary}{
  breakable,
  colback=blue!15!white,
  boxrule=0pt,
  boxsep=1pt,
  left=2pt,right=2pt,top=2pt,bottom=2pt,
  oversize=2pt,
  before skip=\topsep,
  after skip=\topsep,
}
\tcolorboxenvironment{proposition}{
  breakable,
  colback=blue!15!white,
  boxrule=0pt,
  boxsep=1pt,
  left=2pt,right=2pt,top=2pt,bottom=2pt,
  oversize=2pt,
  before skip=\topsep,
  after skip=\topsep,
}
\tcolorboxenvironment{lemma}{
  breakable,
  colback=pink,
  boxrule=0pt,
  boxsep=1pt,
  left=2pt,right=2pt,top=2pt,bottom=2pt,
  oversize=2pt,
  before skip=\topsep,
  after skip=\topsep,
}
\tcolorboxenvironment{remark}{
  breakable,
  colback=orange!15!white,
  boxrule=0pt,
  boxsep=1pt,
  left=2pt,right=2pt,top=2pt,bottom=2pt,
  oversize=2pt,
  before skip=\topsep,
  after skip=\topsep,
}
\tcolorboxenvironment{notation}{
  breakable,
  colback=green!150!black!50,
  boxrule=0pt,
  boxsep=1pt,
  left=2pt,right=2pt,top=2pt,bottom=2pt,
  oversize=2pt,
  before skip=\topsep,
  after skip=\topsep,
}
\tcolorboxenvironment{definition}{
  breakable,
  colback=yellow!70!,
  boxrule=0pt,
  boxsep=1pt,
  left=2pt,right=2pt,top=2pt,bottom=2pt,
  oversize=2pt,
  before skip=\topsep,
  after skip=\topsep,
}

\renewcommand*{\proofname}{\textbf{Proof}}
\renewcommand\qedsymbol{$\bigstar$}
\renewcommand*{\grad}{\nabla\mspace{1mu}}
\renewcommand*{\div}{\text{div}\mspace{1mu}}
\newcommand{\rot}{\text{rot}\mspace{1mu}}
\renewcommand\labelenumi{(\theenumi)}
\newcommand{\vect}[1]{\textbf{#1}}

\let\oldemptyset\emptyset
\let\emptyset\varnothing
\let\oldepsilon\epsilon
\let\epsilon\varepsilon
\let\oldphi\phi
\let\phi\varphi

\newcommand*{\txt}[1]{\text{#1}}

\newcommand{\ind}{\perp\!\!\!\!\perp} 
\newcommand{\measurespace}{(X, \mathcal{M}, \mu)}
\newcommand{\sigalg}{\sigma\mbox{-algebra}}
\newcommand{\boreal}{\mathcal{B}(\mathbb{R})}
\renewcommand{\real}{\mathbb{R}}
\renewcommand{\natural}{\mathbb{N}}
\newcommand{\barreal}{\overline{\mathbb{R}}}
\newcommand{\code}[1]{\texttt{#1}}
\newcommand{\xdownarrow}[1]{%
  {\left\downarrow\vbox to #1{}\right.\kern-\nulldelimiterspace}
}
\newcommand{\xuparrow}[1]{%
  {\left\uparrow\vbox to #1{}\right.\kern-\nulldelimiterspace}
}
\newcommand{\arrvline}{\hfil\kern\arraycolsep\vline\kern-\arraycolsep\hfilneg}
\newcommand{\esssup}{\text{ess}\, \text{sup}}
\newcommand{\normdot}{\norm{\cdot}}
\newcommand{\scalardot}{\langle \cdot,\cdot \rangle}
\newcommand{\scalarproduct}[2]{\langle #1,#2 \rangle}
\newcommand*{\limited}[1]{\vert_{#1}}
\newcommand{\interior}[1]{%
 {\kern0pt#1}^{\mathrm{o}}%
}
\newcommand*{\seminorm}[2]{\left\lvert#1\right\rvert_{#2}}
\newcommand*{\threenorm}[3][\null]{\left\lvert\mspace{-2mu}\left\lvert\mspace{-2mu}\left\lvert#2\right\rvert\mspace{-2mu}\right\rvert\mspace{-2mu}\right\rvert_{#3}^{#1}}

\newcommand{\ltwonorm}[1]{\norm{#1}_{L^2(\Omega)}}
\newcommand{\honenorm}[1]{\norm{#1}_{H^1(\Omega)}}

\newcommand{\jump}[1]{\left[\!\left[#1\right]\!\right]}
\newcommand{\average}[1]{\left\{\mspace{-6.5mu}\left\{#1\right\}\mspace{-6.5mu}\right\}}

\newcommand{\element}{\mathcal{K}}
\newcommand{\triangulation}{\mathcal{T}_h}

\newcommand{\boundary}{{\partial\Omega}}
\newcommand{\eit}{{\element\in\triangulation}}
\newcommand{\scalingfactor}{\tau_{\element}}
\newcommand{\peclet}{\mathbb{P}e}

\newcommand{\vecbabs}{\abs{\vect{b}}}
\newcommand{\loperator}{\mathcal{L}}
\newcommand{\xvec}{\vect{x}}
\newcommand{\tstep}{\Delta t}
\newcommand{\onehalf}{\frac{1}{2}}
\newcommand{\bilinear}{a(\cdot, \cdot)}
\newcommand{\uvec}{\vect{u}}
\newcommand{\normal}{\vect{n}}
\newcommand{\vvec}{\vect{v}}
\newcommand{\honed}{\left[ H^1(\Omega) \right]^d}
\newcommand{\find}{\txt{find }}

\newmdenv[backgroundcolor=blue!20]{background}


\def\stackbelow#1#2{\underset{\displaystyle\overset{\displaystyle\shortparallel}{#2}}{#1}}
\def\stackbelowlittle#1#2{\underset{\textstyle\overset{\textstyle\shortparallel}{#2}}{#1}}



\long\def\symbolfootnotemark[#1]#2{\begingroup%
\def\thefootnote{\fnsymbol{footnote}}\footnotetext[#1]{#2}\footnotemark[#1]\endgroup}

\long\def\symbolfootnotetext[#1]#2{\begingroup%
\def\thefootnote{\fnsymbol{footnote}}\footnotetext[#1]{#2}\endgroup}


\numberwithin{equation}{section}





\begin{document}
\title{Reminders for NAPDE}
\author{Andrea Bonifacio}
\date{\today}
\maketitle
\section*{Reminders on calculus}
\[
    \int_\Omega-\Delta u v = \int_\Omega \grad u \cdot \grad v - \underbrace{\int_{\Gamma_D}  \grad u \cdot \vect{n} v}_{= 0 \txt{ if } v\vert_{\Gamma_D} = 0}
\]
\[
    \int_\Omega \div u = \int_\boundary u \cdot \vect{n} 
\]
\section*{Weak Formulations}
\subsection*{Elliptic equations}
\begin{equation*}
    \begin{cases}
        -\div(\mu \grad u)+ \vect{b}\cdot \grad u + \sigma u = f& \text{in } \Omega \quad g \in L^2(\Gamma_N) \\
        u = 0 & \txt{on }\Gamma_D \quad \partial\Omega = \Gamma_D \cup \Gamma_N \\

        \mu \grad u \cdot \vect{n} = g & \txt{on } \Gamma_N \quad \interior{\Gamma_D} \cap \interior{\Gamma_N} = \emptyset
    \end{cases}
\end{equation*}
\[
    \Downarrow
\]
\begin{equation*}
    \underbrace{\int_\Omega \mu\grad u \cdot \grad v  + \int_\Omega \vect{b} \cdot \grad u v + \int_\Omega \sigma u v}_{=: a(u,v)} = \int_\Omega f v +\underbrace{\int_{\Gamma_D} \mu \grad u \cdot \vect{n} v}_{= 0 \txt{ if } v\vert_{\Gamma_D} = 0} + \int_{\Gamma_N} \underbrace{\mu \grad u \cdot \vect{n}}_{= g} v 
\end{equation*}
\[
    \Downarrow
\]
\begin{equation*}
    \begin{cases}
        \find u \in V & V = \left\{v \in H^1(\Omega), v\vert_{\Gamma_D} = 0\right\} =: H^1_{\Gamma_D}(\Omega)\\
        a(u,v) = \scalarproduct{F}{v} & \forall \; v \in V \label{Weak Formulation of Boundary Value Problems}
    \end{cases}
\end{equation*}
\subsection*{Parabolic equations}
\begin{equation*}
    \begin{cases}
        \partialderivative{u}{t} - \nu \partialderivative{^2u}{x^2} = f & 0 < x <d, t> 0 \\
        u(x,0) = u_0(x) & 0 < x < d \\
        u(0,t) = u(d,t) = 0 & t>0
    \end{cases}
    \label{one_dim_pbl}
\end{equation*}
\[
    \Downarrow
\]
\begin{equation*}
    \int_{\Omega} \partialderivative{u(t)}{t} v \, d\Omega + a(u(t), v) = \int_\Omega f(t)v \, d\Omega \quad \forall \; v \in V 
    \label{weak_formulation_heat_pbl}
\end{equation*}
\[
    \Downarrow
\]
for each \(t > 0\), we need to find \(u_h(t) \in V_h\) s.t. 
\begin{equation*}
    \int_\Omega \partialderivative{u_h(t)}{t} v_h \, d\Omega + a(u_h(t), v_h) = \int_\Omega f(t)v_h \, d\Omega \quad \forall \; v_h \in V_h
\label{semi-discretization_pbl}
\end{equation*}
\section*{Code implementation}
\subsection*{CG-FEM}
\begin{itemize}
    \item Matrix \(A\);
    \[
        A_{ij} = \int_\Omega \grad \phi_j \grad \phi_i  
    \]
    Loop on all the elements and compute locally (elements with \(\hat{\cdot}\) are computed on the reference element):
    \[
        A_{loc_{ij}} = \det(\vect{B}_\element) \int_{\hat{\element}} \hat{\grad} \hat{\phi}^T_j \vect{B}_\element^{-1} \vect{B}^{-1}_\element \hat{\grad} \hat{\phi}_i = \frac{\det(\vect{B})}{2} \hat{\grad} \hat{\phi}^T_j \vect{B}_\element^{-1} \vect{B}^{-T}_\element \hat{\grad} \hat{\phi}_i
    \]
    Can be implemented as 
    \begin{verbatim}
        function [K_loc]=C_lap_loc(Grad,w_2D,nln,BJ)
        K_loc=zeros(nln,nln);
        for i=1:nln
            for j=1:nln
                for k=1:length(w_2D)
                    Binv = inv(BJ(:,:,k));   % inverse
                    Jdet = det(BJ(:,:,k));   % determinant 
                    K_loc(i,j) = K_loc(i,j) + (Jdet.*w_2D(k)) .* ( (Grad(k,:,i)
                                 * Binv) * (Grad(k,:,j) * Binv )');
                end
            end
        end
    \end{verbatim}
    \item Mass matrix \(M\):
    \[
        M_{ij} = \int_\Omega \phi_j, \phi_i
    \]
    Loop on all the elements and calculate the local mass matrix
    \[
        M_{loc_{ij}} =  \det(\vect{B}_\element) \int_{\hat{\element}} \hat{\phi}^T_j \vect{B}_\element^{-1} \vect{B}^{-1}_\element \hat{\phi}_i
    \]
    Can be implemented as
    \begin{verbatim}
        function [M_loc]=C_mass_loc(dphiq,w_2D,nln,BJ)
        M_loc=zeros(nln,nln);
        for i=1:nln
            for j=1:nln
                for k=1:length(w_2D)
                    Binv = inv(BJ(:,:,k));      % inverse
                    Jdet = det(BJ(:,:,k));      % determinant 
                    M_loc(i,j) = M_loc(i,j) + (Jdet.*w_2D(k))
                                 .* dphiq(1,k,i).* dphiq(1,k,j);
                end
            end
        end
    \end{verbatim}
    \item Transport matrix \(T\)
    
    Can be implemented as 
    \begin{verbatim}
        function [ADV_loc]=C_adv_loc(Grad,dphiq,beta,w_2D,nln,BJ)
        ADV_loc=sparse(nln,nln);
        for i=1:nln
            for j=1:nln
                for k=1:length(w_2D)
                    Binv=inv(BJ(:,:,k));    % inverse
                    Jdet=det(BJ(:,:,k));    % determinant 
                    ADV_loc(i,j) = ADV_loc(i,j)+(Jdet.*w_2D(k)).* dphiq(1,k,i)
                                   *( (beta)*(Grad(k,:,j) * Binv )');
                end
            end
        end
    \end{verbatim}
    \item Right hand side \(\vect{b}\):
    \[
        b_i = \int_\Omega f\phi_i
    \]
    which is computed 
    \begin{verbatim}
        function [f]=C_loc_rhs2D(force,dphiq,BJ,w_2D,pphys_2D,nln,mu)
        f = zeros(nln,1);
        x = pphys_2D(:,1);
        y = pphys_2D(:,2);
        F = eval(force);
        for s = 1:nln
            for k = 1:length(w_2D)
                Jdet = det(BJ(:,:,k));  % determinant 
                f(s) = f(s) + w_2D(k)*Jdet*F(k)*dphiq(1,k,s);
            end    
        end
    \end{verbatim}
\end{itemize}
\end{document}
