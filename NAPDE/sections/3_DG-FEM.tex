\section{Discontinuous Galerkin methods}
The idea behind DG methods is to seek the soltion in a discrete space made of polynomials that are completely discontinuous across the elemnts of the mesh.
\[
    V_h \subsetneq V
\]
\subsection{1D case}
Let us consider a Poisson problem 
\[
    \begin{cases}
        -u'' = f & a < x < b \\
        u(a) = u(b) = 0
    \end{cases}
\]
The aim is to use discontinuous piecewise polynomials, so that between every interval \(I_k\) from one node to another we obtain
\[
    \int_{a}^b -u''v = \int_a^b fv \Rightarrow - \sum_{k=0}^{N-1} \int_{I_k} u''v = \sum_{k=0}^{N-1} \int_{I_k} fv
\]
We must know integrate by parts, but our test functions are discontinuous at the nodes, so we must acknowledge it. Let's call \(x_k^-\) and \(x_k^+\) the left and right side of the \(x_k\) node. Then we can:
\begin{equation}
    -\sum_{k=0}^{N-1} \int_{I_k} u''v = \sum_{k=0}^{N-1} \left[ \int_{I_k}u'v' - \left( u'v\vert_{x_{k+1}^-} - u'v\vert_{x_k^+} \right) \right] \label{integration_by_parts_dg}
\end{equation}
    \begin{align}
    \sum_{k=0}^{N-1} (u'v\vert_{x_{k+1}^-} - u'v\vert_{x_k^+}) &= u'(x_1^-)v(x_1^-) - u'(x_0^+)v(x_0^+) \nonumber\\
    & + u'(x_2^-)v(x_2^-) - u'(x_1^+)v(x_1^+) \nonumber\\
    & + \ldots \\
    & + u'(x_N^-)v(x_N^-) - u'(x_{N-1}^+)v(x_{N-1}^+) \nonumber\\
    & = \sum_{k=0}^{N} \jump{u'(x_k)v(x_k)} \nonumber \label{step_for_jump_dg}
\end{align}
where we have defined the jump function
\begin{align}
    \jump{\phi(x_0)} &:= -\phi(x^+_0) & \nonumber\\
    \jump{\phi(x_k)} &:= \phi(x_k^-) -\phi(x^+_k) & x_k :\txt{ interior node} \\
    \jump{\phi(x_{\!N})} &:= \phi(x^-_N) &\nonumber \label{jump_function}
\end{align}
By using \eqref{integration_by_parts_dg} and \eqref{step_for_jump_dg} we obtain 
\begin{equation}
    \sum_{k=0}^{N-1} \int_{I_k} u'v' - \sum_{k=0}^{N} \jump{u'(x_k)v(x_k)} = \sum_{k=0}^{N-1} \int_{I_k} fv \label{almost_bilinear_dg}
\end{equation}
Now define the average operator 
\begin{align}
    \average{\phi(x_0)} &:= \phi(x^+_0) & \nonumber\\
    \average{\phi(x_k)} &:= \frac{1}{2}\phi(x_k^-) +\phi(x^+_k) & x_k : \txt{ interior node} \\
    \average{\phi(x_N)} &:= \phi(x^-_N) &\nonumber \label{average_function}
\end{align}
This way we obtain this formula 
\begin{equation}
    \sum_{k=0}^{N} \jump{u'(x_k)v(x_k)} = \sum_{k=0}^{N} \average{u'(x_k)} \jump{v(x_k)} + \sum_{k=1}^{N-1} \jump{u'(x_k)}\average{v(x_k)}\label{magic_formula_dg}
\end{equation}
If \(u\) is the exact solution and \(u \in \mathcal{C}^1([a,b])\), then \(\jump{u'(x_k)} = 0\) for every interior node, and the second sum in \eqref{magic_formula_dg} drops.

We end up with the formulation (by collecting \eqref{almost_bilinear_dg} and \eqref{magic_formula_dg})
\begin{equation}
    \begin{split}
        \underbrace{\sum_{k=0}^{N-1}\int_{I_k} u'v' - \sum_{k=0}^{N} \average{u'(x_k)}\jump{v(x_k)} - \sum_{k=1}^{N-1} \jump{u'(x_k)}\average{v(x_k)}}_{\mathcal{A}(u,v)} \\
        = \sum_{k=0}^{N-1}\int_{I_k} fv \quad \forall \; v \in V \label{dg_formulation}
    \end{split}
\end{equation}
where 
\[
    V = H^1_{\txt{broken}}(\Omega) := \left\{ v \in L^2(\Omega) : v\vert_{I_k} \in H^1{I_k} \ \forall \; k = 0,\ldots, N-1 \right\}
\]
with the broken norm 
\[
    \norm{v}_{H^1_{\txt{broken}}(\Omega)} = \left( \sum_{k=0}^{N} \norm*{v\vert_{I_k}}_{H^1(\Omega)}^2 \right)^{\frac{1}{2}}
\]
Let now \(V_h \subset V\)
\begin{equation}
    \txt{find } u_h \in V_h : \mathcal{A}(u_h, v_h) = \sum_{k=0}^{N-1} \int_{I_k} fv_h \quad \forall \; v_h \in V_h \label{weak_not_well_posed_formulation_dg}
\end{equation}
\begin{remark}
    \(V_h\) is not a subspace of \(H^1(\Omega)\)
\end{remark}
But \eqref{weak_not_well_posed_formulation_dg} is not well posed, so the \eqref{dg_formulation} must be modified such that:
\begin{itemize}
    \item drop \(3^{rd}\) term because \(\jump{u'(x_k)} = 0\)
    \item add symmetrization term (\(=0\) if \(u\) is the exact solution) 
    \[
        -\sum_{k=0}^{N} \theta \average{v'(x_k)}\jump{u(x_k)}
    \]
    with
    \begin{itemize}
        \item \(\theta = 1\) SIP (Symmetric Interior Penalty)
        \item \(\theta = -1\) NIP (Non-symmetric Interior Penalty)
        \item \(\theta = 0\) IIP (Incomplete Interior Penalty)
    \end{itemize}
    \item add the stabilization term (\(=0\) if \(u\) is the exact solution)
    \[
        +\sum_{k=0}^{N} \gamma \jump{u(x_k)}\jump{v(x_k)}
    \]
\end{itemize}