\newpage
\section{Discontinuous Galerkin methods}
The idea behind DG methods is to seek the soltion in a discrete space made of polynomials that are completely discontinuous across the elements of the mesh.
\[
    V_h \subsetneq V
\]
\subsection{1D case}
Let us consider a Poisson problem 
\[
    \begin{cases}
        -u'' = f & a < x < b \\
        u(a) = u(b) = 0
    \end{cases}
\]
The aim is to use discontinuous piecewise polynomials, so that between every interval \(I_k\) from one node to another we obtain
\[
    \int_{a}^b -u''v = \int_a^b fv \Rightarrow - \sum_{k=0}^{N-1} \int_{I_k} u''v = \sum_{k=0}^{N-1} \int_{I_k} fv
\]
We must know integrate by parts, but our test functions are discontinuous at the nodes, so we must acknowledge it. Let's call \(x_k^-\) and \(x_k^+\) the left and right side of the \(x_k\) node. Then we can:
\begin{equation}
    -\sum_{k=0}^{N-1} \int_{I_k} u''v = \sum_{k=0}^{N-1} \left[ \int_{I_k}u'v' - \left( u'v\vert_{x_{k+1}^-} - u'v\vert_{x_k^+} \right) \right] \label{integration_by_parts_dg}
\end{equation}
    \begin{align}
    \sum_{k=0}^{N-1} (u'v\vert_{x_{k+1}^-} - u'v\vert_{x_k^+}) &= u'(x_1^-)v(x_1^-) - u'(x_0^+)v(x_0^+) \nonumber\\
    & + u'(x_2^-)v(x_2^-) - u'(x_1^+)v(x_1^+) \nonumber\\
    & + \ldots \\
    & + u'(x_N^-)v(x_N^-) - u'(x_{N-1}^+)v(x_{N-1}^+) \nonumber\\
    & = \sum_{k=0}^{N} \jump{u'(x_k)v(x_k)} \nonumber \label{step_for_jump_dg}
\end{align}
where we have defined the jump function
\begin{align}
    \jump{\phi(x_0)} &:= -\phi(x^+_0) & \nonumber\\
    \jump{\phi(x_k)} &:= \phi(x_k^-) -\phi(x^+_k) & x_k :\txt{ interior node} \\
    \jump{\phi(x_{\!N})} &:= \phi(x^-_N) &\nonumber \label{jump_function}
\end{align}
By using \eqref{integration_by_parts_dg} and \eqref{step_for_jump_dg} we obtain 
\begin{equation}
    \sum_{k=0}^{N-1} \int_{I_k} u'v' - \sum_{k=0}^{N} \jump{u'(x_k)v(x_k)} = \sum_{k=0}^{N-1} \int_{I_k} fv \label{almost_bilinear_dg}
\end{equation}
Now define the average operator 
\begin{align}
    \average{\phi(x_0)} &:= \phi(x^+_0) & \nonumber\\
    \average{\phi(x_k)} &:= \frac{1}{2}\phi(x_k^-) +\phi(x^+_k) & x_k : \txt{ interior node} \\
    \average{\phi(x_N)} &:= \phi(x^-_N) &\nonumber \label{average_function}
\end{align}
This way we obtain this formula (magic formula)
\begin{equation}
    \sum_{k=0}^{N} \jump{u'(x_k)v(x_k)} = \sum_{k=0}^{N} \average{u'(x_k)} \jump{v(x_k)} + \sum_{k=1}^{N-1} \jump{u'(x_k)}\average{v(x_k)}\label{magic_formula_dg}
\end{equation}
If \(u\) is the exact solution and \(u \in \mathcal{C}^1([a,b])\), then \(\jump{u'(x_k)} = 0\) for every interior node, and the second sum in \eqref{magic_formula_dg} drops.

We end up with the formulation (by collecting \eqref{almost_bilinear_dg} and \eqref{magic_formula_dg})
\begin{equation}
    \begin{split}
        \underbrace{\sum_{k=0}^{N-1}\int_{I_k} u'v' - \sum_{k=0}^{N} \average{u'(x_k)}\jump{v(x_k)} - \sum_{k=1}^{N-1} \jump{u'(x_k)}\average{v(x_k)}}_{\mathcal{A}(u,v)} \\
        = \sum_{k=0}^{N-1}\int_{I_k} fv \quad \forall \; v \in V \label{dg_formulation}
    \end{split}
\end{equation}
where 
\[
    V = H^1_{\txt{broken}}(\Omega) := \left\{ v \in L^2(\Omega) : v\vert_{I_k} \in H^1{I_k} \ \forall \; k = 0,\ldots, N-1 \right\}
\]
with the broken norm 
\[
    \norm{v}_{H^1_{\txt{broken}}(\Omega)} = \left( \sum_{k=0}^{N} \norm{v\vert_{I_k}}_{H^1(\Omega)}^2 \right)^{\frac{1}{2}}
\]
Let now \(V_h \subset V\)
\begin{equation}
    \find u_h \in V_h : \mathcal{A}(u_h, v_h) = \sum_{k=0}^{N-1} \int_{I_k} fv_h \quad \forall \; v_h \in V_h \label{weak_not_well_posed_formulation_dg}
\end{equation}
\begin{remark}
    \(V_h\) is not a subspace of \(H^1(\Omega)\)
\end{remark}
But \eqref{weak_not_well_posed_formulation_dg} is not well posed, so the \eqref{dg_formulation} must be modified such that:
\begin{itemize}
    \item drop \(3^{rd}\) term because \(\jump{u'(x_k)} = 0\)
    \item add symmetrization term (\(=0\) if \(u\) is the exact solution) 
    \[
        -\sum_{k=0}^{N} \theta \average{v'(x_k)}\jump{u(x_k)}
    \]
    with
    \begin{itemize}
        \item[\qedhere] \(\theta = 1\) SIP (Symmetric Interior Penalty)
        \item[\qedhere] \(\theta = -1\) NIP (Non-symmetric Interior Penalty)
        \item[\qedhere] \(\theta = 0\) IIP (Incomplete Interior Penalty)
    \end{itemize}
    \item add the stabilization term (\(=0\) if \(u\) is the exact solution)
    \[
        +\sum_{k=0}^{N} \gamma \jump{u(x_k)}\jump{v(x_k)}
    \]
\end{itemize}
We can now obtain a new bilinear form
\begin{equation}
    \begin{split}
        \mathcal{A}^*(u_h, v_h) = \underbrace{\sum_{k=0}^{N-1} \int_{I_k} u'_h v'_h}_{(1)} \underbrace{-\sum_{k=0}^{N} \average{u'_h(x_k)} \jump{v_h(x_k)}}_{(2)} \\
        \underbrace{-\sum_{k=0}^{N} \theta \average{v'_h(x_k)} \jump{u_h(x_k)}}_{(3)} \underbrace{+\sum_{k=0}^{N} \gamma \jump{u_h(x_k)}\jump{v_h(x_k)}}_{(4)} \label{new_bilinear_form_dg} 
    \end{split}
\end{equation}
\subsubsection*{Neumann BC}
Impose Neumann BC through \(\average{u'(x_k)}\) in \((2)\). In this case we have \(\sum_{k=1}^{N-1}\) in \((2)\) and, consequently, we write \(\sum_{k=1}^{N-1}\) in \((3)\) for symmetry.
\subsubsection*{Non-homogeneus Dirichlet BC}
Impose Dirichlet BC as follows. In \((3)\) and \((4)\) replace \(\jump{u_h(x_0)}\) and \(\jump{u_h(x_N)}\) with the following definition:
\begin{align*}
    &\jump{u_h(x_0)} := \alpha - u_h(x_0^+) & \txt{if } u(a) = \alpha \\
    &\jump{u_h(x_N)} :=  u_h(x_N^-) - \beta & \txt{if } u(b) = \beta
\end{align*}
In case \(\alpha =  \beta = 0\) we have homogeneus Dirichlet. 

Now in \eqref{new_bilinear_form_dg} split sums as follows: 
\begin{equation}
    \begin{split}        
    & \mathcal{A}^*(u_h, v_h)\\
    & = \sum_{k=0}^{N-1} \int_{I_k} u'_h v'_h \\
        & - \sum_{k=1}^{N-1} \average{u'_h(x_k)}\jump{v_h(x_k)} + u'_h(x_0^+)v_h(x_0^+) - u'_h(x_N^-)v_h(x_N^-) \\
        & - \sum_{k=1}^{N-1} \theta \average{v'_h(x_k)} \jump{u_h(x_k)} - \left[\theta v'_h(x^+_0)(\alpha - u_h(x^+_0)) + \theta v'_h(x^-_N)(u_h(x^-_N) - \beta)\right] \\
        & - \sum_{k=1}^{N-1} \gamma \jump{u_h(x_k)} \jump{v_h(x_k)} + \gamma(\alpha - u_h(x^+_0))(-v_h(x^+_0)) + \gamma (u_h(x_N^-) - \beta)v_h(x_N^-) \label{split_sum_bilinear_dg}
    \end{split}
\end{equation}
Now move terms, including \(\alpha\) and \(\beta\) to the right hand side of the formulation.

On the left hand side it remains 
\begin{equation}
    \begin{split}
        & \tilde{\mathcal{A}} (u_h, v_h) \\
        & = \sum_{k=0}^{N-1} \int_{I_k} u'_hv'_h \\
        & - \sum_{k=1}^{N-1} \average{u'_h(x_k)}\jump{v_h(x_k)} + u'_h(x^+_0)v_h(x^+_0) - u'_h(x_N^-)v_h(x_N^-) \\
        & - \sum_{k=1}^{N-1} \theta \average{v'_h(x_k)} \jump{u_h(x_k)} - \left[\theta v'_h(x^+_0)u_h(x^+_0) + \theta v'_h(x^-_N)u_h(x^-_N)\right] \\
        & - \sum_{k=1}^{N-1} \gamma \jump{u_h(x_k)} \jump{v_h(x_k)} + \gamma u_h(x^+_0)v_h(x^+_0) + \gamma u_h(x_N^-)v_h(x_N^-) \label{split_sum_noalpha_bilinear_dg}
    \end{split}
\end{equation}
On the right hand side instead
\begin{equation}
    \mathcal{F}(v_h) = \sum_{k=0}^{N-1} \int_{I_k} f v_h + \theta(\alpha v'_h(x^+_0) - \beta v'_h(x_N^-)) + \gamma(\alpha v_h(x^+_0) + \beta v_h(x^-_N))
\end{equation}
\begin{remark}
    Note that for \(\theta = 1\), \(\tilde{\mathcal{A}}(u_h, v_h) = \tilde{\mathcal{A}}(v_h, u_h)\), so it's symmetric.
\end{remark}
\subsubsection*{Non-homogeneus Dirichlet conditions}
\begin{equation}
    \find u_h \in V_h : \tilde{\mathcal{A}}(u_h, v_h) = \mathcal{F}(v_h) \quad \forall \; v_h \in V_h
    \label{functional_dg}
\end{equation}
with \(\mathcal{F}\) depending on \(f\), \(\alpha\) and \(\beta\).

Note that in \eqref{split_sum_noalpha_bilinear_dg}, if we define \(\jump{u_h(x_0)}\) and \(\jump{u_h(x_N)}\) as \(\jump{v_h(x_0)}\) and \(\jump{v_h(x_N)}\)
\begin{equation*}
    \begin{split}
        -\sum_{k=1}^{N-1} \average{u'_h(x_k)} \jump{v_h(x_k)} + u'_h(x_0^+)v_h(x^+_0) - u'_h(x_N^-)v_h(x_N^-) \\
        = -\sum_{k=0}^{N} \average{u'_h(x_k)} \jump{v_h(x_h)} \\
         -\sum_{k=1}^{N-1} \theta \average{v'_h(x_k)} \jump{u_h(x_k)} + (\theta u_h(x_0^+)v'_h(x_0^+) - \theta u_h(x_N^-)v'_h(x_N^-)) \\
        = - \sum_{k=0}^{N} \theta \average{v'_h(x_k)}\jump{u_h(x_k)} \\
        + \sum_{k=1}^{N-1} \gamma \jump{u_h(x_k)} \jump{v_h(x_k)} + \gamma u_h(x_0^+)v_h(x_0^+) + \gamma u_h(x_N^-)v_h(x_N^-) \\
        + \sum_{k=0}^{N} \gamma \jump{u_h(x_k)}\jump{v_h(x_k)}
    \end{split}
\end{equation*}
\subsection{Multidimensional case}
We can take our Poisson problem in multidimension 
\begin{equation}
    \begin{cases}
        -\Delta u = f & \txt{in } \Omega \\
        u = 0 & \txt{on }\partial\Omega
    \end{cases}\label{Poisson_multidim_dg}
\end{equation}
with the triangulation \(\mathcal{T}_h\), but this time we cannot assume that the conformity constraint is present. 

So we need to take a test function \(v\) (element-wise smooth), and integrate over an element \(\mathcal{K} \in \mathcal{T}_h\)
\[
    \int_{\mathcal{K}} -\Delta u v = \int_{\mathcal{K}} fv
\]
As usual, integrate by parts, and sum over all the elements \(\mathcal{K} \in \triangulation\) 
\[
    \sum_{\element \in \triangulation} \int_{\element}\grad u \cdot \grad v - \sum_{\element \in \triangulation} \int_{\partial\element} \grad u \cdot \vect{n}_{\element} v = \int_{\Omega} fv
\]
since for any \(F \in \mathcal{F}'_h\) which is the set of interior faces shared by two elements \(\element^+\) and \(\element^-\)
\begin{align*}
    &\average{v} = \frac{(v^+ + v^-)}{2}  & \jump{v} = v^+ \vect{n}^+ + v^- \vect{n}^-\\
    &\average{\bm{\tau}} = \frac{(\bm{\tau}^+ + \bm{\tau}^-)}{2} & \jump{\bm{\tau}} = \bm{\tau}^+ \vect{n}^+ + \bm{\tau}^- \vect{n}^-
\end{align*}
while, for the set of boundary faces \(F \in \mathcal{F}^B_h\)
\begin{align*}
    &\average{v} = v & \jump{v} = v \vect{n} \\
    &\average{\bm{\tau}} = \bm{\tau} & \jump{\bm{\tau}} = \bm{\tau} \cdot \vect{n}
\end{align*}
in this way we can obtain the following formula \(\forall \;\bm{\tau} \txt{ vector function: }
\)
\begin{equation}
    \sum_{\element \in \triangulation} \int_{\partial\element} \bm{\tau} \cdot \vect{n}_{\element} v = \sum_{F \in \mathcal{F}_h}\int_F\average{\bm{\tau}} \cdot \jump{v} + \sum_{F \in \mathcal{F}'_h} \int_{F}\jump{\bm{\tau}}\average{v} \label{magic_formula_multidim_dg}
\end{equation}
and thanks to that we obtain 
\[
    -\sum_{\element \in \triangulation} \int_{\partial\element} \bm{\tau} \cdot \vect{n}_{\element} v =- \sum_{F \in \mathcal{F}_h}\int_F\average{\grad u} \cdot \jump{v} - \sum_{F \in \mathcal{F}'_h} \int_{F}\jump{\grad u}\average{v} 
\]
then 
\[
    \sum_{\element \in \triangulation} \int_{\element}\grad u \cdot \grad v - \sum_{\element \in \triangulation} \int_{\partial\element} \grad u \cdot \vect{n}_{\element} v = \int_{\Omega} fv
\]
so it becomes 
\[
    \sum_{\element \in \triangulation} \int_{\element} \grad u \grad v - \sum_{F \in \mathcal{F}_h} \int_{F} \average{\grad u} \cdot \jump{v} - \sum_{F \in \mathcal{F}'_h} \int_F \jump{\grad u} \average{v} = \int_{\Omega} fv
\]
but, if we assume \(u \in H^2(\Omega)\), then \(\jump{\grad u} = 0 \ \forall \; F \in \mathcal{F}'_h\). This regularity assumption is fullfilled if \(f \in L^2\) and the domain is a convex polygon, thanks to the property of elliptic regularity.
\[
    \sum_{\element \in \triangulation} \int_{\element} \grad u \grad v - \sum_{F \in \mathcal{F}_h} \int_{F} \average{\grad u} \cdot \jump{v} \cancel{- \sum_{F \in \mathcal{F}'_h} \int_F \jump{\grad u} \average{v}} = \int_{\Omega} fv
\]
Now we can assume that \(\jump{u} = 0 \ \forall \; F \in \mathcal{F}_h\) (since \(u \in H^2(\Omega) \cap H^1_0(\Omega)\)) to add a symmetry term 
\[
    \sum_{\element \in \triangulation} \int_\element \grad u \cdot \grad v - \sum_{F \in \mathcal{F}_h} \int_F \average{\grad u} \cdot \jump{v} - \sum_{F \in \mathcal{F}'_h} \int_F \average{\grad_h v} \jump{u} = \int_\Omega fv
\] 
where \(\grad_h\) is the elementwise gradient (\(v\) is only piecewise smooth).
We also add a stabilization term that controls the jumps
\[
    \sum_{\element \in \triangulation} \int_\element \grad u \cdot \grad v  - \sum_{F \in \mathcal{F}_h} \int_F \average{\grad u} \cdot \jump{v} - \sum_{F \in \mathcal{F}_h} \int_F \jump{u} \cdot \average{\grad_h v} + \sum_{F \in \mathcal{F}_h} \int_F \gamma \jump{u} \cdot \jump{v} = \int_\Omega fv
\]
where \(\gamma\) is a stabilization function.

Now we can define the DG discrete space 
\[
    V_h^p = \left\{ v_h \in L^2(\Omega) : v_h\vert_\element \in \mathcal{P}^{p_\element}(\element) \ \forall \; \element \in \triangulation\right\} \not\subseteq H^1_0 (\Omega)
\]
Discretize \(u \leadsto u_h, v \leadsto v_h\) and obtain the following weak formulation
\[
    \find u_h \in V_h^p \txt{ s.t. } \mathcal{A}(u_h, v_h) = \int_\Omega fv \quad \forall \;v_h \in V_h^p
\]
where 
\begin{equation*}
    \begin{split}
        \mathcal{A}(u, v) =  \sum_{\element \in \triangulation} \int_\element \grad u \cdot \grad v  - \sum_{F \in \mathcal{F}_h} \int_F \average{\grad u} \cdot \jump{v}  - \sum_{F \in \mathcal{F}_h} \int_F \jump{u} \cdot \average{\grad_h v} \\+ \sum_{F \in \mathcal{F}_h} \int_F \gamma \jump{u} \cdot \jump{v}
    \end{split}
\end{equation*}
\subsubsection*{Interior Penalty DG methods}
\[
    \find u_h \in V_h^p \txt{ s.t. } \mathcal{A}(u_h, v_h) = \int_\Omega fv \quad \forall \;v_h \in V_h^p
\]
Note that \(\mathcal{A}\) depends on the triangulation and it differs from the original weak formulation in the infinite dimension problem.
\begin{equation*}
    \begin{split}
        \mathcal{A}(u, v) =  \sum_{\element \in \triangulation} \int_\element \grad u \cdot \grad v  - \sum_{F \in \mathcal{F}_h} \int_F \average{\grad u} \cdot \jump{v}  - \theta \sum_{F \in \mathcal{F}_h} \int_F \jump{u} \cdot \average{\grad_h v} \\+ \sum_{F \in \mathcal{F}_h} \int_F \gamma \jump{u} \cdot \jump{v}
    \end{split}
\end{equation*}
where 
\begin{itemize}
    \item \(\theta =  1\) Symmetric Interior Penalty (SIP)
    \item \(\theta =  -1\) Non-symmetric Interior Penalty (NIP)
    \item \(\theta =  0\) Incomplete Interior Penalty (IIP)
\end{itemize}
\subsubsection*{Dirichlet BC}
The above formulation holds when applying homogeneus Dirichlet BC, but in the case of non-homogeneus BC, such as 
\[
    u = g_D \quad \txt{ on } \partial\Omega
\]
the right hand side must be modified as
\[
    \int_\Omega fv -\theta \sum_{F \in \mathcal{F}^B_h}\int_F g_D \grad_h v \cdot \vect{n} + \sum_{F \in \mathcal{F}^B_h} \int_F \gamma g_D v 
\]
\subsubsection*{Neumann BC}
In the case of Neumann BC, like 
\[
    \grad u \cdot \vect{n} = g_N \quad \txt{ on } \partial\Omega
\]
the bilinear form has to be modified such as 
\begin{equation*}
    \begin{split}
        \mathcal{A}(u, v) =  \sum_{\element \in \triangulation} \int_\element \grad u \cdot \grad v  - \sum_{F \in \mathcal{F}'_h} \int_F \average{\grad u} \cdot \jump{v}  - \theta \sum_{F \in \mathcal{F}'_h} \int_F \jump{u} \cdot \average{\grad_h v} \\+ \sum_{F \in \mathcal{F}'_h} \int_F \gamma \jump{u} \cdot \jump{v}
    \end{split}
\end{equation*}
and the right hand side 
\[
    \int_\Omega fv - \sum_{F \in \mathcal{F}^B_h}\int_F g_N v
\]
\subsubsection*{The stabilization function \texorpdfstring{\(\gamma\)}{gamma}}
\[
    \sum_{F \in \mathcal{F}_h} \int_F \gamma \jump{u} \cdot \jump{v} \quad \gamma = \alpha \frac{p^2}{h}
\]
where 
\[
    p =\begin{cases}
        \max\left\{ p_{\element^+}, p_{\element^-} \right\} & \txt{ if } F \in \mathcal{F}'_h \\
        p_\element & \txt{ if } F \in \mathcal{F}^B_h
    \end{cases}
\]
and 
\[
    h =\begin{cases}
        \min\left\{ h_{\element^+}, h_{\element^-} \right\} & \txt{ if } F \in \mathcal{F}'_h \\
        h_\element & \txt{ if } F \in \mathcal{F}^B_h
    \end{cases}
\]
Since we can make some assumptions 
\[
    h_F \approx h_{\element^+} \approx h_{\element^-}, \ p_{\element^+} \approx p_{\element^-} \Rightarrow \gamma = \mathcal{O}\left( \frac{p^2}{h} \right)
\]

\subsection{Theoretical reminders}

For an integer \(s \geq 1\) define the broken Sobolev space
\begin{align*}
    H^s(\triangulation) &= \left\{ v \in L^2(\Omega) : v\limited{\element} \in H^s(\element) \ \forall \;\element \in \triangulation\right\} \\
    \norm{v}^2_{H^s(\element)} &= \sum_{\element \in \triangulation} \norm{v}^2_{H^s(\element)}
\end{align*}
Define also 
\[
    \norm{v}^2_{L^2(\mathcal{F}_h)} = \sum_{F \in \mathcal{F}_h} \norm{v}^2_{L^2(F)}
\]
We define then the following norms 
\begin{align*}
    \norm{v}^2_{DG} &= \norm{\grad_h v}^2_{L^2(\Omega)} + \norm{\gamma^{\frac{1}{2}}\jump{v}}^2_{L^2(\mathcal{F}_h)} \quad \forall \;v \in H^2(\triangulation) \\
    \threenorm{v}{DG} &= \norm{v}_{DG}^2 + \norm{\gamma ^{\frac{1}{2}} \average{\grad_h v}}^2_{L^2(\mathcal{F}_h)} \quad \forall \;v \in H^2(\triangulation)
\end{align*}
where \(\grad_h v\) is the elementwise gradient:
\[
    (\grad_h v)\limited{\element} = \grad(v\limited{\element}) \quad\forall \; \element \in \triangulation
\]
Notice that \(V_h^p \subset H^2(\triangulation)\). It can be shown that 
\begin{align*}
    \norm{v}_{DG} &\underset{(trivial)}{\leq} \threenorm{v}{DG} \not\lesssim \norm{v}_{DG} \quad \forall \; v \in H^2(\triangulation) \\
    \norm{v_h}_{DG} &\underset{(trivial)}{\leq} \threenorm{v_h}{DG} \underset{(on \ slides)}{\not\lesssim}\norm{v_h}_{DG} \quad \forall \; v_h \in V_h^p \\
\end{align*}
Some key properties:
\begin{itemize}
    \item Continuity on \(H^2(\triangulation) \times V^p_h\): 
    \[
        \abs{\mathcal{A}(v, w_h)} \lesssim \threenorm{v}{DG} \norm{w_h}_{DG} \quad  \forall \; v \in H^2(\triangulation), \ \forall \; w_h \in V_h^p
    \]
    Also remind that \(\abs{\mathcal{A}(v, w_h)} \not\lesssim \norm{v}_{DG} \norm{w_h}_{DG}\)
    \item Coercivity on \(V_h^p \times V_h^p\): 
    \[
        \mathcal{A}(v_h, v_h) \gtrsim \norm{v_h}_{DG} \quad \forall \; v_h \in V_h^p
    \]
    For SIP and IIP, the penaly parameter \(\alpha\) should be large enough.
    \item  Strong-consistency (Galerkin orthogonality):
    \[
        \mathcal{A}(u,v_h) = \int_\Omega f v_h \ \ \forall \; v_h \in V_h^p \Rightarrow \mathcal{A}(u-u_h, v_h) = 0 \quad \forall \; v_h \in V_h^p
    \]
    \item Approximation. Let \(\prod_{h}^{p}u \in V_h^p\) be a suitable approximation of \(u\), then 
    \[
        \threenorm{u-{\textstyle \prod_{h}^{p}u}}{DG} \lesssim \frac{h^{\min(p,s)}}{p^{s-\frac{1}{2}}}\norm{u}_{H^{s+1}(\triangulation)}
    \]
    If \(p \geq s\)
    \[
        \threenorm{u-{\textstyle \prod_{h}^{p}u}}{DG} \lesssim \left( \frac{h}{p} \right)^s p^{\frac{1}{2}}\norm{u}_{H^{s+1}(\triangulation)}
    \]
\end{itemize}
\subsection{Error estimates}
Recall the abstract error estimate \(\norm{u - u_h}_{DG} \lesssim \threenorm{u-\prod_{h}^{p}}{DG}\).

If \(u\) is sufficiently regular then
\[
    \norm{u-u_h}_{DG} \lesssim \frac{h^{\min(p,s)}}{p^{s-\frac{1}{2}}}\norm{u}_{H^{s+1}(\triangulation)}
\]
Then, by using a duality argument, one can obtain an estimate for the \(L^2\) norm.

Assuming that \(\Omega\) is such that the following elliptic regularity result holds: for any \(g \in L^2(\Omega)\), the solution \(z\) of the problem
\[
   \begin{cases}
     -\Delta z = g & \txt{in }\Omega \\
     z = 0 & \txt{on } \partial\Omega
   \end{cases}
\]
satisfies \(z \in H^2(\Omega)\) and 
\[
    \norm{z}_{H^2(\Omega)} \lesssim \norm{g}_{L^2(\Omega)}
\]
If the exact solution \(u \in H^s(\Omega), s \geq 2\) and, if \(u_h\) is obtained with the SIP method, it holds 
\[
    \norm{u-u_h}_{L^2(\Omega)} \lesssim \frac{h^{\min(p,s)+1}}{p^{s+\frac{1}{2}}}\norm{u}_{H^{s+1}(\Omega)}
\]
while for NIP and IIP holds
\[
    \norm{u-u_h}_{L^2(\Omega)} \lesssim \frac{h^{\min(p,s)}}{p^{s-\frac{1}{2}}}\norm{u}_{H^{s+1}(\Omega)}
\]