\newpage
\section{Advection-Diffusion-Reaction equations}
\subsection{Formulation of the problem}
Consideting the problem \(\mathcal{L}u = f\) in \(\Omega\), \(u=0\) on \(\partial\Omega\) where 
\begin{align*}
    &\mathcal{L} u = -\div \left(\mu \grad u + \vect{b} u\right) + \sigma u & \txt{(conservative form)} \\
    &\mathcal{L} u = -\div \left(\mu \grad u\right) + \vect{b} \cdot \grad u + \sigma u & \txt{(non-conservative form)}
\end{align*}
with the same assumptions as \eqref{bvp_first_page}. 

The weak formulation is written as 
\begin{equation}
    \find u \in V = H^1_0(\Omega) : a(u,v) = F(v) \ \forall \; v \in V
    \label{weak_formulation_adr}
\end{equation}
with 
\[
    F(v) = \int_\Omega fv
\]
and 
\[
    a(u,v) = \begin{cases}
        \displaystyle\int_\Omega \left( \mu \grad u + \vect{b} u \right) \cdot \grad v + \int_\Omega \sigma u v & \txt{conservative form} \\
        \\
        \displaystyle\int_\Omega \mu \grad u \cdot \grad v + \int_\Omega \vect{b} \cdot \grad u v + \int_\Omega \sigma u v & \txt{non-conservative form}
    \end{cases}
\]
Let's verify the uniqueness of the solution:
\subsubsection*{Coercivity}
Sufficient conditions for coercivity:
\begin{align*}
    &\sigma - \frac{1}{2} \div \vect{b} \geq 0 \txt{ in } \Omega & \txt{non-conservative case} \\
    &\sigma + \frac{1}{2} \div \vect{b} \geq 0 \txt{ in } \Omega & \txt{conservative case}  
\end{align*}
In both cases: \(a(u,v) \geq \mu_0\norm{\grad u}^2 \rightarrow \txt{ coercivity constant } \alpha \simeq \mu_0\)
\subsubsection*{Continuity}
In both cases, continuity constant: \(M \simeq \norm{\mu}_{L^\infty} + \norm{\vect{b}}_{L^\infty} + \norm{\sigma}_{L^2}\)

Given that the hypotheses of Lax-Milgram holds, the solution exists and is unique. We can now bring in the Galerkin formulation
\[
    \find u_h \in V_h : a(u_h, v_h) = (f,v_h) \quad \forall \; v_h \in V_h
\]
and move to the error estimate 
\[
    \norm{u-u_h} \underset{(\txt{Ceà})}{\leq} \frac{M}{\alpha} \inf_{v_h \in V_h} \norm{u-v_h} \underset{\substack{(\txt{interpolation} \\\txt{error estimate})}}{\leq} C \frac{M}{\alpha} h^r \abs{u}_{H^{r+1}(\Omega)}
\]
If it is a convection dominated flow (or reaction dominated), then \(\frac{M}{\alpha} \gg 1\), then we need to find a tradeoff between \(\frac{M}{\alpha}\) and \(h^r\). Also it is numerically prohibitive. 

The Péclet number tells us if the flow is dominated by advection or diffusion if its greater or smaller than \(1\). We can define it as 
\[
    \mathbb{P}e = h\frac{M}{\alpha}
\]
Should be less than \(1\) for stability issues.
\subsection{Stabilization methods}
The idea now is to stabilize the Galerkin method. 
\begin{itemize}
    \item 1D case: Upwind method \(\iff\) Artificial diffusion
    \item 2D case: Streamline diffusion: 
    \[
        +c(h) \int_\Omega \frac{1}{\norm{\vect{b}}} \left( \vect{b} \cdot \grad u_h \right) \left( \vect{b} \cdot \grad v_h \right)
    \]
    Artificial diffusion: 
    \[
        +c(h)\int_\Omega \grad u_h \cdot \grad v_h 
    \]
\end{itemize}
Now the solution is stabilized, but is not fully consistent. So the solution is to find a way to obtain a fully consistent solution 
\[
    \find u_h \in V_h : a(u_h, v_h) + \mathscr{L}_h (u_h, f; v_h) = F(v_h) \quad \forall \; v_h \in V_h
\]
with \(\mathscr{L}_h\) suitably chosen such that
\[
    \mathscr{L}(u_h, f; v_h) = 0 \quad \forall \; v_h \in V_h
\]
so we obtain a strongly consistent approximation of the original problem.

One possibility could be to use an operator proportional to the residual:
\[
    \mathscr{L}_h(u_h, f;v_h) = \sum_{\element \in \triangulation} \int_\element(\mathcal{L}u - f)\tau_\element \phi(v_h) \quad \forall \; v_h \in V_h  
\]
with \(\tau_\element\) as a scaling factor. Typically is chosen, given \(h_\element = diam(\element)\):
\[
    \tau_\element (\vect{x}) = \delta \frac{h_\element}{\abs{\vect{b}(\vect{x})}} \quad \forall \; \vect{x} \in \element, \element \in \triangulation
\]
while, for \(\phi(v_h)\) there are many possibilities. Two of them are 
\begin{itemize}
    \item \(\phi(v_h) = \mathcal{L}v_h \rightarrow\) GLS - Galerkin Least Squares method
    \item \(\phi(v_h) = \mathcal{L}_{ss}v_h \rightarrow\) SUPG - Streamline Upwind Petrov-Galerkin method
\end{itemize}
Brief notation remark: \(\mathcal{L} = \mathcal{L}_s +\mathcal{L}_{ss}\) (symmetric + skew-symmetric part)
Which we define as 
\begin{align*}
    {}_{V'}\langle\mathcal{L}_su, v \rangle_{V} &= {}_{V}\langle v, \mathcal{L}_su \rangle_{V'}  \quad \forall \; u, v \in V \\
    {}_{V'}\langle\mathcal{L}_{ss}u, v \rangle_V &= -{}_{V}\langle v, \mathcal{L}_{ss}u \rangle_{V'} \quad \forall \; u, v \in V 
\end{align*}
For matrices it is \(A = A_S + A_{SS}\) 
\[
  A_S = \frac{1}{2} (A + A^T) \quad A_{SS} = \frac{1}{2} (A -A^T)
\]
Let us see an example in the non conservative form 
\begin{align*}
        \mathcal{L}^1 &= -\mu \Delta u + \vect{b} \cdot \grad u + \sigma u \\
        &= \underbrace{\left[ -\mu \Delta u + \left( \sigma - \frac{1}{2} \div \vect{b} \right)u\right]}_{\mathcal{L}^1_s u} + \underbrace{\left[ \frac{1}{2} \left( \div(\vect{b}u) + \vect{b} \cdot \grad u \right) \right]}_{\mathcal{L}^1_{ss}u}
\end{align*}
Indeed we can see
\begin{align*}
    {}_{V'}\langle \mathcal{L}^1_s, v \rangle_V &= \int_\Omega \mu \grad u \cdot \grad v + \left( \sigma - \frac{1}{2} \div \vect{b} \right) u v \\
    &= \int_\Omega \left[ -\mu \Delta v + \left( \sigma - \frac{1}{2} \div \vect{b} \right) v \right] u = {}_{V}\langle v, \mathcal{L}^1_s \rangle_{V'}
\end{align*}
\begin{align*}
    {}_{V'}\langle \mathcal{L}^1_{ss}, v \rangle_V &= \frac{1}{2} \int_\Omega (\div(\vect{b}u)v+(\vect{b} \cdot \grad u)v) \\
    &= \frac{1}{2} \int_\Omega (-(\vect{b}u) \grad v + (\vect{b}v) \cdot \grad u) \\
    &= \frac{1}{2} \int_\Omega (-(\vect{b}\cdot \grad v)u - \div(\vect{b}v)u) = -{}_V\langle u, \mathcal{L}_{ss}^1 \rangle_{V'}
\end{align*}
\begin{remark}
    If \(\div \vect{b} = 0\), which happens if \(\vect{b}\) is constant, then the conservative and non conservative forms coincide.
\end{remark}
\subsection{GLS method (conservative form)}
\[
    \find u_h \in V_h : a(u_h, v_h) + \sum_{\element \in \triangulation} \int_\Omega \mathcal{L}u_h \tau_\element \mathcal{L}v_h = \int_\Omega f v_h + \sum_{\element \in \triangulation} \int_\Omega f \tau_\element \mathcal{L}v_h \qquad \forall \; v_h \in V_h
\]
\begin{theorem}
    Consider the conservative case. Suppose that 
    \[
        \exists \gamma_0 , \gamma_1 > 0 : 0 < \gamma_0 \leq \gamma(\vect{x}) \leq \gamma_1
    \]
    then, for a suitable constant \(C\), independent of \(h\), we have:
    \[
        \norm{u_h}_{GLS}^2 \leq C \norm{f}_{L^2(\Omega)}^2
    \]
    where \(\normdot_{GLS}\) will be defined later
\end{theorem}
\begin{proof}
    Take \(u_h = v_h\). We have 
    \begin{align*}
        a_h(u_h, u_h) &= \int_\Omega \mu\abs{\grad u_h}^2 + \underbrace{\int_\Omega \div(\vect{b}\,u_h) u_h}_{\mathclap{\substack{=-\int_\Omega \vect{b}\cdot (u_h \grad u_h) \\ = -\frac{1}{2}\int_\Omega \vect{b}\cdot \grad(u^2_h) \\ =\frac{1}{2} \int_\Omega \div \vect{b} \, u^2_h}}} +\int_\Omega \sigma u^2_h + \sum_{\element \in \triangulation} \int_\element \tau_\element (\mathcal{L}u_h)^2 \\
        &= \int_\Omega \mu \abs{\grad u_h}^2 + \int_\Omega \underbrace{\left( \sigma + \frac{1}{2} \div \vect{b} \right)u_h^2}_{=: \gamma(\vect{x})} + \sum_{\element \in \triangulation} \int_\element \tau_\element(\mathcal{L}u_h)^2 \\
        &=: \norm{u_h}_{GLS}^2
    \end{align*}
    On the other hand
    \[
        \abs{F_h(u_h)} \leq \abs{\int_\Omega f u_h} + \abs{{\textstyle \sum_{\element \in \triangulation}}\int_{\element} f \tau_\element \mathcal{L}u_h}
    \]
    where 
    \begin{equation*}
        \begin{split}
            \abs{\int_\Omega f u_h} = \abs{\int_\Omega \frac{1}{\sqrt{\gamma}} f \sqrt{\gamma}u_h} \underset{\txt{\tiny{Cauchy-Schwartz}}}{\leq} \norm{\frac{1}{\sqrt{\gamma}}f}_{L^2(\Omega)} \ltwonorm{\sqrt{\gamma}u_h} \\
            \underset{\txt{\tiny Young}}{\leq} \ltwonorm{\frac{1}{\sqrt{\gamma}}f}^2 + \frac{1}{4} \ltwonorm{\sqrt{\gamma}u_h}^2
        \end{split}
    \end{equation*}
    and where 
    \begin{equation*}
        \begin{split}
            \abs{\sum_{\element \in \triangulation} \int_\element f \tau_\element \mathcal{L}u_h} = \abs{\sum_{\element \in \triangulation} \int_\element \sqrt{\tau_\element} f \sqrt{\tau_\element}\mathcal{L}u_h} \\
            \underset{\txt{\tiny Cauchy-Schwartz}}{\leq} \sum_{\element\in\triangulation}\norm{\sqrt{\tau_\element}f}_{L^2(\element)}\norm{\sqrt{\tau_\element} \mathcal{L}u_h}^2_{L^2(\element)} \\
            \underset{\txt{\tiny Young}}{\leq} \sum_{\element\in\triangulation}\norm{\sqrt{\tau_\element}f}_{L^2(\element)}^2 +\frac{1}{4} \norm{\sqrt{\tau_\element} \mathcal{L}u_h}^2_{L^2(\element)} \\
        \end{split}
    \end{equation*}
    So, \(a_h(u_h, u_h) = F_h(u_h)\) implies:
    \begin{align*}
        \norm{u_h}_{GLS}^2 &= \int_\Omega \mu \abs{\grad u_h}^2 + \int_\Omega \gamma u_h^2 + \sum_{\element \in \triangulation} \int_\element \tau_\element (\mathcal{L}u_h)^2 \\
        &\leq \left[ \norm{\frac{1}{\sqrt{\gamma}}f}_{L^2(\Omega)} + \sum_{\element \in \triangulation} \norm{\sqrt{\tau_\element}f}_{L^2(\Omega)}^2 \right] \\
        & \ + \frac{1}{4} \left[ \int_\Omega \gamma u_h^2 + \sum_{\element \in \triangulation} \tau_\element (\mathcal{L}u_h)^2 \right] \\
        &\leq \underbrace{\left( \frac{1}{\gamma_0} +\max_{\element \in \triangulation} \tau_\element \right)}_{\mathclap{=C(\txt{if }\tau_\element \txt{ uniformly bounded w.r.t. }h)}} \ltwonorm{f}^2 +\frac{1}{4} \norm{u_h}_{GLS}^2
    \end{align*}
    In the end 
    \[
        \norm{u_h}_{GLS}^2 \leq \frac{4}{3} C \norm{f}_{L^2(\Omega)}^2
    \]
\end{proof}
As we already said, a smart choice for \(\scalingfactor\) is \(\delta\frac{h_\element}{\abs{\vect{b}(\vect{x})}}\). But another possibility may be 
\[
    \scalingfactor(\vect{x}) = \frac{h_\element}{2\abs{\vect{b}(\vect{x})}} \xi(\peclet_\element)
\]
with \(\xi(\theta) = \coth(\theta)-\frac{1}{\theta}\). and \(\peclet_\element (\vect{x}) = \frac{\abs{\vect{b}(\vect{x})}}{2\mu(\vect{x})}h_\element\) is the local Péclet number. 
Moreover, if \(\theta \to 0\), then \(\xi(\theta) = \frac{\theta}{3} + o(\theta)\), therefore when \(\peclet_\element(\vect{x}) \ll 1\), we have \(\tau_\element(\vect{x}) \to 0\) and no stabilization is needed.
\subsection{Convergence of GLS}
To state the convergence of GLS we need the inverse inequality, defined as
\begin{equation}
    \sum_{\eit} h^2_\element \int_\element \abs{\Delta v_h}^2 \, d\element \leq C_0 \ltwonorm{\grad v_h}^2 \quad \forall \; v_h \in X_h^r
    \label{inverse_inequality_adr}
\end{equation}
\begin{theorem}[Convergence of GLS]
    Assume that the space \(V_h\) satisfies the following local approximation property: for each \(v \in V \cap H^{r+1}(\Omega)\), there exists a function \(\hat{v}_h \in V_h\) s.t. 
    \begin{equation}
        \norm{v-v_h}_{L^2(\element)} + h_\element \norm{v-\hat{v}_h}_{H^1(\element)} + h^2_\element \abs{v-\hat{v}_h}_{H^2(\element)} \leq Ch^{r+1}_\element \abs{v}_{H^{r+1}}
        \label{convergence_gls_adr}
    \end{equation}
    for each \(\eit\). Moreover, we suppose that for each \(\eit\) the local Péclet number of \(K\) satisfies 
    \begin{equation}
        \peclet_\element (\vect{x}) = \frac{\abs{\vect{b}(\vect{x})}h_\element}{2\mu} > 1 \quad \forall \; \vect{x} \in \element
        \label{local_peclet_adr}
    \end{equation}
    that is, we are in the pre-asymptotic regime. Finally, we suppose that the inverse inequality holds and that the stabilization parameters satisfies the relation \(0 < \delta \leq 2C_0^{-1}\).

    Then, as long as \(u \in H^{r+1}(\Omega)\), the following super-optimal estimate holds:
    \begin{equation}
        \norm{u-u_h}_{GLS} \leq Ch^{r+\frac{1}{2}} \abs{u}_{H^{r+1}(\Omega)}        
        \label{super-estimate_adr}
    \end{equation}
\end{theorem}
\begin{proof}
    First of all, rewrite the error as 
    \begin{equation}
        e_h = u_h - u = \sigma_h - \eta
        \label{error_rewrite_adr}
    \end{equation}
    
    with \(\sigma_h = u_h - \hat{u}_h, \eta = u - \hat{u}_h\), where \(\hat{u}_h\) is a function that depends on \(u\) and that satisfies property \eqref{convergence_gls_adr}. If, for instance, \(V_h = X_h^r \cap H^1_0(\Omega)\), we can choose \(\hat{u}_h = \prod_{h}^{r} u\) that is the finite element interpolant of \(u\). 

    We start by estimating the norm \(\norm{\sigma_h}_{GLS}\). By exploiting the strong consistency of the GLS scheme we obtain 
    \[
        \norm{\sigma_h}^2_{GLS} = a_h(\sigma_h, \sigma_h) = a_h(u_h - u +\eta, \sigma_h) = a_h(\eta, \sigma_h)
    \]
    Now, thanks to the homogeneous Dirichlet boundary conditions it follows that, by adding and subtracting \(\sum_{\eit}\left(\eta, \mathcal{L}\sigma_h\right)_\element\), suitable computations lead to:
    \begin{align*}
        a_h(\eta, \sigma_h) &= \mu \sigma_\Omega \grad \eta \cdot \grad \sigma_h \, d\Omega - \int_\Omega \eta \vect{b} \cdot \grad \sigma_h \, d\Omega + \int_\Omega \sigma \eta \sigma_h \, d\Omega \\
        & \quad + \sum_\eit \delta \left( \mathcal{L}\eta, \frac{h_\element}{\abs{\vect{b}}}\mathcal{L}\sigma_h \right)_{L^2(\element)} \\
        &= \underbrace{\mu\left( \grad \eta, \grad \sigma_h \right)_{L^2(\Omega)}}_{(I)} - \underbrace{\sum_\eit \left(\eta, \mathcal{L}\sigma_h\right)_{L^2(\Omega)}}_{(II)} + \underbrace{2\left( \gamma \eta, \sigma_h \right)_{L^2(\element)}}_{(III)} \\
        &\quad + \underbrace{\sum_\eit \left( \eta, -\mu\Delta \sigma_h \right)_{L^2(\element)}}_{(IV)} + \underbrace{\sum_\eit \delta \left( \mathcal{L}\eta, \frac{h_\element}{\abs{\vect{b}}\mathcal{L}\sigma_h} \right)_{L^2(\element)}}_{(V)} 
    \end{align*}
    Now, we bound each of these terms. By using Cauchy-Schwartz and Young's inequalities we obtain 
    {
        \allowdisplaybreaks
        \begin{align*}
        \abs{(I)} &= \abs{\mu\left( \grad \eta, \grad \sigma_h \right)_{L^2(\element)}} \leq \frac{\mu}{4} \ltwonorm{\grad \sigma_h}^2 + \mu \ltwonorm{\grad \eta}^2 \\
        \abs{(II)} &= \abs{\sum_eit \left( \eta, \mathcal{L}\sigma_h \right)_{L^2(\element)}} \\
        & = \abs{\sum_\eit \left( \sqrt{\frac{\abs{\vect{b}}}{\delta h_\element}} \eta, \sqrt{\frac{\delta h_\element}{\abs{\vect{b}}}} \mathcal{L}\sigma_h \right)_{L^2(\Omega)}} \\
        &\leq \frac{1}{4} \sum_\eit \delta \left( \frac{h_k}{\abs{\vect{b}}} \mathcal{L}\sigma_h, \mathcal{L}\sigma_h \right) \\
        \abs{(III)} &= 2\abs{\left( \gamma \eta, \sigma_h \right)_{L^2(\Omega)}} = 2\abs{\left( \sqrt{\gamma \eta, \sqrt{\gamma}\sigma_h} \right)_{L^2(\Omega)}} \\
        &\leq \frac{1}{2} \ltwonorm{\sqrt{\gamma}\sigma_h}^2 + 2 \ltwonorm{\sqrt{\gamma} \eta}^2
    \end{align*}
    }
    Then, thanks to CS and Young, but also hypotheses \eqref{local_peclet_adr} and \eqref{inverse_inequality_adr}, we obtain 
    \begin{align*}
        \abs{(IV)} &= \abs{\sum_\eit \left( \eta, -\mu \Delta \sigma_h \right)_{L^2(\element)}} \\
        &\leq \frac{1}{4} \sum_\eit \delta \mu^2 \left( \frac{h_\element}{\abs{\vect{b}}}\Delta \sigma_h, \Delta \sigma_h \right)_{L^2(\element)} + \sum_\eit \left( \frac{\abs{\vect{b}}}{\delta h_\element} \eta, \eta \right)_{L^2(\element)} \\
        &\leq \frac{1}{8} \delta \mu \sum_\eit h^2_\element \left( \grad \sigma_h, \grad \sigma_h \right)_{L^2(\element)} + \sum_\eit \left( \frac{\abs{\vect{b}}}{\delta h_\element} \right)_{L^2(\element)} \\
        &\leq \frac{\sigma C_0 \mu}{8} \ltwonorm{\grad \sigma_h}^2 + \sum_\eit \left( \frac{\abs{\vect{b}}}{\delta h_\element}\eta, \eta \right)_{L^2(\element)}
    \end{align*}
    The last one can be bounded once again thanks to CS and Young inequalities as follows 
    \begin{align*}
        \abs{(V)} &= \abs{\sum_\eit \delta \left( \mathcal{L}\eta, \frac{h_\element}{\vecbabs}\mathcal{L}\sigma_h \right)_{L^2(\element)}} \\
        &\leq \frac{1}{4} \sum_\eit \delta \left( \frac{h_\element}{\vecbabs} \mathcal{L}\sigma_h, \mathcal{L}\sigma_h \right)_{L^2(\element)} + \sum_\eit \delta\left( \frac{h_\element}{\vecbabs} \mathcal{L}\eta, \mathcal{L}\eta \right)_{L^2(\element)} 
    \end{align*}
    So we can rewrite everything bounded as 
    \begin{align*}
        \norm{\sigma_h}^2_{GLS} &= a_h(\eta, \sigma_h) \leq \frac{1}{4} \norm{\sigma_h}^2_{GLS} \\
        &\quad + \frac{1}{4} \left( \ltwonorm{\sqrt{\gamma}\sigma_h}^2 + \sum_\eit \delta \left( \frac{h_\element}{\vecbabs} \mathcal{L}\sigma_h,\mathcal{L} \sigma_h \right)_{L^2(\element)} \right) + \frac{\delta C_0 \mu}{8} \ltwonorm{\grad \sigma_h}^2 \\
        &\quad + \underbrace{\mu\ltwonorm{\grad \eta}^2 + 2 \sum_\eit \left( \frac{\vecbabs}{\delta h_\element} \eta, \eta \right)_{L^2(\element)} + 2 \ltwonorm{\sqrt{\gamma}\eta}^2 + \sum_\eit \delta \left( \frac{h_\element}{\vecbabs} \mathcal{L}\eta, \mathcal{L}\eta \right)_{L^2(\element)}}_{\mathcal{E}(\eta)} \\
        &\leq \frac{1}{2} \norm{\sigma_h}_{GLS}^2 + \mathcal{E}(\eta)
    \end{align*}
    Having exploited the assumption that \(\delta \leq 2C^{-1}_0\). We can state then
    \[
        \norm{\sigma_h}_{GLS}^2 \leq 2\mathcal{E}(\eta)
    \]
    It's time to estimate \(\mathcal{E}(\eta)\), by bounding each of it's summands separately. To do this, we will use the local approximation property \eqref{convergence_gls_adr} and the local Péclet \eqref{local_peclet_adr}. Moreover, we observe that the constant \(C\), introduced in the remainder, depends neither on \(h\) nor on \(\peclet_\element\), but can depend on other quantities such as the constant \(\gamma_1\), the reaction constant \(\sigma\) or the norm \(\norm{\vect{b}}_{L^\infty(\Omega)}\), the stabilization parameter \(\delta\).

    Then we have 
    \begin{align*}
        \mu \ltwonorm{\grad \eta}^2 &\leq C\mu h^{2r} \abs{u}_{H^{r+1}(\Omega)}^2 \\
        &\leq C\frac{\norm{\vect{b}}_{L^\infty(\Omega) h}}{2} h^{2r} \abs{u}^2_{H^{r+1}(\Omega)} \leq C h^{2r+1}\abs{u}_{H^{r+1}(\Omega)}^2\\
        2\sum_\eit \left( \frac{\vecbabs}{\delta h_\element} \eta, \eta \right)_{L^2(\element)} &\leq  C\frac{\norm{\vect{b}}_{L^\infty(\Omega) h}}{2} \sum_\eit \frac{1}{h_\element} h^{2r+1}_\element \abs{u}^2_{H^{r+1}(\Omega)} \\
        &\leq c h^{2r+1} \abs{u}^2_{H^{r+1}(\Omega)} \\
        2 \ltwonorm{\sqrt{\gamma}\eta}^2 &\leq 2\gamma_1 \ltwonorm{\eta}^2 \leq Ch^{2r+1} \abs{u}^2_{H^{r+1}(\Omega)} 
    \end{align*} 
    For the fouth term we have 
    \begin{align*}
        \sum_\eit \delta \left( \frac{h_\element}{\vecbabs}\mathcal{L}\eta, \mathcal{L}\eta \right)_{L^2(\element)} &= \sum_\eit \norm{\sqrt{\frac{h_\element}{\vecbabs}} \mathcal{L}\eta}_{L^2(\element)}^2 \\
        &= \sum_\eit \delta \norm{-\mu \sqrt{\frac{h_\element}{\vecbabs}} \Delta \eta + \sqrt{\frac{h_\element}{\vecbabs}} \div(\vect{b}\eta) + \sigma \sqrt{\frac{h_\element}{\vecbabs}} \eta}_{L^2(\element)}^2 \stepcounter{equation}\tag{\theequation}\label{fourth_term_adr}\\
        &\leq C \sum_\eit \delta \Bigg(\norm{\mu \sqrt{\frac{h_\element}{\vecbabs}} \Delta \eta}_{L^2(\element)}^2 + \norm{\sqrt{\frac{h_\element}{\vecbabs}} \div(\vect{b}\eta)}^2_{L^2(\element)} \\
        &\quad + \norm{\sigma \sqrt{\frac{h_\element}{\vecbabs}}\eta}_{L^2(\element)}^2\Bigg) 
    \end{align*}
    Now it is easy to prove that the second and third term of the summands can be bounded using a term or the form \(Ch^{2r+1}\abs{u}_{H^{r+1}(\Omega)}^2\), for a suitable choice of the constant \(C\). For the first term we have 
    \begin{align*}
        \sum_\eit \delta \norm{\mu \sqrt{\frac{h_\element}{\vecbabs}} \Delta \eta}^2_{L^2(\element)} &\leq \sum_\eit \delta \frac{h^2_\element \mu}{2} \norm{\Delta \eta}^2_{L^2(\element)} \\
        &\leq C \delta \norm{\vect{b}}_{L^\infty(\Omega)}\sum_\eit h^3_\element \norm{\Delta}_{L^2(\element)}^2 \leq \abs{u}_{H^{r+1}(\Omega)}^2
    \end{align*}
    having used again \eqref{convergence_gls_adr} and \eqref{local_peclet_adr}. Now we can conclude that 
    \[
        \mathcal{E}(\eta) \leq C h^{2r+1}\abs{u}_{H^{r+1}(\Omega)}^2
    \]
    that is 
    \begin{equation}
        \norm{\sigma_h}_{GLS} \leq C h^{r+\frac{1}{2}}\abs{u}_{H^{r+1}(\Omega)}
        \label{final_proof_adr}
    \end{equation}
    Reverting to \eqref{error_rewrite_adr}, to obtain the desired estimate for the norm \(\norm{u_h-u}_{GLS}\) we need to estimate \(\norm{\eta}_{GLS}\). But thanks to \eqref{fourth_term_adr} we obtain 
    \[
        \norm{\eta}_{GLS} \leq C h^{r+\frac{1}{2}}\abs{u}_{H^{r+1}(\Omega)}
    \]
    Combining this with \eqref{final_proof_adr} we obtain \eqref{super-estimate_adr}.
\end{proof}
