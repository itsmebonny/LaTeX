\newpage
\section{Advection-Diffusion-Reaction equations}
\subsection{Formulation of the problem}
Consideting the problem \(\mathcal{L}u = f\) in \(\Omega\), \(u=0\) on \(\partial\Omega\) where 
\begin{align*}
    &\mathcal{L} u = -\div \left(\mu \grad u + \vect{b} u\right) + \sigma u & \txt{(conservative form)} \\
    &\mathcal{L} u = -\div \left(\mu \grad u\right) + \vect{b} \cdot \grad u + \sigma u & \txt{(non-conservative form)}
\end{align*}
with the same assumptions as \eqref{bvp_first_page}. 

The weak formulation is written as 
\begin{equation}
    \txt{find } u \in V = H^1_0(\Omega) : a(u,v) = F(v) \ \forall \; v \in V
    \label{weak_formulation_adr}
\end{equation}
with 
\[
    F(v) = \int_\Omega fv
\]
and 
\[
    a(u,v) = \begin{cases}
        \displaystyle\int_\Omega \left( \mu \grad u + \vect{b} u \right) \cdot \grad v + \int_\Omega \sigma u v & \txt{conservative form} \\
        \\
        \displaystyle\int_\Omega \mu \grad u \cdot \grad v + \int_\Omega \vect{b} \cdot \grad u v + \int_\Omega \sigma u v & \txt{non-conservative form}
    \end{cases}
\]
Let's verify the uniqueness of the solution:
\subsubsection*{Coercivity}
Sufficient conditions for coercivity:
\begin{align*}
    &\sigma - \frac{1}{2} \div \vect{b} \geq 0 \txt{ in } \Omega & \txt{non-conservative case} \\
    &\sigma + \frac{1}{2} \div \vect{b} \geq 0 \txt{ in } \Omega & \txt{conservative case}  
\end{align*}
In both cases: \(a(u,v) \geq \mu_0\norm{\grad u}^2 \rightarrow \txt{ coercivity constant } \alpha \simeq \mu_0\)
\subsubsection*{Continuity}
In both cases, continuity constant: \(M \simeq \norm{\mu}_{L^\infty} + \norm{\vect{b}}_{L^\infty} + \norm{\sigma}_{L^2}\)

Given that the hypotheses of Lax-Milgram holds, the solution exists and is unique. We can now bring in the Galerkin formulation
\[
    \txt{find }u_h \in V_h : a(u_h, v_h) = (f,v_h) \quad \forall \; v_h \in V_h
\]
and move to the error estimate 
\[
    \norm{u-u_h} \underset{(\txt{Ceà})}{\leq} \frac{M}{\alpha} \inf_{v_h \in V_h} \norm{u-v_h} \underset{\substack{(\txt{interpolation} \\\txt{error estimate})}}{\leq} C \frac{M}{\alpha} h^r \abs{u}_{H^{r+1}(\Omega)}
\]
If it is a convection dominated flow (or reaction dominated), then \(\frac{M}{\alpha} \gg 1\), then we need to find a tradeoff between \(\frac{M}{\alpha}\) and \(h^r\). Also it is numerically prohibitive. 

The Péclet number tells us if the flow is dominated by advection or diffusion if its greater or smaller than \(1\). We can define it as 
\[
    \mathbb{P}e = h\frac{M}{\alpha}
\]
Should be less than \(1\) for stability issues.
\subsection{Stabilization methods}
The idea now is to stabilize the Galerkin method. 
\begin{itemize}
    \item[1D case:] Upwind method \(\iff\) Artificial diffusion
    \item[2D case:] Streamline diffusion: 
    \[
        +c(h) \int_\Omega \frac{1}{\norm{\vect{b}}} \left( \vect{b} \cdot \grad u_h \right) \left( \vect{b} \cdot \grad v_h \right)
    \]
    Artificial diffusion: 
    \[
        +c(h)\int_\Omega \grad u_h \cdot \grad v_h 
    \]
\end{itemize}
Now the solution is stabilized, but is not fully consistent. So the solution is to find a way to obtain a fully consistent solution 
\[
    \txt{find } u_h \in V_h : a(u_h, v_h) + \mathscr{L}_h (u_h, f; v_h) = F(v_h) \quad \forall \; v_h \in V_h
\]
with \(\mathscr{L}_h\) suitably chosen such that
\[
    \mathscr{L}(u_h, f; v_h) = 0 \quad \forall \; v_h \in V_h
\]
so we obtain a strongly consistent approximation of the original problem.

One possibility could be to use an operator proportional to the residual:
\[
    \mathscr{L}_h(u_h, f;v_h) = \sum_{\element \in \triangulation} \int_\element(\mathcal{L}u - f)\tau_\element \phi(v_h) \quad \forall \; v_h \in V_h  
\]
with \(\tau_\element\) as a scaling factor. Typically is chosen, given \(h_\element = diam(\element)\):
\[
    \tau_\element (\vect{x}) = \delta \frac{h_\element}{\abs{\vect{b}(\vect{x})}} \quad \forall \; \vect{x} \in \element, \element \in \triangulation
\]
while, for \(\phi(v_h)\) there are many possibilities. Two of them are 
\begin{itemize}
    \item \(\phi(v_h) = \mathcal{L}v_h \rightarrow\) GLS - Galerkin Least Squares method
    \item \(\phi(v_h) = \mathcal{L}_{ss}v_h \rightarrow\) SUPG - Streamline Upwind Petrov-Galerkin method
\end{itemize}
Brief notation remark: \(\mathcal{L} = \mathcal{L}_s +\mathcal{L}_{ss}\) (symmetric + skew-symmetric part)
Which we define as 
\begin{align*}
    {}_{V'}\langle\mathcal{L}_su, v \rangle_{V} &= {}_{V}\langle v, \mathcal{L}_su \rangle_{V'}  \quad \forall \; u, v \in V \\
    {}_{V'}\langle\mathcal{L}_{ss}u, v \rangle_V &= -{}_{V}\langle v, \mathcal{L}_{ss}u \rangle_{V'} \quad \forall \; u, v \in V 
\end{align*}
For matrices it is \(A = A_S + A_{SS}\) 
\[
  A_S = \frac{1}{2} (A + A^T) \quad A_{SS} = \frac{1}{2} (A -A^T)
\]
Let us see an example in the non conservative form 
\begin{align*}
        \mathcal{L}^1 &= -\mu \Delta u + \vect{b} \cdot \grad u + \sigma u \\
        &= \underbrace{\left[ -\mu \Delta u + \left( \sigma - \frac{1}{2} \div \vect{b} \right)u\right]}_{\mathcal{L}^1_s u} + \underbrace{\left[ \frac{1}{2} \left( \div(\vect{b}u) + \vect{b} \cdot \grad u \right) \right]}_{\mathcal{L}^1_{ss}u}
\end{align*}
Indeed we can see
\begin{align*}
    {}_{V'}\langle \mathcal{L}^1_s, v \rangle_V &= \int_\Omega \mu \grad u \cdot \grad v + \left( \sigma - \frac{1}{2} \div \vect{b} \right) u v \\
    &= \int_\Omega \left[ -\mu \Delta v + \left( \sigma - \frac{1}{2} \div \vect{b} \right) v \right] u = {}_{V}\langle v, \mathcal{L}^1_s \rangle_{V'}
\end{align*}
\begin{align*}
    {}_{V'}\langle \mathcal{L}^1_{ss}, v \rangle_V &= \frac{1}{2} \int_\Omega (\div(\vect{b}u)v+(\vect{b} \cdot \grad u)v) \\
    &= \frac{1}{2} \int_\Omega (-(\vect{b}u) \grad v + (\vect{b}v) \cdot \grad u) \\
    &= \frac{1}{2} \int_\Omega (-(\vect{b}\cdot \grad v)u - \div(\vect{b}v)u) = -{}_V\langle u, \mathcal{L}_{ss}^1 \rangle_{V'}
\end{align*}
\begin{remark}
    If \(\div \vect{b} = 0\), which happens if \(\vect{b}\) is constant, then the conservative and non conservative forms coincide.
\end{remark}
\subsection{GLS method (conservative form)}
\[
    \txt{find }u_h \in V_h : a(u_h, v_h) + \sum_{\element \in \triangulation} \int_\Omega \mathcal{L}u_h \tau_\element \mathcal{L}v_h = \int_\Omega f v_h + \sum_{\element \in \triangulation} \int_\Omega f \tau_\element \mathcal{L}v_h \qquad \forall \; v_h \in V_h
\]
\begin{theorem}
    Consider the conservative case. Suppose that 
    \[
        \exists \gamma_0 , \gamma_1 > 0 : 0 < \gamma_0 \leq \gamma(\vect{x}) \leq \gamma_1
    \]
\end{theorem}