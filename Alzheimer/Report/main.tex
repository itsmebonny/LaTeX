% A LaTeX template for EXECUTIVE SUMMARY of the MSc Thesis submissions to
% Politecnico di Milano (PoliMi) - School of Industrial and Information Engineering
%
% P. F. Antonietti, S. Bonetti, A. Gruttadauria, G. Mescolini, A. Zingaro
% e-mail: template-tesi-ingind@polimi.it
%
% Last Revision: October 2021
%
% Copyright 2021 Politecnico di Milano, Italy. Inc. All rights reserved.

\documentclass[11pt,a4paper]{article}

%------------------------------------------------------------------------------
%	REQUIRED PACKAGES AND  CONFIGURATIONS
%------------------------------------------------------------------------------
% PACKAGES FOR TITLES
\usepackage{titlesec}
\usepackage{color}

% PACKAGES FOR LANGUAGE AND FONT
\usepackage[utf8]{inputenc}
\usepackage[english]{babel}
\usepackage[T1]{fontenc} % Font encoding
% PACKAGES FOR IMAGES
\usepackage{graphicx}
\graphicspath{{Images/}} % Path for images' folder
\usepackage{eso-pic} % For the background picture on the title page
\usepackage{subfig} % Numbered and caption subfigures using \subfloat
\usepackage{caption} % Coloured captions
\usepackage{transparent}

% STANDARD MATH PACKAGES
\usepackage{amsmath}
\usepackage{amsthm}
\usepackage{bm}
\usepackage[overload]{empheq}  % For braced-style systems of equations

% PACKAGES FOR TABLES
\usepackage{tabularx}
\usepackage{longtable} % tables that can span several pages
\usepackage{colortbl}

% PACKAGES FOR ALGORITHMS (PSEUDO-CODE)
\usepackage{algorithm}
\usepackage{algorithmic}

% PACKAGES FOR REFERENCES & BIBLIOGRAPHY
\usepackage[colorlinks=true,linkcolor=black,anchorcolor=black,citecolor=black,filecolor=black,menucolor=black,runcolor=black,urlcolor=black]{hyperref} % Adds clickable links at references
\usepackage{cleveref}
\usepackage[square, numbers, sort&compress]{natbib} % Square brackets, citing references with numbers, citations sorted by appearance in the text and compressed
\bibliographystyle{plain} % You may use a different style adapted to your field

% PACKAGES FOR THE APPENDIX
\usepackage{appendix}

% PACKAGES FOR ITEMIZE & ENUMERATES
\usepackage{enumitem}

% OTHER PACKAGES
\usepackage{amsthm,thmtools,xcolor} % Coloured "Theorem"
\usepackage{comment} % Comment part of code
\usepackage{fancyhdr} % Fancy headers and footers
\usepackage{lipsum} % Insert dummy text
\usepackage{tcolorbox} % Create coloured boxes (e.g. the one for the key-words)
\usepackage{stfloats} % Correct position of the tables
\usepackage{multirow}
\usepackage{multicol}





%-------------------------------------------------------------------------
%	NEW COMMANDS DEFINED
%-------------------------------------------------------------------------
% EXAMPLES OF NEW COMMANDS -> here you see how to define new commands
\newcommand{\bea}{\begin{eqnarray}} % Shortcut for equation arrays
\newcommand{\eea}{\end{eqnarray}}
\newcommand{\e}[1]{\times 10^{#1}}  % Powers of 10 notation
\newcommand{\mathbbm}[1]{\text{\usefont{U}{bbm}{m}{n}#1}} % From mathbbm.sty
\newcommand{\pdev}[2]{\frac{\partial#1}{\partial#2}}
% NB: you can also override some existing commands with the keyword \renewcommand

%----------------------------------------------------------------------------
%	ADD YOUR PACKAGES (be careful of package interaction)
%----------------------------------------------------------------------------
\usepackage{amsfonts} 

%----------------------------------------------------------------------------
%	ADD YOUR DEFINITIONS AND COMMANDS (be careful of existing commands)
%----------------------------------------------------------------------------
\usepackage{import}
\usepackage{xifthen}
\usepackage{pdfpages}
\usepackage{transparent}
\usepackage{wrapfig}

\newcommand{\incfig}[1]{%
    \def\svgwidth{\columnwidth}
    \import{./Images/}{#1.pdf_tex}
}


% Do not change Configuration_files/config.tex file unless you really know what you are doing.
% This file ends the configuration procedures (e.g. customizing commands, definition of new commands)


% Create color bluePoli (-> manuale grafica coordinata:  https://www.polimi.it/fileadmin/user_upload/il_Politecnico/grafica-coordinata/2015_05_11_46xy_manuale_grafica_coordinata.pdf)
\definecolor{bluePoli}{cmyk}{0.4,0.1,0,0.4}

% Custom theorem environments
\declaretheoremstyle[
  shaded={rulecolor=bluePoli!20, rulewidth=1pt, bgcolor=bluePoli!5},
  headfont=\color{bluePoli}\normalfont\bfseries,
  bodyfont=\color{black}\normalfont,
]{colored}

\captionsetup[figure]{labelfont={color=bluePoli}} % Set colour of the captions
\captionsetup[table]{labelfont={color=bluePoli}} % Set colour of the captions
\captionsetup[algorithm]{labelfont={color=bluePoli}} % Set colour of the captions

% \theoremstyle{colored}
% \newtheorem{theorem}{Theorem}[section]
% \newtheorem{proposition}{Proposition}[section]
% \newtheorem{definition}{Definition}[section]
% \newtheorem*{remark}{Remark}
% \newtheorem{lemma}{Lemma}[section]



% Insert here the info that will be displayed into your Title page
% -> title of your work
\renewcommand{\title}{\noindent A Mathematical Model for Alzheimer Disease:

from Microscopic to a Macroscopic Model using
the Two-Scale Homogenization Theory}

% -> author name and surname
\renewcommand{\author}{Andrea Bonifacio}
% -> MSc course
\newcommand\norm[1]{\lVert#1\rVert}
\newcommand{\course}{Methods and Models for Statistical Mechanics}
% -> advisor name and surname
\newcommand{\advisor}{prof. Silvia Lorenzani}
% IF AND ONLY IF you need to modify the co-supervisors you also have to modify the file Configuration_files/title_page.tex (ONLY where it is marked)
%\newcommand{\firstcoadvisor}{Name Surname} % insert if any otherwise comment
%\newcommand{\secondcoadvisor}{Name Surname} % insert if any otherwise comment
% -> academic year
\newcommand{\YEAR}{2023-2024}

%-------------------------------------------------------------------------
%	BEGIN OF YOUR DOCUMENT
%-------------------------------------------------------------------------
\begin{document}

%-----------------------------------------------------------------------------
% TITLE PAGE
%-----------------------------------------------------------------------------
% Do not change Configuration_files/TitlePage.tex (Modify it IF AND ONLY IF you need to add or delete the Co-advisors)
% This file creates the Title Page of the document
% DO NOT REMOVE SPACES BETWEEN LINES!

%\twocolumn[{\begin{@twocolumnfalse}

\AddToShipoutPicture*{\BackgroundPic}
\begin{figure}
    \begin{minipage}{0.5\textwidth}
    \hspace{-0.6cm}\includegraphics[width=0.8\textwidth]{logo_polimi_ing_indinf.eps}
    \end{minipage}
    \begin{minipage}{0.5\textwidth}
        \includegraphics[width=0.5\textwidth]{inr_logo_grisbleu_rvb.eps}
    \end{minipage}
\end{figure}

\vspace{-1mm}
\fontsize{0.3cm}{0.5cm}\selectfont \bfseries \textsc{\color{bluePoli} Project Report}\\

\vspace{-0.2cm}
\Large{\textbf{\color{bluePoli}{\title}}}\\

\vspace{-0.2cm}
\fontsize{0.3cm}{0.5cm}\selectfont \bfseries \textsc{\color{bluePoli} \course}\\

\vspace{-0.2cm}
\fontsize{0.3cm}{0.5cm} \selectfont \bfseries Authors: \textsc{\textbf{\author}}\\

\vspace{-0.4cm}
\fontsize{0.3cm}{0.5cm}\selectfont \bfseries Supervisor: \textsc{\textbf{\advisor}}\\

% if only ONE co-advisor is present:
%\vspace{-0.4cm}
%\fontsize{0.3cm}{0.5cm}\selectfont \bfseries Co-advisor: \textsc{\textbf{\firstcoadvisor}}\\
% if more than one co-advisors are present:
%\vspace{-0.4cm}
%\fontsize{0.3cm}{0.5cm}\selectfont \bfseries Co-advisors: \textsc{\textbf{\firstcoadvisor}}\textsc{\textbf{\secondcoadvisor}}\\

\vspace{-0.4cm}
\fontsize{0.3cm}{0.5cm}\selectfont \bfseries Academic year: \textsc{\textbf{\YEAR}}

\small \normalfont

\vspace{11pt}

\centerline{\rule{1.0\textwidth}{0.4pt}}

%\vspace{15pt}
%\end{@twocolumnfalse}}]

\thispagestyle{plain} % In order to not show the header in the first page


%%%%%%%%%%%%%%%%%%%%%%%%%%%%%%
%%     THESIS MAIN TEXT     %%
%%%%%%%%%%%%%%%%%%%%%%%%%%%%%%
In this work will be presented a mathematical model for Alzheimer's disease (AD) at the microscopic scale. The model will be derived from the Smoluchowski equation, which describes the evolution of the density of diffusing particles that coagulate in pairs. Then will be proved that the model two-scale converges to a macroscopic model asymptotically consistent with the original one.Indeed, the information given on the microscale by the non-homogeneous Neumann boundary condition is transferred into a source term appearing in the limiting (homogenized) equations. Furthermore, on the macroscale, the geometric structure of the perforated domain induces a correction in that the scalar diffusion coefficients defined at the microscale are replaced by tensorial quantities.
%-----------------------------------------------------------------------------
% INTRODUCTION
%-----------------------------------------------------------------------------

\section{Introduction}

Numerical simulations play a critical role in a wide array of scientific and engineering applications, providing insights into the behavior of physical systems under various conditions. Among the most prominent techniques for performing such simulations is Finite Element Modeling (FEM). FEM discretizes a continuous domain into a mesh of finite elements, allowing for the approximation of solutions to complex partial differential equations (PDEs). However, one significant drawback of FEM is its computational intensity, especially when high resolution is required for accurate results. This research aims to explore the potential of Deep Learning (DL) techniques to accelerate FEM simulations, focusing specifically on the deformation of objects subjected to external forces.

The deformation of an object under an applied force is directly tied to the object's discretization. In FEM, the object is represented by a mesh, where the resolution of the mesh—i.e., the size and number of elements—clearly impacts the accuracy and computational cost of the simulation. High-resolution meshes can capture fine details of deformation, leading to more accurate simulations, but they are computationally expensive and time-consuming. 

The main goal of this work is to study the efficacy of a method that combines both Finite Element Modeling (FEM) and DL to obtain a realistic simulation of an object in a fraction of the time that would be required by a traditional FEM simulation. The idea is to, somehow, train a DL model to have inside the information given by the refined discretization and pass them on a coarser discretization.

The idea of using DL techniques to solve scientific problem is not new. Thanks to the rise of new frameworks and libraries, such as TensorFlow and PyTorch, it is now possible to train very complex models on large datasets in a reasonable amount of time. For the problem at hand, a lot of different approaches can be found in the existing literature: a lot of them are based on the idea that the deep learning model should predict the whole dynamic of the system, for example MeshGraphNet \cite{pfaffLearningMeshBasedSimulation2021a} or its multiscale version \cite{fortunatoMultiScaleMeshGraphNets2022}, but these are just two examples of the many possible approaches \cite{jiangMeshfreeFlowNetPhysicsConstrainedDeep2020}, \cite{djeumouNeuralNetworksPhysicsInformed2022}, \cite{hanPredictingPhysicsMeshreduced2022a}. Other methods rely on solving a time independent problem, using various architectures, such as PINNs \cite{djeumouNeuralNetworksPhysicsInformed2022} or GNNs \cite{gaoPhysicsinformedGraphNeural2022}. The proposed method falls into the second category, as it will be explained in the following sections.

The inspiration for this work came from the world of Computational Fluid Dynamics (CFD), particularly from a paper in which the authors propose a data-driven correction to the solution of a coarse grid simulation, using a neural network trained on the fine grid simulation \cite{kienerDatadrivenCorrectionCoarse2023}. The main concept is to create two grids on a fixed domain in a way that the fine grid is a refinement of the coarse grid. In this way, by means of an interpolation operator, it is possible to have both solutions on the same grid and perform computations with them, such as the difference between the two solution or between the derivatives of the solution. The neural network is trained to predict the difference between the fine grid solution and the coarse grid solution, given the coarse grid solution as input. In the paper, they propose various machine learning models, such as a simple feedforward neural network, a random forest and a GNN, both in two and three dimensions. The results show that the neural network is able to predict the difference between the two solutions with a good accuracy. 

This work aims at finding a solution to the problem of accelerating FEM simulations in the context of solid mechanics. To achieve such a goal, a good method should have certain characteristics: \begin{itemize}
    \item It should be geometry independent, removing the need for a new training for each new geometry.ù
    \item Should at least be topology independent, meaning that the method should be able to work with different meshes.
    \item Avoid the ``black-box'' problem of having a model predict the whole system without having any insights on what's happening inside.
    \item It should be really fast, to be able to be used in real-time applications.
\end{itemize}
Of those, the first one is clearly the most difficult to achieve, as encoding a geometry is still an open problem. Some progress were made in the field CITARE GINO, but it's far from solved. Also, for the uses of this method, it is assumed that the geometry is known, so this problem is not addressed here. 

The other three points are addressed here. Let's start by the independence with respect to the topology. In the original paper, the authors found a way to superimpose the two grids, but that would imply that the two meshes must be created together, limiting the possibilities. In this case it is possible to take advantage of the FEM itself, allowing to compute the exact solution everywhere on the domain, and then interpolate it on the mutual grid. By following this approach, it is possible to have a method that is independent of the topology of the mesh, since the position of the points on the grid is always the same with respect to the domain.

The main problem with ``black-box'' models is that they are not interpretable, so if the model isn't working as expected, it is difficult to understand why. One example is the MeshGraphNet, which is a very complex model that tries to fully predict the dynamic of the system. If the prediction is accumulating errors, one must retrain the model from scratch, trying to understand what went wrong. In this case, the method is based on correcting a numerical solution, so the starting point is always the exact solution which the model is trained to correct, so theoretically should be easier to understand the weaknesses of the model.

Finally, the speed of the method is a critical point. The method should be able to be used in real-time applications, so it should be as fast as possible. The method proposed here is based on a neural network, which means that instead of solving a linear system of equations, the solution is obtained by a forward pass of the network, which basically consists of a series of matrix multiplications and non-linear functions. This is a very fast operation, especially if the network is small, so the method should be able to be used in real-time applications.

The rest of the report is organized as follows: Section \ref{sec:problem_setting} introduces the problem setting and the mathematical formulation of the problem. Section \ref{sec:neural_network} gives a brief overview of the neural network architectures used in this work. Section \ref{sec:numerical_results} presents the numerical results obtained by the proposed method on selected tests. Finally, Section \ref{sec:conclusions} summarizes the main findings and outlines possible future research directions.
\section{Smoluchowski equation}
Let's introduce the concept of $i$ cluster which is a cluster of polymers of length $i$: $i $ identical particles (polymers formed by $i$ monomers). Indicate with $u_{i}\geq0$ the concentration of the cluster, and suppose these clusters diffuse with diffusion coefficient $d_{i}>0$.
We can model this phenomena with a nonlinear equation: the Smoluchowski equation
\begin{equation}
\frac{\partial u_{i}}{\partial t}(t, x)-d_{i} \triangle_{x} u_{i}(t, x)=Q_{i}(u) \quad i \geq 1
\label{SmoluchowskiEquation}\end{equation} 
Where $Q_{i}(u)=Q_{g,i}(u)-Q_{l,i}(u)$ is the gain term minus the loss term which  are defined in this way:\\

$$
\begin{aligned}
&Q_{g, i}=\frac{1}{2} \sum_{j=1}^{i-1} a_{i-j, j} u_{i-j} u_{j} \\
&Q_{l, i}=u_{i} \sum_{j=1}^{\infty} a_{i, j} u_{j}
\end{aligned}
$$
$Q_{g,i}$ describes the increasing of concentrations of the cluster of length $i$,
$Q_{l,i}$ describes the depletion of the polymers of size $i$.
An important role is played by the coagulation coefficients $a_{i-i,j}$ which describe a situation where a polymer $i-j$ coagulates with a polymer of length $i$ to form one of length $j$, they are symmetric, indeed $a_{i,j}=a_{j,i}$.\\
The \eqref{SmoluchowskiEquation} is non linear and of infinite dimension: the existence and uniqueness of the solution is not guaranteed by the theory of reaction and diffusion equation. The existence of the solution as a matter of facts depends on the choice of the coagulation coefficients, of which an explicit expression is given in Bretsch Et Al. \\
If there are no sources the total mass of the clusters will be conserved, but actually this is not true. In fact a particular choice of the coefficients produces a non-conservation of the total mass due to the appearance of an infinite cluster called gel, which is born from the growth of longer and longer clusters (this is known as gelation phenomenon), but we will not discuss it in this paper.
\section{The mathematical model for Alzheimer Disease}
As mentioned above, inside the brain there exists a protein called \(\beta\) amyloid peptide ($\mathrm{A} \beta$). Healthy brains produce this protein in monomeric form, but when the mechanism breaks down \(\beta\) starts to auto agglomerate and to form senile plaques. The toxic polymers are the intermediate oligomeric formulation which causes the death of neurons. However, the causes of the AD are not yet completely known, researchers try to study this phenomenon, but the problem is that these intermediate polymers decay very soon in other longer ones. From experiments, it's not easy to study the process of coagulation, therefore the mathematical model helps us to study this process. An assumption that will be made is that 'large' assemblies do not aggregate with each other.
\subsection{Model for the brain}
\begin{figure}[H]
  \centering
  \scalebox{.5}{\incfig{domain_empty}}
  \caption{The domain $\Omega$ representing the cerebral tissue.}
  \label{fig:domain}
\end{figure}
The portion of cerebral tissue considered  in the following is represented by a bounded smooth region $\Omega_{0}\subset \mathbb{R}^3$, whereas the neurons are represented by a family of regular region $\Omega_{j}$ such that:
\begin{enumerate}[label=(\roman*)]
    \item $\bar\Omega_{j}\subset \Omega_{0}$  if   $j=1,2,\dots,\bar{M}$,
    \item $\bar\Omega_{i}\cap \bar\Omega_{j}= \emptyset$  if $i\neq j$.
\end{enumerate}
Let 
$$
\Omega := \Omega_{0} \setminus
\bigcup_{j=1}^{\bar M} \bar\Omega_{j},
$$

\noindent and consider the vector-valued function $u=(u_1,\dots, u_M)$, where $u_j=u_j(t,x)$ $(t\geq 0, t\in \mathbb{R}$ and $x\in \Omega)$ is the molar concentration at the point $x$ and at the time $t$ of a \(\beta\) assembly of $j$ monomers, while $u_M$ takes into account the aggregation of more than $M-1$ monomers. With the definition of $u_M$ it is assumed that `large' assemblies do not aggregate to each other.
Here's a brief presentation of the model made for this problem. Three different stages of the aggregation of \(\beta\) will be considered: its concentration can be modeled with $u_{i}$ with $1\leq i \leq m < M$.
The $u_{1}$ describes the concentration of monomers with the following RDE: 
$$
\frac{\partial u_{1}}{\partial t}(t, x)-d_{1} \Delta_{x} u_{1}(t, x)+u_{1}(t, x) \sum_{j=1}^{M} a_{1, j} u_{j}(t, x)=0.
$$
There is no gain term on the right-hand side because it's impossible for two monomers to coagulate and form another monomer. The loss term is given by the sum of all the possible coagulation of the monomers. 
To solve this equation boundary conditions and initial conditions are necessary.
On the external boundary homogeneous Neumann conditions are assumed, which is meant to artificially isolate the portion of tissue considered from the surrounding environment:
$$
\frac{\partial u_1}{\partial v} \equiv \nabla_{x} u_1\cdot n=0 \quad \text { on }[0, T] \times \partial \Omega.$$
In this way the flux on external fixed boundary is equal to zero, essentially isolating the portion of cerebral tissue from the rest of the world.
Boundary conditions are also needed on the boundaries of the neurons $\partial\Omega_{j}$:
$$ 
\frac{\partial u_{1}}{\partial v} \equiv \nabla_{x} u_1 \cdot n= \psi_{j} \quad \text { on } \partial\Omega_{j}.
$$
This physically models the fact that the neurons produce \(\beta\) in monomeric form. The function $\psi$ is a smooth function and it represents the quantity of \(\beta\) which is produced by the membrane of the neuron, it is a given of the problem.\\
The model considers a portion of the cerebral tissue with a bounded open set $\Omega$ in $\mathbb{R}^{3}$ with a smooth boundary $\partial \Omega$ as seen in Figure \ref{fig:domain}. The neurons in the model will be represented as holes in this domain which are distributed periodically and which have characteristic dimension $0<\epsilon<1$. The set of the domain with the holes is called the perforated domain.
It is assumed that the membrane of the neurons (holes) produces $\mathrm{A} \beta$ and that at time $t=0$ there is no production of  $\mathrm{A} \beta$ from the holes.
Let $Y$ be the unit periodicity cell $\left[0,1\left[{ }^{3}\right.\right.$ having the paving property, shown in Figure \ref{fig:unit_cell}. 
Also, denote by $T$ an open subset of $Y$ with a smooth boundary $\Gamma$, such that $\overline{T} \subset \operatorname{Int} Y$, this means that $T$ can not intersect the boundary of the set $Y$, which is the boundary of the cell. Now call with $Y^{*}=Y \backslash T$ the material part, the "solid part" of the brain. 
\begin{figure}[H]
    \centering
    \scalebox{.5}{\incfig{cell}}
    \caption{The unit periodicity cell $Y$ with the set $T$ of the interior of the hole, bounded by \(\Gamma\)}
    \label{fig:unit_cell}
  \end{figure}

Starting from this it is possible to create a perforated domain: perforate $\Omega$ by removing from it a set $T_{\epsilon}$ of periodically distributed holes defined as before. The set $T$ represents a generic neuron, and $Y^{*}$ the supporting cerebral tissue. Then define $\tau(\epsilon \overline{T})$ to be the set of all translated images $\epsilon \overline{T}$ of the form $\epsilon(k+\overline{T}), k \in \mathbb{Z}^{3}$. Then,

\[ T_{\epsilon} \coloneqq \Omega \cap \tau(\epsilon \overline{T}) . \] is the set of all the holes in the domain.

Introduce now the periodically perforated domain $\Omega_{\epsilon}$ defined by

$$
\Omega_{\epsilon} \coloneqq \Omega \backslash \bar{T}_{\epsilon},
$$
which represent all the set outside the holes. The domain $\Omega_{\epsilon}$ is shown in Figure \ref{fig:perforated_domain}.
\begin{figure}[H]
    \centering
    \scalebox{.5}{\incfig{drawing}}
    \caption{The perforated domain $\Omega_\epsilon$ representing the cerebral tissue. The neurons are represented by the holes.}
    \label{fig:perforated_domain}
  \end{figure}

Ensure that there exists a ``security zone'' where the neurons cannot touch the boundary of \(\Omega\), this is necessary to extend all the functions which live in the solid part also in the holes.
\begin{equation}
  \exists \delta>0 \text { such that dist }\left(\partial\Omega, T_{\epsilon}\right)\geq\delta.
\label{eq:security_zone}
\end{equation}
Therefore can be assumed that $\Omega_{\epsilon}$ is connected. In this scenario there are two boundaries, an internal one $\Gamma_{\epsilon}$, defined by:
$$
\Gamma_{\epsilon} \coloneqq \bigcup\left\{\partial(\epsilon(k+\bar{T})) \mid \epsilon(k+\bar{T}) \subset \Omega\right\}
$$
and an external one that is the fixed exterior boundary denoted by $\partial \Omega$:
$$
\partial\Omega_{\epsilon}\coloneqq\partial\Omega+\Gamma_{\epsilon}.
$$
It is also known that:
\begin{equation}
  \lim _{\epsilon \rightarrow 0} \epsilon\left|\Gamma_{\epsilon}\right|_{N-1}=|\Gamma|_{N-1} \frac{|\Omega|_{N}}{|Y|_{N}}
\label{eq:limit_gamma_eps}
\end{equation}
where $|\cdot|_{N}$ is the $N$-dimensional Hausdorff measure.

The aim is to pass from the microscopic scale to the macroscopic, that is, looking at the brain from 'very high' and we obtain it by performing $\epsilon \rightarrow 0$ which is the scale where clinical data exists. This process is called homogenization theory, and it is a mathematical technique that gives us some mathematical tools to perform this sort of average.
\subsection{The equations}
The system that describes the evolution of monomers is the following:
\begin{equation}
    \begin{dcases}
    \frac{\partial u_{1}^{\epsilon}}{\partial t}(t, x)-d_{1} \Delta_{x} u_{1}^{\epsilon}(t, x)+u_{1}^{\epsilon}(t, x) \sum_{j=1}^{M} a_{1, j} u_{j}^{\epsilon}(t, x)=0, & \text { in }[0, T] \times \Omega_{\epsilon},\\
    \frac{\partial u^{\epsilon}_1}{\partial v} \equiv \nabla_{x} u^{\epsilon}_1\cdot n=0, & \text { on }[0, T]\times\partial\Omega,\\
    \frac{\partial u^{\epsilon}_{1}}{\partial v} \equiv \nabla_{x} u^{\epsilon}_1 \cdot n=\epsilon \psi\left(t, x,\frac{x}{\epsilon}\right), & \text { on }[0, T] \times \Gamma_{\epsilon},\\
    u_{1}^{\epsilon}(0, x)=U_{1}, & \text{ in }\Omega_{\epsilon}.
    \end{dcases}
\label{eq:u_1_eqs}\end{equation}
Note that the $\epsilon$ is put in front of $\psi$ to prevent the divergence of the integral in a further passage, in order to avoid singularities. 
The variable $x$ is called the slow variable and the dependence between $\psi$ and $x$ models the evolution of the disease at a macroscopic scale. The variable
$\frac{x}{\epsilon}$ is called the fast scale variable, which represents the microscopic scale because the main variation of our function are on the scale of the neurons, it's in this small scale that we have a very big change.\\
From our assumptions we suppose that for $t>0$ the brain becomes sick. For technically reasons we have to assume some regularity on $\psi$ and $U_1$:
\begin{enumerate}
    \item $\psi\left(t, x, \frac{x}{\epsilon}\right) \in C^{1}(0, T ; B)$ with $B=C^{1}\left[\bar{\Omega} ; C_{\text {\# }}^{1}(Y)\right]$, where $C_{\#}^{1}(Y)$ is the subset of $C^{1}\left(\mathbb{R}^{N}\right)$ of $Y$-periodic functions;
    \item $\psi\left(t=0, x, \frac{x}{\epsilon}\right)=0$
    \item 
$U_{1}$ is a positive constant such that
\begin{equation}
  U_{1} \leq\|\psi\|_{L^{\infty}(0, T ; B)} .
\label{eq:U_1_norm}\end{equation}
\end{enumerate}
Behind these properties $\psi $ is a generic given function that should be specified in some way if one wants to make the model applicable. We will give an explicit example of it in the last part of the paper.\\
Now we will describe the evolution of oligomers: $1<m<M$. The unknown here is the concentration of a oligomer of generic length $m$
\begin{equation}
    \begin{dcases}
        \frac{\partial u_{m}^{\epsilon}}{\partial t}(t, x)-d_{m} \Delta_{x} u_{m}^{\epsilon}(t, x)+u_{m}^{\epsilon}(t, x) \sum_{j=1}^{M} a_{m, j} u_{j}^{\epsilon}(t, x)= \frac{1}{2} \sum_{j=1}^{m-1} a_{j, m-j} u_{j}^{\epsilon} u_{m-j}^{\epsilon}, & \text { in }[0, T] \times \Omega_{\epsilon},\\
        \frac{\partial u_{m}^{\epsilon}}{\partial v} \equiv \nabla_{x} u_{m}^{\epsilon} \cdot n=0,  & \text { on }[0, T]\times\partial\Omega, \\
        \frac{\partial u_{m}^{\epsilon}}{\partial v} \equiv \nabla_{x} u_{m}^{\epsilon} \cdot n=0, & \text { on }[0, T]\times\Gamma_{\epsilon}.\\ 
     u_{m}^{\epsilon}(0, x)=0 & \text{ in } \Omega_{\epsilon}
    \end{dcases}
\label{eq:u_m_eqs}\end{equation}
In this case we assume homogeneous boundary conditions also on $\Gamma_{\epsilon}$ because we assume that the neurons can't produce \(\beta\) in oligomeric form.
For what it concerns $u_{M}$, it describes the sum of all the densities of all large assemblies of \(\beta\). We are able to do that because in reality when we have a large polymer of \(\beta\) it doesn't  coagulate anymore. 
\begin{equation}
    \begin{dcases}
        \frac{\partial u_{M}^{\epsilon}}{\partial t}(t, x)-d_{M} \Delta_{x} u_{M}^{\epsilon}(t, x)= \frac{1}{2} \sum_{\mathclap{\substack{j+k\geq M\\k,j<M}}} a_{j, k} u_{j}^{\epsilon} u_{k}^{\epsilon}, & \text { in }[0, T] \times \Omega_{\epsilon},\\
        \frac{\partial u_{M}^{\epsilon}}{\partial v} \equiv \nabla_{x} u_{M}^{\epsilon} \cdot n=0,  & \text { on }[0, T]\times\partial\Omega,
        \\
        \frac{\partial u_{M}^{\epsilon}}{\partial v} \equiv \nabla_{x} u_{M}^{\epsilon} \cdot n=0, & \text { on }[0, T]\times\Gamma_{\epsilon},\\ 
        u_{M}^{\epsilon}(0, x)=0, & \text{ in }\Omega_{\epsilon}.
    \end{dcases}
    \label{eq:u_M_eqs}
\end{equation}
In this last system there isn't the loss term because we are assuming that large assemblies do not coagulate with each other.
At this point we want to go from macroscopic to microscopic, trying to compute a sort of average process. Indeed, our aim is to perform the homogenization on the set \eqref{eq:u_1_eqs}-\eqref{eq:u_m_eqs} of equations as $\epsilon \rightarrow 0$, however there is not a clear notion of convergence for the sequence $u_j^{\epsilon}$ $1\geq j \geq M$ which is defined on a varying set $\Omega_{\epsilon}$, which also is random perforated.

\begin{theorem} 
    If $\epsilon>0$, the system \eqref{eq:u_1_eqs}-\eqref{eq:u_m_eqs} has a unique solution
    $$
    \left(u_{1}^{\epsilon}, \ldots, u_{M}^{\epsilon}\right) \in C^{1+\alpha / 2,2+\alpha}\left([0, T] \times \Omega_{\epsilon}\right) \quad(\alpha \in(0,1))
    $$
    such that
    $$
    u_{j}^{\epsilon}(t, x)>0 \text { for }(t, x) \in(0, T) \times \Omega_{\epsilon}, j=1, \ldots, M .
    $$
    \label{thm:3.1}
\end{theorem}


%-----------------------------------------------------------------------------
% CONCLUSION
%-----------------------------------------------------------------------------


%---------------------------------------------------------------------------
%  BIBLIOGRAPHY
%---------------------------------------------------------------------------
%\newpage
% Remember to insert here only the essential bibliography of your work
\bibliography{bibliography.bib} % automatically inserted and ordered with this command

\end{document}
