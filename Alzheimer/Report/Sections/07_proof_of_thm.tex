\section{State and proof of the Main Theorem}
\begin{theorem} Let $u_{m}^{\epsilon}(t, x)(1 \leq m \leq M)$ be a family of classical solutions to problems \eqref{eq:u_1_eqs}-\eqref{eq:u_M_eqs}. The sequences $\widetilde{u_{m}^{\epsilon}}$ and $\widetilde{\nabla_{x} u_{m}^{\epsilon}}(1 \leq m \leq M)$ two-scale converge to: $\left[\chi(y) u_{m}(t, x)\right]$ and $\left[\chi(y)\left(\nabla_{x} u_{m}(t, x)+\nabla_{y} u_{m}^{1}(t, x, y)\right)\right](1 \leq m \leq M)$, respectively, where tilde denotes the extension by zero outside $\Omega_{\epsilon}$ and $\chi(y)$ represents the characteristic function of $Y^{*}$. The limiting functions $\left(u_{m}(t, x), u_{m}^{1}(t, x, y)\right)(1 \leq m \leq M)$ are the unique solutions in $L^{2}\left(0, T ; H^{1}(\Omega)\right) \times L^{2}\left([0, T] \times \Omega ; H_{\#}^{1}(Y) / \mathbb{R}\right)$ of the following two-scale homogenized systems.
If $m=1$:
\begin{equation}
  \begin{dcases}
   \begin{split}
     \theta \frac{\partial u_{1}}{\partial t}(t, x)-\operatorname{div}_{x}\left[d_{1} A \nabla_{x} u_{1}(t, x)\right] \\  +\theta u_{1}(t, x) \sum_{j=1}^{M} a_{1, j} u_{j}(t, x) \\ =d_{1} \int_{\Gamma} \psi(t, x, y) \, d  \sigma(y), 
   \end{split} & \text { in }[0, T]\times\Omega, \\
    {\left[A \nabla_{x} u_{1}(t, x)\right] \cdot n=0}, & \text { on }[0, T]\times\partial\Omega, \\ 
    u_{1}(0, x)=U_{1}, & \text { in }\Omega. \\ 
\end{dcases}
\label{eq 16}
\end{equation}
If $1<m<M$:
\begin{equation}
  \begin{dcases}
    \begin{split}
        \theta \frac{\partial u_{m}}{\partial t}(t, x)-\operatorname{div}_{x}\left[d_{m} A \nabla_{x} u_{m}(t, x)\right] \\ +\theta u_{m}(t, x) \sum_{j=1}^{M} a_{m, j} u_{j}(t, x) & \\ =\frac{\theta}{2} \sum_{j=1}^{m-1} a_{j, m-j} u_{j}(t, x) u_{m-j}(t, x), 
    \end{split}& \text { in }[0, T]\times\Omega,\\ 
    {\left[A \nabla_{x} u_{m}(t, x)\right] \cdot n=0}, & \text { on }[0, T]\times\partial\Omega, \\ 
    u_{m}(0, x)=0, & \text { in }\Omega.
\end{dcases}
\label{eq 17}\end{equation}
If $m=M$:
\begin{equation}
  \begin{dcases}
    \begin{split}
        \theta \frac{\partial u_{M}}{\partial t}(t, x)-d i v_{x}\left[d_{M} A \nabla_{x} u_{M}(t, x)\right] & \\ =\frac{\theta}{2} \sum_{\substack{j+k \geq M \\ j,k<M}} a_{j, k} u_{j}(t, x) u_{k}(t, x),
    \end{split} & \text { in }[0, T]\times\Omega,\\ 
    {\left[A \nabla_{x} u_{M}(t, x)\right] \cdot n=0}, & \text { on }[0, T]\times\partial\Omega, \\ 
    u_{M}(0, x)=0,& \text { in }\Omega.
\end{dcases}
\label{eq 18}\end{equation}
where
$$
\begin{aligned}
u_{m}^{1}(t, x, y) &=\sum_{i=1}^{N} w_{i}(y) \frac{\partial u_{m}}{\partial x_{i}}(t, x) \quad(1 \leq m \leq M), \\
\theta &=\int_{Y} \chi(y) \, d  y=\left|Y^{*}\right|
\end{aligned}
$$
is the volume fraction of material, and $A$ is a matrix with constant coefficients defined by
\begin{equation*}
A_{i j}=\int_{Y^{*}}\left(\nabla_{y} w_{i}+\hat{e}_{i}\right) \cdot\left(\nabla_{y} w_{j}+\hat{e}_{j}\right) \, d  y,
\end{equation*}
with  $\hat{e}_{i}$ being the $ith$ unit vector in $\mathbb{R}^n$ and $(w_i)_1\leq i \leq N$ the family of solutions of the cell problem
\begin{equation}
    \begin{dcases}
        -\operatorname{div}_{y}\left[\nabla_{y} w_{i}+\hat{e}_{i}\right]=0, & \text { in } Y^{*}, \\
        \left(\nabla_{y} w_{i}+\hat{e}_{i}\right) \cdot n=0, & \text { on } \Gamma, \\
        y \rightarrow w_{i}(y), & Y-\text {periodic}.
    \end{dcases}
    \label{eq 19}
\end{equation}
\label{thm 5.1}
\end{theorem}
\begin{proof}
In view of Lemmas \eqref{lemma 5.1}-\eqref{lemma 5.2} and \eqref{lemma 5.4}-\eqref{lemma 5.8}, the sequences $\widetilde{u_{m}^{\epsilon}}$ and $\widetilde{\nabla_{x} u_{m}^{\epsilon}}(1 \leq m \leq M)$ are bounded in $L^{2}([0, T] \times \Omega)$, and by application of Theorem \eqref{thm 7.1} and Theorem \eqref{thm 7.3}, and so:
\begin{equation}
\widetilde{u_{m}^{\epsilon}}  \overset{2s}{\rightharpoonup}
\left[\chi(y) u_{m}(t, x)\right],
\end{equation}
\begin{equation}
\widetilde{\nabla_{x} u_{m}^{\epsilon}}
\overset{2s}{\rightharpoonup}
\left[\chi(y)\left(\nabla_{x} u_{m}(t, x)+\right.\right.\left.\left.\nabla_{y} u_{m}^{1}(t, x, y)\right)\right], \quad (1 \leq m \leq M).
\end{equation}
Similarly, in view of Lemma \eqref{lemma 5.9}, it is possible to prove that
\begin{equation}
\left(\widetilde{\frac{\partial u_{m}^{\epsilon}}{\partial t}}\right)
\overset{2s}{\rightharpoonup}
\left[\chi(y) \frac{\partial u_{m}}{\partial t}(t, x)\right],  \quad (1 \leq m \leq M).
\end{equation}
One can now find the homogenized equations satisfied by $u_{m}(t, x)$ and $u_{m}^{1}(t, x, y)$ $(1 \leq m \leq M)$.
In the case $m=1$, multiply the first equation of \eqref{eq:u_1_eqs} by the test function to obtain the weak formulation of the problem
$$
\phi_{\epsilon} \equiv \phi(t, x)+\epsilon \phi_{1}\left(t, x, \frac{x}{\epsilon}\right),
$$
where $\phi \in C^{1}([0, T] \times \bar{\Omega})$ and $\phi_{1} \in C^{1}\left([0, T] \times \bar{\Omega} ; C_{\#}^{\infty}(Y)\right)$. 
\begin{equation*}
    \frac{\partial u_{1}^{\epsilon}}{\partial t}\phi_{\epsilon}-\operatorname{div}\left(d_{1} \nabla_{x} u_{1}^{\epsilon}\right)\phi_{\epsilon}+u_{1}^{\epsilon} \sum_{j=1}^{M} a_{1, j} u_{1}^{\epsilon}\phi_{\epsilon}=0.
\end{equation*}
Integrating, the divergence theorem yields
\begin{equation}
  \begin{aligned}
&\int_{0}^{T} \int_{\Omega_{\epsilon}} \frac{\partial u_{1}^{\epsilon}}{\partial t} \phi_{\epsilon}\left(t, x, \frac{x}{\epsilon}\right) \, d  t \, d  x+d_{1} \int_{0}^{T} \int_{\Omega_{\epsilon}} \nabla_{x} u_{1}^{\epsilon} \cdot \nabla \phi_{\epsilon} \, d  t \, d  x \\
&+\int_{0}^{T} \int_{\Omega_{\epsilon}} u_{1}^{\epsilon} \sum_{j=1}^{M} a_{1, j} u_{j}^{\epsilon} \phi_{\epsilon} \, d  t \, d  x=\epsilon d_{1} \int_{0}^{T} \int_{\Gamma_{\epsilon}} \psi\left(t, x, \frac{x}{\epsilon}\right) \phi_{\epsilon} \, d  t \, d  \sigma_{\epsilon}(x).
\end{aligned}
\label{eq 86}\end{equation}
Passing to the two-scale limit, one can get
\begin{equation}
\begin{aligned}
&\int_{0}^{T} \int_{\Omega} \int_{Y^{*}} \frac{\partial u_{1}}{\partial t}(t, x) \phi(t, x) \, d  t \, d  x \, d  y \\
&\quad+d_{1} \int_{0}^{T} \int_{\Omega} \int_{Y^{*}}\left[\nabla_{x} u_{1}(t, x)+\nabla_{y} u_{1}^{1}(t, x, y)\right] \cdot\left[\nabla_{x} \phi(t, x)+\nabla_{y} \phi_{1}(t, x, y)\right] \, d  t \, d  x \, d  y+\\&+\int_{0}^{T} \int_{\Omega} \int_{Y^{*}} u_{1}(t, x) \sum_{j=1}^{M} a_{1, j} u_{j}(t, x) \phi(t, x) \, d  t \, d  x \, d  y= d_{1} \int_{0}^{T} \int_{\Omega} \int_{\Gamma} \psi(t, x, y) \phi(t, x) \, d  t \, d  x \, d  \sigma(y).
\end{aligned}
\label{eq 87}\end{equation}
The last term on the left-hand side of \eqref{eq 87} has been obtained by using Theorem \eqref{thm 7.2}, while the term on the right-hand side has been attained by application of Theorem \eqref{thm 7.5}. An integration by parts shows that \eqref{eq 87} is a variational formulation associated with the following homogenized system:

\begin{equation}
  -\operatorname{div}_{y}\left[d_{1}\left(\nabla_{x} u_{1}(t, x)+\nabla_{y} u_{1}^{1}(t, x, y)\right)\right]=0, \quad \text { in }[0, T] \times \Omega \times Y^{*},
\label{eq 88}\end{equation}
\begin{equation}
 {\left[\nabla_{x} u_{1}(t, x)+\nabla_{y} u_{1}^{1}(t, x, y)\right] \cdot n=0, \quad \text { on }[0, T] \times \Omega \times \Gamma},
\label{eq 89}\end{equation}
\begin{equation}
\begin{aligned}
&\theta \frac{\partial u_{1}}{\partial t}(t, x)-\operatorname{div}_{x}\left[d_{1} \int_{Y^{*}}\left(\nabla_{x} u_{1}(t, x)+\nabla_{y} u_{1}^{1}(t, x, y)\right) \, d  y\right] \\&+\theta u_{1}(t, x) \sum_{j=1}^{M} a_{1, j} u_{j}(t, x)-d_{1} \int_{\Gamma} \psi(t, x, y) \, d  \sigma(y)=0, \quad \text { in }[0, T] \times \Omega,
\end{aligned}
\label{eq 90}\end{equation}

\begin{equation}
 {\left[\int_{Y^{*}}\left(\nabla_{x} u_{1}(t, x)+\nabla_{y} u_{1}^{1}(t, x, y)\right) \, d  y\right] \cdot n=0, \quad \text { on }[0, T] \times \partial \Omega}.
\label{eq 91}\end{equation}
To conclude, by continuity, one can have that
$$
u_{1}(0, x)=U_{1} \quad \text { in } \Omega .
$$
Taking advantage of the constancy of the diffusion coefficient $d_{1}$, Eqs. \eqref{eq 88} and \eqref{eq 89} can be rewritten as follows
\begin{equation}
\Delta_{y} u_{1}^{1}(t, x, y)=0, \quad \text { in }[0, T] \times \Omega \times Y^{*},
\label{eq 92}\end{equation}
\begin{equation}
 \nabla_{y} u_{1}^{1}(t, x, y) \cdot n=-\nabla_{x} u_{1}(t, x) \cdot n, \quad \text { on }[0, T] \times \Omega \times \Gamma.
\label{eq 93}\end{equation}
Then, $u_{1}^{1}(t, x, y)$ satisfying (92)-(93) can be written as
\begin{equation}
 u_{1}^{1}(t, x, y)=\sum_{i=1}^{N} w_{i}(y) \frac{\partial u_{1}}{\partial x_{i}}(t, x),
\label{eq 94}\end{equation}
where $\left(w_{i}\right)_{1 \leq i \leq N}$ is the family of solutions of the cell problem
\begin{equation}
 \begin{cases}-\operatorname{div}_{y}\left[\nabla_{y} w_{i}+\hat{e}_{i}\right]=0, & \text { in } Y^{*}, \\ \left(\nabla_{y} w_{i}+\hat{e}_{i}\right) \cdot n=0, & \text { on } \Gamma, \\ y \rightarrow w_{i}(y), & Y-\text {periodic}. \end{cases}
\label{eq 95}\end{equation}
By using the relation \eqref{eq 94} in Eqs. \eqref{eq 90} and \eqref{eq 91}, it is possible to get
\begin{equation}
 \begin{split}
    \theta \frac{\partial u_{1}}{\partial t}(t, x)-\operatorname{div}_{x}\left[d_{1} A \nabla_{x} u_{1}(t, x)\right]+\theta u_{1}(t, x) \sum_{j=1}^{M} a_{1, j} u_{j}(t, x) \\
   -d_{1} \int_{\Gamma} \psi(t, x, y) \, d  \sigma(y)=0, \quad \text { in }[0, T] \times \Omega,
 \end{split}
\label{eq 96}\end{equation}
\begin{equation}
 \left[A \nabla_{x} u_{1}(t, x)\right] \cdot n=0, \quad \text { on }[0, T] \times \partial \Omega,
\label{eq 97}\end{equation}
where $A$ is a matrix with constant coefficients defined by
$$
A_{i j}=\int_{Y^{*}}\left(\nabla_{y} w_{i}+\hat{e}_{i}\right) \cdot\left(\nabla_{y} w_{j}+\hat{e}_{j}\right) \, d  y .
$$
In the case $1<m<M$, multiply the first equation of \eqref{eq:u_m_eqs} by the test function
$$
\phi_{\epsilon} \equiv \phi(t, x)+\epsilon \phi_{1}\left(t, x, \frac{x}{\epsilon}\right)
$$
where $\phi \in C^{1}([0, T] \times \bar{\Omega})$ and $\phi_{1} \in C^{1}\left([0, T] \times \bar{\Omega} ; C_{\#}^{\infty}(Y)\right)$. Following the same arguments as before: integrating, and using the divergence theorem 
\begin{equation}
 \begin{aligned}
&\int_{0}^{T} \int_{\Omega_{\epsilon}} \frac{\partial u_{m}^{\epsilon}}{\partial t} \phi_{\epsilon}\left(t, x, \frac{x}{\epsilon}\right) \, d  t \, d  x+d_{m} \int_{0}^{T} \int_{\Omega_{\epsilon}} \nabla_{x} u_{m}^{\epsilon} \cdot \nabla \phi_{\epsilon} \, d  t \, d  x \\
&\quad+\int_{0}^{T} \int_{\Omega_{\epsilon}} u_{m}^{\epsilon} \sum_{j=1}^{M} a_{m, j} u_{j}^{\epsilon} \phi_{\epsilon} \, d  t \, d  x \\
&=\frac{1}{2} \int_{0}^{T} \int_{\Omega_{\epsilon}} \sum_{j=1}^{m-1} a_{j, m-j} u_{j}^{\epsilon} u_{m-j}^{\epsilon} \phi_{\epsilon} \, d  t \, d  x.
\end{aligned}
\label{eq 98}\end{equation}
Passing to the two-scale limit, one can get 
\begin{equation}
 \begin{aligned}
\int_{0}^{T} & \int_{\Omega} \int_{Y^{*}} \frac{\partial u_{m}}{\partial t}(t, x) \phi(t, x) \, d  t \, d  x \, d  y \\
&+d_{m} \int_{0}^{T} \int_{\Omega} \int_{Y^{*}}\left[\nabla_{x} u_{m}(t, x)+\nabla_{y} u_{m}^{1}(t, x, y)\right] \cdot\left[\nabla_{x} \phi(t, x)+\nabla_{y} \phi_{1}(t, x, y)\right] \, d  t \, d  x \, d  y \\
&+\int_{0}^{T} \int_{\Omega} \int_{Y^{*}} u_{m}(t, x) \sum_{j=1}^{M} a_{m, j} u_{j}(t, x) \phi(t, x) \, d  t \, d  x \, d  y \\
&=\frac{1}{2} \int_{0}^{T} \int_{\Omega} \int_{Y^{*}} \sum_{j=1}^{m-1} a_{j, m-j} u_{j}(t, x) u_{m-j}(t, x) \phi(t, x) \, d  t \, d  x \, d  y.
\end{aligned}
\label{eq 99}\end{equation}
The last term on the left-hand side of \eqref{eq 99} and the term on the right-hand side have been obtained by using Theorem \eqref{thm 7.1}. An integration by parts shows that \eqref{eq 99} is a variational formulation associated with the following homogenized system:
\begin{equation}
 -\operatorname{div}_{y}\left[d_{m}\left(\nabla_{x} u_{m}(t, x)+\nabla_{y} u_{m}^{1}(t, x, y)\right)\right]=0, \quad \text { in }[0, T] \times \Omega \times Y^{*},
\label{eq 100}\end{equation}
\begin{equation}
 \left[\nabla_{x} u_{m}(t, x)+\nabla_{y} u_{m}^{1}(t, x, y)\right] \cdot n=0, \quad \text { on }[0, T] \times \Omega \times \Gamma,
\label{eq 101}\end{equation}
\begin{equation}
\begin{aligned}
&\theta \frac{\partial u_{m}}{\partial t}(t, x)-\operatorname{div}_{x}\left[d_{m} \int_{Y^{*}}\left(\nabla_{x} u_{m}(t, x)+\nabla_{y} u_{m}^{1}(t, x, y)\right) \, d  y\right] \\
&\quad+\theta u_{m}(t, x) \sum_{j=1}^{M} a_{m, j} u_{j}(t, x) \\
&\quad-\frac{\theta}{2} \sum_{j=1}^{m-1} a_{j, m-j} u_{j}(t, x) u_{m-j}(t, x)=0, \quad \text { in }[0, T] \times \Omega,
\end{aligned}
\label{eq 102}\end{equation}
\begin{equation}
 \left[\int_{Y^{*}}\left(\nabla_{x} u_{m}(t, x)+\nabla_{y} u_{m}^{1}(t, x, y)\right) \, d  y\right] \cdot n=0, \quad \text { on }[0, T] \times \partial \Omega.
\label{eq 103}\end{equation}
Moreover, by continuity
$$
u_{m}(0, x)=0 \quad \text { in } \Omega .
$$
Taking advantage of the constancy of the diffusion coefficient $d_{m}$, Eqs. \eqref{eq 100} and \eqref{eq 101} can be rewritten as follows
\begin{equation}
 \Delta_{y} u_{m}^{1}(t, x, y)=0, \quad \text { in }[0, T] \times \Omega \times Y^{*},
\label{eq 104}\end{equation}
\begin{equation}
 \nabla_{y} u_{m}^{1}(t, x, y) \cdot n=-\nabla_{x} u_{m}(t, x) \cdot n, \quad \text { on }[0, T] \times \Omega \times \Gamma.
\label{eq 105}\end{equation}
Then, $u_{m}^{1}(t, x, y)$ satisfying \eqref{eq 104}-\eqref{eq 105} can be written as 
\begin{equation}
 u_{m}^{1}(t, x, y)=\sum_{i=1}^{N} w_{i}(y) \frac{\partial u_{m}}{\partial x_{i}}(t, x),
\label{eq 106}\end{equation}
where $\left(w_{i}\right)_{1 \leq i \leq N}$ is the family of solutions of the cell problem
\begin{equation}
 \begin{dcases}
-\operatorname{div}_{y}\left[\nabla_{y} w_{i}+\hat{e}_{i}\right]=0, & \text { in } Y^{*}, \\
\left(\nabla_{y} w_{i}+\hat{e}_{i}\right) \cdot n=0, & \text { on } \Gamma, \\
y \rightarrow w_{i}(y), & Y-\text {periodic}. 
\end{dcases}
\label{eq 107}\end{equation}
By using the relation \eqref{eq 106} in Eqs. \eqref{eq 102}and \eqref{eq 103}, it is possible to obtain
\begin{equation}
 \begin{aligned}
&\theta \frac{\partial u_{m}}{\partial t}(t, x)-\operatorname{div}_{x}\left[d_{m} A \nabla_{x} u_{m}(t, x)\right]+\theta u_{m}(t, x) \sum_{j=1}^{M} a_{m, j} u_{j}(t, x) \\
&-\frac{\theta}{2} \sum_{j=1}^{m-1} a_{j, m-j} u_{j}(t, x) u_{m-j}(t, x)=0, \quad \text { in }[0, T] \times \Omega. \end{aligned}
\label{eq 108}\end{equation}
\begin{equation}
 \left[A \nabla_{x} u_{m}(t, x)\right] \cdot n=0, \quad \text { on }[0, T] \times \partial \Omega.
\label{eq 109}\end{equation}
where $A$ is a matrix with constant coefficients defined by
$$
A_{i j}=\int_{Y^{*}}\left(\nabla_{y} w_{i}+\hat{e}_{i}\right) \cdot\left(\nabla_{y} w_{j}+\hat{e}_{j}\right) \, d  y .
$$
The proof for the case $m=M$ is achieved by applying exactly the same arguments considered when $1<m<M$.
\end{proof}
Theorem \eqref{thm 5.1} shows that the macroscale (homogenized) model, obtained from Eqs. \eqref{eq:u_1_eqs}-\eqref{eq:u_M_eqs} as $\epsilon \rightarrow 0$, is asymptotically consistent with the original model and resolves both the coarse and the small scale. The information given on the microscale, by the non-homogeneous Neumann boundary condition in \eqref{eq:u_1_eqs}, is transferred into the source term in the first equation of \eqref{eq 16}, describing the limit model. Furthermore, on the macroscale, the geometric structure of the perforated domain induces a correction such that the scalar diffusion coefficients $d_{i}(1 \leq i \leq M)$, defined at the microscale, are replaced by a tensorial quantity with constant coefficients.
