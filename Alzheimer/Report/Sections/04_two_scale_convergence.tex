\section{Two-Scale Convergence}

\begin{definition}[Two-scale convergence]
A sequence of functions $ v ^ {\epsilon } $in $ L^2([ 0 ,T]\times\Omega) $ two-scale converges to $v_{0} \in L^2([ 0 ,T]\times\Omega \times Y)$ if
$$
\lim_{\epsilon \to 0} \int_{0}^{\textrm{T}} \int_{\Omega} v^{\epsilon}(t,x)\psi\left(t,x,\frac{x}{\epsilon}\right) \, dxdt=\int_{0}^{\textrm{T}} \int_{\Omega} \int_{\textrm{Y}} v(t,x,y)\psi(t,x,y) \, dydxdt
$$ 
for all $\psi \in C^1([0,T]\times \bar\Omega;C_{\#}^{\infty}(Y))$
\label{def 7.1}\end{definition}
This definition makes sense because of the next compactness theorem 
\begin{theorem}[Compactness Theorem]
If $v^{\epsilon}$ is a bounded sequence in $ L^2([ 0 ,T]\times\Omega) $, then there exist a function $v_{0}(t,x,y) \in   L^2([0,T]\times\Omega \times Y)$ s.t. $v^{\epsilon}$ two-scale converges to $v_{0}$, we write: 
$v^{\epsilon} \overset{2s}{\rightharpoonup} v_{0}$.
\label{thm 7.1}\end{theorem}
This theorem is important because it shows that the minimal requirement to have two-scale convergence is that $v$ must be bounded. Notice that the two-scale limit contains one more variable with respect to the first one, that is the $y$ variable. It represents the small scale, and it reflects the fluctuations of our model on the microscale.
\begin{remark}[Remark, relation with weak convergence] If we assume that $\psi$ is independent of $y$, and ignoring the dependence on time (since the homogenization is performed on the spatial grid) we obtain:
$$ 
\lim_{\epsilon \to 0} \int_{\Omega} v^{\epsilon}(x)\psi(x)dx= \int_{\Omega} \int_{\textrm{Y}} v(x,y)\psi(x)dxdy=\int_{\Omega} \left[\int_{\textrm{Y}} v(x,y)dy\right]\psi(x)dx.
$$This shows that, in this specific case, the two-scale convergence coincides with the weak convergence.
In this definition the time is only a parameter, and if we assume that $\psi$ does not depend on $y$ we obtain the definition of weak limit. So in this case the weak limit coincides with the two-scale limit. Therefore, the two-scale convergence and the weak convergence are strictly related: the main difference between them is that in the first one the oscillations are captured due to the extra variable $y$.
\end{remark}
