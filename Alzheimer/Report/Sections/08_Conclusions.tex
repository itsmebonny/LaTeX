\section{Conclusions and final remarks}
The model we just presented can be related to the one introduced in \cite{Bertsch}, where the process of diffusion and agglomeration of $\mathrm{A} \beta$ is described on the macroscale by a Smoluchowski system with a source term, coupled with a kinetic-type transport equation for the distribution function of the degree of malfunctioning of neurons that keeps into account the spreading of the disease through a neuron-to-neuron prion-like transmission. At the beginning of our analysis we said that the $\psi(t,x,y)$ is a given for our model. In particular, the mathematical analysis carried out in this paper can be regarded as a formal derivation (that is, neglecting regularity issues), starting from the microscale, of the source term $\mathcal{F}$, introduced in the second equation of (2.16) in \cite{Bertsch}, in order to describe the production of $\mathrm{A} \beta$ in monomeric form by neurons. A comparison between the model Eq. \eqref{eq 16} and the second equation of (2.16) in \cite{Bertsch} allows us to make the following choice of the function $\psi(t, x, y)$ appearing in \eqref{eq 16}:
\begin{equation}
  \psi(t, x, y)=C_{\mathcal{F}} \int_{0}^{1}\left(\mu_{0}+a\right)(1-a) f(x, a, t) \, {d} a g(y)
\label{eq 110}\end{equation}
with
$$
\int_{\Gamma} g(y) \, {d} \sigma(y)=\text { const.}
$$
In Eq. \eqref{eq 110}, the parameter $a \in[0,1]$ describes the degree of malfunctioning of a neuron: $a$ close to 0 stands for 'the neuron is healthy,' whereas $a$ close to 1 for 'the neuron is dead.' Given $x \in \Omega, t \geq 0$ and $a \in[0,1]$,
$$
f(x, a, t) \, {d} a
$$
indicates the fraction of neurons close to $x$ with degree of malfunctioning at time $t$ between $a$ and $a+d a$. Furthermore, the small constant $\mu_{0}>0$ in \eqref{eq 110} accounts for $\mathrm{A} \beta$ production by healthy neurons. In \cite{Bertsch}, an evolution equation for the distribution function $f$ has been proposed:
\begin{equation}
  \partial_{t} f+\partial_{a}(f v[f])=J[f],
\label{eq 111}\end{equation}
where $v=v(x, a, t)$ indicates the deterioration rate of the health state of the neurons. We assume that
\begin{equation}
  v[f]=\iint_{\Omega \times[0,1]} \mathcal{K}(x, a, y, b) f(y, b, t) \, {d} y \, {d} b .
\label{eq 112}\end{equation}
The integral term describes the possible prion-like propagation of AD through the neural pathway. Malfunctioning neighbors are harmful for a neuron's health state, while healthy ones are not:
\begin{align*}
&\mathcal{K}(x, a, y, b) \geq 0, & \forall x, y \in \Omega, a, b \in[0,1], \\
&\mathcal{K}(x, a, y, b)=0, & \text { if } a>b .
\end{align*}
The term $J[f]$ in \eqref{eq 111} accounts for the onset of AD, since it is written in terms of the probability that, in randomly chosen parts of the cerebral tissue, the degree of malfunctioning of neurons randomly jumps to higher values due to external agents or genetic factors. What prevents our homogenization results from being considered a fully rigorous derivation of the source term $\mathcal{F}$, appearing in the second equation of (2.16) in \cite{Bertsch}, is the fact that the solutions of Eq. \eqref{eq 111} do not satisfy, in general, all the regularity properties assumed on $\psi$. However, Eq. \eqref{eq 110} allows us to establish a link, at least formally, between the limit model derived here and the one presented in \cite{Bertsch}, suggesting a possible choice for $\psi$, considered, in the present paper, as a generic given function.
It is worth noting that the plots of $f$, at different times, can be directly compared with medical fluorodeoxyglucose PET images \cite{Bertsch}. The numerical simulations reported in \cite{Bertsch} are in good qualitative agreement with clinical images of the disease distribution in the brain which vary from early to advanced stages.
It's good to notice that the homogenization theory works in every perforated domain so we can follow the previous arguments to homogenize every system in a random perforated domain.