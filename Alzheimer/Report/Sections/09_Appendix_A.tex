\section{On Sobolev spaces}
Recall some theory on Sobolev spaces.
Identify $H^{1}(\Omega)=W^{1,2}(\Omega)$, where the Sobolev space $W^{1, p}(\Omega)$ is defined by
$$
W^{1, p}(\Omega)=\left\{v \mid v \in L^{p}(\Omega), \frac{\partial v}{\partial x_{i}} \in L^{p}(\Omega), i=1, \ldots, N\right\}
$$
and denote by $H_{\#}^{1}(Y)$ the closure of $C_{\#}^{\infty}(Y)$ for the $H^{1}$-norm.
Now, recall that the quotient space of a vector space $V$ and its subspace $U$ is the set denoted with $V/U$ of all equivalence classes in $U$. In particular $v \sim v' \iff v-v'\in U$: two elements of $V$ differ at most to elements that belong to $U$.

\begin{lemma} The following estimate holds: if $v \in \operatorname{Lip}\left(\Omega_{\epsilon}\right)$, then
\begin{equation}
  \|v\|_{L^{2}\left(\Gamma_{\epsilon}\right)}^{2} \leq C_{1}\left[\epsilon^{-1} \int_{\Omega_{\epsilon}}|v|^{2} \, d  x+\epsilon \int_{\Omega_{\epsilon}}\left|\nabla_{x} v\right|^{2} \, d  x\right],
\label{eq 113}\end{equation}
where $C_{1}$ is a constant which does not depend on $\epsilon$.
\label{lemma 7.1}\end{lemma}
The inequality \eqref{eq 113} can be easily obtained from the standard trace theorem by means of a scaling argument (\cite{Allaire_et_al}; \cite{Piat_Piatnitski_2010}; \cite{Piat_Nazarov_Piatnitski_2012}).
\begin{lemma}
  
 Suppose that the domain $\Omega_{\epsilon}$ is such that assumption \eqref{eq:security_zone} is satisfied. Then, there exists a family of linear continuous extension operators

$$
P_{\epsilon}: W^{1, p}\left(\Omega_{\epsilon}\right) \rightarrow W^{1, p}(\Omega)
$$

and a constant $C>0$ independent of $\epsilon$ such that

$$
P_{\epsilon} v=v \quad \text { in } \Omega_{\epsilon}
$$

and
\begin{equation}
  \int_{\Omega}\left|P_{\epsilon} v\right|^{p} \, d  x \leq C \int_{\Omega_{\epsilon}}|v|^{p} \, d  x,
\label{eq 114}\end{equation}
\begin{equation}
  \int_{\Omega}\left|\nabla\left(P_{\epsilon} v\right)\right|^{p} \, d  x \leq C \int_{\Omega_{\epsilon}}|\nabla v|^{p} \, d  x
\label{eq 115}\end{equation}

for each $v \in W^{1, p}\left(\Omega_{\epsilon}\right)$ and for any $p \in(1,+\infty)$.
\label{lemma 7.2}\end{lemma}


As a consequence of the existence of extension operators, one can derive the Sobolev inequalities in $W^{1, p}\left(\Omega_{\epsilon}\right)$ with a constant independent of $\epsilon$.

\begin{lemma} (Anisotropic Sobolev inequalities in perforated domains)

(i) For arbitrary $v \in H^{1}\left(0, T ; L^{2}\left(\Omega_{\epsilon}\right)\right) \cap L^{2}\left(0, T ; H^{1}\left(\Omega_{\epsilon}\right)\right)$ and $q_{1}$ and $r_{1}$ satisfying the conditions
\begin{equation}
  \left\{\begin{array}{l}
\frac{1}{r_{1}}+\frac{N}{2 q_{1}}=\frac{N}{4} \\
r_{1} \in[2, \infty], q_{1} \in\left[2, \frac{2 N}{N-2}\right] \quad \text { for } N>2
\end{array}\right.
\label{eq 116}\end{equation}

the following estimate holds
\begin{equation}
  \|v\|_{L^{r_{1}}\left(0, T ; L^{q_{1}}\left(\Omega_{\epsilon}\right)\right)} \leq c\|v\|_{Q_{\epsilon}(T)}
\label{eq 117}\end{equation}
where $\mathrm{c}$ is a positive constant independent of $\epsilon$ and
\begin{equation}
  \|v\|_{Q_{\epsilon}(T)}^{2}:=\sup _{0 \leq t \leq T} \int_{\Omega_{\epsilon}}|v(t)|^{2} \, d  x+\int_{0}^{T} \, d  t \int_{\Omega_{\epsilon}}|\nabla v(t)|^{2} \, d  x
\label{eq 118}\end{equation}


(ii) For arbitrary $v \in H^{1}\left(0, T ; L^{2}\left(\Omega_{\epsilon}\right)\right) \cap L^{2}\left(0, T ; H^{1}\left(\Omega_{\epsilon}\right)\right)$ and $q_{2}$ and $r_{2}$ satisfying the conditions
\begin{equation}
  \left\{\begin{array}{l}
\frac{1}{r_{2}}+\frac{(N-1)}{2 q_{2}}=\frac{N}{4} \\
r_{2} \in[2, \infty], q_{2} \in\left[2, \frac{2(N-1)}{(N-2)}\right] \quad \text { for } N \geq 3
\end{array}\right.
\label{eq 119}\end{equation}


the following estimate holds
\begin{equation}
  \|v\|_{L^{r_{2}}\left(0, T ; L^{q_{2}}\left(\Gamma_{\epsilon}\right)\right)} \leq c \epsilon^{-\frac{N}{2}-\frac{(1-N)}{q_{2}}}\|v\|_{Q_{\epsilon}(T)}
\label{eq 120}\end{equation}


where $c$ is a positive constant independent of $\epsilon$ and the norm $\|v\|_{Q_{\epsilon}(T)}$ is defined as in \eqref{eq 118}.
\label{lemma 7.3}\end{lemma}
\begin{proof}
(i) The extension Lemma \eqref{lemma 7.2} ensures the well definiteness of a linear continuous extension operator $P_{\epsilon}$ which satisfies \eqref{eq 114} and \eqref{eq 115}. By the classical multiplicative Sobolev inequalities valid in $\Omega$ (see \cite{Ladyzenskaja_Solonnikov_Uralceva_1968} and \cite{Nittka_2011}), one can have that
\begin{equation}
  \left\|P_{\epsilon} v\right\|_{L_{1}^{r_{1}}\left(0, T ; L^{q_{1}(\Omega)}\right.} \leq c_{1}\left\|P_{\epsilon} v\right\|_{Q(T)},
\label{eq 121}\end{equation}


where $c_{1} \geq 0$ depends only on $\Omega, r_{1}, q_{1}$, with $r_{1}$ and $q_{1}$ satisfying the conditions \eqref{eq 116} and
\begin{equation}
  \left\|P_{\epsilon} v\right\|_{Q(T)}^{2}:=\sup _{0 \leq t \leq T} \int_{\Omega}\left|P_{\epsilon} v(t)\right|^{2} \, d  x+\int_{0}^{T} \, d  t \int_{\Omega}\left|\nabla\left(P_{\epsilon} v(t)\right)\right|^{2} \, d  x.
\label{eq 122}\end{equation}

By using \eqref{eq 114},\eqref{eq 115} and \eqref{eq 121}, it is possible to conclude that
\begin{equation}
  \begin{aligned}
\|v\|_{L^{r_{1}}\left(0, T ; L^{q_{1}}\left(\Omega_{\epsilon}\right)\right)} & \leq C^{\prime}\left\|P_{\epsilon} v\right\|_{L^{r_{1}}\left(0, T ; L^{q_{1}}(\Omega)\right)} \\
& \leq C^{\prime} c_{1}\left\|P_{\epsilon} v\right\|_{Q(T)} \leq C^{\prime} c_{1} C\|v\|_{Q_{\epsilon}(T)},
\end{aligned}
\label{eq 123}\end{equation}


where $c:=C^{\prime} c_{1} C$ is independent of $\epsilon$.

(ii) eRwrite the anisotropic Sobolev inequality valid on $\partial \Omega$ (see \cite{Ladyzenskaja_Solonnikov_Uralceva_1968} and \cite{Nittka_2011}):
\begin{equation}
  \begin{aligned}
&{\left[\int_{0}^{T} \, d  t\left[\int_{\partial \Omega}|v(t)|^{q_{2}} d \mathcal{H}^{N-1}\right]^{\frac{r_{2}}{q_{2}}}\right]^{\frac{1}{r_{2}}}} \\
&\leq c_{1}\left[\sup _{0 \leq t \leq T} \int_{\Omega}|v(t)|^{2} \, d  y+\int_{0}^{T} \, d  t \int_{\Omega}|\nabla v(t)|^{2} \, d  y\right]^{1 / 2},
\end{aligned}
\label{124}\end{equation}
where $c_{1} \geq 0$ depends only on $r_{2}, q_{2}$ and on local properties of the surface $\partial \Omega$ (which is assumed to be piecewise smooth) with $r_{2}$ and $q_{2}$ satisfying the conditions \eqref{eq 119}. By performing the change of variable $y=\frac{x}{\epsilon}$, it is easy to obtain the corresponding re-scaled estimates:
\begin{equation}
  \begin{aligned}
&\epsilon^{\frac{(1-N)}{q_{2}}}\left[\int_{0}^{T} \, d  t\left[\int_{\Gamma_{\epsilon}}|v(t)|^{q_{2}} d \mathcal{H}^{N-1}\right]^{\frac{r_{2}}{q_{2}}}\right]^{\frac{1}{r_{2}}} \\
&\quad \leq c_{1} \epsilon^{-\frac{N}{2}}\left[\sup _{0 \leq t \leq T} \int_{\Omega_{\epsilon}}|v(t)|^{2} \, d  x+\epsilon^{2} \int_{0}^{T} \, d  t \int_{\Omega_{\epsilon}}|\nabla v(t)|^{2} \, d  x\right]^{1 / 2},
\end{aligned}
\label{eq 125}\end{equation}
\begin{equation}
\begin{aligned}
&{\left[\int_{0}^{T} \, d  t\left[\int_{\Gamma_{\epsilon}}|v(t)|^{q_{2}} d \mathcal{H}^{N-1}\right]^{\frac{r_{2}}{q_{2}}}\right]^{\frac{1}{r_{2}}}} \\
&\quad \leq c \epsilon^{-\frac{N}{2}-\frac{(1-N)}{q_{2}}}\left[\sup _{0 \leq t \leq T} \int_{\Omega_{\epsilon}}|v(t)|^{2} \, d  x+\int_{0}^{T} \, d  t \int_{\Omega_{\epsilon}}|\nabla v(t)|^{2} \, d  x\right]^{1 / 2},
\end{aligned}
\label{eq 126}\end{equation}
where $c$ is a positive constant independent of $\epsilon$.
\end{proof}
