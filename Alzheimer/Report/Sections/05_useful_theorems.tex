\section{More on Two-Scale Convergence}\label{appendix:b}
The next theorems yield a characterization of the two-scale limit of the gradients of bounded sequences $v^{\epsilon}$ which is a critical result in order to apply the homogenization on problems.
\begin{theorem}
Let $v^{\epsilon}$ be a sequence of functions in $L^{2}([0, T] \times \Omega)$ which two-scale converges to a limit $v_{0} \in L^{2}([0, T] \times \Omega \times Y)$. Suppose, furthermore, that
$$
\lim _{\epsilon \rightarrow 0} \int_{0}^{T} \int_{\Omega}\left|v^{\epsilon}(t, x)\right|^{2} \, dxdt=\int_{0}^{T} \int_{\Omega} \int_{Y}\left|v_{0}(t, x, y)\right|^{2} \, dy dx dt.
$$
Then, for any sequence $w^{\epsilon}$ in $L^{2}([0, T] \times \Omega)$ that two-scale converges to a limit $w_{0} \in L^{2}([0, T] \times \Omega \times Y)$, we have
$$
\begin{aligned}
\lim _{\epsilon \rightarrow 0} \int_{0}^{T} \int_{\Omega} v^{\epsilon}(t, x) & w^{\epsilon}(t, x) \phi\left(t, x, \frac{x}{\epsilon}\right) \, dxdt \\
&=\int_{0}^{T} \int_{\Omega} \int_{Y} v_{0}(t, x, y) w_{0}(t, x, y) \phi(t, x, y) \, dy dx dt
\end{aligned}
$$for all $\phi \in C^{1}\left([0, T] \times \bar{\Omega} ; C_{\#}^{\infty}(Y)\right)$.
\label{thm 7.2}\end{theorem}
This theorem shows that, under specific hypotheses, the limit of the product coincides with the product of the limits. Whereas when the limit is performed it appears an extra variable and one needs a theorem that defines how to manage the gradient of this variable.
\begin{theorem}
Let $v^{\epsilon}$ be a bounded sequence belonging in $L^{2}\left(0, T ; H^{1}(\Omega)\right)$ such that $v_{\epsilon}(t,x)\rightharpoonup v(t,x)$  in $L^{2}\left(0, T ; H^{1}(\Omega)\right)$. Then $v^{\epsilon} \overset{2s}{\rightharpoonup} v(t, x)$, and there exists a function $v_{1}(t, x, y)$ in $L^{2}\left([0, T] \times \Omega ; H_{\#}^{1}(Y) / \mathbb{R}\right)$ such that, up to a subsequence, $\nabla v^{\epsilon}  \overset{2s}{\rightharpoonup} \left(\nabla_{x} v(t, x)+\nabla_{y} v_{1}(t, x, y)\right)$. 
\label{thm 7.3}\end{theorem}
\begin{theorem} Let $v^{\epsilon}$ and $\epsilon \nabla v^{\epsilon}$ be two bounded sequences in $L^{2}([0, T] \times \Omega)$. Then, there exists a function $v_{1}(t, x, y)$ in $L^{2}\left([0, T] \times \Omega ; H_{\#}^{1}(Y) / \mathbb{R}\right)$ such that, up to a subsequence, $v^{\epsilon}$ and $\epsilon \nabla v^{\epsilon}$ two-scale converge to $v_{1}(t, x, y)$ and $\nabla_{y} v_{1}(t, x, y)$, respectively.
\label{theorem 7.4}\end{theorem}
The main result of two-scale convergence can be generalized to the case of sequences defined on the boundary of the holes: $L^{2}\left([0, T] \times \Gamma_{\epsilon}\right)$.
\begin{theorem}
Let $v^{\epsilon}$ be a sequence in $L^{2}\left([0, T] \times \Gamma_{\epsilon}\right)$ such that
$$
\epsilon \int_{0}^{T} \int_{\Gamma_{\epsilon}}\left|v^{\epsilon}(t, x)\right|^{2} \, d\sigma_{\epsilon}(x)dt \leq C,
$$
where $C$ is a positive constant, independent of $\epsilon$. There exists a subsequence (still denoted by $\epsilon$) and a two-scale limit $v_{0}(t, x, y) \in L^{2}\left([0, T] \times \Omega ; L^{2}(\Gamma)\right)$ such that $v^{\epsilon}(t, x)$ two-scale converges to $v_{0}(t, x, y)$ in the sense that
$$
\begin{aligned}
&\lim _{\epsilon \rightarrow 0} \epsilon \int_{0}^{T} \int_{\Gamma_{\epsilon}} v^{\epsilon}(t, x) \phi\left(t, x, \frac{x}{\epsilon}\right) \mathrm{d} t \mathrm{~d} \sigma_{\epsilon}(x) \\
&=\int_{0}^{T} \int_{\Omega} \int_{\Gamma} v_{0}(t, x, y) \phi(t, x, y) \mathrm{d} t \mathrm{~d} x \mathrm{~d} \sigma(y)
\end{aligned}
$$
for any function $\phi \in C^{1}\left([0, T] \times \bar{\Omega} ; C_{\#}^{\infty}(Y)\right)$.
\label{thm 7.5}\end{theorem}
This two-scale convergence theorem on the boundary led to the following lemma:
\begin{lemma} Let $B=C\left[\bar{\Omega} ; C_{\#}(Y)\right]$ be the space of continuous functions $\phi(x, y)$ on $\bar{\Omega} \times Y$ which are $Y$-periodic in $y$. Then, B is a separable Banach space which is dense in $L^{2}\left(\Omega ; L^{2}(\Gamma)\right)$, and such that any function $\phi(x, y) \in B$ satisfies
$$
\epsilon \int_{\Gamma_{\epsilon}}\left|\phi\left(x, \frac{x}{\epsilon}\right)\right|^{2} \mathrm{~d} \sigma_{\epsilon}(x) \leq C\|\phi\|_{B}^{2},
$$
and
$$
\lim _{\epsilon \rightarrow 0} \epsilon \int_{\Gamma_{\epsilon}}\left|\phi\left(x, \frac{x}{\epsilon}\right)\right|^{2} \mathrm{~d} \sigma_{\epsilon}(x)=\int_{\Omega} \int_{\Gamma}|\phi(x, y)|^{2} \mathrm{~d} x \mathrm{~d} \sigma(y) .
$$
\label{lemma 7.4}\end{lemma}
Since $\phi$ is a data of the problem one can assume every regularity and use exactly the previous result.
