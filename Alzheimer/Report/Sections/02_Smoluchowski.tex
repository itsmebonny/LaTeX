\section{Smoluchowski equation}
The Smoluchowski coagulation equation models various kinds of phenomena as, for example: the evolution of a system of solid or liquid particles suspended in a gas (in
aerosol science), polymerization (in chemistry), aggregation of colloidal particles (in physics), formation of stars and planets (in astrophysics), red blood cell aggregation (in
hematology), behavior of fuel mixtures in engines (in engineering). By convention, only binary reaction will be considered.
This phenomenon is called coalescence and can be written formally as 
\[
    P_i + P_j \rightarrow P_{i+j}
\]
Under the assumption that the aggregation of cluster is the result of only Brownian motion or diffusion, the equations can be written as:
\begin{equation}
    \pdev{u_i}{t}(t,x) - d_i \Delta_x u_i(t,x) = Q_i(u) \quad \text{in } [0,T] \times \Omega,
    \label{eq:smoluchowski}
\end{equation}
where \(u_i\) is the concentration of the cluster of length \(i\), \(d_i\) is the diffusion coefficient, \(\Omega\) is the domain of the system and \(Q_i(u)\) is the gain term minus the loss term which  are defined in this way:
\begin{equation*}
    \begin{aligned}
    &Q_{g, i}=\frac{1}{2} \sum_{j=1}^{i-1} a_{i-j, j} u_{i-j} u_{j} \\
    &Q_{l, i}=u_{i} \sum_{j=1}^{\infty} a_{i, j} u_{j}
    \end{aligned}
\end{equation*}
$Q_{g,i}$ describes the increasing of concentrations of the cluster of length $i$,
$Q_{l,i}$ describes the depletion of the polymers of size $i$.
An important role is played by the coagulation coefficients $a_{i-i,j}$ which describe a situation where a polymer $i-j$ coagulates with a polymer of length $i$ to form one of length $j$, they are symmetric, indeed $a_{i,j}=a_{j,i}$.\\
The \eqref{eq:smoluchowski} is nonlinear and of infinite dimension: the existence and uniqueness of the solution is not guaranteed by the theory of RDEs. The existence of the solution as a matter of facts depends on the choice of the coagulation coefficients, of which an explicit expression is given in \cite{Bertsch}.
If there are no sources the total mass of the clusters will be conserved, but actually this is not true. In fact a particular choice of the coefficients produces a non-conservation of the total mass due to the appearance of an infinite cluster called gel, which is born from the growth of longer and longer clusters (this is known as gelation phenomenon), but will not be discussed here.