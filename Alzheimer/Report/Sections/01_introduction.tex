\section{Introduction}
Alzheimer's disease (AD) is a neurodegenerative disease that usually starts slowly and progressively worsens. It is the most common form of senile dementia but the cause of it is still poorly understood. 
In our brain exists a protein called \(\beta\) peptide and it has been proven that it has a substantial role in the process of synaptic degeneration leading to neuronal death and eventually to dementia. This protein is produced in monomeric form by every healthy brain, the problem occurs when  by unknown reasons (partially genetic) some neurons start to present an imbalance between production and clearance of $\mathrm{A} \beta$ amyloid during aging and, at elevated levels, it produces pathological aggregates such as long insoluble amyloid fibrils which accumulate in spherical deposits known as senile plaques. Actually has been proved that the most toxic form of the \(\beta\) is not the senile plaque but the oligomeric aggregation of the peptide outside the neurons. The $\mathrm{A} \beta$ amyloid is involved in the release of neurotoxic products that are connected in neuronal and synaptic damage. 
The importance of having a mathematical model for this disease is that the most toxic polymers don't live long enough to be studied from the experimental point of view. For this reason the mathematical model is the only way that we have to study the behavior of the amyloid.\\
The modellization process carried out in this paper starts from the Smoluchowski equation: a system of partial differential equations which describes the evolving densities of diffusing particles that coagulate in pairs. We will study the application of the Smoluchowski equation to the description of agglomeration of $\mathrm{A} \beta$ peptide. Indeed, one wants to start from differential equations that are assumed to hold on the microscale and to transform them into equations on macroscale. To do this not trivial passage, which involves a sort of "average process", we use a method called homogenization. It consists in performing the limits of the solutions of PDE's and finding the set of equations of which these limits are solutions. \\
So our main aim is to derive a macroscopic model starting from a microscopic one: this strategy is particularly relevant because allows to model the onset and progression of the disease at the proper neuronal scale and then, through an asymptotic procedure, to obtain consistent macroscopic equations whose outcomes can be directly compared with the clinical data.
