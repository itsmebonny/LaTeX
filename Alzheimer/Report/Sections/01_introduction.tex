\section{Introduction}
Alzheimer's disease (AD) is a neurodegenerative disease that usually starts slowly and progressively worsens. It is the most common form of senile dementia but the cause of it is still poorly understood. 
There are a few hypotheses about the cause of the disease, but the most accepted is the amyloid cascade hypothesis, which is the one that will be considered in this work.
Inside the brain exists a protein called \(\beta\)-peptide, and it has been proven that it has a substantial role in the process of synaptic degeneration leading to neuronal death and eventually to dementia. This protein is produced in monomeric form by every healthy brain, but some problems arise when,  by unknown reasons (partially genetic), some neurons start to present an imbalance between production and clearance of $\mathrm{A} \beta$ amyloid during aging. At elevated levels of concentration, it produces pathological aggregates such as long insoluble amyloid fibrils which accumulate in spherical deposits known as senile plaques. Actually, it has been proved that the most toxic form of the \(\beta\) is not the senile plaque, but the oligomeric aggregation of the peptide outside the neurons. The $\mathrm{A} \beta$ amyloid is involved in the release of neurotoxic products that are connected in neuronal and synaptic damage.
Having a mathematical model for Alzheimer's disease is crucial because the highly toxic polymers involved in the disease have a short lifespan and cannot be easily studied experimentally. Therefore, a mathematical model provides the only means to investigate the behavior of amyloid in this context.
In Bertsch et al. \cite{Bertsch} the authors present a model for the evolution of AD at a macroscopic scale and over the entire lifetime of the patient. In this case, the whole brain is represented by a region of the three-dimensional space, and the process of diffusion and aggregation of \(\mathrm{A}\beta\) is modeled by a Smoluchowski system with a source term, coupled with a kinetic-type transport equation that keeps into account the spreading of the disease. Clearly, at this scale, neurons are no more visible so that they can be described mathematically as points.
The modeling process carried out in this paper starts again from the Smoluchowski equation: a system of PDEs which describes the evolving densities of diffusing particles that coagulate in pairs. In this report it will be studied the application of the Smoluchowski equation to the description of agglomeration of $\mathrm{A} \beta$ peptide. Indeed, one wants to start from differential equations that are assumed to hold on the microscale and to transform them into equations on macroscale. To do this not trivial passage, which involves a sort of "average process", we use a method called homogenization. It consists in performing the limits of the solutions of PDE's and finding the set of equations of which these limits are solutions. 

So the main objective is to derive a macroscopic model starting from a microscopic one: this strategy is particularly relevant because allows to model the onset and progression of the disease at the proper neuronal scale and then, through an asymptotic procedure, to obtain consistent macroscopic equations whose outcomes can be directly compared with the clinical data.
