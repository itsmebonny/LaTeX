\documentclass[a4paper, 11pt]{article}

% --- PREAMBOLO ---
\usepackage[utf8]{inputenc}
\usepackage[T1]{fontenc}
\usepackage{lmodern}
\usepackage[italian]{babel}
\usepackage{geometry}
\geometry{a4paper, left=2.5cm, right=2.5cm, top=2.5cm, bottom=2.5cm}
\usepackage{array}
\usepackage{graphicx}
\usepackage{xcolor}
\usepackage{titlesec}
\usepackage{booktabs} % Per tabelle più eleganti
\usepackage{enumitem} % Per personalizzare le liste
\usepackage{colortbl} % Per colorare le celle delle tabelle

% --- DEFINIZIONE COLORI E STILI ---
\definecolor{darkblue}{RGB}{0, 80, 150}
\definecolor{lightgray}{gray}{0.95}

\titleformat{\section}
  {\normalfont\Large\bfseries\color{darkblue}}
  {\thesection}{1em}{}
\titleformat{\subsection}
  {\normalfont\large\bfseries}
  {\thesubsection}{1em}{}

\setlength{\parindent}{0pt} % Rimuove l'indentazione dei paragrafi
\setlength{\parskip}{1em} % Aggiunge spazio tra i paragrafi

% --- DOCUMENTO ---
\begin{document}

\begin{tabular}{@{}p{0.45\textwidth} p{0.45\textwidth}@{}}
\toprule
\textbf{Proteine e Latticini} & \textbf{Verdura e Frutta} \\
\midrule
\begin{itemize}[leftmargin=*]
    \item Petto di pollo: circa 600g
    \item Uova: 6
    \item Tonno al naturale in scatola: 2 lattine da 80g
    \item Ceci precotti: 1 barattolo grande (circa 400g)
    \item Lenticchie secche: 200g
    \item Ricotta fresca: 250g
    \item Yogurt magro (bianco o frutta): 6 vasetti
\end{itemize} & 
\begin{itemize}[leftmargin=*]
    \item Zucchine: 3-4 medie
    \item Pomodori pelati o passata: 1 barattolo grande (800g)
    \item Pomodorini freschi: 1 confezione (circa 250g)
    \item Cipolle: 2
    \item Aglio: 1 testa
    \item Lattuga o insalata mista: 1 cespo grande o 2 buste
    \item Carote: 3-4
    \item Spinaci (freschi o surgelati): 300g
    \item Patate: 3-4 medie (circa 500g)
    \item Mele: 4-5
    \item Banane: 4-5
\end{itemize} \\
\addlinespace[1em]
\bottomrule
\toprule
\textbf{Carboidrati e Dispensa} & \\
\midrule
\begin{itemize}[leftmargin=*]
    \item Pasta (integrale o di semola): 500g
    \item Riso (per risotti o bianco): 300g
    \item Pane integrale o bianco: 1 filone/confezione
    \item Fette biscottate integrali: 1 pacco
    \item Marmellata a ridotto tenore di zuccheri
    \item Olio extra vergine di oliva
    \item Sale, pepe, origano/basilico
\end{itemize} & \\
\bottomrule
\end{tabular}

\newpage

\section{Piano Alimentare Dettagliato (per 2 persone)}

\renewcommand{\arraystretch}{1.5}
\begin{tabular}{|>{\bfseries}m{2.5cm}|m{12.5cm}|}
\hline
\multicolumn{2}{|c|}{\cellcolor{lightgray}\bfseries LUNEDÌ} \\
\hline
\textbf{Colazione} & 
\begin{itemize}[noitemsep, topsep=0pt, leftmargin=*]
    \item 2 vasetti di yogurt magro con pezzi di banana.
\end{itemize} \\
\hline
\textbf{Pranzo} & 
\begin{itemize}[noitemsep, topsep=0pt, leftmargin=*]
    \item \textbf{Pasta con sugo di zucchine e pomodoro}: Pasta (160g) condita con un sugo preparato con 2 zucchine, mezza cipolla e metà del barattolo di pomodori pelati.
    \item Olio EVO (2 cucchiaini a persona).
\end{itemize} \\
\hline
\textbf{Cena} & 
\begin{itemize}[noitemsep, topsep=0pt, leftmargin=*]
    \item \textbf{Petto di pollo alla griglia}: Circa 300g di pollo grigliato con aromi.
    \item \textbf{Insalata mista}: Lattuga e carote grattugiate.
    \item Pane (40g a persona).
\end{itemize} \\
\hline
\multicolumn{2}{|c|}{\cellcolor{lightgray}\bfseries MARTEDÌ} \\
\hline
\textbf{Colazione} & 
\begin{itemize}[noitemsep, topsep=0pt, leftmargin=*]
    \item Fette biscottate integrali (2-3 a persona) con marmellata.
\end{itemize} \\
\hline
\textbf{Pranzo} & 
\begin{itemize}[noitemsep, topsep=0pt, leftmargin=*]
    \item \textbf{Riso e lenticchie}: Riso (160g) cotto insieme alle lenticchie (200g secche) con un soffritto di cipolla e carota.
    \item Olio EVO (2 cucchiaini a persona).
\end{itemize} \\
\hline
\textbf{Cena} & 
\begin{itemize}[noitemsep, topsep=0pt, leftmargin=*]
    \item \textbf{Frittata di patate e spinaci}: Frittata preparata con 4 uova, 2 patate lesse a cubetti e metà degli spinaci.
    \item Pane (40g a persona).
\end{itemize} \\
\hline
\multicolumn{2}{|c|}{\cellcolor{lightgray}\bfseries MERCOLEDÌ} \\
\hline
\textbf{Colazione} & 
\begin{itemize}[noitemsep, topsep=0pt, leftmargin=*]
    \item 2 vasetti di yogurt magro con una mela a pezzi.
\end{itemize} \\
\hline
\textbf{Pranzo} & 
\begin{itemize}[noitemsep, topsep=0pt, leftmargin=*]
    \item \textbf{Insalatona con tonno e ceci}: Abbondante insalata mista con 2 lattine di tonno, i ceci precotti e pomodorini freschi.
    \item Pane (40g a persona) e Olio EVO.
\end{itemize} \\
\hline
\textbf{Cena} & 
\begin{itemize}[noitemsep, topsep=0pt, leftmargin=*]
    \item \textbf{Pasta con ricotta e spinaci}: Pasta (160g) condita con la ricotta (250g) e gli spinaci rimasti, saltati in padella con aglio.
    \item Una spolverata di pepe.
\end{itemize} \\
\hline
\end{tabular}

\newpage

\begin{tabular}{|>{\bfseries}m{2.5cm}|m{12.5cm}|}
\hline
\multicolumn{2}{|c|}{\cellcolor{lightgray}\bfseries GIOVEDÌ} \\
\hline
\textbf{Colazione} & 
\begin{itemize}[noitemsep, topsep=0pt, leftmargin=*]
    \item Fette biscottate integrali (2-3 a persona) con marmellata.
\end{itemize} \\
\hline
\textbf{Pranzo} & 
\begin{itemize}[noitemsep, topsep=0pt, leftmargin=*]
    \item \textbf{Riso al pomodoro}: Riso (160g) condito con la salsa di pomodoro avanzata dal lunedì (o preparata al momento).
    \item Una porzione di pane per accompagnare (30g a persona).
\end{itemize} \\
\hline
\textbf{Cena} & 
\begin{itemize}[noitemsep, topsep=0pt, leftmargin=*]
    \item \textbf{Bocconcini di pollo con zucchine}: Il restante pollo (circa 300g) tagliato a cubetti e cotto in padella con le zucchine avanzate, aglio e pomodorini.
    \item Pane (40g a persona).
\end{itemize} \\
\hline
\multicolumn{2}{|c|}{\cellcolor{lightgray}\bfseries VENERDÌ} \\
\hline
\textbf{Colazione} & 
\begin{itemize}[noitemsep, topsep=0pt, leftmargin=*]
    \item 2 vasetti di yogurt magro con frutta a scelta (banana o mela).
\end{itemize} \\
\hline
\textbf{Pranzo} & 
\begin{itemize}[noitemsep, topsep=0pt, leftmargin=*]
    \item \textbf{Pasta al tonno}: La pasta restante (circa 180g) condita con un sugo semplice di pomodoro e le ultime scatolette di tonno (se avanzate, altrimenti usare una nuova).
    \item Olio EVO e origano.
\end{itemize} \\
\hline
\textbf{Cena} & 
\begin{itemize}[noitemsep, topsep=0pt, leftmargin=*]
    \item \textbf{Uova strapazzate e verdure}: Le 2 uova rimaste, strapazzate.
    \item \textbf{Contorno di patate e carote}: Le patate e le carote avanzate, cotte al forno o lesse.
    \item Pane (40g a persona).
\end{itemize} \\
\hline
\end{tabular}
\vspace{1cm}
\begin{center}
    \textit{Spuntini di metà mattina e merenda: una mela o una banana a persona.}
\end{center}

\section*{Consigli Utili}
\begin{itemize}
    \item \textbf{Cotture}: Prediligere metodi di cottura sani come griglia, vapore, forno o padella con poco olio.
    \item \textbf{Idratazione}: Ricordare di bere circa 1,5-2 litri di acqua al giorno a persona.
    \item \textbf{Flessibilità}: Gli ingredienti delle insalate o dei contorni possono essere scambiati in base alla disponibilità (es. usare carote invece di pomodorini).
    \item \textbf{Condimenti}: Limitare il sale e usare liberamente spezie e aromi per insaporire i piatti. L'olio EVO va dosato come da piano (circa 4 cucchiaini a persona al giorno).
\end{itemize}

\end{document}