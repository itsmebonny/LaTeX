\section{Lesson 27/10/2022}

\(\left(X, d\right)\) metric space.
\begin{definition}
    \(E \subset X\) is compact if for any open covering \(\{A_i\}_{i \in I}\) has a finite subcover.
\end{definition}

\begin{definition}
    \(E \subset X\) is sequentially compact if \(\forall \; \{x_n\} \subset E \) 
    there exists \(\{x_{n_k}\}\) subsequence convergent to some limit \(x \in E\)
\end{definition}

Well known fact: if \(\left(X, d\right)\) is a metric space, then \(E\) is compact \(\iff E \) is sequentially compact.

\begin{theorem}[Riesz Theorem]
    \(X\) normed space, dim\(X = \infty\) \(\iff \bar{B}_1(0)\) is not compact. 
\end{theorem}

\begin{lemma}[Riesz quasi orthogonality Lemma]
    \(X\) normed space, \(E \subsetneq X\) closed subspace. Then \(\forall \; \epsilon \in \left(0, 1\right) \ \exists \; x \in X \) s.t. 
    \[
        \norm{x}=1 \text{ and } \text{dist}(x, E) = \inf_{y \in E} \norm{x-y} \geq 1- \epsilon\]
\end{lemma}

\begin{remark}

    \begin{itemize}
        \item \(E \in X\) closed. Then dist\((x, E)=0 \iff x \in E\)
        \item By definition of infimum, if \(d =\) dist\((x, E)\), then \(\forall \rho >0 \ \exists \; z \in E\) s.t. 
        \[
            \norm{x-z} < (1+\rho) d
        \]
    \end{itemize}
\end{remark}

\begin{proof}
    Let \(y \in X \setminus E\), and \(d := \) dist\((y, E) >0\), since \(E\) is closed. 
    
    \(\forall \; \rho > 0 \ \exists z \in E \) s.t.
    \[
        \norm{y-z} \leq (1+\rho)d = \frac{d}{1-\epsilon} \tag{1}
    \]
    since we choose \(\rho\) s.t. \(1+\rho = \frac{1}{1-\epsilon}\). Now we set \(x = \frac{y-z}{\norm{y-z}}\).

    Clearly \(\norm{x}=1\). Moreover, \(\forall \; u \in E\), we have that
    \[
        \norm{x-u} = \norm{ \frac{y-z}{\norm{y-z}} - u }
        = \norm{ \frac{y-z -\norm{y-z}u }{\norm{y-z}} }
        = \frac{1}{\norm{y-z}} \norm{y-(z + \norm{y-z}u)} =
    \]
    \[
        = \frac{1}{\norm{y-z}} \norm{y-w}
        \geq \frac{1}{\norm{y-z}} \text{dist}(y, E)
        \overset{(1)}{\geq} \frac{1-\epsilon}{d} d = 1 - \epsilon
    \]
    Since this is true \(\forall \; u \in E\), we deduce that
    \[
        \text{dist}(x, E) \geq 1-\epsilon
    \]
\end{proof}

\subsection*{Compactness on \(\mathcal{C}^0(\left[a, b\right])\)}
\begin{definition}
    \(\{f_n\}\) sequence in \(\mathcal{C}^0(\left[a, b\right])\). 
    We say that \(\{f_n\}\) is uniformly equicontinuous in \([a, b]\) if \(\forall \; \epsilon >0 \; \exists \; \delta >0 \) depending only on \(\epsilon \) s.t. 
    \[
        |t-\tau| < \delta \Rightarrow \| f_n(t) - f_n(\tau) \| < \epsilon \qquad \forall \; n
    \]
\end{definition}

\begin{remark}
    With respect to the uniform continuity, in this case \(\delta\) does not depend on \(f\). \(\delta\) is the same for all the \(f_n\) s
\end{remark}

\begin{theorem}
    \(\{f_n\} \subseteq \mathcal{C}^0(\left[a, b\right])\). Suppose that:
    \begin{itemize}
        \item \(\{f_n\}\) is uniformly equicontinuous
        \item \(\{f_n\}\) is bounded: \(\exists \; M>0\) s.t. \(\norm{f_n}_\infty <M \qquad \forall \; n\)
    \end{itemize}
    Then \(\exists\) a subsequence \(\{f_{n_k}\}\) and \(f \in \mathcal{C}^0(\left[a, b\right])\) s.t. \(f_{n_k} \rightarrow f \) uniformly.
\end{theorem}

Lebesgue spaces. \\
\((X, \mathcal{M}, \mu)\) measure space, \(p \in \left[1, \infty\right]\). We defined \(L^1(X)\) and \(L^\infty(X)\). 
In a similar way, we define \(L^p(X)\) \(\forall \; p \in \left[1, \infty\right]\)
\[
    \mathcal{L}^p (X, \mathcal{M}, \mu) := \{ f: X \rightarrow \barreal \text{ measurable s.t. } \int_X |f|^p \, d\mu < \infty\}
\]
On \(\mathcal{L}^p\) we introduce the equivalent relation
\[
    f \sim g \text{ in } \mathcal{L}^p \iff f=g \text{ a.e. on } X 
\]
and define 
\[
    {L}^p (X, \mathcal{M}, \mu) := \frac{\mathcal{L}^p (X, \mathcal{M}, \mu)}{\sim}
\]
We want to show that this is a normed space with
\[
    \norm{f}_p := 
    \begin{cases}
        \left( \int_X |f|^p \, d\mu \right)^{\frac{1}{p}}  & p \in [1, \infty) \\
        \esssup_X |f| & p = \infty
    \end{cases}
\]

The fact that \(L^p\) is a vector space is easy to prove. The only non trivial part is that \(f, g \in L^p \Rightarrow f+g \in L^p\).

This comes directly from the 
\begin{lemma}
    \(p \in [1, \infty), \ a, b \geq 0\). Then 
    \[
        \left(a+b\right)^p \leq 2^{p-1} \left(a^p+b^p\right) 
    \]
\end{lemma}

\(f, g \in L^p, \ p \in [1, \infty)\)
\[
    \int_X |f+g|^p \, d\mu \leq \int_X (|f|+|g|)^p \, d\mu 
    \leq 2^{p-1} \int_X (|f|^p+|g|^p) \, d\mu
\]
\[
    = 2^{p-1} \int_X |f|^p \, d\mu + 2^{p-1} \int_X |g|^p \, d\mu < \infty
\]
\(L^p\) is a vector space, \(\forall \; p \in [1, \infty)\).


\(f, g \in L^\infty\). Then a.e. 
\[
    \Rightarrow |f+g| \leq |f|+|g| \leq \norm{f}_\infty + \norm{g}_\infty < \infty
    \Rightarrow f+g \in L^\infty
\]
\(L^\infty\) is a vector space. 

\begin{remark}
    \(l^p := L^p (\mathbb{N}, \mathcal{P}(\mathbb{N}), \mu_c )\). \(l^p\) is a particular case of \(L^p\)
    \[
    \begin{array}{ll}
        l^p = \{ x = \left(x^{(k)}\right)_{k \in \mathbb{N}} : \sum_{k=1}^\infty |x^{(k)}|^p < \infty \} 
        & \norm{x}_p = \left( \sum_{k=1}^\infty |x^{(k)}|^p \right)^{\frac{1}{p}} \quad p \in [1, \infty)
        \\ l^\infty = \{ x = \left(x^{(k)}\right)_{k \in \mathbb{N}} : \sup_{k \in \mathbb{N}} |x^{(k)}| < \infty \} 
        & \norm{x}_\infty = \sup_{k \in \mathbb{N}} |x^{(k)}|
    \end{array}
    \]
\end{remark}

Now we prove that \(\norm{.}_p\) is actually a norm in \(L^p\). 
We will concentrate on \(p < \infty\) (\(p = \infty\) is the easy case) 

Properties 1 and 2 of the norm are immediate to check:
\begin{enumerate}
    \item \(\norm{f}_p = 0 \iff \int_X |f|^p \, d\mu =0 \iff f=0 \text{ a.e. on } X \iff f=0 \in L^p\)
    \item Obvious, by linearity
    \item About triangle inequality? We need some preliminaries
\end{enumerate}

\begin{theorem}[Young's Inequality]
    Let \(p \in (1, \infty)\), \(a, b \geq 0\). We say that \(q\) is the conjugate exponent of p if 
    \[
        \frac{1}{p} + \frac{1}{q} = 1 \iff q = \frac{p}{p-1}
    \]
    Then 
    \[
        ab \leq \frac{a^p}{p} + \frac{b^q}{q}
    \]
\end{theorem}
\begin{remark}
    \(p \in (1, \infty) \Rightarrow q \in (1, \infty)\). Moreover, we say that \(1\) and \(\infty\) are conjugate
\end{remark}
\begin{proof}
    \(\phi(x)= e^{x}\) is convex: 
    \[
        \phi((1-t)x + ty) \leq (1-t)\phi(x) + t \phi(y) \qquad \forall x, y \in \real \quad \forall \; t \in [0, 1]
    \]

    If \(a=0\) or \(b=0\), then the thesis holds. \\
    If \(a, b >0\)
    \[
        ab = e^{\log{a}} e^{\log{b}}
        = e^{\log{a}^{\frac{p}{p}}} e^{\log{b}^{\frac{q}{q}}}
        = e^{\frac{1}{p}\log{a}^p} e^{\frac{1}{q}\log{b}^q}
    \]
    Since \(\phi \) is convex
    \[
        \frac{1}{p} e^{\log{a}^p} + \frac{1}{q} e^{\log{b}^q} = \frac{1}{p} a^p + \frac{1}{q} b^q
    \]
    \(x = \log{a^p}\), \(y= \log{b^q}\) \(\qquad 1-t = \frac{1}{p}\), \(t=\frac{1}{q}\)
\end{proof}

\begin{theorem}
    \(\left(X, \mathcal{M}, \mu \right)\) measure space. \(f, g\) measurable functions. \(p, q \in [1, \infty]\) conjugate exponents.

    Then 
    \[
        \norm{fg}_1 \leq \norm{f}_p \norm{g}_q
    \]
\end{theorem}
\begin{proof}
    Case \(p, q \in (1, \infty)\). Obvious if \(\norm{f}_p \norm{g}_q = \infty\). \\
    If \(\norm{f}_p \norm{g}_q = 0 \Rightarrow\)  either \(f=0\) a.e. on \(X\) or \(g=0\) a.e. on X
    \(\Rightarrow fg=0\) a.e. on \(X\) \(\Rightarrow \norm{fg}_1 =0\). Let then \(\norm{f}_p\), \(\norm{g}_p \in (0, \infty)\). \\
    For \(x \in X\), we set 
    \[
        a := \frac{|f(x)|}{\norm{f}_p} \text{, } b := \frac{|g(x)|}{\norm{g}_q} 
    \]
    and use Young:
    \[
        \frac{|f(x)g(x)|}{\norm{f}_p \norm{g}_q} 
        \leq \frac{1}{p} \frac{|f(x)|^p}{\norm{f}_p^p} + \frac{1}{q} \frac{|g(x)|^q}{\norm{g}_q^q}
    \]
    \(\forall \; x \in X \). By integrating, we obtain
    \[
        \frac{1}{\norm{f}_p \norm{g}_q} \int_X |fg| \, d\mu \leq 
        \frac{1}{p \norm{f}_p^p} \int_X |f|^p \, d\mu + \frac{1}{q \norm{g}_q^q} \int_X |g|^q \, d\mu 
        = \frac{1}{p} + \frac{1}{q} = 1
    \]
    \[
        \Rightarrow \norm{fg} \leq \norm{f}_p \norm{g}_q
    \]
    Case \(p=1\), \(q= \infty\). Exercise 
\end{proof}
\begin{theorem}[Minkowski Inequality]
    \(f, g \in L^p(X, \mathcal{M}, \mu)\), \(p \in [1, \infty]\). Then 
    \[
        \norm{f+g}_p \leq \norm{f}_p + \norm{g}_p
    \] 
\end{theorem}
\begin{proof}
    \(p \in (1, \infty)\)
    \[
        \norm{f+g}_p^p = \int_X |f+g|^p \, d\mu = \int_X |f+g| |f+g|^{p-1} \, d\mu
    \]
    \[    
        \leq \int_X \left( |f|+|g| \right) |f+g|^{p-1} \, d\mu
        = \int_X |f| |f+g|^{p-1} \, d\mu + \int_X |g| |f+g|^{p-1} \, d\mu 
    \]
    Using Holder with \(p\), \(q = \frac{p}{p-1}\)
    \[
        \leq \norm{f}_p \left( \int_X \left( |f+g|^{p-1} \right)^{\frac{p}{p-1}} \, d\mu \right) ^ {\frac{p-1}{p}}
        + \norm{g}_p \left( \int_X \left( |f+g|^{p-1} \right)^{\frac{p}{p-1}} \, d\mu \right) ^ {\frac{p-1}{p}}
    \]
    \[
        = \norm{f}_p \norm{f+g}^{p-1}_p + \norm{g}_p \norm{f+g}_p^{p-1}
    \]
    We divide left hand side and right hand side by \(\norm{f+g}_p^{p-1}\):
    \[
        \norm{f+g}_p \leq \norm{f}_p + \norm{g}_p
    \]
\end{proof}