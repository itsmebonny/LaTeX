\section{Lecture 30/11/2022}

We introduced the weak convergence. \(X\) banach space. \({x_n} \subset X\) converges weakly to \(x\), \(x_n \rightharpoonup x\) weakly in \(X\), if 
\[
    L x_n \to Lx \text { in } \real, \quad \forall\; L \in X^* = \mathcal{L}(X, \real)
\]

Recall that: 
\begin{itemize}
    \item \(x_n \to x \) strongly in \(X\), namely \(\norm{x_n - x}_X \to 0\) \(\Rightarrow x_n \rightharpoonup x\) and \(\nLeftarrow\)
    \item \(x_n \rightharpoonup x \Rightarrow \{x_n\}\) is bounded, the weak limit \(x\) is unique, and 
    \[
        \norm{x} \leq \liminf _{n \to \infty} \norm{x_n}
    \]
\end{itemize}

\begin{remark}
    In \(\real^n\) (or any finite dimensional Banach space) \(x_n \rightharpoonup x\) weakly \(\iff x_n \to x\) strongly (ex.)
\end{remark}

With the same philosophy we introduce: 
\begin{definition}
    \(X \) Banach \(\Rightarrow X^* \) is Banach as well.
\end{definition}
\begin{definition}
    A sequence \(\{L_n \} \subset X^*\) is weakly\(^*\) convergent to \(L \in  X^*\), namely \(L_n \overset{*}{\rightharpoonup} L\) in \(X^*\), if 
    \[
        L_n x \to Lx \in \real \quad \forall\; x \in X
    \]
\end{definition}
\begin{remark}
    Observe thar a sequence \(\{L_n\} \) tends weakly to \(L\) in \(X^*\) if 
    \[
        \Lambda L_n \to \Lambda L \quad \forall \Lambda \in X^{**}
    \]
\end{remark}
We know that \(\exists \; \tau:X \to X^{**}\) canonical map s.t. 
\[
    \underset{X^{**}}{\langle}\tau(x), L \underset{X^*}{\rangle} = Lx \quad \forall \; L \in X^*
\]
Thus \(L_n \rightharpoonup L\) weakly in \(X^*\) \(\rightarrow\) \(\langle \tau(x), L_n \rangle \to \langle \tau(x), L \rangle\) \(\forall x \in X:\) namely
\[
    L_n x \to Lx \qquad \forall \; x \in X
\]
namely \(L_n \overset{*}{\rightharpoonup}\) weakly* in \(X^*\). In general the converse is false. However
\begin{proposition}
    If \(X\) is reflexive, then \(L_n \rightharpoonup L\) weakly in \(X^* \iff L_n \overset{*}{\rightharpoonup}\) weakly* in \(X^*\)
\end{proposition}
\begin{proof}
    If \(X\) is reflexive, every element \(\Lambda\) of \(X^{**}\) is of type \(\Lambda = \tau(x)\) for some \(x\)
\end{proof}
\begin{proposition}
    \(X\) banach space, \(X^*\) dual space, \(L_n \overset{*}{\rightharpoonup} L\) in \(X^*\). Then
    \begin{itemize}
        \item The weak * limit is unique
        \item \(\{L_n\}\) is bounded
        \item \(\norm{L}_{X^*} \leq \liminf _{n \to \infty} \norm{L_n}_{X^*}\)
        \item If in addition \(x_n \to x\) strongly in \(X\) \(\Rightarrow\) \(L_nx_n \to Lx\) 
    \end{itemize}
\end{proposition}

\begin{theorem}[Banach Alaoglu]
    \(X \) separable Banach space. Then every bounded sequence in \(X^*\) has a weakly* convergent subsequence. (bounded sets in \(X^* \) sequentially compact for the weak* convergence)
\end{theorem}

\begin{proof}
    \(\{L_n\}\) bounded sequence in \(X^*\), nammely
    \[
        \sup_{n} \norm{L_n}_{X^*} = M < \infty
    \]
    Since \(X\) is separable, \(\exists \; \{x_k\}_{k \in \natural}\) dense in \(X\). 
    Now, consider \(\{L_n x_1\}:\) it is bounded in \(\real\):
    \[
        \abs{L_n x_1} \leq \norm{L_n}_{X^*} \norm{x_1}_X \leq M \norm{x_1}_X < \infty
    \]
    \(\Rightarrow \exists \; \{L_{n_j}\}\) s.t. \(L_{n_j} x_1 \to l_j\) in \(\real\). Now, consider \(\{L_{n_j} x_2\}\): it is bounded,
    \[
        \abs{L_{n_j}x_2} \leq \norm{L_{n_j}}_{X^*} \norm{x_2}_X \leq M \norm{x_2}_X < \infty
    \]
    \(\Rightarrow \exists\; \{L_{n_{ij}}\}\) subsequence of \(\{L_{n_j}\}\) s.t. \( L_{n_{ij}} x_2 \to l_2\) in \(\real\)
    We can iterate the process. \(\forall \; k \quad \{L_n^k\}\) is a subsequence of \(\{L_n^{k-1}\}\). \(\Rightarrow \{L_n^k\}\) is a subsequence of \(\{L_n^j\}\) \(\forall i < k\). In particular, 
    \[
        L_n^k x_j \to l_j \quad \forall \; j \leq k
    \]
    We pick up \(T_n = L_n^n\) (diagonal selection). By construction, \(\forall m \in \natural\) fixed, \(\{T_n: n \geq m\}\) is a subsequence of \(\{L_n^m: n \geq m\}\) 
    \[
        \Rightarrow T_n x_m \to l_m \quad \text{ as } n \to \infty
    \]
    We want to show now that \(T_n x \to l_x\) \(\forall x \in X\), and that \(l_x = Tx\) is such that \(T \in X^*\). Since \(\{x_k\}\) is dense, \(\forall \; x \in X\) and \( \forall\; \epsilon >0\) \(\exists\; k \in \natural \) s.t. 
    \[
        \norm{x-x_k}_X < \frac{\epsilon}{2M}
    \]
    Thus
    \[
        \abs{T_n x - T_m x} \leq \abs{T_n x - T_n x_k} + \abs{T_n x_k - T_m x_k} + \abs{T_m x_k - T_m x} \leq 
    \]
    \[
    \leq \norm{T_n}_{X^*} \norm{x -x_k}_X + \abs{T_n x_k - T_m x_k} + \norm{T_m}_{X^*} \norm{x -x_k}_X \leq
    \]
    \[
        \leq M \frac{\epsilon}{2M} + \abs{T_n x_k - T_m x_k} + M \frac{\epsilon}{2M} 
        < \epsilon + \abs{T_n x_k - T_m x_k} < 2 \epsilon
    \]
    \(\forall \; n, m > \overline{n}\), since \(\{T_n x_k\}\) is convergent and so a Cauchy sequence.

    This means that \(\{T_n x\}\) is a Cauchy sequence in \(\real\)
    \[
        T_n x \to l_x \text{ in } \real \quad \forall x \in \real
    \]
    It only remains to show that \(l_x = Tx\) for some \(T \in X^*\). This is a consequence of a corollary of Banach Steinhaus.

    To sum up: \(\{L_n\}\) bounded in \(X^*\)
    \[
        \Rightarrow \exists \; \{T_n\} \text{ subsequence s.t. } T_n x \to Tx
    \]
    for every \(x \in X\), namely \(T_n \overset{*}{\rightharpoonup} T\) in \(X^*\)
\end{proof}

\begin{theorem}[Variant of BA for reflexive spaces]
    \(X\) reflexive and Banach. Then every bounded sequence in \(X\) has a weakly convergent subsequence
\end{theorem}

\begin{proof}
    For simplicity, we assume that \(X\) is separable (not necessary). \(X \) separable and reflexive \(\Rightarrow X^*\) is separable. \(\tau:X \to X^{**}\) canonical map: it is an isometric isometry.

    \(\{x_n\}\) bounded sequence in \(X \iff \{ \tau(x_n) \}\) is bounded in \(X^{**} = (X^*)^*\)
        
    \(\Rightarrow \) by Banach Alaoglu, \(\exists \{x_{n_k}\}\) s.t. \(\tau(x_{n_k}) \overset{*}{\rightharpoonup} \Lambda \) in \(X^{**}\):
    \[
        \underset{X^{**}}{\langle}\tau(x_{n_k}), L \underset{X^*}{\rangle} \to \underset{X^{**}}{\langle}\Lambda, L \underset{X^*}{\rangle} \quad K \to \infty
    \]
    \(\forall\; L \in X^*.\) Since \(X\) is reflexive, \(\forall \Lambda \in X^{**} \) \(\exists ! \; x \in X\) s.t. \(\Lambda = \tau (x)\). Therefore, 
    \[
        L x_{n_k} = \underset{X^{**}}{\langle}\tau(x_{n_k}), L \underset{X^*}{\rangle} \to \underset{X^{**}}{\langle}\Lambda, L \underset{X^*}{\rangle} = Lx
    \]
    \(\forall\; L \in X^*\). We proved that 
    \[
        \lim_{k \to \infty} L x_{n_k} = Lx \qquad \forall\; L \in X^*
    \]
    namely \(x_{n_k} \rightharpoonup x\) in \(X\)
\end{proof}

\subsection*{Compact Operators}
\(X, Y\) Banach spaces. 
\begin{definition}
    A linear operator \(K:X \to Y\) is said to be compact if \(\forall E \subseteq X\) bounded, the set \(K(E)\) is relatively compact, namely \(\overline{K(E)}\) is compact. 

    Equivalently, \(K\) is compact if \(\forall \; \{x_n\} \subset X \) bounded, the sequence \(\{K(x_n)\}\) has a strongly convergent subsequence. 
\end{definition}
\begin{proposition}
    \(K:X \to Y\) linear and compact. Then \(K \in \mathcal{L}(X, Y)\)
\end{proposition}
\begin{proof}
    Define \(B:=\overline{B_1(0)}\) in \(X\). If \(K\) is compact 
    \(\Rightarrow K(B) \) is relatively compact 
    \(\Rightarrow \overline{K(B)}\) is compact 
    \(\Rightarrow \overline{K(B)}\) is bounded
    \(\Rightarrow K(B)\) is bounded: \(\exists \; M>0 \) s.t. 
    \[
        \norm{Kx}_Y \leq M \qquad \forall \; x \in \overline{B_1(0)} = B
    \]
    \[
        \Rightarrow \stackbelow{\underbrace{\sup_{\norm{x} \leq 1} \norm{Kx}_Y}}{\norm{K}_{\mathcal{L}(X, Y)}} \leq M
    \]
\end{proof}

\begin{definition}
    \(T \in \mathcal{L}(X, Y)\) has finite rank if 
    \[
        \underset{\dim (T(X))}{\text{is the image of }T= \{y \in Y: y = Tx\} \text{ for some} x \in X} < \infty
    \]
\end{definition}
\(T(X) \subset Y\) is a subspace.
\begin{proposition}
    \(T \in \mathcal{L}(X, Y)\) has finite rank \(\Rightarrow T\) is compact.
\end{proposition}
\begin{proof}
    \(X \subset X \) bounded. Since \(T \in \mathcal{L}(X, Y)\), then \(T(A)\) is bounded. \(T(A) \subset T(X) \approx \real^n\), since \(T\) has finite rank. 
    
    Thus \(T(A)\) is a bounded set of \(\real^n\) \(\Rightarrow T(A) \) is relatively compact. 
\end{proof}

\begin{definition}
    We denote by \(\mathcal{K}(X, Y)\) the class of linear compact operators from \(X \) to \(Y\). This is a linear subspace.

    If \(Y=X\), we write \(\mathcal{K}(X)\)
\end{definition}
\begin{proposition}
    \(X, Y\) Banach spaces, \(T:X \to Y\) linear and compact, \(Y\) in \(\infty\) dim. Then \(T\) cannot be surjective.
\end{proposition}
\begin{proof}
    Recall that \(C\) compact set, \(S \subset C\) closed \(\Rightarrow S\) is compact (in any metric space)

    Assume by contradiction that \(K\) is surjective. By the OMT, \(T\) is an open map. Take 
    \[
        \emptyset \neq A \subset X
    \]
    open and bounded. \(T(A)\) is relatively compact (since \(T\) is compact), and is open (since \(T\) is an open map) 
    and \(\neq \emptyset\)
    \[
        \Rightarrow T(A) \supset B_r(y_0)
    \]
    for some \(y_0 \in Y \) and \(r>0\). Thus 
    \[
        \overline{T(A)} \supset \overline{B_r(y_0)} \Rightarrow \overline{B_R(y_0)} 
    \] 
    is compact in \(Y\). This contradicts the Riesz theorem, since in \(\infty\) dimension balls are never compact. 
\end{proof}

\begin{proposition}
    \(X\), \(Y\), \(Z\) Banach spaces. \(T \in \mathcal{L}(X, Y)\), \(S \in \mathcal{K}(Y, Z)\) (or \(T \in \mathcal{K}(X, Y)\), \(S \in \mathcal{L}(Y, Z)\)). Then \(S \circ T\) is compact.
\end{proposition}
\begin{theorem}
    \(\mathcal{K}(X, Y) \) is a closed subspace of \(\mathcal{L}(X, Y)\). \(\Rightarrow (\mathcal{K}(X, Y), \normdot_{\mathcal{K}(X, Y)})\) is a Banach space.
\end{theorem}
Consequence: if we want to check that \(T \in \mathcal{L}(X, Y)\) is compact, we can prove that \(\exists\; \{T_n\} \subseteq \mathcal{K}(X, Y)\) s.t. 
\[
    \norm{T_n - T}_{\mathcal{L}} \to 0
\]
Since \(\mathcal{K}(X, Y)\), it follows that \(T\) is compact.