\section{Lecture 30/11/2022}

We introduced the weak convergence. \(X\) banach space. \({x_n} \subset X\) converges weakly to \(x\), \(x_n \rightharpoonup x\) weakly in \(X\), if 
\[
    L x_n \to Lx \text { in } \real, \quad \forall\; L \in X^* = \mathcal{L}(X, \real)
\]

Recall that: 
\begin{itemize}
    \item \(x_n \to x \) strongly in \(X\), namely \(\norm{x_n - x}_X \to 0\) \(\Rightarrow x_n \rightharpoonup x\) and \(\nLeftarrow\)
    \item \(x_n \rightharpoonup x \Rightarrow \{x_n\}\) is bounded, the weak limit \(x\) is unique, and 
    \[
        \norm{x} \leq \liminf _{n \to \infty} \norm{x_n}
    \]
\end{itemize}

\begin{remark}
    In \(\real^n\) (or any finite dimensional Banach space) \(x_n \rightharpoonup x\) weakly \(\iff x_n \to x\) strongly (ex.)
\end{remark}

With the same philosopy we introduce: 
\begin{definition}
    \(X \) Banach \(\Rightarrow X^* \) is Banach as well.
\end{definition}
\begin{definition}
    A sequence \(\{L_n \} \subset X^*\) is weakly\(^*\) convergent to \(L \in  X^*\), namely \(L_n \overset{*}{\rightharpoonup} L\) in \(X^*\), if 
    \[
        L_n x \to Lx \in \real \quad \forall\; x \in X
    \]
\end{definition}
\begin{remark}
    Observe thar a sequence \(\{L_n\} \) tends weakly to \(L\) in \(X^*\) if 
    \[
        \Lambda L_n \to \Lambda L \quad \forall \Lambda \in X^{**}
    \]
\end{remark}
We know that \(\exists \tau:X \to X^{**}\) canonical map s.t. 
    \[
        \underset{X^{**}}{<}\tau(x), L \underset{X^*}{>} = Lx \quad \forall \; L \in X^*
    \]
    Thus \(L_n \rightharpoonup L\) weakly in \(X^*\) \(\rightarrow\) \(<\tau(x), L_n> \to <\tau(x), L>\) \(\forall x \in X:\) namely
    \[
        L_n x \to Lx \qquad \forall \; x \in X
    \]
    namely \(L_n \overset{*}{\rightharpoonup}\) weakly* in \(X^*\). In general the converse is false. However
    \begin{proposition}
        If \(X\) is reflexive, then \(L_n \rightharpoonup L\) weakly in \(X^* \iff L_n \overset{*}{\rightharpoonup}\) weakly* in \(X^*\)
    \end{proposition}
    \begin{proof}
        If \(X\) is reflexive, every element \(\Lambda\) of \(X^{**}\) is of type \(\Lambda = \tau(x)\) for some \(x\)
    \end{proof}
    \begin{proposition}
        \(X\) banach space, \(X^*\) dual space, \(L_n \overset{*}{\rightharpoonup} L\) in \(X^*\). Then
        \begin{itemize}
            \item The weak * limit is unique
            \item \(\{L_n\}\) is bounded
            \item \(\norm{L}_{X^*} \leq \liminf _{n \to \infty} \norm{L_n}_{X^*}\)
            \item If in addition \(x_n \to x\) strongly in \(X\) \(\Rightarrow\) \(L_nx_n \to Lx\) 
        \end{itemize}
    \end{proposition}

    \begin{theorem}[Banach Alaoglu]
        \(X \) separable Banach space. Then every bounded sequence in \(X^*\) has a weakly* convergent subsequence. (bounded sets in \(X^* \) sequentially compact for the weak* convergence)
    \end{theorem}

    \begin{proof}
        sparati
    \end{proof}

    \begin{theorem}[Variant of BA for reflexive spaces]
        \(X\) reflexive and Banach. Then every bounded sequence in \(X\) has a weakly convergent subsequence
    \end{theorem}

    \begin{proof}
        For simplicity, we assume that \(X\) is separable (not necessary). \(X \) separable and reflexive \(\Rightarrow X^*\) is separable. \(\tau:X \to X^{**}\) canonical map: it is an isometric isometry.

        \(\{x_n\}\) bounded sequence in \(X \iff \{ \tau(x_n) \}\) is bounded in \(X^{**} = (X^*)^*\)
        
        \(\Rightarrow \) by Banach Alaoglu, \(\exists \{x_{n_k}\}\) s.t. \(\tau(x_{n_k}) \overset{*}{\rightharpoonup} \Lambda \) in \(X^{**}\):
        \[
            \underset{X^{**}}{<}\tau(x_{n_k}), L \underset{X^*}{>} \to \underset{X^{**}}{<}\Lambda, L \underset{X^*}{>}
        \]
    \end{proof}

