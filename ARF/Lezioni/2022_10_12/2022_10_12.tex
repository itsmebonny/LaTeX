\section{Lecture 12/10/2022}
\begin{remark}
    Convergence in measure \(\Rightarrow\) a.e convergence?

    No, not even if \(\mu(X) < +\infty\).

    Consider \(\chi_{n,k} = \chi_{[\frac{k-1}{n}, \frac{k}{n}]}\) with \(n \in \mathbb{N}, k = 1, \ldots, n\)
    \[
        \begin{array}{ccc}
            \chi_{1,1}(x) = \chi_{[0, 1]}(x) & & \\
            \chi_{2,1}(x) = \chi_{[0, \frac{1}{2}]}(x) & \chi_{2,2}(x) = \chi_{[\frac{1}{2}, 1]}(x) & \\
            \chi_{3,1}(x) = \chi_{[0, \frac{1}{3}]}(x) & \chi_{3,2}(x) = \chi_{[\frac{1}{3}, \frac{2}{3}]}(x) &\chi_{3,3}(x) = \chi_{[\frac{2}{3}, 1]}(x) \\
        \end{array}
    \]
    For \(n\) fixed and \(k\) variable, we move the \(\chi\) from the left to right. When the \(\chi\) reaches \(1\), we switch \(n\), and \(\chi\) reappear from the left, being thinner. 
    \[
        \begin{array}{cc}
            \int_{[0,1]} \chi_{n,k} \, d\lambda = \frac{1}{n} & \int_{[0,1]} \chi_{n+1, k} \, d\lambda = \frac{1}{n+1}
        \end{array}
    \]
    We can order the elements of \(\chi_{n,k}\) in a sequence, letting
    \(f_p = \chi_{n,k}\) for \(p=1+2+\ldots+(n-1)+k\). We will prove that \(\left\{ f_p \right\}\) converges in measure, but not a.e. 

    This is the \textbf{typewriter sequence} (\(p(n,k)\)). For every \(x \in [0,1]\) there are \(\infty\) many indexes s.t. \(f_p(x) = 1\) and \(\infty\) many indexes s.t. \(f_p(x) = 0\), meaning that \(\nexists \; \lim_{p\to \infty} f_p(x)\)
    \( f_p \nrightarrow 0 \) a.e. on \(\left[ 0, 1\right]\).

    But we do have convergence in measure to \(0\): \(\alpha \in (0,1)\)
    \[
        \lambda\left(\left\{ \abs{f_{p(n,k)}} \geq \alpha \right\}\right) = \lambda \left(\left[ \frac{k-1}{n}, \frac{k}{n} \right]\right) = \frac{1}{n} \to 0 \mbox{ as } \begin{array}{l}
            n\to \infty \\
            \Updownarrow \\
            p \to \infty        
        \end{array}
    \]
\end{remark}
\begin{remark}
    So, \(f_p \nrightarrow 0\) a.e. on \([0,1]\). But consider \( \{ f_{p(n,1)}: n \in \mathbb{N} \} \). 
    This is a subsequence and, by definition 
    \[ 
        f_{p(n, 1)}(x) = \chi_{n, 1}(x)= \chi_{\left[0, \frac{0}{n} \right]}(x) 
    \] 
    For this subsequence, we have \( f_{p(n,1)}(x) \rightarrow 0 \) as \( n \to\infty \; \forall x \in (0, 1] \), then a.e. on \(\left[0, 1\right]\)
    
    This is not random!

\end{remark}

\begin{proposition}
    If \(\mu(X) < \infty \) and \(f_n \rightarrow f \) in measure, then \(\exists\) a subsequence \(\{f_{n_k} \}\) s.t. \(f_{n_k} \to f \) a.e. on \(X\).
\end{proposition}
Now we analize the relation between convergence in \(L^1(X)\) and the other convergences.

\begin{theorem}
    \( \{f_n\} \subset L^1(X), f \in L^1(X) \). If \(f_n \rightarrow f \) in \(L^1(X)\) then \(f_n \rightarrow f \) in measure on \(X\)
\end{theorem}
\begin{proof}
    By contradiction. Suppose that \(f_n \nrightarrow f \) in measure on X: 
    \( \exists \; \bar{\alpha} > 0 \) s.t. 
    \[ 
        \limsup_{n\to\infty} \mu(\{ |f_n-f| \geq \bar{\alpha} \}) > 0 
    \]
    \(\Rightarrow \exists \; \bar{\epsilon}\) and a subsequence \( \{ f_{n_k} \} \) s.t.
    \[ 
        \mu(\{ |f_{n_k}-f| \geq \bar{\alpha} \}) > \bar{\epsilon} 
    \]
    Consider then 
    \[  
        \begin{array}{l}
        d_1(f_{n_k}, f) 
        = \int_X |f_{n_k} - f| \, d\mu  
        \overset{\text{monot. } \int}{\geq} \int_{\left\{|f_{n_k}-f|\geq \bar{\alpha}  \right\}} \abs{f_{n_k} -f} \, d\mu  \geq\\
        \geq \int_{\left\{|f_{n_k}-f|\geq \bar{\alpha}  \right\}} 1 \, d\mu 
        = \bar{\alpha} \; \mu (\{|f_{n_k } - f| \geq \bar{\alpha}\}) 
        > \bar{\alpha} \; \bar{\epsilon}    
        \end{array}
    \]
    But, by assumption, \(d_1(f_n, f) \rightarrow 0\)
    \[ 
        \Rightarrow d_1(f_{n_k}, f) \rightarrow 0 
    \] 
    Contradiction.
\end{proof}

\begin{remark}
    The convergence in measure doesn't imply the convergence in \(L^1\). \\ For example, consider 
    \[ 
        f_n (x) = n \chi_{\left[0, \frac{1}{n} \right]}(x) 
    \]
    \( \underbrace{\mu \left( \left\{ |f_n| \geq \alpha \right\}\right)}_{= \frac{1}{n}} \to 0 \) for every \(\alpha\) \\
    On the other hand 
    \[ 
        \int _{\left[0, 1\right]} n \chi_{\left[0, \frac{1}{n} \right]} \, d\lambda 
        = \int_{\left[0, \frac{1}{n}\right]} n \, d\lambda 
        = n \frac{1}{n} = 1
    \]
    \( f_n \nrightarrow 0\) in \(L^1\) 
\end{remark}

Convergence a.e. \(\nRightarrow\) convergence in \(L^1\): \\
Use the same example above, \(f_n \rightarrow 0\) a.e. on \([0, 1] \nRightarrow f_n \rightarrow 0\) in \(L^1\)

Convergence in \(L^1\) \(\nRightarrow\) convergence a.e.: \\
Consider the typewriter sequence: \( d_1(f_{p(n, k)}, 0) \to 0\) when \( p \to\infty\) \\
But we don't have a.e. convergence. \\
However, recall the dominated convergence theorem: (DOM)
\[ 
    f_n \rightarrow f \text{ a.e. } + \exists \text{ of a dominating function } \Rightarrow d(f_n, f)\rightarrow 0 
\]
It is also possible to show a reverse DOM: \\
If \(f_n \to f \) in \(L^1(X)\), then \(\exists\) a subsequence \(\left\{f_{n_k}\right\}\) and \(w \in L^1(X)\) s.t. 
\begin{enumerate}
    \item \(f_{n_k} \rightarrow f\) a.e. on X
    \item \( \abs{f_{n_k}(x)} \leq w(x) \) for a.e. \(x \in X\)
\end{enumerate}

\subsection*{Derivatives of measures}
\(\left(X, \mathcal{M}, \mu \right)\) measure space,
\(\oldphi : X \to \left[0, \infty \right]\) measurable.  

We learned that \(\nu: \mathcal{M} \to \left[0, \infty \right]\) by 
\[
    \nu(E)= \int_E \oldphi \, d\mu \mbox{ is a measure on }(X, \mathcal{M})
\] 

If the equation above holds, then we say that \(\oldphi\) is the \textbf{Radon Nykodym derivative} of \(\nu\) with respect to \(\mu\) and we write 
\[
    \oldphi = \frac{d\nu}{d\mu}
\]
\begin{definition}
    \(\mu, \nu  \) measures on \(\left(X, \mathcal{M}\right)\). 
    We say that \(\nu\) is absolutely continuous with respect to \(\mu\), \(\nu << \mu \) if 
    \[
        \mu(E) = 0 \Rightarrow \nu(E)=0
    \]
\end{definition}

\begin{lemma}
    There is a necessary condition:
    \[ 
        \exists \; \frac{d \nu}{d \mu} \Rightarrow \nu << \mu 
    \]
\end{lemma}

\begin{proof}
    \[
        \nu(E) = \int_E \left(\frac{d\nu}{d\mu}\right) \, d\mu = 0
    \] 
    if \(\mu(E)=0\) by basic properties of \(\int\)
\end{proof}

\begin{theorem}[Radon Nykodim Theorem]
    \(\left(X, \mathcal{M}\right) \) measurable space, \(\mu, \nu\) measures. \\
    If \(\nu << \mu \) and moreover \(\mu \) is \(\sigma\)-finite, then \(\oldphi : X \to \left[0, \infty\right]\) measurable s.t.
    \[
        \oldphi = \frac{d \nu}{d \mu} \quad  \text{ namely } \nu(E)= \int_E \oldphi \, d\mu \quad \forall \; E \in \mathcal{M}
    \]
\end{theorem}

\begin{remark}
    If \(\mu\) is not sigma finite the theorem may fail. \\
    In \(\left(\left[0, 1\right], \mathcal{L}\left(\left[0, 1\right]\right)\right)\) consider the counting measure \(\mu = \mu_C\) and the Lebesgue measure \(\nu= \lambda\)
    \(\nu << \mu\) since \(\mu(E)= 0 \Leftrightarrow E= \emptyset \Rightarrow \lambda(E) = \nu(E)=0\) \\
    But we can check that \( \nexists \; \oldphi : \left[0, 1\right] \rightarrow \left[0, \infty \right]\) measurable s.t. \(\lambda(E)= \int_E \oldphi \, d\mu_C\)
\end{remark}

Check by contradiction: assume that \(\oldphi \) does exist, and take \(x_0 \in \left[0, 1\right]\)
\[ 
    0 = \lambda (\left\{x_0\right\}) = \int_{\left\{x_0\right\}} \oldphi \, d\mu_C = \oldphi (x_0) \; \overbrace{\mu_C (\left\{x_0\right\})}^{=1}= \oldphi (x_0)
\]
\(\Rightarrow \oldphi (x_0) = 0 \; \forall \; x_0 \in \left[0, 1\right]\).

But then \(1 = \lambda(\left[0, 1\right]) = \int_{\left[0, 1\right]} 0 \, d\mu_C = 0\). Contradiction \\
Note that \(\mu_C (\left[0, 1\right]) = \infty \) and \(\left( \left[0,1\right], \mathcal{L}(\left[0, 1\right]), \mu_C\right)\) is not \(\sigma\)-finite (\(\left[0,1\right]\) is uncountable)


\subsection*{Product Measure}
\( (X, \mathcal{M}, \mu), (Y, \mathcal{N}, \nu) \) measure spaces.
The goal is to define a measure space on \(X \times Y\)
\begin{definition}
    We call \textbf{measurable rectangle} in \(X \times Y\) a set of type \(A \times B\) where \(A \in \mathcal{M}, B \in \mathcal{N}\)
    \[  R = \{ A \times B \subset X\times Y \text{ s.t. } A \in \mathcal{M}, B \in \mathcal{N}\}\]
    We define the product \(\sigma\)-algebra \(\mathcal{M} \otimes \mathcal{N}\) as \(\sigma_0(R)\). \\
    This is a \(\sigma\)-algebra in \(X \times Y\)
\end{definition}

\begin{definition}
    Let \(E \subset X \times Y \). For \( \bar{x} \in X \) and \(\bar{y} \in Y \) we define
\[ 
    \begin{array}{ll}
        E_{\bar{x}} = \{ y \in Y: \left( \bar{x}, y \right) \in E \} \subseteq Y & \qquad \bar{x} \text{-section of } E \\
        E_{\bar{y}} = \{ x \in X: \left( x, \bar{y} \right) \in E \} \subseteq X & \qquad \bar{y} \mbox{-section of } E \\
    \end{array}
\]
\end{definition}

\begin{proposition}
    \(\left( X, \mathcal{M} \right), \left( Y, \mathcal{N} \right)\) measurable spaces. \(E \in \mathcal{M} \otimes \mathcal{N}\) \\
    Then \(E_x \in \mathcal{M} \) and \(E_y \in \mathcal{N} \) 
    \(\Rightarrow \) we can define 
    \[
    \begin{array}{rlrl}
        \phi : & X \rightarrow \left[ 0, \infty \right] & \qquad  \psi : &Y \rightarrow \left[ 0, \infty \right] \\
                & x \mapsto \nu(E_x) & & y \mapsto \mu(E_y) 
        
    \end{array}    
    \]
\end{proposition}


\begin{theorem}
    If \(\left(X, \mathcal{M}, \mu \right)\) and \(\left(Y, \mathcal{N}, \nu \right)\) are \(\sigma\) finite spaces, then:
    \begin{enumerate}
        \item \(\phi\) is \(\mathcal{M}\)-measurable and \(\psi\) is \( \mathcal{N}\)-measurable
        \item we have that \(\int_X \nu(E_x) \, d\mu = \int_Y \mu(E_y) \, d\nu \)
    \end{enumerate}
\end{theorem}

Using the fact that \(\mu \) and \(\nu\) are measures, and that \(\int\) of non negative function is a measure, we deduce the following

\begin{theorem}[Iterated integrals for characteristic functions]
    \(\mu \otimes \nu : \mathcal{M} \otimes \mathcal{N} \rightarrow \mathbb{R} \) defined by
    \[ \left(\mu \otimes \nu \right)(E) = \int_X \nu(E_x) \, d\mu = \int_Y \mu(E_y) \, d\nu\]
    is a measure, the product measure.
\end{theorem}

\begin{remark}[On the complection of product measure spaces] 
    \(\left(X, \mathcal{M}, \mu \right), \left(Y, \mathcal{N}, \nu \right)\) complete measures spaces. 
    
    In general it is not true that \((X \times Y, \mathcal{M} \otimes \mathcal{N}, \mu \otimes \nu)\) is complete.
\end{remark}
\noindent\underline{Example}: \(X = Y = \real\), \(\mathcal{M} = \mathcal{N} = \mathcal{L}(\real), \mu = \nu = \lambda\).

Given \(A \mbox{ non meas. set }, A \subseteq [0,1], B = \left\{ y_0 \right\}, E = A \times B\). 
If \(E\) were measurable, then its sections must be measurable. But \(E_{y_0} = A\) which is not measurable.

However, \(E\) is negligible:
\[
    E \subseteq [0,1] \times \left\{ y_0 \right\}, \mbox{ and } \left(\lambda \otimes \lambda\right)\left([0,1] \times \left\{ y_0 \right\}\right) = 0
\]
Then \((\lambda \otimes \lambda)\) is not complete 
\[
    \Rightarrow (\real^2, \mathcal{L}(\real) \otimes \mathcal{L}(\real), \lambda \otimes \lambda) \neq (\real^2, \mathcal{L}(\real^2), \lambda_2)
\]
\begin{theorem}
    Let \(\lambda_n\) be the Lebesgue measure in \(\mathbb{R}^n\). 
    If \(n= K+m\), then \(\left(\mathbb{R}^n, \mathcal{L}(\mathbb{R}^n), \lambda_n \right)\) is the complection of \( ( \real^k \times \real^m, \mathcal{L}(\real^k) \otimes \mathcal{L}(\real^m),\lambda_k \otimes \lambda_m )\)
\end{theorem}
