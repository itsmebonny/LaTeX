\section{Lecture 16/11/2022}
Let \(T\) be a linear operator from \(X\) to \(Y\).
\begin{definition}
    The \textbf{kernel} of \(T\) is the set 
    \[
        \ker(T) = \{ x \in X: Tx =0\} \subset X
    \]

\end{definition}

This is a vector subspace of \(X\). 

\(T\) is injective \(\Leftrightarrow \ker (T) = \{0\}\). If \(T\) is continuous, \(\ker(T)\) is closed 
\[
    \ker(T) = T^{-1} (\{0\})
\]
\begin{definition}
    \(X\), \(Y\) normed spaces. \(X\) and \(Y\) are isomorphic if \(\exists \; T \in \mathcal{L}(X, Y)\) bijective, and such that \(T^{-1} \in \mathcal{L}(X, Y)\)
\end{definition}
\begin{definition}
    \(T \in \mathcal{L}(X, Y)\) is an isometry if
    \[
        \norm{Tx}_Y = \norm{x}_X \quad \forall \; x \in X
    \]
\end{definition}
\begin{definition}
    If \(X \subseteq Y\) is a vector subspace, and \(\left(X, \normdot_X\right)\) and \(\left(Y, \normdot_Y\right)\) are normed space, then we can consider 
    \[
        \begin{array}{rcr}
            J: & X \to Y & \text{(inclusion map)}
            \\ & x \mapsto x
        \end{array}
    \] 
    If \(J \in \mathcal{L}(X, Y)\) (namely, if \(\exists \; M>0 \) s.t. \(\norm{x}_Y \leq M \norm{x}_X\) \(\forall \; x \in X\)), 
    then we say that \(J\) is an embedding of \(X\) into \(Y\), and we write \(X \hookrightarrow Y\)
\end{definition}

\noindent\underline{Ex}: \(\mu(X) < \infty \), \(1 \leq p < q \leq \infty\)
\[
    L^q(X) \hookrightarrow L^p(X) \tag*{(inclusion of \(L^p\) spaces)}
\]

\subsection*{Some fundamentals theorems on linear operators}
\begin{definition}
    \((X, d)\) metric space. \(A \subset X\). \(x \in X \) is an \textbf{adherence point} of \(A\) if \(\forall \; r>0: B_r(x) \cap A \neq \emptyset\)
    \[
        \bar{A} = \{ x \in X: x \text{ is an adherence point of } A \} = A \cup \partial A
    \]
\end{definition}
\begin{definition}
    \(A \subset X\) is dense in \(X\) if \(\bar{A} = X\).
\end{definition}
For example, \(\mathbb{Q}\) is dense in \(\real\), and \((a, b)\) is dense in \([a, b]\).
\begin{definition}
    \(A \subset X\) is nowhere dense if the interior of the closure of \(A\) is empty, namely
    \[
       \text{int} (\bar{A}) = \interior{\bar{A}} = \emptyset  
    \]
\end{definition}

Ex: \(\interior{\bar{\{x\}}} = \interior{\{x\}} = \emptyset\)

\(\mathbb{Z} \subset \real\): \(\interior{\bar{\mathbb{Z}}}=\interior{\mathbb{Z}} = \emptyset\)

\(\mathbb{Q} \) is not nowhere dense: \(\interior{(\bar{\mathbb{Q}})} = \interior{(\mathbb{R})} = \real \)
\begin{definition}
    \(A \subset X\) is called \textbf{of first category} (or \textbf{meager set}) in \(X\) if \(A\) is the (at most) countable union of nowhere dense sets. 
\end{definition}

Ex: \(\mathbb{Q}\) is of first category in \(\real\): countable union of nowhere dense sets
\[
    \mathbb{Q} = \bigcup_{q \in \mathbb{Q}} \{q\}
\]
\begin{definition}
    \(A \subset X\) is of second category if it is not of first category.
\end{definition}
\begin{theorem}[Baire category theory]
    \((X, d)\) complete metric space. Then 
    \begin{itemize}
        \item \(\{U_n\}_{n=0}^\infty\) is a sequence of open and dense sets in \(X\) \(\Rightarrow \cap_{n=0}^\infty U_n\) is dense in \(X\).
        \item \(X\) is of second category in itself: \(X\) cannot be the countable union of nowhere dense sets. 
    \end{itemize}
\end{theorem}


\noindent\underline{Preliminaries}:
\begin{itemize}
    \item \(A \subset X\) is dense \(\Leftrightarrow \forall \; W \subset X\), \(W\) open, \(W \neq \emptyset\), we have that \(A \cap W \neq \emptyset\)
    \item \(A\) is nowhere dense \(\Leftrightarrow \left(\bar{A}\right)^C\) is open and dense
\end{itemize}

\begin{proof} Here's the proof of the two parts of the theorem:
    \begin{itemize}
        \item[(a)] Thanks to the first preliminary, we show that \(\forall W \subset X\) open and non empty we have \((\cap_n U_n) \cap W \neq \emptyset\)
        
        \[
            \begin{array}{rl}
                U_0 \text{ is open and dense:} & \overset{1^{st}\text{\tiny{prel.}}}{\Rightarrow} \underbrace{U_0 \cap W}_{\text{is open}} \neq \emptyset \\
                & \Rightarrow \text{it contains an open ball} \\
                & \Rightarrow (U_0 \cap W) \supset B_{r_0}(x_0) \mbox{ for some } x_0 \in X \mbox{ and } r_0 > 0
        \end{array}
        \] 
        For \(n>0\), we choose \(x_n \in X\) and \(r_n > 0\) inductively in the following way: we have 
        \[
            U_n \cap B_{r_{n-1}}(x_{n-1}) \neq \emptyset 
        \tag*{(\(1^{st}\) prel. \(+ U_n\) is dense)}\]
        \[
            \Rightarrow \overline{B_{r_n}(x_n)} \subset \underset{\begin{array}{l}\text{\small{all these balls}} \\ \text{\small{are included in}} \\
            \displaystyle{B_{r_0}(x_0)}\end{array}}{(U_n \cap B_{r_{n-1}}(x_{n-1}))}
        \]
        with \(x_n \in X\) and \(0 < r_n < \frac{1}{2^n}\)

        By the condition on \(r_n\), we see that 
        \[
            x_n, x_m \in B_{r_N}(x_N) \quad \forall \; n,m > N
        \]
        \(\Rightarrow \left\{ x_n \right\}\) is a Cauchy sequence in \(X\)
        \[
            d(x_n, x_m) \leq \frac{1}{2^N} \quad \forall \; n,m > N
        \]
        \(X\) is complete: \(x_n \overset{d}{\rightarrow} x \in X\)
        Since 
        \[
            \begin{array}{lr}
                x_n \in B_{r_N}(x_N) & \forall \; n > N \\
                \Rightarrow x = \lim_n x_n \in \overline{B_{r_N}(x_N)} \subset (U_n \cap B_{r_0}(x_0)) \subset (U_N \cap W) & \forall \; n \in \mathbb{N} \\
                \Rightarrow x_n \in \bigcap_n (U_n \cap W) = \left(\bigcap_n U_n\right) \cap W
            \end{array}
        \]
        This means that \(\bigcap_n U_n\) is dense.
        \item[(b)] It follows from (a):
        
        If \(\left\{ E_n \right\}\) is a sequence of nowhere dense sets in \(X\), then, by the second preliminary \(\left\{ (E_n)^C \right\}\) is a sequence of open and dense sets. By (a) 
        \[
            \begin{array}{l}
                \bigcap_n (\overline{E_n})^C \neq \emptyset \\
                \Rightarrow \bigcup_n E_n \subset \bigcup_n \overline{E_n} = X \setminus \overset{= \emptyset}{\left(\bigcap_n (\overline{E_n})^C\right)} \neq X \\

            \end{array} 
        \]
    \end{itemize}
\end{proof}
\noindent\underline{Ex}: \((X, \normdot)\) \(\infty-\dim\) Banach space. \(\left\{ e_i \right\}_{i \in I}\) Hamel basis. 

Then \(I\) is uncountable.
\begin{theorem}[Banach Steinhaus]
    \(X\) Banach space, \(Y\) normed space, \(\mathcal{F} \subseteq \mathcal{L}(X, Y)\) family.
    Suppose that \(\mathcal{F}\) is pointwise bounded: 
    \[
        \forall\; x \in X \quad \exists \; M_x >0 \text{ s.t. } \sup_{T \in \mathcal{F}} \norm{Tx}_Y \leq M_x \tag*{(PB)}
    \]
    Then \(\mathcal{F} \) is uniformly bounded: 
    \[
        \exists \; M \geq 0 \text{ s.t. } \sup_{T \in \mathcal{F}} \norm{T}_{\mathcal{L}(X, Y)} \leq M \tag*{(UB)}
    \]
\end{theorem}
\begin{proof}
    \(\forall \; n \in \natural\), let 
    \[
        C_n := \{ x \in X: \norm{Tx}_Y \leq n \quad \forall \; T \in \mathcal{F} \} 
        = \cap_{T \in \mathcal{F}} \{ x \in X: \norm{Tx}_Y \leq n \}
    \]
    \(C_n\) is a closed set \(\forall \; n\), since \(T\) is continuous. (also \(\phi: X \to \real \) \(\phi(x)=\norm{Tx}_Y\) is continuous)

    By (PB), every \(x \in X\) stays in some \(C_n\): \(X = \cup_{n=1}^\infty C_n\). 
    Since \(X\) is Banach, by the Baire theorem it is necessary that \(\exists \; n_0 \in \natural\) s.t. \(\interior{C_{n_0}} \neq \emptyset \Rightarrow\) a ball \(\overline{B_r(x_0)} \subset C_{n_0}\): then
    \[
        n_0 \geq \norm{T(x_0 + rz)}_Y \overset{\text{\tiny{linearity}}}{=} \norm{Tx_0 + rTz}_Y \overset{\text{\tiny{triangle ineq}}}{\leq} r\norm{Tz}_Y - \norm{Tx_0}_Y \quad \forall \; T \in \mathcal{F} \; \forall \; z \in \bar{B_1(0)}
    \]
    To sum up: \(\forall \; T \in \mathcal{F} \), \(\forall \; z \in \bar{B_1(0)}\) we have 
    \[
        r \norm{Tz}_Y - \norm{Tx_0}_Y \leq n_0 \Rightarrow \norm{Tz}_Y \leq \frac{1}{r}(n_0 + M_{x_0})
    \]
    We take sup over \(T \in \mathcal{F}\):
    \[
        \sup_{T \in \mathcal{F}} \norm{T}_{\mathcal{L}(X, Y)} \leq \frac{1}{r} \left(n_0 + M_{x_0}\right) =: M
    \]
\end{proof}

\begin{corollary}
    \(X\) Banach space, \(Y\) normed space. \(\{ T_n \} \subseteq \mathcal{L}(X, Y)\) s.t. \(\{T_n x\}\) have a limit, denoted by \(Tx\), \(\forall \; x \in X\) (pointwise convergence). 
    Then \(T \in \mathcal{L}(X, Y)\)
\end{corollary}
\begin{proof}
    \(T\) is linear: 
    \[
        \begin{array}{ccc}
            T_n(\alpha_1 x_1 + \alpha_2 x_2 ) & = & \alpha_1 T_n x_1 + \alpha_2 T_n x_2 \\
            \downarrow && \downarrow \\
            T(\alpha_1 x_1 + \alpha_2 x_2) & = & \alpha_1 Tx_1 + \alpha_2 Tx_2
        \end{array}
    \]
    Now we observe that we have (PB): if \(\{ T_n x \}\) is convergent \(\Rightarrow \{T_n x\} \) is bounded \(\Rightarrow \) by Banach Steinhaus, \(\{ T_n \}\) is uniformly bounded: 
    \[
        \exists M>0 \text{ s.t. } \sup_n \norm{T_n}_{\mathcal{L}(X, Y)} \leq M 
    \]
    Therefore, \(\forall \; x \in X\):
    \[
        \norm{Tx}_Y = \norm{\lim_n (T_n x)}_Y = \lim_n \norm{T_n x}_Y \leq \lim_n \norm{T_n}_\mathcal{L} \norm{x}_X \leq \lim_n M \norm{x}_X  = M \norm{x}_X
    \]
    Thus, \(T\) is bounded: \(T \in \mathcal{L}(X, Y)\)
\end{proof}