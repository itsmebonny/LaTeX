\section{Lecture 20/10/2022}
\begin{theorem}[The \(2^{nd}\) fundamental theorem of calculus.]
    \(G \in \mbox{AC}([a,b]) \Leftrightarrow\) \(G\) is differentiable a.e. on \([a,b]\), \(G' \in L^1([a,b])\), and (FC) holds.
\end{theorem}

Example and comments:
\begin{itemize}
    \item If \(f\) is bounded and monotone \(\Rightarrow\) \(f \in \mbox{BV}\)
    \[
        V_a^b (f) = \abs{f(b) - f(a)}
    \]
    Note that \(f\) may not be continuous
    \[
        f(x) = \begin{cases}
            1 & x \geq 0 \\
            0 & x < 0
        \end{cases}
        \Rightarrow f \in \mbox{BV}([-1,1])
    \]
    \item \(f \in \mbox{BV}([a,b]) \Rightarrow f\) is bounded. Indeed
    \[
        \sup_{x \in [a,b]} \abs{f(x)} \leq \abs{f(x)} + V_a^b(f) \overset{f \in \text{BV}}{<} +\infty
    \]
    \item \(f\) is continuous on \([a,b]\), or even if  \(f\) is differentiable everywhere in \([a,b]\) \(\nRightarrow f \in \mbox{BV}([a,b])\)
    \[
        f(x) = \begin{cases}
            x^2\cos{\frac{2\pi}{x^2}} & x \in (0,1] \\
            0 & x = 0
        \end{cases}
    \]
    It is continuous in \([0,1]\), but \(f \notin \mbox{BV}([0,1])\)
    \item \(f \in \mbox{BV}([a,b]) \cap \mbox{UC}([a,b])  \nRightarrow f \in \mbox{AC}([a,b])\)
    \[\begin{array}{l}
        v \mbox{ a Vitali-Cantor function} \\ v \mbox{ is bounded and monotone} \\ 
        v \in \mbox{UC}([0,1]) \\ 
        
    \end{array}
    \Rightarrow v \in \mbox{BV}([0,1]) 
    \]
    But \(v \not \in \mbox{AC}([0,1])\)
    \item If \(f \in \mbox{BV}([a,b]) \Rightarrow f\) is differentiable a.e. on \([a,b]\), and \(f' \in L^1([a,b])\)
\end{itemize}
We can now come back to the proof of Lemma 1 of the last lesson.

\noindent\underline{Preliminary result}: \(A \in \real\) open. Then 
\[
    A = \bigcup_{n=1}^{\infty}(a_n, b_n) \mbox{ disjoint}
\]
    any open set of \(\real\) is the (at most) countable union of open disjoint intervals.

\noindent\underline{Preliminary result (equivalent definition for AC)}: \(f \in \mbox{AC}([a,b]) \Leftrightarrow \forall \; \epsilon > 0 \exists \; \delta > 0\) depending on \(\epsilon\) s.t. 
\[
    \forall \; \bigcup_{n=1}^{\infty} [a_n, b_n], \quad [a_n, b_n] \mbox{ have disjoint interiors}
\]
\[
    \sum_{n=1}^{\infty} (b_n -a_n) < \delta \Rightarrow \sum_{n=1}^{\infty} \abs{f(b_n) - f(a_n)} < \epsilon
\]
\begin{proof}
    We defined \(\lambda\) starting from two properties
    \begin{itemize}
        \item invariance under translations
        \item \(\lambda((x,y)) = y - x \quad \forall \; a \leq  y \leq b\)
    \end{itemize}
    Now, \(G\) is monotone, say \(G\) increasing (if \(G \searrow\), take \(-G\)). We can repeat the construction of \(\lambda\) in order to obtain a measure \(\mu\) s.t. 
    \begin{itemize}
        \item \(\mu\) is invariant under translations
        \item \(\mu((x,y)) = \underbrace{G(y) - G(x)}_{\geq 0}\) \(\forall \; a \leq x < y \leq b\) (for \(\lambda\), take \(G(t) = t\))
    \end{itemize}
It can be proved that we obtain a measure on \((\real, \mathcal{L}(\real))\), complete.

On \((\real, \mathcal{L}(\real))\) we have two measures: \(\lambda \mbox{ and } \mu\).

\noindent\underline{Idea}: We take these measures on \(([a,b], \mathcal{L}([a,b]))\), and we want to show that \(\exists \; \frac{d\mu}{d\Lambda}\) (Radon-Nikodym)

We can check the hypothesis of the Radon-Nikodym theorem:
\begin{itemize}
    \item \(\lambda\) is \(\sigma\)-finite: \(\lambda([a,b]) = b-a < +\infty\)
    \item \(\mu << \lambda\): \(E \in \mathcal{L}([a,b])\), \(\lambda(E) = 0 \Rightarrow \mu(E) = 0\)
\end{itemize}
Assume \(\lambda(E) = 0\). \(G\) is \(\mbox{AC}([a,b])\): then \(\forall \; \epsilon > 0\) \(\exists \; \delta = \delta(\epsilon) > 0\) s.t. 
\[
    \forall \; \bigcup_{n=1}^{\infty} [a_n, b_n], \quad [a_n, b_n] \mbox{ have disjoint interiors}
\]
\[
    \lambda\left(\bigcup_{n=1}^{\infty} [a_n, b_n]\right) < \delta \Rightarrow \sum_{n=1}^{\infty} \abs{G(b_n) - G(a_n)} < \epsilon
\]
Take this \(\delta\). By regularity of \(\lambda\), \(\exists \; A \) open set of \(\left[a, b\right]\)s.t. \(A \supset E\) and \(\lambda(A) < \delta\)
\[
    A \mbox{ is open} \Rightarrow A = \left( \bigcup_{n=1}^{\infty} I_n \symbolfootnotemark[2]{\mbox{open intervals } = (x, y)} \right), \mbox{ disjoint}
    \symbolfootnotetext[2]{open intervals = \((x_n, y_n)\)}
\] 
it is a countable union of open intervals (maybe two of them contains \(a\) or \(b\))
\[
    \lambda(A) < \delta \Leftrightarrow \sum_{n=1}^{\infty} (y_n - x_n) < \delta
\]
But then, since \(\mu\) is a measure it is countably additive
\[
    \mu (E) \leq \mu(A) = \sum_n \mu (I_n) = \sum_n G(y_n) - G(x_n) < \epsilon
\]
by the choice of \(\delta \) and the fact that \(G \in\) AC.
We proved that 
\[
    \lambda(E) = 0 \Rightarrow \forall \; \epsilon >0 : \, \mu(E) < \epsilon \Rightarrow \mu(E) =0
\]
So \(\mu << \lambda\). We can apply Radon Nikodym \(\exists \; \oldphi : \left[ a, b\right] \to \left[0, \infty\right]\) s.t. 
\[
    G(x) - G(a) = \int_a^x \oldphi \, d\lambda
\]
Since \(G\) is bounded, then \(\oldphi \in L^1(\left[a, b\right])\)
\[
    G(x)=G(a) + \int_a^x \oldphi(t) \, dt
\]
By the first fundamental theorem of calculus, this is differentiable a.e. 
\[
    \Rightarrow G'(x) = \oldphi(x) \text{ a.e. on } \left[a, b\right]
\]
\[
    \Rightarrow G'(x) = G(a) + \int_a^x G'(t) \, dt
\]
\end{proof}
Now we want to get rid of the additional assumption (monotonicity).

\noindent\underline{Preliminary result}: \(f \in \mbox{BV}([a,b])\). Then
\[
    \phi(x) = V_a^x(f), \quad \forall \, x \in [a,b] 
\]
is an increasing function.
\begin{proof}
    By \(a \leq x < y \leq b\). Then
    \[
        V_a^y (f) = V_a^x (f) + \underbrace{V_x^y(f)}_{\geq 0} \geq V_a^x(f)
    \]
\end{proof}
\noindent\underline{Preliminary result}: If \(G \in \mbox{AC}([a,b])\), then \(G \in \mbox{BV}([a,b])\), and moreover 
\[
    \phi(x) = V_a^x(G) \mbox{ is in AC}([a,b])
\]
\begin{proof}[Proof of the second fundamental theorem of calculus in the general case]\(G \in \mbox{AC}([a,b])\)


    We want to write \(G= G_1 + G_2\) where \(G_1 \nearrow \) and \(G_2 \searrow\), both AC. 

    Then the second fundamental theorem holds for \(G_1 \) and \(G_2\) so it holds for \(G\) by linearity of the integral. 
    
    We pose:
    \[
        G_1(x) = \frac{G(x) + V_a^x(G)}{2}
    \]
    \[
        G_2(x) = \frac{G(x) - V_a^x(G)}{2}
    \]
    Clearly, \(G_1+G_2 = G\), \(G_1, G_2\) are AC, by the last preliminary result.

    \noindent\underline{\(G_1 \nearrow\)}: Let \(a \leq x < y \leq b\)
    \[
        \abs{G(y) - G(x)} \leq V_x^y(G)
    \]
    Therefore, 
    \[
        G_1 (y) - G_1(x) = \frac{1}{2} (\underbrace{G(y) - G(x)}_{\begin{array}{l}\geq - \abs{G(y) - G(x)} \\ \geq - V_x^y(G)\end{array}} + V_a^y(G) + V_a^x(G))  \geq \frac{1}{2}(-V_x^y(G) + V_x^y(G)) = 0
    \]
    
       
    
    So \(G_1\) is decreasing. In an analogue way, we can prove that \(G_2\) is decreasing.
\end{proof}

\section*{Functional analysis}
Normed spaces and Banach spaces
\begin{definition}
    Given \(X\) vector space, a norm on \(X\) is a function \(\norm{\cdot} : X \to [0, \infty)\) s.t.
    \begin{itemize}
        \item \(\norm{x} = 0 \Leftrightarrow x = 0\)
        \item \(\forall \; \alpha \in \real, \forall \; x \in X\) : 
        \[
            \norm{\alpha x} = \abs{\alpha}\norm{x}  \tag*{(positive homogeneity)}
        \]
        \item \(\forall \; x,y \in X\): 
        \[
            \norm{x+y} \leq \norm{x} + \norm{y} \tag*{(triangle inequality)}
        \]
    \end{itemize}
    Then, \((X, \normdot)\) is called a \textbf{normed space}
\end{definition}
\noindent\underline{Ex}: \(\abs{\norm{x} - \norm{y}} \leq \norm{x - y}\) \(\forall \; x,y \in X\)
\begin{proposition}
    \((X, \normdot)\) normed space. Then \((X, d)\) is a metric space for 
    \[
        d(x,y) = \norm{x - y}
    \]
\end{proposition}
\begin{remark}
    Normed space \(\underset{\displaystyle\nleftarrow}{\rightarrow}\) metric space
\end{remark}
\noindent\underline{Examples}:

\begin{itemize}
    \item \(\real^N\)
    \[
        \norm{x}_p := (\sum_{i = 1}^N \abs{x_i}^p)^{\frac{1}{p}} \; \; \forall p \in [1, +\infty) \quad
        \norm{x}_{\infty} := \max_{i = 1, \ldots, N} \abs{x_i}
    \]
    \item \(\mathcal{C}^0([a,b])\) \[\norm{f}_{\infty} := \max_{x \in [a,b]} \abs{f(x)}\]
    \item \(L^1(X, \mathcal{M}, \mu)\) \[\norm{f}_1 := \int_X \abs{f} \, d\mu\] This is a norm in \(L^1\), but not on \(\mathcal{L}^1\) \((\int_x \abs{f} \, d\mu = 0 \Rightarrow f = 0 \mbox{ a.e.})\)
    \item \(L^{\infty}(X, \mathcal{M}, \mu)\) \[\norm{f}_{\infty} := \underset{[a,b]}{\esssup} \abs{f}\]
\end{itemize}
\((X, \normdot)\) normed space \(\to\) \((X, d)\) metric space \(\to\) convergent sequences on \(X\): \(\left\{ x_n \right\} \subset X\) is convergent in \(X\) iff 
\[
    d(x_n, x) \to 0 \Leftrightarrow \norm{x_n -x} \to 0 \mbox{ as } n \to \infty
\]
\noindent\underline{Ex}: \(x_n \to x\) in \(X\), then \(\norm{x_n} \to \norm{x}\) (the norm is a continuous function on \(X\))
\begin{definition}
    \(\left\{ x_n \right\}\) is a \textbf{Cauchy sequence} in \((X, \normdot)\) if \(\forall \; \epsilon > 0\) \(\exists \; \bar{n} \in \mathbb{N}\) s.t. 
    \[
        n,m > \bar{n} \Rightarrow \norm{x_n-x_m} < \epsilon
    \]
\end{definition}
\begin{definition}
    \((X, \normdot)\) is called a \textbf{Banach space} if \((X, d)\) is complete, namely if any Cauchy sequence in \((X, d)\) is convergent.
\end{definition}
If \((X, \normdot)\) is a normed space, we can speak about series in \(X\). Let \(\left\{ x_n \right\} \subset X\) and \(s_n = x_0 + x_1 + \ldots + x_n\), then \(\sum_{n=0}^{+\infty} x_n = \left\{ s_n \right\}\). 

Then \(\sum x_n\) is convergent if \(\left\{ s_n \right\}\) is convergent. If \(\sum x_n\) is convergent, we write 
\[
    s = \sum_{n = 0}^{+\infty} x_n \Leftrightarrow s_n \to s
\]
For numerical series
\[
    \sum_{n=1}^{\infty} \abs{a_n} < +\infty \Rightarrow \sum a_n \mbox{ is convergent}
\]
In general, in normed spaces 
\[
    \sum_{n=1}^{\infty} \norm{x_n} < +\infty \nRightarrow \sum_{n=1}^{\infty} x_n \mbox{ is convergent}
\]
\subsubsection*{Characterization}
\((X, \normdot)\) is a Banach space \(\Leftrightarrow\) every series s.t. \(\sum \norm{x_n} < +\infty\) is also s.t. \(\sum x_n\) is convergent