\section{Lesson 29/09/2022}
\begin{remark}
    On the relation between \((\real, \boreal, \lambda)\) and  \((\real, \mathcal{L}(\real), \lambda)\) = (\(\lambda =\) Lebesgue measure)

    \((\real, \boreal, \lambda)\) is not complete. In fact, \((\real, \mathcal{L}(\real), \lambda)\) is the completion of \((\real, \boreal, \lambda)\).

    Note that, \(\forall \; E \in \mathcal{L}(\real) \exists \; \mbox{ a } G_{\delta}-\mbox{set } A\) and an \(F_{\delta}-\mbox{set } B\) s.t.
    \[
        \begin{array}{l}
            A \supset E \mbox{ and } \lambda(A \backslash E) = 0 \\
            B \subset E \mbox{ and } \lambda(E \backslash B) = 0
        \end{array}
    \]
\end{remark}
\((X, \mathcal{M}, \mu)\) a complete measure space. Let \(P(x)\) be a proposition depending on \(x \in X\). We say that \(P(x)\) is true \((\mu-)\)almost everywhere if 
\[
    \mu\left(\left\{ x \in X : P(x) \mbox{ is false }\right\}\right) = 0
\]
\(P(x)\) is true \(\underset{(\mu-\text{a.e.})}{\mbox{a.e.}}\) on \(X\).

\underline{Ex}: \((\real, \mathcal{L}(\real), \lambda)\), \(f(x) = x^2\).  
Then \(f(x) > 0\) a.e. on \(\real\) (for a.e. \(x\)):
\[
    \left\{ f(x) \leq 0 \right\} = \left\{ 0 \right\}, \mbox{ and } \lambda(\left\{ 0 \right\}) = 0
\]
\((\real, \mathcal{P}(\real), \mu_C)\) with \(\mu_C\) counting measure. Then it is not true that \(f(x) > 0\) \(\mu_C\)-a.e. 
\[
    \mu_C \left(\left\{ 0 \right\}\right) = 1
\]
It will be useful to consider sequences converging a.e.: 
\[
    f_n \to f \qquad \mbox{a.e. on }X_{\mbox{mbox} \text{text}}
\]
if \(\mu\left( \left\{ x \in X : \lim_n f_n(x) \neq f(x), \mbox{ or does not exist } \right\}\right) = 0\)
\begin{proposition}
    \((X, \mathcal{M}, \mu)\) complete measure space. 
    \begin{enumerate}
        \item \(f: X \to \real\) is measurable, and \(g = f \) a.e. on \(X\), then \(g\) is measurable
        \item \(f_n \to f\) a.e. on \(X\), \(f_n : X \to \real\) measurable for all \(n\), then \(f\) is measurable
    \end{enumerate}
\end{proposition}
\subsubsection*{Integration of non-negative functions}
\underline{Notation}: \[
    \begin{array}{c}
        \left\{ x \in X : f(x) \geq 0 \right\} = \left\{ f \geq 0 \right\} \\
        \left\{ x \in X : f(x) > 0 \right\} = \left\{ f > 0 \right\}   \\
        \vdots
    \end{array}
    \]
\((X, \mathcal{M}, \mu)\) complete measure space.
We consider measurable functions \(f: X \to [0, +\infty]\)

\underline{Convention}: we define 
\[
    \begin{array}{l}
        a + \infty = +\infty \quad \forall \; a \in \real \\
        a \cdot (+\infty) = \begin{cases}
            +\infty & \mbox{if } a \neq 0, a > 0 \\
            0 & \mbox{if } a = 0
        \end{cases}        
    \end{array}
\]
With this convention, \(+ and \cdot\) of measurable functions are measurable functions.
\begin{definition}
    Let \(s: X \to [0, +\infty]\) be a measurable simple function, 
    \[
        s(x) = \sum_{n=1}^m a_n \chi_{D_n}(\bar{x})
    \]
    where \(D_1,\ldots,D_m\) are measurable, disjoint, and \(\bigcup_{n=1}^m D_n = X\). Let also \(E \in \mathcal{M}\). Then we define 
    \[
        \int_E s \, d\mu := \sum_{n=1}^m a_n \mu(D_n \cap E)
    \]
\end{definition}
\begin{remark}
    \[s:[a,b] \to \real, \lambda = \mu \Longrightarrow \int_E s \, d\mu \mbox{ is the area under the curve}\]
\end{remark}
\begin{remark}
    There are several points:
    \begin{itemize}
        \item In the definition we have already used the convention \(\mu(D_n \cap E = +\infty) \quad \mbox{ for some }n\)
        \item \(E \in \mathcal{M} \Longrightarrow \chi_E\) is a simple function
        \[
            \chi_E(x) = 1 \cdot \chi_E + 0 \cdot \chi_{X\backslash E}(x)
        \] 
        In this case 
        \[
            \int_X \chi_E \, d\mu = 1\cdot \mu(E) + 0 \cdot \mu(X\backslash E) = \mu(E)
        \]
        \item \(s\chi_E = \sum_{n=1}^m a_n\chi_{D_n \cap E} \Longrightarrow \int_E s\, d\mu = \int_X s\chi_E \, d\mu\)
    \end{itemize}
\end{remark}
\begin{definition}
    \(f:X \to [0, +\infty]\) measurable, \(E \in \mathcal{M}\). The \textbf{Lebesgue integral} of \(f\) on \(E\), with respect to (w.r.t.) \(\mu\), is 
    \[
        \int_E f \, d\mu = \sup \left\{ \int_E d\mu \vert \begin{array}{l}s\text{ is simple} \\ 0 \leq s \leq f \end{array}\right\}
    \]
\end{definition}