\section{Lecture 15/09/2022}
The elements of \(\overline{\mathcal{M}}\) are sets of the type \(E \cup N\), with \(E \in \mathcal{M}\) and \(\bar{\mu}(N) = 0\).
\subsubsection*{Outer measure}
We wish to define a measure \(\lambda\) ``on \(\real\)'' with the following properties:
\begin{enumerate}
    \item \(\lambda((a,b)) = b-a\)
    \item \(\lambda(E + t)\symbolfootnotemark[2]{\(\left\lbrace x \in \mathbb{R} : x=y+t, \mbox{ with } y \in E\right\rbrace\)}  \; = \lambda(E)\) for every measurable set \(E \subset \mathbb{R}\) and \(t \in \mathbb{R}\)
\end{enumerate}
It would be nice to define such a measure on \(\mathcal{P}(\mathbb{R})\). In such case, note that \(\lambda(\left\lbrace x \right\rbrace) = 0\), \(\forall \; x \in \mathbb{R}\)
But then 
\begin{theorem}[Ulam]
    The only measure on \(\mathcal{P}(\mathbb{R})\) s.t. \(\lambda(\left\lbrace x \right\rbrace) = 0 \quad \forall \; x\) is the trivial measure. Thus, a measure satisfying the two properties of the outer measure cannot be defined on \(\mathcal{P}(\real)\)
\end{theorem}
We'll learn in what follows how to create a measure space on \(\real\), with a \(\sigalg\) including all the Borel sets, and a measure satisfying properties of the outer measure. This is the so called \textbf{Lebesgue measure}.
\begin{definition}
    Given a set \(X\). An \textbf{outer measure} is a function \(\mu^* : \mathcal{P}(\mathbb{X}) \to [0, +\infty]\) s.t. 
    \begin{itemize}
        \item \(\mu^*(\emptyset) = 0\)
        \item \(\mu^*(A) \leq \mu^*(B)\) if \(A \subseteq B\) (Monotonicity)
        \item \(\mu^*(\bigcup_{n=1}^{\infty} E_n) \leq \sum_{n=1}^{\infty} \mu^*(E_n)\) (\(\sigma\)-subadditivity)
    \end{itemize}
\end{definition}
The common way to define an outer measure is to start with a family of elementary sets \(\mathcal{E}\) on which a notion of measure is defined (e.g. intervals on \(\real\), rectangles on \(\real^2, \ldots\)) and then to approximate arbitrary sets from outside by \textbf{countable} unions of members of \(\mathcal{E}\).
\begin{proposition}
    Let \(\mathcal{E} \subset \mathcal{P}(\mathbb{R})\) and \(\rho : \mathcal{E} \to [0, +\infty]\) be such that \(\emptyset \in \mathcal{E}, X \in \mathcal{E}\) and \(\rho(\emptyset) = 0\). For any \(A \in \mathcal{P}(X)\), let 
    \[\mu^*(A) := \inf \left\lbrace \sum_{n=1}^{\infty} \rho (E_n) : E_n \in \mathcal{E} \mbox{ and } A \subset \bigcup_{n=1}^{\infty} E_n \right\rbrace\]
    Then \(\mu^*\) is an outer measure, the outer measure generated by \((\mathcal{E}, \rho)\).
\end{proposition}
\begin{proof}
    \(\forall \; A \subset X \; \exists \left\lbrace E_n \right\rbrace \subset \mathcal{E}\) s.t. \(A \subset \bigcup_n E_n : \mbox{ take } E_n = X \; \forall \; n,\)
    then \(\mu^*\) is well defined. Obviously, \(\mu^*(\emptyset) = 0\) (with \(E_n = \emptyset \quad \forall\; n\)), and \(\mu^*(A) \leq \mu^*(B)\) for \(A \subset B\) (any covering of \(B\) with elements of \(\mathcal{E}\) is also a covering of \(A\).)

    We have to prove the \(\sigma\)-subadditivity. 
    
    \noindent Let \(\left\lbrace A_n \right\rbrace_{n \in \mathbb{N}} \subseteq \mathcal{P}(X)\) and \(\epsilon > 0\). For each \(n, \, \exists \left\lbrace E_{n_j} \right\rbrace_{j \in \mathbb{N}} \in \mathcal{E}\) s.t. \(A_n \subset \bigcup_{i = 1}^{\infty} E_{n_j}\) and \(\sum_{j=1}^{\infty} \rho(E_{n_j}) \leq \mu^*(A_n) + \frac{\epsilon}{2^n}\).  
    But then, if \(A = \bigcup_{n=1}^{\infty} A_n\), we have that \(A \subset \bigcup_{n,j \in \mathbb{N}^2} E_{n_j}\) and
    \[
        \mu^*(A) \leq \sum_{n,j} \rho(E_{n_j}) \leq \sum_{n} \left(\mu^*(A_n) + \frac{\epsilon}{2^n}\right) = \sum_{n} \mu^*(A_n) + \epsilon
    \]
    Since \(\epsilon\) is arbitrary, we are done.
\end{proof}
Ex:  
\begin{enumerate}
    \item \(X \in \mathbb{R}, \mathcal{E} = \left\lbrace (a,b) : a \leq b, a,b \in \mathbb{R} \right\rbrace \mbox{ family of open intervals:} \)
    \[
        \rho((a,b)) = b-a
    \]
    
    \item \(X = \mathbb{R}^n, \mathcal{E} = \left\lbrace (a_1, b_1) \times \ldots \times (a_n, b_n) : a_i \leq b_i, a_i, b_i \in \mathbb{R} \right\rbrace\): 
    \[
        \rho((a_1, b_1)\times \ldots \times (a_n, b_n)) = (b_1 -a_1) \cdot \ldots \cdot (b_n - a_n)
    \]
\end{enumerate}
\begin{remark}
    \(E \in \mathcal{E} \Rightarrow \mu^*(E) = \rho(E)\).  

    In examples 1 and 2, we have in fact 
    \[
        \mu^*((a,b)) = b-a, \mu^*\left((a_1, b_1) \times \ldots \times (a_n, b_n)\right) = \prod_{i=1}^{n} (b_i - a_i)
    \] 
\end{remark}

To pass from the outer measure to a measure there is a condition: 
\begin{definition}[Caratheodory condition]
    If \(\mu^*\) is an outer measure on \(X\), a set \(A \subset X\) is called \(\mu^*\)-\textbf{measurable} if 
    \[
        \mu^*(E) = \mu^*(E \cap A) + \mu^*(E \cap A^C) \quad \forall \; E \subset X
    \]
\end{definition}
\begin{remark}
    If \(E\) is a ``nice'' set containing \(A\), then the above equality says that the outer measure of \(A\), \(\mu^*(E \cap A)\), is equal to \(\mu^*(E) - \mu^*(E \cap A^C)\), which can be thought as an ``inner measure''. So basically we are saying that \(A\) is measurable if the outer and inner measure coincide. (Like the definition of Riemann integration with lower and upper sum)
\end{remark}
\begin{remark}
    \(\mu^*\) is subadditive by def \(\Rightarrow \mu^*(E) \leq \mu^*(E \cap A) + \mu^*(E \cap A^C) \quad \forall \; E, A \subset X\).  
    So, to prove that a set is \(\mu^*\)-measurable it is enough to prove the reverse inequality, \(\forall \; E \subset X\). In fact, if \(\mu^*(E) = +\infty\), then \(+\infty \geq \mu^*(E \cap A) + \mu^*(E \cap A^C)\), and hence \(A\) is \(\mu^*\)-measurable iff 
    \[
        \mu^*(E) \geq \mu(E \cap A) + \mu^*(E \cap A^C) \quad \forall \; E \subset X \mbox{ with } \mu^*(E) < +\infty
    \] 
\end{remark}
Their relevance to the notion of \(\mu^*\)-measurability is clarified by the following
\begin{theorem}[Caratheodory]
    If \(\mu^*\) is an outer measure on \(X\), the family
    \[
        \mathcal{M} = \left\lbrace A \subseteq X : A \mbox{ is }\mu^*\mbox{-measurable}\right\rbrace
    \]
    is a \(\sigalg\) and \(\mu^*\vert_{\mathcal{M}}\) is a complete measure.
\end{theorem}
\begin{lemma}
    If \(A \subset X\) and \(\mu^*(A) = 0\), then \(A\) is \(\mu^*\)-measurable.
\end{lemma}
\begin{proof}
    Let \(E \subset X\) with \(\mu^*(E) < +\infty\). Then 
    \[
        \mu^*(E) \geq \mu^*(E) + \mu^*(A) \overset{\symbolfootnotemark[3]{\(E \cap A^C \subseteq E\) and \(E\cap A \subseteq A\) + monotonicity}}{\geq}  \mu^*(E \cap A) + \mu^*(E \cap A^C)
    \]
    \symbolfootnotetext[3]{\(E \cap A^C \subseteq E\) and \(E\cap A \subseteq A\) + monotonicity}
    This implies that A is \(\mu^*\)-measurable.
\end{proof}
To sum up: \(X \mbox{ set}, (\mathcal{E}, \rho)\) elementary and measurable sets, so \(\mu^*\) is an outer measure. Then given \(\mu^*\) and the Caratheodory condition, we have \((X, \mathcal{M}, \mu)\) that is a complete measure space.
\begin{remark}
    So far we did not prove that \(\mathcal{E} \subseteq \mathcal{M}\). We will do it in a particular case.
\end{remark}
\subsubsection*{Lebesgue measure}
\begin{itemize}
\item \(X = \mathbb{R}\), \(\mathcal{E}\) family of open intervals, \(\rho((a,b)) = b-a = \lambda((a,b))\), the complete measure space is \((\mathbb{R}, \mathcal{L}(\mathbb{R}), \lambda)\) with \(\mathcal{L}(\mathbb{R})\) the Lebesgue-measurable sets on \(\mathbb{R}\) and \(\lambda\) the Lebesgue measure on \(\mathbb{R}\).
\item \(X = \mathbb{R}^n\), \(\mathcal{E} = \left\lbrace \prod_{k = 1}^n (a_k, b_k): a_k \leq b_k \quad \forall \; k = 1,\ldots, n \right\rbrace\), \(\rho\left(\prod_{k = 1}^n (a_k, b_k)\right) = \prod_{k=1}^n (b_k - a_k)\) and this is a complete measure space \((\mathbb{R}^n, \mathcal{L}(\mathbb{R}^n), \lambda_n)\)
\end{itemize}
