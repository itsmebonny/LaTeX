\section{Lecture 24/11/2022}
\noindent\underline{Recap on reflexivity}:

\noindent\(X\) Banach space. \(X^{**} = (X^*)^*\) is the \textbf{bidual space}, \(\mathcal{L}(X^*, \real)\)
\[
    \forall \; x \in X \; \exists \; \Lambda_x : X^* \to \real \mbox{ defined by } \Lambda_x(L) = Lx \quad \forall \; L \in X^*
\]
We proved that \(\Lambda_x \in X^{**}\). Thus we can define the \textbf{canonical map}:
\[
        \begin{array}{lc}
            \tau : & X \to X^{**} \\
            & x \mapsto \Lambda_x
        \end{array}
    \tag*{(Canonical Map)}
\]
We stated that \(\tau\) is an isometric isomorphism from \(X\) into \(\tau(X)\). This is true but for our purpose it's even too much, and it's difficult to prove in details. However, we can prove a slightly weaker result 
\begin{theorem}
    \(\tau\) is linear, continuous, and is an isometry
    \[
        \norm{\tau(x)}_{X^{**}} = \norm{x}_X \quad \forall \; x \in X
    \]
    Moreover, \(\tau\) is injective. If \(\tau\) is also surjective, it is an isometric isomorphism between \(X\) and \(X^{**}\)
\end{theorem}
\begin{proof}
    There are two parts:
    \begin{itemize}
        \item \(\tau\) is linear and continuous: exercise. 
        
        \noindent \(\tau\) is an isometry: \(\norm{\tau(x)}_{X^{**}} = \norm{\Lambda_x}_{X^{**}} = \norm{x}_X\)

        \noindent \(\tau\) is injective: \(x\neq y \Rightarrow \tau(x) \neq \tau(y)\)? 

        \noindent \(x \neq y \Rightarrow\) by the second corollary to Hahn-Banach \(\exists\; L \in X^*\) s.t. \(Lx \neq Ly\).
        \[
            \underset{X^{**}}{<}\tau(x), L \underset{X^*}{>} = \Lambda_x(L) = Lx \neq Ly = \Lambda_y(L) = \underset{X^{**}}{<}\tau(x), L \underset{X^*}{>}
        \]
        Then, \(\tau(x) \neq \tau(y)\) and \(\tau\) is injective.
        \item Let now \(\tau\) be surjective. Then \(\tau \in \mathcal{L}(x, X^{**})\) and is bijective \(\Rightarrow\) by a corollary of the open map theorem, \(\tau^{-1} \in \mathcal{L}(X^{**}, X)\)
    \end{itemize}
\end{proof}
\begin{definition}
    \(X\) is reflexive if \(\tau\) is surjective. In this case, \(\tau\) is an isometric isomorphism between \(X\) and \(X^{**}\)
\end{definition}

We formally mentioned that
\begin{theorem}
    If \((X, \normdot)\) is uniformly convex \(\Rightarrow\) \((X, \normdot)\) is reflexive.
\end{theorem}

\begin{remark}
    \begin{proposition}
        If \((X, \normdot)\) is uniformly convex \(\Rightarrow\) \(\overline{B_1(0)}\) is strictly convex 
    
    \end{proposition}
    \begin{proof}
        Is it true that if \(x, y \in \overline{B_1(0)}\), then \(\frac{x+y}{2} \in B_1(0)\)? Since \((X, \normdot)\) is uniformly convex, we know that \((\norm{x-y}=: \overline{\epsilon}>0)\)
        \[
            \forall \; \overline{\epsilon}>0 \quad \exists \; \overline{\delta} >0 \text{ s.t. } \norm{x} \leq 1 \; \norm{y} \leq \; \norm{x-y} > \frac{\overline{\epsilon}}{2} \Rightarrow \norm{\frac{x+y}{2}} < 1-\overline{delta} 
        \]
        In particular, 
        \[
            \frac{x+y}{2} < 1-\overline{\delta} < 1 \Rightarrow \frac{x+y}{2} \in B_1(0)
        \] 
    \end{proof}
    Consequence: \((\real^2, \normdot_1)\) and \((\real^2, \normdot_\infty)\) are not uniformly convex.
\end{remark}

\begin{proposition}
    \((\real^2, \normdot_2)\) is uniformly convex
\end{proposition}
\begin{proof}
    Suppose by contradiction that this is false:
    \( \exists \; \overline{\epsilon} >0 \) and \( \{x_n\}, \{y_n\} \subset \overline{B_1(0)} \) s.t. 
    \[
        \norm{x_n - y_n} > \overline{\epsilon} \text{, but } \norm{\frac{x_n + y_n}{2}} \geq 1 \tag{*}
    \]
    \(\overline{B_1(0)}\) is compact (since we are in \(\real^2\)) \(\Rightarrow\) UTS \(x_n \to \overline{x} \), \(y_n \to \overline{y}\) as \(n \to\infty\). Taking the limit in (*), we deduce that \(\overline{x}\), \(\overline{y} \in \overline{B_1(0)}\)
    \[
        \norm{\overline{x}- \overline{y}} \geq \overline{\epsilon} \text{ , and } \norm{\frac{\overline{x}+\overline{y}}{2}} \geq 1
    \]
    This is not possible, since \(\overline{B_1(0)}\) is strictly convex.
\end{proof}

\begin{theorem}
    \((X, \mathcal{M}, \mu)\) complete measure space. Then \(L^{p}(X)\) is reflexive \(\forall \; p \in (1, \infty)\)
\end{theorem}
\begin{proof}
    \((L^p(X), \normdot_p)\) is uniformly convex \(\forall \; p \in (1, \infty)\) (Clarkson inequalities)
\end{proof}
\(L^1(X) and L^\infty(X)\) are not uniformly convex, and not reflexive.

\subsection*{Dual space of \(L^p\)}

\begin{theorem}[Riesz representation theorem]
    \((X, \mathcal{M}, \mu)\) complete measure space, \(p \in (1, \infty)\). Then 
    \[
        \forall\; L \in (L^p(X))^* \quad \exists! \; g \in L^{p'}(X)
    \]
    with \(p' \) conjugate exponent s.t. \(L=L_g\), namely
    \[
        Lf = \int_X fg \, d\mu \qquad \forall\; f \in L^p(X)
    \]
    Moreover \(\norm{L_g}_{(L^p)^*} = \norm{g}_{p'}\)
    
    Thus: \(T: g \in L^{p'} \mapsto L_g \in (L^p)^*\) is an isometric isomorphism.
\end{theorem}
\begin{proof}
    \(1 < p < \infty\). Consider \(T: L^{p'} \to (L^p)^*\) with \(g \mapsto Tg: <Tg, f> = \int_X fg \, d\mu\) (namely \(Tg = L_g\)). We already know that
    \[
        \norm{Tg}_* = \norm{L_g}_* = \norm{g}_{p'}
    \]
    \(T\) is injective: for exercise.

    \(T\) is surjective. Indeed, let \(F:= T(L^{p'}) \subseteq (L^p)^*\) subspace. Since \(T\) is an isometry and \(L^{p'}\) is complete, it can be shown that \(T(L^{p'})\) is also complete \(\Rightarrow T(L^{p'})\) is closed.

    If by contradiction \(F \neq (L^p)^*\), then we can apply corollary 3 to Hahn Banach \((X = (L^p)^*), Y=F, x_0 = \lambda\):
    \[
        \exists h \in (L^p)^{**} \text{ s.t. } <h, \lambda> \neq 0 \text{ and } h|_F = 0: <h, Tg> =0 \quad \forall g \in L^{p'} \tag*{1} 
    \]
    But \(L^p \) is reflexive \((1 <p < \infty)\), then \(h \in L^p \setminus \{0\}\):
    \[
        <h, Tg> = (Tg)h = \int_X hg \, d\mu
    \]
    Therefore, (1) tells us that
    \[
        \int_X hg \, d\mu =0 \quad \forall \; g \in L^{p'}(X)
    \]
    Take \(g = \abs{h}^{p-2}h\). Therefore
    \[
        0 = \int_X hg \, d\mu = \int_X h \abs{h}^{p-2} h \, d\mu = \int_X \abs{h}^p \, d\mu \Rightarrow h=0 \in L^p
    \]
    which is the desired contradiction.

    \(T\) is an isomorphism: for exercise.
\end{proof}

