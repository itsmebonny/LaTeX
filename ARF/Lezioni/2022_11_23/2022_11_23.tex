\section{Lecture 23/11/2022}

\begin{theorem}[Hahn-Banach regarding continuous extension]
    \(X\) (real) normed space. \(Y\) subspace of \(X\), \(f \in Y^* = \mathcal{L}(Y, \real)\)

    Then \(\exists \; F \in X^* = \mathcal{L}(X, \real)\) s.t. 
    \[
        \begin{array}{lr}
            
            F(y) = f(y) & \forall \; y \in Y \\

            \norm{F}_{X^*} = \norm{f}_{Y^*}   & \\
        \end{array}
    \]
\end{theorem}
\begin{proof}
    Define \(p:X \to \real\), \(p(x) = \norm{f}_{Y^*} \norm{x}_X\) \(\forall \; x \in X\). Then \(p\) is sublinear (from the properties of \(\normdot_X\)).

    Moreover, \(f(y) \leq \abs{f(y)} \leq \norm{f}_{Y^*}\norm{y}_X = p(y)\) \(\forall \; y \in Y\). Then, by Hahn-Banach theorem (general version), \(\exists \; F : X \to \real\) s.t. \(F\) is an extension of \(f\) and \(F(x) \leq p(x) \; \forall \; x \in X\).

    Now, if \(F(x) \geq 0\)
    \[
        \abs{F(x)} = F(x) \leq p(x) = \norm{f}_{Y^*}\norm{x}_X
    \]
    If \(F(x) < 0\)
    \[
        \abs{F(x)} = -F(x) = F(-x) \leq p(-x) = \norm{f}_{Y^*}\norm{-x}_X = \norm{f}_{Y^*}\norm{x}_X
    \]
    \(\forall \; x \in X\)
    \[
        \abs{F(x)} \leq \norm{f}_{Y^*}\norm{x}_X
    \]
    namely, \(F \in X^*\) (it is bounded), and 
    \[
        \norm{F}_{X^*} \leq \norm{f}_{Y^*}
    \]

    Also, \(\norm{F}_{X^*} \geq \norm{f}_{Y^*}\) since \(F\) extends \(f\):
    \[
        \norm{F}_{X^*} = \sup_{\norm{x}_X \leq 1} \abs{F(x)} \geq \sup_{\norm{y}_Y \leq 1}\abs{F(y)} 
        = \sup_{\norm{y}_X \leq 1, y \in Y} \abs{f(y)} = \norm{f}_{Y^*}
    \]
\end{proof}
\noindent\underline{Consequence 1}
\begin{theorem}
    \((L^\infty (X))^*\) `strictly contains' \(L^1(X)\)
\end{theorem}

\begin{proof}
    We must show that \(\exists \; L \in (L^\infty(X))^*\) s.t. \(\nexists \; g \in L^1(X)\) s.t. 
    \[
        Lf = \int_X fg \, d\mu \quad \forall f \in L^\infty(X)
    \]

For simplicity, we consider \((X, \mathcal{M}, \mu) = ([-1, 1], \mathcal{L}([-1,1]), \lambda)\). Let \(Y\) be the subspace of \(L^\infty([-1,1])\) of the bounded continuous functions \(\mathcal{C}^0([-1,1])\). On \(Y\) we define 
\[
    \Lambda f = f(0) \quad \forall \; f \in Y
\]
We can do it since \(f \in \mathcal{C}^0([-1,1])\) (for elements in \(L^\infty\) we cannot speak about pointwise values!). 

\(\Lambda\) is linear:
\[
    \Lambda(\alpha f + \beta g) = \alpha \Lambda f + \beta \Lambda g
\]
Moreover, \(\Lambda\) is in \(Y^*\):
\[
    \abs{\Lambda f} = \abs{f(0)} < \max_{[-1,1]} \abs{f} = \norm{f}_{\infty}
\]
This proves that \(\Lambda \in Y^*\), \(\norm{\Lambda}_{Y^*} \leq 1\). By Hahn-Banach, \(\exists \; L \in (L^\infty(X))^*\) which is an extension of \(\Lambda\), and is s.t. 
\[
    \norm{L}_{(L^\infty)^*}
\]
Can we have 
\[
    Lf=\int_{-1}^1 fg \, d\mu \quad \mbox{for some } g \in L^1(X)\mbox{?}
\]
Suppose by contradiction that this is true, take 
\[
    f_n \in \mathcal{C}^0([-1,1])
\]
defined in this way:
\[
    f_n(x) = \phi(nx)
\]
where \(\phi\) is continuous, \(\mbox{supp}\, \phi \subseteq \left[ \frac{-1}{2} \frac{1}{2} \right]\) 
\[
    \phi(0) = 1, \phi(nx) = 0 \quad \forall \; x \mbox{ s.t. } \abs{nx} > \frac{1}{2} \Leftrightarrow \abs{x} > \frac{1}{2n}  
\]
By contradiction, 
\[
    \sup f_n \subseteq \left[-\frac{1}{2n}, \frac{1}{2n}\right] \Rightarrow f_n(x) \to 0
\]
Therefore, if \(g \in L^1([-1,1])\) is s.t. 
\[
    \int_{-1}^1 f_n g \, d\lambda = L f_n
\]
Then, on one side 
\[
    \int_{-1}^1 f_n g \, d\mu = L f_n = f_n (0) = 1 \quad \forall \; n
\tag*{(1)}\]
But on the other side 
\begin{itemize}
    \item \(f_n(x)g(x) \to 0\) a.e. in \([-1,1]\)
    \item \(\abs{f_n(x) g(x)} \leq g(x) \in L^1([-1,1])\)
    \[
        \overset{\mbox{DOM}}{\Rightarrow} \int_{-1}^1 f_n g \, d\lambda \to 0
    \tag*{(2)}\]
\end{itemize}
But \((1)\) and \((2)\) are in contradiction.
In conclusion, there is no \(g \in L^!([-1,1])\) s.t. 
\[
    \int_{-1}^1 f g \, d\lambda = L f \quad \forall \; f \in L^\infty ([-1,1])
\]
\end{proof}
\noindent\underline{Other consequences of the Hahn-Banach theorem}
\begin{corollary}
    \(X\) (real) normed space, \(x_0 \in X \setminus \left\{ 0 \right\}\).
    Then \(\exists \; L_{x_0} \in X^*\) s.t. 
    \[
        \norm{L_{x_0}}_{X^*} = 1 \mbox{ and } L_{x_0}(x_0) = \norm{x_0}
    \]
\end{corollary}
\begin{proof}
    Take \(Y = \left\{ \lambda x_0 : \lambda \in \real \right\}\) (1-d vector space generated by \(x_0\))  
    
    \[
        \begin{array}{rl}
            L_0: & Y \to \real \\
            & \lambda x_0 \mapsto \lambda \norm{x_0}_X
        \end{array}    
    \]

    This is linear and continuous on \(Y\) \(\Rightarrow\) by Hahn-Banach (continuous extension) \(\exists \; \tilde{L}_0 \in X^*\) s.t. \(\tilde{L}_0\) extends \(L_0\) and 
    \[
        \norm{\tilde{L}_0}_{X^*} = \norm{L_0}_{Y^*} = \sup_{\begin{array}{c}\lambda x_0 \in Y \\ \norm{\lambda x_0} = 1\end{array}}\abs{L_0(\lambda x_0)}= \sup \abs{\lambda \norm{x_0}_X} = 1
    \]
    Thus \(\tilde{L}_0\) is precisely the desired functional.
    \[
        \tilde{L}_0 (x_0) = L_0 = \norm{x_0}_X
    \]
    and 
    \[
        \norm{\tilde{L}_0}_{X^*} = 1
    \]
\end{proof}
\begin{corollary}[The bounded linear functionals separate points]
    If \(x,y \in X\) and \(Lx = Ly\) \(\forall \; L \in X^* \Rightarrow x = y\) (if \(x \neq y, \exists \; L \in X^*\) s.t. \(Lx \neq Ly\))
\end{corollary}
\begin{proof}
    Assume \(x-y \neq 0\). Then, by the previous corollary, \(\exists \; L \in X^*\) s.t. 
    \[
        \norm{L}_{X^*} \mbox{ and } L(x-y) = \norm{x-y}_X \Rightarrow Lx -Ly = L(x-y) = \norm{x-y}_X \neq 0
    \]
\end{proof}
\begin{corollary}
    \(X\) normed space, \(Y\) closed subspace of \(X\), \(x_0 \in X \setminus Y\). 

    Then \(\exists \; L \in X^*\) s.t. \(L\vert_Y = 0\) and \(L_{x_0} \neq 0\)
\end{corollary}
\subsection*{Reflexive spaces}
\(X\) Banach space, \(X^*\) dual space.

\noindent \underline{Notation}: \(L \in X^*: Lx = L(x) = <L,x> = \underset{X^*}{<}L,x\underset{X}{>}\)

\((X^*)^*\) dual space of \(X^*\) is called the \textbf{bidual} of \(X\), denoted by \(X^*\)
\[
    X^{**} = \mathcal{L}(X^*, \real)
\]
We can describe many elements of \(X^{**}\) in the following way: for \(x \in X\), define 
\[
    \begin{array}{ll}
        \Lambda: & X^* \to \real \\
        & L \mapsto Lx = \underset{X^*}{<}L,x\underset{X}{>}
    \end{array}
\]
(\(\Lambda_x\) evaluates functionals in \(X^*\) in the point \(x\)).

\(\Lambda_x\) is linear:
\[
    \Lambda_x (\alpha L_1 + \beta L_2) = (\alpha L_1 + \beta L_2)(x) = \alpha L_1 x + \beta L_2 x = \alpha \Lambda_x L_1 + \beta \Lambda_x L_2
\]
Moreover, it is bounded 
\[
    \abs{\Lambda_x(L)} = \abs{Lx} \underset{L \in X^*}{\leq} \norm{L}_{X^*}\norm{x}_X \quad \forall \; L \in X^*
\]
Moreover, 
\[
    \norm{\Lambda_x}_{\mathcal{L}(X^*, \real)} = \sup_{L \neq 0} \frac{\abs{\Lambda_x L}}{\norm{L}_X^*}
\]
We claim that \(\norm{\Lambda_x}_{\mathcal{L}} = \norm{x}_X\). Indeed, by the first corollary of Hahn-Banach, given any \(x \in X\setminus \left\{ 0 \right\} \exists \; Lx \in X^*\) 
\[
    \exists \; L_x \in X^* \mbox{ s.t } \abs{L_x x} = \norm{x}_X, \mbox{ and } \norm{L_x}_{X^*} = 1 
\]
\[
    \Rightarrow \sup_{L \neq 0} \frac{\abs{\Lambda_x L}}{\norm{L}_{X^*}} = \sup_{L \neq 0} \frac{\abs{Lx}}{\norm{L}_{X^*}} \geq \frac{\abs{L_x x}}{\norm{L_x}_{X^*}} = \norm{x}_X 
\]
\[
    \Rightarrow \norm{\Lambda_x}_{X^{**}} = \norm{x}_X
\]
\begin{theorem}
    \(\exists\) a map 
    \[
        \begin{array}{lc}
            \tau : & X \to X^{**} \\
            & x \mapsto \Lambda_x
        \end{array}
    \tag*{(Canonical Map)}\]
    which is linear, continuous and an isometry. Namely, the canonical map is an isometric isomorphism from \(X\) into \(\tau(X) \subseteq X^{**}\)
\end{theorem}
\noindent\underline{Question}: are there other elements in \(X^{**}\)?
\begin{definition}
    If the canonical map is surjective, then we say that \(X\) is \textbf{reflexive}, \(X \approxeq X^{**}\). Otherwise, \(\tau(X)\) will be a strict close subspace of \(X\).
\end{definition}
\begin{remark}
    \(X\) reflexive \(\nLeftarrow\) \(X\) and \(X^{**}\) are isometrically isomorphic.
\end{remark}
\begin{theorem}
    \(X\) reflexive space. Then every closed subspace of \(X\) is reflexive.
\end{theorem}
\begin{theorem}
    \(X\) Banach. 
    \[
        X \mbox{ reflexive } \Leftrightarrow X^* \mbox{ reflexive }
    \]
\end{theorem}
\begin{theorem}
    \(X\) Banach.
    \begin{itemize}
        \item If \(X^*\) is separable \(\Rightarrow X\) is separable
        \item If \(X\) is separable and reflexive \(\Rightarrow X^*\) is separable
    \end{itemize}
\end{theorem}
To show that a space is reflexive, it is convenient to introduce the following notion.
\begin{definition}
    \(X\) Banach space. \(X\) is called \textbf{uniformly convex} if \(\forall \; \epsilon > 0 \; \exists \; \delta > 0 \) s.t. 
    \[
        \forall \; x, y \in X \mbox{ with } \norm{x} \leq 1, \norm{y} \leq 1, \norm{x-y} > \epsilon
    \]
    then we have 
    \[
        \norm{\frac{x+y}{2}} < 1-\delta
    \]
    This is a quantitative version of the strict convexity.
\end{definition}
\begin{definition}
    \(C \subset X\) is convex  \(\Leftrightarrow \forall \; x,y \in C : \frac{x+y}{2} \in C\)

    \noindent \(C \subset X\) is \textbf{strictly convex} \(\Leftrightarrow \forall \; x,y \in C : \frac{x+y}{2} \in \interior{C}\)
\end{definition}
Roughly speaking, \(X\) is uniformly convex if \(\overline{B_1(0)}\) is strictly convex in a quantitative way.
\begin{theorem}[Milman-Pettis]
    Every uniformly convex Banach space is reflexive.
\end{theorem}