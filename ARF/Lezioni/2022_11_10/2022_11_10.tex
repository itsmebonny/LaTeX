\section{Lecture 10/11/2022}
\noindent\underline{Quick recap about the `delirium' on the separability}

The thing that you need to know, in \(\to L^p(\real, \mathcal{L}(\real), \lambda)\), are:
\begin{enumerate}
    \item \(L^p\) is separable \(\forall\; p \in [1, \infty)\)
    \item \(\tilde{S}(\real)\) is dense in \(L^p(\real)\) \(\forall \; p \in [1, \infty)\), 
    namely \(\forall p \in L^p (\real)\) and \(\forall \; \epsilon >0 \) \(\exists \; s \in \tilde{S}(\real)\) s.t. 
    \[
        \norm{f-s}_p < \epsilon
    \]
    \item \(\mathcal{C}_C^0 (\real)\) is dense in \(L^p\), namely \(\forall p \in L^p (\real)\) and \(\forall \; \epsilon >0 \) \(\exists \; g \in \mathcal{C}_C^0(\real)\) s.t. 
    \[
        \norm{f-g}_p < \epsilon
    \]
\end{enumerate}
Everything remains true if you replace \(\real\) with \(X\) open or closed, or with \(X \in L(\real^n)\), and consider \((X, L(X), \lambda)\).

What happens for \(L^{\infty}(\real, \mathcal{L}(\real), \lambda)\)? 

\(\mathcal{C}(\real)\) is not dense in \(L^\infty\).

By the simple approximation theorem, we have that simple functions are dense in \(L^\infty\).
\begin{theorem}
    \(L^{\infty}(\real, \mathcal{L}(\real), \lambda)\) is not separable.
\end{theorem}
\begin{proof}
    \(\{\chi_{[-\alpha, \alpha]}: \alpha >0\} \subseteq L^{\infty}(\real, \mathcal{L}(\real), \lambda)\)
    \(\chi_\alpha = \chi_{[-\alpha, \alpha]}\)

    This is an uncountable family of functions. \(\norm{\chi_\alpha - \chi_{\alpha'}}_{\infty}=1\) \(\forall \; \alpha \neq \alpha'\), indeed

    \[
        \abs{\chi_\alpha(x) - \chi_{\alpha'}(x)} =
        \begin{cases}
            0 & \text{if} x \in [-\alpha, \alpha] \cup (\alpha', \infty) \cup (-\infty, -\alpha')\\
            1 & \text{if} x \in (\alpha, \alpha'] \cup [-\alpha', \alpha)    
        \end{cases}
    \]

    In particular, \(B_{\frac{1}{2}}(\chi_\alpha) \cap B_{\frac{1}{2}}(\chi_{\alpha '}) = \emptyset \) \(\forall \; \alpha \neq \alpha'\)

    Assume by contradiction that \(L^\infty(\real)\) is separable: \(\exists \; Z \subset L^\infty\) which is countable and dense. In particular, \(\forall \; f \subset L^\infty\) \(\exists \; g \in Z\) s.t. 
    \[
        \norm{g-f }_\infty < \frac{1}{2}
    \]
    Therefore, \(\forall \; \alpha\), \(\exists \;g_\alpha \in B_{\frac{1}{2}}(\chi_\alpha) \cap Z\). 
    But \( B_{\frac{1}{2}}(\chi_\alpha) \cap B_{\frac{1}{2}}(\chi_{\alpha'}) = \emptyset \)

    \[
        \Rightarrow \alpha \neq \alpha' \text{, we have } g_\alpha \neq g_{\alpha'}
    \]
    \(Z \supseteq \{ g_\alpha : \alpha >0 \}\), which is uncountable. This is not possible, since \(Z\) is countable.
\end{proof}

\begin{remark}
    The same is true if \((\real, \mathcal{L}(\real), \lambda)\) is swapped with \((X, \mathcal{L}(X), \lambda)\), \(X\) is open or closed on \(\real\) or \(\real^n\)
\end{remark}

\subsection*{Linear operators}
\((X, \normdot_X)\), \((Y, \normdot_Y)\) normed spaces.
\begin{definition}
    \(T : D(T) \subseteq X \to Y\) is a \textbf{linear operator} (or map) if 
    \[
        T(\alpha_1 x_1 + \alpha_2 x_2) = \alpha_1 T(x_1) + \alpha_2 T(x_2) \quad \forall \; x_1, x_2, \in D(T) \quad \forall \; \alpha_1, \alpha_2 \in \real
    \]
    \(D(T)\) is a linear subspace of \(X\), and is called the domain of T. When \(D(T) = X\) and \(Y = \real\), \(T\) is called linear functional.
\end{definition}
\begin{definition}
    A linear operator \(T : D(T) \subseteq X \to Y\) is bounded if \(D(T) = X\) and \(\exists \; M >0\) s.t. 
    \[
        \norm{T_X}_Y \leq M \abs{x}_X \forall \; x \in X
    \]
    Recall that \(T\) is continuous in \(x_0 \in X\) iff 
    \[
        \forall \; \left\{ x_n \right\} \subset X, x_n \overset{X}{\to} x_0 \Rightarrow Tx_n \overset{Y}{\to} Tx_0
    \]
\end{definition}
\noindent\underline{Ex}:
\begin{itemize}
    \item \(L: \real^n \to \real\)  is a linear functional . Then \(\exists \; y \in \real^n\) s.t. 
    \[
        Lx = \langle y, x \rangle = (y, x) = y \cdot x
    \]
    In particular, then \(L\) is continuous on \(\real^n\) and bounded:
    \[
        \abs{L_X} < \abs{\langle y,x \rangle} \overset{\text{\tiny{Cauchy-Schwarz}}}{\leq} \norm{y} \norm{x} \qquad \forall\; x \in \real^n
    \]
    So \(L\) is bounded with \(M=\norm{y}\).

    \item Linear operators in \(\infty\)-dim may not be defined everywhere, and many may not be continuous:
    \((X, \normdot_X) = (Y, \normdot_Y) = (\mathcal{C}([0, 1]), \normdot_\infty)\).
    
    Consider 
    \[
        \begin{array}{cc}
            \frac{d}{dx}: \mathcal{C}'([0,1]) \subseteq X \to Y & \frac{d}{dx}(\alpha f + \beta g) = \alpha \frac{d}{dx}f + \beta \frac{d}{dx} g \\
            f \mapsto f'
        \end{array}
    \]
    This is not continuous or bounded. For example, take \(f_n(x) = \frac{1}{n} \sin{2\pi n x}\). \(\norm{f_n}_\infty \to 0\) but \(\norm{f_n'}_\infty =1\)

    In this case \(f_n \to 0 \nRightarrow \frac{d}{dx} f_n \to 0\), then \(\frac{d}{dx} \) is not bounded as well.
    \item Let \((X, \normdot_X)\) be a normed space. If \(\dim X = 0\), is it possible to find linear functionals which are not bounded? Yes.
\end{itemize}
\begin{definition}
    A subset \(\left\{ e_i \right\}_{i \in I}\) is called \textbf{Hamel basis} of \(X\) if 
    \[
        \norm{e_i}_X = 1 \quad \forall  \; i
    \]
    and if every \(x \in X\) can be written in a unique way as 
    \[
        x = \sum_{k=1}^n x_k e_{i_k}, \quad x_k \in \real, \ n \in \mathbb{N}
    \]
\end{definition}
Every \(x\) can be written uniquely as a finite linear combination of element of the basis.
If \(\dim X = \infty\) is not immediate that the Hamel basis exists. This can be proved using the axiom of choice. (Zorn's lemma). 

Any normed space has a Hamel basis \(\dim X = \infty \Rightarrow \{e_i\}_{i \in I}\) has \(\infty\) many elements.

\noindent Let then \((X, \normdot_X)\) be \(\infty -\dim\), with Hamel basis \(\{e_i \}_{i \in I}\). \(I\) is infinite \(\Rightarrow I \supseteq \mathbb{N}\).

\noindent We define \(L:X \to \real\) in the following way 
\[
    \begin{array}{ccccc}
        L e_0 = 0 & L e_1 = 1 & \dots & L e_n = n & \dots \\
        L e_i = 0 \quad \forall \; i \in I \setminus \mathbb{N} &&&&
    \end{array}
\]
Then, for \(x \in X\) we set
\[
    Lx = L \left( \sum_{k=1}^n x_k e_{i_k} \right) = \sum_{k=1}^n x_k L e_{i_k}
\]
\(L\) is linear by contradiction, and it is not bounded:
\[
    \begin{array}{c}
        \abs{L e_n} = n \to \infty \quad \norm{e_n }_X = 1 \; \forall\, n \\
        \frac{\abs{L e_n}}{\norm{e_n}_X} \to \infty \Rightarrow L \text{ is not bounded}
    \end{array}
\]
\begin{remark}
    In practice, Hamel basis are hard to use. They differ from Hilbertian basis.
\end{remark}

For linear operators, boundedness and continuity are equivalent.
\begin{theorem}
    \(T:X \to Y\) linear map. Then the following are equivalent
    \begin{enumerate}
        \item \(T\) is continuous in \(0 \in X\)
        \item \(T\) is continuous everywhere in \(X\)
        \item \(T\) is bounded
    \end{enumerate}
\end{theorem}
\begin{remark}
    \(T\) linear \(\Rightarrow T0 = 0\). Indeed
    \[
        T0 = T(0x) =  0 Tx = 0
    \]
\end{remark}
\begin{proof}
    
    \begin{itemize}
        \item \((2) \Rightarrow (1)\) obvious.
        \item \((1) \Rightarrow (3)\) Suppose by contradiction that \(T\) is not bounded. 
        
        Then \(\exists \; \{ x_n \} \subset X \), \(x_n \neq 0\), s.t. 
        \[
            \frac{\norm{T x_n}_Y}{\norm{x_n}_X} \geq n \quad \forall\; n
        \]
        Define
        \[
            z_n := \frac{x_n}{n \norm{x_n}_X}
        \]
        Then \(\norm{z_n }_X = \frac{1}{n \norm{x_n}} \norm{x_n}_X \to 0\),
        namely \(z_n \to 0 \) in \( X \Rightarrow (T \text{ is continuous in }0)\) \(T z_n \to T0 =0\).
        However, 
        \[
            \norm{T z_n}_Y = \norm{T\left(\frac{x_n}{n \norm{x_n}_X}\right)} = \frac{1}{n \norm{x_n}_X} \norm{T x_n}_Y \geq 1 \ \forall\; n
        \]
        Contradiction.
        \item \((3) \Rightarrow (2)\) 
        We observe that 
        \[
            \norm{Tx_1 - Tx_2}_Y = \norm{T(x_1 - x_2)}_Y \leq M \norm{x_1 - x_2}_X  \quad \forall \; x_1 \, x_2 \in X
        \]
        Then, let \(x \in X \) and let \(x_n \to x\) in \(X\): \(\norm{x_n - x}_X \to 0\). But then
        \[
            \norm{T x_n - Tx}_Y \leq M \norm{x_n - x}_X \to 0
        \]
        namely \(Tx_n \to Tx \) in \(Y\). This is the continuity.
    \end{itemize}
\end{proof}

\begin{definition}
    The set of linear operators \(T : X \to Y\) which are also bounded (continuous) is denoted by \(\mathcal{L}(X, Y)\).    If \(Y=X\), one simply writes \(\mathcal{L}(X)\)
\end{definition}
This is a vector space. \( \forall\; T,\, S \in \mathcal{L}(X, Y) \), \(\forall\; \alpha, \beta \in \real:\)
\[
     (\alpha T + \beta S)(x) = \alpha Tx + \beta Sx \qquad \in \mathcal{L}(X, Y)
\]
We can also introduce a norm:
\[
    \norm{T}_{\mathcal{L}(X, Y)} = \norm{T}_\mathcal{L} := \sup_{\norm{x}_X \leq 1} \norm{Tx}_Y
\]

Also, 
\[
    \norm{T}_{\mathcal{L}(X, Y)} = \sup_{\norm{x}_X = 1} \norm{Tx}_Y = \sup_{x \neq 0} \frac{\norm{Tx}_Y}{\norm{x}_X} = \inf{M >0 \text{ s.t. } \norm{Tx}_Y \leq M \norm{x}_X \quad \forall \; x \in X}
\]

\begin{theorem}
    \(X\) normed space, \(Y\) Banach space. Then \((\mathcal{L}(X, Y), \normdot_{\mathcal{L}(X, Y)})\) is a Banach space.
\end{theorem}
\begin{proof}
    Let \(\{T_n\}\) be a Cauchy sequence in \(\mathcal{L}(X, Y)\). We want to show that \(\exists \; T \in \mathcal{L}(X, Y) \) s.t.
    \[
        \norm{T_n - T}_\mathcal{L} \to 0
    \]
    \(\{T_n\}\) cauchy: \(\forall \; \epsilon >0 \) \(\exists \; \bar{n} \in \mathbb{N}\) s.t. 
    \[
        n, m > \bar{n} \Rightarrow \norm{T_n - T_m }_\mathcal{L} < \epsilon
    \]
    Consider then \(\{ T_n x \}\), \(x \in X\)
    \[
        \norm{T_n x - T_m x }_Y = \norm{(T_n - T_m)x}_Y \leq \norm{T_n - T_m}_Y \norm{x}_X \leq \epsilon \norm{x}_X \tag{*}
    \]
    This means that \(\{ T_n x \}\) is a Cauchy sequence in \(Y\), which is complete: then \(\forall \; x \in X\) \(\exists \) a vector \(y_x \in Y\) s.t. \(T_n x \to y_x\) in \(Y\).

    Define 
    \[
        T: X \to Y \qquad x \mapsto y_x = Tx
    \]
    \(T\) is linear: indeed, \(\forall \; x_1\), \(x_2 \in X\) and \(\alpha_1\), \(\alpha_2 \in \real\):
    \[
        T(\alpha_1 x_1 + \alpha_2 x_2) = \lim_{n \to \infty} T_n (\alpha_1 x_1 + \alpha_2 x_2) = \lim_{n \to \infty} (\alpha_1 T_n x_1 + \alpha_2 T_n x_2) = \alpha_1 Tx_1 + \alpha_2 Tx_2
    \]
    So \(T \) is linear. It remains to show that \(T\) is bounded, and that \(\norm{T_n - T}_{\mathcal{L}} \to 0\).
    To show that \(T\) is bounded, note that, by (*), \(\forall \; \epsilon >0 \; \exists \; \bar{n}\) s.t.
    \[
        n, m > \bar{n} \Rightarrow \norm{T_n x - T_m x}_Y \leq \epsilon \norm{x}_X \quad \forall \; x 
    \]
    Take the limit for \(m \to \infty\): 
    \[
        \norm{T_n x - Tx}_Y \leq \epsilon \norm{x}_X
    \]
    But then, since \(T_n\) is bounded, 
    \[
        \norm{Tx}_Y = \norm{Tx \pm T_n x}_Y \leq \norm{T_n x}_Y + \norm{Tx - T_n x}_Y \leq M_n \norm{x}_X + \epsilon \norm{x}_X = (M_n + \epsilon) \norm{x}_X
    \]
    and \(T\) is bounded. To show that \(\norm{T_n - T}_\mathcal{L} \to 0\), observe that \(\forall \; \epsilon >0 \; \exists \; \bar{n} \) s.t. \(n > \bar{n}\)
    \[
        \norm{T_n x - Tx}_Y \leq \epsilon \norm{x}_X 
        \Leftrightarrow \frac{\norm{(T_n - T)x}_Y}{\norm{x}_X} \leq \epsilon \quad \forall \; x \in X \setminus {0}
        \overset{\text{\tiny{take sup over \(x \neq 0\)} }}{\Rightarrow} \norm{T_n - T}_\mathcal{L} < \epsilon
    \]
    namely, \(T_n \to T\) in \(\mathcal{L}\)
\end{proof}