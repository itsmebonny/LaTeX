\section{Lesson 10/11/2022}
\underline{Quick recap about the `delirium' on the separability}

The thing that you need to know \(\to L^p(\real, \mathcal{L}(\real), \lambda)\).
\begin{enumerate}
    \item \(L^p\)
\end{enumerate}
What happens for \(L^{\infty}(\real, \mathcal{L}(\real), \lambda)\)? 

\(\mathcal{C}(\real)\) is on \(L^\infty\).

By the simple approximation theorem, we have that simple functions are dense in \(L^\infty\).
\begin{theorem}
    \(L^{\infty}(\real, \mathcal{L}(\real), \lambda)\) is not separable.
\end{theorem}
\begin{proof}
    \(\{\chi_{[-\alpha, \alpha]}: \alpha >0\} \subseteq L^{\infty}(\real, \mathcal{L}(\real), \lambda)\)
    \(\chi_\alpha = \chi_{[-\alpha, \alpha]}\)

    This is an uncountable family of functions. \(\norm{\chi_\alpha - \chi_{\alpha'}}\) \(\forall \; \alpha \neq \alpha'\)

    ...

    In particular, \(B_{\frac{1}{2}}(\chi_\alpha) \cap B_{\frac{1}{2}}(\chi_{\alpha '}) = \emptyset \) \(\forall \; \alpha \neq \alpha'\)

    Assume by contradiction that \(L^\infty(\real)\) is separable: \(\exists Z \subset L^\infty\) which is countable and dense. In particular, \(\forall \; f \subset L^\infty\) \(\exists \; g \in Z\) s.t. 
    \[
        \norm{g-f }_\infty < \frac{1}{2}
    \]
    Therefore, ...

    \[
        \Rightarrow \alpha \neq \alpha' \text{, we have } g_\alpha \neq g_{\alpha'}
    \]
    ....
\end{proof}

\begin{remark}
    The same is true if \((\real, \mathcal{L}(\real), \lambda)\) is swapped with \((X, \mathcal{L}(X), \lambda)\), \(X\) is open or closed on \(\real\) or \(\real^n\)
\end{remark}
\subsection*{Linear operators}
\((X, \normdot_X)\), \((Y, \normdot_Y)\) normed spaces.
\begin{definition}
    \(T : D(T) \subseteq X \to Y\) is a \textbf{linear operator} if 
    \[
        T(...) = \alpha_1 T(x_1) + \alpha_2 T(x_2) \forall \; x_1, x_2, \in D(T) \quad \forall \; \alpha_1, \alpha_2 \in \real
    \]
    \(D(T)\) is a ... linear subspace
\end{definition}
\begin{definition}
    A linear operator \(T : D(T) \subseteq X \to Y\) is bounded if \(D(T) = X\) and \(\exists \; M >0\) s.t. 
    \[
        \norm{T_X}_Y \leq M \abs{x}_X \forall \; x \in X
    \]
    Recall that \(T\) is continuous in ... iff 
    \[
        \forall \; \left\{ x_n \right\} \subset X
    \]
\end{definition}
\underline{Ex}:
\begin{itemize}
    \item \(L: \real^n \to \real\)  is a linear functional . Then \(\exists \; y \in \real^n\) s.t. 
    \[
        qualcosa
    \]
    In particular, then \(L\) is continuous on \(\real^n\) and 
    \[
        \abs{L_X} < \abs{<y,x>} \overset{\text{\tiny{Cauchy-Schwarz}}}{\leq}
    \]
    \item Linear operators in \(\infty\)-dim may not be defined everywhere, and many may not be continuous.
    
    Consider 
    \[
        \begin{array}{cc}
            cose & cose \\
        \end{array}
    \]
    This is not continuous or bounded. For example, take \(f_n(x) = \frac{1}{n} \sin{2\pi n x}\)
    In this case \(f_n \to 0 \nRightarrow \frac{d}{dx} f_n \to 0\), then \(\frac{d}{dx}\) is not bounded as well.
    \item Let \((X, \normdot_X)\) be a normed space. If \(\dim X = 0\), is it possible to find linear functionals which are not bounded? Yes.
\end{itemize}
\begin{definition}
    A subset \(\left\{ qualcosa \right\}\) is called \textbf{Hamel basis} of \(X\) if 
    \[
        dawdwa
    \]
    and if every \(x \in X\) can be written in a unique way as 
    \[
        x = \sum_{k=1}^n x_k e_{i_k}, dwdqw
    \]
\end{definition}
Every \(x\) can be written uniquely as a finite linear combination of element of the basis.
If \(\dim X = \infty\) is not immediate that the Hamel basis exists. This can be proved using the axiom of choice. (Zorn's lemma). 

Any normed space has a Hamel basis \(\dim X = \infty \Rightarrow \) has \(\infty\) many elements.

cose

We define \(L:X \to \real\) in the following way 
\[
    qualcosa array
\]
Then, for \(x \in X\) we set
\[
d<s    
\]
 
\(L\) is linear by contradiction, and it is not bounded.

\[
    \begin{array}{c}
        cose \\
    \end{array}
\]
\begin{remark}
    In practice, Hamel basis are 
\end{remark}
For linear operators, boundedness and continuity are equivalent.
\begin{theorem}
    \(T:X \to Y\) linear map. Then the following are equivalent
    \begin{enumerate}
        \item \(T\) is continuous in \(0 \in X\)
        \item \(T\) is continuous everywhere in \(X\)
        \item \(T\) is bounded
    \end{enumerate}
\end{theorem}
\begin{remark}
    \(T\) linear \(\Rightarrow T_0 = 0\). Indeed
    \[
        T_0 = T(0x) =  0
    \]
\end{remark}
\begin{proof}
    
    \begin{itemize}
        \item \((2) \Rightarrow (1)\) obvious.
        \item \((1) \Rightarrow (3)\) Suppose by contradiction that \(T\) is not bounded. TANTE COSE
    \end{itemize}
\end{proof}