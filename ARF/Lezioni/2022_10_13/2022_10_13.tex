\documentclass[a4paper,12pt]{article}
\usepackage{amssymb}
\usepackage{amsmath}
\usepackage{hhline}
\usepackage{hyperref}
\usepackage{bm}
\usepackage[margin=2cm]{geometry}

\usepackage{amsthm}
\usepackage{tikz}
\usepackage{tabularx}
\usepackage{graphicx}
\usetikzlibrary{shapes.geometric, arrows}
\tikzstyle{startstop} = [rectangle, rounded corners, minimum width=3cm, minimum height=1cm,text centered, draw=black, fill=red!30]
\tikzstyle{io} = [trapezium, trapezium left angle=70, trapezium right angle=110, minimum width=3cm, minimum height=1cm, text centered, draw=black, fill=blue!30]
\tikzstyle{process} = [rectangle, minimum width=3cm, minimum height=1cm, text centered, draw=black, fill=orange!30]
\tikzstyle{decision} = [diamond,aspect = 2, text centered, draw=black, fill=green!30]
\tikzstyle{arrow} = [thick,->,>=stealth]
\usepackage{newunicodechar}
\newunicodechar{≠}{\ensuremath{\not =}}
\usepackage{textcomp}
\usepackage[makeroom]{cancel}

\newlength\mylength
\setlength\mylength{0.1cm}
\newcolumntype{Y}{>{\Centering\arraybackslash}X}

\AtBeginEnvironment{array}{\everymath{\displaystyle}}
\newtheoremstyle{break}
  {\partopsep}{\topsep}%  
  {\normalfont}{}
  {\bfseries}{}%
  {\newline}{}%
  \theoremstyle{break}
\newtheorem{theorem}{Theorem}[section]
\newtheorem{corollary}{Corollary}[section]
\newtheorem{proposition}{Proposition}[section]
\newtheorem{remark}[section]{Remark}
\newtheorem{lemma}{Lemma}[section]
\renewcommand*{\proofname}{\textbf{Proof}}
\renewcommand\qedsymbol{$\bigstar$}
\newtheorem{definition}{Definition}[section]
\renewcommand\labelenumi{(\theenumi)}

\let\oldemptyset\emptyset
\let\emptyset\varnothing

\newcommand{\ind}{\perp\!\!\!\!\perp} 
\newcommand{\measurespace}{(X, \mathcal{M}, \mu)}
\newcommand{\sigalg}{\sigma\mbox{-algebra}}
\newcommand{\boreal}{\mathcal{B}(\mathbb{R})}
\newcommand{\real}{\mathbb{R}}
\newcommand{\code}[1]{\texttt{#1}}
\newcommand{\xdownarrow}[1]{%
  {\left\downarrow\vbox to #1{}\right.\kern-\nulldelimiterspace}
}
\newcommand{\xuparrow}[1]{%
  {\left\uparrow\vbox to #1{}\right.\kern-\nulldelimiterspace}
}
\newcommand{\arrvline}{\hfil\kern\arraycolsep\vline\kern-\arraycolsep\hfilneg}

\long\def\symbolfootnotemark[#1]#2{\begingroup%
\def\thefootnote{\fnsymbol{footnote}}\footnotetext[#1]{#2}\footnotemark[#1]\endgroup}

\long\def\symbolfootnotetext[#1]#2{\begingroup%
\def\thefootnote{\fnsymbol{footnote}}\footnotetext[#1]{#2}\endgroup}


\numberwithin{equation}{section}





\begin{document}
\section*{Lesson 13/10/2022}
\subsubsection*{Integration on product spaces}
\((X, \mathcal{M}, \mu), (Y, \mathcal{N},\nu)\) measure spaces. \(f : X \times Y \to \bar{\real}\) measurable.

If \(f \geq 0\), then 
\[
    \iint_{X \times Y} f d\mu\otimes d\nu
\]
Goal: obtain a formula of iterated integral like the one in Analysis 2.

\(\forall \bar{x} \in X\) and \(\bar{y} \in Y\)
\[
    cose
\]
\begin{proposition}
    If \(f\) is measurable \(\Rightarrow\) \(f_{\bar{x}}\) is \((\mathcal{N}, \boreal)\)-measurable and \(f_{\bar{y}}\) is \((\mathcal{M}, \mathcal{B}(\bar{\real}))\)-measurable.
    Then we can conclude
    \( \phi : X \to \bar\real \):
    \[
        \phi(x) = \int_Y f_x d\nu = \int_Y f(x,y) d\nu(y)
    \]
    and \(\psi : Y \to \bar{\real}\)
    \[
        \psi(y) = \int_X f_y d\mu = \int_X f(x,y) d\mu(x)
    \]
\end{proposition}
\underline{Questions:} what is the solution of \(\iint_{X \times Y}\)
cose cose
\begin{theorem}[Tonelli's theorem]
    \((X, \mathcal{M}, \mu)\) and \((Y, \mathcal{N}, \nu)\) complete measure spaces and \(\sigma\)-finite.
    Suppose that \(f\) is \((\mathcal{M} \otimes \mathcal{N}, \mathcal{B}(\bar{\real}))\)-measurable and that \(f > 0\) a.e. on \(X \times Y\). Then \(\psi\) and \(\phi\) are measurable and
    \[
        \iint_{X \times Y} f d\mu \otimes d\nu = cose
    \]
    Equally holds also if one of the integrals is \(\infty\).
\end{theorem}
\begin{remark}
    The double integral can be reduced to single integrals, iterated. Moreover we can always change the order of the integrals
    For sign changing functions the situation is more involved.
\end{remark}
\begin{theorem}[Fubini's theorem]
    \((X, \mathcal{M}, \mu)\) and \((Y, \mathcal{N}, \nu)\) complete measure spaces and \(\sigma\)-finite.
    If \(f \in L^1(X \times Y)\), then \(\psi\) and \(\phi\) defined above are measurable, and cose holds, and all the integrals are finite.
\end{theorem}
\underline{Question}: how to check if \(f\in L^1(X \times Y)\)? Typically, to check cosette

If \(\iiiint_{X \times Y} \vert f \vert d\mu \otimes d\nu < \infty\) then we can apply Fubini for \(\iint_{X \times Y} f d\mu \otimes d\nu\)
\begin{remark}
    the proof of Fubini's and Tonelli's theorems is based for the iterated integrals for characteristic functions.
    (Note that \((\mu \otimes \nu)(E) = \int_X ()\) e altre cosette)
\end{remark}
\begin{remark}
    Sometimes double integrals are very useful to compute single integrals.
\end{remark}
Ex: \(\int_{-\infty}^{+\infty}\exp{-x^2} = \sqrt{\pi}\)


\subsection*{The first fundamental theorem of calculus}

Consider \(f \in L^1\left([a,b]\right)\)nWe can define the \textbf{integral function}
\[F(x) = \int_{[a,b]} f d\lambda = \int_a^b f(t)dt , \quad \]

If the function cose 

What happens if ?
\end{document}