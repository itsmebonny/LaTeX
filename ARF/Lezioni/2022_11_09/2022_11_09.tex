\section{Lecture 09/11/2022}
We introduced \(L^p(X, \mathcal{M}, \mu)\), and we proved that this is a normed space with 
\[
    \norm{f}_p := \begin{cases}
        \left( \int_X \abs{f}^p \; d\mu \right)^{\frac{1}{p}} & \text{if } p\in [1, +\infty) \\
        \underset{X}{\esssup}\abs{f} & \text{if } p = +\infty
    \end{cases}
\]
\subsection*{Inclusion of \(L^p\) spaces}
\begin{theorem}
    Suppose that \(\mu(X) < +\infty\). Then 
    \[
        1 \leq p \leq q \leq \infty \Rightarrow L^q(X) \subseteq L^p(X)
    \]
    Meaning that any \(f \in L^q\) is also in \(L^p\). More precisely, \(\exists \; C > 0\) depending on \(\mu(X), p, q\) s.t.
    \[
        \norm{f}_p \leq C \norm{f}_q \quad f \in L^q(X)
    \]
\end{theorem}
\begin{proof}
    If \(q = +\infty\)
    
    \(f \in L^\infty(X)\): then \(\abs{f(x)} \leq \underset{X}{\esssup}\abs{f} = \norm{f}_\infty\) for a.e. \(x \in X\), say \(\forall \; x \in X \setminus A\), with \(\mu(A) = 0\). Then 
    \[
        \int_X \abs{f}^p \, d\mu = \int_{X\setminus A} \abs{f}^p \, d\mu \leq \norm{f}_{\infty}^p \int_{X\setminus A} 1 \,d\mu = \norm{f}_\infty^p \underbrace{\mu(X)}_{= \mu(X\setminus A)}
    \]
    If \(q < +\infty\)

    Then \(\frac{q}{p} > 1\), and we can use Hölder\(\left(\frac{q}{p},\left( \frac{q}{p} \right)' \right)\), where \(\left( \frac{q}{p} \right)' = \frac{\frac{q}{p}}{\frac{q}{p}-1} = \frac{q}{q-p}\)
    \[
        \norm{f}_p^p = \int_X \abs{f}^p \, d\mu \overset{\text{\tiny{Hölder}}}{\leq} \left( \int_X \left( \abs{f}^{\not p} \right)^{\frac{q}{\not p}} \, d\mu\right)^{\frac{p}{q}}\cdot \left( \int_X 1 \, d\mu \right)^{\frac{q-p}{p}} = \left( \int_X \abs{f}^{q} \, d\mu\right)^{\frac{p}{q}}\cdot \left( \mu(X)\right)^{\frac{q-p}{p}}
    \]
    \[
        \Rightarrow \norm{f}_p \leq \mu(X)^{\frac{q-p}{qp}} \norm{f}_q
    \]
\end{proof}

The assumption \(\mu(X)< \infty\) is essential. For example, in \(X = [1, \infty]\)
\[
    \frac{1}{x} \in L^2([1, \infty]) \Leftrightarrow \int_1^\infty \frac{dx}{x^2} < \infty
\]
\[
    \frac{1}{x} \notin L^1 ([1, \infty]) \Leftrightarrow \int_1^\infty \frac{dx}{x} = \infty
\]

In particular, the previous theorem is false for \(l^p\)-spaces
\[
    l^p = L^p(\natural, \mathcal{P}(\natural), \mu_C)
\]
\[
    1 \leq p \leq q \leq \infty \Rightarrow l^p \subseteq l^q \text{, and } \exists \; C>0 \text{ s.t. } \norm{x}_q \leq C \norm{x}_p \quad \forall\; x \in l^p
\]
Without assumptions on \(\mu(X)\), in general one has the interpolation inequality.
\begin{theorem}
    \((X, \mathcal{M}, \mu)\) measure space. 
    Let \(1 \leq p \leq q \leq \infty\). If \(f \in L^p(X) \cap L^q(X)\), then 
    \[
        f \in L^r(X) \quad \forall \; r \in (p, q)
    \]
    and moreover 
    \[
        \norm{f}_r \leq \norm{f}_p^\alpha \norm{f}_q^{1-\alpha}
    \]
    where \(\alpha\) is such that \(\frac{1}{r}=\frac{\alpha}{p} + \frac{1-\alpha}{q}\)
\end{theorem}
\begin{proof}
    For exercise. Use Holder
\end{proof}

\subsection*{Completeness and Separability}
\begin{theorem}
    For \(1 \leq p \leq \infty\), \(L^p(X, \mathcal{M}, \mu)\) is a Banach space (with reference to \(\normdot_p\))
\end{theorem}
\begin{proof}
    \item \(p < \infty\).
    
    By using the characterization of completeness with the series, we want to show that 
    if \(\{f_n\} \subseteq L^p(X)\), and \(\sum_{k=1}^\infty\norm{f_k}_p < \infty \Rightarrow \sum_{k=1}^\infty f_k \) is convergent in \(L^p\), 
    namely \(s_n = \sum_{k=1}^n f_k\) has a limit in \(L^p\): \(\norm{s_n -s}_p \to 0 \) as \(n \to \infty\). 

    Let then \(\{f_n\} \subseteq L^p(X)\) s.t. 
    \[
        \sum_{k=1}^\infty \norm{f_k}_p = M < \infty 
    \]
    Define 
    \[
        g_n(x) = \sum_{k=1}^n \abs{f_k(x)}
    \]
    By Minkowski, \(\norm{g_n}_p \leq \sum_{k=1}^n \norm{f_k}_p \leq M < \infty\). 
    Moreover, for every \(x \in X\) fixed, \(\{g_n(x)\}\) is increasing \(\Rightarrow g_n(x) \to g(x)\) as \(n \to \infty\), \(\forall\; x \in X\)
    \[
        \int_X \abs{g}^p \, d\mu \overset{\text{\tiny{Monot conv}}}{=} \lim_n \int_X \abs{g_n}^p \leq M^p < \infty \Rightarrow g \in L^p(X)
    \]
    \(\Rightarrow \abs{g}^p \) is finite a.e.:
    \[
        \sum_{k=1}^\infty \abs{f_k(x)} < \infty \text{  for a.e. } x \in X
    \]
    \[
        \Rightarrow \sum_{k=1}^\infty f_k(x) \text{ is convergent a.e. to a limit } s(x)
    \]
    Thus, we proved that \(s_n(x) = \sum_{k=1}^n f_k(x) \to s(x)\) a.e. in \(X\). Namely \(\abs{s_n - s }^p \to 0\) a.e. in \(X\). To find a dominating function for \(\abs{s_n -s}^p\), we start by observing that
    \[
        \abs{s_n(x)} = \abs{\sum_{k=1}^n f_k(x)} \leq \sum_{k=1}^n \abs{f_k(x)} = g_n(x) \leq g(x) \text{ for a.e. } x \in X
    \]
    Therefore
    \[
        \abs{s_n -s }^p \leq 2^{p-1}(\abs{s_n}^p + \abs{s}^p) \leq 2^{p-1} (g^p + g^p ) = 2^p g^p \in L^1(X)
    \]
    By the dominated convergence theorem
    \[
        \int_X \abs{s_n -s }^p \, d\mu \to 0 \Leftrightarrow \norm{s_n -s}_p \to 0
    \]
    Thus \(L^p\) is complete.
    \item \(p=\infty\) exercise
\end{proof}

To speak about separability, we give a 
\begin{definition}
    \(g: \Omega \subset \real^n \to \real\). The support of \(g\) is
    \[
      \mbox{supp}\, g = \overline{\left\{ x \in \Omega : g(x) \neq 0 \right\}}
    \]
    Also 
    \[
        \mathcal{C}^0_C = \left\{ f \in \mathcal{C}^0\left( \Omega \right) : \mbox{supp} \, f \mbox{ is compact in } \Omega\right\} = \mathcal{C}^0_O(\Omega) = \mathcal{C}_C(\Omega)
    \]
\end{definition}
\begin{theorem}[Lusin Theorem]
    \(\Omega \in \mathcal{L}(\real), \lambda(\real) < +\infty\). Let also \(f : \real \to \real\) measurable, s.t. \(f\equiv 0\) in \(\Omega^C\).

    Then \(\forall \; \epsilon > 0 \; \exists \; g \in \mathcal{C}^0_C(\real)\) s.t.
    \[
        \lambda \left( \left\{ x \in \real : g(x) \neq f(x) \right\} \right) < \epsilon
    \]
    and
    \[
        \sup_{\real} \abs{g} \leq \sup_\real \abs{f}
    \]
\end{theorem}
\begin{definition}
    Given \(s \mbox{ simple function } = \sum_{k=1}^n a_k \chi_{E_k}\), where \(E_1, \ldots, E_n\) are \(\mathcal{L}\)-measurable sets, \(a_1, \ldots, a_n \in \real\). 
    \[
        E_1 \cup E_2 \cup \ldots \cup E_n = \real
    \]
    We consider
    \[
        \tilde{\mathcal{S}}(\real) = \left\{ s \mbox{ simple in } \real \mbox{ s.t. } \lambda\left( \left\{ s \neq 0 \right\} \right) < +\infty \right\}
    \]
    What does it mean for a simple function to be in \(L^p(\real)\)? 
    \[
        \int_{\real} \abs{s}^p \, d\mu = \sum_{k=1}^n a_k^p \lambda(E_k) < +\infty
    \tag*{\(1 \leq p \leq +\infty\)}\]
    iff \(s \equiv 0\) outside a set of finite measure \(\Leftrightarrow s \in \tilde{\mathcal{S}}(\real)\).

    \(\tilde{\mathcal{S}}(\real)\) is the set of integrable simple functions.
\end{definition}
\begin{theorem}
    \(\tilde{\mathcal{S}}(\real)\) is dense in \(L^p\), \(\forall \; p \in (1, +\infty)\)
\end{theorem}

\begin{proof}
    \(f \in L^p(\real), f \geq 0\) a.e. in \(\real\).

    We want to show that \(\exists \; \left\{ s_n \right\} \subseteq \tilde{\mathcal{S}}(\real)\) s.t. \(\norm{s_n - f}_p \to 0\).

    By the simple approximation theorem, \(\exists \; \left\{ s_n \right\}\) of simple functions s.t. \(\left\{ s_n(x) \right\}\) is increasing, for every \(x\), and \(s_n \to f\) pointwise in \(\real\).

    Since \(\abs{s_n}^p \leq f^p \Rightarrow s_n \in L^p\) for every n \(\Rightarrow \left\{ s_n \right\} \subseteq \tilde{\mathcal{S}}(\real)\). Moreover
    \[
        \abs{s_n -f}^p \to 0 \qquad \text{a.e. in } \real
    \]
    \[
        \abs{s_n -f }^p \leq 2^{p-1} (\abs{s_n}^p + \abs{f}^p) \leq 2^p \abs{f}^p \in L^1
    \]
    \(\Rightarrow \) by dominated convergence
    \[
        \int_\real \abs{s_n - f }^p \, d\lambda \to 0 \text{ , namely } \norm{s_n -f}_p \to 0
    \]
    If \(f\) is sign changing, then \(f = f^+ - f^-\)  and argue as before on \(f^+\) and \(f^-\)
\end{proof}

\begin{theorem}
    \( \forall p \in [1, \infty)\), the space \(L^p(\real)\) is separable. 
\end{theorem}
\begin{proof}
    \noindent\underline{sketch}
    \begin{itemize}
        \item Step 1: \(\mathcal{C}_C^0(\real) \) is dense in \(L^p(\real )\), \(\forall \; \leq p \leq \infty\).
        
        Take \(s \in \tilde{\mathcal{S}}(\real)\). Then, by Lusin theorem, \(\exists \; \{f_n\} \subseteq \mathcal{C}_C^0(\real)\) s.t. \(\norm{f_n -s }_p \to 0\).
        Then, since any \(f \in L^p\) can be approximated by simple integrable functions, we have that \(f\) can be approximated by functions in \(\mathcal{C}_C^0(\real)\).

        \item Step 2: 
        
        By Stone Weierstrass, the set of polinomials \(\mathcal{P}(\real)\) is dense in \(\mathcal{C}_C^0(\real)\) with the \(\normdot_\infty\) norm. 
        Since we work with functions with compact support, this implies that \(\mathcal{P}(\real)\) is dense in \(\mathcal{C}_C^0(\real)\) also with respect to \(\normdot_p\)
        \[
            \int_{-M}^M \abs{f - p_n }^p \, d\lambda \leq \norm{f - p_n }^p_\infty 2M \to 0
        \] 
        if \(\norm{f - p_n}_\infty \to 0\), \(\Rightarrow \mathcal{P}(\real)\) is dense in \(L^p(\real)\).

        \(\tilde{\mathcal{P}}(\real) = \{\)polynomials with rational coefficients\(\}\). 
        This is countable, and is dense in \((\mathcal{P}(\real), \normdot_p)\). \(\Rightarrow \) is dense in \(L^p\) 
    \end{itemize}
\end{proof}

What about \(L^\infty(\real)\)? In this case \(\mathcal{C}(\real)\) are not dense in \(L^\infty(\real)\).
For example, consider 
\[
    f(x) =
    \begin{cases}
        1 \quad x \geq 0
        \\ 0 \quad x<0
    \end{cases}
\] 
If \(g \in L^\infty\) s.t. \(\norm{g-f}_\infty < \frac{1}{3}\), then \(g\) cannot be continuous. Assume by contradiction that \(\exists \; g \in \mathcal{C}(\real)\) s.t. \(\norm{g-f}_\infty < \frac{1}{3}\). Then
\[
    \esssup _\real \abs{g(x)- f(x)} < \frac{1}{3}
\]
In particular, \(g(x) < \frac{1}{3}\) \(\forall \; x<0\)
\[
    \Rightarrow \lim_{x \to 0^-} g(x) \leq \frac{1}{3}
\]
On the other hand, \(g(x) > \frac{2}{3} \quad \forall \; x >0\)
\[
    \Rightarrow g(0)=\lim_{x \to 0^+} g(x) \geq \frac{2}{3}
\]