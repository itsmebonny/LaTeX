\documentclass[a4paper,12pt]{article}
\usepackage{amssymb}
\usepackage{amsmath}
\usepackage{hhline}
\usepackage{hyperref}
\usepackage{bm}
\usepackage[margin=2cm]{geometry}

\usepackage{amsthm}
\usepackage{tikz}
\usepackage{tabularx}
\usepackage{graphicx}
\usetikzlibrary{shapes.geometric, arrows}
\tikzstyle{startstop} = [rectangle, rounded corners, minimum width=3cm, minimum height=1cm,text centered, draw=black, fill=red!30]
\tikzstyle{io} = [trapezium, trapezium left angle=70, trapezium right angle=110, minimum width=3cm, minimum height=1cm, text centered, draw=black, fill=blue!30]
\tikzstyle{process} = [rectangle, minimum width=3cm, minimum height=1cm, text centered, draw=black, fill=orange!30]
\tikzstyle{decision} = [diamond,aspect = 2, text centered, draw=black, fill=green!30]
\tikzstyle{arrow} = [thick,->,>=stealth]
\usepackage{newunicodechar}
\newunicodechar{≠}{\ensuremath{\not =}}
\usepackage{textcomp}
\usepackage[makeroom]{cancel}

\newlength\mylength
\setlength\mylength{0.1cm}
\newcolumntype{Y}{>{\Centering\arraybackslash}X}

\AtBeginEnvironment{array}{\everymath{\displaystyle}}
\newtheoremstyle{break}
  {\partopsep}{\topsep}%  
  {\normalfont}{}
  {\bfseries}{}%
  {\newline}{}%
  \theoremstyle{break}
\newtheorem{theorem}{Theorem}[section]
\newtheorem{corollary}{Corollary}[section]
\newtheorem{proposition}{Proposition}[section]
\newtheorem{remark}[section]{Remark}
\newtheorem{lemma}{Lemma}[section]
\renewcommand*{\proofname}{\textbf{Proof}}
\renewcommand\qedsymbol{$\bigstar$}
\newtheorem{definition}{Definition}[section]
\renewcommand\labelenumi{(\theenumi)}

\let\oldemptyset\emptyset
\let\emptyset\varnothing

\newcommand{\ind}{\perp\!\!\!\!\perp} 
\newcommand{\measurespace}{(X, \mathcal{M}, \mu)}
\newcommand{\sigalg}{\sigma\mbox{-algebra}}
\newcommand{\boreal}{\mathcal{B}(\mathbb{R})}
\newcommand{\real}{\mathbb{R}}
\newcommand{\code}[1]{\texttt{#1}}
\newcommand{\xdownarrow}[1]{%
  {\left\downarrow\vbox to #1{}\right.\kern-\nulldelimiterspace}
}
\newcommand{\xuparrow}[1]{%
  {\left\uparrow\vbox to #1{}\right.\kern-\nulldelimiterspace}
}
\newcommand{\arrvline}{\hfil\kern\arraycolsep\vline\kern-\arraycolsep\hfilneg}

\long\def\symbolfootnotemark[#1]#2{\begingroup%
\def\thefootnote{\fnsymbol{footnote}}\footnotetext[#1]{#2}\footnotemark[#1]\endgroup}

\long\def\symbolfootnotetext[#1]#2{\begingroup%
\def\thefootnote{\fnsymbol{footnote}}\footnotetext[#1]{#2}\endgroup}


\numberwithin{equation}{section}





\begin{document}
\section{Lesson 22/09/2022}
We will mainly focus on 2 situations:
\begin{enumerate}
    \item  \(((X, \mathcal{M}))\) is a measurable space obtained by means of an outer measure. Ex: \((\mathbb{R}^n, \mathcal{L}(\mathbb{R}^n))\), \((Y, d_y)\) metric space
    If \(X \to Y\) is (Lebesgue) measurable \(\Longleftrightarrow\) \((\mathcal{M}, \mathcal{B}(Y))\) is measurable
    \item \((X, d_X), (Y, d_Y)\) are metric spaces \(\longrightarrow (X, \mathcal{B}(X))\)
    If \(X \to Y\) are borel measurable \(\Longleftrightarrow (\mathcal{B}(X), \mathcal{B}(Y)) \)measurable
\end{enumerate}
\begin{remark}
     \(f\) is Lebesgue measurable if the continuity of the borel set is a Lebesgue-measurable set.
\end{remark}
\begin{proposition}
    \begin{enumerate}
        \item \((X, d_X), (Y, d_Y)\) metric spaces. If \(X \to Y\) is continuous, then is Borel measurable
        \item \((Y, d_Y)\) metric space. If \(\mathbb{R}^n \to Y\) is continuous, then it is a Lebesgue measure.
    \end{enumerate}
\end{proposition}
\begin{proof}
    \begin{enumerate}
        \item \(f\) is continuous \(\Longleftrightarrow f^{-1}(A)\) is open \(\forall \; A \in Y\)
        open \(\Longrightarrow\) \(f^{-1}(A) \in \mathcal{B}(Y) \; \forall \; A \in Y\) open
        Since \(\mathcal{B}(Y) = \sigma_0(\mbox{open sets})\) by proposition \(1\) thus implies that \(f\) is Borel measurable
        \item \(f\) is continuous \(\Longrightarrow\) \(f\) is Borel measurable
        mancano pezzi namely \(f\) is Lebesgue measurable
    \end{enumerate}
\end{proof}
\begin{proposition}
    \((X, \mathcal{M})\) measurable space, \((X, d_Y), (Y, d_Y)\) metric spaces. 
    if \(f: X \to Y\) is \(\mathcal{M}, \mathcal{B}(Y)\)-measurable and \(g : Y \to Z\) is continuous \(\Longrightarrow\) \(g \circ f : x \to Z\) is \(\mathcal{M}, \mathcal{B}(Y)\)-measurable
\end{proposition}
\begin{proposition}
    \((X, \mathcal{M})\) measurable space 
    Let \(\Phi : \mathbb{R}^n \to Y\) be continuous where \((Y, d_Y)\) is a metric space. Then \(h: X\to Y\) defined by \(h(x) = \Phi(u(x), boh)\) is \(\mathcal{M}, \mathcal{B}(Y)\)-measurable.
\end{proposition}
\begin{proof}
    Define \(f: X \to \mathbb{R}^n\), \(f(x) = u(x), v(x)\). By def \(h = \Phi \circ f\) by prop 3 if we show that \(f\) is measurable, then \(h\) is measurable. It can be proved that \[\mathcal{B}(\mathbb{R}^2) = \sigma_0 \left(\left\lbrace (a_1, b_1) \times (a_2, b_2): a,b \in \mathbb{R}\right\rbrace\right)\]
    pezzi
    \(f^{-1}(\mathcal{R} \in \mathcal{M}) \quad \forall \mbox{open rectangle in }\mathcal{R}^2\)
    \(R = I \times J\)
    \(F^{-1} = \left\lbrace x \in X \right\rbrace\) 
\end{proof}
\begin{remark}
    roba 
    \[
        g(x) = \begin{cases}
            x & \mbox{where } x \geq 0 \\
            0 & \mbox{where } x < 0
        \end{cases}
    \]
\end{remark}
cosine 
\((X,\mathcal{M})\) measurable space, then such a  function f is measurable iff
\[
    f^{-1}(a, +\infty)] \in \mathcal{M} \quad \forall a \in \mathcal{R}
\]
LEt now \(\left\lbrace f_n \right\rbrace\) be a Sequence of measurable functions from \(X\) to \(\bar{\mathcal{R}}\). Then we define \[
    (\inf_n f_n)(x) = \inf_n f_n(x)
\]
\[
    (\sup_n f_n)(x) = \sup_n f_n(x)
\]
\[
    (\liminf_n f_n)(x) = \liminf_n f_n(x)
\]
\[
    (\limsup_n f_n)(x) = \limsup_n f_n(x)
\]
\[
    (\lim_n f_n)(x) = \lim_n f_n(x) \quad \mbox{if the limit exists}
\]
\begin{proposition}
    \((X, \mathcal{M})\) measurable space, \(f_n : X \to \bar{\mathcal{R}}\) measurable, then 
    \(\sup \inf \liminf \limsup\) of \(f_n\) are measurable, in particular if \(\lim f_n\) exists, then \(f\) is measurable
\end{proposition}
\begin{proof}
    \((\sup f_n)^{-1} ((a, \infty]) = \left\lbrace x \in X : \sup f_n(x) > a \right\rbrace\) (manca pezzi)
    \[
        \bigcup \left\lbrace x \in X : f_n(x) > a \right\rbrace
    \]
    Then \((\sup f_n)^{-1} ((a, \infty])\) is measurable, cose da aggiungere
Noe the limsup
\[\limsup_n f_n = \lim_n (\sup_{k > n} f_n(x))\]
cose cose o
\end{proof}
\subsection*{Simple functions}
\begin{definition}
    \((X, \mathcal{M})\) measurable space. A measurable function s: \(X \to \bar{\mathcal{R}}\) is said to be simple if \(s(X)\) is a finite set 
    altre cose 
    Then \(s(x) = \sum_{n = 1} a_n \chi_{E_n}(x)\) where \(E_n\) is a measurable set 
    sistemare. 
\end{definition}
\underline{Particular case}: if s:\(\mathbb{R} \to \bar{\mathbb{R}}\) and each \(E_n\) is a finite union of intervals, then \(s\) is said to be a STEP FUNCTION.

The idea is to approximate functions with simple functions.
\begin{theorem}
    \((X,\mathcal{M})\) measurable space, \(f: X \to [0, \infty]\) measurable. Then \(\exists\) a sequence \(\left\lbrace s_n \right\rbrace\) of simple functions s.t. 
    \[
        0 \leq s_1 \leq \ldots \leq s_n \leq \ldots \leq f \quad \mbox{pointwise}
    \]
    and \(s_n(x) \to f(x)\)
Moreover if f is bounded then \(s_n \to f\) uniformly on \(X\) as \(n \to \infty\)
\end{theorem}
\begin{proof}[f is bounded]
    Fix \(n \in \mathbb{N}\) and divide \([0,n)\) in \(n \cdot 2^n\) intervals called \(I_j = [a_j,b_j)\) with lenght \(\frac{1}{2^n}\)

    Let \(E_0 = f^{-1}([n, \infty)), E_j = f^{-1}([a_j, b_j))\) for \(j = 1, \ldots, n\cdot 2^n\)
    
    We let Array

    Namely we define 
    \[
    s_n (x) = n\chi_{E_0}(X) + \sum_{j =1}^{n \cdot 2^n} a_j \chi_{E_j}(x)    
    \]
    Then \(s_n \leq s_{n+1}\) by contradiction

    Then any \(x \in X\) stays in \(f^{-1}([a_j, b_j))\) for some j \(\Longrightarrow \) 
\end{proof}
\end{document}