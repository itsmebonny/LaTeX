\section{Lecture 14/12/2022}

\subsection*{Dual space of a Hilbert space}
Observe that, if \(y \in H\), then we can define \(\Lambda_y : H \to \real\) as
\[
    \Lambda_y x = \langle y, x\rangle  
\]
It is linear (\(\scalardot   \) is bilinear), and it is bounded: 
\[
    \abs{\Lambda_y x}= \abs{\langle y, x\rangle  } \leq \norm{y} \norm{x} \quad \forall\; x, y
\]
\(\Rightarrow \Lambda_y\) is bounded, with \(\norm{\Lambda_y}_* \leq \norm{y}\)

Moreover, 
\[
    \Lambda_y \left(\frac{y}{\norm{y}}\right) = \langle y, \frac{y}{\norm{y}}\rangle   = \norm{y}
\]
\[
    \Rightarrow \norm{\Lambda_y}_* = \sup_{\norm{x} \leq 1} \abs{\Lambda_y x} \geq \abs{\Lambda_y (\frac{y}{\norm{y}})} = \norm{y}
\]
Thus \( \norm{\Lambda_y}_* = \norm{y}\), and the map 
\[
    \begin{array}{rl}
        i: & H \to H^* \\
        & y \mapsto \Lambda_y    
    \end{array}
\]
is an isometry from \(H\) into \(i(H) \subset {H^*}\).

Are there other elements in \(H^*\)?

\begin{theorem}[Riesz Representation Theorem]
    \(\forall \; \Lambda \in H^* \) \(\exists! \; y \in H\) s.t. \(\Lambda = \Lambda_y\), namely
    \[
        \Lambda x = \langle y, x\rangle   \qquad \forall x \in H
    \]
    Moreover, the map \(i\) is an isometric isomorphism. We can identify \(H^*\) with \(H\)
\end{theorem}

\begin{corollary}
    Any Hilbert space is reflexive. 
\end{corollary}
\begin{remark}
    Any Hilbert space is uniformly convex.
\end{remark}

\begin{itemize}
    \item Riesz in \(L^p\): \(L^p \) is uniformly convex \(\Rightarrow L^p\) is reflexive. We used this fact to prove Riesz in \(L^p\)
    \item Riesz in Hilbert: direct proof of \(H^* = H \Rightarrow H\) is reflexive.
\end{itemize}
Both strategies can be adopted in both contexts.

\begin{proof}
    \begin{itemize}
        \item We show that \(\forall\; \Lambda \in H^* \) \(\exists\; y \in H\) s.t. \(\Lambda = \Lambda_y\)
        
        If \(\Lambda = 0 \Rightarrow \Lambda = \Lambda_0\) \((\Lambda_0 x = \langle 0, x\rangle   = 0)\)

        Suppose \(\Lambda \neq 0. \) \(\ker(\Lambda) = \Lambda^{-1}(\{0\}) \) is a closed (since \(\Lambda\) is continuous) subspace, \(\neq H\). \(\Rightarrow\) we consider \(\ker(\Lambda)^\perp \neq \{0\}\). Let
        \[
            z \in \ker(\Lambda)^\perp, \quad \norm{z}=1
        \]
        For \(x \in H\), we have
        \[
            x - \frac{\Lambda x}{\Lambda z} z \in \ker(\Lambda)
        \]
        Indeed, \(\Lambda \left( \frac{\Lambda x}{\Lambda z} z \right) \overset{linearity}{=} \Lambda x - \frac{\Lambda x}{\Lambda z} \Lambda z = 0\). Then, since \(z\) is orthogonal to any element of \(\ker(\Lambda)\), 
        \[
            \langle z, x - \frac{\Lambda x}{\Lambda z} z\rangle   = 0 \qquad \forall\; x \in H
        \] 
        \(\scalardot  \) is bilinear: the left hand side is 
        \[
            \langle z, x\rangle   - \frac{\Lambda x}{\Lambda z} \norm{z}^2 \Rightarrow \langle z, x\rangle   = \frac{\Lambda x}{\Lambda z} 
        \]
        \[
            \Lambda x = \langle (\Lambda z)z, x\rangle   \qquad \forall \; x \in H
        \]
        So the thesis is proved for \(y = (\Lambda z)z\).

        \item The uniqueness of \(y\) is easy.
        \[
            \langle x, y_1\rangle   = \langle x, y_2\rangle   \quad \forall \; x \in H
        \]
        Then \(\langle x, y_1-y_2\rangle   = 0\) \(\forall \; x \in H\). We choose \(x = y_1 -y_2\):
        \[
            \norm{y_1 - y_2}^2 = 0 \Rightarrow y_1 = y_2
        \]
    \end{itemize}
\end{proof}

Consequence: \(H\) Hilbert space.
\[
    x_n \rightharpoonup x \text{ weakly in } H \Leftrightarrow \langle x_n, y\rangle   \to \langle x, y\rangle   \quad \forall\; y \in H
\]
Sometimes weak convergence + something else \(\Rightarrow\) strong convergence. For instance
\begin{proposition}
    \(H\) Hilbert. If \(x_n \rightharpoonup x\) weakly in \(H\), and \(\norm{x_n} \to \norm{x}\) \(\Rightarrow x_n \to x\) in \(H\), namely \(\norm{x_n - x} \to 0\)
\end{proposition}
\begin{proof}
    \[
        \norm{x_n - x }^2 = \norm{x_n}^2 - 2 \langle x_n, x\rangle   + \norm{x}^2 = (*)
    \]
    \(\langle x_n, x\rangle   \to \langle x, x\rangle   = \norm{x}^2 \) by weak convergence.
    \[
        (*) = \norm{x}^2 - 2 \norm{x}^2 + \norm{x}^2 = 0
    \]
\end{proof}

\subsection*{Orthonormal Basis}

In \(\real^n\), we have the canonical basis
\[
    e_1, ..., e_n \in \real^n
\]
s.t.
\[
    e_j^{(k)} = 
    \begin{cases}
        1 & k=j \\
        0 & k \neq j   
    \end{cases}
\]
There elements are \(\perp\): \(\langle e_i, e_j\rangle   = 0\) \(\forall\; i \neq j\). \(\norm{e_i}=1\) \(\forall\; i\).

Moreover, \(e_1, ..., e_n\) are a basis, namely \(\forall\; v \in \real^n\) \(\exists! \;\) expression
\[
    v = \sum_{i=1}^n v_i e_i = \sum_{i=1}^n \langle v, e_i\rangle   e_i
\]
In particular, \(v=0 \Leftrightarrow \langle v, e_i\rangle  = 0\) \(\forall\; i\). Do we have an analogue in Hilbert spaces?

\begin{definition}
    \(S \subset H\) is called orthonormal if 
    \begin{itemize}
        \item \(x \perp y\) \(\forall\; x \neq y\), \(x\), \(y \in S\)
        \item \(\norm{x} = 1\) \(\forall x \in S\)
    \end{itemize}
\end{definition}
\begin{definition}
    An orthonormal set is an Hilbert Basis (or is \textbf{complete}) if \(S^\perp = \{0\}\), namely if 
    \[
        \langle u, x\rangle   = 0 \quad \forall x \in S \Rightarrow u=0
    \]
\end{definition}
\begin{theorem}
    \(H\) Hilbert space, \(H \neq \{0\}\). Then \(H\) has an Hilbert basis.

\noindent Moreover, \(H\) is a separable Hilbert space \(\Leftrightarrow \) it has a finite and countable Hilbert basis.
\end{theorem}

Example:
\begin{itemize}
    \item \(H = l^2\). \(H\) is separable
    
    An Hilbert basis is \(\{e_n\}_{n \in \natural}\) defined By
    \[
        e_n^{(k)} = 
        \begin{cases}
            1 & k=n \\
            0 & k \neq n
        \end{cases}
    \]

    \item \(H = L^2([-\pi, \pi])\)
    
    An Hilbert basis is
    \[
        \lbrace \frac{1}{\sqrt{2\pi}}, \frac{\sin(nx)}{\sqrt{\pi}}, \frac{cos(nx)}{\sqrt{\pi}} \rbrace \quad n \in \natural
    \]
\end{itemize}

\begin{remark}
    Hamel basis \(\neq\) Hilbert basis.

    \(X \; \infty-\) dimensional \(\Rightarrow\) any Hamel basis of \(X\) is uncountable.

    \(H \; \infty-\) dimensional and separable \(\Rightarrow\) any Hilbert basis is countable
\end{remark}

The usefulness of Hilbert basis stays in the fact that they allow us to reason component by component.

\begin{theorem}[Bessel inequality]
    \(H\) separable Hilbert space. \(\{u_n\}_{n \in \natural}\) orthonormal set. Then \(\forall \; x \in H\):
    \[
        \sum_{n=1}^\infty \abs{\langle x, u_n\rangle }^2 \leq \norm{x}^2
    \]
\end{theorem}
\begin{theorem}[Generalized Fourier Series]
    \(H\) separable Hilbert space, \(\{u_n\}\) Hilbert basis.
    Then any \(x \in H\) can be written in a unique way as 
    \[
        x = \sum_{n=1}^\infty \langle x, u_n\rangle   u_n \qquad \langle x, u_n\rangle   \text{ Fourier coefficient of }x
    \]
    Moreover, \(\forall\; y \in H\) we have
    \[
        \langle x, y\rangle   = \sum_{n=1}^\infty \langle x, u_n\rangle   \langle y, u_n \rangle  
    \]
    and 
    \[
        \norm{x}^2 = \sum_{n=1}^\infty (\langle x, u_n\rangle  )^2 \tag*{Parseval identity}
    \]
\end{theorem}
\begin{theorem}
    \(H\) separable Hilbert space. Then \(H\) is isomorphic to \(l^2\) as Hilbert space: namely \(\exists\) an isomorphism \(\phi: H \to l^2\) s.t.
    \[
        \langle x, y\rangle  _H = \langle \phi(x), \phi(y)\rangle  _{l^2} \qquad \forall x, y \in H
    \]
\end{theorem}
\begin{proof}
    \(\exists\) a countable Hilbert basis, and \(\forall\; x \in H\)
    \[
        x = \sum_{n=1}^\infty \langle x, u_n\rangle   u_n
    \]
    Then the desired isomorphism is
    \[
        \begin{array}{rl}
            \phi: & H \to l^2 \\
            & x \to \sum_{n=1}^\infty \langle x, u_n\rangle   e_n
        \end{array}
    \]
\end{proof}

\begin{corollary}
    \(H\) separable Hilbert space, \(\dim H = \infty\). \(\{u_n\}_{n \in \natural}\) Hilbert basis. Then \(u_n \rightharpoonup 0\) weakly in \(H\), but \(u_n \nrightarrow 0\) in \(H\).
\end{corollary}
\begin{proof}
    \[
        \norm{u_n}=1 \quad \forall\; n \Rightarrow \norm{u_n -0} \nrightarrow 0 
    \]
    On the other hand, we know that \(\forall \; x \in H\)
    \[
        \norm{x}^2 = \sum_{n=1}^\infty \abs{\langle x, u_n\rangle  }^2 < \infty
    \]
    It is then necessary that
    \[
        \langle x, u_n\rangle   \to 0 \text{ as } n \to \infty \qquad \forall\; x \in H
    \]
    By Riesz, this means that \(u_n \rightharpoonup 0\) in \(H\)
\end{proof}

Ex: \(H = L^2([-\pi, \pi])\). Then the previous corollary tells that (Riemann - Lebesgue lemma)
\[
    \int_{-\pi}^\pi f(x) \sin (nx) \, dx \to 0 \text{ as } n \to \infty
\]                
\(\sin(nx) \rightharpoonup 0\) in \(L^2([-\pi, \pi])\) as \(n \to \infty\). Note that \(\{sin(nx)\}_{n \in \natural}\) does not converge for a.e. \(x\).

Weak convergence in \(L^2\) and pointwise or a.e. convergence are not related.

\(\{sin(nx)\}\) does not converge a.e. on \([-\pi, \pi]\), not even up to subsequences. 
The same is true in \(L^p\), \(p \neq 2\). Even in this case
\[
    \sin(nx) \rightharpoonup 0 \text{ weakly in } L^p([a,b]) \quad (p \in [1, \infty)
\] 
but we don't have a.e. convergence.

\begin{proposition}
    \(p \in [1, \infty)\). Suppose that \(f_n \rightharpoonup f\) in \(L^p(X)\), and that \(f_n \to g\) a.e. in \(X\). Then \(f=g\) a.e.
\end{proposition}

\begin{proposition}
    \(X\) Banach, \(V\) subspace of \(X^*\), dense in \(X^*\). Suppose that \(\{x_n\} \subset X\) is bounded, and that 
    \[
        L x_n \to Lx \qquad \forall\; L \in V
    \]
    Then
    \[
        L x_n \to Lx \qquad L \in X^*
    \]
    namely \(x_n \rightharpoonup x\) weakly in \(X\).
\end{proposition}
\begin{proof}
    (ex)
\end{proof}

Consequence: \(I \subset \real\) interval (\(I= \real\) is fine). \(\{f_n\} \subseteq L^p(I)\), \(p \in (1, \infty)\). 
\(\{f_n\}\) bounded in \(L^p\): \(\exists \; C>  0\) s.t \(\norm{f_n}_{L^p} \leq C\), \(\forall \; n\). Then:

\begin{itemize}
    \item If 
    \[
        \int_I f_n \phi \to \int_I f \phi \qquad \forall \; \phi \in \mathcal{C}_C(I)
    \]
    \(\Rightarrow f_n \rightharpoonup f\) weakly in \(L^p(I)\)
    \item If 
    \[
        \int_a^b f_n \to \int_a^b f \quad \forall (a, b) \subset I
    \]
    \(\Rightarrow f_n \rightharpoonup f\) weakly in \(L^p(I)\)
\end{itemize}

Some useful facts on bounded operators in Hilbert Spaces.

\(H\) Hilbert space. 
\begin{proposition}
    If \(T \in \mathcal{L}(H)\), then
    \[
        \norm{T}_{\mathcal{L}(H)} = \sup_{\norm{x}=\norm{y}=1} \abs{\langle Tx, y\rangle  }
    \]
\end{proposition}
\begin{definition}
    \(T \) is called \textbf{symmetric} (or self adjoint) if 
    \[
        \langle Tx, y\rangle   = \langle x, Ty\rangle   \qquad \forall x, y \in H
    \]
\end{definition}
\begin{proposition}
    Let \(T \in \mathcal{L}(H)\) be symmetric. Then 
    \[
        \norm{T}_{\mathcal{L}(H)} = \sup_{\norm{x}= 1} \abs{\langle Tx, x\rangle  }
    \]
\end{proposition}

Example: \(K \in L^2([0, 1]x[0, 1])\). Let \(T: L^2([0, 1]) \to L^2([0, 1])\) be defined by 
\[
    (Tf)(t) = \int_0^1 K(s, t) f(s) \,ds
\]
\(T \in \mathcal{L}(L^2([0, 1]))\). It is symmetric \(\Leftrightarrow K(s, t) = K(t, s)\) \(\forall \; s, t\)

\subsection*{Spectral Theory}
In what follows, \(E\) is a Banach space and \(T \in \mathcal{L}(E)\).

\begin{definition}
    The \textbf{resolvent} of \(T\) is 
    \[
        \rho(T) = \{ \lambda \in \real: T - \lambda I \text{ is bijective from } E \text{ to } E\}
    \]
\end{definition}
\begin{definition}
    The \textbf{spectrum} of \(T\) is    
    \[
        \sigma(T) = \real \setminus \rho(T)
    \]
\end{definition}
\begin{definition}
    \(\lambda\) is an \textbf{eigenvalue} of \(T\), \(\lambda \in EV(T) \), if 
    \[
        \ker(T- \lambda I) \neq \emptyset
    \]
    (\(T - \lambda I\) is not injective), namely if \(\exists \; u \in E\) s.t. \(u \neq 0\) and
    \[
        T u = \lambda u
    \]
    In this case, \(u\) is called eigenvector and \(\ker(T-\lambda I)\) is the eigenspace of \(\lambda\).
\end{definition}
\begin{remark}
    \(EV(T) \subset \sigma(T)\)
\end{remark}
\begin{remark}
    In finite dimension, linear operators can be represented by matrices.

    \(A \; n \times n\) matrix. We know that \(x \mapsto Ax\) is bijective \(\Leftrightarrow \) it is injective \(\Leftrightarrow \) det\(A \neq 0\). In particular, in finite dimension \(\sigma(A) = EV(A)\). This is false in \(\infty \) dimension.
\end{remark}

Basic fact:
\begin{theorem}
    \(E\) Banach, \(T \in \mathcal{L}(E)\). Then \(\sigma(T) \subset \real\) is compact, and 
    \[
        \sigma(T) \subset \left[ - \norm{T}_\mathcal{L}, \norm{T}_\mathcal{L} \right]
    \]
\end{theorem}

In general we cannot say much more.

Ex: in \(l^2\), consider the left shift:
\[
    T_l(x^{(0)}, x^{(1)}, x^{(2)},... x^{(n)},...) = (x^{(1)}, x^{(2)}, x^{(3)},... x^{(n+1)},...)
\]
\(T_l \in \mathcal{L}(l^2)\), \(\norm{T_l}=1\). \(EV(T_l)=?\). We have to solve
\[
    T_l x = \lambda x \qquad \text{ for some } \lambda \in \real, x \in l^2 \setminus \{0\}
\]
\(\Rightarrow x^{(1)}=\lambda x^{(0)}\). \(x^{(n+1)} = \lambda x^{(n)} = \lambda^{n+1} x^{(0)}\). \(\forall \; \lambda \in \real\), the sequence
\[
    x = x^{(0)} \left(1, \lambda,\lambda^2, ..., \lambda^n, ...\right)
\]
is a solution of \(T_l x = \lambda x\).
\[
    x \in l^2 \Leftrightarrow \sum_{n=0}^\infty (\lambda^n)^2 < \infty \Leftrightarrow \sum_{n=0}^\infty (\lambda^2)^n < \infty \Leftrightarrow \abs{\lambda} < 1
\]
Any \(\lambda \in (-1, 1)\) is an e.v. of \(T_l\). Moreover, \(\sigma(T_l)\) is a compact set which is included in \([-1, 1]\) and contains \(EV(T_l) =(-1, 1)\)
\[
    \Rightarrow \sigma(T) = [-1, 1]
\]
