\section{Lecture 15/12/2022}

We focus in what follows on the following case: \(H\) separable Hilbert space, \(T \in \mathcal{K}(H)\) and symmetric.

\begin{proposition}
    Let \(d = \norm{T}_{\mathcal{L}(H)}\). Then either \(d\) or \(-d\) is an eigenvalue of \(T\)
\end{proposition}
Recall: \(T \in \mathcal{K}(H)\), \(u_n \rightharpoonup u\) weakly \(\Rightarrow T u_n \to Tu\) strongly in \(H\)

\begin{proof}
    \(d \neq 0\) (otherwise \(T=0\)). We know that
    \[
        d = \sup_{\norm{u}=1} \abs{\langle Tu, u\rangle}
    \]
    Take a maximizing sequence for \(d\):
    \[
        \exists \{u_n\} \subset H \text{ s.t. } \norm{u_n}=1 \qquad \abs{\langle Tu_n, u_n\rangle} \to d    
    \]
    \(\{u_n\}\) is bounded \(\Rightarrow\) by Banach Alaoglu in reflexive spaces (any Hilbert space is reflexive) we can extract \(\{u_{n_k}\}\) s.t. \(u_{n_k} \rightharpoonup u\) weakly in \(H\), for some \(u\).

    By weak strong continuity, \(T u_{n_k} \to Tu\) strongly in \(H\). From this, we deduce that 
    \[
        \abs{ \langle T u_{n_k}, u_{n_k}\rangle - \langle Tu, u\rangle \to 0} \qquad \text{ as } K \to \infty    
    \]
    We know
    \[
        \abs{ \langle Tu_{n_k}, u_{n_k}\rangle} \leq \abs{\langle T u_{n_k} - Tu, u_{n_k}\rangle} + \abs{ \langle Tu, u_{n_k} - u\rangle} \to 0
    \]
    and also that \(\abs{\langle T u_{n_k}, u_{n_k}\rangle} \to d\)
    \[
        \Rightarrow \abs{\langle Tu, u\rangle} = d
    \]
    and hence \(u \neq 0\)

    \begin{itemize}
        \item Suppose that \(\langle Tu, u\rangle = d\). Then
        \[
            \norm{Tu- du}^2 = \norm{Tu}^2 - 2d \langle Tu, u\rangle + d^2 \norm{u}^2 \leq d^2 - 2d^2 + d^2 = 0
        \]
        \[
            \Rightarrow \norm{Tu - du} = 0 \Rightarrow Tu = du
        \]
        and \(d\) is an eigenvalue.
        \item \(\langle Tu, u\rangle = -d\). Then one can prove that \(-d\) is an eigenvalue.
    \end{itemize}
\end{proof}

\begin{proposition}
    \(\lambda \neq 0\) is an eigenvalue of a compact operator \(T \in \mathcal{K}(E)\), \(E \) Banach. Let \(V_\lambda\) be the eigenspace of \(\lambda\). Then \(\dim V_\lambda < \infty\)
\end{proposition}
\begin{proof}
    Recall that \(I: F \to F \), with \(F \infty\) dimensional. Banach space, cannot be compact. Assume by contradiction that \(V_\lambda\) has \(\infty\) dim. Consider
    \[
        \frac{1}{\lambda} T |_{V_\lambda}: V_\lambda \to V_\lambda 
    \]
    is the identity \(\frac{1}{\lambda} T u = \frac{1}{\lambda} \times \lambda u = u\) \(\forall\; u \in V_\lambda\) 
    So \(\frac{1}{\lambda} T |_{V_\lambda}\) cannot be compact. On the other hand, \(\frac{1}{\lambda} T |_{V_\lambda}\) is compact by assumption.
\end{proof}

\begin{proposition}
    \(H\) Hilbert, \(T \in \mathcal{L}(H)\) symmetric. Then eigenvectors associated with different eigenvalues are orthogonal. 
\end{proposition}
\begin{proof}
    \[
        \begin{array}{ccc}
            Tu_1 = \lambda_1 u_1 & u_1, u_2 \neq 0 & \lambda_1 \neq \lambda_2 \\
            Tu_2 = \lambda_2 u_2 
        \end{array}
    \]
    \[
        \lambda_1 \langle u_1, u_2 \rangle = \langle Tu_1, u_2\rangle = \langle u_1, T u_2\rangle = \lambda_2 \langle u_1, u_2\rangle
    \]
    \[
        \Rightarrow (\lambda_1 - \lambda_2) \langle u_1, u_2\rangle = 0 \Rightarrow \langle u_1, u_2\rangle = 0
    \]
\end{proof}

\begin{theorem}[Spectral Theorem]
    \(H\) separable Hilbert, \(T \in \mathcal{K}(H)\) symmetric. Then
    \begin{enumerate}
        \item \(\sigma(T) \setminus \{0\} = EV (T) \setminus \{0\}\)
        \item \(0 \in \sigma(T)\)
    \end{enumerate}
    and the following alternative holds:
    \begin{enumerate}
        \item either \(T\) has infinitely many distinct eigenvalues, and in this case \(0 \in EV(T)\) and \(\ker T\) is infinite dimensional
        \item or \(EV(T) \setminus \{0\} \) is a sequence tending to \(0\)
    \end{enumerate}
    Moreover, the eigenvectors can be chosen in such a way to form a Hilbert basis of \(H\) (if necessary adding an orthonormal basis of \(\ker T\))
\end{theorem}

\begin{remark}
    \begin{itemize}
        \item \(\forall\) symmetric matrix \(A\), \(n x n \), \(\exists\) an orthonormal basis of \(\real^n\) of eigenvectors
        \item If \(T \in \mathcal{K}(E)\), \(E\) Banach, we can still say that \(0 \in \sigma(T)\) (if \(E\) has \(\infty\) dimension), that \(EV(T) \setminus \{0\} = \sigma(T) \setminus \{0\}\) and that either there are finitely many distinct eigenvectors, or \(EV(T) \setminus \{0\}\) is a sequence tending to \(0\)
    \end{itemize}
\end{remark}

\begin{proof}
    \begin{itemize}
        \item \(0 \in \sigma(T)\) is simple: \(T \) is compact, \(H\) has \(\infty\) dimension \(\Rightarrow\) it can't be surjective.
        \[
            T = T - 0 I \text{ is not bijective: } 0 \notin \rho(T)
        \]
        \item From proposition 1, \(\exists \) an eigenvalue \(\lambda\) with \(\abs{\lambda_0} = \norm{T}_{\mathcal{L}(H)}\). Let \(V_0\) be the associated eigenspace. By proposition \(2\), \(\dim V_0 = N_0 \langle \infty\).
        
        Let \(\{w_1^0, ..., w_{N_0}^0\}\) be an orthonormal basis for \(V_0\). Consider now \(H_1 = V_0^\perp\), so that \(H = V_0 \oplus H_1\). We claim that \(T|_{H_1} \in \mathcal{K}(H_1)\) symmetric. 

        \(T|_{H_1}\) is compact and symmetric, by assumption. We have to check that \(T|_{H_1}: H_1 \to H_1\)
        \[
            u \in H_1 \iff \langle u, w\rangle = 0 \quad \forall\; w \in V_0
        \]
        \[
            \langle Tu, w\rangle = \langle u, Tw\rangle = \langle u, \lambda_0 w\rangle = \lambda_0 \langle u, w\rangle
        \]
        \(\forall\; w \in V_0\), namely \(Tu \in H_1\), \(\forall\; u \in H_1\). 
        \[
            H_1 = V_0^\perp
        \]
        \(\Rightarrow\) it is a closed subspace of \(H\) \(\Rightarrow H_1\) is a Hilbert space.
        \(T|_{H_1}\) is a compact symmetric operator on a separable Hilbert space. Therefore, arguing as before, we have an eigenvalue for \(T\) given by \(\lambda_1\) s.t. 
        \[
            \abs{\lambda_1} = \sup_{
            \begin{array}{l}
                \norm{u} = 1 \\
                u \in H_1    
            \end{array}}
            \abs{\langle Tu, u\rangle}
        \]
        Clearly, \(\abs{\lambda_1} \leq \abs{\lambda_0} = \sup \abs{\langle Tu, u\rangle}\).
        We have an eigenspace \(V_1 \) for \(\lambda_1\), with dimension \(N_1\), and an orthonormal basis \(\{w_1^1, ..., w_{N_1}^1\}\) for \(V_1\). We iterate the process. 
        Either after a finite number of steps we have
        \[
            \lambda_N = \sup{\begin{array}{l}
                \norm{u} = 1 \\
                u \in H_1    
            \end{array}} \abs{\langle Tu, u\rangle} = 0
        \]
        Or \(\{\lambda_n\}\) forms a sequence, s.t. \(\abs{\lambda_n}\) is decreasing.

        Case 1: We can say that 
        \[
            H = V_0 \oplus V_1 \oplus V_2 \oplus ... V_{N-1} \oplus \ker T
        \]
        \(\ker T\) is a closed subspace of \(H\), separable \(\Rightarrow\) we have an orthonormal countable basis \(\{z_1, ... , z_n\}\) of \(\ker T\). Then
        \[
            \{w_1^0, ..., w_{N_0}^0, w_1^1, ..., w_{N_1}^1, ..., w_1^{N-1}, ..., w_{N_{N_1}}{N-1}, z_0, ..., z_n \} 
        \]
        is an orthonormal basis of \(H\), made of eigenvectors.

        Case 2: at first, we show that \(\lambda_n \to 0\). If not, \(\abs{\lambda_n} \to \eta > 0\). 
        Consider then \(\{\frac{w_n}{\lambda_n}\}\), where \(w_n\) is an eigenfunction of \(\lambda\) with \(\norm{w_n}=1\). Then \(\{\frac{w_n}{\lambda_n}\}\) is bounded, and
        \[
            T(\frac{w_n}{\lambda_n}) = \frac{1}{\lambda_n} T w_n = \frac{1}{\lambda_n} \lambda_n w_n = w_n
        \]
        \(\Rightarrow\) by compactness, there exists a subsequence of \(T(\frac{w_n}{\lambda_n}= w_n)\) which is strongly convergent.
        This is not possible, since
        \[
            \norm{w_i - w_j}^2 = 2 \quad \forall\; i \neq j
        \]
        \(\norm{w_i}^2 + \norm{w_j}^2 -2\langle w_i, w_j\rangle\) \(\Rightarrow \lambda_n \to 0\).

        It remains to show that \(x \in V_i^\perp\), \(\forall\; i\), then \(x \in \ker T\). To this end
        \[
            \norm{Tx} = \norm{T|_{H_i} x} \leq \norm{T|_{H_i}}_{\mathcal{H_i}} \norm{x} = \abs{\lambda_i} \norm{x} \quad \forall\;i 
        \]
        Taking \(i \to \infty\), we deduce
        \[
            \norm{Tx} \leq \lim_{i \to \infty} \abs{\lambda_i} \norm{x} = 0
        \]
        \(\Rightarrow\) \(x \in \ker T\). Even in this case, 
        \[
            H = \ker T \oplus V_1 \oplus V_2 \oplus ... \oplus V_n \oplus ...
        \]
        and once again we can consider a basis of eigenvectors.
    \end{itemize}
\end{proof}

\begin{corollary}[Fredholm Alternative]
    \(H\) separable Hilbert space, \(T \in \mathcal{K}(H) \) and symmetric. Then:
    \begin{enumerate}
        \item either \(\forall\; y \in H\) the equation
        \[
            x - Tx = y
        \]
        has a unique solution
        \item or \(\lambda=1\) is an eigenvalue of \(T\), and in this case \(x - TX = y\) can have no solution or infinitely many solutions, depending on \(y\).
    \end{enumerate}
\end{corollary}

\begin{remark}
    \begin{itemize}
        \item Rouché Capelli: \(Ax = y\). \(A\) matrix. Either \(\det A \neq 0\), and then \(\forall\; y \in \real^n \exists!\; \) solution; or \(Ax = y\) can have \(0\) or \(\infty \) many solution.
        \item \(T\) symmetric is not necessary, and the corollary also holds in Banach spaces. 
        \item The corollary is very useful to treat integral equations:
        \[
            u(t) - \int_0^1 K(s, t) u(s) ds = g(t)
        \]
    \end{itemize}
\end{remark}
\begin{proof}
    By the Spectral Theorem, \(\forall \; x \in H\), we can write 
    \[
        x = \sum_{n=1}^\infty \langle x, u_n\rangle u_n
    \] 
    where \(\{u_n \}\) is a Hilbert basis of eigenvectors of \(T\). Also, we have 
    \[
        Tx = \sum_{n=1}^\infty \lambda_n \langle x, u_n\rangle u_n
    \]
    and 
    \[
        y = \sum_{n=1}^\infty \langle y, u_n\rangle u_n
    \]
    Then, the equation \(x - Tx = y\) becomes
    \[
        \sum_{n=1}^\infty (1- \lambda_n) \langle x, u_n\rangle u_n = \sum_{n=1}^\infty \langle y, u_n \rangle u_n
    \]
    \[
        \Rightarrow (1 - \lambda_n) \langle x, u_n\rangle = \langle y, u_n\rangle \quad \forall\; n
    \]
    If \(\lambda_n \neq 1\) \(\forall \; n\), then we take 
    \[
        \langle x, u_n\rangle = \frac{\langle y, u_n \rangle}{1 - \lambda_n} \quad \forall\; n
    \]
    \[
        x = \sum_{n=1}^\infty \frac{\langle y, u_n\rangle}{1 - \lambda_n} u_n
    \]
    is the solution. If instead \(\lambda_n = 1\) for some \(n\), then there are no solution if \(y\) is such that \(\langle y, u_n\rangle \neq 0\):
    \[
        (1 - \lambda_n) \langle x, u_n\rangle = \langle y, u_n\rangle
    \]
    For \(y\) s.t. \(\langle y, u_n\rangle = 0\), we have \(\infty\) many solutions.
\end{proof}