\documentclass[a4paper,12pt]{article}
\usepackage{amssymb}
\usepackage{amsmath}
\usepackage{hhline}
\usepackage{hyperref}
\usepackage{mathtools}
\usepackage{bm}
\usepackage[margin=2cm]{geometry}

\usepackage{amsthm}

\usepackage{tabularx}
\usepackage{graphicx}
\usepackage{physics}
\usepackage{textcomp}


\newlength\mylength
\setlength\mylength{0.1cm}
\newcolumntype{Y}{>{\Centering\arraybackslash}X}

\AtBeginEnvironment{array}{\everymath{\displaystyle}}
\newtheoremstyle{break}
  {\partopsep}{\topsep}%  
  {\normalfont}{}
  {\bfseries}{}%
  {\newline}{}%
  \theoremstyle{break}
\newtheorem{theorem}{Theorem}[section]
\newtheorem{corollary}{Corollary}[section]
\newtheorem{proposition}{Proposition}[section]
\newtheorem{remark}{Remark}[section]
\newtheorem{lemma}{Lemma}[section]
\renewcommand*{\proofname}{\textbf{Proof}}
\renewcommand\qedsymbol{$\bigstar$}
\newtheorem{definition}{Definition}[section]
\renewcommand\labelenumi{(\theenumi)}

\let\oldemptyset\emptyset
\let\emptyset\varnothing
\let\oldepsilon\epsilon
\let\epsilon\varepsilon
\let\oldphi\phi
\let\phi\varphi



\newcommand{\ind}{\perp\!\!\!\!\perp} 
\newcommand{\measurespace}{(X, \mathcal{M}, \mu)}
\newcommand{\sigalg}{\sigma\mbox{-algebra}}
\newcommand{\boreal}{\mathcal{B}(\mathbb{R})}
\renewcommand{\real}{\mathbb{R}}
\renewcommand{\natural}{\mathbb{N}}
\newcommand{\barreal}{\overline{\mathbb{R}}}
\newcommand{\code}[1]{\texttt{#1}}
\newcommand{\xdownarrow}[1]{%
  {\left\downarrow\vbox to #1{}\right.\kern-\nulldelimiterspace}
}
\newcommand{\xuparrow}[1]{%
  {\left\uparrow\vbox to #1{}\right.\kern-\nulldelimiterspace}
}
\newcommand{\arrvline}{\hfil\kern\arraycolsep\vline\kern-\arraycolsep\hfilneg}
\newcommand{\esssup}{\text{ess}\, \text{sup}}
\newcommand{\normdot}{\norm{\cdot}}
\newcommand{\scalardot}{
  \langle \cdot,\cdot \rangle}

\newcommand{\interior}[1]{%
  {\kern0pt#1}^{\mathrm{o}}%
}
\def\stackbelow#1#2{\underset{\displaystyle\overset{\displaystyle\shortparallel}{#2}}{#1}}
\def\stackbelowlittle#1#2{\underset{\textstyle\overset{\textstyle\shortparallel}{#2}}{#1}}



\long\def\symbolfootnotemark[#1]#2{\begingroup%
\def\thefootnote{\fnsymbol{footnote}}\footnotetext[#1]{#2}\footnotemark[#1]\endgroup}

\long\def\symbolfootnotetext[#1]#2{\begingroup%
\def\thefootnote{\fnsymbol{footnote}}\footnotetext[#1]{#2}\endgroup}


\numberwithin{equation}{section}





\begin{document}
\title{Notes from Real and Functional Analysis}
\author{Ilaria Bonfanti \& Andrea Bonifacio}
\date{\today}
\maketitle
\newpage

\section{Lecture 12/09/2022}

\subsection*{Element of set theory}
Let \(X\) be a set. Then 
\[
    \mathcal{P}(X) = \left\lbrace Y \; | \; Y \subseteq X \right\rbrace \tag{\textbf{Power Set}}
\]
Let \(I \subseteq \mathbb{R}\) be a set of indexes. A family of sets induced by \(I\) is 
\[
    \left\lbrace E_i \right\rbrace_{i \in I}, \quad E_i \subseteq X  \tag{\textbf{Family/Collection}}
\]
If \(I = \mathbb{N} \) is called a 
\[
    \left\lbrace E_n \right\rbrace_{n \in \mathbb{N}} \tag{\textbf{Sequence}}
\]
\begin{definition}
    \( \left\lbrace E_n \right\rbrace \subseteq \mathcal{P}(X) \) is monotone increasing (resp. decreasing) if 
    \[
        E_n \subseteq E_{n+1} \,\forall n \qquad (\mbox{resp. } E_n \supseteq E_{n+1} \, \forall n)
    \]
    and is written as 
    \[
        \left\lbrace E_n \right\rbrace \nearrow \qquad (\mbox{resp. }\left\lbrace E_n \right\rbrace \searrow)
    \]
\end{definition}
Given a family of sets \(\left\lbrace E_i \right\rbrace_{i \in I} \subseteq \mathcal{P}(X)\), will be often considered
\[
    \bigcup_{i \in I} E_i = \left\lbrace x \in X : \exists \; i \in I \, s.t. \, x \in E_i \right\rbrace 
\]
\[
    \bigcap_{i \in I} E_i = \left\lbrace x \in X : x \in E_i, \, \forall i \in I \right\rbrace
\]
\(\left\lbrace E_i \right\rbrace\) is said to be \textbf{disjoint} if \(E_i \cap E_j = \emptyset \; \forall i \not = j\).

Examples:
\[
    [a,b] = \bigcap_{n = 1}^{\infty} (a - \frac{1}{n}, b + \frac{1}{n}) 
\]
\[
    (a,b) = \bigcup_{n = 1}^{\infty}[a + \frac{1}{n}, b - \frac{1}{n}]
\]
\begin{definition}
    \(\left\lbrace E_n \right\rbrace \subseteq \mathcal{P}(X)\). We define 
    \[
        \limsup_{n} E_n := \bigcap_{k = 1}^{\infty} \bigl(\bigcup_{n = k}^{\infty} E_n\bigr)
    \]
    \[
        \liminf_{n} E_n := \bigcup_{k = 1}^{\infty} \bigl(\bigcap_{n = k}^{\infty} E_n\bigr)
    \]
    If these two sets are equal, then 
    \[
        \lim_n E_n = \limsup_n E_n = \liminf_n E_n
    \]
\end{definition}
\begin{proposition}
    Some limits are:
    \begin{itemize}
        \item \(\limsup_n E_n = \left\lbrace x \in X :\, x \in E_n \; \mbox{for }\infty-\mbox{many indexes }n \right\rbrace\)
        \item \(\liminf_n E_n = \left\lbrace x \in X :\, x \in E_n \; \mbox{for all but finitely many indexes }n \right\rbrace\)
        \item \(\liminf_n E_n \subseteq \limsup_n E_n\)
        \item \(\left( \liminf_n E_n\right)^C = \limsup_n E_n^C\) 
    \end{itemize}
\end{proposition}
\begin{proof}
    We can define:
    \[
    \begin{array}{ccc}
        x \in \limsup_n E_n & \Leftrightarrow & x \in \bigcap_{k = 1}^{\infty} \left(\bigcup_{n = k}^{\infty} E_n\right) \\
        & \Leftrightarrow & \forall k \in \mathbb{N} \, : \; \bigcup_{n = k}^{\infty} E_n \\
        & \Leftrightarrow &  \forall k \in \mathbb{N} \; \exists n_k \geq k \, s.t. \, x \in E_{n_k}
        
    \end{array}
\]
So \(x \in \limsup_n E_n \; \Rightarrow\) \(\begin{array}[t]{l}
    \exists m_1 = n_1 \, s.t. \, x \in E_{n_1} \\
    \exists m_2 := n_{m_1 + 1} \geq m_1 + 1 \, s.t. \, x \in E_{n_2} \\
    \vdots \\
    \exists m_k := n_{m_{k-1} + 1} \geq m_{k-1} + 1 \, s.t. \, x \in E_{n_k} \\
    \vdots \\
    x \in E_{m_1}, \ldots, E_{m_k}, \ldots 
\end{array}
\)

On the other hand, assume that \(x \in E_n\) for \(\infty\)-many indexes.
We claim that \(\forall k \in \mathbb{N} \, \exists n_k \geq k \, s.t. \, x \in E_{n_k} \, \Leftrightarrow \, x \in \limsup_n E_n\). 
If that claim is not true, then \(\exists \bar{k} \, s.t. \, x \not \in E_n \, \forall n > \bar{k} \Rightarrow x\) belongs at most to \(E_1, \ldots, E_{\bar{k}}\), a contradiction. 
\end{proof}

\begin{definition}
    \(\left\lbrace E_i \right\rbrace_{i \in I}\) is a \textbf{covering} of \(X\) if 
    \[
        X \subseteq \bigcup_{i \in I} E_i
    \]
A subfamily of \(E_i\) that is still a covering is called a \textbf{subcovering}
\end{definition}
\begin{definition}
    Let \(E \subseteq X\). The function \(\chi_E \, : X \rightarrow \mathbb{R}\) 
    \[
        \chi_E (x):= \begin{cases}
            1 & \mbox{if } x \in E \\
            0 & \mbox{if } x \in X\setminus E
        \end{cases}
    \]
    is called \textbf{characteristic function} of \(E\)
\end{definition}
Let \(E_1, E_2\) be sets:
\[
    \chi_{E_1 \cap E_2} = \chi_{E_1} \cdot \chi_{E_2}
\]
\[
    \chi_{E_1 \cup E_2} = \chi_{E_1} + \chi_{E_2} - \chi_{E_1 \cap E_2} 
\]
\[
    \left\lbrace E_n \right\rbrace \subseteq \mathcal{P}(X), \mbox{ disjoint}, E = \bigcup_{n = 1}^{\infty} E_n \Rightarrow \mathcal{X_E} = \sum_{n = 1}^{\infty} \chi_{E_n}
\]
\[
    \left\lbrace E_n \right\rbrace \subseteq \mathcal{P}, P = \liminf_n E_n, Q = \limsup_n E_n \Rightarrow \chi_P = \liminf \chi_{E_n}, \chi_Q = \limsup_n \chi_{E_n}
\]
Recall that \(\limsup_n a_n = \lim_{k \to \infty} \sup_{n \geq k} a_n\) and \(\liminf_n a_n = \lim_{k \to \infty} \inf_{n \geq k} a_n\)


Let's also check that \(\chi_Q = \limsup_n \chi_{E_n}\)
\[
    \begin{array}{ccl}   
    x \in \limsup_n E_n & \Leftrightarrow & \chi_Q(x) = 1 \\
    & \Leftrightarrow & \forall \, k \in \mathbb{N} \, \exists \, n_k \geq k \; s.t. \; x \in E_{n_k}
    \end{array}
    \]
If we fix \(k\) then 
\[
    \begin{array}{c}
        \sup_{n \geq k} \chi_{E_n}(x) = \chi_{E_{n_k}}(x) = 1 \\
        \lim_{k \to \infty} \sup_{n \geq k} \chi_{E_n}(x) = \limsup_n \chi_{E_n}(x) = 1
    \end{array}
\]
    Let now \(x \not \in \limsup E_n \Leftrightarrow \chi_Q(x) = 0\).
    Then \(x\) belongs at most to finitely many \(E_n\) \(\Rightarrow \exists \, \bar{k}\; s.t. \; x \not \in E_n, \forall n \geq \bar{k}\)
    
    If \(k \geq \bar{k}\), then \(\sup_{n \geq k} \chi_{E_n} (x) = 0 \Rightarrow \lim_{k \to \infty} \sup_{n \geq k} \chi_{E_n}(x) = \limsup_n \chi_{E_n} (x) = 0\)

    \subsubsection*{Relations}
    Given \(X, Y\) sets, is called a \textbf{relation} of \(X\) and \(Y\) a subset of \(X \times Y\)
    \[
        R \subseteq X + Y \quad R = \left\lbrace (x,y) \, : \, x \in X, y \in Y \right\rbrace
    \]
    \[
        (x,y) \in R \Leftrightarrow xRy
    \]
    \[
        X = \left\lbrace 0,1,2,3 \right\rbrace \quad R = \left\lbrace (0,1), (1,2), (2,1) \right\rbrace \mbox{ is a relation in } X
    \]
\begin{definition}
    A \textbf{function} from \(X\) to \(Y\) is a relation \(R\) s.t. for any element \(x\) of \(X\) \(\exists !\) element \(y\) of \(Y\) s.t. \(xRy\)
\end{definition}
\begin{definition}
    \(R\) on \(X\) is an \textbf{equivalence relation} if 
    \begin{enumerate}
        \item \(xRx\) \(\forall \; x \in X\) (\(R\) is \textbf{reflexive})
        \item \(xRy \Rightarrow yRx\) (\(R\) is \textbf{symmetric})
        \item \(xRy, yRz \Rightarrow xRz\) (\(R\) is \textbf{transitive})
    \end{enumerate}
    If \(R\) is an equivalence relation, the set 
    \(
        E_X := \left\lbrace y \in X \, : \, yRx \right\rbrace, \; x \in X
    \)
    is called the \textbf{equivalence class} of \(X\)
\end{definition}
\begin{definition}
    \(\frac{X}{R} := \left\lbrace E_X \, : \, x \in X \right\rbrace\) is the \textbf{quotient set}
\end{definition}
Ex: \(X = \mathbb{Z}\), let's say that \(nRm\) if \(n-m\) is even. This is an equivalence relation.
\[
    E_n = \left\lbrace \ldots, n-4, n-2, n, n+2, n+4, \ldots \right\rbrace
\]
in this case if \(n\) is even, \(E_n = \left\lbrace \mbox{even numbers} \right\rbrace\) and if \(n\) is odd, \(E_n = \left\lbrace \mbox{odd numbers} \right\rbrace\)
\subsection*{Measure theory}
\begin{definition}
    A family \(\mathcal{M} \subseteq \mathcal{P}(X)\) is called a \textbf{\(\mathbf{\sigma}\)-algebra} if 
    \begin{enumerate}
        \item \(X \in \mathcal{M}\)
        \item \(E \in \mathcal{M} \Rightarrow E^C = X\setminus E \in \mathcal{M}\)
        \item If \(E = \bigcup_{n \in \mathbb{N}}E_n\) and \(E_n \in \mathcal{M} \; \forall \, n\), then \(E \in \mathcal{M}\)
    \end{enumerate}
\end{definition}
If \(\mathcal{M}\) is a \(\sigalg\), \((X, \mathcal{M})\) is called \textbf{measurable space} and the sets in \(\mathcal{M}\) are called \textbf{measurable}.  
Ex: \begin{itemize}
    \item \((X, \mathcal{P}(X))\) is a measurable space
    \item Let \(X\) be a set, then \(\left\lbrace \emptyset, X \right\rbrace\) is a \(\sigalg\)
\end{itemize}
\begin{remark}
    \(\sigma\) is often used to denote the closure w.r.t. countably many operators. If we replace the countable unions with finite unions in the definition of \(\sigalg\), we obtain an \textbf{algebra}.
\end{remark}
Some \textbf{basic properties} of a measurable space \((X, \mathcal{M})\):
\begin{enumerate}
    \item \(\emptyset \in \mathcal{M}\): \(\emptyset = X^C\) and \(X \in \mathcal{M}\)
    \item \(\mathcal{M}\) is an algebra, and \(E_1, \ldots, E_n \in \mathcal{M}\)
    \[
        E_1 \cup \ldots \cup E_n = E_1 \cup \ldots \cup E_n \cup \underbrace{\emptyset}_{\in \mathcal{M}} \cup \emptyset \ldots \in \mathcal{M} 
    \]
    \item \(E_n \in \mathcal{M}\), \(\bigcap_{n \in \mathbb{N}} E_n \in \mathcal{M}\)
    \[
        \bigcap_{n \in \mathbb{N}} E_n = \biggl(\underbrace{\bigcup_{n \in \mathbb{N}} \underbrace{E_n^C}_{\in \mathcal{M}}}_{\in \mathcal{M}}\biggr)^C \qquad (\mathcal{M} \mbox{ is also closed under finite intersection})
    \]
    \end{enumerate}
\begin{itemize}
    \item \(E, F \in \mathcal{M} \Rightarrow E \setminus F \in \mathcal{M} = E \setminus F = E \cap F^C \in \mathcal{M}\)
    \item If \(\Omega \subset X\), then the \textbf{restriction} of \(\mathcal{M}\) to \(\Omega\), written as \[\mathcal{M}\vert_{\Omega} := \left\lbrace F \subseteq \Omega: F = E \cap \Omega, \mbox{ with } E \in \mathcal{M} \right\rbrace\] is a \(\sigalg\) on \(\Omega\)
\end{itemize}
\begin{theorem}
    \(\mathcal{S} \subseteq \mathcal{P}(X)\). Then it is well defined the smallest \(\sigalg\) containing \(\mathcal{S}\), the \(\sigalg\) generated by \(\mathcal{S} := \sigma_0(\mathcal{S})\):
    \begin{itemize}
        \item \(\mathcal{S} \subseteq \sigma_0(\mathcal{S})\) and thus is a \(\sigalg\)
        \item \(\forall \sigma(\mathcal{M})\) s.t. \(\mathcal{M} \supseteq \mathcal{S}\), we have \(\mathcal{M} \supseteq \sigma_0(\mathcal{S})\)
    \end{itemize}
\end{theorem}
\begin{proof}[Proof idea]
    \[
        \mathcal{V} = \left\lbrace \mathcal{M} \subseteq \mathcal{P}(X): \mathcal{M} \mbox{ is a } \sigalg \mbox{ and }\mathcal{S} \subseteq \mathcal{M}\right\rbrace \not = \emptyset \mbox{ since } \mathcal{P}(X) \in \mathcal{V}
    \]
    We define \(\sigma_0(\mathcal{S}) = \bigcap \left\lbrace \mathcal{M} \; : \; \mathcal{M} \in \mathcal{V}\right\rbrace\), so it can be proved that this is the desired \(\sigalg\)
\end{proof}
\subsubsection*{Borel sets}
Given \((X, d)\) metric space, the \(\sigalg\) generated by the open sets is called \textbf{Borel} \(\sigalg\), written as \(\mathcal{B}(X)\). The sets in \(\mathcal{B}(X)\) are called \textbf{Borel sets}. The following sets are Borel sets:
\begin{itemize}
    \item open sets
    \item closed sets
    \item countable intersections of open sets: \(G_{\sigma}\) sets
    \item countable unions of closed sets: \(F_{\sigma}\) sets
\end{itemize}

\begin{remark}
    \(\mathcal{B}(\mathbb{R})\) can be equivalently defined as the \(\sigalg\) generated by 
    \[
        \left\lbrace (a,b): \; a,b \in \real, a < b \right\rbrace
    \]
    \[
        \left\lbrace (-\infty,b): \; b \in \real \right\rbrace
    \]
    \[
        \left\lbrace (a,+\infty ): \; a \in \real \right\rbrace
    \]
    \[
        \left\lbrace [a,b): \; a,b \in \real, a < b \right\rbrace
    \]
    \[
        \vdots
    \]
\end{remark}

\section{Lecture 14/09/2022}
\noindent\underline{Question}: What is \(\boreal\)?
Is \(\boreal \not = \mathcal{P}(\mathbb{R})\)? No.
\begin{definition}
    \((X, \mathcal{M})\) measurable space. A function \(\mu : \; \mathcal{M} \to [0, +\infty]\) is called a \textbf{positive measure} if \(\mu(\emptyset) = 0\) and if \(\mu\) is countably additive, that is 
    \[
        \forall \left\lbrace E_n \right\rbrace \subseteq \mathcal{M} \quad \mbox{disjoint}
    \]
    we have that \[
        \mu\left(\bigcup_{n=1}^{\infty} E_n\right) = \sum_{n = 1}^{\infty} \mu(E_n) \tag*{\(\sigma\)-additivity}
    \]
\end{definition}
\begin{remark}
    A set \(A\) is countable if \(\exists \; f \; 1-1\) s.t. \(f: A\to \mathbb{N}\)

    Examples: \(\mathbb{Z}, \mathbb{Q}\) are countable, while \(\mathbb{R}\) is not, also \((0,1)\) is uncountable.
\end{remark}
We always assume that \(\exists \; E \not = \emptyset, E \in \mathcal{M}\) s.t. \(\mu(E) \not = \infty\). 

If \((X,\mathcal{M})\) is a measurable space, and \(\mu\) is a measure on it, then \((X, \mathcal{M}, \mu)\) is a measure space.

Then:
\begin{enumerate}
    \item \(\mu\) is \textbf{finitely additive}: 
    \[
        \forall \; E,F \in \mathcal{M}, \mbox{ with } E \cap F \not = \emptyset \Rightarrow
    \mu(E \cup F) = \mu(E) + \mu(F)
    \]
    \item the \textbf{excision property}
    \[
        \forall \; E, F \in \mathcal{M}, \mbox{ with } E \subset F \mbox{ and } \mu(E) < +\infty \Rightarrow \mu(F\setminus E) = \mu(F) - \mu(E)
    \]
    \item \textbf{monotonicity}
    \[
        \forall \; E, F \in \mathcal{M}, \mbox{ with } E \subset F \Rightarrow \mu(E) \leq \mu(F)
    \]
    \item if \(\Omega \in \mathcal{M}\) then \((\Omega, \mathcal{M}\vert_{\Omega}, \mu\vert_{\mathcal{M}\vert_{\Omega}})\) is a measure space
\end{enumerate}
\begin{proof}
     \begin{enumerate}
        \item \(E_1 = E, E_2 = F, E_3 = \ldots = E_n = \ldots = \emptyset\) 
        This is a disjoint sequence \(\Rightarrow\) by \(\sigma\)-additivity. 
        \[
            \mu(E \cup F) = \mu\left(\bigcup_{n} E_n\right) = \sum_n \mu(E_n) = \mu(E) + \mu(F) + \underbrace{\mu(E_k)}_{= \mu(\emptyset)}
        \] 
        \item \(E \subset F\), so \(F = E \cup (F \setminus E)\) and this is disjoint \(\overset{(i)}{\Rightarrow} \mu(F) = \mu(E) + \mu(F\setminus E)\), and since \(\mu(E) < \infty\), the property follows.
        \item \(E \subset F \Rightarrow \mu(F) = \mu(E) + \underbrace{\mu(F\setminus E)}_{\geq 0} \geq \mu(E)\)
        \item 
     \end{enumerate}
    \end{proof}
\begin{definition}
    \((X, \mathcal{M}, \mu)\) measure space. 
    \begin{itemize}
    \item If \(\mu(X) < +\infty\), we say that \(\mu\) is \textbf{finite}.

    \item If \(\mu (X) = +\infty\), and \(\exists \; \left\lbrace E_n \right\rbrace \subset \mathcal{M}\) s.t. \(X = \bigcup_n E_n\) and each \(E_n\) has finite measure, then we say that \(\mu\) is \(\sigma\)-finite. 

    \item If \(\mu(X) = 1\) we say that \(\mu\) is a \textbf{probability measure}.
    \end{itemize}
\end{definition}
    Some examples:
    \begin{itemize}
        \item Trivial Measure: \((X, \mathcal{M})\) measurable space. \(\mu : \mathcal{M} \to [0, \infty]\) defined by \(\mu(E) = 0 \quad \forall \; E \in \mathcal{M}\) 
        \item Counting Measure: \((X, \mathcal{P}(X))\) measurable space. We define 
        \[
            \mu_C :  \mathcal{P}(X) \to [0, \infty], \quad \mu_C (E) = \begin{cases}
                n & \mbox{if } E \mbox{ has } n \mbox{ elements} \\
                \infty &  \mbox{if } E \mbox{ has } \infty \mbox{-many elements} 
            \end{cases}
        \]
        \item Dirac Measure: \((X, \mathcal{P}(X))\) measurable space, \(t \in X\). We define 
        \[
            \delta_t  :  \mathcal{P}(X) \to [0, +\infty], \quad \delta_t(E) = \begin{cases}
                1 & \mbox{if } t \in E\\
                0 & \mbox{otherwise}
            \end{cases}
        \]
    \end{itemize}
\noindent\underline{Continuity of the measure along monotone sequences}

\((X, \mathcal{M}, \mu)\) measure space
\begin{enumerate}
    \item \(\left\lbrace E_i \right\rbrace \subset \mathcal{M}, \; E_i \subseteq E_{i+1} \; \forall i\) and let \[
        E = \bigcup_{i = 1}^{\infty} E_i = \lim_i E_i
    \]
    Then:
    \[
        \mu(E) = \lim_i \mu(E_i)
    \]
    \item \(\left\lbrace E_i \right\rbrace \subset \mathcal{M}, \; E_{i+1} \subseteq E_{i} \; \forall i\) and let \(E = \bigcap_{i = 1}^{\infty} E_i = \lim_i E_i\).
\end{enumerate}
\begin{proof}
    \begin{enumerate}
        \item if \(\exists \; i\) s.t. \(\mu(E_i) = +\infty\), then is trivial. Assume then that every \(E_i\) has a finite measure, so that \(E = \bigcup_{i=1}^{\infty} E_i = \bigcup_{i=0}^{\infty}(E_{i+1}\setminus E_i)\) with \(E_0 = \emptyset\).
        
        So, by \(\sigma\)-additivity \[\mu(E) = \mu\left(\bigcup_{i=0}^{\infty}(E_{i+1}\setminus E_i)\right) = \]
        \[
            = \sum_{i = 0}^{\infty} \mu(E_{i + 1} \setminus E_i) \overset{(excision)}{=} \sum_{i=0}^{\infty}\left(\mu(E_{i+1} - \mu(E_i))\right) = 
        \]
        \[
            \overset{(telescopic \; series)}{=} \lim_n \mu(E_n) - \underbrace{\mu(E_0)}_{= 0} = \lim_n \mu(E_n)
        \]
        \item For simplicity, suppose \(\tau = 1\), and define \(F_k = E_i\setminus E_k\) 
        \[
            \left\lbrace E_k \right\rbrace \searrow \Rightarrow \left\lbrace F_k \right\rbrace \nearrow
        \]
        \[
            \mu(E_i) = \mu(E_k) + \mu(F_k) \mbox{ and } \bigcup_k F_k = E_i \setminus (\bigcap_k E_k)
        \]
        \[
            \mu(E_i) = \mu(\bigcup_k F_k) + \underbrace{\mu(\bigcap_k E_k)}_{\mu(E)} =
        \]
        \[
            \overset{(i)}{=} \lim_k \mu(F_k) + \mu(E) = \lim_k \left(\mu(E_i) - \mu(E_k)\right) + \mu(E)
        \]
        Since \(\mu(E_i) < \infty\) we can subtract it from both sides
        \[
            0 = -\lim_k \mu(E_k) + \mu(E)
        \]
    \end{enumerate}
\end{proof}
Counterexample: given \((\mathbb{N}, \mathcal{P}(\mathbb{N}), \mu_C)\) measure space. Let \(E_n = \left\lbrace n, n+1, n+2, \ldots\right\rbrace\). In this case \(\mu_C (E_n) = +\infty, E_{n+1} \subseteq E_n \forall \; n\), but \(\bigcap_n E_n = \emptyset \Rightarrow \mu\left(\bigcap_n E_n\right) = 0 \)
\begin{theorem}[\(\sigma\)-subadditivity of the measure]
\((X, \mathcal{M}, \mu)\) is a measure space. \(\forall \left\lbrace E_n \right\rbrace \subseteq \mathcal{M}\) (not necessarily disjoint): \(\mu\left(\bigcup_n E_n\right) \leq \sum_n \mu(E_n)\)
\end{theorem}
\begin{proof}
    \(E_1, E_2 \in \mathcal{M}\) and also \(E_1 \cup E_2 = E_1 \cup (E_2 \setminus E_1)\) disjoint sets.
    \[
        \mu(E_1 \cup E_2) = \mu(E_1) + \mu(\underbrace{E_2 \setminus E_1}_{\subseteq E_2}) \overset{(monotonicity)}{\leq} \mu(E_1) + \mu(E_2)
    \]
    that means that we have the subadditivity for finitely many sets.

    \noindent Given \(
        A = \bigcup_{n=1}^{\infty} E_n, A_k = \bigcup_{n = 1}^{k} E_n,
        \left\lbrace A_k \right\rbrace \nearrow, \; A_{k+1} \supseteq A_k, \; \lim_k A_k = A:
    \)
    \[
        \mu\left(\bigcup_{n = 1}^{\infty} E_n\right) \overset{(continuity)}{=} \lim_k \mu(A_k) = \lim_k \mu \left(\bigcup_{n=1}^{k} E_n\right) \leq \lim_k \sum_{n=1}^k \mu(E_n) = \sum_{n = 1}^{\infty} \mu(E_n)
\]
\end{proof}
Exercise: \((X, \mathcal{M})\) measurable space. \(\mu : \mathcal{M} \to [0, +\infty]\) s.t. \(\mu\) is finitely additive, \(\sigma\)-subadditive and \(\mu(\emptyset) = 0\) \(\Rightarrow\) \(\mu\) is \(\sigma\)-additive, and hence is a measure.  

Exercise: the Borel-Cantelli lemma states that, given \((X, \mathcal{M}, \mu)\) and \(\left\lbrace E_n \right\rbrace \subseteq \mathcal{M}\). Then
\[
    \sum_{n=0}^{\infty} \mu(E_n) < \infty \Rightarrow \mu(\limsup_n E_n) = 0
\]
It can be phrased as: \begin{quote}
    If the series of the probability of the events \(E_n\) is convergent, then the probability that \(\infty\)-many events occur is \(0\)
\end{quote}
\begin{proof}
    The thesis is: \[\mu(\limsup_n E_n) = \mu\biggl(\bigcap_{n=1}^{\infty} \underbrace{\bigcup_{k \geq n} E_k}_{A_n := \bigcup_{k\geq n}E_k}\biggr)\]
    Is it true that \(\left\lbrace A_n \right\rbrace \searrow\)? Yes.
    \[
         A_{n+1} = \bigcup_{k \geq n+1}  E_k \subseteq \bigcup_{k \geq n} E_k = A_n
    \]
    Does some \(A_n\) have a finite measure? 
    \[
        \mu(A_n) = \mu\left(\bigcup_{k \geq n} E_k\right) \leq \sum_{k \geq n} \mu(E_k) < \infty
    \]
    by assumption. Therefore, we can use the continuity along decreasing sequences: 
    \[
        \mu(\limsup_n E_n) = \lim_n \mu(A_n) = \lim_n \mu \left(\bigcup_{k \geq n} E_k\right) \overset{\sigma-sub.}{\leq} \lim_n \sum_{k=n}^{\infty} \mu(E_k) = 0
    \]
\end{proof}
\subsubsection*{Sets of \(0\) measure}
\((X, \mathcal{M}, \mu)\) measure space.
\begin{itemize}
    \item \(N \subseteq X\) is a set of \(0\) measure if \(N \in \mathcal{M}\) and \(\mu(N) = 0\)
    \item \(E \subseteq X\) is called \textbf{negligible set} if \(\exists \; N \in \mathcal{M}\) with \(0\) measure s.t. \(E \subseteq N\) (\(E\) does not necessarily stays in \(\mathcal{M}\))
\end{itemize} 
\begin{definition}
    \((X, \mathcal{M}, \mu)\) measure space s.t. every negligible set is measurable (and hence of \(0\) measure), then \(\measurespace\) is said to be a \textbf{complete measure space}.

    A measure space may not be complete. However, let 
    \[
        \overline{\mathcal{M}} := \left\lbrace E \subseteq X : \exists\; F, G \in \mathcal{M} \mbox{ with } F\subseteq E \subseteq G \mbox{ and } \mu(G\setminus F) = 0\right\rbrace
    \]
    Clearly \(\mathcal{M} \subseteq \overline{\mathcal{M}}\). For \(E \in \overline{\mathcal{M}}\), take \(F\) and \(G\) as above and let \(\bar{\mu}(E) = \bar{\mu}(F)\) then \(\bar{\mu}\vert_{\mathcal{M}} = \mu\), and moreover:
\end{definition}
\begin{theorem}
    \(\measurespace\) is a complete measure space. Let's observe that \(\bar{\mu}\) is well defined: let \(E \subseteq X\) and \(F_1,F_2, G_1, G_2\) s.t. \(F_i \subset E \subset G_i \quad i = 1,2\). Then \(\mu(G_i\setminus F_i) = 0\). Now we have to check that \(\mu(F_1) = \mu(F_2)\). 

    Since \[
        F_1 \setminus F_2 \subseteq E\setminus F_2 \subseteq G_2 \setminus F_2
    \] 
    and \(G_2 \setminus F_2\) has \(0\) measure \(\Rightarrow \mu(F_1 \setminus F_2) = 0\). Then \(F_1 = (F_1 \setminus F_2) \cup (F_1 \cap F_2) \Rightarrow \mu(F_1) = \mu(F_1 \cap F_2).\) In the same way, \(\mu(F_2) = \mu(F_1 \cap F_2)\)
\end{theorem}

\section{Lecture 15/09/2022}
The elements of \(\overline{\mathcal{M}}\) are sets of the type \(E \cup N\), with \(E \in \mathcal{M}\) and \(\bar{\mu}(N) = 0\).
\subsubsection*{Outer measure}
We wish to define a measure \(\lambda\) ``on \(\real\)'' with the following properties:
\begin{enumerate}
    \item \(\lambda((a,b)) = b-a\)
    \item \(\lambda(E + t)\symbolfootnotemark[2]{\(\left\lbrace x \in \mathbb{R} : x=y+t, \mbox{ with } y \in E\right\rbrace\)}  \; = \lambda(E)\) for every measurable set \(E \subset \mathbb{R}\) and \(t \in \mathbb{R}\)
\end{enumerate}
It would be nice to define such a measure on \(\mathcal{P}(\mathbb{R})\). In such case, note that \(\lambda(\left\lbrace x \right\rbrace) = 0\), \(\forall \; x \in \mathbb{R}\)
But then 
\begin{theorem}[Ulam]
    The only measure on \(\mathcal{P}(\mathbb{R})\) s.t. \(\lambda(\left\lbrace x \right\rbrace) = 0 \quad \forall \; x\) is the trivial measure. Thus, a measure satisfying the two properties of the outer measure cannot be defined on \(\mathcal{P}(\real)\)
\end{theorem}
We'll learn in what follows how to create a measure space on \(\real\), with a \(\sigalg\) including all the Borel sets, and a measure satisfying properties of the outer measure. This is the so called \textbf{Lebesgue measure}.
\begin{definition}
    Given a set \(X\). An \textbf{outer measure} is a function \(\mu^* : \mathcal{P}(\mathbb{R}) \to [0, +\infty]\) s.t. 
    \begin{itemize}
        \item \(\mu^*(\emptyset) = 0\)
        \item \(\mu^*(A) \leq \mu^*(B)\) if \(A \subseteq B\) (Monotonicity)
        \item \(\mu^*(\bigcup_{n=1}^{\infty} E_n) \leq \sum_{n=1}^{\infty} \mu^*(E_n)\) (\(\sigma\)-subadditivity)
    \end{itemize}
\end{definition}
The common way to define an outer measure is to start with a family of elementary sets \(\mathcal{E}\) on which a notion of measure is defined (e.g. intervals on \(\real\), rectangles on \(\real^2, \ldots\)) and then to approximate arbitrary sets from outside by \textbf{countable} unions of members of \(\mathcal{E}\).
\begin{proposition}
    Let \(\mathcal{E} \subset \mathcal{P}(\mathbb{R})\) and \(\rho : \mathcal{E} \to [0, +\infty]\) be such that \(\emptyset \in \mathcal{E}, X \in \mathcal{E}\) and \(\rho(\emptyset) = 0\). For any \(A \in \mathcal{P}(X)\), let 
    \[\mu^*(A) := \inf \left\lbrace \sum_{n=1}^{\infty} \rho (E_n) : E_n \in \mathcal{E} \mbox{ and } A \subset \bigcup_{n=1}^{\infty} E_n \right\rbrace\]
    Then \(\mu^*\) is an outer measure, the outer measure generated by \((\mathcal{E}, \rho)\).
\end{proposition}
\begin{proof}
    \(\forall \; A \subset X \; \exists \left\lbrace E_n \right\rbrace \subset \mathcal{E}\) s.t. \(A \subset \bigcup_n E_n : \mbox{ take } E_n = X \forall \; n\)
    then \(\mu^*\) is well defined. Obviously, \(\mu^*(\emptyset) = 0\) (with \(E_n = \emptyset \quad \forall\; n\)), and \(\mu^*(A) \leq \mu^*(B)\) for \(A \subset B\) (any covering of \(B\) with elements of \(\mathcal{E}\) is also a covering of \(A\).)

    We have to prove the \(\sigma\)-subadditivity. Let \(\left\lbrace A_n \right\rbrace_{n \in \mathbb{N}} \subseteq \mathcal{P}(X)\) and \(\epsilon > 0\). For each \(n, \exists \left\lbrace E_{n_j} \right\rbrace_{j \in \mathbb{N}} \in \mathcal{E}\) s.t. \(A_n \subset \bigcup_{i = 1}^{\infty} E_{n_j}\) and \(\sum_{j=1}^{\infty} \rho(E_{n_j}) \leq \mu^*(A_n) + \frac{\epsilon}{2^n}\).  
    But then, if \(A = \bigcup_{n=1}^{\infty} A_n\), we have that \(A \subset \bigcup_{n,j \in \mathbb{N}^2} E_{n_j}\) and
    \[
        \mu^*(A) \leq \sum_{n,j} \rho(E_{n_j}) \leq \sum_{n} \left(\mu^*(A_n) + \frac{\epsilon}{2^n}\right) = \sum_{n} \mu^*(A_n) + \epsilon
    \]
    Since \(\epsilon\) is arbitrary, we are done.
\end{proof}
Ex:  
\begin{enumerate}
    \item \(X \in \mathbb{R}, \mathcal{E} = \left\lbrace (a,b) : a \leq b, a,b \in \mathbb{R} \right\rbrace \mbox{ family of open intervals:} \)
    \[
        \rho((a,b)) = b-a
    \]
    
    \item \(X = \mathbb{R}^n, \mathcal{E} = \left\lbrace (a_1, b_1) \times \ldots \times (a_n, b_n) : a_i \leq b_i, a_i, b_i \in \mathbb{R} \right\rbrace\): 
    \[
        \rho((a_1, b_1)\times \ldots \times (a_n, b_n)) = (b_1 -a_1) \cdot \ldots \cdot (b_n - a_n)
    \]
\end{enumerate}
\begin{remark}
    \(E \in \mathcal{E} \Rightarrow \mu^*(E) = \rho(E)\).  

    In examples 1 and 2, we have in fact 
    \[
        \mu^*((a,b)) = b-a, \mu^*\left((a_1, b_1) \times \ldots \times (a_n, b_n)\right) = \prod_{i=1}^{n} (b_i - a_i)
    \] 
\end{remark}

To pass from the outer measure to a measure there is a condition: 
\begin{definition}[Caratheodory condition]
    If \(\mu^*\) is an outer measure on \(X\), a set \(A \subset X\) is called \(\mu^*\)-\textbf{measurable} if 
    \[
        \mu^*(E) = \mu^*(E \cap A) + \mu^*(E \cap A^C) \quad \forall \; E \subset X
    \]
\end{definition}
\begin{remark}
    If \(E\) is a ``nice'' set containing \(A\), then the above equality says that the outer measure of \(A\), \(\mu^*(E \cap A)\), is equal to \(\mu^*(E) - \mu^*(E \cap A^C)\), which can be thought as an ``inner measure''. So basically we are saying that \(A\) is measurable if the outer and inner measure coincide. (Like the definition of Riemann integration with lower and upper sum)
\end{remark}
\begin{remark}
    \(\mu^*\) is subadditive by def \(\Rightarrow \mu^*(E) \leq \mu^*(E \cap A) + \mu^*(E \cap A^C) \quad \forall \; E, A \subset X\).  
    So, to prove that a set is \(\mu^*\)-measurable it is enough to prove the reverse inequality, \(\forall \; E \subset X\). In fact, if \(\mu^*(E) = +\infty\), then \(+\infty \geq \mu^*(E \cap A) + \mu^*(E \cap A^C)\), and hence \(A\) is \(\mu^*\)-measurable iff 
    \[
        \mu^*(E) \geq \mu(E \cap A) + \mu^*(E \cap A^C) \quad \forall \; E \subset X \mbox{ with } \mu^*(E) < +\infty
    \] 
\end{remark}
Their relevance to the notion of \(\mu^*\)-measurability is clarified by the following
\begin{theorem}[Caratheodory]
    If \(\mu^*\) is an outer measure on \(X\), the family
    \[
        \mathcal{M} = \left\lbrace A \subseteq X : A \mbox{ is }\mu^*\mbox{-measurable}\right\rbrace
    \]
    is a \(\sigalg\) and \(\mu^*\vert_{\mathcal{M}}\) is a complete measure.
\end{theorem}
\begin{lemma}
    If \(A \subset X\) and \(\mu^*(A) = 0\), then \(A\) is \(\mu^*\)-measurable.
\end{lemma}
\begin{proof}
    Let \(E \subset X\) with \(\mu^*(E) < +\infty\). Then 
    \[
        \mu^*(E) \geq \mu^*(E) + \mu^*(A) \overset{\symbolfootnotemark[3]{\(E \cap A^C \subseteq E\) and \(E\cap A \subseteq A\) + monotonicity}}{\geq}  \mu^*(E \cap A) + \mu^*(E \cap A^C)
    \]
    \symbolfootnotetext[3]{\(E \cap A^C \subseteq E\) and \(E\cap A \subseteq A\) + monotonicity}
    This implies that A is \(\mu^*\)-measurable.
\end{proof}
To sum up: \(X \mbox{ set}, (\mathcal{E}, \rho)\) elementary and measurable sets, so \(\mu^*\) is an outer measure. Then given \(\mu^*\) and the Caratheodory condition, we have \((X, \mathcal{M}, \mu)\) that is a complete measure space.
\begin{remark}
    So far we did not prove that \(\mathcal{E} \subseteq \mathcal{M}\). We will do it in a particular case.
\end{remark}
\subsubsection*{Lebesgue measure}
\begin{itemize}
\item \(X = \mathbb{R}\), \(\mathcal{E}\) family of open intervals, \(\rho((a,b)) = b-a = \lambda((a,b))\), the complete measure space is \((\mathbb{R}, \mathcal{L}(\mathbb{R}), \lambda)\) with \(\mathcal{L}(\mathbb{R})\) the Lebesgue-measurable sets on \(\mathbb{R}\) and \(\lambda\) the Lebesgue measure on \(\mathbb{R}\).
\item \(X = \mathbb{R}^n\), \(\mathcal{E} = \left\lbrace \prod_{k = 1}^n (a_k, b_k): a_k \leq b_k \quad \forall \; k = 1,\ldots, n \right\rbrace\), \(\rho\left(\prod_{k = 1}^n (a_k, b_k)\right) = \prod_{k=1}^n (b_k - a_k)\) and this is a complete measure space \((\mathbb{R}^n, \mathcal{L}(\mathbb{R}^n), \lambda_n)\)
\end{itemize}

\section{Lecture 21/09/2022}
\subsubsection*{Lebesgue measure}
\(\mathcal{E}\) = family of open intervals (a,b), \(a,b \in \mathbb{R}^*, a < b\). \(\rho =\) lenght \(l\).
\(\rho((a,b)) = b - a\). 

\noindent\underline{Notations:} open interval \(I\) with lenght \(l(I)\)
\subsubsection*{Outer measure}
\(E \subset \real\). The outer measure of \(E\) is 
\[
    \lambda^*(E) = \inf \left\lbrace \sum_{n=1}^{+\infty} l(I_n) \vert I_n \mbox{ is an open interval, } E \subset \bigcup_{n = 1}^{\infty} I_n \right\rbrace
\]
\subsubsection*{Caratheodory condition (CC)}
\(A \subset \real\) is \(\lambda^*\)-measurable if 
\[
    \lambda^*(E) = \lambda^*(E \cap A) + \lambda^*(E \cap A^C) \qquad \forall \; E \subset \real
\]
\[
    \left\lbrace A \subset \real : A \mbox{ is }\lambda^*\mbox{-measurable} \right\rbrace =: \mathcal{L}(\real)
\tag*{(Lebesgue \(\sigalg\))}\]
\[
    \lambda := \lambda^* \vert_{\mathcal{L}(\real)}
\tag*{(Lebesgue measure on \(\real\))}\]
Then, \((\real, \mathcal{L}(\real), \lambda)\) is a complete measure space. In particular, \(\lambda^*(A) = 0 \Rightarrow A \in \mathcal{L}(\real)\) and \(\lambda(A) = 0\).
\begin{remark}[CC-Criterion for measurability]
    To check that \(A\) is \(\lambda^*\)-measurable, it is sufficient to check that 
    \[
        \lambda^* \geq \lambda^*(E \cap A) + \lambda^*(E \cap A^C)
    \] for every \(E \subset \real\) with \(\lambda^*(E) < +\infty\)
\end{remark}
\begin{proposition}
    Any countable set is measurable, with \(0\) Lebesgue measure.
\end{proposition}
\begin{proof}
    Let \(a \in \real\), \[\left\lbrace a \right\rbrace \subseteq (a-\epsilon, a+\epsilon), \forall \; \epsilon > 0 \overset{\mbox{by def.}}{\Rightarrow} \lambda^*(\left\lbrace a \right\rbrace) \leq 2\epsilon \overset{\mbox{lim}}{\Rightarrow} \lambda^*(\left\lbrace a \right\rbrace) = 0\]
    \(\left\lbrace a \right\rbrace\) is measurable with \(\lambda(\left\lbrace a \right\rbrace) = 0, \forall \; a \in \real\). Now if a set \(A\) is countable, \(A = \left\lbrace a_n \right\rbrace_{n \in \mathbb{N}} = \bigcup_n \left\lbrace a_n \right\rbrace\) (disjoint) \(\Rightarrow \lambda(A) \underset{\sigma-add}{=} \sum_n \lambda(\left\lbrace a_n \right\rbrace) = 0\)
\end{proof}
\begin{remark}
    \(\lambda(\mathbb{Q} = 0)\). \(\mathbb{Q}\) is dense on \(\real\), \(\bar{\mathbb{Q}} = \real\). In general, measure theoretical info and topological info cannot be compared.
\end{remark}
\begin{proposition}
    \(\boreal \subseteq \mathcal{L}(\real)\)
\end{proposition}
\begin{remark}
    So far we didn't prove the fact that open intervals are \(\mathcal{L}\)-measurable.
\end{remark}
\begin{proof}
    We know that \(\boreal\) is generated by \(\left\lbrace (a, +\infty) : a \in \real \right\rbrace\). Then, we can directly show that \((a, +\infty) \in \mathcal{L}(\real) \quad \forall \; a \in \real\). Let \(a \in \real\) be fixed. We use the criterion for measurability and we check that 
    \[
        \lambda^* (E) \geq \lambda^*\underbrace{(E \cap (a, +\infty)}_{=: E_1} + \lambda^*\underbrace{(E \cap (-\infty, a])}_{=:E_2} \quad \forall\; E \subset \real, \, \lambda^* < +\infty
    \]
    Since \(\lambda^*(E) < +\infty\), \(\exists\) a countable union \(\bigcup_n I_n \supset E\), where \(I_n\) is an open interval \(\forall \; n\) and 
    \[
        \sum_n l(I_n) \leq \lambda^*(E) + \epsilon
    \]
    Let \(I^1_n := I_n \cap E_1, I^2_n := I_n \cap (-\infty, a + \frac{\epsilon}{2^n})\). These are open intervals:
    \[
        E_1 \subset \bigcup_n I^1_n \qquad E_2 \subset_n I^2_n
    \tag*{countable unions}\]
    and moreover 
    \[
        l(I_n) \geq l(I^1_n) + l(I^2_n) - \frac{\epsilon}{2^n}
    \]
    By definition of \(\lambda^*\), \(\lambda^*(E_1) \leq \sum_n l(I^1_n)\) and \(\lambda^* (E_2) \leq \sum_n l(I^2_n)\), therefore 
    \[
        \lambda^*(E_1) + \lambda^*(E_2) \leq \sum_n l(I^1_n) + \sum_n l(I^2_n) \leq \sum_n \left(l(I_n) +\frac{\epsilon}{2^n}\right) = \left(\sum_n l(I_n)\right) + \epsilon \leq \lambda^*(E) + 2\epsilon
    \]
    Since \(\epsilon\) was arbitrarily chosen, we have
    \[
        \lambda^*(E) \geq \lambda^*(E_1) + \lambda^*(E_2)
    \]  
    which is the thesis.
\end{proof}
So, the Lebesgue measure measures all the open, closed \(G_{\delta}\), \(F_{\delta}\) sets. Clearly
\[
    \lambda((a,b)) = b-a
\]
One can also show that \(\lambda\) is invariant under translation. 

\noindent\underline{Questions:} \(\boreal \subseteq \mathcal{L}(\mathbb{R}) \subseteq \mathcal{P}(\real)\), is it a strict inclusion or not?
\begin{itemize}
    \item By Ulam's theorem, if a measure is such that \(\lambda \left(\left\{ a \right\}\right) = 0, \forall \; a\) and all the sets in \(\mathcal{P}(\real)\) are measurable, then \(\lambda \equiv 0\). This and the fact that \(\lambda\left(\left(a,b\right)\right) \not = 0\) simply that \(\mathcal{L}(\real) \subsetneqq\symbolfootnotemark[3]{I had no choice} \mathcal{P}(\real) : \exists \mbox{ non-measurable sets}\) called Vitali sets. Every measurable set with positive measure contains a Vitali set. (\href{https://math.stackexchange.com/questions/137949/the-construction-of-a-vitali-set}{Explanation})
    \item \(\boreal \subsetneqq \mathcal{L}(\real)\). The construction of a \(\mathcal{L}\)-measurable se which is not a Borel set will be done during exercise classes.
\end{itemize}
The relation between \(\boreal\) and \(\mathcal{L}(\real)\) is clarified by 

\begin{theorem}[Regularity of \(\lambda\)]
    The following sentences are equivalent:
    \begin{enumerate}
        \item \(E \in \mathcal{L}(\real)\)
        \item \(\forall \; \epsilon > 0 \exists \; A \supset E\), \(A \mbox{ open}\) s.t.
        \[
            \lambda \left(A \setminus E\right) < \epsilon
        \]
        \item \(\exists \; G \supset E\), \(G \mbox{ of class } G_{\delta}\), s.t. 
        \[
            \lambda(G\setminus E) = 0
        \]
        \item \(\exists \; C \subset E\), \(C \mbox{ closed}\), s.t. 
        \[
            \lambda(E\setminus C) = 0
        \]
        \item \(\exists \; F \subset E\), \(F \mbox{ of class } F_{\delta}\), s.t. 
        \[
            \lambda(E\setminus F) = 0
        \]
    \end{enumerate}
\end{theorem}
\noindent\underline{\textbf{Consequence:}} \(E \in \mathcal{L}(\real) \Rightarrow E = F \cup N\), where \(F\) is of class \(F_{\delta}\), and \(\lambda(N) = 0\).
\begin{proof}[Partial proof]
    For simplicity, we will consider only sets with finite measure.
    \begin{itemize}
        \item[(1)]\( \Rightarrow (2)\) \(E \in \mathcal{L}(\real)\). By definition of \(\lambda^*\), \(\forall \; \epsilon > 0 \exists \; \bigcup_n I_n \supset E\) s.t. each \(I_n\) is an open interval, and 
        \[
            \lambda(E) = \lambda^*(E) \geq \sum_n l(I_n) -\epsilon
        \]
        We define \(A = \bigcup_n I_n\), which is open. Also \(A \supset E\) and 
        \[
            \lambda(A)= \lambda\left(\bigcup_n I_n\right) \overset{\sigma-\mbox{sub.}}{\leq} \sum_n l(I_n) \leq \lambda(E) + \epsilon
        \]
        Then, by excision
        \[
            \lambda(A \setminus E) = \lambda(A) - \lambda(E) \leq \epsilon
        \]
        \item[(2)]\( \Rightarrow (3)\) Define, for every \(K \in \mathbb{N}\), an open set \(A_k\) s.t. \(A_k \supset E\) and \(\lambda(A_k \setminus E) < \frac{1}{k}\). Let \(A = \bigcap_k A_k\). This is a \(G_{\delta}\) set, it contains \(E\) (since each \(A_k\) contains \(E\)) and 
        \[
            \lambda(A \setminus E) \underset{(A \subset A_k \; \forall \; k)}{\leq} \lambda(A_k \setminus E) < \frac{1}{k} \Rightarrow \lambda(A \setminus E) = 0 \quad \forall \; k
        \]
        \item[(3)]\( \Rightarrow (1)\)] If \(E \subset \real\) and \(\exists\; G \supset E\), with \(G\) of class \(G_{\delta},\) s.t. \(\lambda(G \setminus E) = 0\), then
        \[
            E = G \setminus(G \setminus E) \mbox{ is measurable}
        \]
        since \(G\) is a Borel set and \((G \setminus E)\) has \(0\) measure, then both are in \(\mathcal{L}\)
    \end{itemize}
\end{proof}
\begin{remark}
    Any countable set has \(0\) measure. he inverse is false. An example is given by the \textbf{Cantor set}.
    
    Let \(T_0 = [0,1]\). Then we define \(T_{n+1}\) stating from \(T_n\) in the following way:
    given \(T_n\), finite union of closed disjoint intervals of lenght \(l_n (\frac{1}{3})^n\), \(T_{n+1}\) is obtained by removing from each interval of \(T_n\), the open central subinterval of lenght \(\frac{l_n}{3}\).

    The Cantor set is \(T := \bigcap_{k=0}^{+\infty}\). It can be proved that \(T\) is compact, \(\lambda(T) = 0\) and \(T\) is uncountable.

    If, instead of removing intervals of size \(\frac{1}{3}, \frac{1}{9}, \ldots, \frac{1}{3^k}\), we remove sets of size \(\left(\frac{\epsilon}{3}\right)^k\), with \(\epsilon \in (0,1)\), we obtain the \textbf{generalized Cantor set} (or \textbf{fat Cantor set}) \(T_{\epsilon}\). \(T_{\epsilon}\) is uncountable, compact and has no interior points (it contains no intervals). However, \(\lambda(T_{\epsilon}) = \frac{3(1 -\epsilon)}{3 - 2\epsilon} > 0\)
\end{remark}
\begin{remark}
    We worked on \(\real\), but everything can be adapted to \(\real^n\)
\end{remark}
\subsubsection*{Measurable functions and integration}
\begin{definition}
    \(f:X \to Y\), then it is well defined the counterimage 
    \[
        f^{-1} : \mathcal{P}(Y) \to \mathcal{P}(Y)
    \]
    \[
        E \to f^{-1}(E) = \left\{ x \in X : f(x) \in E \right\}
    \]
\end{definition}
\begin{definition}
    \((X, \mathcal{M}), (Y, \mathcal{N})\) measurable spaces. \(f:X \to Y\) is called \textbf{measurable} or \((\mathcal{M}, \mathcal{N})\)-measurable if 
    \[
        f^{-1}(E) \in \mathcal{M} \mbox{ for every } E \in \mathcal{M}
    \]
    so, the counterimage of measurable sets in \(Y\) is a measurable set on \(X\).
\end{definition}

\section{Lecture 22/09/2022}
To check if a function is measurable or not, it is often used the following proposition
\begin{proposition}
    \((X, \mathcal{M}), (Y, \mathcal{N})\) measurable spaces. Let \(\mathcal{F} \subseteq \mathcal{P}(Y)\) be s.t. \(\mathcal{N} = \sigma_0(\mathcal{F})\). Then
    \[
        f: X \to Y \mbox{ is } (\mathcal{M}, \mathcal{N})-\mbox{measurable} \Leftrightarrow f^{-1}(E) \in \mathcal{M} \mbox{ for every } E \in \mathcal{F}
    \]
\end{proposition}

We will mainly focus on 2 situations:
\begin{enumerate}
    \item  \((X, \mathcal{M})\) is a measurable space obtained by means of an outer measure. \\
    Ex: \((\mathbb{R}^n, \mathcal{L}(\mathbb{R}^n))\), \((Y, d_y)\) metric space \(\to (Y, \mathcal{B}(Y))\). 

    If \(X \to Y\) is (Lebesgue) measurable \(\Leftrightarrow\) \((\mathcal{M}, \mathcal{B}(Y))\) is measurable

    \item \((X, d_X), (Y, d_Y)\) are metric spaces \(\longrightarrow (X, \mathcal{B}(X)), (Y, \mathcal{B}(Y))\) \\
    \(f: X \to Y\) is Borel measurable \(\Leftrightarrow (\mathcal{M}(X), \mathcal{B}(Y)) \)-measurable.
\end{enumerate}
\begin{remark}
     \(f\) is Lebesgue measurable if the continuity of the Borel set is a Lebesgue-measurable set.
\end{remark}
\begin{proposition}
    There are two parts:
    \begin{enumerate}
        \item \((X, d_X), (Y, d_Y)\) metric spaces. If \(f:X \to Y\) is continuous, then is Borel measurable
        \item \((Y, d_Y)\) metric space. If \(f:\mathbb{R}^n \to Y\) is continuous, then it is a Lebesgue measure.
    \end{enumerate}
\end{proposition}
\begin{proof}
    The proof is divided in:
    \begin{enumerate}
        \item \(f\) is continuous \(\Leftrightarrow f^{-1}(A)\) is open \(\forall \; A \subset Y\)
        open \(\Rightarrow\) \(f^{-1}(A) \in \mathcal{B}(Y) \; \forall \; A \subset Y\) open.
        Since \(\mathcal{B}(Y) = \sigma_0 \; (\mbox{open sets})\) by proposition \((1)\) this implies that \(f\) is Borel measurable
        \item \(f\) is continuous \(\overset{(1)}{\Rightarrow}\) \(f\) is Borel measurable.
        \(f^{-1}(A) \in \mathcal{B}(\real^n) \subseteq \mathcal{L}(\real^n) \forall \; A \in \mathcal{B}(Y)\). Namely \(f\) is Lebesgue measurable
    \end{enumerate}
\end{proof}
\begin{proposition}
    \((X, \mathcal{M})\) measurable space, \((X, d_Y), (Y, d_Y)\) metric spaces. 
    If \(f: X \to Y\) is \((\mathcal{M}, \mathcal{B}(Y))\)-measurable and \(g : Y \to Z\) is continuous \(\Rightarrow\) \(g \circ f : x \to Z\) is \((\mathcal{M}, \mathcal{B}(Y))\)-measurable
\end{proposition}
\begin{proposition}
    \((X, \mathcal{M})\) measurable space , \(u,v : X \to  \real\) measurable functions.
    Let \(\Phi : \mathbb{R}^2 \to Y\) be continuous where \((Y, d_Y)\) is a metric space. Then \(h: X\to Y\) defined by \(h(x) = \Phi(u(x), v(x))\) is \((\mathcal{M}, \mathcal{B}(Y))\)-measurable.
\end{proposition}
\noindent\underline{Consequence}: \(u, v\) measurable \(\Rightarrow\) \(u+v\) is measurable.
\begin{proof}
    Define \(f: X \to \mathbb{R}^2\), \(f(x) = u(x), v(x)\). By definition \(h = \Phi \circ f\) by proposition (3) if we show that \(f\) is \((\mathcal{M},\mathcal{B}(\real^2))\)-measurable, then \(h\) is measurable. It can be proved that \[\mathcal{B}(\mathbb{R}^2) = \sigma_0 (\lbrace \underbrace{(a_1, b_1) \times (a_2, b_2)}_{\mbox{open rectangle}}: a,b \in \mathbb{R}\rbrace)\]
    Thanks to proposition (1), to check that \(f\) is measurable. We can simply check that
    \(f^{-1}(\real) \in \mathcal{M} \quad \forall \mbox{ open rectangle in }\real^2\) and
    \(R = I \times J\), with \(I\) and \(J\) open intervals:
    \[
        \begin{array}{c}
            F^{-1}(\real) = \left\lbrace x \in X : (u(x), v(x)) \in \real \right\rbrace \\
            \Updownarrow \\
            u(x) \in I \mbox{ and } v(x) \in J \\
            = \left\{ x \in X : u(x) \in I \right\} \cap \left\{ x \in X : v(x) \in J \right\} \\
            = \underbrace{u^{-1}(I)}_{\in \mathcal{M}} \cap \underbrace{v^{-1}(J)}_{\in \mathcal{M}} \in \mathcal{M} \\
            \mbox{ since both }u,v \mbox{ are measurable}
        \end{array}
    \]
    This completes the proof
\end{proof}
\noindent\underline{Consequences}: by proposition 3 and 4, if \(u\) and \(v\) are measurable, then also \(u+v\), \(u \cdot v\). Other measurable functions include \(u^+ = \max\left\{ u, 0 \right\}, u^- =- \min\left\{u,0\right\}, \abs{u} = u^+ + u^-, u^2, \ldots\)

Recall that \(u = u^+ - u^-\)
\begin{remark}
    \(u^+\) is measurable since \(u^+ = g\circ u\), where:
    \[
        g(x) = \begin{cases}
            x & \mbox{where } x \geq 0 \\
            0 & \mbox{where } x < 0
        \end{cases}
    \]
\end{remark}
Most of the times we will work with functions \(f: X \to \real\) or \(f: X \to \underbrace{\barreal}_{\real \cup \left\{\pm \infty \right\}}\) 
\((X,\mathcal{M})\) measurable space, then such a  function \(f\) is measurable iff
\[
    f^{-1}((a, +\infty)]\symbolfootnotemark[2]{We use ) if \(f\) takes values in \(\real\) and ] if \(f\) takes values in \(\barreal\)}) \in \mathcal{M} \quad \forall a \in \real
\]
or equivalently
\[
    f^{-1}([a, +\infty)]) \in \mathcal{M} \quad \forall a \in \real
\]
Let now \(\left\lbrace f_n \right\rbrace\) be a sequence of measurable functions from \(X\) to \(\barreal\). Then we define 
\[
    (\inf_n f_n)(x) = \inf_n f_n(x)
\]
\[
    (\sup_n f_n)(x) = \sup_n f_n(x)
\]
\[
    (\liminf_n f_n)(x) = \liminf_n f_n(x)
\]
\[
    (\limsup_n f_n)(x) = \limsup_n f_n(x)
\]
\[
    (\lim_n f_n)(x) = \lim_n f_n(x) \quad \mbox{if the limit exists}
\]
\begin{proposition}
    \((X, \mathcal{M})\) measurable space, \(f_n : X \to \barreal\) measurable, then 
    \[\sup_n f_n \; \inf_n f_n \; \liminf_n f_n \; \limsup_n f_n\] are measurable, in particular if \(\lim_n f_n\) is well defined, then \(f\) is measurable
\end{proposition}
\begin{proof}
    \((\sup f_n)^{-1} ((a, \infty]) = \lbrace x \in X : \underset{\begin{array}{c}
        \Updownarrow \\
        \exists \mbox{ some indexes }n\mbox{ s.t. } f_n(x) > a
    \end{array}}{\sup f_n(x) > a} \rbrace\) 
    \[
        \bigcup_n \left\lbrace x \in X : f_n(x) > a \right\rbrace = \bigcup_n \underbrace{f_n^{-1}((a, +\infty])}_{\in \mathcal{M}}
    \]
    Then \((\sup f_n)^{-1} ((a, \infty])\) is measurable, since it is the countable union of measurable sets.

Now we check that the \(\limsup_n f_n\) is measurable
\[\limsup_n f_n(x) = \lim_n \underbrace{(\sup_{k > n} f_k(x))}_{\mbox{is decreasing on } n} = \inf_n (\sup_{k \geq n} f_k(x))\]
If we write \(g_n(x) = \sup_{k \geq n} f_k(x)\), then 
\begin{itemize}
    \item \(g_n\) is measurable, by what we proved previously
    \item \(\limsup_n f_n = \inf_n g_n\) is measurable
\end{itemize}
\end{proof}
\subsection*{Simple functions}
\begin{definition}
    \((X, \mathcal{M})\) measurable space. A measurable function \(s:X \to \barreal\) is said to be simple if \(s(X)\) is a finite set. 
    \[
        s(X) = \{a_1 \ldots, a_n\} \mbox{ for some }n \in \mathbb{N}, a_i \not= a_j
    \]
    Then 
    \[
        s(x) = \sum_{n = 1} a_n \chi_{E_n}(x)
    \]
    where \(E_n\) is a measurable set, \(E_n = \left\{ x \in X : s(X) = a_n \right\}\), and \(E_i \cap E_j = \emptyset\) for \(i \not = j\), and \(\bigcup_{n = 1}^N E_n = X\).
\end{definition}
\noindent\underline{Particular case}: if \(s: \mathbb{R} \to \barreal\), and each \(E_n\) is a finite union of intervals, then \(s\) is said to be a \textbf{step} function.

\noindent\underline{Goal}: to approximate arbitrary measurable functions with simple functions.
\begin{theorem}
    \((X,\mathcal{M})\) measurable space, \(f: X \to [0, \infty]\) measurable. Then \(\exists\) a sequence \(\left\lbrace s_n \right\rbrace\) of simple functions s.t. 
    \[
        0 \leq s_1 \leq \ldots \leq s_n \leq \ldots \leq f \quad \underset{\forall \; x \in X}{\mbox{(pointwise)}}
    \]
    and \(s_n(x) \to f(x) \quad \forall \; x \in X \mbox{ as }n \to \infty\).

Moreover if f is bounded then \(s_n \to f\) uniformly on \(X\) as \(n \to \infty\)
\end{theorem}

\begin{proof}[proof - for f bounded]
    Fix \(n \in \mathbb{N}\) and divide \([0,n)\) in \(n \cdot 2^n\) intervals called \(I_j = [a_j,b_j)\) with length \(\frac{1}{2^n}\)

    Let \(E_0 = f^{-1}([n, +\infty)), E_j = f^{-1}([a_j, b_j))\) for \(j = 1, \ldots, n\cdot 2^n\)
    
    We let \(\begin{array}{cc}
        s_n(x) = a_j & \mbox{for } x \in E_j \\
        s_n(x) = n & \mbox{for } x \in E_0
    \end{array}\)

    Namely we define the simple function \(s_n\) as
    \[
    s_n (x) = n\chi_{E_0}(X) + \sum_{j =1}^{n \cdot 2^n} a_j \chi_{E_j}(x)    
    \]
    Then \(s_n \leq s_{n+1}\) by contradiction, and, since \(f\) is bounded, \(E_0 = \emptyset\) for \(n\) sufficiently large (\(n > \sup f\)).

    Then any \(x \in X\) stays in \(f^{-1}([a_j, b_j))\) for some \(j\) 
    \[
        \begin{array}{l}
            \Rightarrow \stackbelow{a_j}{s_n(x)} \leq f(x) < b_j \\
            \Rightarrow 0 \leq f(x)-s_n(x) < b_j - a_j = \frac{1}{2^n} \\
            \Rightarrow \sup_{x \in X} \abs{f(x)- s_n({x})} < \frac{1}{2^n} \to 0 \mbox{ as } n \to \infty
        \end{array}
    \]
    Namely, \(s_n \to f\) uniformly on X.
\end{proof}

\section{Lecture 29/09/2022}
\begin{remark}
    On the relation between \((\real, \boreal, \lambda)\) and  \((\real, \mathcal{L}(\real), \lambda)\) \qquad (\(\lambda =\) Lebesgue measure)

    \((\real, \boreal, \lambda)\) is not complete. In fact, \((\real, \mathcal{L}(\real), \lambda)\) is the completion of \((\real, \boreal, \lambda)\).

    Note that, \(\forall \; E \in \mathcal{L}(\real) \exists \; \mbox{ a } G_{\delta}-\mbox{set } A\) and an \(F_{\delta}-\mbox{set } B\) s.t.
    \[
        \begin{array}{l}
            A \supset E \mbox{ and } \lambda(A \setminus E) = 0 \\
            B \subset E \mbox{ and } \lambda(E \setminus B) = 0
        \end{array}
    \]
\end{remark}
\((X, \mathcal{M}, \mu)\) complete measure space. \\
Let \(P(x)\) be a proposition depending on \(x \in X\). We say that \(P(x)\) is true \((\mu-)\) almost everywhere if 
\[
    \mu\left(\left\{ x \in X : P(x) \mbox{ is false}\right\}\right) = 0
\]
\(P(x)\) is true \(\underset{(\mu-\text{a.e.})}{\mbox{a.e.}}\) on \(X\).

\noindent\underline{Ex}: \((\real, \mathcal{L}(\real), \lambda)\), \(f(x) = x^2\).  
Then \(f(x) > 0\) a.e. on \(\real\) (for a.e. \(x\)):
\[
    \left\{ f(x) \leq 0 \right\} = \left\{ 0 \right\}, \mbox{ and } \lambda(\left\{ 0 \right\}) = 0
\]
\((\real, \mathcal{P}(\real), \mu_C)\) with \(\mu_C\) counting measure. Then it is not true that \(f(x) > 0\) \(\mu_C\)-a.e. 
\[
    \mu_C \left(\left\{ 0 \right\}\right) = 1
\]
It will be useful to consider sequences converging a.e.: 
\[
    f_n \to f \qquad \mbox{a.e. on }X
\]
if \(\mu\left( \left\{ x \in X : \lim_n f_n(x) \neq f(x), \mbox{ or does not exist} \right\}\right) = 0\)

\begin{proposition}
    \((X, \mathcal{M}, \mu)\) complete measure space. 
    \begin{enumerate}
        \item \(f: X \to \real\) is measurable, and \(g = f \) a.e. on \(X\), then \(g\) is measurable
        \item \(f_n \to f\) a.e. on \(X\), \(f_n : X \to \real\) measurable for all \(n\), then \(f\) is measurable
    \end{enumerate}
\end{proposition}

\subsubsection*{Integration of non-negative functions}
\noindent\underline{Notation}: \[
    \begin{array}{c}
        \left\{ x \in X : f(x) \geq 0 \right\} = \left\{ f \geq 0 \right\} \\
        \left\{ x \in X : f(x) > 0 \right\} = \left\{ f > 0 \right\}   \\
        \vdots
    \end{array}
    \]
\((X, \mathcal{M}, \mu)\) complete measure space.
We consider measurable functions \(f: X \to [0, +\infty]\)

\noindent\underline{Convention}: we define 
\[
    \begin{array}{l}
        a + \infty = +\infty \quad \forall \; a \in \real \\
        a \cdot (+\infty) = \begin{cases}
            +\infty & \mbox{if } a \neq 0, a > 0 \\
            0 & \mbox{if } a = 0
        \end{cases}        
    \end{array}
\]
With this convention, \(+ \) and \( \cdot\) of measurable functions are measurable functions.
\begin{definition}
    Let \(s: X \to [0, +\infty]\) be a measurable simple function, 
    \[
        s(x) = \sum_{n=1}^m a_n \chi_{D_n}(\bar{x})
    \]
    where \(D_1,\ldots,D_m\) are measurable, disjoint, and \(\bigcup_{n=1}^m D_n = X\). Let also \(E \in \mathcal{M}\). Then we define 
    \[
        \int_E s \, d\mu := \sum_{n=1}^m a_n \mu(D_n \cap E)
    \]
\end{definition}
\begin{remark}
    Given a simple function \(s\):
    \[s:[a,b] \to \real, \lambda = \mu \Rightarrow \int_E s \, d\mu \mbox{ is the area under the curve}\]
\end{remark}
\begin{remark}
    There are several points:
    \begin{itemize}
        \item In the definition we have already used the convention \(\mu(D_n \cap E = +\infty) \quad \mbox{ for some }n\)
        \item \(E \in \mathcal{M} \Rightarrow \chi_E\) is a simple function
        \[
            \chi_E(x) = 1 \cdot \chi_E + 0 \cdot \chi_{X\setminus E}(x)
        \] 
        In this case 
        \[
            \int_X \chi_E \, d\mu = 1\cdot \mu(E) + 0 \cdot \mu(X\setminus E) = \mu(E)
        \]
        \item \(s\chi_E = \sum_{n=1}^m a_n\chi_{D_n \cap E} \Rightarrow \int_E s\, d\mu = \int_X s\chi_E \, d\mu\)
    \end{itemize}
\end{remark}
\begin{definition}
    \(f:X \to [0, +\infty]\) measurable, \(E \in \mathcal{M}\). The \textbf{Lebesgue integral} of \(f\) on \(E\), with respect to (w.r.t.) \(\mu\), is 
    \[
        \int_E f \, d\mu = \sup \left\{ \int_E d\mu \vert \begin{array}{l}s\text{ is simple} \\ 0 \leq s \leq f \end{array}\right\}
    \]
\end{definition}

\begin{enumerate}
    \item If \(f \) is simple, the definitions are consistent
    \item Also for \(f\) measurable: \( \int_E f \, d\mu = \int_X f \chi_E \, d\mu\)
    \item \( \left( \mathbb{N}, \mathcal{\mathbb{N}}, \mu_C \right)\). \(f: \mathbb{N} \to \mathbb{R}\) is a sequence \( \left\{ a_n \right\}_{n \in \mathbb{N}}\) \[ \int_\mathbb{N} \{a_n\} \, d\mu_C = \sum_{n=0}^\infty a_n\]
\end{enumerate}

Basic Properties. \\
Let \(f, g : X \to \left[0, \infty\right]\) measurable. \(E, F \in \mathcal{M}, \ \alpha \geq 0\). Then: 
\begin{enumerate}
    \item \(\mu(E)=0 \Rightarrow \int_E f \, d\mu = 0\)
    \item \(f \leq g \) on \(E \Rightarrow \int_E f \, d\mu \leq \int_E g \, d\mu \)
    \item \(E \subset F \Rightarrow \int_E f \, d\mu \leq \int_F f \, d\mu\)
    \item \(\alpha \geq o \Rightarrow \int_E \alpha f \, d\mu = \alpha \int_E d \, d\mu\)
\end{enumerate}

\begin{remark}
    \(\left(\left[0, 1\right], \mathcal{L}(\left[0, 1\right]), \lambda  \right)\) \\
    Consider \(\chi_\mathbb{Q}\), it is the Dirichlet function on \(\left[0, 1\right]\). This is not Riemann integrable. \\ 
    However, \(\int_{\left[0,1\right]} \chi_{\mathbb{Q}} \, d\lambda = \lambda \left( \mathbb{Q} \cap \left[0,1\right] \right) =0 \)
\end{remark}

\begin{theorem}[Chebychev's inequality]
    \(f: X \to \left[0, \infty \right]\) measurable, \(c > 0\). Then \[ \mu\left(\{f \geq c \}\right) \leq \frac{1}{c} \int{\{f \geq c \}} f \, d\mu \leq \frac{1}{c} \int_X f \, d\mu \]
\end{theorem}
\begin{proof}
    \[ \int_X f \, d\mu \overset{X \supset \{f \geq c\}}{\geq} \int_{\{f \geq c \}} f \, d\mu \geq \int_{\{f \geq c\}} c \, d\mu 
    = c \int_{\{f \geq c\}} \, d\mu 
    = c \mu \left(\{f \geq c\}\right) \]
\end{proof}

\begin{theorem}
    \(s : X \to \left[0, \infty\right]\) simple. Define \(\phi : \mathcal{M} \to \left[0, \infty\right] \\ \phi(E) = \int_E s \, d\mu \\ \Rightarrow \phi \) is a measure. 
\end{theorem}

\begin{proof}
    \(\mu(\emptyset) =0 \Rightarrow \phi(\emptyset)=0 \) by definition. 
    \begin{definition}[sigma additivity]
        \(\{E_n \subset \mathcal{M}\}\) disjoint, and let \(E = \bigcup_{n=1}^\infty E_n \Rightarrow s = \sum_{k=1}^m a_k \chi_{D_k} \; D_k \in \mathcal{M}
        \)
    \end{definition}
    Then, by definition and since \(\mu\) is a measure and \(E \cap D_k = \bigcup_n (E_n \cap D_k)  \) 
    \[
        \phi(E) = \sum_{k=1}^m a_k \mu(D_k \cap E) = 
        \sum_{k=1}^\infty a_k \sum_{n=1}^\infty \mu(E_n \cap D_k)= 
    \]
    \[    
        \sum_{n=1}^\infty \left( \sum_{k=1}^m a_k \mu (E_n \cap D_k) \right) = 
        \sum_{n=1}^\infty \int_{E_n} s \, d\mu = 
        \sum_{n=1}^\infty \phi(E_n)
    \]
\end{proof}

\begin{theorem}[Vanishing Lemma]
    \(f: X \to \left[0, \infty\right]\) measurable. \(E \subset X \) measurable 
    \[\int_E f \, d\mu =0 \Leftrightarrow f=0 \text{ a.e. on } E \]
\end{theorem}
\begin{proof}
    \( \Leftarrow \) easy \\
    \( \Rightarrow \) Consider \( E \cap \{f >0\} = \bigcup_{n=1}^\infty \underbrace{\left(E_n \{ f \geq \frac{1}{n} \} \right)}_{=:E_n} \) \\
    Then \(\{E_n\}\) is an increasing sequence. By Chebyshev 
    \[
        \mu (E_n) \leq \frac{1}{\frac{1}{n}} \int_E f \, d\mu =0 \; \forall n \Rightarrow \mu(E_n)=0 \quad \forall \; n 
    \]
    \(\mu(E) \cup \{f>0\} \overset{\text{continuity}}{=} \lim_n \mu (E_n)=0\), namely \(f=0\) a.e. on \(E\)
\end{proof}

The \(\int\) does not see sets with 0 measure.

\begin{definition}
    If \( f:X \to \left[0, \infty\right] \) is measurable, and \( \int_X f \, d\mu < \infty \) then we say that \(f\) is integrable.
\end{definition}

\begin{theorem}[Monotone Convergence - Beppo Levi]
    \(f_n:X\to \left[0, \infty\right]\) measurable. Suppose that 
    \begin{itemize}
        \item \(f_n(x) \leq f_{n+1}(x)\) for a.e. \(x \in X\) for every \(n\)
        \item \(f_n \to f \) a.e. on \(X\)
    \end{itemize} 
    Then \[ \int_X f \, d\mu = \lim_n \int_X f_n \, d\mu\]
\end{theorem}
\begin{proof}
    Part 1. \\
    Assume that the two hypotesis hold everywhere. First, if \(f\) is measurable 
    \[
        \int_X f_n \, d\mu \nearrow \quad \Rightarrow \exists \; \alpha = \lim_n \int_X f_n \, d\mu
    \]
    
    Also, \(f_n \leq f \) everywhere \(\Rightarrow \int_X f_n \, d\mu \leq \int_X f \, d\mu \quad \forall \; n\) 
    \[
        \Rightarrow \alpha \leq \int_X f \, d\mu 
    \]
    We want to show that also \(\geq \) is true. Let \(s\) be a simple function s.t. \(0 \leq s \leq f\) and \(c \in \left(0,1\right)\)
    Let \(E_n = \{f_n \geq cs\} \in \mathcal{M}\)
    \begin{itemize}
        \item \(E_n \in E_{n+1} \; \forall \; n:\) 
        \\ if \(x \in E_n, \) then \(f_n(x) \geq cs(x) \Rightarrow f_{n+1}(x) \geq cs(x)\) \\ \(\Rightarrow f_{n+1}(x) \geq f_n(x) \geq cs(x) \Rightarrow x \in E_{n+1}\)
        \item Moreover, \(X = \bigcup_{n=1}^\infty E_n\). Indeed: 
        \\ - if \(f(x)=0\), then \(s(x)=0 \Rightarrow f_1(x)=0 = cs(x), \; x \in E_1\) 
        \\ - if \(f(x)>0\), then \(cs(x) < f(x)=\lim_n f_n(x)\) since \(s \leq f \) and \(c <1\) 
        \\ \(\Rightarrow cs(x) < f_n(x)\) for \(n \) sufficiently large, namely \(x \in E_n \) for \(n \) sufficiently large. 
    \end{itemize} 
    Therefore, 
    \[
        \alpha \geq \int_X f_n \, d\mu \geq \int_{E_n} f_n \, d\mu \geq c \int_{E_n} s \, d\mu = c \phi(E_n)
    \]
    \(\forall n, \ \forall 0 \leq s \leq f, \forall c \in \left[0, 1\right]\quad \phi(E_n) = \int_{E_n} s \, d\mu\). 
    \(\phi\) is a measure, and \(\{E_n\} \nearrow\) \\
    Therefore, taking the lim when \(n \to \infty\) by continuity 
    \[
        \alpha \geq \lim_n c \int_{E_n} s \, d\mu = c \int_X s \, d\mu \; \quad \forall c \in \left[0, 1\right]
    \]
    Take the limit when \(c \to 1^-: \ \alpha \geq \int_X s \, d\mu  \quad \forall \; 0 \leq s \leq f \) \\
    Take the sup over s: \(\alpha \geq \int_X f \, d\mu \).
    We proved both inequalities, so the thesis holds. \\
    Part 2. \\
    Note that \(\{x \in X: \text{ assumpions of the theorem do not hold}\}\) is a set of zero measure. Take \(F. \ X = E \cup F \) since we have the assumpion on \(E\) and \(\mu (F)=0\). \\ 
    Then, by the Vanishing Lemma, since \((f - f \chi_E)=0\) a.e. and \((f_n - f_n \chi_E)=0\) we have that 
    \[ 
        \int_X f \, d\mu = \int_E d \, d\mu = \lim_n \int_E f_n \, d\mu = \lim_n \int_X f_n \, d\mu 
    \]
\end{proof}
\section{Lecture 05/10/2022}
\begin{theorem}[Monotone Convergence or Beppo Levi's theorem]
    \(f_n : X \to [0, +\infty]\) measurable. Suppose that 
    \begin{enumerate}
        \item \(f_n(x) \leq f_{n+1}(x)\) for a.e. \(x \in X\), for every \(n\)
        \item \(f_n \to f\) a.e. on \(X\) 
    \end{enumerate}
    Then 
    \[
        \int_X f \, d\mu = \lim_n \int_X f_n \, d\mu
    \]
\end{theorem}

\begin{corollary}[Monotone convergence for series]
    \(f_n : X \to [0, +\infty]\) measurable, then 
    \[
        \int_X \left( \sum_{n=0}^{\infty} f_n\right) \, d\mu = \sum_{n=0}^{\infty} \int_X f_n \, d\mu
    \]
\end{corollary}

\begin{theorem}[Approximation with simple functions]
    Given \((X, \mathcal{M})\) measure space, \(f: X \to [0, +\infty]\) measurable, then \(\exists\) a sequence \(\left\{ s_n \right\}\) of simple functions s.t. 
    \[
        0 \leq s_1 \leq \ldots \leq s_n \leq \ldots \leq f \qquad \text{pointwise } \forall \; x \in X
    \]
    and 
    \[
        s_n (x) \to f(x) \qquad \forall \; x \in X \text{as } n \to \infty 
    \]
    Moreover, if \(f\) is bounded, then \(s_n \to f\) uniformly on \(X\) as \(n \to \infty\).
\end{theorem}

\begin{remark}
    There is also
    \[
        \int_X f \, d\mu = \sup \left\{ \int_X s \, d\mu \, \bigg\vert \begin{array}{l}s\text{ is simple} \\ 0 \leq s \leq f \end{array}\right\}
    \]
\end{remark}

But let \(\left\{ s_n \right\}\) be the sequence given by the simple approximation theorem. By monotone convergence 
\[
    \int_X f \, d\mu = \lim_n \int_X s_n \, d\mu
\]
Ex: \(f, g : X \to [0, +\infty]\). Then 
\[
    \int_X (f+g) \, d\mu = \int_X f \, d\mu + \int_X g \, d\mu
\]

\begin{lemma}[Fatou's Lemma]
    Given \(f_n \to [0, +\infty]\) measurable \(\forall \, n\). Then 
    \[
        \int_X (\liminf_n f_n) \, d\mu \leq \liminf_n \int_X f_n \, d\mu
    \]
    In particular, if \(f_n \to f\) a.e. on \(X\), then
    \[
        \int_X f \, d\mu \leq \liminf_n \int_X f_n d\mu
    \]
\end{lemma}

\begin{proof}
    Given that \((\liminf_n f_n)(x) = \lim_n (\underbrace{\inf_{k \geq n} f_k(x)}_{= g_n (x)})\). Now, for every \(x \in X\), \(\left\{ g_n(x) \right\}\nearrow\)
    \[
        g_{n+1}(x) = \inf_{k \geq n+1} f_k(x) \geq \inf_{k \geq n} f_k(x) = g_n (x)
    \]
    Also, \(g_n \geq 0\) on \(X\). Thus, by monotone convergence
    \[
        \int_X \liminf_n f_n \, d\mu = \int_X \lim_n g_n \, d\mu = \lim_n \int_X g_n \, d\mu = \liminf \int_X g_n \, d\mu
    \]
    Now, note that \(g_n (x) = \inf_{k\geq n} f_k(x) \leq f_n(x) \leq \liminf_n \int_X f_n \, d\mu\) 
\end{proof}

\begin{theorem}[\(\sigma\)-additivity of \(\int\)]
    Given \((X, \mathcal{M}, \mu)\) measure space, \(\oldphi : X \to [0, +\infty]\). Define \(\nu(E) = \int_E \oldphi \, d\mu\), with \(E \in \mathcal{M}\). 
    \(\nu : \mathcal{M} \to [0, +\infty]\) is a measure. Moreover, let \(f:X \to [0, +\infty]\) measurable
    \[
        \int_X f \, d\nu = \int_X f\oldphi \, d\mu \tag*{*}
    \]
\end{theorem}
\begin{proof}
    \noindent\underline{\(\nu\) is a measure}:  

    \(\nu(\emptyset) = 0\), since \(\mu(\emptyset) = 0\). 
    Now, let \(E = \bigcup_{n=1}^{\infty} E_k\), \(\left\{ E_k \right\}\) disjoint. Then 
    \[
        \nu(E) = \int_X \oldphi\chi_{E} \, d\mu = \int_X \oldphi\sum_n \chi_{E_n} \, d\mu \underset{\begin{array}{l} \text{\scriptsize{monot. conv.}} \\ 
            \text{\scriptsize{for }}\textstyle\sum\end{array}}{=}  \sum_n \int_X \oldphi \chi_{E_n} \, d\mu = \sum_n \int_X \oldphi \, d\mu = \sum_n \nu(E_n) 
    \]
    We have proven \(\sigma\) additivity, so \(\nu\) is a measure.

    \noindent\underline{(*) holds}:
    Let \(E \in \mathcal{M}\). Then
    \[
        \int_X \chi_E \, d\nu = \int_E 1 \, d\nu = \nu(E) = \int_E = \int_E \oldphi \, d\mu = \int_X \oldphi \chi_E \, d\mu
    \]
    This shows that \((*)\) holds for \(\chi_E\), \(\forall \; E\). Then it holds for simple functions. \\
    Let now f be any measurable function, positive. Then we can take \(\{s_n\}\) given by the simple approximation theorem
    \[
        \int_X f \, d\nu \overset{\text{monot}}{=} \lim_n \int_X s_n \, d\nu = \lim_n \int_X s_n \oldphi \, d\mu \overset{\text{monot}}{=} \int_X f \oldphi \, d\mu 
    \]
    which is \((*)\) 
\end{proof}

\begin{remark}
    \(X, \mathcal{M}, \mu\) complete measure space. Then, by the vanishing lemma, it is not difficult to deduce that 
    \[
        f=g \text{ a.e. on } X \Leftrightarrow \int_E f \, d\mu = \int_E g \, d\mu \qquad \forall\; E \in \mathcal{M}
    \]
    The \(\int\) does not see differences of sets with 0 measure. As a consequence, in all the theorems, it is sufficient to assume that the assumptions are satisfid a.e. 
\end{remark}


\subsection*{Integrals for real valued functions}
\(X, \mathcal{M}, \mu\) complete measure space. \\
\(f: X \to \barreal = [-\infty, \infty]\) measurable. Recall \(f= f^+ - f^- \) where \(f^+ = \max{\{f, 0\}}\), \(f^- = -\min{\{f, 0\}} \) and \(|f|= f^+ + f^-\). 
Note that both are positive and measurable. 

\begin{definition}
    we say that \(f:X \to \barreal\) measurable is integrable on \(X\) if 
    \[
        \int_X |f| \, d\mu < \infty
    \]
\end{definition}

If \(f\) is integrable, we define \(\int_X f \, d\mu = \int_X f^+ \, d\mu + \int_X f^- \, d\mu\) \\
The set of integrable functions is denoted by 
\[
    \mathcal{L}^1 (X, \mathcal{M}, \mu) := \{f:X \to \barreal \text{ integrable} \} 
\]
\[
    \mathcal{L}^1 (X, \mathcal{M}, \mu) 
    = \mathcal{L}^1 (X) 
    = \mathcal{L}^1 
\]

If \(E \in \mathcal{M}\), we define
\[
    \int_E f \, d\mu = \int_X f \chi_E \, d\mu
\]

\begin{remark}
    \(f \in \mathcal{L}^1(X) \Rightarrow \int_X f \, d\mu \in \real\). Indeed \(0 \leq f^\pm \leq |f|\)
    \[
        \Rightarrow 0 \leq \int_X f^+ \, d\mu ,\ \int_X f^- \, d\mu \leq \int_X |f| \, d\mu < \infty 
    \] 
    \[
        \Rightarrow \int_X f \, d\mu = \int_X f^+ \, d\mu - \int_X f^- \, d\mu \in \real
    \]
\end{remark}

\begin{proposition}
    \(f : X \to \barreal \) measurable. Then
    \begin{enumerate}
        \item \(f \in \mathcal{L}^1 \Leftrightarrow |f| \in \mathcal{L}^1 \Leftrightarrow \) both \(f^+, \ f^-\) \(\in \mathcal{L}^1\)
        \item \(f \in \mathcal{L}^1 \), then 
        \[
            \left| \int_X f \, d\mu \right| \leq \int_X |f| \, d\mu  \tag{triangle inequality}
        \]
    \end{enumerate}
\end{proposition}

\begin{proof}
    Of the second part.
    \[
        \left| \int_X f \, d\mu \right| = \left| \int_X f^+ \, d\mu + \int_X f^- \, d\mu\right| \leq \int_X f^+ \, d\mu + \int_X f^- \, d\mu = \int_X |f| \, d\mu  
    \]
\end{proof}

\begin{proposition}
    \(\mathcal{L}^1(X, \mathcal{M}, \mu)\) is a vector space, and \(f, \, g \in \mathcal{L}^1\), \(\alpha \in \real\)
    \[
        \Rightarrow \int_X \left(\alpha f + g \right) \, d\mu = \alpha \int_X f \, d\mu + \int_X g \, d\mu 
    \]  
    by linearity of the integrals.
\end{proposition}

Many results can be extended from non negative functions to general functions.

\begin{theorem}
    \((X, \mathcal{M}, \mu)\) complete measure space. \(f\), \(g \in \mathcal{L}^1\). Then
    \[
        f= g \text{ a.e. on } X \Leftrightarrow \int_X |f-g| \, d\mu =0 \Leftrightarrow \int_E f \, d\mu = \int_E g \, d\mu \qquad \forall \; E \in \mathcal{M} 
    \]
\end{theorem}
The most relevant theorem for convergence is the following
\begin{theorem}[Dominated convergence theorem]
    \(\{f_n\}\) sequence of measurable functions \(X \to \barreal\). Suppose that
    \begin{enumerate}
        \item \(f_n \to f \) a.e. on \(X\)
        \item \(\exists \; g : X \to \barreal \), \(g \in \mathcal{L}^1(X)\), such that \(|f_n(x) \leq g(x)|\) a.e. on \(X\) \(\forall \; n \in \mathbb{N}\)
    \end{enumerate}
    Then \(f \in \mathcal{L}^1\) and 
    \[
        \lim_n \int_X |f_n -f| \, d\mu = 0 
        \qquad \left( \Rightarrow \int_X f \, d\mu = \lim_n \int_X f_n \, d\mu \right)  
    \]
\end{theorem}
\begin{proof}
    Note that \(f_n \in \mathcal{L}^1\) \(\forall \; n\), since \(|f_n| \leq g\) and we have the monotonicity of \(\int\) for non negative functions
    \[
        |f_n(x)| \leq g(x) \quad
        n \to \infty \qquad
        |f(x)| \leq g(x) \text{  a.e. on } X 
    \]
    \[ 
        \Rightarrow f \in \mathcal{L}^1(X)
    \]  
    Now, consider \(\oldphi_n = 2g - |f_n - f|\). We have
    \[
        |f_n-f| \leq |f_n|+|f| \leq 2g \quad \text{ a.e. on }X \quad \oldphi_n \geq 0 \quad \text{ a.e. on }X  
    \]
    We can use Fatou's lemma:
    \[\begin{array}{l}
        \int_X (\underbrace{\liminf_n \oldphi_n}_{\stackbelowlittle{\mbox{= 2g a.e.}}{\int_X 2g \; d\mu}}) \; d\mu \leq \liminf_n \int_X \oldphi_n \, d\mu = \liminf_n \int_X (2g - \abs{f_n - f}) \, d\mu = \\
        = \int_X 2g \, d\mu + \liminf_n (-\int_X \abs{f_n - f} \, d\mu) =\int_X 2g \, d\mu - \limsup_n \int_X \abs{f_n - f} \, d\mu
    \end{array}
    \]
    Subtracting \(\int_X 2g \, d\mu\) from both sides 
    \[
        0 \leq -\limsup_n \int_X \abs{f_n - f} \, d\mu \Rightarrow 0 \leq \liminf_n \int_X \abs{f_n - f} \, d\mu \leq \limsup_n \int_X \abs{f_n - f} \, d\mu \leq 0 
    \]  
\end{proof}
\begin{remark}
    If \(\mu(X) < +\infty\), and \(\exists \; M > 0\) s.t. \(\abs{f_n} \leq M\) a.e. on \(X, \; \forall \; n\), then we can take \(g = M\) as dominating function.
\end{remark}

\noindent\underline{Comments on the relation between Riemann and Lebesgue integrals}

Let \(f : I \subset \real \to \real\), \(I\) interval, be bounded. Assume also that \(I\) is closed and bounded.
\begin{theorem}
    Let \(f\) be Riemann-integrable on \(I \ (f \in R(I))\). Then 
    \[
        f\in \mathcal{L}^1(I, \mathcal{L}(I), \lambda)
    \]
    and 
    \[
        \int_I f \, d\lambda = \int_I f(x) \, dx
    \]
\end{theorem}
\begin{theorem}
    \(f \in R(I) \Leftrightarrow f\) is continuous on \(x\), for a.e. \(x \in I\).
\end{theorem}
\noindent\underline{Ex}: \(\chi_{\mathbb{Q}}\) on \([0,1]\) is not Riemann integrable, because it is discontinuous at any point. Note that, instead, \(\chi_{\mathbb{Q}} = 0\) a.e. on \([0,1]\) \(\Rightarrow \int_{[0,1]} \chi_{\mathbb{Q}} \, d\lambda = 0\)
\section{Lecture 06/10/2022}
Let \(f \not \in R(I)\). Is it true that \(\exists \; g \in R(I)\) s.t. \(g = f\) a.e. on \(I\)? No.

For instance, consider \(T_{\mathcal{E}}\), the generalized Cantor set (\(\lambda(T_{\mathcal{E}}) = 0\)) and then consider \(\chi_{T_{\mathcal{E}}}\). \\
In general, \(\chi_{A}\) is discontinuous on \(\delta A\).  But \(T_{\mathcal{E}}\) has no interior parts \(\Rightarrow T_{\mathcal{E}} = \delta T_{\mathcal{E}}\) \(\Rightarrow \chi_{T_{\mathcal{E}}}\) is discontinuous on \(T_{\mathcal{E}}\), which has positive measure
\(\Rightarrow \) by the last theorem, \(\chi_{T_\epsilon}\) is not \(R(I)\)

Clearly 
\[
    \int_{[0,1]} \chi_{T_{\mathcal{E}}} d\lambda = \lambda(T_{\mathcal{E}})
\]
so \(\chi_{T_{\mathcal{E}}} \in \mathcal{L}^1([0,1])\).  

If \(g = \chi_{T_{\mathcal{E}}}\) a.e., then \(g\) is discontinuous at almost every part of \(T_{\mathcal{E}} \Rightarrow\) \(g\) is discontinuous on a set of positive measure \(\Rightarrow g \not \in R(I)\). 
So, the Lebesgue integral is a true extension of the Riemann one.

Regarding generalized integrals we have

\begin{theorem}
    \(-\infty \leq a < b \leq +\infty, \quad f \in R^g([a,b])\) where 
    \[
        R^g([a,b]) = \left\lbrace \mbox{Riemann-int functions on }[a,b]\mbox{ in the generalized sense} \right\rbrace
    \]
    Then, \(f\) is \(([a,b], \mathcal{L}([a,b]))\)-measurable. Moreover
    \begin{enumerate}
        \item \(f \geq 0\) on \([a,b] \Rightarrow f \in \mathcal{L}^1([a,b])\)
        \item \(\vert f \vert \in R^g([a,b]) \Rightarrow f \in \mathcal{L}^1 ([a,b])\)
    \end{enumerate}
    and in both cases
    \[
        \int_{[a,b]} fd\lambda = \int_a^b f(x)dx
    \]
    If \(f\) is in \(R^g([a,b])\), but \(\vert f\vert \not \in R^g([a,b])\), then the two notions of \(\int\) are not really related
\end{theorem}

Ex:
\(f(x) = \frac{\sin x}{x},  x \in [1, \infty]\)
\[\int_1^{\infty} \vert f(x) \vert dx = +\infty \Rightarrow f \not \in \mathcal{L}^1([1, +\infty])\].
But on the other hand
\[
    \int_1^{\infty} \frac{\sin x}{x} dx = \lim_{\omega \to \infty} \int_1^{\omega} \frac{\sin x}{x} dx = \frac{\pi}{2}
\]

\begin{proposition}
    \((X, \mathcal{M}, \mu)\) complete measure space. Let \(\{f_n\} \subseteq \mathcal{L}^1(X, \mathcal{M}, \mu)\). 
    
    Suppose that \(\sum_{n=1}^\infty \int_X |f_n| \, d\mu < \infty\)
    Then the series \(\sum_{n=1}^\infty f_n\) converges a.e. on \(X\), it is in \(\mathcal{L}^1(X)\) and 
    \[
        \int_X \left( \sum_{n=1}^\infty f_n  \right) \, d\mu = \sum_{n=1}^\infty \int_X f_n \, d\mu
    \]
\end{proposition}


\subsubsection*{Spaces of integrable functions}
\((X, \mathcal{M}, \mu)\) complete measure space.
\[
    \mathcal{L}^1 = \left\lbrace f: X \to \barreal : \mbox{ f is integrable}\right\rbrace
\]
\(\mathcal{L}^1\) is a vector space. On \(\mathcal{L}^1\) we can introduce \(d : \mathcal{L}^1 \times \mathcal{L}^1 \to [0, +\infty)\) defined by 
\[
    d_1 (f,g) =\int_{X} \vert f-g \vert 
\]

It is immediate to check that 
\[
    d_1 (f, g) = d_1(g, f) \tag*{(symmetry)}
\]  

\[
    d_1(f, g) \leq d_1(f, h) + d_1(h, g) \; \;\forall f, g, h \in \mathcal{L}^1(X) \tag*{(triangular inequality)}
\]  
However, \(d_1\) is not a distance on \(\mathcal{L}^1(X)\), since 
\[
    d_1(f,g) = 0 \Rightarrow f=g \quad \mbox{a.e on }X
\tag*{(pseudo-distance)}\]
To overcome this problem, we introduce an equivalent relation in \(\mathcal{L}^1(X)\): we say that 
\[
    f \sim g \Leftrightarrow f = g \quad \mbox{a.e. on }X
\]
If \(f \in \mathcal{L}^1(X)\), we can consider the equivalence class
\[
    [f] = \left\lbrace g \in \mathcal{L}^1(X) : g = f \mbox{ a.e on }X \right\rbrace
\]
We define
\[
    L^1(X) = \frac{\mathcal{L}^1(X)}{\sim} = \{[f]: f \in \mathcal{L}^1(X)\}
\]
\(L^1(X)\) is a vector space, and on \(L^1(X)\) the function \(d_1\) is a distance: 
\[
    d_1([f], [g]) = 0 \Leftrightarrow \int_X |[f]-[g]|\, d\mu =0 \Leftrightarrow [f]= [g] \text{ a.e. } \Leftrightarrow f=g \text{ a.e. }
\]
To simplify the notations, the elements of \(L^1(X)\) are called functions, and one writes \(f \in L^1(X)\). With this, we means that we choose a representative in \([f]\), and f denotes both the representative and the equivalence class. The representative can be arbitrarily modified on any set with \(0\) measure.

Another relevant space of measurable functions is the space of \textbf{essentially bounded} functions.
\begin{definition}
    \(f : X \to \overline{\mathbb{R}}\) measurable is called essentially bounded if \(\exists \; M > 0\) s.t.
    \[
        \mu(\left\lbrace x \in X : \vert f(x) \vert \geq M \right\rbrace) = 0
    \]
\end{definition}
Ex: 
\[f(x) = \begin{cases}
    1 & x > 0 \\
+\infty & x = 0 \\
0 & x < 0 \\
\end{cases}
\] 
For \(M > 1\), \(\lambda(\left\lbrace x \in \mathbb{R} : \vert f(x) \vert > M\right\rbrace) = \lambda(\left\lbrace 0 \right\rbrace) = 0 \Rightarrow f\mbox{ is essentially bounded}\).


If \(f\) is essentially bounded, it is well defined the \textbf{essential supremum} of \(f\).
\[
    \underset{X}{\esssup} f := \inf \left\lbrace M > 0 \mbox{ s.t. } f \leq M \mbox{ a.e. on }X\right\rbrace = \inf \left\lbrace M > 0 \mbox{ s.t. } \mu(\{f \geq M \})=0 \right\rbrace
\]
It can also be defined an essential inf.
\begin{remark}
    Note that, by def of inf, \(\forall \; \epsilon > 0\) we have 
    \[
        f \leq (\underset{X}{\esssup} f) + \epsilon \qquad \text{a.e. on }X
    \]
\end{remark}
We define 
\[
    L^{\infty} (X, \mathcal{M}, \mu) = \frac{\mathcal{L}^{\infty}(X, \mathcal{M}, \mu)}{\sim}
\]
\(L^{\infty}(X)\) is a vector space, and it is also a metric space for \(d_{\infty}(f,g) = \underset{X}{\esssup} \vert f-g \vert\)
\subsubsection*{Relation between different types of convergence}
\(\left\lbrace f_n \right\rbrace\) sequence of measurable functions \(X \to \barreal\)
\begin{itemize}
    \item \(f_n \to f\) pointwise (everywhere) on \(X\) if \(f_n(x) \overset{n \to \infty}{\to} f(x) \; \forall \; x \in X\)
    \item \(f_n \to f\) uniformly on \(X\) if \(\sup_{x \in X} \abs{f_n(x) - f(x)} \overset{n \to \infty}{\to} 0\)
    \item \(f_n \to f\) a.e. on \(X\) if 
    \[
        \begin{array}{c}
            \mu\left(\left\{ x \in X : \lim_n f_n(x) \neq f(x) \mbox{ or does not exist} \right\}\right) = 0 \\
            \Updownarrow \\
            f_n(x) \to f(x) \mbox{ for a.e } x \in X
        \end{array}
    \]
    \item \underline{Convergence in \(L^1(X)\)}: \(f_n \to f\) in \(L^1(X)\) if 
    \[
        \int_X \stackbelow{\abs{f_n - f}}{d_1(f_n, f)} \, d\mu \overset{n\to \infty}{\to} 0
    \]
    \item \underline{Convergence in measure/probability}: \(f_n \to f\) in measure if \(\forall \, \alpha > 0\)
    \[
        \lim_{n\to \infty} \mu\left(\left\{ \abs{f_n - f} \geq \alpha \right\}\right) = 0
    \]
\end{itemize}
\noindent\underline{Basic facts}: uniformly convergence \(\underset{\displaystyle\nLeftarrow}{\Rightarrow}\) pointwise \(\underset{\displaystyle\nLeftarrow}{\Rightarrow}\) a.e. convergence.

\noindent\underline{Ex:} \(f_n (x) = \exp{-nx}, x \in [0,1]\)
\[
    f(x) = 0, \quad g(x) = \begin{cases}
        0 & x \in (0, 1]\\ 1 & x = 0
    \end{cases}
\]
Then \(f_n \to g\) pointwise, \(g = f\) a.e. \(\Rightarrow f_n \to f\) a.e. on \([0,1]\). But \(f(0) \neq g(0) \Rightarrow f_n \to f\) pointwise. 

\(f_n \nrightarrow g\) uniformly on \([0,1]\) \(\bigg\lvert \begin{array}{l}
    f_n \in \mathcal{C}([0,1]) \\ f_n \to g \Rightarrow g \in \mathcal{C}([0,1])
\end{array}\)

a.e. \(\nRightarrow\) uniform, but not all is lost...
\begin{theorem}[Egorov]
    Let \(\mu(X) < +\infty\), and suppose that \(f_n \to f\) a.e. on \(X\). Then, \(\forall \; \epsilon > 0, \exists \; X_{\epsilon} \subset X\), measurable, s.t. 
    \[
        \mu(X \setminus X_{\epsilon}) < \epsilon
    \]
    and \(f_n \to f\) uniformly on \(X_{\epsilon}\)
\end{theorem}
\noindent\underline{Ex}: in an example \(f_n \to 0\) a.e., \(f_n \to 0\) uniformly on \([0,1]\), but \(f_n \to 0\) uniformly on \([\epsilon, 1]\).

Regarding a.e. convergence and in measure convergence there is the following theorem
\begin{theorem}
    If \(\mu(X) < +\infty\) and \(f_n \to f\) a.e. on \(X\) \(\Rightarrow f_n \to f\) in measure on \(X\)
\end{theorem}
\begin{proof}
    Let \(\alpha > 0\). We want to show that \(\forall \; \epsilon > 0\) \(\exists \; \bar{n} \in \mathbb{N}\) s.t. 
    \[
        n > \bar{n} \Rightarrow \mu(\left\lbrace \abs{f_n -f} \geq \alpha\right\rbrace) <\epsilon
    \]
    \(f_n \to f\) a.e. on \(X\), \(\mu(X) < +\infty\) \(\overset{\text{Egorov}}{\Rightarrow} \exists \; X_{\epsilon} \subseteq X\) s.t. \(\mu(X \setminus X_{\epsilon}) < \epsilon\) and \(f_n \to f\) uniformly on \(X_{\epsilon}\) \(\Leftrightarrow \sup_{X_{\epsilon}} \abs{f_n - f} \overset{n\to \infty}{\to} 0\).

    In particular, this means that \(\exists \; \bar{n} \in \mathbb{N}\) s.t. \(n > \bar{n} \Rightarrow \abs{f_n -f} < \alpha\) on \(X_{\epsilon}\).

    Therefore 
    \[
        \left\{ \abs{f_n -f} \geq \alpha \right\} \cap  X_{\epsilon} = \emptyset \Rightarrow \left\{ \abs{f_n -f} \geq \alpha \right\} \subseteq X \setminus X_{\epsilon} \qquad \text{for } n> \bar{n} 
    \]
    By monotonicity of \(\mu\), we deduce that
    \[
        \mu\left(\left\{ \abs{f_n -f} \geq \alpha \right\}\right) \leq \mu(X \setminus X_{\epsilon}) < \epsilon \quad \mbox{for }n> \bar{n}
    \]
    Namely, \(f_n \to f\) in measure.
\end{proof}
\begin{remark}
    \(\mu(X) < +\infty\) is essential
\end{remark}
For example, in \((\mathbb{R}, \mathcal{L}(\mathbb{R}), \lambda)\) consider
\[
    f_n (x) = \chi_{[n, n+1)}(x)
\]
\(f_n(x) \to 0\) for every \(x \in \mathbb{R}\). However, \(\lambda(\left\lbrace \vert f_n \vert \geq \frac{1}{2}\right\rbrace) = \lambda([n, n+1)) = 1\) not \(0\)

\section{Lecture 12/10/2022}
\begin{remark}
    Convergence in measure \(\Rightarrow\) a.e convergence?

    No, not even if \(\mu(X) < +\infty\).

    Consider \(\chi_{n,k} = \chi_{[\frac{k-1}{n}, \frac{k}{n}]}\) with \(n \in \mathbb{N}, k = 1, \ldots, n\)
    \[
        \begin{array}{ccc}
            \chi_{1,1}(x) = \chi_{[0, 1]}(x) & & \\
            \chi_{2,1}(x) = \chi_{[0, \frac{1}{2}]}(x) & \chi_{2,2}(x) = \chi_{[\frac{1}{2}, 1]}(x) & \\
            \chi_{3,1}(x) = \chi_{[0, \frac{1}{3}]}(x) & \chi_{3,2}(x) = \chi_{[\frac{1}{3}, \frac{2}{3}]}(x) &\chi_{3,3}(x) = \chi_{[\frac{2}{3}, 1]}(x) \\
        \end{array}
    \]
    For \(n\) fixed and \(k\) variable, we move the \(\chi\) from the left to right. When the \(\chi\) reaches \(1\), we switch \(n\), and \(\chi\) reappear from the left, being thinner. 
    \[
        \begin{array}{cc}
            \int_{[0,1]} \chi_{n,k} \, d\lambda = \frac{1}{n} & \int_{[0,1]} \chi_{n+1, k} \, d\lambda = \frac{1}{n+1}
        \end{array}
    \]
    We can order the elements of \(\chi_{n,k}\) in a sequence, letting
    \(f_p = \chi_{n,k}\) for \(p=1+2+\ldots+(n-1)+k\). We will prove that \(\left\{ f_p \right\}\) converges in measure, but not a.e. 

    This is the \textbf{typewriter sequence} (\(p(n,k)\)). For every \(x \in [0,1]\) there are \(\infty\) many indexes s.t. \(f_p(x) = 1\) and \(\infty\) many indexes s.t. \(f_p(x) = 0\), meaning that \(\nexists \; \lim_{p\to \infty} f_p(x)\)
    \( f_p \nrightarrow 0 \) a.e. on \(\left[ 0, 1\right]\).

    But we do have convergence in measure to \(0\): \(\alpha \in (0,1)\)
    \[
        \lambda\left(\left\{ \abs{f_{p(n,k)}} \geq \alpha \right\}\right) = \lambda \left(\left[ \frac{k-1}{n}, \frac{k}{n} \right]\right) = \frac{1}{n} \to 0 \mbox{ as } \begin{array}{l}
            n\to \infty \\
            \Updownarrow \\
            p \to \infty        
        \end{array}
    \]
\end{remark}
\begin{remark}
    So, \(f_p \nrightarrow 0\) a.e. on \([0,1]\). But consider \( \{ f_{p(n,1)}: n \in \mathbb{N} \} \). 
    This is a subsequence and, by definition 
    \[ 
        f_{p(n, 1)}(x) = \chi_{n, 1}(x)= \chi_{\left[0, \frac{0}{n} \right]}(x) 
    \] 
    For this subsequence, we have \( f_{p(n,1)}(x) \rightarrow 0 \) as \( n \to\infty \; \forall x \in (0, 1] \), then a.e. on \(\left[0, 1\right]\)
    
    This is not random!

\end{remark}

\begin{proposition}
    If \(\mu(X) < \infty \) and \(f_n \rightarrow f \) in measure, then \(\exists\) a subsequence \(\{f_{n_k} \}\) s.t. \(f_{n_k} \to f \) a.e. on \(X\).
\end{proposition}
Now we analize the relation between convergence in \(L^1(X)\) and the other convergences.

\begin{theorem}
    \( \{f_n\} \subset L^1(X), f \in L^1(X) \). If \(f_n \rightarrow f \) in \(L^1(X)\) then \(f_n \rightarrow f \) in measure on \(X\)
\end{theorem}
\begin{proof}
    By contradiction. Suppose that \(f_n \nrightarrow f \) in measure on X: 
    \( \exists \; \bar{\alpha} > 0 \) s.t. 
    \[ 
        \limsup_{n\to\infty} \mu(\{ |f_n-f| \geq \bar{\alpha} \}) > 0 
    \]
    \(\Rightarrow \exists \; \bar{\epsilon}\) and a subsequence \( \{ f_{n_k} \} \) s.t.
    \[ 
        \mu(\{ |f_{n_k}-f| \geq \bar{\alpha} \}) > \bar{\epsilon} 
    \]
    Consider then 
    \[  
        \begin{array}{l}
        d_1(f_{n_k}, f) 
        = \int_X |f_{n_k} - f| \, d\mu  
        \overset{\text{monot. } \int}{\geq} \int_{\left\{|f_{n_k}-f|\geq \bar{\alpha}  \right\}} \abs{f_{n_k} -f} \, d\mu  \geq\\
        \geq \int_{\left\{|f_{n_k}-f|\geq \bar{\alpha}  \right\}} 1 \, d\mu 
        = \bar{\alpha} \; \mu (\{|f_{n_k } - f| \geq \bar{\alpha}\}) 
        > \bar{\alpha} \; \bar{\epsilon}    
        \end{array}
    \]
    But, by assumption, \(d_1(f_n, f) \rightarrow 0\)
    \[ 
        \Rightarrow d_1(f_{n_k}, f) \rightarrow 0 
    \] 
    Contradiction.
\end{proof}

\begin{remark}
    The convergence in measure doesn't imply the convergence in \(L^1\). \\ For example, consider 
    \[ 
        f_n (x) = n \chi_{\left[0, \frac{1}{n} \right]}(x) 
    \]
    \( \underbrace{\mu \left( \left\{ |f_n| \geq \alpha \right\}\right)}_{= \frac{1}{n}} \to 0 \) for every \(\alpha\) \\
    On the other hand 
    \[ 
        \int _{\left[0, 1\right]} n \chi_{\left[0, \frac{1}{n} \right]} \, d\lambda 
        = \int_{\left[0, \frac{1}{n}\right]} n \, d\lambda 
        = n \frac{1}{n} = 1
    \]
    \( f_n \nrightarrow 0\) in \(L^1\) 
\end{remark}

Convergence a.e. \(\nRightarrow\) convergence in \(L^1\): \\
Use the same example above, \(f_n \rightarrow 0\) a.e. on \([0, 1] \nRightarrow f_n \rightarrow 0\) in \(L^1\)

Convergence in \(L^1\) \(\nRightarrow\) convergence a.e.: \\
Consider the typewriter sequence: \( d_1(f_{p(n, k)}, 0) \to 0\) when \( p \to\infty\) \\
But we don't have a.e. convergence. \\
However, recall the dominated convergence theorem: (DOM)
\[ 
    f_n \rightarrow f \text{ a.e. } + \exists \text{ of a dominating function } \Rightarrow d(f_n, f)\rightarrow 0 
\]
It is also possible to show a reverse DOM: \\
If \(f_n \to f \) in \(L^1(X)\), then \(\exists\) a subsequence \(\left\{f_{n_k}\right\}\) and \(w \in L^1(X)\) s.t. 
\begin{enumerate}
    \item \(f_{n_k} \rightarrow f\) a.e. on X
    \item \( \abs{f_{n_k}(x)} \leq w(x) \) for a.e. \(x \in X\)
\end{enumerate}

\subsection*{Derivatives of measures}
\(\left(X, \mathcal{M}, \mu \right)\) measure space,
\(\oldphi : X \to \left[0, \infty \right]\) measurable.  

We learned that \(\nu: \mathcal{M} \to \left[0, \infty \right]\) by 
\[
    \nu(E)= \int_E \oldphi \, d\mu \mbox{ is a measure on }(X, \mathcal{M})
\] 

If the equation above holds, then we say that \(\oldphi \) is the Radon Nykodym derivative of \(\nu\) with respect to \(\mu\) and we write 
\[
    \oldphi = \frac{d\nu}{d\mu}
\]
\begin{definition}
    \(\mu, \nu  \) measures on \(\left(X, \mathcal{M}\right)\). 
    We say that \(\nu\) is absolutely continuous with respect to \(\mu\), \(\nu << \mu \) if 
    \[
        \mu(E) = 0 \Rightarrow \nu(E)=0
    \]
\end{definition}

\begin{lemma}
    There is a necessary condition:
    \[ 
        \exists \; \frac{d \nu}{d \mu} \Rightarrow \nu << \mu 
    \]
\end{lemma}

\begin{proof}
    \[
        \nu(E) = \int_E \left(\frac{d\nu}{d\mu}\right) \, d\mu = 0
    \] 
    if \(\mu(E)=0\) by basic properties of \(\int\)
\end{proof}

\begin{theorem}[Radon Nykodim Theorem]
    \(\left(X, \mathcal{M}\right) \) measurable space, \(\mu, \nu\) measures. \\
    If \(\nu << \mu \) and moreover \(\mu \) is \(\sigma\)-finite, then \(\oldphi : X \to \left[0, \infty\right]\) measurable s.t.
    \[
        \oldphi = \frac{d \nu}{d \mu} \quad  \text{ namely } \nu(E)= \int_E \oldphi \, d\mu \quad \forall \; E \in \mathcal{M}
    \]
\end{theorem}

\begin{remark}
    If \(\mu\) is not sigma finite the theorem may fail. \\
    In \(\left(\left[0, 1\right], \mathcal{L}\left(\left[0, 1\right]\right)\right)\) consider the counting measure \(\mu = \mu_C\) and the Lebesgue measure \(\nu= \lambda\)
    \(\nu << \mu\) since \(\mu(E)= 0 \Leftrightarrow E= \emptyset \Rightarrow \lambda(E) = \nu(E)=0\) \\
    But we can check that \( \nexists \; \oldphi : \left[0, 1\right] \rightarrow \left[0, \infty \right]\) measurable s.t. \(\lambda(E)= \int_E \oldphi \, d\mu_C\)
\end{remark}

Check by contradiction: assume that \(\oldphi \) does exist, and take \(x_0 \in \left[0, 1\right]\)
\[ 
    0 = \lambda (\left\{x_0\right\}) = \int_{\left\{x_0\right\}} \oldphi \, d\mu_C = \oldphi (x_0) \; \overbrace{\mu_C (\left\{x_0\right\})}^{=1}= \oldphi (x_0)
\]
\(\Rightarrow \oldphi (x_0) = 0 \; \forall \; x_0 \in \left[0, 1\right]\).

But then \(1 = \lambda(\left[0, 1\right]) = \int_{\left[0, 1\right]} 0 \, d\mu_C = 0\). Contradiction \\
Note that \(\mu_C (\left[0, 1\right]) = \infty \) and \(\left( \left[0,1\right], \mathcal{L}(\left[0, 1\right]), \mu_C\right)\) is not \(\sigma\)-finite (\(\left[0,1\right]\) is uncountable)


\subsection*{Product Measure}
\( (X, \mathcal{M}, \mu), (Y, \mathcal{N}, \nu) \) measure spaces.
The goal is to define a measure space on \(X \times Y\)
\begin{definition}
    We call \textbf{measurable rectangle} in \(X \times Y\) a set of type \(A \times B\) where \(A \in \mathcal{M}, B \in \mathcal{N}\)
    \[  R = \{ A \times B \subset X\times Y \text{ s.t. } A \in \mathcal{M}, B \in \mathcal{N}\}\]
    We define the product \(\sigma\)-algebra \(\mathcal{M} \otimes \mathcal{N}\) as \(\sigma_0(R)\). \\
    This is a \(\sigma\)-algebra in \(X \times Y\)
\end{definition}

\begin{definition}
    Let \(E \subset X \times Y \). For \( \bar{x} \in X \) and \(\bar{y} \in Y \) we define
\[ 
    \begin{array}{ll}
        E_{\bar{x}} = \{ y \in Y: \left( \bar{x}, y \right) \in E \} \subseteq Y & \qquad \bar{x} \text{-section of } E \\
        E_{\bar{y}} = \{ x \in X: \left( x, \bar{y} \right) \in E \} \subseteq X & \qquad \bar{y} \mbox{-section of } E \\
    \end{array}
\]
\end{definition}

\begin{proposition}
    \(\left( X, \mathcal{M} \right), \left( Y, \mathcal{N} \right)\) measurable spaces. \(E \in \mathcal{M} \otimes \mathcal{N}\) \\
    Then \(E_x \in \mathcal{M} \) and \(E_y \in \mathcal{N} \) 
    \(\Rightarrow \) we can define 
    \[
    \begin{array}{rlrl}
        \phi : & X \rightarrow \left[ 0, \infty \right] & \qquad  \psi : &Y \rightarrow \left[ 0, \infty \right] \\
                & x \mapsto \nu(E_x) & & y \mapsto \mu(E_y) 
        
    \end{array}    
    \]
\end{proposition}


\begin{theorem}
    If \(\left(X, \mathcal{M}, \mu \right)\) and \(\left(Y, \mathcal{N}, \nu \right)\) are \(\sigma\) finite spaces, then:
    \begin{enumerate}
        \item \(\phi\) is \(\mathcal{M}\)-measurable and \(\psi\) is \( \mathcal{N}\)-measurable
        \item we have that \(\int_X \nu(E_x) \, d\mu = \int_Y \mu(E_y) \, d\nu \)
    \end{enumerate}
\end{theorem}

Using the fact that \(\mu \) and \(\nu\) are measures, and that \(\int\) of non negative function is a measure, we deduce the following

\begin{theorem}[Iterated integrals for characteristic functions]
    \(\mu \otimes \nu : \mathcal{M} \otimes \mathcal{N} \rightarrow \mathbb{R} \) defined by
    \[ \left(\mu \otimes \nu \right)(E) = \int_X \nu(E_x) \, d\mu = \int_Y \mu(E_y) \, d\nu\]
    is a measure, the product measure.
\end{theorem}

\begin{remark}[On the complection of product measure spaces] 
    \(\left(X, \mathcal{M}, \mu \right), \left(Y, \mathcal{N}, \nu \right)\) complete measures spaces. 
    
    In general it is not true that \((X \times Y, \mathcal{M} \otimes \mathcal{N}, \mu \otimes \nu)\) is complete.
\end{remark}
\noindent\underline{Example}: \(X = Y = \real\), \(\mathcal{M} = \mathcal{N} = \mathcal{L}(\real), \mu = \nu = \lambda\).

Given \(A \mbox{ non meas. set }, A \subseteq [0,1], B = \left\{ y_0 \right\}, E = A \times B\). 
If \(E\) were measurable, then its sections must be measurable. But \(E_{y_0} = A\) which is not measurable.

However, \(E\) is negligible:
\[
    E \subseteq [0,1] \times \left\{ y_0 \right\}, \mbox{ and } \left(\lambda \otimes \lambda\right)\left([0,1] \times \left\{ y_0 \right\}\right) = 0
\]
Then \((\lambda \otimes \lambda)\) is not complete 
\[
    \Rightarrow (\real^2, \mathcal{L}(\real) \otimes \mathcal{L}(\real), \lambda \otimes \lambda) \neq (\real^2, \mathcal{L}(\real^2), \lambda_2)
\]
\begin{theorem}
    Let \(\lambda_n\) be the Lebesgue measure in \(\mathbb{R}^n\). 
    If \(n= K+m\), then \(\left(\mathbb{R}^n, \mathcal{L}(\mathbb{R}^n), \lambda_n \right)\) is the complection of \( ( \real^k \times \real^m, \mathcal{L}(\real^k) \otimes \mathcal{L}(\real^m),\lambda_k \otimes \lambda_m )\)
\end{theorem}

\section{Lecture 13/10/2022}
\subsubsection*{Integration on product spaces}
\((X, \mathcal{M}, \mu), (Y, \mathcal{N},\nu)\) measure spaces. \(f : X \times Y \to \barreal\) measurable.

If \(f \geq 0\), then 
\[
    \iint_{X \times Y} f d\mu\otimes d\nu
\]
Goal: obtain a formula of iterated integral like the one in Analysis 2.

\(\forall \; \bar{x} \in X\) and \(\bar{y} \in Y\), we define
\[
    \begin{array}{llll}
        f_{\bar{x}} :& Y \to \barreal & f_{\bar{y}}:& X \to \barreal  \\
        & y \mapsto f(\bar{x}, y) & & x \mapsto f(x, \bar{y})

    \end{array}
\]
\begin{proposition}
    If \(f\) is measurable \(\Rightarrow\) \(f_{\bar{x}}\) is \((\mathcal{N}, \boreal)\)-measurable and \(f_{\bar{y}}\) is \((\mathcal{M}, \mathcal{B}(\barreal))\)-measurable.
    Then we can consider
    \[
        \begin{array}{ll}    
        \phi : X \to \barreal & 
        \phi(x) = \int_Y f_x d\nu = \int_Y f(x,y) \underbrace{d\nu(y)}_{dy} \\
        \psi : Y \to \barreal &
        \psi(y) = \int_X f_y d\mu = \int_X f(x,y) d\mu(x)
    \end{array}
    \]
\end{proposition}
\noindent\underline{Questions:} what is the solution of \(\iint_{X \times Y}\), \(\phi\) and \(\psi\)?

\begin{theorem}[Tonelli's theorem]
    \((X, \mathcal{M}, \mu)\) and \((Y, \mathcal{N}, \nu)\) complete measure spaces and \(\sigma\)-finite. \\
    Suppose that \(f\) is \((\mathcal{M} \otimes \mathcal{N}, \mathcal{B}(\barreal))\)-measurable and that \(f > 0\) a.e. on \(X \times Y\). Then \(\psi\) and \(\phi\) are measurable and
    \[
        \iint_{X \times Y} f d\mu \otimes d\nu = \int_X \phi(x) \, d\mu(x) = \int_Y \psi(y) \, d\nu(y) \tag*{Integration formula}
    \]
    Equally holds also if one of the integrals is \(\infty\).
    \[
        \begin{array}{l}
            \int_X \phi(x) \, d\mu(x) = \int_X \left(\int_Y f(x, y) \, d\nu(y) \right) \, d\mu(x) \\
            \int_Y \psi(y) \, d\nu(y) = \int_Y \left(\int_X f(x, y) \, d\mu(x) \right) \, d\nu(y)      
    \end{array}  
    \]
\end{theorem}
\begin{remark}
    The double integral can be reduced to single integrals, iterated. Moreover we can always change the order of the integrals.
    For sign changing functions the situation is more involved.
\end{remark}
\begin{theorem}[Fubini's theorem]
    \((X, \mathcal{M}, \mu)\) and \((Y, \mathcal{N}, \nu)\) complete measure spaces and \(\sigma\)-finite.
    If \(f \in L^1(X \times Y)\), then \(\psi\) and \(\phi\) defined above are measurable, the integration formula holds, and all the integrals are finite.
\end{theorem}
\noindent\underline{Question}: how to check if \(f\in L^1(X \times Y)\)? Typically, to check that \(f \in L^1(X \times Y)\) one uses Tonelli: 
\[
    f \in L^1(X \times Y) \Leftrightarrow \iint_{X \times Y} \abs{f} \, d\mu \otimes d\nu
\]
We use Tonelli to check that this is finite. 
If \(\iint_{X \times Y} \vert f \vert d\mu \otimes d\nu < \infty\) then we can apply Fubini for \(\iint_{X \times Y} f d\mu \otimes d\nu\)
\begin{remark}
    the proof of Fubini's and Tonelli's theorems is based for the iterated integrals for characteristic functions.
    Note that 
    \[(\mu \otimes \nu)(E) = \begin{array}{l}
        \int_X \phi(x) \, d\mu(x) = \int_X \left(\int_Y f(x, y) \, d\nu(y) \right) \, d\mu(x) \\
        \int_Y \psi(y) \, d\nu(y) = \int_Y \left(\int_X f(x, y) \, d\mu(x) \right) \, d\nu(y)
    \end{array}
    \]
\end{remark}
\begin{remark}
    Sometimes double integrals are very useful to compute single integrals.
\end{remark}
Ex: \(\int_{-\infty}^{+\infty}e^{-x^2} = \sqrt{\pi}\)




\section{Lecture 19/10/2022}
\subsection*{The first fundamental theorem of calculus}

Consider \(f \in L^1\left([a,b]\right)\). We can define the \textbf{integral function}
\[F(x) = \int_{[a,b]} f d\lambda = \int_a^b f(t)dt , \quad x \in [a,b]\]

If \(f \in \mathcal{C}\left(\left[a, b\right]\right)\), then \(F\) is differentiable on \(\left[a, b\right]\), and \(F'(x)=f(x)\)

What happens if \(f \in L^1([a, b])\)?



\begin{definition}
    Given \(f \in L^1([a,b])\). We say that \(x \in [a,b]\) is a \textbf{Lebesgue point} for \(f\) if 
    \[
        \lim_{h \to 0} \frac{1}{h} \int_x^{x+h} \abs{f(t) - f(x)} \, dt = 0
    \]
    If \(x=a\) or \(x=b\), this is the left/right \(\lim\).
\end{definition}
\begin{remark}
    A point \(x\) is called a Lebesgue point for \(f\) if \(f\) `does not oscillate too much' close to \(x\):
    \begin{itemize}
        \item \(f\) \(\mathcal{C}([a,b]) \to \text{ every } x \in [a,b]\) is a Lebesgue point.
        \item \[
            f(x) = \begin{cases}
                1  & x > 0 \\
            0 & x < 0
            \end{cases}
        \]
        \[
            \lim_{h \to 0} \frac{1}{h} \int_0^{h} \abs{f(t) - f(0)} \, dt = \lim_{h \to 0} \frac{1}{\abs{h}} \int_0^{h} \abs{0 - 1} \, dt = 0
        \]
    \end{itemize}
\end{remark}
\begin{theorem}[Lebesgue]
    If \(f \in L^1([a.b])\) then a.e. \(x \in [a,b]\) is a Lebesgue point for \(f\)
\end{theorem}
\begin{remark}
    In the definition of Lebesgue point, the pointwise values of \(f\) are relevant 
    \[
        f = g \in L^1 \Leftrightarrow f = g \text{ a.e.}
    \]
    Then the Lebesgue point of \(f\) could be different from the one of \(g\).  
    This is not a big problem if \(f = g\) a.e. on \([a,b] \Rightarrow f = g \in [a,b]\setminus N\) where \(\lambda(N) = 0\); \(x\) is a Lebesgue point for \(f\), \(\forall \; x \in \left[a, b\right] \setminus M, \ \lambda(M)=0\) \\
    \(\Rightarrow x \) is a Lebesgue point for \(g\), \(\ \forall x \in \left[a, b\right] \setminus (M \cup N)\) \\
    \(\left[a, b\right] \setminus (M \cup N)\) is a set of full measure of Lebesgue points for \(f\) and \(g\).
\end{remark}
To speak about Lebesgue points, one has to choose a specific representative \(f \in L^1([a,b])\). If you change representative, you obtain the same set of Lebesgue points up to sets with \(0\)-measure.

\begin{theorem}[First fundamental theorem of calculus]
Given \(f \in L^1([a,b]),\ F(x) = \int_a^xf(t) \, dt\) \\
Then \(f\) is differentiable a.e. on \([a,b]\) and \(F'(x) = f(x) \text{ a.e. in } [a,b]\)    
\end{theorem}
\begin{proof}
    Let \(x \in [a,b]\) for any Lebesgue point for \(f\) (a.e. \(x \in [a,b]\) is fine). Consider
    \[
        \left|{\frac{F(x+h)-F(x)}{h} - f(x)}\right| = \abs{\frac{1}{h} \int_x^{x+h} (f(t) - f(x)) \,dt } \leq \frac{1}{h} \int_x^{x+h} |f(t) - f(x)| \,dt \to 0 
    \]
    Since \(x\) is a Lebesgue point.
\end{proof}
\begin{definition}
    Given \(f : I \to \real\) is called \textbf{absolutely continuous} in \(I\), \(f \in AC(I)\), if \(\forall \; \epsilon >0 \; \exists \; \delta >0\)  
    s.t. 
    \[
        \bigcup_{k=1}^n [a_k, b_k] \in I \text{ with disjoint interiors}
    \] 
    \[
        \lambda(\bigcup_{k=1}^n [a_k, b_k]) = \sum_{k=1}^n (b_k -a_k) < \delta
    \]
    \[
        \Rightarrow \sum_{k=1}^n \abs{f(b_k)-f(a_k)} < \epsilon
    \]
\end{definition}
\begin{remark}
    \(f\) is uniformly continuous on \([a,b]\) if \(\forall \; \epsilon > 0\) \(\exists \; \delta > 0\) s.t. 
    \[
        \abs{t - \tau} < \delta \Rightarrow \abs{f(t)-f(\tau)} < \epsilon
    \]
    An absolutely continuous function is also uniformly continuous. \\
    But the converse is false.
\end{remark}

\begin{itemize}
    \item If \(f\) is Lipschitz on \([a,b] \Rightarrow f \in \text{AC}([a,b])\) 
\end{itemize}
Recall that \(f \in \mbox{Lip}([a,b])\) if \(\exists \; L > 0\) s.t. 
\[
    \abs{f(x) - f(y)} \leq L\abs{x-y} \qquad \forall x, y \in \left[a, b\right]
\]
\noindent\underline{Check}: For any \(\epsilon > 0\), and consider 
\[
    \sum_{k=1}^n \abs{f(b_k)-f(a_k)} \leq \sum_{k=1}^n L(b_k-a_k) = L \sum_{k=1}^n (b_k-a_k)
\]
If we take \(\delta = \delta(\epsilon) = \frac{\epsilon}{L}\), then 
\[
    \sum_{k=1}^n (b_x, a_x) < \delta \Rightarrow \sum_{k=1}^n \abs{f(b_k)-f(a_k) } \leq L \sum_{k=1}^n (b_k-a_k)
\]
\qed
\[
    \mbox{Lip}([a,b]) \subsetneqq \mbox{AC}([a,b]) \subsetneqq \mbox{UC}([a,b])
\]
\begin{theorem}[Regularity of integral functions]
    Given \(f \in L^1([a,b]), F(x) = \int_a^xf(t) \, dt\), then \(F \in \mbox{AC}([a,b])\)
\end{theorem}
To prove the theorem we need the
\begin{theorem}[Absolute continuity of the integral]
    Given \(f \in L^1(X, \mathcal{M}, \mu)\). Then \(\forall \; \epsilon >0 \) \(\exists \; \delta > 0\) s.t.     
    \[
        \begin{array}{l}
            E \in \mathcal{M} \\
            \mu(E) < \delta
        \end{array} 
        \Rightarrow \int_E \abs{f} \, d\mu < \epsilon
    \]
\end{theorem}
\begin{proof}
    We fix \(\epsilon > 0\). Let \(F_n := \left\{ \abs{f} < n \right\}\), \(n \in \mathbb{N}\). Also \(F_n \in \mathcal{M}\; \forall \; n\), \(F_n \subseteq F_{n+1}\) and 
    \[
        \bigcup_{n=1}^{\infty} F_n = \left\{ \abs{f} < \infty \right\} =: F
    \]
    \(f \in L^1 \Rightarrow \abs{f}\) is finite a.e.: \(\mu(X \setminus F) =0\).
    Therefore:
    \[
        \int_X \abs{f} \, d\mu = \int_{X \setminus F} \abs{f} \, d\mu + \int_F \abs{f} \, d\mu 
        = \lim_{n \to\infty} \int_{F_n} \abs{f} \, d\mu
    \]
    \[
        \lim_{n\to\infty} \int_X \abs{f} \left(\- \chi_{F_n} \right) \, d\mu =0
    \]
    \(\forall \; \epsilon >0 \; \exists \; \bar{n} \in \mathbb{N}\) s.t. 
    \[
        n > \bar{n} \Rightarrow \abs{\int_X \abs{f} \chi_{F_n^C} \, d\mu} < \frac{\epsilon}{2}
    \]
    Now, fix \(\epsilon >0 \), and take \(n > \bar{n}(\epsilon)\). If \(E \in \mathcal{M}\), then
    \[
        \int_E \abs{f} \, d\mu = \int_{E \cap F_n} \abs{f} \, d\mu + \int_{E \cap F_n^C} \abs{f} \, d\mu
        \leq n \int_E 1 \, d\mu + \int_{F_n^C} \abs{f} \, d\mu 
    \]
    If we suppose that \(\mu(E) < \frac{\epsilon}{2n} =: \delta(\epsilon)\), we deduce that
    \[
        n \int_E 1 \, d\mu = n \mu(E) < \frac{\epsilon}{2}
    \]
    Also, since \(n > \bar{n}\)
    \[
        \int_{F_n^C} \abs{f} \, d\mu < \frac{\epsilon}{2}
    \]
    \[
        \Rightarrow \int_E \abs{f} \, d\mu < \epsilon \]
\end{proof}
\begin{proof}[Proof - Regularity of integral functions]
    Let \(\epsilon > 0\), and \(\delta = \delta(\epsilon) > 0\) be the value given by the absolute continuity of \(\int \abs{f} \, d\mu\). 
    Take 
    \[
        E = \bigcup_{k=1}^n \left[ a_k, b_k \right] \qquad E \subseteq \left[a, b\right]
    \]
    If \(\lambda(E) < \delta\), then
    \[
        \sum_{k=1}^n \abs{F(b_k) - F(a_k)} = \sum_{k=1}^n \abs{\int_{a_k}^{b_k} f(t)\, dt} 
        \leq \sum_{k=1}^n \int_{a_k}^{b_k} \abs{f(t)}\, dt
        = \int_E \abs{f} \, d\lambda < \epsilon
    \]
    by absolute continuity of \(\int\) 
\end{proof}
\begin{remark}
    \(\sqrt{x}\) is AC\((\left[0, 1\right])\), but is not Lip\((\left[0, 1\right])\).
    \[
        \sqrt{x} = \int_0^x \frac{1}{2\sqrt{t}} \, dt
    \]
    \(\Rightarrow \sqrt{x}\) is the \(\int\) function of a \(L^1\) function \\
    \(\Rightarrow \sqrt{x} \in \text{AC}(\left[0, 1\right])\)
\end{remark}

To sum up: the \(\int \) function of a \(L^1\) function is AC, it is differentiable a.e., and 
\[
    F(x) - F(a) = \int_a^x F'(t) \, dt \tag*{FC}
\]

Suppose \(G\) is differentiable a.e. on \(\left[a, b\right]\) and FC holds for \(G\):
\[
    G(x) - G(a) = \int_a^x G'(t) \, dt
\]
What can we say about \(G\)?
\begin{remark}
    If \(G\) \(\in \mathcal{C}^1(\left[a, b\right]) \Rightarrow\) FC holds. \\
    If FC holds, then \(G' \in L^1(\left[a, b\right]) \) (necessary condition).
    \noindent
    Is the necessary condition also sufficient? In general not. 
    Take \(v(x)\), the Vital Cantor function: \(v \in \mathcal{C}([0,1]), v(0)=0, v(1)=1\). \(v\) is differentiable a.e. on \([0,1]\) but the calculus formula doesn't hold!
\end{remark}
\begin{remark}
    A function which is differentiable a.e. on an interval can behave very badly
\end{remark}
\begin{theorem}
    \(G \in \text{AC}([a, b])\). Then \(G\) is differentiable a.e. on \([a, b]\), \(G' \in L^1([a, b])\), and FC holds.
\end{theorem}
\begin{remark}
    These theorems say that AC function are precisely the ones for which FC holds:
    \begin{itemize}
        \item \(G \in \) AC \(\Rightarrow\) FC holds. 
        \item If FC holds, then \(G' \in L^1 ([a, b])\) 
    \end{itemize}
    \(\Rightarrow \int_a^x G'(t) \, dt \in \) AC \\
    \(\Rightarrow G(x) - G(a) = \int_a^x G'(t) \, dt \in \) AC   
\end{remark}
\begin{remark}
    \(v \in \) UC\(([0, 1])\) by continuity and Heine Cantor, but \(v \notin \) AC\(([0, 1])\) because FC does not hold.
\end{remark}

The proof of the second fundamental theorem of calculus is divided into two steps. 
\begin{lemma}
    The second fundamental theorem hold under the additional assumption that \(G\) is monotone.
\end{lemma}
Second step: to get rid of the monotonicity. \\
For step 2, is it useful to give the
\begin{definition}
    \(\left[a, b\right] \subset \mathbb{R}\). Let 
    \[
        \mathcal{P}_{\left[a, b\right]} := \lbrace (x_0, x_1, \dots, x_n): n \in \mathbb{N} \text{ and } a=x_0 < x_1 < x_2 < \dots < x_n = b \rbrace
    \]
    For \(P \in \mathcal{P}_{\left[a, b\right]}\) and \(f: [a, b] \to \barreal\), define
    \[
        v_a^b(f, P) := \sum_{k=1}^n \abs{f(x_k) - f(x_{k-1})}
    \]
    The total variation of \(f \) on \([a, b]\) is 
    \[
        V_a^b (f) := \sup_{P \in \mathcal{P}_{\left[a, b\right]}} v_a^b (f, P)
        = \sup \lbrace \sum_{k=1}^n \abs{f(x_k)-f(x_{k-1})}:n \in \mathbb{N}, a=x_0 < x_1 < \dots <x_n = b \rbrace
    \]
\end{definition}
If \(V_a^b(f) < \infty\), we say that \(f\) is a function with \textbf{bounded variation}, \(f \in \) BV \(([a, b])\)
\section{Lecture 20/10/2022}
\begin{theorem}[The \(2^{nd}\) fundamental theorem of calculus.]
    \(G \in \mbox{AC}([a,b]) \Leftrightarrow\) \(G\) is differentiable a.e. on \([a,b]\), \(G' \in L^1([a,b])\), and (FC) holds.
\end{theorem}

Example and comments:
\begin{itemize}
    \item If \(f\) is bounded and monotone \(\Rightarrow\) \(f \in \mbox{BV}\)
    \[
        V_a^b (f) = \abs{f(b) - f(a)}
    \]
    Note that \(f\) may not be continuous
    \[
        f(x) = \begin{cases}
            1 & x \geq 0 \\
            0 & x < 0
        \end{cases}
        \Rightarrow f \in \mbox{BV}([-1,1])
    \]
    \item \(f \in \mbox{BV}([a,b]) \Rightarrow f\) is bounded. Indeed
    \[
        \sup_{x \in [a,b]} \abs{f(x)} \leq \abs{f(x)} + V_a^b(f) \overset{f \in \text{BV}}{<} +\infty
    \]
    \item \(f\) is continuous on \([a,b]\), or even if  \(f\) is differentiable everywhere in \([a,b]\) \(\nRightarrow f \in \mbox{BV}([a,b])\)
    \[
        f(x) = \begin{cases}
            x^2\cos{\frac{2\pi}{x^2}} & x \in (0,1] \\
            0 & x = 0
        \end{cases}
    \]
    It is continuous in \([0,1]\), but \(f \notin \mbox{BV}([0,1])\)
    \item \(f \in \mbox{BV}([a,b]) \cap \mbox{UC}([a,b])  \nRightarrow f \in \mbox{AC}([a,b])\)
    \[\begin{array}{l}
        v \mbox{ a Vitali-Cantor function} \\ v \mbox{ is bounded and monotone} \\ 
        v \in \mbox{UC}([0,1]) \\ 
        
    \end{array}
    \Rightarrow v \in \mbox{BV}([0,1]) 
    \]
    But \(v \not \in \mbox{AC}([0,1])\)
    \item If \(f \in \mbox{BV}([a,b]) \Rightarrow f\) is differentiable a.e. on \([a,b]\), and \(f' \in L^1([a,b])\)
\end{itemize}
We can now come back to the proof of Lemma 1 of the last lesson.

\noindent\underline{Preliminary result}: \(A \in \real\) open. Then 
\[
    A = \bigcup_{n=1}^{\infty}(a_n, b_n) \mbox{ disjoint}
\]
    any open set of \(\real\) is the (at most) countable union of open disjoint intervals.

\noindent\underline{Preliminary result (equivalent definition for AC)}: \(f \in \mbox{AC}([a,b]) \Leftrightarrow \forall \; \epsilon > 0 \exists \; \delta > 0\) depending on \(\epsilon\) s.t. 
\[
    \forall \; \bigcup_{n=1}^{\infty} [a_n, b_n], \quad [a_n, b_n] \mbox{ have disjoint interiors}
\]
\[
    \sum_{n=1}^{\infty} (b_n -a_n) < \delta \Rightarrow \sum_{n=1}^{\infty} \abs{f(b_n) - f(a_n)} < \epsilon
\]
\begin{proof}
    We defined \(\lambda\) starting from two properties
    \begin{itemize}
        \item invariance under translations
        \item \(\lambda((x,y)) = y - x\) \(\forall \; a \leq < y \leq b\)
    \end{itemize}
    Now, \(G\) is monotone, say \(G\) increasing (if \(G \searrow\), take \(-G\)). We can repeat the construction of \(\lambda\) in order to obtain a measure \(\mu\) s.t. 
    \begin{itemize}
        \item \(\mu\) is invariant under translations
        \item \(\mu((x,y)) = \underbrace{G(y) - G(x)}_{\geq 0}\) \(\forall \; a \leq x < y \leq b\) (for \(\lambda\), take \(G(t) = t\))
    \end{itemize}
It can be proved that we obtain a measure on \((\real, \mathcal{L}(\real))\), complete.

On \((\real, \mathcal{L}(\real))\) we have two measures: \(\lambda \mbox{ and } \mu\).

\noindent\underline{Idea}: We take these measures on \(([a,b], \mathcal{L}([a,b]))\), and we want to show that \(\exists \; \frac{d\mu}{d\Lambda}\) (Radon-Nikodym)

We can check the hypothesis of the Radon-Nikodym theorem:
\begin{itemize}
    \item \(\lambda\) is \(\sigma\)-finite: \(\lambda([a,b]) = b-a < +\infty\)
    \item \(\mu << \lambda\): \(E \in \mathcal{L}([a,b])\), \(\lambda(E) = 0 \Rightarrow \mu(E) = 0\)
\end{itemize}
Assume \(\lambda(E) = 0\). \(G\) is \(\mbox{AC}([a,b])\): then \(\forall \; \epsilon > 0\) \(\exists \; \delta = \delta(\epsilon) > 0\) s.t. 
\[
    \forall \; \bigcup_{n=1}^{\infty} [a_n, b_n], \quad [a_n, b_n] \mbox{ have disjoint interiors}
\]
\[
    \lambda\left(\bigcup_{n=1}^{\infty} [a_n, b_n]\right) < \delta \Rightarrow \sum_{n=1}^{\infty} \abs{G(b_n) - G(a_n)} < \epsilon
\]
Take this \(\delta\). By regularity of \(\lambda\), \(\exists \; A \) open set of \(\left[a, b\right]\)s.t. \(A \supset E\) and \(\lambda(A) < \delta\)
\[
    A \mbox{ is open} \Rightarrow A = \left( \bigcup_{n=1}^{\infty} I_n \symbolfootnotemark[2]{\mbox{open intervals } = (x, y)} \right), \mbox{ disjoint}
    \symbolfootnotetext[2]{open intervals = \((x_n, y_n)\)}
\] 
it is a countable union of open intervals (maybe two of them contains \(a\) or \(b\))
\[
    \lambda(A) < \delta \Leftrightarrow \sum_{n=1}^{\infty} (y_n - x_n) < \delta
\]
But then, since \(\mu\) is a measure it is countably additive
\[
    \mu (E) \leq \mu(A) = \sum_n \mu (I_n) = \sum_n G(y_n) - G(x_n) < \epsilon
\]
by the choice of \(\delta \) and the fact that \(G \in\) AC.
We proved that 
\[
    \lambda(E) = 0 \Rightarrow \forall \; \epsilon >0 : \, \mu(E) < \epsilon \Rightarrow \mu(E) =0
\]
So \(\mu << \lambda\). We can apply Radon Nikodym \(\exists \; \oldphi : \left[ a, b\right] \to \left[0, \infty\right]\) s.t. 
\[
    G(x) - G(a) = \int_a^x \oldphi \, d\lambda
\]
Since \(G\) is bounded, then \(\oldphi \in L^1(\left[a, b\right])\)
\[
    G(x)=G(a) + \int_a^x \oldphi(t) \, dt
\]
By the first fundamental theorem of calculus, this is differentiable a.e. 
\[
    \Rightarrow G'(x) = \oldphi(x) \text{ a.e. on } \left[a, b\right]
\]
\[
    \Rightarrow G'(x) = G(a) + \int_a^x G'(t) \, dt
\]
\end{proof}
Now we want to get rid of the additional assumption (monotonicity).

\noindent\underline{Preliminary result}: \(f \in \mbox{BV}([a,b])\). Then
\[
    \phi(x) = V_a^x(f), \quad \forall \, x \in [a,b] 
\]
is an increasing function.
\begin{proof}
    By \(a \leq x < y \leq b\). Then
    \[
        V_a^y (f) = V_a^x (f) + \underbrace{V_x^y(f)}_{\geq 0} \geq V_a^x(f)
    \]
\end{proof}
\noindent\underline{Preliminary result}: If \(G \in \mbox{AC}([a,b])\), then \(G \in \mbox{BV}([a,b])\), and moreover 
\[
    \phi(x) = V_a^x(G) \mbox{ is in AC}([a,b])
\]
\begin{proof}[Proof of the second fundamental theorem of calculus in the general case]\(G \in \mbox{AC}([a,b])\)


    We want to write \(G= G_1 + G_2\) where \(G_1 \nearrow \) and \(G_2 \searrow\), both AC. 

    Then the second fundamental theorem holds for \(G_1 \) and \(G_2\) so it holds for \(G\) by linearity of the integral. 
    
    We pose:
    \[
        G_1(x) = \frac{G(x) + V_a^x(G)}{2}
    \]
    \[
        G_2(x) = \frac{G(x) - V_a^x(G)}{2}
    \]
    Clearly, \(G_1+G_2 = G\), \(G_1, G_2\) are AC, by the last preliminary result.

    \noindent\underline{\(G_1 \nearrow\)}: Let \(a \leq x < y \leq b\)
    \[
        \abs{G(y) - G(x)} \leq V_x^y(G)
    \]
    Therefore, 
    \[
        G_1 (y) - G_1(x) = \frac{1}{2} (\underbrace{G(y) - G(x)}_{\begin{array}{l}\geq - \abs{G(y) - G(x)} \\ \geq - V_x^y(G)\end{array}} + V_a^y(G) + V_a^x(G))  \geq \frac{1}{2}(-V_x^y(G) + V_x^y(G)) = 0
    \]
    
       
    
    So \(G_1\) is decreasing. In an analogue way, we can prove that \(G_2\) is decreasing.
\end{proof}

\section*{Functional analysis}
Normed spaces and Banach spaces
\begin{definition}
    Given \(X\) vector space, a norm on \(X\) is a function \(\norm{\cdot} : X \to [0, \infty)\) s.t.
    \begin{itemize}
        \item \(\norm{x} = 0 \Leftrightarrow x = 0\)
        \item \(\forall \; \alpha \in \real, \forall \; x \in X\) : 
        \[
            \norm{\alpha x} = \abs{\alpha}\norm{x}  \tag*{(positive homogeneity)}
        \]
        \item \(\forall \; x,y \in X\): 
        \[
            \norm{x+y} \leq \norm{x} + \norm{y} \tag*{(triangle inequality)}
        \]
    \end{itemize}
    Then, \((X, \normdot)\) is called a \textbf{normed space}
\end{definition}
\noindent\underline{Ex}: \(\abs{\norm{x} - \norm{y}} \leq \norm{x - y}\) \(\forall \; x,y \in X\)
\begin{proposition}
    \((X, \normdot)\) normed space. Then \((X, d)\) is a metric space for 
    \[
        d(x,y) = \norm{x - y}
    \]
\end{proposition}
\begin{remark}
    Normed space \(\underset{\displaystyle\nleftarrow}{\rightarrow}\) metric space
\end{remark}
\noindent\underline{Examples}:

\begin{itemize}
    \item \(\real^N, \norm{x}_p := (\sum_{i = 1}^N \abs{x_i}^p)^{\frac{1}{p}}\) \(\forall p \in [1, +\infty)\)
    \[
        \norm{x}_{\infty} := \max_{i = 1, \ldots, N} \abs{x_i}
    \]
    \item \(\mathcal{C}^0([a,b])\) \[\norm{f}_{\infty} := \max_{x \in [a,b]} \abs{f(x)}\]
    \item \(L^1(X, \mathcal{M}, \mu)\) \[\norm{f}_1 := \int_X \abs{f} \, d\mu\] This is a norm in \(L^1\), but not on \(\mathcal{L}^1\) \((\int_x \abs{f} \, d\mu = 0 \Rightarrow f = 0 \mbox{ a.e.})\)
    \item \(L^{\infty}(X, \mathcal{M}, \mu)\) \[\norm{f}_{\infty} := \underset{[a,b]}{\esssup} \abs{f}\]
\end{itemize}
\((X, \normdot)\) normed space \(\to\) \((X, d)\) metric space \(\to\) convergent sequences on \(X\): \(\left\{ x_n \right\} \subset X\) is convergent in \(X\) iff 
\[
    d(x_n, x) \to 0 \Leftrightarrow \norm{x_n -x} \to 0 \mbox{ as } n \to \infty
\]
\noindent\underline{Ex}: \(x_n \to x\) in \(X\), then \(\norm{x_n} \to \norm{x}\) (the norm is a continuous function on \(X\))
\begin{definition}
    \(\left\{ x_n \right\}\) is a \textbf{Cauchy sequence} in \((X, \normdot)\) if \(\forall \; \epsilon > 0\) \(\exists \; \bar{n} \in \mathbb{N}\) s.t. 
    \[
        n,m > \bar{n} \Rightarrow \norm{x_n-x_m} < \epsilon
    \]
\end{definition}
\begin{definition}
    \((X, \normdot)\) is called a \textbf{Banach space} if \((X, d)\) is complete, namely if any Cauchy sequence in \((X, d)\) is convergent.
\end{definition}
If \((X, \normdot)\) is a normed space, we can speak about series in \(X\). Let \(\left\{ x_n \right\} \subset X\) and \(s_n = x_0 + x_1 + \ldots + x_n\), then \(\sum_{n=0}^{+\infty} x_n = \left\{ s_n \right\}\). 

Then \(\sum x_n\) is convergent if \(\left\{ s_n \right\}\) is convergent. If \(\sum x_n\) is convergent, we write 
\[
    s = \sum_{n = 0}^{+\infty} x_n \Leftrightarrow s_n \to s
\]
For numerical series
\[
    \sum_{n=1}^{\infty} \abs{a_n} < +\infty \Rightarrow \sum a_n \mbox{ is convergent}
\]
In general, in normed spaces 
\[
    \sum_{n=1}^{\infty} \norm{x_n} < +\infty \nRightarrow \sum_{n=1}^{\infty} x_n \mbox{ is convergent}
\]
\subsubsection*{Characterization}
\((X, \normdot)\) is a Banach space \(\Leftrightarrow\) every series s.t. \(\sum \norm{x_n} < +\infty\) is also s.t. \(\sum x_n\) is convergent
\section{Lecture 26/10/2022}
\((X, \norm{\cdot}) \to (X, d) \to \) open sets, closed sets, bounded sets....

In \(\real^n\) we are used to work with \(\norm{\cdot}_2\), but we could have many different norms.
\begin{definition}
    Let \(\normdot\) and \(\normdot_2\) be two norms on the same vector space \(X\). We say that these norms are \textbf{equivalent} if \(\exists \; m, M >0\) s.t. 
    \[
        m\norm{x} \leq \norm{x}_2 \leq M\norm{x} \quad \forall\; x \in X
    \]
\end{definition}
It can be proved that if two norms are equivalent they lead to different metric spaces, but to the same open sets, closed sets, convergent sequences, compact sets \dots
\begin{theorem}
    If \(X\) is any finite dimension vector space, then all the norms on \(X\) are equivalent.
\end{theorem}
\begin{remark}
    This is why in \(\real^n\) usually one does not specify the choice of the norm. One choose the Euclidean norm, since it comes from a scalar product. (ref. Hilbert spaces)
\end{remark}
\noindent\underline{Preliminary fact}: The set \(S_1 = \left\{ s \in \real^n : \norm{s}_1  = 1\right\}\) is compact in \((\real^n, d)\)
\begin{proof}
    We show that any norm is equivalent to \(\normdot_1 = \sum_{i=1}^n \abs{x_i}\)
    \[
        x = \sum_{i=1}^n x_i e_i \qquad \left\{ e_i \right\}_{i= 1,\ldots, n} \mbox{ canonical basis}
    \]
    Let's introduce the norm star 
    \[
        \norm{x}_* = \norm{\sum_{i=1}^n x_i e_i}_* 
        \leq \sum_{i=1}^n \norm{x_i e_i}_* = \sum_{i=1}^n \abs{x_i} \norm{e_i}_* 
        \leq \left(\max_{1 \leq i \leq n} \norm{e_i}_*\right) \sum_{i=1}^n \abs{x_i}
        = M \norm{x}_1
    \]
    We proved that \(\exists \; M> 0\) s.t.
    \[
        \norm{x}_* \leq M \norm{x}_1 \quad \forall \; x \in X \tag{1}
    \]
    Note that this proves that \(\phi(x) = \norm{x}_*\) is continuous in (\(X, d\)). Indeed 
    \[
        x_n \to x \Leftrightarrow d_1(x_n, x) \to 0
    \]
    then 
    \[
        \abs{\phi(x_n) - \phi(x)} = \abs{\norm{x_n}_* -\norm{x}} \leq \norm{x_n - x}_* 
        \overset{(1)}{\leq} M\norm{x_n - x}_1 \to 0
    \]
    Therefore, by the Weierstrass theorem, \(\exists\) a minimum point \(x_0 \in S_1\) s.t. 
    \[
        \phi(x) \geq \phi(x_0) = m \quad \forall\; x \in S_1
    \]
    (recall that \(S_1\) is compact)
    \[
        \norm{x}_* \geq m \quad \forall \; x \in S_1
    \]
    We claim that \(m>0\). If \(m=0\) then \(\norm{x_0}_* = 0 \Rightarrow x_0  = 0\) that is impossible, since \(x_0 \in S_1\).

    Thus \(m>0\). Let now \(y \in \real^n, y \neq 0\). Then 
    \[
        \frac{y}{\norm{y}_1} \in S_1 
        \Rightarrow \norm{\frac{y}{\norm{y}_1}}_* \geq m 
        \Rightarrow \frac{1}{\norm{y}_1} \norm{y}_* \geq m 
        \Rightarrow \norm{y}_* \geq m \norm{y}_1 \quad \forall \; y \in \real^n 
    \]
\end{proof}
If \(\dim X = +\infty\), then there are many non-equivalent norms.

\noindent\underline{Ex}: In \(\mathcal{C}^0([a,b])\), we can define \(\normdot_{\infty}\) and \(\norm{f}_1 = \int_a^b\abs{f(t)} \, dt\).

This is a norm in \(\mathcal{C}^0\), but these norms are not equivalent. 


\subsection*{Separability}
\((X, d)\) metric space. 
\begin{definition}
    We say that \(X\) is separable if \(\exists \; A \subset X\) which is dense (\(\bar{A} = X\)) and countable 
\end{definition}
In \(\real^n\), \(\mathbb{Q}^n\) which is dense and countable. In \(\infty-\dim\) we can have separable spaces or not. 

For instance, \((L^{\infty}, \normdot_{\infty})\) is not separable. Instead \((\mathcal{C}^0([a,b]), \normdot_{\infty})\) is a separable space. 

\begin{proof}[Sketch of the proof]
    We will use the \textbf{Stone-Weierstrass theorem}.

    The set of polynomials is dense on \(\mathcal{C}^0([a,b])\) and is an uncountable set. 
    However it can be proved that the set of polynomials with coefficients in \(\mathbb{Q}\) is dense in the set of all polynomials

    Moreover this set is countable. Then, by Stone-Weierstrass this is a countable dense set in \(\mathcal{C}^0([a,b])\)
\end{proof}
\begin{remark}
    One can show that \(\mathcal{C}^0(K)\) is separable whenever \(K\) is a compact set of a metric space \((X,d)\)
\end{remark}

\subsection*{Compactness}
In finite dimension (in \(\real^n\)), one has that
\[
    E \subset X \mbox{ is compact } \Leftrightarrow E \mbox{ is closed and bounded}
\]
If \(\dim X = \infty\), then only `\(\Rightarrow\)' is true. In finite dimension, we know that the closed unit ball is compact
\[
    \bar{B}_1(0) = \left\{ x \in \real^n : \norm{x} \leq 1\right\}
\]
What happens now if \((X, \normdot)\) is on \(\infty-\dim\) normed space?
\begin{theorem}[Riesz's theorem]
    \(X\) normed space, \(\dim X = +\infty\) \(\Rightarrow \overline{B_1(0)}\) is not compact
\end{theorem}
\begin{remark}
    It is well known that if \(E \in \real^n\) is compact, then \(\forall \; \left\{ x_n \right\} \in E \) \(\exists \; \left\{ x_{n_k} \right\}\) subsequence s.t. \(x_{n_k} \to x \in E\).  
    This proposition is much harder to prove in \(\infty-\dim\).
\end{remark}
The proof of the Riesz's theorem is based on the Riesz's \textbf{quasi-orthogonality lemma}.
\begin{lemma}[Riesz Quasi-Orthogonality Lemma]
    Let \(X\) be a normed space, \(E \subsetneq X\) a closed subspace. 
    Then \(\forall \; \epsilon \in (0,1)\) \(\exists \; x \in X\) s.t.
    \[
        \norm{x}=1 \text{ and dist}(x, E) = \inf_{y \in E} \norm{x-y} \geq 1- \epsilon
    \]
\end{lemma}
\begin{proof}
    Of the Riesz's theorem. Assume that \(\overline{B_1(0)}\) is compact, and \(X \) has infinite dimension. 

    \(\exists\) a sequence \(\{E_n\}\) of finite dimensional subspaces (hence closed) of \(X\) s.t. 
    \[
        E_{n-1} \subset E_n \text{ and } E_{n-1} \neq E_n
    \]
    \(E_{n-1}\) is a proper closed subspace of \(E_n\) \(\forall\; n\)

    We can apply the Riesz Lemma with \(X = E_n\), \(E=E_{n-1}\), \(\epsilon=\frac{1}{2}\). 
    Then \(\forall \; n\) \(\exists \; u_n \in E_n  \) s.t. \(\norm{u_n}=1\) and dist\((u_n, E_{n-1}) \geq \frac{1}{2} \quad \forall\; n\)
    
    Therefore, we have a sequence \(\{u_n\}\) with the following properties 
    \[
        \norm{u_n}=1 \qquad \forall\; n
    \]
    \[
        \norm{u_n-u_m} \geq \frac{1}{2} \qquad \forall\; n \neq m
    \]
    \(\Rightarrow\) this sequence cannot have any convergent subsequence. 
    But then \(\overline{B_1(0)} \supseteq \{u_n\}\), this implies that \(\overline{B_1(0)}\) is not compact.
    Contradiction.

    (In any \(\left(X, \normdot\right)\) normed space, if \(E\) is compact, 
    then \(\forall\; \{x_n\} \subset E \) \(\exists \; \{x_{n_k}\}\) s.t. \(x_{n_k} \rightarrow x \in E\))
\end{proof}
\section{Lecture 27/10/2022}

\(\left(X, d\right)\) metric space.
\begin{definition}
    \(E \subset X\) is compact if for any open covering \(\{A_i\}_{i \in I}\) has a finite subcover.
\end{definition}

\begin{definition}
    \(E \subset X\) is sequentially compact if \(\forall \; \{x_n\} \subset E \) 
    there exists \(\{x_{n_k}\}\) subsequence convergent to some limit \(x \in E\)
\end{definition}

Well known fact: if \(\left(X, d\right)\) is a metric space, then \(E\) is compact \(\Leftrightarrow E \) is sequentially compact.

\begin{theorem}[Riesz Theorem]
    \(X\) normed space, dim\(X = \infty\) \(\Leftrightarrow \overline{B_1(0)}\) is not compact. 
\end{theorem}

\begin{lemma}[Riesz quasi orthogonality Lemma]
    \(X\) normed space, \(E \subsetneq X\) closed subspace. Then \(\forall \; \epsilon \in \left(0, 1\right) \ \exists \; x \in X \) s.t. 
    \[
        \norm{x}=1 \text{ and } \text{dist}(x, E) = \inf_{y \in E} \norm{x-y} \geq 1- \epsilon\]
\end{lemma}

\begin{remark}
    Also:
    \begin{itemize}
        \item \(E \in X\) closed. Then dist\((x, E)=0 \Leftrightarrow x \in E\)
        \item By definition of infimum, if \(d =\) dist\((x, E)\), then \(\forall \; \rho >0 \ \exists \; z \in E\) s.t. 
        \[
            \norm{x-z} < (1+\rho) d
        \]
    \end{itemize}
\end{remark}

\begin{proof}
    Let \(y \in X \setminus E\), and \(d := \) dist\((y, E) >0\), since \(E\) is closed. 
    
    \(\forall \; \rho > 0 \ \exists \; z \in E \) s.t.
    \[
        \norm{y-z} \leq (1+\rho)d = \frac{d}{1-\epsilon} \tag{1}
    \]
    since we choose \(\rho\) s.t. \(1+\rho = \frac{1}{1-\epsilon}\). Now we set \(x = \frac{y-z}{\norm{y-z}}\).

    Clearly \(\norm{x}=1\). Moreover, \(\forall \; u \in E\), we have that
    \[
        \norm{x-u} = \norm{ \frac{y-z}{\norm{y-z}} - u }
        = \norm{ \frac{y-z -\norm{y-z}u }{\norm{y-z}} }
        = \frac{1}{\norm{y-z}} \norm{y-(z + \norm{y-z}u)} =
    \]
    \[
        = \frac{1}{\norm{y-z}} \norm{y-w}
        \geq \frac{1}{\norm{y-z}} \text{dist}(y, E)
        \overset{(1)}{\geq} \frac{1-\epsilon}{d} d = 1 - \epsilon
    \]
    Since this is true \(\forall \; u \in E\), we deduce that
    \[
        \text{dist}(x, E) \geq 1-\epsilon
    \]
\end{proof}

\subsubsection*{Compactness on \(\mathcal{C}^0(\left[a, b\right])\)}
\begin{definition}
    \(\{f_n\}\) sequence in \(\mathcal{C}^0(\left[a, b\right])\). 
    We say that \(\{f_n\}\) is \textbf{uniformly equicontinuous} in \([a, b]\) if \(\forall \; \epsilon >0 \; \exists \; \delta >0 \) depending only on \(\epsilon \) s.t. 
    \[
        |t-\tau| < \delta \Rightarrow \| f_n(t) - f_n(\tau) \| < \epsilon \qquad \forall \; n
    \]
\end{definition}

\begin{remark}
    With respect to the uniform continuity, in this case \(\delta\) does not depend on \(f\). \(\delta\) is the same for all the \(f_n\) s
\end{remark}

\begin{theorem}[Ascolì Arzelà]
    \(\{f_n\} \subseteq \mathcal{C}^0(\left[a, b\right])\). Suppose that:
    \begin{itemize}
        \item \(\{f_n\}\) is uniformly equi-continuous
        \item \(\{f_n\}\) is equi-bounded: \(\exists \; M>0\) s.t. \(\norm{f_n}_\infty <M \qquad \forall \; n\)
    \end{itemize}
    Then \(\exists\) a subsequence \(\{f_{n_k}\}\) and \(f \in \mathcal{C}^0(\left[a, b\right])\) s.t. \(f_{n_k} \rightarrow f \) uniformly.
\end{theorem}

\subsection*{Lebesgue spaces} 
\((X, \mathcal{M}, \mu)\) measure space, \(p \in \left[1, \infty\right]\). We defined \(L^1(X)\) and \(L^\infty(X)\). 
In a similar way, we define \(L^p(X)\) \(\forall \; p \in \left[1, \infty\right]\)
\[
    \mathcal{L}^p (X, \mathcal{M}, \mu) := \{ f: X \rightarrow \barreal \text{ measurable s.t. } \int_X |f|^p \, d\mu < \infty\}
\]
On \(\mathcal{L}^p\) we introduce the equivalent relation
\[
    f \sim g \text{ in } \mathcal{L}^p \Leftrightarrow f=g \text{ a.e. on } X 
\]
and define 
\[
    {L}^p (X, \mathcal{M}, \mu) := \frac{\mathcal{L}^p (X, \mathcal{M}, \mu)}{\sim}
\]
We want to show that this is a normed space with
\[
    \norm{f}_p := 
    \begin{cases}
        \left( \int_X |f|^p \, d\mu \right)^{\frac{1}{p}}  & p \in [1, \infty) \\
        \underset{X}{\esssup} |f| & p = \infty
    \end{cases}
\]

The fact that \(L^p\) is a vector space is easy to prove. The only non trivial part is that \(f, g \in L^p \Rightarrow f+g \in L^p\).

This comes directly from the 
\begin{lemma}
    \(p \in [1, \infty), \ a, b \geq 0\). Then 
    \[
        \left(a+b\right)^p \leq 2^{p-1} \left(a^p+b^p\right) 
    \]
\end{lemma}

\(f, g \in L^p, \ p \in [1, \infty)\)
\[
    \int_X |f+g|^p \, d\mu \leq \int_X (|f|+|g|)^p \, d\mu 
    \leq 2^{p-1} \int_X (|f|^p+|g|^p) \, d\mu
\]
\[
    = 2^{p-1} \int_X |f|^p \, d\mu + 2^{p-1} \int_X |g|^p \, d\mu < \infty
\]
\(L^p\) is a vector space, \(\forall \; p \in [1, \infty)\).


\(f, g \in L^\infty\). Then a.e. 
\[
    \Rightarrow |f+g| \leq |f|+|g| \leq \norm{f}_\infty + \norm{g}_\infty < \infty
    \Rightarrow f+g \in L^\infty
\]
\(L^\infty\) is a vector space. 

\begin{remark}
    \(l^p := L^p (\mathbb{N}, \mathcal{P}(\mathbb{N}), \mu_c )\). \(l^p\) is a particular case of \(L^p\)
    \[
    \begin{array}{ll}
        l^p = \{ x = \left(x^{(k)}\right)_{k \in \mathbb{N}} : \sum_{k=1}^\infty |x^{(k)}|^p < \infty \} 
        & \norm{x}_p = \left( \sum_{k=1}^\infty |x^{(k)}|^p \right)^{\frac{1}{p}} \quad p \in [1, \infty)
        \\ l^\infty = \{ x = \left(x^{(k)}\right)_{k \in \mathbb{N}} : \sup_{k \in \mathbb{N}} |x^{(k)}| < \infty \} 
        & \norm{x}_\infty = \sup_{k \in \mathbb{N}} |x^{(k)}|
    \end{array}
    \]
\end{remark}

Now we prove that \(\norm{.}_p\) is actually a norm in \(L^p\). 
We will concentrate on \(p < \infty\) (\(p = \infty\) is the easy case) 

Properties 1 and 2 of the norm are immediate to check:
\begin{enumerate}
    \item \(\norm{f}_p = 0 \Leftrightarrow \int_X |f|^p \, d\mu =0 \Leftrightarrow f=0 \text{ a.e. on } X \Leftrightarrow f=0 \in L^p\)
    \item Obvious, by linearity
    \item About triangle inequality? We need some preliminaries
\end{enumerate}

\begin{theorem}[Young's Inequality]
    Let \(p \in (1, \infty)\), \(a, b \geq 0\). We say that \(q\) is the conjugate exponent of p if 
    \[
        \frac{1}{p} + \frac{1}{q} = 1 \Leftrightarrow q = \frac{p}{p-1}
    \]
    Then 
    \[
        ab \leq \frac{a^p}{p} + \frac{b^q}{q}
    \]
\end{theorem}
\begin{remark}
    \(p \in (1, \infty) \Rightarrow q \in (1, \infty)\). Moreover, we say that \(1\) and \(\infty\) are conjugate
\end{remark}
\begin{proof}
    \(\phi(x)= e^{x}\) is convex: 
    \[
        \phi((1-t)x + ty) \leq (1-t)\phi(x) + t \phi(y) \qquad \forall x, y \in \real \quad \forall \; t \in [0, 1]
    \]

    If \(a=0\) or \(b=0\), then the thesis holds. \\
    If \(a, b >0\)
    \[
        ab = e^{\log{a}} e^{\log{b}}
        = e^{\log{a}^{\frac{p}{p}}} e^{\log{b}^{\frac{q}{q}}}
        = e^{\frac{1}{p}\log{a}^p} e^{\frac{1}{q}\log{b}^q}
    \]
    Since \(\phi \) is convex
    \[
        \frac{1}{p} e^{\log{a}^p} + \frac{1}{q} e^{\log{b}^q} = \frac{1}{p} a^p + \frac{1}{q} b^q
    \]
    \(x = \log{a^p}\), \(y= \log{b^q}\) \(\qquad 1-t = \frac{1}{p}\), \(t=\frac{1}{q}\)
\end{proof}

\begin{theorem}[Holder's Inequality]
    \(\left(X, \mathcal{M}, \mu \right)\) measure space. \(f, g\) measurable functions. \(p, q \in [1, \infty]\) conjugate exponents.

    Then 
    \[
        \norm{fg}_1 \leq \norm{f}_p \norm{g}_q
    \]
\end{theorem}
\begin{proof}
    Case \(p, q \in (1, \infty)\). Obvious if \(\norm{f}_p \norm{g}_q = \infty\). \\
    If \(\norm{f}_p \norm{g}_q = 0 \Rightarrow\)  either \(f=0\) a.e. on \(X\) or \(g=0\) a.e. on X
    \(\Rightarrow fg=0\) a.e. on \(X\) \(\Rightarrow \norm{fg}_1 =0\). Let then \(\norm{f}_p\), \(\norm{g}_p \in (0, \infty)\). \\
    For \(x \in X\), we set 
    \[
        a := \frac{|f(x)|}{\norm{f}_p} \text{, } b := \frac{|g(x)|}{\norm{g}_q} 
    \]
    and use Young:
    \[
        \frac{|f(x)g(x)|}{\norm{f}_p \norm{g}_q} 
        \leq \frac{1}{p} \frac{|f(x)|^p}{\norm{f}_p^p} + \frac{1}{q} \frac{|g(x)|^q}{\norm{g}_q^q}
    \]
    \(\forall \; x \in X \). By integrating, we obtain
    \[
        \frac{1}{\norm{f}_p \norm{g}_q} \int_X |fg| \, d\mu \leq 
        \frac{1}{p \norm{f}_p^p} \int_X |f|^p \, d\mu + \frac{1}{q \norm{g}_q^q} \int_X |g|^q \, d\mu 
        = \frac{1}{p} + \frac{1}{q} = 1
    \]
    \[
        \Rightarrow \norm{fg} \leq \norm{f}_p \norm{g}_q
    \]
    Case \(p=1\), \(q= \infty\). Exercise 
\end{proof}
\begin{theorem}[Minkowski Inequality]
    \(f, g \in L^p(X, \mathcal{M}, \mu)\), \(p \in [1, \infty]\). Then 
    \[
        \norm{f+g}_p \leq \norm{f}_p + \norm{g}_p
    \] 
\end{theorem}
\begin{proof}
    \(p \in (1, \infty)\)
    \[
        \norm{f+g}_p^p = \int_X |f+g|^p \, d\mu = \int_X |f+g| |f+g|^{p-1} \, d\mu
    \]
    \[    
        \leq \int_X \left( |f|+|g| \right) |f+g|^{p-1} \, d\mu
        = \int_X |f| |f+g|^{p-1} \, d\mu + \int_X |g| |f+g|^{p-1} \, d\mu 
    \]
    Using Holder with \(p\), \(q = \frac{p}{p-1}\)
    \[
        \leq \norm{f}_p \left( \int_X \left( |f+g|^{p-1} \right)^{\frac{p}{p-1}} \, d\mu \right) ^ {\frac{p-1}{p}}
        + \norm{g}_p \left( \int_X \left( |f+g|^{p-1} \right)^{\frac{p}{p-1}} \, d\mu \right) ^ {\frac{p-1}{p}}
    \]
    \[
        = \norm{f}_p \norm{f+g}^{p-1}_p + \norm{g}_p \norm{f+g}_p^{p-1}
    \]
    We divide left hand side and right hand side by \(\norm{f+g}_p^{p-1}\):
    \[
        \norm{f+g}_p \leq \norm{f}_p + \norm{g}_p
    \]
\end{proof}
\section{Lecture 09/11/2022}
We introduced \(L^p(X, \mathcal{M}, \mu)\), and we proved that this is a normed space with 
\[
    \norm{f}_p := \begin{cases}
        \left( \int_X \abs{f}^p \; d\mu \right)^{\frac{1}{p}} & \text{if } p\in [1, +\infty) \\
        \underset{X}{\esssup}\abs{f} & \text{if } p = +\infty
    \end{cases}
\]
\subsection*{Inclusion of \(L^p\) spaces}
\begin{theorem}
    Suppose that \(\mu(X) < +\infty\). Then 
    \[
        1 \leq p \leq q \leq \infty \Rightarrow L^q(X) \subseteq L^p(X)
    \]
    Meaning that any \(f \in L^q\) is also in \(L^p\). More precisely, \(\exists \; C > 0\) depending on \(\mu(X), p, q\) s.t.
    \[
        \norm{f}_p \leq C \norm{f}_q \quad f \in L^q(X)
    \]
\end{theorem}
\begin{proof}
    If \(q = +\infty\)
    
    \(f \in L^\infty(X)\): then \(\abs{f(x)} \leq \underset{X}{\esssup}\abs{f} = \norm{f}_\infty\) for a.e. \(x \in X\), say \(\forall \; x \in X \setminus A\), with \(\mu(A) = 0\). Then 
    \[
        \int_X \abs{f}^p \, d\mu = \int_{X\setminus A} \abs{f}^p \, d\mu \leq \norm{f}_{\infty}^p \int_{X\setminus A} 1 \,d\mu = \norm{f}_\infty^p \underbrace{\mu(X)}_{= \mu(X\setminus A)}
    \]
    If \(q < +\infty\)

    Then \(\frac{q}{p} > 1\), and we can use Hölder\(\left(\frac{q}{p},\left( \frac{q}{p} \right)' \right)\), where \(\left( \frac{q}{p} \right)' = \frac{\frac{q}{p}}{\frac{q}{p}-1} = \frac{q}{q-p}\)
    \[
        \norm{f}_p^p = \int_X \abs{f}^p \, d\mu \overset{\text{\tiny{Hölder}}}{\leq} \left( \int_X \left( \abs{f}^{\not p} \right)^{\frac{q}{\not p}} \, d\mu\right)^{\frac{p}{q}}\cdot \left( \int_X 1 \, d\mu \right)^{\frac{q-p}{p}} = \left( \int_X \abs{f}^{q} \, d\mu\right)^{\frac{p}{q}}\cdot \left( \mu(X)\right)^{\frac{q-p}{p}}
    \]
    \[
        \Rightarrow \norm{f}_p \leq \mu(X)^{\frac{q-p}{qp}} \norm{f}_q
    \]
\end{proof}

The assumption \(\mu(X)< \infty\) is essential. For example, in \(X = [1, \infty]\)
\[
    \frac{1}{x} \in L^2([1, \infty]) \Leftrightarrow \int_1^\infty \frac{dx}{x^2} < \infty
\]
\[
    \frac{1}{x} \notin L^1 ([1, \infty]) \Leftrightarrow \int_1^\infty \frac{dx}{x} = \infty
\]

In particular, the previous theorem is false for \(l^p\)-spaces
\[
    l^p = L^p(\natural, \mathcal{P}(\natural), \mu_C)
\]
\[
    1 \leq p \leq q \leq \infty \Rightarrow l^p \subseteq l^q \text{, and } \exists \; C>0 \text{ s.t. } \norm{x}_q \leq C \norm{x}_p \quad \forall\; x \in l^p
\]
Without assumptions on \(\mu(X)\), in general one has the interpolation inequality.
\begin{theorem}
    \((X, \mathcal{M}, \mu)\) measure space. 
    Let \(1 \leq p \leq q \leq \infty\). If \(f \in L^p(X) \cap L^q(X)\), then 
    \[
        f \in L^r(X) \quad \forall \; r \in (p, q)
    \]
    and moreover 
    \[
        \norm{f}_r \leq \norm{f}_p^\alpha \norm{f}_q^{1-\alpha}
    \]
    where \(\alpha\) is such that \(\frac{1}{r}=\frac{\alpha}{p} + \frac{1-\alpha}{q}\)
\end{theorem}
\begin{proof}
    For exercise. Use Holder
\end{proof}

\subsection*{Completeness and Separability}
\begin{theorem}
    For \(1 \leq p \leq \infty\), \(L^p(X, \mathcal{M}, \mu)\) is a Banach space (with reference to \(\normdot_p\))
\end{theorem}
\begin{proof}
    \item \(p < \infty\).
    
    By using the characterization of completeness with the series, we want to show that 
    if \(\{f_n\} \subseteq L^p(X)\), and \(\sum_{k=1}^\infty\norm{f_k}_p < \infty \Rightarrow \sum_{k=1}^\infty f_k \) is convergent in \(L^p\), 
    namely \(s_n = \sum_{k=1}^n f_k\) has a limit in \(L^p\): \(\norm{s_n -s}_p \to 0 \) as \(n \to \infty\). 

    Let then \(\{f_n\} \subseteq L^p(X)\) s.t. 
    \[
        \sum_{k=1}^\infty \norm{f_k}_p = M < \infty 
    \]
    Define 
    \[
        g_n(x) = \sum_{k=1}^n \abs{f_k(x)}
    \]
    By Minkowski, \(\norm{g_n}_p \leq \sum_{k=1}^n \norm{f_k}_p \leq M < \infty\). 
    Moreover, for every \(x \in X\) fixed, \(\{g_n(x)\}\) is increasing \(\Rightarrow g_n(x) \to g(x)\) as \(n \to \infty\), \(\forall\; x \in X\)
    \[
        \int_X \abs{g}^p \, d\mu \overset{\text{\tiny{Monot conv}}}{=} \lim_n \int_X \abs{g_n}^p \leq M^p < \infty \Rightarrow g \in L^p(X)
    \]
    \(\Rightarrow \abs{g}^p \) is finite a.e.:
    \[
        \sum_{k=1}^\infty \abs{f_k(x)} < \infty \text{  for a.e. } x \in X
    \]
    \[
        \Rightarrow \sum_{k=1}^\infty f_k(x) \text{ is convergent a.e. to a limit } s(x)
    \]
    Thus, we proved that \(s_n(x) = \sum_{k=1}^n f_k(x) \to s(x)\) a.e. in \(X\). Namely \(\abs{s_n - s }^p \to 0\) a.e. in \(X\). To find a dominating function for \(\abs{s_n -s}^p\), we start by observing that
    \[
        \abs{s_n(x)} = \abs{\sum_{k=1}^n f_k(x)} \leq \sum_{k=1}^n \abs{f_k(x)} = g_n(x) \leq g(x) \text{ for a.e. } x \in X
    \]
    Therefore
    \[
        \abs{s_n -s }^p \leq 2^{p-1}(\abs{s_n}^p + \abs{s}^p) \leq 2^{p-1} (g^p + g^p ) = 2^p g^p \in L^1(X)
    \]
    By the dominated convergence theorem
    \[
        \int_X \abs{s_n -s }^p \, d\mu \to 0 \Leftrightarrow \norm{s_n -s}_p \to 0
    \]
    Thus \(L^p\) is complete.
    \item \(p=\infty\) exercise
\end{proof}

To speak about separability, we give a 
\begin{definition}
    \(g: \Omega \subset \real^n \to \real\). The support of \(g\) is
    \[
      \mbox{supp}\, g = \overline{\left\{ x \in \Omega : g(x) \neq 0 \right\}}
    \]
    Also 
    \[
        \mathcal{C}^0_C = \left\{ f \in \mathcal{C}^0\left( \Omega \right) : \mbox{supp} \, f \mbox{ is compact in } \Omega\right\} = \mathcal{C}^0_O(\Omega) = \mathcal{C}_C(\Omega)
    \]
\end{definition}
\begin{theorem}[Lusin Theorem]
    \(\Omega \in \mathcal{L}(\real), \lambda(\real) < +\infty\). Let also \(f : \real \to \real\) measurable, s.t. \(f\equiv 0\) in \(\Omega^C\).

    Then \(\forall \; \epsilon > 0 \; \exists \; g \in \mathcal{C}^0_C(\real)\) s.t.
    \[
        \lambda \left( \left\{ x \in \real : g(x) \neq f(x) \right\} \right) < \epsilon
    \]
    and
    \[
        \sup_{\real} \abs{g} \leq \sup_\real \abs{f}
    \]
\end{theorem}
\begin{definition}
    Given \(s \mbox{ simple function } = \sum_{k=1}^n a_k \chi_{E_k}\), where \(E_1, \ldots, E_n\) are \(\mathcal{L}\)-measurable sets, \(a_1, \ldots, a_n \in \real\). 
    \[
        E_1 \cup E_2 \cup \ldots \cup E_n = \real
    \]
    We consider
    \[
        \tilde{\mathcal{S}}(\real) = \left\{ s \mbox{ simple in } \real \mbox{ s.t. } \lambda\left( \left\{ s \neq 0 \right\} \right) < +\infty \right\}
    \]
    What does it mean for a simple function to be in \(L^p(\real)\)? 
    \[
        \int_{\real} \abs{s}^p \, d\mu = \sum_{k=1}^n a_k^p \lambda(E_k) < +\infty
    \tag*{\(1 \leq p \leq +\infty\)}\]
    iff \(s \equiv 0\) outside a set of finite measure \(\Leftrightarrow s \in \tilde{\mathcal{S}}(\real)\).

    \(\tilde{\mathcal{S}}(\real)\) is the set of integrable simple functions.
\end{definition}
\begin{theorem}
    \(\tilde{\mathcal{S}}(\real)\) is dense in \(L^p\), \(\forall \; p \in (1, +\infty)\)
\end{theorem}

\begin{proof}
    \(f \in L^p(\real), f \geq 0\) a.e. in \(\real\).

    We want to show that \(\exists \; \left\{ s_n \right\} \subseteq \tilde{\mathcal{S}}(\real)\) s.t. \(\norm{s_n - f}_p \to 0\).

    By the simple approximation theorem, \(\exists \; \left\{ s_n \right\}\) of simple functions s.t. \(\left\{ s_n(x) \right\}\) is increasing, for every \(x\), and \(s_n \to f\) pointwise in \(\real\).

    Since \(\abs{s_n}^p \leq f^p \Rightarrow s_n \in L^p\) for every n \(\Rightarrow \left\{ s_n \right\} \subseteq \tilde{\mathcal{S}}(\real)\). Moreover
    \[
        \abs{s_n -f}^p \to 0 \qquad \text{a.e. in } \real
    \]
    \[
        \abs{s_n -f }^p \leq 2^{p-1} (\abs{s_n}^p + \abs{f}^p) \leq 2^p \abs{f}^p \in L^1
    \]
    \(\Rightarrow \) by dominated convergence
    \[
        \int_\real \abs{s_n - f }^p \, d\lambda \to 0 \text{ , namely } \norm{s_n -f}_p \to 0
    \]
    If \(f\) is sign changing, then \(f = f^+ - f^-\)  and argue as before on \(f^+\) and \(f^-\)
\end{proof}

\begin{theorem}
    \( \forall p \in [1, \infty)\), the space \(L^p(\real)\) is separable. 
\end{theorem}
\begin{proof}
    \noindent\underline{sketch}
    \begin{itemize}
        \item Step 1: \(\mathcal{C}_C^0(\real) \) is dense in \(L^p(\real )\), \(\forall \; \leq p \leq \infty\).
        
        Take \(s \in \tilde{\mathcal{S}}(\real)\). Then, by Lusin theorem, \(\exists \; \{f_n\} \subseteq \mathcal{C}_C^0(\real)\) s.t. \(\norm{f_n -s }_p \to 0\).
        Then, since any \(f \in L^p\) can be approximated by simple integrable functions, we have that \(f\) can be approximated by functions in \(\mathcal{C}_C^0(\real)\).

        \item Step 2: 
        
        By Stone Weierstrass, the set of polinomials \(\mathcal{P}(\real)\) is dense in \(\mathcal{C}_C^0(\real)\) with the \(\normdot_\infty\) norm. 
        Since we work with functions with compact support, this implies that \(\mathcal{P}(\real)\) is dense in \(\mathcal{C}_C^0(\real)\) also with respect to \(\normdot_p\)
        \[
            \int_{-M}^M \abs{f - p_n }^p \, d\lambda \leq \norm{f - p_n }^p_\infty 2M \to 0
        \] 
        if \(\norm{f - p_n}_\infty \to 0\), \(\Rightarrow \mathcal{P}(\real)\) is dense in \(L^p(\real)\).

        \(\tilde{\mathcal{P}}(\real) = \{\)polynomials with rational coefficients\(\}\). 
        This is countable, and is dense in \((\mathcal{P}(\real), \normdot_p)\). \(\Rightarrow \) is dense in \(L^p\) 
    \end{itemize}
\end{proof}

What about \(L^\infty(\real)\)? In this case \(\mathcal{C}(\real)\) are not dense in \(L^\infty(\real)\).
For example, consider 
\[
    f(x) =
    \begin{cases}
        1 \quad x \geq 0
        \\ 0 \quad x<0
    \end{cases}
\] 
If \(g \in L^\infty\) s.t. \(\norm{g-f}_\infty < \frac{1}{3}\), then \(g\) cannot be continuous. Assume by contradiction that \(\exists \; g \in \mathcal{C}(\real)\) s.t. \(\norm{g-f}_\infty < \frac{1}{3}\). Then
\[
    \esssup _\real \abs{g(x)- f(x)} < \frac{1}{3}
\]
In particular, \(g(x) < \frac{1}{3}\) \(\forall \; x<0\)
\[
    \Rightarrow \lim_{x \to 0^-} g(x) \leq \frac{1}{3}
\]
On the other hand, \(g(x) > \frac{2}{3} \quad \forall \; x >0\)
\[
    \Rightarrow g(0)=\lim_{x \to 0^+} g(x) \geq \frac{2}{3}
\]
\section{Lecture 10/11/2022}
\noindent\underline{Quick recap about the `delirium' on the separability}

The thing that you need to know, in \(\to L^p(\real, \mathcal{L}(\real), \lambda)\), are:
\begin{enumerate}
    \item \(L^p\) is separable \(\forall\; p \in [1, \infty)\)
    \item \(\tilde{S}(\real)\) is dense in \(L^p(\real)\) \(\forall \; p \in [1, \infty)\), 
    namely \(\forall p \in L^p (\real)\) and \(\forall \; \epsilon >0 \) \(\exists \; s \in \tilde{S}(\real)\) s.t. 
    \[
        \norm{f-s}_p < \epsilon
    \]
    \item \(\mathcal{C}_C^0 (\real)\) is dense in \(L^p\), namely \(\forall p \in L^p (\real)\) and \(\forall \; \epsilon >0 \) \(\exists \; g \in \mathcal{C}_C^0(\real)\) s.t. 
    \[
        \norm{f-g}_p < \epsilon
    \]
\end{enumerate}
Everything remains true if you replace \(\real\) with \(X\) open or closed, or with \(X \in L(\real^n)\), and consider \((X, L(X), \lambda)\).

What happens for \(L^{\infty}(\real, \mathcal{L}(\real), \lambda)\)? 

\(\mathcal{C}(\real)\) is not dense in \(L^\infty\).

By the simple approximation theorem, we have that simple functions are dense in \(L^\infty\).
\begin{theorem}
    \(L^{\infty}(\real, \mathcal{L}(\real), \lambda)\) is not separable.
\end{theorem}
\begin{proof}
    \(\{\chi_{[-\alpha, \alpha]}: \alpha >0\} \subseteq L^{\infty}(\real, \mathcal{L}(\real), \lambda)\)
    \(\chi_\alpha = \chi_{[-\alpha, \alpha]}\)

    This is an uncountable family of functions. \(\norm{\chi_\alpha - \chi_{\alpha'}}\) \(\forall \; \alpha \neq \alpha'\)

    \[
        \abs{\chi_\alpha(x) - \chi_{\alpha'}(x)} =
        \begin{cases}
            0 & if x \in [-\alpha, \alpha] \cup (\alpha', \infty) \cup (-\infty, -\alpha')\\
            1 & if x \in (\alpha, \alpha'] \cup [-\alpha', \alpha)    
        \end{cases}
    \]

    In particular, \(B_{\frac{1}{2}}(\chi_\alpha) \cap B_{\frac{1}{2}}(\chi_{\alpha '}) = \emptyset \) \(\forall \; \alpha \neq \alpha'\)

    Assume by contradiction that \(L^\infty(\real)\) is separable: \(\exists Z \subset L^\infty\) which is countable and dense. In particular, \(\forall \; f \subset L^\infty\) \(\exists \; g \in Z\) s.t. 
    \[
        \norm{g-f }_\infty < \frac{1}{2}
    \]
    Therefore, \(\forall \; \alpha\) \(\exists g_\alpha \in B_{\frac{1}{2}}(\chi_\alpha) \cap Z\). 
    But \( B_{\frac{1}{2}(\chi_\alpha) \cap B_\{\frac{1}{2}(\chi_{\alpha'})} = \emptyset \)

    \[
        \Rightarrow \alpha \neq \alpha' \text{, we have } g_\alpha \neq g_{\alpha'}
    \]
    \(Z \supseteq \{ g_\alpha : \alpha >0 \}\), which is uncountable. This is not possible, since \(Z\) is countable.
\end{proof}

\begin{remark}
    The same is true if \((\real, \mathcal{L}(\real), \lambda)\) is swapped with \((X, \mathcal{L}(X), \lambda)\), \(X\) is open or closed on \(\real\) or \(\real^n\)
\end{remark}

\subsection*{Linear operators}
\((X, \normdot_X)\), \((Y, \normdot_Y)\) normed spaces.
\begin{definition}
    \(T : D(T) \subseteq X \to Y\) is a \textbf{linear operator} (or map) if 
    \[
        T(\alpha_1 x_1 + \alpha_2 x_2) = \alpha_1 T(x_1) + \alpha_2 T(x_2) \quad \forall \; x_1, x_2, \in D(T) \quad \forall \; \alpha_1, \alpha_2 \in \real
    \]
    \(D(T)\) is a linear subspace of \(X\), and is called the domain of T. When \(D(T) = X\) and \(Y = \real\), \(T\) is called linear functional.
\end{definition}
\begin{definition}
    A linear operator \(T : D(T) \subseteq X \to Y\) is bounded if \(D(T) = X\) and \(\exists \; M >0\) s.t. 
    \[
        \norm{T_X}_Y \leq M \abs{x}_X \forall \; x \in X
    \]
    Recall that \(T\) is continuous in \(x_0 \in X\) iff 
    \[
        \forall \; \left\{ x_n \right\} \subset X, x_n \overset{X}{\to} x_0 \Rightarrow Tx_n \overset{Y}{\to} Tx_0
    \]
\end{definition}
\noindent\underline{Ex}:
\begin{itemize}
    \item \(L: \real^n \to \real\)  is a linear functional . Then \(\exists \; y \in \real^n\) s.t. 
    \[
        Lx = <y, x> = (y, x) = y \cdot x
    \]
    In particular, then \(L\) is continuous on \(\real^n\) and bounded:
    \[
        \abs{L_X} < \abs{<y,x>} \overset{\text{\tiny{Cauchy-Schwarz}}}{\leq} \norm{y} \norm{x} \qquad \forall\; x \in \real^n
    \]
    So \(L\) is bounded with \(M=\norm{y}\).

    \item Linear operators in \(\infty\)-dim may not be defined everywhere, and many may not be continuous:
    \((X, \normdot_X) = (Y, \normdot_Y) = (\mathcal{C}([0, 1]), \normdot_\infty)\).
    
    Consider 
    \[
        \begin{array}{cc}
            \frac{d}{dx}: \mathcal{C}'([0,1]) \subseteq X \to Y & \frac{d}{dx}(\alpha f + \beta g) = \alpha \frac{d}{dx}f + \beta \frac{d}{dx} g \\
            f \mapsto f'
        \end{array}
    \]
    This is not continuous or bounded. For example, take \(f_n(x) = \frac{1}{n} \sin{2\pi n x}\). \(\norm{f_n}_\infty \to 0\) but \(\norm{f_n'}_\infty =1\)

    In this case \(f_n \to 0 \nRightarrow \frac{d}{dx} f_n \to 0\), then \(\frac{d}{dx} \) is not bounded as well.
    \item Let \((X, \normdot_X)\) be a normed space. If \(\dim X = 0\), is it possible to find linear functionals which are not bounded? Yes.
\end{itemize}
\begin{definition}
    A subset \(\left\{ qualcosa \right\}\) is called \textbf{Hamel basis} of \(X\) if 
    \[
        \norm{e_i}_X = 1 \ \forall  \; i
    \]
    and if every \(x \in X\) can be written in a unique way as 
    \[
        x = \sum_{k=1}^n x_k e_{i_k}, \quad x_k \in \real, \ n \in \mathbb{N}
    \]
\end{definition}
Every \(x\) can be written uniquely as a finite linear combination of element of the basis.
If \(\dim X = \infty\) is not immediate that the Hamel basis exists. This can be proved using the axiom of choice. (Zorn's lemma). 

Any normed space has a Hamel basis \(\dim X = \infty \Rightarrow \{e_i\}_{i \in I}\) has \(\infty\) many elements.

Let then \((X, \normdot_X)\) be \(\infty \dim\), with Hamel basis \(\{e_i \}_{i \in I}\). \(I\) is infinite \(\Rightarrow I \supseteq \mathbb{N}\).

We define \(L:X \to \real\) in the following way 
\[
    \begin{array}{ccccc}
        L e_0 = 0 & L e_1 = 1 & \dots & L e_n = n & \dots \\
        L e_i = 0 \quad \forall \; i \in I \setminus \mathbb{N} &&&&
    \end{array}
\]
Then, for \(x \in X\) we set
\[
    Lx = L \left( \sum_{k=1}^n x_k e_{i_k} \right) = \sum_{k=1}^n x_k L e_{i_k}
\]
 
\(L\) is linear by contradiction, and it is not bounded:

\[
    \begin{array}{c}
        \abs{L e_n} = n \to \infty \quad \norm{e_n }_X = 1 \; \forall\, n \\
        \frac{\abs{L e_n}}{\norm{e_n}_X} \to \infty \Rightarrow L \text{ L is not bounded}
    \end{array}
\]
\begin{remark}
    In practice, Hamel basis are hard to use. They differ from Hilbertian basis.
\end{remark}

For linear operators, boundedness and continuity are equivalent.
\begin{theorem}
    \(T:X \to Y\) linear map. Then the following are equivalent
    \begin{enumerate}
        \item \(T\) is continuous in \(0 \in X\)
        \item \(T\) is continuous everywhere in \(X\)
        \item \(T\) is bounded
    \end{enumerate}
\end{theorem}
\begin{remark}
    \(T\) linear \(\Rightarrow T0 = 0\). Indeed
    \[
        T0 = T(0x) =  0 Tx = 0
    \]
\end{remark}
\begin{proof}
    
    \begin{itemize}
        \item \((2) \Rightarrow (1)\) obvious.
        \item \((1) \Rightarrow (3)\) Suppose by contradiction that \(T\) is not bounded. 
        
        Then \(\exists \{ x_n \} \subset X \), \(x_n \neq 0\), s.t. 
        \[
            \frac{\norm{T x_n}_Y}{\norm{x_n}_X} \geq n \quad \forall\; n
        \]
        Define
        \[
            z_n := \frac{x_n}{n \norm{x_n}_X}
        \]
        Then \(\norm{z_n }_X = \frac{1}{n \norm{x_n}} \norm{x_n}_X \to 0\),
        namely \(z_n \to 0 \) in \( X \Rightarrow (T \text{ is continuous in }0)\) \(T z_n \to T0 =0\).
        However, 
        \[
            \norm{T z_n}_Y = \norm{T\left(\frac{x_n}{n \norm{x_n}_X}\right)} = \frac{1}{n \norm{x_n}_X} \norm{T x_n}_Y \geq 1 \ \forall\; n
        \]
        Contradiction.
        \item \((2) \Rightarrow (2)\) 
        We observe that \(\norm{Tx_1 - Tx_2}_Y = \norm{T(x_1 - x_2)}_Y \leq M \norm{x_1 - x_2}_X \) \(\forall \; x_1\) \(x_2 \in X\)
        Then, let \(x \in X \) and let \(x_n \to x\) in \(X\): \(\norm{x_n - x}_X \to 0\). But then
        \[
            \norm{T x_n - Tx}_Y \leq M \norm{x_n - x}_X \to 0
        \]
        namely \(Tx_n \to Tx \) in \(Y\). This is the continuity.
    \end{itemize}
\end{proof}

\begin{definition}
    The set of linear operators \(T : X \to Y\) which are also bounded (continuous) is denoted by \(\mathcal{L}(X, Y)\)

    If \(Y=X\), one simply writes \(\mathcal{L}(X)\)
\end{definition}
This is a vector space. \( \forall T, S \in \mathcal{L}(X, Y) \), \(\forall\; \alpha, \beta \in \real:\)
\[
     (\alpha T + \beta S)(x) = \alpha Tx + \beta Sx \qquad \in \mathcal{L}(X, Y)
\]
We can also introduce a norm:
\[
    \norm{T}_{\mathcal{L}(X, Y)} = \norm{T}_\mathcal{L} := \sup_{\norm{x}_X \leq 1} \norm{Tx}_Y
\]

Also, 
\[
    \norm{T}_{\mathcal{L}(X, Y)} = \sup_{\norm{x}_X = 1} \norm{Tx}_Y = \sup_{x \neq 0} \frac{\norm{Tx}_Y}{\norm{x}_X} = \inf{M >0 \text{ s.t. } \norm{Tx}_Y \leq M \norm{x}_X \quad \forall \; x \in X}
\]

\begin{theorem}
    \(X\) normed space, \(Y\) Banach space. Then \((\mathcal{L}(X, Y), \normdot_{\mathcal{L}(X, Y)})\) is a Banach space.
\end{theorem}
\begin{proof}
    Let \(\{T_n\}\) be a Cauchy sequence in \(\mathcal{L}(X, Y)\). We want to show that \(\exists T \in \mathcal{L}(X, Y) \) s.t.
    \[
        \norm{T_n - T}_\mathcal{L} \to 0
    \]
    \(\{T_n\}\) cauchy: \(\forall \; \epsilon >0 \) \(\exists \bar{n} \in \mathbb{N}\) s.t. 
    \[
        n, m > \bar{n} \Rightarrow \norm{T_n - T_m }_\mathcal{L} < \epsilon
    \]
    Consider then \(\{ T_n x \}\), \(x \in X\)
    \[
        \norm{T_n x - T_m x }_Y = \norm{(T_n - T_m)x}_Y \leq \norm{T_n - T_m}_Y \norm{x}_X \leq \epsilon \norm{x}_X \tag{*}
    \]
    This means that \(\{ T_n x \}\) is a Cauchy sequence in \(Y\), which is complete: then \(\forall \; x \in X\) \(\exists \) a vector \(y_x \in Y\) s.t. \(T_n x \to y_x\) in \(Y\).

    Define 
    \[
        T: X \to Y \qquad x \mapsto y_x = Tx
    \]
    \(T\) is linear: indeed, \(\forall \; x_1\), \(x_2 \in X\) and \(\alpha_1\), \(\alpha_2 \in \real\):
    \[
        T(\alpha_1 x_1 + \alpha_2 x_2) = \lim_{n \to \infty} T_n (\alpha_1 x_1 + \alpha_2 x_2) = \lim_{n \to \infty} (\alpha_1 T_n x_1 + \alpha_2 T_n x_2) = \alpha_1 Tx_1 + \alpha_2 Tx_2
    \]
    So \(T \) is linear. It remains to show that \(T\) is bounded, and that \(\norm{T_n - T}_{\mathcal{L}} \to 0\).
    To show that \(T\) is bounded, note that, by (*), \(\forall \; \epsilon >0 \; \exists \; \bar{n}\) s.t.
    \[
        n, m > \bar{n} \Rightarrow \norm{T_n x - T_m x}_Y \leq \epsilon \norm{x}_X \quad \forall \; x 
    \]
    Take the limit for \(m \to \infty\): 
    \[
        \norm{T_n x - Tx}_Y \leq \epsilon \norm{x}_X
    \]
    But then, since \(T_n\) is bounded, 
    \[
        \norm{Tx}_Y = \norm{Tx \pm T_n x}_Y \leq \norm{T_n x}_Y + \norm{Tx - T_n x}_Y \leq M_n \norm{x}_X + \epsilon \norm{x}_X = (M_n + \epsilon) \norm{x}_X
    \]
    and \(T\) is bounded. To show that \(\norm{T_n - T}_\mathcal{L} \to 0\), observe that \(\forall \; \epsilon >0 \; \exists \; \bar{n} \) s.t. \(n > \bar{n}\)
    \[
        \norm{T_n x - Tx}_Y \leq \epsilon \norm{x}_X 
        \Leftrightarrow \frac{\norm{(T_n - T)x}_Y}{\norm{x}_X} \leq \epsilon \quad \forall \; x \in X \setminus {0}
        \overset{\text{\tiny{take sup over \(x \neq 0\)} }}{\Rightarrow} \norm{T_n - T}_\mathcal{L} < \epsilon
    \]
    namely, \(T_n \to T\) in \(\mathcal{L}\)
\end{proof}
\section{Lecture 16/11/2022}
Let \(T\) be a linear operator from \(X\) to \(Y\).
\begin{definition}
    The \textbf{kernel} of \(T\) is the set 
    \[
        \ker(T) = \{ x \in X: Tx =0\} \subset X
    \]

\end{definition}
This is a vector subspace of \(X\). 

\noindent \(T\) is injective \(\Leftrightarrow \ker (T) = \{0\}\). If \(T\) is continuous, \(\ker(T)\) is closed 
\[
    \ker(T) = T^{-1} (\{0\})
\]
\begin{definition}
    \(X\), \(Y\) normed spaces. \(X\) and \(Y\) are isomorphic if \(\exists \; T \in \mathcal{L}(X, Y)\) bijective, and such that \(T^{-1} \in \mathcal{L}(X, Y)\)
\end{definition}
\begin{definition}
    \(T \in \mathcal{L}(X, Y)\) is an isometry if
    \[
        \norm{Tx}_Y = \norm{x}_X \quad \forall \; x \in X
    \]
\end{definition}
\begin{definition}
    If \(X \subseteq Y\) is a vector subspace, and \(\left(X, \normdot_X\right)\) and \(\left(Y, \normdot_Y\right)\) are normed space, then we can consider 
    \[
        \begin{array}{rcr}
            J: & X \to Y & \text{(inclusion map)}
            \\ & x \mapsto x
        \end{array}
    \] 
    If \(J \in \mathcal{L}(X, Y)\) (namely, if \(\exists \; M>0 \) s.t. \(\norm{x}_Y \leq M \norm{x}_X\) \(\forall \; x \in X\)), 
    then we say that \(J\) is an embedding of \(X\) into \(Y\), and we write \(X \hookrightarrow Y\)
\end{definition}

\noindent\underline{Ex}: \(\mu(X) < \infty \), \(1 \leq p < q \leq \infty\)
\[
    L^q(X) \hookrightarrow L^p(X) \tag*{(inclusion of \(L^p\) spaces)}
\]

\subsection*{Some fundamentals theorems on linear operators}
\begin{definition}
    \((X, d)\) metric space. \(A \subset X\). \(x \in X \) is an \textbf{adherence point} of \(A\) if \(\forall \; r>0: B_r(x) \cap A \neq \emptyset\)
    \[
        \bar{A} = \{ x \in X: x \text{ is an adherence point of } A \} = A \cup \partial A
    \]
\end{definition}
\begin{definition}
    \(A \subset X\) is \textbf{dense} in \(X\) if \(\bar{A} = X\).
\end{definition}
For example, \(\mathbb{Q}\) is dense in \(\real\), and \((a, b)\) is dense in \([a, b]\).
\begin{definition}
    \(A \subset X\) is \textbf{nowhere dense} if the interior of the closure of \(A\) is empty, namely
    \[
       \text{int} (\bar{A}) = \interior{\bar{A}} = \emptyset  
    \]
\end{definition}

Ex: \(\interior{\bar{\{x\}}} = \interior{\{x\}} = \emptyset\)

\(\mathbb{Z} \subset \real\): \(\interior{\bar{\mathbb{Z}}}=\interior{\mathbb{Z}} = \emptyset\)

\(\mathbb{Q} \) is not nowhere dense: \(\interior{(\bar{\mathbb{Q}})} = \interior{(\mathbb{R})} = \real \)
\begin{definition}
    \(A \subset X\) is called \textbf{of first category} (or \textbf{meager set}) in \(X\) if \(A\) is the (at most) countable union of nowhere dense sets. 
\end{definition}

Ex: \(\mathbb{Q}\) is of first category in \(\real\): countable union of nowhere dense sets
\[
    \mathbb{Q} = \bigcup_{q \in \mathbb{Q}} \{q\}
\]
\begin{definition}
    \(A \subset X\) is of second category if it is not of first category.
\end{definition}
\begin{theorem}[Baire category theory]
    \((X, d)\) complete metric space. Then 
    \begin{itemize}
        \item \(\{U_n\}_{n=0}^\infty\) is a sequence of open and dense sets in \(X\) \(\Rightarrow \cap_{n=0}^\infty U_n\) is dense in \(X\).
        \item \(X\) is of second category in itself: \(X\) cannot be the countable union of nowhere dense sets. 
    \end{itemize}
\end{theorem}


\noindent\underline{Preliminaries}:
\begin{itemize}
    \item \(A \subset X\) is dense \(\Leftrightarrow \forall \; W \subset X\), \(W\) open, \(W \neq \emptyset\), we have that \(A \cap W \neq \emptyset\)
    \item \(A\) is nowhere dense \(\Leftrightarrow \left(\bar{A}\right)^C\) is open and dense
\end{itemize}

\begin{proof} Here's the proof of the two parts of the theorem:
    \begin{itemize}
        \item[(a)] Thanks to the first preliminary, we show that \(\forall \; W \subset X\) open and non empty we have \((\cap_n U_n) \cap W \neq \emptyset\)
        
        \[
            \begin{array}{rl}
                U_0 \text{ is open and dense:} & \overset{1^{st}\text{\tiny{prel.}}}{\Rightarrow} \underbrace{U_0 \cap W}_{\text{is open}} \neq \emptyset \\
                & \Rightarrow \text{it contains an open ball} \\
                & \Rightarrow (U_0 \cap W) \supset B_{r_0}(x_0) \mbox{ for some } x_0 \in X \mbox{ and } r_0 > 0
        \end{array}
        \] 
        For \(n>0\), we choose \(x_n \in X\) and \(r_n > 0\) inductively in the following way: we have 
        \[
            U_n \cap B_{r_{n-1}}(x_{n-1}) \neq \emptyset 
        \tag*{(\(1^{st}\) prel. \(+ \, U_n\) is dense)}\]
        \[
            \Rightarrow \overline{B_{r_n}(x_n)} \subset \underset{\begin{array}{l}\text{\small{all these balls}} \\ \text{\small{are included in}} \\
            \displaystyle{B_{r_0}(x_0)}\end{array}}{(U_n \cap B_{r_{n-1}}(x_{n-1}))}
        \]
        with \(x_n \in X\) and \(0 < r_n < \frac{1}{2^n}\)

        By the condition on \(r_n\), we see that 
        \[
            x_n, x_m \in B_{r_N}(x_N) \quad \forall \; n,m > N
        \]
        \(\Rightarrow \left\{ x_n \right\}\) is a Cauchy sequence in \(X\)
        \[
            d(x_n, x_m) \leq \frac{1}{2^N} \quad \forall \; n,m > N
        \]
        \(X\) is complete: \(x_n \overset{d}{\rightarrow} x \in X\)
        Since 
        \[
            \begin{array}{lr}
                x_n \in B_{r_N}(x_N) & \forall \; n > N \\
                \Rightarrow x = \lim_n x_n \in \overline{B_{r_N}(x_N)} \subset (U_n \cap B_{r_0}(x_0)) \subset (U_N \cap W) & \forall \; n \in \mathbb{N} \\
                \Rightarrow x_n \in \bigcap_n (U_n \cap W) = \left(\bigcap_n U_n\right) \cap W
            \end{array}
        \]
        This means that \(\bigcap_n U_n\) is dense.
        \item[(b)] It follows from (a):
        
        If \(\left\{ E_n \right\}\) is a sequence of nowhere dense sets in \(X\), then, by the second preliminary \(\left\{ (E_n)^C \right\}\) is a sequence of open and dense sets. By (a) 
        \[
            \begin{array}{l}
                \bigcap_n (\overline{E_n})^C \neq \emptyset \\
                \Rightarrow \bigcup_n E_n \subset \bigcup_n \overline{E_n} = X \setminus \overset{= \emptyset}{\left(\bigcap_n (\overline{E_n})^C\right)} \neq X \\

            \end{array} 
        \]
    \end{itemize}
\end{proof}
\noindent\underline{Ex}: \((X, \normdot)\) \(\infty-\dim\) Banach space. \(\left\{ e_i \right\}_{i \in I}\) Hamel basis. 

Then \(I\) is uncountable.
\begin{theorem}[Banach Steinhaus]
    \(X\) Banach space, \(Y\) normed space, \(\mathcal{F} \subseteq \mathcal{L}(X, Y)\) family.
    Suppose that \(\mathcal{F}\) is pointwise bounded: 
    \[
        \forall\; x \in X \quad \exists \; M_x >0 \text{ s.t. } \sup_{T \in \mathcal{F}} \norm{Tx}_Y \leq M_x \tag*{(PB)}
    \]
    Then \(\mathcal{F} \) is uniformly bounded: 
    \[
        \exists \; M \geq 0 \text{ s.t. } \sup_{T \in \mathcal{F}} \norm{T}_{\mathcal{L}(X, Y)} \leq M \tag*{(UB)}
    \]
\end{theorem}
\begin{proof}
    \(\forall \; n \in \natural\), let 
    \[
        C_n := \{ x \in X: \norm{Tx}_Y \leq n \quad \forall \; T \in \mathcal{F} \} 
        = \cap_{T \in \mathcal{F}} \{ x \in X: \norm{Tx}_Y \leq n \}
    \]
    \(C_n\) is a closed set \(\forall \; n\), since \(T\) is continuous. (also \(\phi: X \to \real \) \(\phi(x)=\norm{Tx}_Y\) is continuous)

    By (PB), every \(x \in X\) stays in some \(C_n\): \(X = \cup_{n=1}^\infty C_n\). 
    Since \(X\) is Banach, by the Baire theorem it is necessary that \(\exists \; n_0 \in \natural\) s.t. \(\interior{C_{n_0}} \neq \emptyset \Rightarrow\) a ball \(\overline{B_r(x_0)} \subset C_{n_0}\): then
    \[
        \norm{T(x_0+rz)}_Y \leq n_0 \quad \forall \; z \in \overline{B_1(0)}
    \]
    \[
        \norm{T(x_0 + rz)}_Y \overset{\text{\tiny{linearity}}}{=} \norm{Tx_0 + rTz}_Y \overset{\text{\tiny{triangle ineq}}}{\leq} r\norm{Tz}_Y - \norm{Tx_0}_Y \quad \forall \; T \in \mathcal{F} \; 
    \]
    To sum up: \(\forall \; T \in \mathcal{F} \), \(\forall \; z \in \overline{B_1(0)}\) we have 
    \[
        r \norm{Tz}_Y - \norm{Tx_0}_Y \leq n_0 \Rightarrow \norm{Tz}_Y \leq \frac{1}{r}(n_0 + M_{x_0})
    \]
    We take sup over \(T \in \mathcal{F}\):
    \[
        \sup_{T \in \mathcal{F}} \norm{T}_{\mathcal{L}(X, Y)} \leq \frac{1}{r} \left(n_0 + M_{x_0}\right) =: M
    \]
\end{proof}

\begin{corollary}
    \(X\) Banach space, \(Y\) normed space. \(\{ T_n \} \subseteq \mathcal{L}(X, Y)\) s.t. \(\{T_n x\}\) has a limit, denoted by \(Tx\), \(\forall \; x \in X\) (pointwise convergence). 
    Then \(T \in \mathcal{L}(X, Y)\)
\end{corollary}
\begin{proof}
    \(T\) is linear: 
    \[
        \begin{array}{ccc}
            T_n(\alpha_1 x_1 + \alpha_2 x_2 ) & = & \alpha_1 T_n x_1 + \alpha_2 T_n x_2 \\
            \downarrow && \downarrow \\
            T(\alpha_1 x_1 + \alpha_2 x_2) & = & \alpha_1 Tx_1 + \alpha_2 Tx_2
        \end{array}
    \]
    Now we observe that we have (PB): if \(\{ T_n x \}\) is convergent \(\Rightarrow \{T_n x\} \) is bounded \(\Rightarrow \) by Banach Steinhaus, \(\{ T_n \}\) is uniformly bounded: 
    \[
        \exists M>0 \text{ s.t. } \sup_n \norm{T_n}_{\mathcal{L}(X, Y)} \leq M 
    \]
    Therefore, \(\forall \; x \in X\):
    \[
        \norm{Tx}_Y = \norm{\lim_n (T_n x)}_Y = \lim_n \norm{T_n x}_Y \leq \lim_n \norm{T_n}_\mathcal{L} \norm{x}_X \leq \lim_n M \norm{x}_X  = M \norm{x}_X
    \]
    Thus, \(T\) is bounded: \(T \in \mathcal{L}(X, Y)\)
\end{proof}
\section{Lecture 17/11/2022}
Let \(X, Y\) be normed spaces.
\begin{definition}
    \(T: X \to Y\) is called \textbf{open map} if, \(\forall \; A \subset X \mbox{ open}\), the set \(T(A)\subset Y\) is open.
\end{definition}
\begin{remark}
    Recall that \(T\) is continuous on \(X\) if \(T^{-1}(O)\) is open on \(X\), \(\forall \; O \mbox{ open in } Y\).
\end{remark}
\noindent\underline{Ex}: \(f(x) : \mbox{constant}\) is continuous, but not open. \(f((a,b)) = \left\{ \mbox{const} \right\}\)
\begin{theorem}[Open map theorem]
    \(X, Y\) Banach spaces. \(T \in \mathcal{L}(X,Y)\) is surjective. Then \(T\) is an open map.
\end{theorem}
\begin{corollary}
    \(X,Y\) Banach spaces, \(T \in \mathcal{L}(X,Y)\) is bijective. Then \(T\) is an isomorphism: \(T^{-1} \in \mathcal{L}(X,Y)\)
\end{corollary}
\begin{proof}
    \begin{itemize}
        \item \(T : Y \to X\) is linear. (Exercise. Hint: Use \(T^{-1} \circ T = \text{Id} + \) linearity of \(T\))
        \item We want now to check that \(T^{-1}\) is continuous on \(Y\): \((T^{-1})^{-1}(O)\) is open in \(Y\), \(\forall \; O\) open in \(X\). We know that \(T\) is an open map thanks to the open map theorem.
        \[
            (T^{-1})^{-1}(O) = \left\{ y \in Y, T^{-1}(y) \in O \right\} = \left\{ y \in Y, T^{-1}(y) = x, \mbox{ for some } x \in O \right\} =
        \]
        \[
            = \left\{ y \in Y, y = Tx, \mbox{ for some } x \in O\right\} = T(O) \mbox{ is open}
        \]
    Since \(T\) is an open map, \(\forall \; O \subset X\), open.
    \end{itemize}
\end{proof}
\begin{corollary}
    \(X\) vector space, \(\normdot, \normdot_*\) norms on \(X\). Assume \((X, \normdot), (X, \normdot_*)\) are Banach spaces. 
    Assume that \(\exists \; C_1 > 0\) s.t. 
    \[
        \norm{x}_* \leq C_1\norm{x} \quad \forall \; x \in X
    \]
    Then \(\normdot\) and \(\normdot_*\) are equivalent, namely \(\exists \; C_2 > 0\) s.t. 
    \[
        \norm{x} \leq C_2 \norm{x}_*
    \]
\end{corollary}
\begin{proof}
    Consider
    \[
        \begin{array}{lrl}
            I :& (X, \normdot) & \to (X, \normdot_*) \\
            &x & \mapsto x
        \end{array}
    \]
    By assumption, \(I\) is bounded: \(\exists \; C_1 > 0\) s.t. 
    \[
        \norm{Ix}_* = \norm{x}_* \leq C_2 \norm{x}
    \]
    \(I\) is bijective.

    Thus, by the corollary before
    \[
        I^{-1} = I \in \mathcal{L}((X, \normdot_*), (X, \normdot))
    \]
    namely \(\exists \; C_2 > 0\) s.t. 
    \[
        \stackbelow{\norm{Ix}}{\norm{x}} \leq C_2 \norm{x}_*
    \]
\end{proof}
\begin{definition}
    \(T : D(T) \subset X \to Y\) linear operator. We say that \(T\) is \textbf{closed} if \(\forall \; \left\{ x_n \right\} \subset D(t)\). 
    \[
        \begin{rcases*}
            x_n \to x & \mbox{in } X \\
            Tx_n \to y & \mbox{in } Y
        \end{rcases*} \Rightarrow x \in D(T) \mbox{ and } Tx = y
    \]

\end{definition}
\noindent\underline{Ex}: \(X = Y = \mathcal{C}^0([0,1])\) with the supremum norm.
\[
    T = \frac{d}{dx}
\]
\(T\) is not continuous. But it is closed: it can be proved that if \(\left\{ f_n \right\} \subset \mathcal{C}^1([0,1])\) is s.t.
\[
    \begin{rcases*}
        f_n \to f & \mbox{uniformly} \\
        f_n' \to g & \mbox{uniformly}
    \end{rcases*} \Rightarrow f \mbox{ is } \mathcal{C}^1([0,1]) \mbox{ and } f' = g
\] 
\noindent\underline{Ex}: \(T \in \mathcal{L}(X,Y) \Rightarrow T \mbox{ is closed}\)
\begin{remark}
    \(T\) is a closed operator \(\Leftrightarrow\) the graph of \(T\) is closed.
    \[
        \mbox{graph}(T) = \left\{ (x, Tx): x \in X \right\}
    \]
\end{remark}
\begin{theorem}[Closed graph theorem]
    \(X, Y\) Banach spaces. 
    
    \(T : X \to Y\) linear closed operator (\(D(T) = X\)). 
    
    Then \(T \in \mathcal{L}(X,Y)\).
\end{theorem}
\begin{remark}
    In general it is easier to prove that an operator is closed, rather than it it continuous.
\end{remark}
\begin{proof}
    Define on \(X\) the graph-norm of \(T\)
    \[
        \norm{x}_* = \norm{x}_X + \norm{Tx}_Y
    \]
    Then is a norm on \(X\). If \(\left\{ x_n \right\} \in X\) is a Cauchy sequence for \(\normdot_*\), then \(\left\{ x_n \right\}\) is a Cauchy sequence in \((X, \normdot_X)\) and \(\left\{ Tx_n \right\}\) is a Cauchy sequence on \((Y, \normdot_Y)\)
    \[
        \Rightarrow \begin{rcases*}
            x_n \to x & \mbox{in } X \\
            Tx_n \to y & \mbox{in } Y
        \end{rcases*} \mbox{ since } T \mbox{ is closed, we deduce that } y = Tx
    \]
    Thus 
    \[
        \norm{x_n - x}_X + \norm{Tx_n - Tx} \to 0
    \]
    This proves that \((X, \normdot_*)\) is a Banach space. Also, we know that 
    \[
        \norm{x}_X \leq \norm{x}_X + \norm{Tx}_Y = \norm{x}_*
    \]
    By the last corollary of the open map theorem, \(\exists \; C_2\) s.t. 
    \[
        \norm{x}_* \leq C_2 \norm{x_X}
    \]
    \[
        \norm{Tx}_Y \leq \norm{x}_* \leq C_2 \norm{x}_X \quad \forall \; x \in X
    \]
    This means that \(T\) is bounded.
\end{proof}
\subsection*{Dual spaces}
\(X\) normed space: 
\[
    X^* = \mathcal{L}(X, \real) \mbox{ is called \textbf{dual space of} }X
\]
\(X\) normed space, \(Y\) Banach space \(\Rightarrow \mathcal{L}(X, Y)\) is a Banach space with \(\normdot_{\mathcal{L}}\).

Since \(\real\) is a Banach space, the dual space \(X^*\) is a Banach space with 
\[
    \norm{L}_* = \sup_{\norm{x}_X \leq 1} \abs{Lx}
\]
\noindent\underline{Ex}: 
\begin{itemize}
    \item In \(\real^n\), only linear functional is separated by a scalar product:
        \[
            L : \real^n \to \real \mbox{ is linear } \Rightarrow \exists ! \; y \in \real^n \mbox{ s.t. } Lx = <y,x>
        \]
        It can be proved that 
        \[
            L \in (\real^n)^* \mapsto y \in \real^n
        \]
        is an isometric isomorphism 
        \[
            (\real^n)^* \approxeq \real^n
        \]
        Then \(X^*\) is very complicated.
    \item Dual of \(L^p\)?
    
    \(\left(X, \mathcal{M}, \mu\right)\) measure space. \(p \in \left[1, \infty\right]\), \(p'\) conjugate exponent.
    \[
        \frac{1}{p} + \frac{1}{p'} = 1 \iff 
        \begin{cases}
            \begin{array}{ll}
                p' = \frac{p}{p-1} & p \in (1, \infty) \\
                p' = \infty & p = 1 \\
                p'=1 & p = \infty
            \end{array}     
        \end{cases}
    \]
    For \(g \in L^{p'}(X)\), define \(L_g : L^p(X) \to \real\) by
    \[
        l_g f := \int_X fg \, d\mu \qquad \forall \; f \in L^p(\Omega)
    \]
    This is well defined, by the Holder inequality:
    \[
        \abs{\int_X fg \, d\mu } \leq \int_X \abs{fg} \, d\mu = \norm{fg}_1 \leq \norm{g}_{p'} \norm{f}_p \tag*{*} 
    \]
    If \(g \in L^{p'}\), this shows that \(L_g f \in \real \quad \forall \; f \in L^p\)
\end{itemize}

\begin{proposition}
    If \(p \in [1, \infty]\) then \(L_g \in (L^p(X)^*)\). Moreover, 
    \begin{itemize}
        \item if \(p > 1\), then \(\norm{L_g}_* = \norm{g}_{p'}\)
        \item if \(p=1\) then \(\norm{L_g}_* = \norm{g}_{\infty}\) with more assumptions (they are satisfied in \((X, \mathcal{L}(X), \lambda)\))
    \end{itemize}
\end{proposition}

\begin{remark}
    We are saying that \(L^{p'}\) can be identified with a subspace of the dual space \((L^p)^*\) and this identification is an isometry.
\end{remark}

Question: are there functional in \((L^p)^*\)?

\begin{proof}
    (of the proposition)
    \begin{itemize}
        \item Case \(p=\infty\) ex
        \item Case \(p=1\) but difficult it's ok if you don't do it
        \item Case \(p \in (1, \infty)\)
        
        \(L_g\) is clearly linear, by linearity of \(\int\), indeed:
        \(\forall \; \alpha \; \beta \in \real\), \(f_1 \; f_2 \in L^p(X)\). Then
        \[
            L_g(\alpha f_1 + \beta f_2) = \int_X g (\alpha f_1 + \beta f_2) \, d\mu = \alpha \int_X g f_1 \, d\mu + \beta \int_X g f_2 \, d\mu = \alpha L_g f_1 + \beta L_g f_2
        \]
        We want to show now that \(L_g \) is bounded. 
        We proved in (*) that 
        \[
            \abs{L_g f} \leq \norm{g}_{p'} \norm{f}_p \quad \forall \; f \in L^p(\Omega)
        \]
        This shows that \(L_g\) is bounded, with norm  \(\norm{L_g}_* \leq \norm{g}_{p'}\) (remember that \(\norm{T}_\mathcal{L} = \inf \{ M>0: \norm{Tx} _Y \leq M \norm{x}_X \quad \forall x \in X\} \))

        We want to show that \(\norm{L_g}_* = \norm{g}_{p'}\). If \(\norm{L_g}_* < \norm{g}_{p'}\), then \(\exists \; M < \norm{g}_{p'}\) s.t. 
        \[
            \abs{L_g f } \leq M \norm{f}_p \quad \forall \; f \in L^p
        \]
        We rule out this possibility by choosing an explicit \(\tilde{f} \in L^p \) s.t.
        \[
            \abs{L_g \tilde{f} } = \norm{g}_{p'} \norm{\tilde{f}}_p
        \]
        We take 
        \[
            \tilde{f} = \frac{\abs{g}^{p'-1}}{\norm{g}_{p'}^{p'-1}} \frac{g}{\abs{g}}
        \]
        Now, 
        \[
            \norm{\tilde{f}}_p^p = \int_X \abs{\tilde{f}}^p \, d\mu = \int_X \frac{\abs{g}^{p(p'-1)}}{\norm{g}_p^{p(p'-1)}} \, d\mu = (*)
        \]
        \((p')' = p \Rightarrow p=\frac{p'}{p'-1} \Rightarrow p(p'-1) = p'\)
        \[
            (*) = \frac{1}{\norm{g}_{p'}^{p'}} \int_X \abs{g}^{p'} \, d\mu = \frac{\norm{g}_{p'}^{p'}}{\norm{g}_{p'}^{p'}} = 1
        \]    
        \[
            \begin{array}{l}
            \abs{L_g \tilde{f}} = \abs{\int_X \frac{\abs{g}^{p'-1}}{\norm{g}_{p'}^{p'-1}} \abs{g} \, d\mu}
            = \abs{\int_X \frac{\abs{g}^{p'}}{\norm{g}_{p'}^{p'-1}} \, d\mu }
            = \frac{1}{\norm{g}_{p'-1}^{p'}} \norm{g}_{p'}^{p'} 
            = \norm{g}_{p'} = \norm{g}_{p'}\norm{\tilde{f}}_p
        \end{array}
            \]
    \end{itemize}
\end{proof}

\subsubsection*{Hahn Banach}

\begin{definition}
    \(X\) vector space. A map \(p: X \to \real\) is called \textbf{sublinear functional} if 
    \begin{itemize}
        \item \(p(\alpha x) = \alpha p(x) \qquad \forall \; x \in X\), \(\alpha >0\)
        \item \(p(x+y) \leq p(x) + p(y) \qquad \forall x, y \in X\)  
    \end{itemize}
\end{definition}

\begin{theorem}[Hahn Banach]
    \(X\) real vector space, \(p: X \to \real\) sublinear functional. 
    \(Y\) subspace of \(X\) and suppose that \(\exists \; f: Y \to \real \) linear on \(Y\) s.t. 
    \[
        f(y) \leq p(y) \quad \forall \, y \in Y
    \]
    Then \(\exists\) a linear functional \(F: X \to \real \) s.t. 
    \[
        F(y) = f(y) \quad \forall \; y \in Y \tag*{\(F\) is an extension of \(f\)}
    \]
    Moreover,
    \[
        F(x) \leq p(x) \quad \forall \; x \in X
    \] 
\end{theorem}

\section{Lecture 23/11/2022}

\begin{theorem}[Hahn-Banach regarding continuous extension]
    \(X\) (real) normed space. \(Y\) subspace of \(X\), \(f \in Y^* = \mathcal{L}(Y, \real)\)

    Then \(\exists \; F \in X^* = \mathcal{L}(X, \real)\) s.t. 
    \[
        \begin{array}{lr}
            
            F(y) = f(y) & \forall \; y \in Y \\

            \norm{F}_{X^*} = \norm{f}_{Y^*}   & \\
        \end{array}
    \]
\end{theorem}
\begin{proof}
    Define \(p:X \to \real\), \(p(x) = \norm{f}_{Y^*} \norm{x}_X\) \(\forall \; x \in X\). Then \(p\) is sublinear (from the properties of \(\normdot_X\)).

    Moreover, \(f(y) \leq \abs{f(y)} \leq \norm{f}_{Y^*}\norm{y}_X = p(y)\) \(\forall \; y \in Y\). Then, by Hahn-Banach theorem (general version), \(\exists \; F : X \to \real\) s.t. \(F\) is an extension of \(f\) and \(F(x) \leq p(x) \; \forall \; x \in X\).

    Now, if \(F(x) \geq 0\)
    \[
        \abs{F(x)} = F(x) \leq p(x) = \norm{f}_{Y^*}\norm{x}_X
    \]
    If \(F(x) < 0\)
    \[
        \abs{F(x)} = -F(x) = F(-x) \leq p(-x) = \norm{f}_{Y^*}\norm{-x}_X = \norm{f}_{Y^*}\norm{x}_X
    \]
    \(\forall \; x \in X\)
    \[
        \abs{F(x)} \leq \norm{f}_{Y^*}\norm{x}_X
    \]
    namely, \(F \in X^*\) (it is bounded), and 
    \[
        \norm{F}_{X^*} \leq \norm{f}_{Y^*}
    \]

    Also, \(\norm{F}_{X^*} \geq \norm{f}_{Y^*}\) since \(F\) extends \(f\):
    \[
        \norm{F}_{X^*} = \sup_{\norm{x}_X \leq 1} \abs{F(x)} \geq \sup_{\norm{y}_Y \leq 1}\abs{F(y)} 
        = \sup_{\norm{y}_X \leq 1, y \in Y} \abs{f(y)} = \norm{f}_{Y^*}
    \]
\end{proof}
\noindent\underline{Consequence 1}
\begin{theorem}
    \((L^\infty (X))^*\) `strictly contains' \(L^1(X)\)
\end{theorem}

\begin{proof}
    We must show that \(\exists \; L \in (L^\infty(X))^*\) s.t. \(\nexists \; g \in L^1(X)\) s.t. 
    \[
        Lf = \int_X fg \, d\mu \quad \forall f \in L^\infty(X)
    \]

For simplicity, we consider \((X, \mathcal{M}, \mu) = ([-1, 1], \mathcal{L}([-1,1]), \lambda)\). Let \(Y\) be the subspace of \(L^\infty([-1,1])\) of the bounded continuous functions \(\mathcal{C}^0([-1,1])\). On \(Y\) we define 
\[
    \Lambda f = f(0) \quad \forall \; f \in Y
\]
We can do it since \(f \in \mathcal{C}^0([-1,1])\) (for elements in \(L^\infty\) we cannot speak about pointwise values!). 

\(\Lambda\) is linear:
\[
    \Lambda(\alpha f + \beta g) = \alpha \Lambda f + \beta \Lambda g
\]
Moreover, \(\Lambda\) is in \(Y^*\):
\[
    \abs{\Lambda f} = \abs{f(0)} < \max_{[-1,1]} \abs{f} = \norm{f}_{\infty}
\]
This proves that \(\Lambda \in Y^*\), \(\norm{\Lambda}_{Y^*} \leq 1\). By Hahn-Banach, \(\exists \; L \in (L^\infty(X))^*\) which is an extension of \(\Lambda\), and is s.t. 
\[
    \norm{L}_{(L^\infty)^*}
\]
Can we have 
\[
    Lf=\int_{-1}^1 fg \, d\mu \quad \mbox{for some } g \in L^1(X)\mbox{?}
\]
Suppose by contradiction that this is true, take 
\[
    f_n \in \mathcal{C}^0([-1,1])
\]
defined in this way:
\[
    f_n(x) = \phi(nx)
\]
where \(\phi\) is continuous, \(\mbox{supp}\, \phi \subseteq \left[ \frac{-1}{2} \frac{1}{2} \right]\) 
\[
    \phi(0) = 1, \phi(nx) = 0 \quad \forall \; x \mbox{ s.t. } \abs{nx} > \frac{1}{2} \Leftrightarrow \abs{x} > \frac{1}{2n}  
\]
By contradiction, 
\[
    \sup f_n \subseteq \left[-\frac{1}{2n}, \frac{1}{2n}\right] \Rightarrow f_n(x) \to 0
\]
Therefore, if \(g \in L^1([-1,1])\) is s.t. 
\[
    \int_{-1}^1 f_n g \, d\lambda = L f_n
\]
Then, on one side 
\[
    \int_{-1}^1 f_n g \, d\mu = L f_n = f_n (0) = 1 \quad \forall \; n
\tag*{(1)}\]
But on the other side 
\begin{itemize}
    \item \(f_n(x)g(x) \to 0\) a.e. in \([-1,1]\)
    \item \(\abs{f_n(x) g(x)} \leq g(x) \in L^1([-1,1])\)
    \[
        \overset{\mbox{DOM}}{\Rightarrow} \int_{-1}^1 f_n g \, d\lambda \to 0
    \tag*{(2)}\]
\end{itemize}
But \((1)\) and \((2)\) are in contradiction.
In conclusion, there is no \(g \in L^!([-1,1])\) s.t. 
\[
    \int_{-1}^1 f g \, d\lambda = L f \quad \forall \; f \in L^\infty ([-1,1])
\]
\end{proof}
\noindent\underline{Other consequences of the Hahn-Banach theorem}
\begin{corollary}
    \(X\) (real) normed space, \(x_0 \in X \setminus \left\{ 0 \right\}\).
    Then \(\exists \; L_{x_0} \in X^*\) s.t. 
    \[
        \norm{L_{x_0}}_{X^*} = 1 \mbox{ and } L_{x_0}(x_0) = \norm{x_0}
    \]
\end{corollary}
\begin{proof}
    Take \(Y = \left\{ \lambda x_0 : \lambda \right\}\) cose 
    

    This is linear and continuous on \(Y\) \(\Rightarrow\) by Hahn-Banach (continuous extension) \(\exists \; \tilde{L}_0 \in X^*\) s.t. \(\tilde{L}_0\) extends \(L_0\) and 
    \[
        \norm{\tilde{L}_0}_{X^*} = \norm{L_0}_{Y^*} = \sup_{\begin{array}{c}\lambda x_0 \in Y \\ \norm{\lambda x_0} = 1\end{array}}\abs{L_0(\lambda x_0)}= \sup \abs{\lambda \norm{x_0}_X} = 1
    \]
    Thus \(\tilde{L}_0\) is precisely the desired functional.
    \[
        \tilde{L}_0 (x_0) = L_0 = \norm{x_0}_X
    \]
    and 
    \[
        \norm{\tilde{L}_0}_{X^*} = 1
    \]
\end{proof}
\begin{corollary}[The bounded linear functionals separate points]
    If \(x,y \in X\) and \(Lx = Ly\) \(\forall \; L \in X^* \Rightarrow x = y\) (if \(x \neq y, \exists \; L \in X^*\) s.t. \(Lx \neq Ly\))
\end{corollary}
\begin{proof}
    Assume \(x-y \neq 0\). Then, by the previous corollary, \(\exists \; L \in X^*\) s.t. 
    \[
        \norm{L}_{X^*} \mbox{ and } L(x-y) = \norm{x-y}_X \Rightarrow Lx -Ly = L(x-y) = \norm{x-y}_X \neq 0
    \]
\end{proof}
\begin{corollary}
    \(X\) normed space, \(Y\) closed subspace of \(X\), \(x_0 \in X \setminus Y\). 

    Then \(\exists \; L \in X^*\) s.t. \(L\vert_Y = 0\) and \(L_{x_0} \neq 0\)
\end{corollary}
\subsection*{Reflexive spaces}
\(X\) Banach space, \(X^*\) dual space.

\noindent \underline{Notation}: \(L \in X^*: Lx = L(x) = <L,x> = \underset{X^*}{<}L,x\underset{X}{>}\)

\((X^*)^*\) dual space of \(X^*\) is called the \textbf{bidual} of \(X\), denoted by \(X^*\)
\[
    X^{**} = \mathcal{L}(X^*, \real)
\]
We can describe many elements of \(X^{**}\) in the following way: for \(x \in X\), define 
\[
    \begin{array}{ll}
        \Lambda: & X^* \to \real \\
        & L \mapsto Lx = \underset{X^*}{<}L,x\underset{X}{>}
    \end{array}
\]
(\(\Lambda_x\) evaluates functionals in \(X^*\) in the point \(x\)).

\(\Lambda_x\) is linear:
\[
    \Lambda_x (\alpha L_1 + \beta L_2) = (\alpha L_1 + \beta L_2)(x) = \alpha L_1 x + \beta L_2 x = \alpha \Lambda_x L_1 + \beta \Lambda_x L_2
\]
Moreover, it is bounded 
\[
    \abs{\Lambda_x(L)} = \abs{Lx} \underset{L \in X^*}{\leq} \norm{L}_{X^*}\norm{x}_X \quad \forall \; L \in X^*
\]
Moreover, 
\[
    \norm{\Lambda_x}_{\mathcal{L}(X^*, \real)} = \sup_{L \neq 0} \frac{\abs{\Lambda_x L}}{\norm{L}_X^*}
\]
We claim that \(\norm{\Lambda_x}_{\mathcal{L}} = \norm{x}_X\). Indeed, by the first corollary of Hahn-Banach, given any \(x \in X\setminus \left\{ 0 \right\} \exists \; Lx \in X^*\) 
\[
    \exists \; L_x \in X^* \mbox{ s.t } \abs{L_x x} = \norm{x}_X, \mbox{ and } \norm{L_x}_{X^*} = 1 
\]
\[
    \Rightarrow \sup_{L \neq 0} \frac{\abs{\Lambda_x L}}{\norm{L}_{X^*}} = \sup_{L \neq 0} \frac{\abs{Lx}}{\norm{L}_{X^*}} \geq \frac{\abs{L_x x}}{\norm{L_x}_{X^*}} = \norm{x}_X 
\]
\[
    \Rightarrow \norm{\Lambda_x}_{X^{**}} = \norm{x}_X
\]
\begin{theorem}
    \(\exists\) a map 
    \[
        \begin{array}{lc}
            \tau : & X \to X^{**} \\
            & x \mapsto \Lambda_x
        \end{array}
    \tag*{(Canonical Map)}\]
    which is linear, continuous and an isometry. Namely, the canonical map is an isometric isomorphism from \(X\) into \(\tau(X) \subseteq X^{**}\)
\end{theorem}
\noindent\underline{Question}: are there other elements in \(X^{**}\)?
\begin{definition}
    If the canonical map is surjective, then we say that \(X\) is \textbf{reflexive}, \(X \approxeq X^{**}\). Otherwise, \(\tau(X)\) will be a strict close subspace of \(X\).
\end{definition}
\begin{remark}
    \(X\) reflexive \(\nLeftarrow\) \(X\) and \(X^{**}\) are isometrically isomorphic.
\end{remark}
\begin{theorem}
    \(X\) reflexive space. Then every closed subspace of \(X\) is reflexive.
\end{theorem}
\begin{theorem}
    \(X\) Banach. 
    \[
        X \mbox{ reflexive } \Leftrightarrow X^* \mbox{ reflexive }
    \]
\end{theorem}
\begin{theorem}
    \(X\) Banach.
    \begin{itemize}
        \item If \(X^*\) is separable \(\Rightarrow X\) is separable
        \item If \(X\) is separable and reflexive \(\Rightarrow X^*\) is separable
    \end{itemize}
\end{theorem}
To show that a space is reflexive, it is convenient to introduce the following notion.
\begin{definition}
    \(X\) Banach space. \(X\) is called \textbf{uniformly convex} if \(\forall \; \epsilon > 0 \; \exists \; \delta > 0 \) s.t. 
    \[
        \forall \; x, y \in X \mbox{ with } \norm{x} \leq 1, \norm{y} \leq 1, \norm{x-y} > \epsilon
    \]
    then we have 
    \[
        \norm{\frac{x+y}{2}} < 1-\delta
    \]
    This is a quantitative version of the strict convexity.
\end{definition}
\begin{definition}
    \(C \subset X\) is convex  \(\Leftrightarrow \forall \; x,y \in C : \frac{x+y}{2} \in C\)

    \noindent \(C \subset X\) is \textbf{strictly convex} \(\Leftrightarrow \forall \; x,y \in C : \frac{x+y}{2} \in \interior{C}\)
\end{definition}
Roughly speaking, \(X\) is uniformly convex if \(\overline{B_1(0)}\) is strictly convex in a quantitative way.
\begin{theorem}[Milman-Pettis]
    Every uniformly convex Banach space is reflexive.
\end{theorem}
\section{Lecture 24/11/2022}
\noindent\underline{Recap on reflexivity}:

\noindent\(X\) Banach space. \(X^{**} = (X^*)^*\) is the \textbf{bidual space}, \(\mathcal{L}(X^*, \real)\)
\[
    \forall \; x \in X \; \exists \; \Lambda_x : X^* \to \real \mbox{ defined by } \Lambda_x(L) = Lx \quad \forall \; L \in X^*
\]
We proved that \(\Lambda_x \in X^{**}\). Thus we can define the \textbf{canonical map}:
\[
        \begin{array}{lc}
            \tau : & X \to X^{**} \\
            & x \mapsto \Lambda_x
        \end{array}
    \tag*{(Canonical Map)}
\]
We stated that \(\tau\) is an isometric isomorphism from \(X\) into \(\tau(X)\). This is true but for our purpose it's even too much, and it's difficult to prove in details. However, we can prove a slightly weaker result 
\begin{theorem}
    \(\tau\) is linear, continuous, and is an isometry
    \[
        \norm{\tau(x)}_{X^{**}} = \norm{x}_X \quad \forall \; x \in X
    \]
    Moreover, \(\tau\) is injective. If \(\tau\) is also surjective, it is an isometric isomorphism between \(X\) and \(X^{**}\)
\end{theorem}
\begin{proof}
    There are two parts:
    \begin{itemize}
        \item \(\tau\) is linear and continuous: exercise. 
        
        \noindent \(\tau\) is an isometry: \(\norm{\tau(x)}_{X^{**}} = \norm{\Lambda_x}_{X^{**}} = \norm{x}_X\)

        \noindent \(\tau\) is injective: \(x\neq y \Rightarrow \tau(x) \neq \tau(y)\)? 

        \noindent \(x \neq y \Rightarrow\) by the second corollary to Hahn-Banach \(\exists\; L \in X^*\) s.t. \(Lx \neq Ly\).
        \[
            \underset{X^{**}}{\langle}\tau(x), L \underset{X^*}{\rangle} = \Lambda_x(L) = Lx \neq Ly = \Lambda_y(L) = \underset{X^{**}}{\langle}\tau(x), L \underset{X^*}{\rangle}
        \]
        Then, \(\tau(x) \neq \tau(y)\) and \(\tau\) is injective.
        \item Let now \(\tau\) be surjective. Then \(\tau \in \mathcal{L}(x, X^{**})\) and is bijective \(\Rightarrow\) by a corollary of the open map theorem, \(\tau^{-1} \in \mathcal{L}(X^{**}, X)\)
    \end{itemize}
\end{proof}
\begin{definition}
    \(X\) is reflexive if \(\tau\) is surjective. In this case, \(\tau\) is an isometric isomorphism between \(X\) and \(X^{**}\)
\end{definition}

We formally mentioned that
\begin{theorem}
    If \((X, \normdot)\) is uniformly convex \(\Rightarrow\) \((X, \normdot)\) is reflexive.
\end{theorem}

\noindent\textbf{\underline{Remarks}:}
    \begin{proposition}
        If \((X, \normdot)\) is uniformly convex \(\Rightarrow\) \(\overline{B_1(0)}\) is strictly convex.
    \end{proposition}
    \begin{proof}
        Is it true that if \(x, y \in \overline{B_1(0)}\), then \(\frac{x+y}{2} \in B_1(0)\)? Since \((X, \normdot)\) is uniformly convex, we know that \((\norm{x-y}=: \overline{\epsilon}>0)\)
        \[
            \forall \; \overline{\epsilon}>0 \quad \exists \; \overline{\delta} >0 \text{ s.t. } \norm{x} \leq 1 \; \norm{y} \leq \; \norm{x-y} > \frac{\overline{\epsilon}}{2} \Rightarrow \norm{\frac{x+y}{2}} < 1-\overline{delta} 
        \]
        In particular, 
        \[
            \frac{x+y}{2} < 1-\overline{\delta} < 1 \Rightarrow \frac{x+y}{2} \in B_1(0)
        \] 
    \end{proof}
    Consequence: \((\real^2, \normdot_1)\) and \((\real^2, \normdot_\infty)\) are not uniformly convex.
\begin{proposition}
    \((\real^2, \normdot_2)\) is uniformly convex
\end{proposition}
\begin{proof}
    Suppose by contradiction that this is false:
    \( \exists \; \overline{\epsilon} >0 \) and \( \{x_n\}, \{y_n\} \subset \overline{B_1(0)} \) s.t. 
    \[
        \norm{x_n - y_n} > \overline{\epsilon} \text{, but } \norm{\frac{x_n + y_n}{2}} \geq 1 \tag{*}
    \]
    \(\overline{B_1(0)}\) is compact (since we are in \(\real^2\)) \(\Rightarrow\) UTS \(x_n \to \overline{x} \), \(y_n \to \overline{y}\) as \(n \to\infty\). Taking the limit in (*), we deduce that \(\overline{x}\), \(\overline{y} \in \overline{B_1(0)}\)
    \[
        \norm{\overline{x}- \overline{y}} \geq \overline{\epsilon} \text{ , and } \norm{\frac{\overline{x}+\overline{y}}{2}} \geq 1
    \]
    This is not possible, since \(\overline{B_1(0)}\) is strictly convex.
\end{proof}

\begin{theorem}
    \((X, \mathcal{M}, \mu)\) complete measure space. Then \(L^{p}(X)\) is reflexive \(\forall \; p \in (1, \infty)\)
\end{theorem}
\begin{proof}
    \((L^p(X), \normdot_p)\) is uniformly convex \(\forall \; p \in (1, \infty)\) (Clarkson inequalities)
\end{proof}
\(L^1(X) and L^\infty(X)\) are not uniformly convex, and not reflexive.

\subsection*{Dual space of \(L^p\)}

\begin{theorem}[Riesz representation theorem]
    \((X, \mathcal{M}, \mu)\) complete measure space, \(p \in (1, \infty)\). Then 
    \[
        \forall\; L \in (L^p(X))^* \quad \exists! \; g \in L^{p'}(X)
    \]
    with \(p' \) conjugate exponent s.t. \(L=L_g\), namely
    \[
        Lf = \int_X fg \, d\mu \qquad \forall\; f \in L^p(X)
    \]
    Moreover \(\norm{L_g}_{(L^p)^*} = \norm{g}_{p'}\)
    
    Thus: \(T: g \in L^{p'} \mapsto L_g \in (L^p)^*\) is an isometric isomorphism.
\end{theorem}
\begin{proof}
    \(1 < p < \infty\). Consider \(T: L^{p'} \to (L^p)^*\) with \(g \mapsto Tg: \langle Tg, f \rangle = \int_X fg \, d\mu\) (namely \(Tg = L_g\)). We already know that
    \[
        \norm{Tg}_* = \norm{L_g}_* = \norm{g}_{p'}
    \]
    \(T\) is injective: for exercise.

    \(T\) is surjective. Indeed, let \(F:= T(L^{p'}) \subseteq (L^p)^*\) subspace. Since \(T\) is an isometry and \(L^{p'}\) is complete, it can be shown that \(T(L^{p'})\) is also complete \(\Rightarrow T(L^{p'})\) is closed.

    If by contradiction \(F \neq (L^p)^*\), then we can apply corollary 3 to Hahn Banach \((X = (L^p)^*), Y=F, x_0 = \lambda\):
    \[
        \exists h \in (L^p)^{**} \text{ s.t. } \langle h, \lambda \rangle \neq 0 \text{ and } h|_F = 0: \langle h, Tg \rangle =0 \quad \forall g \in L^{p'} \tag*{1} 
    \]
    But \(L^p \) is reflexive \((1 <p < \infty)\), then \(h \in L^p \setminus \{0\}\):
    \[
        \langle h, Tg \rangle = (Tg)h = \int_X hg \, d\mu
    \]
    Therefore, (1) tells us that
    \[
        \int_X hg \, d\mu =0 \quad \forall \; g \in L^{p'}(X)
    \]
    Take \(g = \abs{h}^{p-2}h\). Therefore
    \[
        0 = \int_X hg \, d\mu = \int_X h \abs{h}^{p-2} h \, d\mu = \int_X \abs{h}^p \, d\mu \Rightarrow h=0 \in L^p
    \]
    which is the desired contradiction.

    \(T\) is an isomorphism: for exercise.
\end{proof}

\[
    (L^p)^* = L^{p'}
\]

\begin{remark}
    \(p=1\). one can prove the
\end{remark}
\begin{theorem}
    \((X; \mathcal{M}, \mu)\) complete measure space, \(\sigma\)-finite.
    Then \(\forall\; L \in (L^1(X))^* \quad \exists! \; g \in L^\infty(X)\) s.t. \(L=L_g\):
    \[
        Lf = \int_X fg \, d\mu \quad \forall f \in L^1(X)
    \]
    Moreover, the map \(L \in (L^1)^* \mapsto g \in L^\infty\) is an isometric isomorphism.
\end{theorem}

Recall that \((X, \mathcal{M}, \mu)\) is finite if \(\mu(X) < \infty\).

\begin{definition}
    \((X, \mathcal{M}, \mu)\) is \(\sigma\)-finite if either \(\mu(X) < \infty\), or \(X = \sum_{n=1}^\infty X_n\), where \(\mu(X_n) < \infty\)
\end{definition}

\subsection*{Weak Convergence}

We know that \(\overline{B_1(0)}\) is never compact in \(\infty\) dimension. This is a problem is in proving convergence of sequences. a way to approach this issue consists in weakening the notion of convergence.
\begin{definition}
    \(X\) Banach space. \(\{x_n\} \subset X\) sequence, \(x \in X\). We say that \(x_n\) tends to \(x\) weakly (in \(X\)) as \(n \to \infty\), \(x_n \rightharpoonup x\) in \(X\), if 
    \[
        L x_n \to Lx \qquad \forall \; L \in X^*
    \]
\end{definition}
\begin{remark}
    Assume that \(x_n \to x\) in \(X\), namely \(\norm{x_n -x}_X \to 0\). If \(f:X \to \real\) is continuous, then
    \[
        \lim_{n \to \infty} f(x_n) = f(x)
    \]
    In particular, this is true if \(f = L \in X^*\):
    \[
        \begin{array}{rlc}
            x_n \to x \in X \Rightarrow & L x_n \to Lx & \forall\; L \in X^* \\
            & x_n \rightharpoonup x \in X & \\
            x_n \to x \in X \Rightarrow & x_n \rightharpoonup x \text{ weakly } \in X & \\
             \nLeftarrow & 
        \end{array}
    \]
\end{remark}
\begin{remark}
    We will be interested in weak convergence in \(L^p\). If \(p \in [1, \infty)\), then \(f_n \rightharpoonup f\) weakly in \(L^p(X) \iff \int_X f_n g d, d\mu \to \int_X fg \, d\mu\) \(\forall \; g \in L^{p'}\)

    Similarily, in \(l^p = L^P(\natural, \mathcal{P}(\natural), \mu_c)\), \(x_n \rightharpoonup x \) weakly in \(l^p\) \(\iff \sum_{k=1}^\infty x_n^{(k)} y^{(k)} \to \sum_{k=1}^\infty x^{(k)} y^{(k)}\) \(\forall \; y \in l^{p'}\)
\end{remark}

\begin{proposition}
    The weak limit is unique (if it exists)
\end{proposition}
\begin{proof}
    By contradiction, suppose that \(\exists \{ x_n\} \subset X\) s.t. \(x_n \rightharpoonup x_1\), \(x_n \rightharpoonup x_2\) weakly in \(X\), \(x_1 \neq x_2\). Then
    \[
        \begin{array}{lc}
            Lx_n \to Lx_1 & \forall\; L \in X^* \\
            Lx_n \to L x_2 & \forall\; L \in X^* \\
            \Rightarrow Lx_1 = Lx_2 & \forall\; L \in X^*
    \end{array} 
    \] 
    By Hahn Banach (corollary 2), this implies \(x_1 = x_2\), a contradiction.
\end{proof}

\begin{proposition}
    If \(x_n \rightharpoonup x\) weakly in \(X\), then \(\{x_n\}\) is bounded, and 
    \[
        \norm{x} \leq \liminf_{n \to \infty} \norm{x_n} \tag*{weak lower semi continuity of \(\normdot\)}
    \]
\end{proposition}
\begin{proof}
    \begin{itemize}
        \item \(\{x_n\}\) is bounded
        
        \(x_n \rightharpoonup x\) weakly \(\Rightarrow \{L x_n\}\) is bounded in \(\real\), \(\forall \; L \in X^*\). Consider \(\Lambda_n \in X^{**}\) def by 
        \[
            \Lambda_n L = L x_n \quad \forall\; L \in X^{*}
        \]
        \(\forall\; L \in X^* \exists\; M_L >0 \) s.t.
        \[
            \abs{\Lambda_n L} = \abs{L x_n } \leq M_L \qquad \forall\; n \tag*{PB}
        \]
        (pointwise boundedness of \(\{\Lambda_n \} \subset X^{**}\))
        \[
            \Lambda_n: X^{*}= \mathcal{L}(X, \real) \to \real    
        \]
        By Banach Steinhaus, \(\{\Lambda_n\}\) is uniformly bounded:
        \[
            \sup_n \norm{ \Lambda_n}_{\mathcal{L}(X^*, \real)} \leq M
        \]
        Moreover, by Hahn Banach, \(\forall \; n \in \natural \quad \exists L_n \in X^*\) s.t. \(\norm{L_n}_* = 1 \) and \(L_n x_n = \norm{x_n}\). Therefore
        \[
            \norm{x_n} = \abs{L_n x_n} = \abs{\Lambda_n L_n} \leq \norm{\Lambda_n}_\mathcal{L} \norm{L_n}_* \leq M \quad \forall n \in \natural
        \] 
        \item \(x_n \rightharpoonup x\) weakly. 
        
        By corollary 1 of Hahn Banach, \(\exists L_x \in x^*\) s.t. \(\norm{L_x}_* = 1\) and \(L_x x = \norm{x}\). Then
        \[
            \norm{x} = \abs{L_x x} = \lim_n \abs{L_x x_n} = \liminf_n \abs{L_x x_n} \leq \liminf_n \norm{L_x}_* \norm{x_n }_X = \liminf_n \norm{x_n }_X
        \]    
    \end{itemize}
\end{proof}

\begin{proposition}
    \(x_n \rightharpoonup x \) in \(X\) weakly, and \(L_n \to L\) (strongly) in \(X^*\). Then
    \[
        L_n X_n \to Lx \quad \text{ in } \real
    \]
\end{proposition}
\begin{proposition}
    \(X, Y\) Banach, \(T \in \mathcal{L}(X, Y)\)
    \[
        x_n \rightharpoonup x \text{ weakly } \Rightarrow Tx_n \rightharpoonup Tx \text{ weakly }
    \]
\end{proposition}
\section{Lecture 31/11/2022}

\section{Lecture 01/12/2022}

\subsection*{Hilbert Spaces}

\begin{definition}
    \(H\) vector space on \(\real\). A function \(p:H \times H \to \real\) is called scalar (or inner) product if it is positive definite, symmetric, and bilinear; namely if 
    \begin{enumerate}
        \item \(p(x, x) \geq 0\) \(\forall \; x \in H\) and \(p(x, x) = 0 \Rightarrow x=0\)
        \item \(p(x, y) = p(y, x)\) \(\forall \; x, y \in H\)
        \item \(p(\alpha x_1 + \beta x_2, y) = \alpha p (x_1, y) + \beta p(x_2, y)\) \(\forall \; \alpha, \beta \in \real\), \(x_1, x_2, y \in H\)
    \end{enumerate}
\end{definition}

Notation: \(p(x, y) = <x, y> = (x, y) = x \dot y\)

\begin{definition}
    A vector space \(H\) with a scalar product is called a pre Hilbertian space.
\end{definition}

\begin{proposition}
    \((H, <.,.>)\) pre Hilbertian space.
    \begin{itemize}
        \item Cauchy Schwarz inequality \[
            \abs{<x, y>} \leq \sqrt{<x, x>} \sqrt{<y, y>} \quad \forall \; x, y \in H
        \]
        \item \(\sqrt{<x, x>} =: \norm{x}\) is a norm on \(H\)
    \end{itemize}
\end{proposition}

\((H, <.,.>)\) pre Hilbert \(\to (H, \normdot)\) normed space \(\to (H, d)\) metric space where \(d(x, y) = \norm{x-y}\)

\begin{definition}
    We say that \(H, <.,.>\) is a Hilbert space if \((H, \normdot)\) is a Banach space. (namely, if \((H, d)\) is a complete metric space)
\end{definition}

Examples:
\begin{itemize}
    \item \(\real^n\), \(<x, y> = \sum_{i=1}^n x_i y_i\)
    \item \(L^2(X, \mathcal{M}, \mu)\) \((X, \mathcal{M}, \mu)\) complete measure space. \(<f, g> = \int_X fg \, d\mu\). \(\norm{f} = (\int_X f^2 \, d\mu)^\frac{1}{2} = \norm{f}_2\). \((L^2(X), \normdot_2)\) is a Banach space \(\Rightarrow (L^2(X), <.,.>)\) is a Hilbert space.
    \item \(l^2\) is a Hilbert space. \(<x, y> = \sum_{k=1}^\infty x^{(k)} y^{(k)}\), \(x = (x^{(k)})\), \(y = (y^{(k)})\)
    \item \((\mathcal{C}^0([a, b]), <.,.> )\) is a pre Hilbertian space. \((\mathcal{C}^0([a, b]), \normdot_2 )\) is not a Banach space. 
\end{itemize}

\begin{definition}
    \(x, y\) are orthogonal if \(<x, y>=0\). We write \(x \perp y\)
\end{definition}
\begin{remark}
    Hilbert spaces are particular cases of Banach spaces. The converse is not true. In any Hilbert space, the norm induced by \(<.,.>\) must satisfy the parallelogram rule
    \[
        \norm{x+y}^2 + \norm{x-y}^2 = 2\norm{x}^2 + 2\norm{y}^2 \quad \forall \; x, y \in H \tag*{PR}
    \]
\end{remark}

\begin{proposition}
    \(H\) Banach space with respect to \(\normdot\). If \(\normdot\) satisfies (PR), then \(H\) is a Hilbert space with scalar product 
    \[
        <x, y> := \frac{1}{2} [\norm{x+y}^2 - \norm{x}^2 - \norm{y}^2], \quad <x,x> = \norm{x}^2
    \]
\end{proposition}

Consequence: we can check that a Banach space is not a Hilbert space by showing that (PR) does not hold.
Ex: \((L^p, \normdot_p)\) is not a Hilbert space \(\forall\; p \neq 2\). The same for \((\mathcal{C}^0([a, b]), \normdot_\infty )\)

\subsection*{Orthogonal projection}

Recall:
\begin{definition}
    \(C \subset H \) is convex if \(\forall\; x, y \in C: \, \frac{x+y}{2} \in C\)
\end{definition}
\begin{definition}
    \(S \subset H\), \(f \in H\). \[ dist (f, S) = \inf_{g \in S} \norm{f-g}\]
\end{definition}

\begin{theorem}[projection on closed convex sets]
    \(H\) Hilbert space. Let \(S \subseteq H\) non empty, closed, convex. Then \(\forall \; f \in H \exists! \; h \in S\) s.t. 
    \[
        \norm{f-h} = dist (f, S) = \min_{g \in S} \norm{f-g} \tag*{1}
    \]
    Moreover, \(h\) is characterized by the variational inequality:
    \[
        <f-h, g-h> \leq 0 \quad \forall \; g \in S \tag*{*}
    \]
    namely \(h\) is the projection of \(f \) on \(S\) (\(f\) satisfies (1)) \(\iff (*) holds\)
\end{theorem}
\begin{remark}
    \(h\) satisfies 1: \(h\) is the projection of \(f\) on \(S\), \(h = P_S f\)
\end{remark}
\begin{proof}
    Only of the existence of \(h\). \\
    \(S \subset H\). \(dist(f, S) >0\) \((f \notin S )\). 
\end{proof}
\section{Lecture 14/12/2022}

\subsection*{Dual space of a Hilbert space}
Observe that, if \(y \in H\), then we can define \(\Lambda_y : H \to \real\) as
\[
    \Lambda_y x = \langle y, x\rangle  
\]
It is linear (\(\scalardot   \) is bilinear), and it is bounded: 
\[
    \abs{\Lambda_y x}= \abs{\langle y, x\rangle  } \leq \norm{y} \norm{x} \quad \forall\; x, y
\]
\(\Rightarrow \Lambda_y\) is bounded, with \(\norm{\Lambda_y}_* \leq \norm{y}\)

Moreover, 
\[
    \Lambda_y (\frac{y}{\norm{y}}) = \langle y, \frac{y}{\norm{y}}\rangle   = \norm{y}
\]
\[
    \Rightarrow \norm{\Lambda_y}_* = \sup_{\norm{x} \leq 1} \abs{\Lambda_y x} \geq \abs{\Lambda_y (\frac{y}{\norm{y}})} = \norm{y}
\]
Thus \( \norm{\Lambda_y}_* = \norm{y}\), and the map 
\[
    \begin{array}{rl}
        i: & H \to H^* \\
        & y \mapsto \Lambda_y    
    \end{array}
\]
is an isometry from \(H\) into \(i(H) \subset {H^*}\).

Are there other elements in \(H^*\)?

\begin{theorem}[Riesz Representation Theorem]
    \(\forall \; \Lambda \in H^* \) \(\exists! \; y \in H\) s.t. \(\Lambda = \Lambda_y\), namely
    \[
        \Lambda x = \langle y, x\rangle   \qquad \forall x \in H
    \]
    Moreover, the map \(i\) is an isometric isomorphism. We can identify \(H^*\) with \(H\)
\end{theorem}

\begin{corollary}
    Any Hilbert space is reflexive. 
\end{corollary}
\begin{remark}
    Any Hilbert space is uniformly convex.
\end{remark}

\begin{itemize}
    \item Riesz in \(L^p\): \(L^p \) is uniformly convex \(\Rightarrow L^p\) is reflexive. We used this fact to prove Riesz in \(L^p\)
    \item Riesz in Hilbert: direct proof of \(H^* = H \Rightarrow H\) is reflexive.
\end{itemize}
Both strategies can be adopted in both contexts.

\begin{proof}
    \begin{itemize}
        \item We show that \(\forall\; \Lambda \in H^* \) \(\exists\; y \in H\) s.t. \(\Lambda = \Lambda_y\)
        
        If \(\Lambda = 0 \Rightarrow \Lambda = \Lambda_0\) \((\Lambda_0 x = \langle 0, x\rangle   = 0)\)

        Suppose \(\Lambda \neq 0. \) \(\ker(\Lambda) = \Lambda^{-1}(\{0\}) \) is a closed (since \(\Lambda\) is continuous) subspace, \(\neq H\). \(\Rightarrow\) we consider \(\ker(\Lambda)^\perp \neq \{0\}\). Let
        \[
            z \in \ker(\Lambda)^\perp, \quad \norm{z}=1
        \]
        For \(x \in H\), we have
        \[
            x - \frac{\Lambda x}{\Lambda z} z \in \ker(\Lambda)
        \]
        Indeed, \(\Lambda \left( \frac{\Lambda x}{\Lambda z} z \right) \overset{linearity}{=} \Lambda x - \frac{\Lambda x}{\Lambda z} \Lambda z = 0\). Then, since \(z\) is orthogonal to any element of \(\ker(\Lambda)\), 
        \[
            \langle z, x - \frac{\Lambda x}{\Lambda z} z\rangle   = 0 \qquad \forall\; x \in H
        \] 
        \(\scalardot  \) is bilinear: the left hand side is 
        \[
            \langle z, x\rangle   - \frac{\Lambda x}{\Lambda z} \norm{z}^2 \Rightarrow \langle z, x\rangle   = \frac{\Lambda x}{\Lambda z} 
        \]
        \[
            \Lambda x = \langle (\Lambda z)z, x\rangle   \qquad \forall \; x \in H
        \]
        So the thesis is proved for \(y = (\Lambda z)z\).

        \item The uniqueness of \(y\) is easy.
        \[
            \langle x, y_1\rangle   = \langle x, y_2\rangle   \quad \forall \; x \in H
        \]
        Then \(\langle x, y_1-y_2\rangle   = 0\) \(\forall; x \in H\). We choose \(x = y_1 -y_2\):
        \[
            \norm{y_1 - y_2}^2 = 0 \Rightarrow y_1 = y_2
        \]
    \end{itemize}
\end{proof}

Consequence: \(H\) Hilbert space.
\[
    x_n \rightharpoonup x \text{ weakly in } H \iff \langle x_n, y\rangle   \to \langle x, y\rangle   \quad \forall\; y \in H
\]
Sometimes weak convergence + something else \(\Rightarrow\) strong convergence. For instance
\begin{proposition}
    \(H\) Hilbert. If \(x_n \rightharpoonup x\) weakly in \(H\), and \(\norm{x_n} \to \norm{x}\) \(\Rightarrow x_n \to x\) in \(H\), namely \(\norm{x_n - x} \to 0\)
\end{proposition}
\begin{proof}
    \[
        \norm{x_n - x }^2 = \norm{x_n}^2 - 2 \langle x_n, x\rangle   + \norm{x}^2 = (*)
    \]
    \(\langle x_n, x\rangle   \to \langle x, x\rangle   = \norm{x}^2 \) by weak convergence.
    \[
        (*) = \norm{x}^2 - 2 \norm{x}^2 + \norm{x}^2 = 0
    \]
\end{proof}

\subsection*{Orthonormal Basis}

In \(\real^n\), we have the canonical basis
\[
    e_1, ..., e_n \in \real^n
\]
s.t.
\[
    e_j^{(k)} = 
    \begin{cases}
        1 & k=j \\
        0 & k \neq j   
    \end{cases}
\]
There elements are \(\perp\): \(\langle e_i, e_j\rangle   = 0\) \(\forall\; i \neq j\). \(\norm{e_i}=1\) \(\forall\; i\).

Moreover, \(e_1, ..., e_n\) are a basis, namely \(\forall\; v \in \real^n\) \(\exists! \;\) expression
\[
    v = \sum_{i=1}^n v_i e_i = \sum_{i=1}^n \langle v, e_i\rangle   e_i
\]
In particular, \(v=0 \iff \langle v, e_i\rangle  = 0\) \(\forall\; i\). Do we have an analogue in Hilbert spaces?

\begin{definition}
    \(S \subset H\) is called orthonormal if 
    \begin{itemize}
        \item \(x \perp y\) \(\forall\; x \neq y\), \(x\), \(y \in S\)
        \item \(\norm{x} = 1\) \(\forall x \in S\)
    \end{itemize}
\end{definition}
\begin{definition}
    An orthonormal set is an Hilbert Basis (or is complete) if \(S^\perp = \{0\}\), namely if 
    \[
        \langle u, x\rangle   = 0 \quad \forall x \in S \Rightarrow u=0
    \]
\end{definition}
\begin{theorem}
    \(H\) Hilbert space, \(H \neq \{0\}\). Then \(H\) has an Hilbert basis.

    Moreover, \(H\) is a separable Hilbert space \(\iff \) it has a finite and countable Hilbert basis.
\end{theorem}

Example:
\begin{itemize}
    \item \(H = l^2\). \(H\) is separable
    
    An Hilbert basis is \(\{e_n\}_{n \in \natural}\) defined By
    \[
        e_n^{(k)} = 
        \begin{cases}
            1 & k=n \\
            0 & k \neq n
        \end{cases}
    \]

    \item \(H = L^2([-\pi, \pi])\)
    
    An Hilbert basis is
    \[
        \lbrace \frac{1}{\sqrt{2\pi}}, \frac{\sin(nx)}{\sqrt{\pi}}, \frac{cos(nx)}{\sqrt{\pi}} \rbrace \quad n \in \natural
    \]
\end{itemize}

\begin{remark}
    Hamel basis \(\neq\) Hilbert basis.

    \(X \infty\) dimensional \(\Rightarrow\) any Hamel basis of \(X\) is uncountable.

    \(H \infty\) dimensional and separable \(\Rightarrow\) any Hilbert basis is countable
\end{remark}

The usefulness of Hilbert basis stays in the fact that they allow us to reason component by component.

\begin{theorem}[Bessel inequality]
    \(H\) separable Hilbert space. \(\{u_n\}_{n \in \natural}\) orthonormal set. Then \(\forall \; x \in H\):
    \[
        \sum_{n=1}^\infty \abs{\langle x, u_n\rangle  } \leq \norm{x}^2
    \]
\end{theorem}
\begin{theorem}[Generalized Fourier Series]
    \(H\) separable Hilbert space, \(\{u_n\}\) Hilbert basis.
    Then any \(x \in H\) can be written in a unique way as 
    \[
        x = \sum_{n=1}^\infty \langle x, u_n\rangle   u_n \qquad \langle x, u_n\rangle   \text{ Fourier coefficient of }x
    \]
    Moreover, \(\forall\; y \in H\) we have
    \[
        \langle x, y\rangle   = \sum_{n=1}^\infty \langle x, u_n\rangle   \langle y, u_n \rangle  
    \]
    and 
    \[
        \norm{x}^2 = \sum_{n=1}^\infty (\langle x, u_n\rangle  )^2 \tag*{Parseval identity}
    \]
\end{theorem}
\begin{theorem}
    \(H\) separable Hilbert space. Then \(H\) is isomorphic to \(l^2\) as Hilbert space: namely \(\exists\) an isomorphism \(\phi: H \to l^2\) s.t.
    \[
        \langle x, y\rangle  _H = \langle \phi(x), \phi(y)\rangle  _{l^2} \qquad \forall x, y \in H
    \]
\end{theorem}
\begin{proof}
    \(\exists\) a countable Hilbert basis, and \(\forall\; x \in H\)
    \[
        x = \sum_{n=1}^\infty \langle x, u_n\rangle   u_n
    \]
    Then the desired isomorphism is
    \[
        \begin{array}{rl}
            \phi: & H \to l^2 \\
            & x \to \sum_{n=1}^\infty \langle x, u_n\rangle   e_n
        \end{array}
    \]
\end{proof}

\begin{corollary}
    \(H\) separable Hilbert space, \(\dim H = \infty\). \(\{u_n\}_{n \in \natural}\) Hilbert basis. Then \(u_n \rightharpoonup 0\) weakly in \(H\), but \(u_n \nrightarrow 0\) in \(H\).
\end{corollary}
\begin{proof}
    \[
        \norm{u_n}=1 \quad \forall\; n \Rightarrow \norm{u_n -0} \nrightarrow 0 
    \]
    On the other hand, we know that \(\forall \; x \in H\)
    \[
        \norm{x}^2 = \sum_{n=1}^\infty \abs{\langle x, u_n\rangle  }^2 \langle  \infty
    \]
    It is then necessary that
    \[
        \langle x, u_n\rangle   \to 0 \text{ as } n \to \infty \qquad \forall\; x \in H
    \]
    By Riesz, this means that \(u_n \rightharpoonup 0\) in \(H\)
\end{proof}

Ex: \(H = L^2([-\pi, \pi])\). Then the previous corollary tells that (Riemann - Lebesque lemma)
\[
    \int_{-\pi}^\pi f(x) \sin (nx) \, dx \to 0 \text{ as } n \to \infty
\]                
\(\sin(nx) \rightharpoonup 0\) in \(L^2([-\pi, \pi])\) as \(n \to \infty\). Note that \(\{sin(nx)\}_{n \in \natural}\) does not converge for a.e. \(x\).

Weak convergence in \(L^2\) and pointwise or a.e. convergence are not related.

\(\{sin(nx)\}\) does not converge a.e. on \([-\pi, \pi]\), not even up to subsequences. 
The same is true in \(L^p\), \(p \neq 2\). Even in this case
\[
    \sin(nx) \rightharpoonup 0 \text{ weakly in } L^p([a,b]) \quad (p \in [1, \infty)
\] 
but we don't have a.e. convergence.

\begin{proposition}
    \(p \in [1, \infty)\). Suppose that \(f_n \to f\) weakly in \(L^p(X)\), and that \(f_n \to g\) a.e. in \(X\). Then \(f=g\) a.e.
\end{proposition}

\begin{proposition}
    \(X\) Banach, \(V\) subspace of \(X^*\), dense in \(X^*\). Suppose that \(\{x_n\} \subset X\) is bounded, and that 
    \[
        L x_n \to Lx \qquad \forall\; L \in V
    \]
    Then
    \[
        L x_n \to Lx \qquad L \in X^*
    \]
    namely \(x_n \rightharpoonup x\) weakly in \(X\).
\end{proposition}
\begin{proof}
    (ex)
\end{proof}

Consequence: \(I \subset \real\) interval (\(I= \real\) is fine). \(\{f_n\} \subseteq L^p(I)\), \(p \in (1, \infty)\). 
\(\{f_n\}\) bounded in \(L^p\): \(\exists \; C>  0\) s.t \(\norm{f_n}_{L^p} \leq C\), \(\forall n\). Then:

\begin{itemize}
    \item If 
    \[
        \int_I f_n \phi \to \int_I f \phi \qquad \forall \; \phi \in \mathcal{C}_C(I)
    \]
    \(\Rightarrow f_n \rightharpoonup f\) weakly in \(L^p(I)\)
    \item If 
    \[
        \int_a^b f_n \to \int_a^b f \quad \forall (a, b) \subset I
    \]
    \(\Rightarrow f_n \rightharpoonup f\) weakly in \(L^p(I)\)
\end{itemize}

Some useful facts on bounded operators in Hilbert Spaces.

\(H\) Hilbert space. 
\begin{proposition}
    If \(T \in \mathcal{L}(H)\), then
    \[
        \norm{T}_{\mathcal{L}(H)} = \sup_{\norm{x}=\norm{y}=1} \abs{\langle Tx, y\rangle  }
    \]
\end{proposition}
\begin{definition}
    \(T \) is called \textbf{symmetric} (or self adjoint) if 
    \[
        \langle Tx, y\rangle   = \langle x, Ty\rangle   \qquad \forall x, y \in H
    \]
\end{definition}
\begin{proposition}
    Let \(T \in \mathcal{L}(H)\) be symmetric. Then 
    \[
        \norm{T}_{\mathcal{L}(H)} = \sup_{\norm{x}= 1} \abs{\langle Tx, x\rangle  }
    \]
\end{proposition}

Example: \(K \in L^2([0, 1]x[0, 1])\). Let \(T: L^2([0, 1]) \to L^2([0, 1])\) be defined by 
\[
    (Tf)(t) = \int_0^1 K(s, t) f(s) \,ds
\]
\(T \in \mathcal{L}(L^2([0, 1]))\). It is symmetric \(\iff K(s, t) = K(t, s)\) \(\forall \; s, t\)

\subsection*{Spectral Theory}
In what follows, \(E\) is a Banach space and \(T \in \mathcal{L}(E)\).

\begin{definition}
    The \textbf{resolvent} of \(T\) is 
    \[
        \rho(T) = \{ \lambda \in \real: T - \lambda I \text{ is bijective from } E \text{ to } E\}
    \]
\end{definition}
\begin{definition}
    The \textbf{spectrum} of \(T\) is    
    \[
        \sigma(T) = \real \setminus \rho(T)
    \]
\end{definition}
\begin{definition}
    \(\lambda\) is an \textbf{eigenvalue} of \(T\), \(\lambda \in EV(T) \), if 
    \[
        \ker(T- \lambda I) \neq \emptyset
    \]
    (\(T - \lambda I\) is not injective), namely if \(\exists \; u \in E\) s.t. \(u \neq 0\) and
    \[
        T u = \lambda u
    \]
    In this case, \(u\) is called eigenvector and \(\ker(T-\lambda I)\) is the eigenspace of \(\lambda\).
\end{definition}
\begin{remark}
    \(EV(T) \subset \sigma(T)\)
\end{remark}
\begin{remark}
    In finite dimension, linear operators can be represented by matrixes.

    \(A \; n \times n\) matrix. We know that \(x \mapsto Ax\) is bijective \(\iff \) it is injective \(\iff \) det\(A \neq 0\). In particular, in finite dimension \(\sigma(A) = EV(A)\). This is false in \(\infty \) dimension.
\end{remark}

Basic fact
\begin{theorem}
    \(E\) Banach, \(T \in \mathcal{L}(E)\). Then \(\sigma(T) \subset \real\) is compact, and 
    \[
        \sigma(T) \subset \left[ - \norm{T}_\mathcal{L}, \norm{T}_\mathcal{L} \right]
    \]
\end{theorem}

In general we cannot say much more.

Ex: in \(l^2\), consider the left shift:
\[
    T_l(x^{(0)}, x^{(1)}, x^{(2)},... x^{(n)},...) = (x^{(1)}, x^{(2)}, x^{(3)},... x^{(n+1)},...)
\]
\(T_l \in \mathcal{L}(l^2)\), \(\norm{T_l}=1\). \(EV(T_l)=?\). We have to solve
\[
    T_l x = \lambda x \qquad \text{ for some } \lambda \in \real, x \in l^2 \setminus \{0\}
\]
\(\Rightarrow x^{(1)}=\lambda x^{(0)}\). \(x^{(n+1)} = \lambda x^{(n)} = \lambda^{n+1} x^{(0)}\). \(\forall \; \lambda \in \real\), the sequence
\[
    x = x^{(0)} \left(1, \lambda,\lambda^2, ..., \lambda^n, ...\right)
\]
is a solution of \(T_l x = \lambda x\).
\[
    x \in l^2 \iff \sum_{n=0}^\infty (\lambda^n)^2 < \infty \iff \sum_{n=0}^\infty (\lambda^2)^n < \infty \iff \abs{\lambda} < 1
\]
Any \(\lambda \in (-1, 1)\) is an e.v. of \(T_l\). Moreover, \(\sigma(T_l)\) is a compact set which is included in \([-1, 1]\) and contains \(EV(T_l) =(-1, 1)\)
\[
    \Rightarrow \sigma(T) = [-1, 1]
\]

\section{Lecture 14/12/2022}

\end{document}

